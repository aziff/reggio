\documentclass[12pt]{article}
\usepackage[margin=1in]{geometry}

\usepackage{amsmath}
\usepackage{amssymb}
\usepackage{booktabs}
\usepackage{datetime}
\usepackage{float}
\usepackage{geometry}
\usepackage{graphicx}
\usepackage{natbib}
\usepackage{sectsty}
\usepackage{setspace}

\sectionfont{\fontsize{12}{12}\selectfont}

\settimeformat{hhmmsstime}

\begin{document}

\title{Update on Reggio Project}
\author{Reggio Team}
\date{Original version: Monday 15$^{\text{th}}$ August, 2016 \\ Current version: \today \\ \vspace{1em} Time: \currenttime}
\maketitle

\doublespace

The data used in the project evaluating the effect of the Reggio Approach are collected from a sample of individuals in three cities: Reggio Emilia, Parma, and Padova. The sample is comprised of six cohorts, described in Table \ref{tab:cohorts}. All individuals in the sample attended infant-toddler centers (asilo; ages 0-3) and/or preschool centers (materna; ages 3-5).\footnote{We currently only analyze those who attended materna.} We currently divide these centers in such a way to capture the different organizational oversight. These divisions are: municipal, state, religious, and private.\footnote{There are select individuals who did not attend any center. We categorize them as ``none" although they ostensibly received home-based care. These categorizations loosely follow the administrative entities that organize early childhood education \citet{della2001early}.} The municipal centers in the town of Reggio Emilia are the centers explicitly stated to be influenced by the work of Loris Malaguzzi.\footnote{\citet{Cagliari-etal-eds_2016_BOOK_Loris-Malaguzzi}.} This document presents evidence that this categorization of centers may not reflect the actual provision of specialized curricula. That is, although we are trying to evaluate the Reggio Approach, it is possible that other centers besides the municipal ones in Reggio Emilia may contain key components of the Reggio Approach. The information compiled below provides discusses open questions that complicate analysis of the Reggio Approach.

\begin{table}[htbp]
\small
\begin{center}
\caption{Cohort Structure}\label{tab:cohorts}
\begin{tabular}{l l l l}
\toprule
\multicolumn{1}{c}{(1)} & \multicolumn{1}{c}{(2)} & \multicolumn{1}{c}{(3)} & \multicolumn{1}{c}{(4)} \\
\multicolumn{1}{c}{Cohort} & \multicolumn{1}{c}{Birth Year} & \multicolumn{1}{c}{Age} & \multicolumn{1}{c}{$N$} \\
\midrule
Children & 2006 & 6 & 880 \\
Migrants & 2006 & 6 & 281 \\
Adolescents & 1994 & 18 & 836 \\
Adults 30s & 1980-1981 & 32 & 782 \\
Adults 40s & 1969-1970  & 43 & 791 \\
Adults 50s & 1954-1959 & 54-60 & 449 \\
\bottomrule
\end{tabular}
\end{center}
\footnotesize \raggedright
Note: This table describes the cohorts in the data used for the Reggio Project. The Children cohort is only comprised of children who are native Italians. Column (3) indicates the age at the time of the interview and Column (4) indicates the number of individuals in that cohort for which we have collected survey data.
\end{table}

\section*{Evidence of Spillover}
Although the Reggio Approach was formally started in 1963 with the founding of the first materna center in Reggio Emilia, influences before this founding and collaborations after hint a spillover of the Reggio Approach. For example, Malaguzzi was heavily influenced by Bruno Ciari, who directed the municipal schools of Bologna between 1966 and 1970.\footnote{\citet{forman1998hundred}.} Although Malaguzzi incorporated new programmatic elements, the horizontal hierarchy of the school, including the interaction with families, and the presence of mixed-age classrooms are adapted, in part at least, from Ciari.

There is further evidence that Malaguzzi actively spread his approach to a neighboring municipality, Modena. He was the director of the municipal schools in both Reggio Emilia and Modena until 1974.\footnote{\citet{Cagliari-etal-eds_2016_BOOK_Loris-Malaguzzi}.} Spillover since then is possible given the centralized system of early childhood education and several regulations culminating in universal preschool.\footnote{See \citet{della2001early} for details of these regulations, especially Law 444 enacted in 1968 mandating the provision of universal preschool. Mandated guidelines enacted in 1991 concerning curricular focus and teacher-child interactions present evidence of more modern homogenizing of the quality of the early childhood education centers.} This spillover is especially likely in Parma which is geographically and socially close to Reggio Emilia.

\section*{Undocumented Curricula}
To understand the differences and similarities between the school types, we have been seeking out information on the different curricula and approaches. However, there are many gaps in this knowledge. Much of what we know about programmatic elements comes from enacted regulations that require certain elements to receive funding. Regulations for religious schools not receiving state funding are unclear prior to 2000, which covers the majority of the cohorts.\footnote{\citet{ribolzi2013italy}.}

\section*{Historical Information for Older Cohorts}
Not only do we need to fully understand the approaches of the different schools, but we also need to understand the historical development of those approaches. Information on schools back to the 1950s has been difficult to procure, especially for religious schools. 

\section*{Selection Issues}
Although Italy has a centralized provision early childhood education, one aspect that is left to localities is the admission criteria. In Reggio Emilia, the municipal schools prioritize enrolling families who are disadvantaged. Other options, such as private schools, have a cost changing the demographic of families attending. Therefore, we are presented with a selection issue. When we see lower IQ scores in the individuals who went to municipal centers, we do not know if this is because of the different school types or because of selection of disadvantaged families into these centers.

\bibliography{../heckman}
\bibliographystyle{chicago}

\end{document}