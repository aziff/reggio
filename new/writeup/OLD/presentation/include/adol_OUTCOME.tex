\begin{table}[H]
\caption{Baseline characteristics by child-care type, adolescents (age 18)}
\label{tab:adol_CONTROLasilo}
\centering
\begin{adjustbox}{width=0.65\textwidth,center=\textwidth}
%\begin{adjustbox}{max height=\dimexpr\textheight-5.5cm\relax,max width=\textwidth}
\small
\begin{tabular}{l cccc}
\hline \hline 
 & Reggio & Reggio & Parma and & Any \\
 & Approach & Other & Padova & Other \\
 &   (1)    &  (2)  & (3)    &  (4) \\
%
\hline 
\multicolumn{5}{c}{\textit{Infant-toddler center}} \\
\hline 
%

%
\hline 
\multicolumn{5}{c}{\textit{Preschool}} \\
\hline 
%

\hline
\end{tabular}
\end{adjustbox}
 \vspace{1ex}

 \raggedright{
 \tiny{Average of baseline characteristics, by treatment (column 1) and comparison group.
 The first comparison group (column 2) are individuals living in Reggio Emilia who did not attend the RA. The second comparison group (column 3) are individuals living in Parma or Padova. The final comparison group (column 4) are individuals living in Reggio Emilia, Parma, or Padova who did not attend RA.
 Standard errors of means in brackets. Test for difference in means between each column and the first column (Reggio Municipal, the treatment group) was performed; *** significant difference at 1\%, ** significant difference at 5\%, * significant difference at 10\%. Source: authors calculation using survey data.}
 }
\end{table}
