%\section{Model of School Selection (Sketch)}
%\label{sec:model}

In this section we sketch a theoretical framework which helps motivate and interpret our empirical results. In order to better understand the family decision to send the child to a Reggio Children Approach (RA) child-care center, consider a generalized Roy model framework. $Y_{1,i}$ represents the potential outcome of the child $i$ when attending an RA center, and $Y_{0,i}$ the potential outcome of the child when not attending an RA center. Define potential outcomes as follows:

\begin{align*}
Y_{1,i}& =\mu_{1}(X_{i})+U_{1,i} \\
Y_{0,i}& =\mu_{0}(X_{i})+U_{0,i}
\end{align*}%
where $\mu_{d}(X_{i})=E(Y_{d,i}|X=x)$ for $d=0,1$. The individual return to the Reggio Children Approach can be defined as the difference between these two potential outcomes, $Y_{1,i}-Y_{0,i}=\mu_{1}(X_{i})-\mu_{0}(X_{i})+U_{1,i}-U_{0,i}$.

The total costs of attending the two types of schools will, in general, systematically differ as well. Let the costs associated with attending school type $d$ for family $i$ be given by
\begin{equation*}
V_{d}(Z_{i})=\delta_{d}(Z_{i})+\varepsilon_{d,i},\ d=0,1,
\end{equation*}%
where $Z_{i}$ are observable characteristics of household $i$ and $EV_{d}(Z_{i})=\delta_{d}(Z_{i}).$ A family will choose to send their child to a Reggio Children Approach center, $R_{i}=1$, depending on the perceived net benefit$,I_{R,i}$, which is given by
\begin{eqnarray*}
I_{R,i} &=&(Y_{1,i}-V_{1,i})-(Y_{0,i}-V_{0,i}) \\
&=&(Y_{1,i}-Y_{0,1})-(V_{1,i}-V_{0,i}) \\
&=&(\mu_{1}(X_{i})-\mu_{0}(X_{i})+U_{1,i}-U_{0,i})-(\delta_{1}(Z_{i})-\delta_{0}(Z_{i})+\varepsilon_{1,i}-\varepsilon_{0,i}) \\
&=&\mu_{1}(X_{i})-\mu_{0}(X_{i})-\delta_{1}(Z_{i})+\delta_{0}(Z_{i})-W_{i},
\end{eqnarray*}%
where $W_{i}\equiv U_{0,i}-U_{1,i}+\varepsilon_{1,i}-\varepsilon_{0,i}.$ Then define the net systematic return to choice $i$ as $R_{d}(X_{i},Z_{i})=\mu_{d}(X_{i})-\delta_{d}(Z_{i}).$ Then household $i$ chooses RA when $I_{R,i} \geq 0$, that is when \[R_{1}(X_{i},Z_{i}) - R_{0}(X_{i},Z_{i}) \geq W_{i}\]

We note that the observable variables $Z_{i}$ are considered to be characteristics of the household that shift the cost but do not affect the potential academic and non-academic benefits of attending the two types of school. That is, the difference in attendance likelihoods for household $i$ are a function of the $X_{i}$ and $Z_{i},$ but the difference in school benefits between an RA school and a non-RA school are functions only of the
observable characteristics $X_{i}.$

The variables observable to the econometrician are $R_{i},Y_{R_{i},i},X_{i},$ and $Z_{i}.$ Then we have%
\begin{equation}
EY_{R_{i},i}=\mu_{R_{i}}(X_{i})+E(U_{R_{i},i}|R_{i}).  \label{heckit}
\end{equation}%
Now assume for simplicity that all of the conditional expectation functions in the model are linear, so that%
\begin{equation*}
EY_{R_{i},i}=X_{i}\beta_{R_{i}},
\end{equation*}%
and%
\begin{equation*}
R_{1}(X_{i},Z_{i})-R_{0}(X_{i},Z_{i})=X_{i}(\beta_{1}-\beta_{0})-Z_{i}(\gamma_{1}-\gamma_{0}),
\end{equation*}%
and assume that all of the disturbance terms in the model are distributed as a multivariate normal. Then $W_{i}$ is normally distributed, and the conditional expectation $E(U_{R_{i},i}|R_{i})$ in (\ref{heckit}) can be consistently estimated (up to a scalar constant) from a first stage probit, as in Heckman (1976). We can then estimate the regression function with the estimated (up to scale) value $E(U_{R_{i},i}|R_{i}),$ and this enables identification of $\beta_{1},\beta_{0}$ and $(\gamma_{1}-\gamma_{0})$ and a small set of elements of the covariance matrix of $W_{i}.$ More efficient estimates of the identified parameters can be obtained through joint estimation of the $R_{i}$ and $Y_{R_{i},i}$ dependent variables.

%\end{document}
