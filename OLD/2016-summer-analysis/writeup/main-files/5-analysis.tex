\section{Analysis}
\label{sec:analysis}
This section reports the preliminary results of the analysis. 
Section \ref{sec:balance} provides information across cities and school types about the balance of characteristics that can be observed at baseline, before the child attends any type of child care.
Section \ref{sec:OLS} reports evidence of the difference in the relevant outcomes once these initial characteristics are controlled for.
%Section \ref{sec:PSM} takes a different approach and compares the children who had a similar probability of attending the RA school system. 
%Finally, section \ref{sec:IV} leverages information about distance from the different centers as an instrument for participation in the RA.
\todo[backgroundcolor=orange!30,size=\tiny]{Propensity score matching and IV}
\subsection{Balance of observable characteristics}
\label{sec:balance}
Given the non experimental nature of the analysis, selection of the proper control group is crucial. The following tables describe the characteristics of the family and child before going to the different types of child care. 

%\todo[backgroundcolor=orange!30,size=\tiny]{Insert here all the tables of balance of background characteristics.}
\singlespacing

%% children, infant-toddler
\begin{table}[H]
\caption{Baseline characteristics by infant-toddler-center type, children (age 6)}
% this is the top part of the tables that display the summary of the baseline characteristics, by city and child-care type
\centering
\begin{adjustbox}{width=1.2\textwidth,center=\textwidth}
\small
\begin{tabular}{m{4.0cm} cccccccccccc}
\hline \hline 
 & Reggio & Reggio & Reggio & Reggio & Parma & Parma & Parma & Parma & Padova & Padova & Padova & Padova \\
 & Municipal & Religious & Private & Not Attended & Municipal & Religious & Private & Not Attended & Municipal & Religious & Private & Not Attended \\

\hline 

CAPI & 0.60 & 0.56 & 0.17{*} & 0.51 & 0.44{***} & 0.43 & 0.36{**} & 0.46{**} & 0.49 & 0.31{***} & 0.32{***} & 0.55 \\
 & [0.04] & [0.10] & [0.17] & [0.05] & [0.04] & [0.20] & [0.08] & [0.05] & [0.06] & [0.09] & [0.07] & [0.04] \\
Male dummy & 0.57 & 0.52 & 0.67 & 0.49 & 0.58 & 0.43 & 0.56 & 0.54 & 0.56 & 0.62 & 0.59 & 0.47{*} \\
 & [0.04] & [0.10] & [0.21] & [0.05] & [0.04] & [0.20] & [0.08] & [0.05] & [0.06] & [0.10] & [0.08] & [0.04] \\
Age & 6.77 & 6.76 & 6.83 & 6.84 & 6.70{*} & 6.7 & 6.77 & 6.70{*} & 6.77 & 6.74 & 6.71 & 6.59{***} \\
 & [0.03] & [0.07] & [0.15] & [0.03] & [0.03] & [0.11] & [0.05] & [0.03] & [0.04] & [0.08] & [0.05] & [0.03] \\
%Age sq. & 46 & 45.77 & 46.76 & 46.95 & 44.97{*} & 44.9 & 46 & 44.97{*} & 45.89 & 45.64 & 45.14 & 43.52{***} \\
% & [0.38] & [0.97] & [2.12] & [0.47] & [0.35] & [1.47] & [0.75] & [0.45] & [0.56] & [1.04] & [0.66] & [0.37] \\
Middle school edu mom & 0.07 & 0.04 & 0.00 & 0.1 & 0.05 & 0.00 & 0.06 & 0.06 & 0.09 & 0.04 & 0.1 & 0.1 \\
 & [0.02] & [0.04] & [0.00] & [0.03] & [0.02] & [0.00] & [0.04] & [0.02] & [0.03] & [0.04] & [0.05] & [0.02] \\
High school edu mom & 0.47 & 0.41 & 0.5 & 0.42 & 0.37{*} & 0.57 & 0.42 & 0.46 & 0.31{**} & 0.35 & 0.34 & 0.57 \\
 & [0.04] & [0.10] & [0.22] & [0.05] & [0.04] & [0.20] & [0.08] & [0.05] & [0.06] & [0.10] & [0.07] & [0.04] \\
University edu mom & 0.33 & 0.37 & 0.5 & 0.17{***} & 0.51{***} & 0.43 & 0.47 & 0.38 & 0.50{**} & 0.42 & 0.54{**} & 0.23{*} \\
 & [0.04] & [0.09] & [0.22] & [0.03] & [0.04] & [0.20] & [0.08] & [0.05] & [0.06] & [0.10] & [0.08] & [0.04] \\
Middle school edu dad & 0.09 & 0.04 & 0.00 & 0.08 & 0.07 & 0.00 & 0.14 & 0.12 & 0.06 & 0.12 & 0.1 & 0.09 \\
 & [0.02] & [0.04] & [0.00] & [0.02] & [0.02] & [0.00] & [0.06] & [0.03] & [0.03] & [0.06] & [0.05] & [0.02] \\
High school edu dad & 0.37 & 0.22 & 0.67 & 0.33 & 0.34 & 0.14 & 0.36 & 0.41 & 0.37 & 0.23 & 0.44 & 0.48{*} \\
 & [0.04] & [0.08] & [0.21] & [0.04] & [0.04] & [0.14] & [0.08] & [0.05] & [0.06] & [0.08] & [0.08] & [0.04] \\
University edu dad & 0.23 & 0.52{***} & 0.33 & 0.21 & 0.37{***} & 0.57{*} & 0.33 & 0.32 & 0.32 & 0.38{*} & 0.29 & 0.28 \\
 & [0.03] & [0.10] & [0.21] & [0.04] & [0.04] & [0.20] & [0.08] & [0.05] & [0.06] & [0.10] & [0.07] & [0.04] \\
Mom born in province & 0.44 & 0.56 & 0.67 & 0.60{**} & 0.58{**} & 0.29 & 0.75{***} & 0.61{**} & 0.56 & 0.54 & 0.54 & 0.82{***} \\
 & [0.04] & [0.10] & [0.21] & [0.05] & [0.04] & [0.18] & [0.07] & [0.05] & [0.06] & [0.10] & [0.08] & [0.03] \\
Dad born in province & 0.45 & 0.70{**} & 0.67 & 0.57{*} & 0.56{*} & 0.57 & 0.67{**} & 0.60{**} & 0.49 & 0.5 & 0.54 & 0.77{***} \\
 & [0.04] & [0.09] & [0.21] & [0.05] & [0.04] & [0.20] & [0.08] & [0.05] & [0.06] & [0.10] & [0.08] & [0.04] \\
Caregiver is religious & 0.82 & 0.81 & 0.67 & 0.9 & 0.85 & 0.57 & 0.89 & 0.91{*} & 0.72 & 0.92 & 0.88 & 0.79 \\
 & [0.03] & [0.08] & [0.21] & [0.03] & [0.03] & [0.20] & [0.05] & [0.03] & [0.05] & [0.05] & [0.05] & [0.03] \\
Own Home & 0.55 & 0.74{*} & 0.5 & 0.58 & 0.70{***} & 0.43 & 0.67 & 0.76{***} & 0.66 & 0.69 & 0.61 & 0.66{**} \\
 & [0.04] & [0.09] & [0.22] & [0.05] & [0.04] & [0.20] & [0.08] & [0.04] & [0.06] & [0.09] & [0.08] & [0.04] \\
Income 5k-10k eur & 0.01 & 0.00 & 0.00 & 0.02 & 0.03 & 0.00 & 0.03 & 0.00 & 0.00 & 0.00 & 0.02 & 0.02 \\
 & [0.01] & [0.00] & [0.00] & [0.01] & [0.01] & [0.00] & [0.03] & [0.00] & [0.00] & [0.00] & [0.02] & [0.01] \\
Income 10k-25k eur & 0.21 & 0.11 & 0.17 & 0.15 & 0.17 & 0.00 & 0.19 & 0.22 & 0.22 & 0.19 & 0.2 & 0.10{**} \\
 & [0.03] & [0.06] & [0.17] & [0.03] & [0.03] & [0.00] & [0.07] & [0.04] & [0.05] & [0.08] & [0.06] & [0.03] \\
Income 25k-50k eur & 0.38 & 0.33 & 0.00 & 0.25{**} & 0.44 & 0.43 & 0.36 & 0.36 & 0.43 & 0.38 & 0.39 & 0.23{***} \\
 & [0.04] & [0.09] & [0.00] & [0.04] & [0.04] & [0.20] & [0.08] & [0.05] & [0.06] & [0.10] & [0.08] & [0.04] \\
Income 50k-100k eur & 0.21 & 0.3 & 0.33 & 0.15 & 0.21 & 0.00 & 0.19 & 0.17 & 0.16 & 0.23 & 0.24 & 0.06{***} \\
 & [0.03] & [0.09] & [0.21] & [0.03] & [0.03] & [0.00] & [0.07] & [0.04] & [0.04] & [0.08] & [0.07] & [0.02] \\
Income 100k-250k eur & 0.03 & 0.04 & 0.00 & 0.02 & 0.03 & 0.00 & 0.03 & 0.01 & 0.04 & 0.04 & 0.00 & 0.03 \\
 & [0.01] & [0.04] & [0.00] & [0.01] & [0.01] & [0.00] & [0.03] & [0.01] & [0.03] & [0.04] & [0.00] & [0.01] \\
%Income more 250k eur & 0.00 & 0.00 & 0.00 & 0.00 & 0.00 & 0.00 & 0.00 & 0.00 & 0.00 & 0.00 & 0.00 & 0 \\
% & [0.00] & [0.00] & [0.00] & [0.00] & [0.00] & [0.00] & [0.00] & [0.00] & [0.00] & [0.00] & [0.00] & [0.00] \\
%Income below 5k eur & 0.01 & 0.04 & 0.00 & 0.01 & 0.01 & 0.00 & 0.03 & 0.04 & 0.01 & 0.00 & 0.00 & 0.06{*} \\
% & [0.01] & [0.04] & [0.00] & [0.01] & [0.01] & [0.00] & [0.03] & [0.02] & [0.01] & [0.00] & [0.00] & [0.02] \\
Low birthweight & 0.12 & 0.00 & 0.17 & 0.04{**} & 0.05{**} & 0.00 & 0.03 & 0.11 & 0.09 & 0.04 & 0.05 & 0.03{***} \\
 & [0.03] & [0.00] & [0.17] & [0.02] & [0.02] & [0.00] & [0.03] & [0.03] & [0.03] & [0.04] & [0.03] & [0.01] \\
Premature & 0.14 & 0.07 & 0.17 & 0.04{**} & 0.06{**} & 0.00 & 0.11 & 0.09 & 0.15 & 0.04 & 0.02{*} & 0.05{**} \\
 & [0.03] & [0.05] & [0.17] & [0.02] & [0.02] & [0.00] & [0.05] & [0.03] & [0.04] & [0.04] & [0.02] & [0.02]\\ \hline


% it contains the notes, assuming they are the same for all the tables.
\end{tabular}

\end{adjustbox}
\raggedright{
\footnotesize{Average of baseline characteristcs, by city and type of child-care attended. Standard errors of means in brackets. Test for difference in means between each column and the first column (Reggio Municipal, the treatment group) was performed; *** significant difference at 1\%, ** significant difference at 5\%, * significant difference at 10\%. Source: authors calculation using survey data.}
}
\end{table}
  
%\end{table}

%% adolescents, infant-toddler
\begin{table}[H]
\caption{Baseline characteristics by infant-toddler-center type, adolescents (age 18)}
% this is the top part of the tables that display the summary of the baseline characteristics, by city and child-care type
\centering
\begin{adjustbox}{width=1.2\textwidth,center=\textwidth}
\small
\begin{tabular}{m{4.0cm} cccccccccccc}
\hline \hline 
 & Reggio & Reggio & Reggio & Reggio & Parma & Parma & Parma & Parma & Padova & Padova & Padova & Padova \\
 & Municipal & Religious & Private & Not Attended & Municipal & Religious & Private & Not Attended & Municipal & Religious & Private & Not Attended \\

\hline 
  
CAPI & 0.44 & 0.56 & 0.33 & 0.39 & 0.53 & 0.70 & 0.85*** & 0.52 & 0.43 & 0.50 & & 0.50\\
  &  [0.04]  &  [0.18]  &  [0.33]  &  [0.04]  &  [0.05]  &  [0.15]  &  [0.10]  &  [0.04]  &  [0.06]  &  [0.19]  & . & [0.03] \\
Male dummy  &  0.42  &  0.44  &  0.67  &  0.41  &  0.46  &  0.60  &  0.54  &  0.41  &  0.49  &  0.62  &  . & 0.47 \\
  &  [0.04]  &  [0.18]  &  [0.33]  &  [0.04]  &  [0.05]  &  [0.16]  &  [0.14]  &  [0.04]  &  [0.06]  &  [0.18]  & . & [0.03] \\
Age  &  18.71  &  18.83  &  18.96  &  18.69  &  18.84***  &  18.86  &  18.79  &  18.71  &  18.69  &  18.74  &  . & 18.72 \\
  &  [0.03]  &  [0.06]  &  [0.05]  &  [0.03]  &  [0.03]  &  [0.13]  &  [0.11]  &  [0.03]  &  [0.04]  &  [0.11]  & . & [0.02] \\
%Age sq.  &  350.15  &  354.75  &  359.37  &  349.53  &  355.15***  &  355.92  &  353.15  &  350.11  &  349.45  &  351.26  & . & 350.42 \\
%  &  [0.99]  &  [2.16]  &  [1.90]  &  [1.18]  &  [1.23]  &  [4.83]  &  [3.96]  &  [1.12]  &  [1.67]  &  [4.11]  & . & [0.90] \\
Middle school edu mom  &  0.08  &  0.11  &  0.00  &  0.10  &  0.15*  &  0.00  &  0.00  &  0.08  &  0.08  &  0.00  & . & 0.12 \\
  &  [0.02]  &  [0.11]  &  [0.00]  &  [0.03]  &  [0.04]  &  [0.00]  &  [0.00]  &  [0.02]  &  [0.04]  &  [0.00]  & . & [0.02] \\
High school edu mom  &  0.51  &  0.44  &  0.67  &  0.45  &  0.38*  &  0.60  &  0.23*  &  0.50  &  0.39  &  0.62  & . & 0.44 \\
  &  [0.04]  &  [0.18]  &  [0.33]  &  [0.04]  &  [0.05]  &  [0.16]  &  [0.12]  &  [0.04]  &  [0.06]  &  [0.18]  & . & [0.03] \\
University edu mom  &  0.29  &  0.44  &  0.33  &  0.20*  &  0.35  &  0.20  &  0.69***  &  0.30  &  0.46**  &  0.12  & . & 0.25 \\
  &  [0.04]  &  [0.18]  &  [0.33]  &  [0.04]  &  [0.05]  &  [0.13]  &  [0.13]  &  [0.04]  &  [0.06]  &  [0.12]  & . & [0.03] \\
Middle school edu dad  &  0.05  &  0.11  &  0.00  &  0.12*  &  0.10  &  0.00  &  0.00  &  0.06  &  0.13*  &  0.00  & . & 0.10 \\
  &  [0.02]  &  [0.11]  &  [0.00]  &  [0.03]  &  [0.03]  &  [0.00]  &  [0.00]  &  [0.02]  &  [0.04]  &  [0.00]  & . & [0.02] \\
High school edu dad  &  0.42  &  0.33  &  0.67  &  0.38  &  0.35  &  0.60  &  0.15*  &  0.38  &  0.36  &  0.38  & . & 0.40 \\
  &  [0.04]  &  [0.17]  &  [0.33]  &  [0.04]  &  [0.05]  &  [0.16]  &  [0.10]  &  [0.04]  &  [0.06]  &  [0.18]  & . & [0.03] \\
University edu dad  &  0.22  &  0.44  &  0.33  &  0.13*  &  0.25  &  0.10  &  0.54**  &  0.25  &  0.30  &  0.25  & . & 0.28 \\
  &  [0.03]  &  [0.18]  &  [0.33]  &  [0.03]  &  [0.04]  &  [0.10]  &  [0.14]  &  [0.04]  &  [0.06]  &  [0.16]  & . & [0.03] \\
Mom born in province  &  0.75  &  0.67  &  0.33  &  0.61**  &  0.61**  &  0.70  &  0.92  &  0.71  &  0.67  &  0.75  & . & 0.81 \\
  &  [0.04]  &  [0.17]  &  [0.33]  &  [0.04]  &  [0.05]  &  [0.15]  &  [0.08]  &  [0.04]  &  [0.06]  &  [0.16]  & . & [0.03] \\
Dad born in province  &  0.62  &  0.78  &  0.33  &  0.55  &  0.53  &  0.80  &  0.69  &  0.64  &  0.62  &  0.50  & . & 0.78*** \\
  &  [0.04]  &  [0.15]  &  [0.33]  &  [0.04]  &  [0.05]  &  [0.13]  &  [0.13]  &  [0.04]  &  [0.06]  &  [0.19]  & . & [0.03] \\
Caregiver is religious  &  0.74  &  1.00  &  1.00  &  0.79  &  0.86**  &  0.80  &  0.85  &  0.89***  &  0.72  &  0.38**  & . & 0.76 \\
  &  [0.04]  &  [0.00]  &  [0.00]  &  [0.04]  &  [0.04]  &  [0.13]  &  [0.10]  &  [0.03]  &  [0.06]  &  [0.18]  & . & [0.03] \\
Own Home  &  0.89  &  1.00  &  1.00  &  0.78**  &  0.77**  &  0.80  &  1.00  &  0.84  &  0.80  &  0.75  & . & 0.76*** \\
  &  [0.03]  &  [0.00]  &  [0.00]  &  [0.04]  &  [0.04]  &  [0.13]  &  [0.00]  &  [0.03]  &  [0.05]  &  [0.16]  & . & [0.03] \\
Income 5k-10k eur  &  0.01  &  0.00  &  0.00  &  0.01  &  0.00  &  0.00  &  0.00  &  0.02  &  0.00  &  0.00  & . & 0.01 \\
  &  [0.01]  &  [0.00]  &  [0.00]  &  [0.01]  &  [0.00]  &  [0.00]  &  [0.00]  &  [0.01]  &  [0.00]  &  [0.00]  & . & [0.01] \\
Income 10k-25k eur  &  0.16  &  0.22  &  0.33  &  0.20  &  0.19  &  0.00  &  0.23  &  0.16  &  0.11  &  0.00  &  . & 0.10* \\
  &  [0.03]  &  [0.15]  &  [0.33]  &  [0.04]  &  [0.04]  &  [0.00]  &  [0.12]  &  [0.03]  &  [0.04]  &  [0.00]  & . & [0.02] \\
Income 25k-50k eur  &  0.33  &  0.33  &  0.33  &  0.32  &  0.31  &  0.80***  &  0.23  &  0.25  &  0.33  &  0.12  & . & 0.22** \\
  &  [0.04]  &  [0.17]  &  [0.33]  &  [0.04]  &  [0.05]  &  [0.13]  &  [0.12]  &  [0.04]  &  [0.06]  &  [0.12]  & . & [0.03] \\
Income 50k-100k eur  &  0.29  &  0.00  &  0.00  &  0.18*  &  0.28  &  0.20  &  0.38  &  0.21  &  0.15**  &  0.12  & . & 0.10*** \\
  &  [0.04]  &  [0.00]  &  [0.00]  &  [0.03]  &  [0.05]  &  [0.13]  &  [0.14]  &  [0.04]  &  [0.05]  &  [0.12]  & . & [0.02] \\
Income 100k-250k eur  &  0.05  &  0.22*  &  0.00  &  0.03  &  0.02  &  0.00  &  0.00  &  0.05  &  0.00  &  0.00  & . & 0.03 \\
  &  [0.02]  &  [0.15]  &  [0.00]  &  [0.02]  &  [0.01]  &  [0.00]  &  [0.00]  &  [0.02]  &  [0.00]  &  [0.00]  & . & [0.01] \\
%Income more 250k eur  &  0.01  &  0.00  &  0.00  &  0.00  &  0.00  &  0.00  &  0.00  &  0.00  &  0.00  &  0.00  & . & 0.00 \\
%  &  [0.01]  &  [0.00]  &  [0.00]  &  [0.00]  &  [0.00]  &  [0.00]  &  [0.00]  &  [0.00]  &  [0.00]  &  [0.00]  & . & [0.00] \\
%Income below 5k eur  &  0.00  &  0.00  &  0.00  &  0.01  &  0.03*  &  0.00  &  0.00  &  0.02*  &  0.03*  &  0.00  & . & 0.04** \\
%  &  [0.00]  &  [0.00]  &  [0.00]  &  [0.01]  &  [0.02]  &  [0.00]  &  [0.00]  &  [0.01]  &  [0.02]  &  [0.00]  & . & [0.01] \\
Low birthweight  &  0.05  &  0.11  &  0.00  &  0.05  &  0.08  &  0.00  &  0.00  &  0.06  &  0.07  &  0.00  & . & 0.04 \\
  &  [0.02]  &  [0.11]  &  [0.00]  &  [0.02]  &  [0.03]  &  [0.00]  &  [0.00]  &  [0.02]  &  [0.03]  &  [0.00]  & . & [0.01] \\
Premature  &  0.06  &  0.11  &  0.00  &  0.07  &  0.13*  &  0.00  &  0.00  &  0.09  &  0.10  &  0.12  & . & 0.06 \\
 & [0.02] & [0.11] & [0.00] & [0.02] & [0.03] & [0.00] & [0.00] & [0.03] & [0.04] & [0.12] & & [0.02]\\ \hline


% it contains the notes, assuming they are the same for all the tables.
\end{tabular}

\end{adjustbox}
\raggedright{
\footnotesize{Average of baseline characteristcs, by city and type of child-care attended. Standard errors of means in brackets. Test for difference in means between each column and the first column (Reggio Municipal, the treatment group) was performed; *** significant difference at 1\%, ** significant difference at 5\%, * significant difference at 10\%. Source: authors calculation using survey data.}
}
\end{table}
  
%\end{table}

%% adults, infant-toddler
\begin{table}[H]
\caption{Baseline characteristics by infant-toddler-center type, adults (age 30-50)}
% this is the top part of the tables that display the summary of the baseline characteristics, by city and child-care type
\centering
\begin{adjustbox}{width=1.2\textwidth,center=\textwidth}
\small
\begin{tabular}{m{4.0cm} cccccccccccc}
\hline \hline 
 & Reggio & Reggio & Reggio & Reggio & Parma & Parma & Parma & Parma & Padova & Padova & Padova & Padova \\
 & Municipal & Religious & Private & Not Attended & Municipal & Religious & Private & Not Attended & Municipal & Religious & Private & Not Attended \\

\hline 
  
CAPI & 0.69 & 0.25 & 0.33 & 0.54*** & 0.30*** & 0.53 & 0.36** & 0.37*** & 0.50** & 0.28*** & 0.00 & 0.33***\\
  &  [0.05]  &  [0.25]  &  [0.33]  &  [0.02]  &  [0.05]  &  [0.13]  &  [0.15]  &  [0.02]  &  [0.08]  &  [0.11]  &  [0.00]  &  [0.02] \\
Male dummy  &  0.62  &  0.25  &  1.00  &  0.53  &  0.44**  &  0.53  &  0.36  &  0.49**  &  0.63  &  0.50  &  0.33  &  0.49** \\
  &  [0.05]  &  [0.25]  &  [0.00]  &  [0.02]  &  [0.06]  &  [0.13]  &  [0.15]  &  [0.02]  &  [0.08]  &  [0.12]  &  [0.33]  &  [0.02] \\
Age  &  36.81  &  39.56  &  32.92  &  44.03***  &  40.21**  &  41.39  &  33.91  &  41.65***  &  38.14  &  44.54***  &  32.77  &  43.01*** \\
  &  [0.54]  &  [6.44]  &  [0.20]  &  [0.37]  &  [1.00]  &  [2.80]  &  [1.08]  &  [0.37]  &  [0.91]  &  [2.19]  &  [0.39]  &  [0.38] \\
%Age sq.  &  1382.43  &  1689.38  &  1084.10  &  2027.19***  &  1693.41**  &  1823.06  &  1161.36  &  1804.89***  &  1485.28  &  2065.43***  &  1074.13  &  1935.41*** \\
%  &  [41.05]  &  [592.43]  &  [13.06]  &  [33.21]  &  [88.03]  &  [250.51]  &  [83.17]  &  [32.88]  &  [69.12]  &  [195.13]  &  [25.30]  &  [34.33] \\
Middle school edu mom  &  0.19  &  0.25  &  0.33  &  0.19  &  0.39***  &  0.27  &  0.00  &  0.20  &  0.11  &  0.28  &  0.00  &  0.28* \\
  &  [0.04]  &  [0.25]  &  [0.33]  &  [0.02]  &  [0.06]  &  [0.12]  &  [0.00]  &  [0.02]  &  [0.05]  &  [0.11]  &  [0.00]  &  [0.02] \\
High school edu mom  &  0.43  &  0.75  &  0.00  &  0.43  &  0.21***  &  0.20  &  0.36  &  0.33*  &  0.42  &  0.22  &  0.33  &  0.31** \\
  &  [0.05]  &  [0.25]  &  [0.00]  &  [0.02]  &  [0.05]  &  [0.11]  &  [0.15]  &  [0.02]  &  [0.08]  &  [0.10]  &  [0.33]  &  [0.02] \\
University edu mom  &  0.39  &  0.00  &  0.67  &  0.37  &  0.40  &  0.47  &  0.64  &  0.46  &  0.39  &  0.50  &  0.67  &  0.40 \\
  &  [0.05]  &  [0.00]  &  [0.33]  &  [0.02]  &  [0.06]  &  [0.13]  &  [0.15]  &  [0.02]  &  [0.08]  &  [0.12]  &  [0.33]  &  [0.02] \\
Middle school edu dad  &  0.20  &  0.25  &  0.33  &  0.17  &  0.38**  &  0.27  &  0.00  &  0.20  &  0.11  &  0.28  &  0.00  &  0.22 \\
  &  [0.04]  &  [0.25]  &  [0.33]  &  [0.01]  &  [0.06]  &  [0.12]  &  [0.00]  &  [0.02]  &  [0.05]  &  [0.11]  &  [0.00]  &  [0.02] \\
High school edu dad  &  0.34  &  0.50  &  0.00  &  0.41  &  0.23  &  0.07**  &  0.18  &  0.32  &  0.34  &  0.11*  &  0.00  &  0.27 \\
  &  [0.05]  &  [0.29]  &  [0.00]  &  [0.02]  &  [0.05]  &  [0.07]  &  [0.12]  &  [0.02]  &  [0.08]  &  [0.08]  &  [0.00]  &  [0.02] \\
University edu dad  &  0.45  &  0.25  &  0.67  &  0.41  &  0.39  &  0.60  &  0.82**  &  0.47  &  0.45  &  0.61  &  1.00  &  0.51 \\
  &  [0.05]  &  [0.25]  &  [0.33]  &  [0.02]  &  [0.06]  &  [0.13]  &  [0.12]  &  [0.02]  &  [0.08]  &  [0.12]  &  [0.00]  &  [0.02] \\
Mom born in province  &  0.82  &  1.00  &  1.00  &  0.81  &  0.78  &  0.80  &  0.55**  &  0.73*  &  0.66*  &  0.56**  &  1.00  &  0.69** \\
  &  [0.04]  &  [0.00]  &  [0.00]  &  [0.02]  &  [0.05]  &  [0.11]  &  [0.16]  &  [0.02]  &  [0.08]  &  [0.12]  &  [0.00]  &  [0.02] \\
Dad born in province  &  0.89  &  1.00  &  1.00  &  0.82  &  0.84  &  0.73  &  0.55**  &  0.79**  &  0.66***  &  0.78  &  0.33**  &  0.77** \\
  &  [0.03]  &  [0.00]  &  [0.00]  &  [0.02]  &  [0.04]  &  [0.12]  &  [0.16]  &  [0.02]  &  [0.08]  &  [0.10]  &  [0.33]  &  [0.02] \\
Caregiver is religious  &  0.00  &  0.00  &  0.00  &  0.00  &  0.00  &  0.00  &  0.00  &  0.00  &  0.00  &  0.00  &  0.00  &  0.00 \\
  &  [0.00]  &  [0.00]  &  [0.00]  &  [0.00]  &  [0.00]  &  [0.00]  &  [0.00]  &  [0.00]  &  [0.00]  &  [0.00]  &  [0.00]  &  [0.00] \\
%Own Home  &  0.31  &  0.50  &  0.33  &  0.59***  &  0.68***  &  0.73***  &  0.64**  &  0.61***  &  0.87***  &  0.78***  &  0.67  &  0.66*** \\
%  &  [0.05]  &  [0.29]  &  [0.33]  &  [0.02]  &  [0.05]  &  [0.12]  &  [0.15]  &  [0.02]  &  [0.06]  &  [0.10]  &  [0.33]  &  [0.02] \\
%Income 5k-10k eur  &  0.00  &  0.00  &  0.00  &  0.00  &  0.00  &  0.00  &  0.00  &  0.00  &  0.00  &  0.00  &  0.00  &  0.00 \\
%  &  [0.00]  &  [0.00]  &  [0.00]  &  [0.00]  &  [0.00]  &  [0.00]  &  [0.00]  &  [0.00]  &  [0.00]  &  [0.00]  &  [0.00]  &  [0.00] \\
%Income 10k-25k eur  &  0.00  &  0.00  &  0.00  &  0.00  &  0.00  &  0.00  &  0.00  &  0.00  &  0.00  &  0.00  &  0.00  &  0.00 \\
%  &  [0.00]  &  [0.00]  &  [0.00]  &  [0.00]  &  [0.00]  &  [0.00]  &  [0.00]  &  [0.00]  &  [0.00]  &  [0.00]  &  [0.00]  &  [0.00] \\
%Income 25k-50k eur  &  0.00  &  0.00  &  0.00  &  0.00  &  0.00  &  0.00  &  0.00  &  0.00  &  0.00  &  0.00  &  0.00  &  0.00 \\
%  &  [0.00]  &  [0.00]  &  [0.00]  &  [0.00]  &  [0.00]  &  [0.00]  &  [0.00]  &  [0.00]  &  [0.00]  &  [0.00]  &  [0.00]  &  [0.00] \\
%Income 50k-100k eur  &  0.00  &  0.00  &  0.00  &  0.00  &  0.00  &  0.00  &  0.00  &  0.00  &  0.00  &  0.00  &  0.00  &  0.00 \\
%  &  [0.00]  &  [0.00]  &  [0.00]  &  [0.00]  &  [0.00]  &  [0.00]  &  [0.00]  &  [0.00]  &  [0.00]  &  [0.00]  &  [0.00]  &  [0.00] \\
%Income 100k-250k eur  &  0.00  &  0.00  &  0.00  &  0.00  &  0.00  &  0.00  &  0.00  &  0.00  &  0.00  &  0.00  &  0.00  &  0.00 \\
%  &  [0.00]  &  [0.00]  &  [0.00]  &  [0.00]  &  [0.00]  &  [0.00]  &  [0.00]  &  [0.00]  &  [0.00]  &  [0.00]  &  [0.00]  &  [0.00] \\
%%Income more 250k eur  &  0.00  &  0.00  &  0.00  &  0.00  &  0.00  &  0.00  &  0.00  &  0.00  &  0.00  &  0.00  &  0.00  &  0.00 \\
%%  &  [0.00]  &  [0.00]  &  [0.00]  &  [0.00]  &  [0.00]  &  [0.00]  &  [0.00]  &  [0.00]  &  [0.00]  &  [0.00]  &  [0.00]  &  [0.00] \\
%%Income below 5k eur  &  0.00  &  0.00  &  0.00  &  0.00  &  0.00  &  0.00  &  0.00  &  0.00  &  0.00  &  0.00  &  0.00  &  0.00 \\
%%  &  [0.00]  &  [0.00]  &  [0.00]  &  [0.00]  &  [0.00]  &  [0.00]  &  [0.00]  &  [0.00]  &  [0.00]  &  [0.00]  &  [0.00]  &  [0.00] \\
%Low birthweight  &  0.00  &  0.00  &  0.00  &  0.00  &  0.00  &  0.00  &  0.00  &  0.00  &  0.00  &  0.00  &  0.00  &  0.00 \\
% & [0.00] & [0.00] & [0.00] & [0.00] & [0.00] & [0.00] & [0.00] & [0.00] & [0.00] & [0.00] & [0.00] & [0.00]\\ 
\hline


% it contains the notes, assuming they are the same for all the tables.
\end{tabular}

\end{adjustbox}
\raggedright{
\footnotesize{Average of baseline characteristcs, by city and type of child-care attended. Standard errors of means in brackets. Test for difference in means between each column and the first column (Reggio Municipal, the treatment group) was performed; *** significant difference at 1\%, ** significant difference at 5\%, * significant difference at 10\%. Source: authors calculation using survey data.}
}
\end{table}
  
%\end{table}

%%% children, preschool
\begin{table}[H]
\caption{Baseline characteristics by preschool type, children (age 6)}
% this is the top part of the tables that display the summary of the baseline characteristics, by city and child-care type
\centering
\begin{adjustbox}{width=1.2\textwidth,center=\textwidth}
\small
\begin{tabular}{m{4.0cm} ccccccccccccccc}
\hline \hline 
 & Reggio & Reggio & Reggio & Reggio & Reggio & Parma & Parma & Parma & Parma & Parma & Padova & Padova & Padova & Padova & Padova \\
 & Municipal & State & Religious & Private & Not Attended & Municipal & State & Religious & Private & Not Attended & Municipal & State & Religious & Private & Not Attended \\

\hline 
  
CAPI  &  0.60  &  0.40**  &  0.55  &  0.40  &  0.50  &  0.42***  &  0.37***  &  0.44**  &  0.78  &  0.50  &  0.45**  &  0.55  &  0.48**  &  0.42  &  0.00 \\
  &  [0.04]  &  [0.07]  &  [0.05]  &  [0.24]  &  [0.50]  &  [0.04]  &  [0.07]  &  [0.06]  &  [0.15]  &  [0.22]  &  [0.06]  &  [0.08]  &  [0.04]  &  [0.15]  &  [0.00] \\
Male dummy  &  0.55  &  0.53  &  0.53  &  0.40  &  0.50  &  0.54  &  0.56  &  0.57  &  0.67  &  0.67  &  0.59  &  0.62  &  0.48  &  0.25*  &  0.50 \\
  &  [0.04]  &  [0.08]  &  [0.05]  &  [0.24]  &  [0.50]  &  [0.04]  &  [0.08]  &  [0.06]  &  [0.17]  &  [0.21]  &  [0.05]  &  [0.08]  &  [0.04]  &  [0.13]  &  [0.50] \\
Age  &  6.80  &  6.92*  &  6.74  &  6.69  &  6.93  &  6.70***  &  6.71  &  6.70**  &  6.83  &  6.80  &  6.66***  &  6.66**  &  6.67***  &  6.74  &  6.39* \\
  &  [0.03]  &  [0.05]  &  [0.04]  &  [0.18]  &  [0.16]  &  [0.03]  &  [0.05]  &  [0.04]  &  [0.09]  &  [0.14]  &  [0.04]  &  [0.06]  &  [0.03]  &  [0.10]  &  [0.04] \\
%Age sq.  &  46.39  &  47.93*  &  45.58  &  44.92  &  48.01  &  44.99***  &  45.17  &  44.96**  &  46.76  &  46.38  &  44.40***  &  44.46**  &  44.60***  &  45.51  &  40.78* \\
%  &  [0.38]  &  [0.68]  &  [0.54]  &  [2.48]  &  [2.28]  &  [0.35]  &  [0.63]  &  [0.51]  &  [1.18]  &  [1.91]  &  [0.48]  &  [0.83]  &  [0.39]  &  [1.32]  &  [0.54] \\
Middle school edu mom  &  0.08  &  0.02  &  0.11  &  0.00  &  0.00  &  0.06  &  0.02  &  0.08  &  0.00  &  0.00  &  0.06  &  0.05  &  0.12  &  0.08  &  0.00 \\
  &  [0.02]  &  [0.02]  &  [0.03]  &  [0.00]  &  [0.00]  &  [0.02]  &  [0.02]  &  [0.03]  &  [0.00]  &  [0.00]  &  [0.03]  &  [0.03]  &  [0.03]  &  [0.08]  &  [0.00] \\
High school edu mom  &  0.46  &  0.44  &  0.43  &  0.60  &  0.00  &  0.38  &  0.44  &  0.44  &  0.56  &  0.17  &  0.43  &  0.40  &  0.47  &  0.58  &  0.00 \\
  &  [0.04]  &  [0.07]  &  [0.05]  &  [0.24]  &  [0.00]  &  [0.04]  &  [0.08]  &  [0.06]  &  [0.18]  &  [0.17]  &  [0.05]  &  [0.08]  &  [0.04]  &  [0.15]  &  [0.00] \\
University edu mom  &  0.29  &  0.18  &  0.32  &  0.40  &  0.00  &  0.46***  &  0.53***  &  0.39  &  0.44  &  0.83**  &  0.44**  &  0.35  &  0.32  &  0.33  &  0.50 \\
  &  [0.04]  &  [0.06]  &  [0.05]  &  [0.24]  &  [0.00]  &  [0.04]  &  [0.08]  &  [0.06]  &  [0.18]  &  [0.17]  &  [0.06]  &  [0.08]  &  [0.04]  &  [0.14]  &  [0.50] \\
Middle school edu dad  &  0.08  &  0.11  &  0.05  &  0.00  &  0.50  &  0.08  &  0.12  &  0.13  &  0.11  &  0.00  &  0.09  &  0.05  &  0.11  &  0.00  &  0.00 \\
  &  [0.02]  &  [0.05]  &  [0.02]  &  [0.00]  &  [0.50]  &  [0.02]  &  [0.05]  &  [0.04]  &  [0.11]  &  [0.00]  &  [0.03]  &  [0.03]  &  [0.03]  &  [0.00]  &  [0.00] \\
High school edu dad  &  0.34  &  0.31  &  0.38  &  0.40  &  0.50  &  0.34  &  0.37  &  0.39  &  0.44  &  0.33  &  0.40  &  0.40  &  0.43  &  0.50  &  0.00 \\
  &  [0.04]  &  [0.07]  &  [0.05]  &  [0.24]  &  [0.50]  &  [0.04]  &  [0.07]  &  [0.06]  &  [0.18]  &  [0.21]  &  [0.05]  &  [0.08]  &  [0.04]  &  [0.15]  &  [0.00] \\
University edu dad  &  0.23  &  0.18  &  0.29  &  0.40  &  0.00  &  0.33*  &  0.44**  &  0.35*  &  0.33  &  0.33  &  0.29  &  0.30  &  0.30  &  0.42  &  0.50 \\
  &  [0.03]  &  [0.06]  &  [0.05]  &  [0.24]  &  [0.00]  &  [0.04]  &  [0.08]  &  [0.05]  &  [0.17]  &  [0.21]  &  [0.05]  &  [0.07]  &  [0.04]  &  [0.15]  &  [0.50] \\
Mom born in province  &  0.51  &  0.38  &  0.59  &  0.40  &  1.00  &  0.60  &  0.72**  &  0.56  &  0.67  &  0.50  &  0.62  &  0.70**  &  0.74***  &  0.58  &  0.00 \\
  &  [0.04]  &  [0.07]  &  [0.05]  &  [0.24]  &  [0.00]  &  [0.04]  &  [0.07]  &  [0.06]  &  [0.17]  &  [0.22]  &  [0.05]  &  [0.07]  &  [0.04]  &  [0.15]  &  [0.00] \\
Dad born in province  &  0.51  &  0.44  &  0.57  &  0.40  &  1.00  &  0.58  &  0.70**  &  0.62  &  0.33  &  0.17  &  0.61  &  0.50  &  0.70***  &  0.67  &  0.50 \\
  &  [0.04]  &  [0.07]  &  [0.05]  &  [0.24]  &  [0.00]  &  [0.04]  &  [0.07]  &  [0.06]  &  [0.17]  &  [0.17]  &  [0.05]  &  [0.08]  &  [0.04]  &  [0.14]  &  [0.50] \\
Caregiver is religious  &  0.81  &  0.89  &  0.90*  &  0.80  &  0.50  &  0.86  &  0.91  &  0.86  &  1.00  &  0.83  &  0.76  &  0.75  &  0.83  &  0.92  &  1.00 \\
  &  [0.03]  &  [0.05]  &  [0.03]  &  [0.20]  &  [0.50]  &  [0.03]  &  [0.04]  &  [0.04]  &  [0.00]  &  [0.17]  &  [0.05]  &  [0.07]  &  [0.03]  &  [0.08]  &  [0.00] \\
Own Home  &  0.54  &  0.44  &  0.73***  &  0.40  &  0.50  &  0.69***  &  0.74**  &  0.69**  &  0.78  &  0.83  &  0.63  &  0.60  &  0.68**  &  0.75  &  0.50 \\
  &  [0.04]  &  [0.07]  &  [0.05]  &  [0.24]  &  [0.50]  &  [0.04]  &  [0.07]  &  [0.05]  &  [0.15]  &  [0.17]  &  [0.05]  &  [0.08]  &  [0.04]  &  [0.13]  &  [0.50] \\
Income 5k-10k eur  &  0.01  &  0.00  &  0.02  &  0.00  &  0.00  &  0.01  &  0.00  &  0.04  &  0.00  &  0.00  &  0.05*  &  0.00  &  0.00  &  0.00  &  0.00 \\
  &  [0.01]  &  [0.00]  &  [0.02]  &  [0.00]  &  [0.00]  &  [0.01]  &  [0.00]  &  [0.02]  &  [0.00]  &  [0.00]  &  [0.02]  &  [0.00]  &  [0.00]  &  [0.00]  &  [0.00] \\
Income 10k-25k eur  &  0.17  &  0.22  &  0.16  &  0.00  &  0.00  &  0.21  &  0.21  &  0.13  &  0.00  &  0.33  &  0.13  &  0.25  &  0.16  &  0.00  &  0.00 \\
  &  [0.03]  &  [0.06]  &  [0.04]  &  [0.00]  &  [0.00]  &  [0.03]  &  [0.06]  &  [0.04]  &  [0.00]  &  [0.21]  &  [0.04]  &  [0.07]  &  [0.03]  &  [0.00]  &  [0.00] \\
Income 25k-50k eur  &  0.34  &  0.29  &  0.30  &  0.40  &  0.00  &  0.45*  &  0.37  &  0.31  &  0.44  &  0.67  &  0.34  &  0.25  &  0.33  &  0.17  &  0.00 \\
  &  [0.04]  &  [0.07]  &  [0.05]  &  [0.24]  &  [0.00]  &  [0.04]  &  [0.07]  &  [0.05]  &  [0.18]  &  [0.21]  &  [0.05]  &  [0.07]  &  [0.04]  &  [0.11]  &  [0.00] \\
Income 50k-100k eur  &  0.19  &  0.07*  &  0.27  &  0.20  &  0.00  &  0.17  &  0.26  &  0.23  &  0.11  &  0.00  &  0.17  &  0.10  &  0.13  &  0.00  &  0.00 \\
  &  [0.03]  &  [0.04]  &  [0.05]  &  [0.20]  &  [0.00]  &  [0.03]  &  [0.07]  &  [0.05]  &  [0.11]  &  [0.00]  &  [0.04]  &  [0.05]  &  [0.03]  &  [0.00]  &  [0.00] \\
Income 100k-250k eur  &  0.04  &  0.00  &  0.01  &  0.00  &  0.00  &  0.02  &  0.02  &  0.03  &  0.00  &  0.00  &  0.01  &  0.03  &  0.03  &  0.08  &  0.50* \\
  &  [0.01]  &  [0.00]  &  [0.01]  &  [0.00]  &  [0.00]  &  [0.01]  &  [0.02]  &  [0.02]  &  [0.00]  &  [0.00]  &  [0.01]  &  [0.03]  &  [0.01]  &  [0.08]  &  [0.50] \\
%Income more 250k eur  &  0.00  &  0.00  &  0.00  &  0.00  &  0.00  &  0.00  &  0.00  &  0.00  &  0.00  &  0.00  &  0.00  &  0.00  &  0.00  &  0.00  &  0.00 \\
%  &  [0.00]  &  [0.00]  &  [0.00]  &  [0.00]  &  [0.00]  &  [0.00]  &  [0.00]  &  [0.00]  &  [0.00]  &  [0.00]  &  [0.00]  &  [0.00]  &  [0.00]  &  [0.00]  &  [0.00] \\
%Income below 5k eur  &  0.00  &  0.02  &  0.03**  &  0.00  &  0.00  &  0.02  &  0.02  &  0.03*  &  0.11*  &  0.00  &  0.01  &  0.03  &  0.05***  &  0.00  &  0.00 \\
%  &  [0.00]  &  [0.02]  &  [0.02]  &  [0.00]  &  [0.00]  &  [0.01]  &  [0.02]  &  [0.02]  &  [0.11]  &  [0.00]  &  [0.01]  &  [0.03]  &  [0.02]  &  [0.00]  &  [0.00] \\
Low birthweight  &  0.10  &  0.07  &  0.05  &  0.00  &  0.00  &  0.04**  &  0.14  &  0.08  &  0.11  &  0.00  &  0.07  &  0.05  &  0.03**  &  0.08  &  0.00 \\
  &  [0.02]  &  [0.04]  &  [0.02]  &  [0.00]  &  [0.00]  &  [0.02]  &  [0.05]  &  [0.03]  &  [0.11]  &  [0.00]  &  [0.03]  &  [0.03]  &  [0.01]  &  [0.08]  &  [0.00] \\
Premature  &  0.10  &  0.09  &  0.10  &  0.00  &  0.00  &  0.03**  &  0.14  &  0.12  &  0.22  &  0.00  &  0.06  &  0.07  &  0.08  &  0.00  &  0.00 \\
  &  [0.02]  &  [0.04]  &  [0.03]  &  [0.00]  &  [0.00]  &  [0.01]  &  [0.05]  &  [0.04]  &  [0.15]  &  [0.00]  &  [0.03]  &  [0.04]  &  [0.02]  &  [0.00]  &  [0.00] \\

% it contains the notes, assuming they are the same for all the tables.
\end{tabular}

\end{adjustbox}
\raggedright{
\footnotesize{Average of baseline characteristcs, by city and type of child-care attended. Standard errors of means in brackets. Test for difference in means between each column and the first column (Reggio Municipal, the treatment group) was performed; *** significant difference at 1\%, ** significant difference at 5\%, * significant difference at 10\%. Source: authors calculation using survey data.}
}
\end{table}
  
%\end{table}
%
%%% adolescents, preschool
\begin{table}[H]
\caption{Baseline characteristics by preschool type, adolescents (age 18)}
% this is the top part of the tables that display the summary of the baseline characteristics, by city and child-care type
\centering
\begin{adjustbox}{width=1.2\textwidth,center=\textwidth}
\small
\begin{tabular}{m{4.0cm} ccccccccccccccc}
\hline \hline 
 & Reggio & Reggio & Reggio & Reggio & Reggio & Parma & Parma & Parma & Parma & Parma & Padova & Padova & Padova & Padova & Padova \\
 & Municipal & State & Religious & Private & Not Attended & Municipal & State & Religious & Private & Not Attended & Municipal & State & Religious & Private & Not Attended \\

\hline 
  
CAPI  &  0.47  &  0.41  &  0.38  &  0.33  &  0.43  &  0.53  &  0.47  &  0.59  &  1.00  &  0.50  &  0.43  &  0.55  &  0.53  &  0.33  &  . \\
  &  [0.04]  &  [0.11]  &  [0.05]  &  [0.21]  &  [0.20]  &  [0.05]  &  [0.08]  &  [0.05]  &  [0.00]  &  [0.29]  &  [0.05]  &  [0.07]  &  [0.04]  &  [0.21]  &  . \\
Male dummy  &  0.42  &  0.55  &  0.40  &  0.50  &  0.57  &  0.40  &  0.42  &  0.52  &  0.67  &  0.50  &  0.44  &  0.45  &  0.50  &  0.50  &  . \\
  &  [0.04]  &  [0.11]  &  [0.05]  &  [0.22]  &  [0.20]  &  [0.05]  &  [0.08]  &  [0.06]  &  [0.21]  &  [0.29]  &  [0.05]  &  [0.07]  &  [0.04]  &  [0.22]  &  . \\
Age  &  18.70  &  18.75  &  18.73  &  18.64  &  18.67  &  18.77  &  18.75  &  18.80*  &  18.81  &  18.59  &  18.75  &  18.82**  &  18.64*  &  18.74  &  . \\
  &  [0.03]  &  [0.07]  &  [0.03]  &  [0.14]  &  [0.17]  &  [0.03]  &  [0.05]  &  [0.04]  &  [0.17]  &  [0.08]  &  [0.04]  &  [0.05]  &  [0.03]  &  [0.18]  &  . \\
%Age sq.  &  349.95  &  351.75  &  350.76  &  347.57  &  348.77  &  352.39  &  351.55  &  353.45*  &  354.07  &  345.74  &  351.60  &  354.38**  &  347.68*  &  351.29  &  . \\
%  &  [0.99]  &  [2.69]  &  [1.27]  &  [5.27]  &  [6.49]  &  [1.21]  &  [2.04]  &  [1.36]  &  [6.50]  &  [3.14]  &  [1.39]  &  [1.88]  &  [1.07]  &  [6.84]  &  . \\
Middle school edu mom  &  0.09  &  0.09  &  0.11  &  0.00  &  0.00  &  0.09  &  0.16  &  0.10  &  0.00  &  0.00  &  0.15  &  0.11  &  0.08  &  0.17  &  . \\
  &  [0.02]  &  [0.06]  &  [0.03]  &  [0.00]  &  [0.00]  &  [0.03]  &  [0.06]  &  [0.03]  &  [0.00]  &  [0.00]  &  [0.04]  &  [0.05]  &  [0.02]  &  [0.17]  &  . \\
High school edu mom  &  0.51  &  0.36  &  0.42  &  0.67  &  0.71  &  0.44  &  0.47  &  0.45  &  0.00  &  0.75  &  0.39*  &  0.43  &  0.45  &  0.50  &  . \\
  &  [0.04]  &  [0.10]  &  [0.05]  &  [0.21]  &  [0.18]  &  [0.05]  &  [0.08]  &  [0.06]  &  [0.00]  &  [0.25]  &  [0.05]  &  [0.07]  &  [0.04]  &  [0.22]  &  . \\
University edu mom  &  0.22  &  0.27  &  0.31  &  0.17  &  0.00  &  0.37***  &  0.16  &  0.33*  &  1.00  &  0.25  &  0.31  &  0.34  &  0.27  &  0.33  &  . \\
  &  [0.03]  &  [0.10]  &  [0.05]  &  [0.17]  &  [0.00]  &  [0.05]  &  [0.06]  &  [0.05]  &  [0.00]  &  [0.25]  &  [0.05]  &  [0.07]  &  [0.04]  &  [0.21]  &  . \\
Middle school edu dad  &  0.07  &  0.09  &  0.10  &  0.33*  &  0.00  &  0.09  &  0.12  &  0.02  &  0.00  &  0.00  &  0.11  &  0.13  &  0.08  &  0.17  &  . \\
  &  [0.02]  &  [0.06]  &  [0.03]  &  [0.21]  &  [0.00]  &  [0.03]  &  [0.05]  &  [0.02]  &  [0.00]  &  [0.00]  &  [0.03]  &  [0.05]  &  [0.02]  &  [0.17]  &  . \\
High school edu dad  &  0.40  &  0.41  &  0.40  &  0.50  &  0.14  &  0.35  &  0.40  &  0.38  &  0.17  &  0.25  &  0.32  &  0.47  &  0.41  &  0.17  &  . \\
  &  [0.04]  &  [0.11]  &  [0.05]  &  [0.22]  &  [0.14]  &  [0.04]  &  [0.08]  &  [0.05]  &  [0.17]  &  [0.25]  &  [0.05]  &  [0.07]  &  [0.04]  &  [0.17]  &  . \\
University edu dad  &  0.19  &  0.09  &  0.21  &  0.00  &  0.43  &  0.24  &  0.16  &  0.32**  &  0.50*  &  0.25  &  0.31**  &  0.23  &  0.27  &  0.50*  &  . \\
  &  [0.03]  &  [0.06]  &  [0.04]  &  [0.00]  &  [0.20]  &  [0.04]  &  [0.06]  &  [0.05]  &  [0.22]  &  [0.25]  &  [0.05]  &  [0.06]  &  [0.04]  &  [0.22]  &  . \\
Mom born in province&  0.72  &  0.50**  &  0.64  &  0.83  &  0.57  &  0.68  &  0.74  &  0.66  &  0.67  &  0.25*  &  0.71  &  0.77  &  0.82*  &  1.00  & . \\
  &  [0.03]  &  [0.11]  &  [0.05]  &  [0.17]  &  [0.20]  &  [0.04]  &  [0.07]  &  [0.05]  &  [0.21]  &  [0.25]  &  [0.05]  &  [0.06]  &  [0.03]  &  [0.00]  &  . \\
Dad born in province&  0.61  &  0.45  &  0.57  &  0.67  &  0.43  &  0.58  &  0.70  &  0.63  &  0.50  &  0.00  &  0.65  &  0.72  &  0.79***  &  0.67  &  . \\
  &  [0.04]  &  [0.11]  &  [0.05]  &  [0.21]  &  [0.20]  &  [0.05]  &  [0.07]  &  [0.05]  &  [0.22]  &  [0.00]  &  [0.05]  &  [0.07]  &  [0.04]  &  [0.21]  &  . \\
Caregiver is religious  &  0.69  &  0.73  &  0.91***  &  1.00  &  0.86  &  0.88***  &  0.86**  &  0.87***  &  0.83  &  0.75  &  0.77  &  0.62  &  0.78*  &  0.50  &  . \\
  &  [0.04]  &  [0.10]  &  [0.03]  &  [0.00]  &  [0.14]  &  [0.03]  &  [0.05]  &  [0.04]  &  [0.17]  &  [0.25]  &  [0.04]  &  [0.07]  &  [0.04]  &  [0.22]  &  . \\
Own Home  &  0.86  &  0.77  &  0.85  &  0.83  &  0.43**  &  0.79  &  0.81  &  0.83  &  0.83  &  1.00  &  0.75**  &  0.64***  &  0.82  &  0.83  &  . \\
  &  [0.03]  &  [0.09]  &  [0.04]  &  [0.17]  &  [0.20]  &  [0.04]  &  [0.06]  &  [0.04]  &  [0.17]  &  [0.00]  &  [0.04]  &  [0.07]  &  [0.03]  &  [0.17]  &  . \\
Income 5k-10k eur  &  0.01  &  0.00  &  0.01  &  0.00  &  0.00  &  0.01  &  0.00  &  0.00  &  0.00  &  0.00  &  0.00  &  0.02  &  0.01  &  0.00  &  . \\
  &  [0.01]  &  [0.00]  &  [0.01]  &  [0.00]  &  [0.00]  &  [0.01]  &  [0.00]  &  [0.00]  &  [0.00]  &  [0.00]  &  [0.00]  &  [0.02]  &  [0.01]  &  [0.00]  &  . \\
Income 10k-25k eur  &  0.17  &  0.36**  &  0.17  &  0.00  &  0.29  &  0.16  &  0.28  &  0.12  &  0.50*  &  0.25  &  0.12  &  0.09  &  0.11  &  0.00  &  . \\
  &  [0.03]  &  [0.10]  &  [0.04]  &  [0.00]  &  [0.18]  &  [0.03]  &  [0.07]  &  [0.04]  &  [0.22]  &  [0.25]  &  [0.03]  &  [0.04]  &  [0.03]  &  [0.00]  &  . \\
Income 25k-50k eur  &  0.32  &  0.27  &  0.34  &  0.17  &  0.43  &  0.25  &  0.37  &  0.30  &  0.50  &  0.25  &  0.30  &  0.09***  &  0.27  &  0.17  &  . \\
  &  [0.04]  &  [0.10]  &  [0.05]  &  [0.17]  &  [0.20]  &  [0.04]  &  [0.07]  &  [0.05]  &  [0.22]  &  [0.25]  &  [0.05]  &  [0.04]  &  [0.04]  &  [0.17]  &  . \\
Income 50k-100k eur  &  0.25  &  0.09  &  0.25  &  0.33  &  0.00  &  0.26  &  0.23  &  0.26  &  0.00  &  0.00  &  0.10***  &  0.13*  &  0.11***  &  0.17  &  . \\
  &  [0.03]  &  [0.06]  &  [0.04]  &  [0.21]  &  [0.00]  &  [0.04]  &  [0.07]  &  [0.05]  &  [0.00]  &  [0.00]  &  [0.03]  &  [0.05]  &  [0.03]  &  [0.17]  &  . \\
Income 100k-250k eur  &  0.04  &  0.00  &  0.06  &  0.00  &  0.00  &  0.03  &  0.00  &  0.05  &  0.00  &  0.00  &  0.01  &  0.00  &  0.05  &  0.00  &  . \\
  &  [0.02]  &  [0.00]  &  [0.02]  &  [0.00]  &  [0.00]  &  [0.01]  &  [0.00]  &  [0.02]  &  [0.00]  &  [0.00]  &  [0.01]  &  [0.00]  &  [0.02]  &  [0.00]  &  . \\
%Income more 250k eur  &  0.00  &  0.00  &  0.01  &  0.00  &  0.00  &  0.00  &  0.00  &  0.00  &  0.00  &  0.00  &  0.00  &  0.00  &  0.00  &  0.00  &  . \\
%  &  [0.00]  &  [0.00]  &  [0.01]  &  [0.00]  &  [0.00]  &  [0.00]  &  [0.00]  &  [0.00]  &  [0.00]  &  [0.00]  &  [0.00]  &  [0.00]  &  [0.00]  &  [0.00]  &  . \\
%Income below 5k eur  &  0.00  &  0.00  &  0.01  &  0.00  &  0.00  &  0.03*  &  0.02  &  0.02  &  0.00  &  0.00  &  0.03**  &  0.02  &  0.05***  &  0.00  &  . \\
%  &  [0.00]  &  [0.00]  &  [0.01]  &  [0.00]  &  [0.00]  &  [0.01]  &  [0.02]  &  [0.02]  &  [0.00]  &  [0.00]  &  [0.02]  &  [0.02]  &  [0.02]  &  [0.00]  &  . \\
Low birthweight  &  0.05  &  0.00  &  0.05  &  0.17  &  0.14  &  0.06  &  0.05  &  0.07  &  0.00  &  0.25  &  0.06  &  0.06  &  0.02  &  0.00  &  . \\
  &  [0.02]  &  [0.00]  &  [0.02]  &  [0.17]  &  [0.14]  &  [0.02]  &  [0.03]  &  [0.03]  &  [0.00]  &  [0.25]  &  [0.03]  &  [0.04]  &  [0.01]  &  [0.00]  &  . \\
Premature  &  0.04  &  0.09  &  0.08  &  0.17  &  0.14  &  0.09*  &  0.07  &  0.12**  &  0.00  &  0.25  &  0.10  &  0.09  &  0.04  &  0.00  &  . \\
  &  [0.02]  &  [0.06]  &  [0.03]  &  [0.17]  &  [0.14]  &  [0.03]  &  [0.04]  &  [0.04]  &  [0.00]  &  [0.25]  &  [0.03]  &  [0.04]  &  [0.02]  &  [0.00]  &  . \\

\hline
% it contains the notes, assuming they are the same for all the tables.
\end{tabular}

\end{adjustbox}
\raggedright{
\footnotesize{Average of baseline characteristcs, by city and type of child-care attended. Standard errors of means in brackets. Test for difference in means between each column and the first column (Reggio Municipal, the treatment group) was performed; *** significant difference at 1\%, ** significant difference at 5\%, * significant difference at 10\%. Source: authors calculation using survey data.}
}
\end{table}
  
%\end{table}
%
%%% adults, preschool
\begin{table}[H]
\caption{Baseline characteristics by preschool type, adults (age 30-50)}
% this is the top part of the tables that display the summary of the baseline characteristics, by city and child-care type
\centering
\begin{adjustbox}{width=1.2\textwidth,center=\textwidth}
\small
\begin{tabular}{m{4.0cm} ccccccccccccccc}
\hline \hline 
 & Reggio & Reggio & Reggio & Reggio & Reggio & Parma & Parma & Parma & Parma & Parma & Padova & Padova & Padova & Padova & Padova \\
 & Municipal & State & Religious & Private & Not Attended & Municipal & State & Religious & Private & Not Attended & Municipal & State & Religious & Private & Not Attended \\

\hline 
  
CAPI  &  0.63  &  0.59  &  0.59  &  0.62  &  0.47***  &  0.25***  &  0.38***  &  0.54  &  0.50  &  0.34***  &  0.32***  &  0.56  &  0.29***  &  0.00  &  0.35*** \\
  &  [0.03]  &  [0.07]  &  [0.05]  &  [0.18]  &  [0.03]  &  [0.03]  &  [0.05]  &  [0.05]  &  [0.22]  &  [0.03]  &  [0.05]  &  [0.07]  &  [0.03]  &  [0.00]  &  [0.04] \\
Male dummy  &  0.63  &  0.48*  &  0.55  &  0.62  &  0.47***  &  0.52**  &  0.46**  &  0.48**  &  0.50  &  0.48***  &  0.45***  &  0.46**  &  0.48***  &  0.00  &  0.56 \\
  &  [0.03]  &  [0.07]  &  [0.05]  &  [0.18]  &  [0.03]  &  [0.04]  &  [0.05]  &  [0.05]  &  [0.22]  &  [0.03]  &  [0.06]  &  [0.07]  &  [0.03]  &  [0.00]  &  [0.04] \\
Age  &  38.39  &  40.01  &  42.95***  &  45.28**  &  48.17***  &  37.91*  &  38.19  &  40.17*  &  34.35*  &  45.57***  &  40.66**  &  38.86  &  42.04***  &  49.37  &  45.46*** \\
  &  [0.37]  &  [1.19]  &  [0.79]  &  [2.59]  &  [0.57]  &  [0.57]  &  [0.83]  &  [0.69]  &  [1.92]  &  [0.55]  &  [1.00]  &  [0.90]  &  [0.50]  &  [8.69]  &  [0.69] \\
%Age sq.  &  1513.95  &  1681.25  &  1919.39***  &  2097.11**  &  2411.03***  &  1489.30*  &  1516.04  &  1668.54*  &  1198.50*  &  2147.34***  &  1725.28**  &  1551.57  &  1849.30***  &  2588.00  &  2151.50*** \\
%  &  [30.48]  &  [103.92]  &  [70.72]  &  [239.00]  &  [52.25]  &  [47.83]  &  [71.17]  &  [59.28]  &  [146.55]  &  [50.96]  &  [87.94]  &  [73.52]  &  [44.44]  &  [786.18]  &  [63.12] \\
Middle school edu mom  &  0.17  &  0.26  &  0.23  &  0.38  &  0.17  &  0.20  &  0.26*  &  0.15  &  0.00  &  0.28***  &  0.33***  &  0.13  &  0.27***  &  0.33  &  0.28*** \\
  &  [0.02]  &  [0.06]  &  [0.04]  &  [0.18]  &  [0.02]  &  [0.03]  &  [0.05]  &  [0.03]  &  [0.00]  &  [0.03]  &  [0.06]  &  [0.05]  &  [0.02]  &  [0.33]  &  [0.03] \\
High school edu mom  &  0.44  &  0.34  &  0.51  &  0.50  &  0.38  &  0.25***  &  0.29**  &  0.31**  &  0.50  &  0.36*  &  0.32**  &  0.46  &  0.30***  &  0.33  &  0.27*** \\
  &  [0.03]  &  [0.06]  &  [0.05]  &  [0.19]  &  [0.03]  &  [0.03]  &  [0.05]  &  [0.04]  &  [0.22]  &  [0.03]  &  [0.05]  &  [0.07]  &  [0.03]  &  [0.33]  &  [0.03] \\
University edu mom  &  0.37  &  0.40  &  0.25**  &  0.12  &  0.43  &  0.55***  &  0.45  &  0.54***  &  0.50  &  0.34  &  0.33  &  0.37  &  0.42  &  0.33  &  0.42 \\
  &  [0.03]  &  [0.06]  &  [0.04]  &  [0.12]  &  [0.03]  &  [0.04]  &  [0.05]  &  [0.05]  &  [0.22]  &  [0.03]  &  [0.06]  &  [0.07]  &  [0.03]  &  [0.33]  &  [0.04] \\
Middle school edu dad  &  0.17  &  0.24  &  0.22  &  0.25  &  0.14  &  0.18  &  0.25  &  0.12  &  0.00  &  0.29***  &  0.25  &  0.13  &  0.22  &  0.00  &  0.20 \\
  &  [0.02]  &  [0.06]  &  [0.04]  &  [0.16]  &  [0.02]  &  [0.03]  &  [0.05]  &  [0.03]  &  [0.00]  &  [0.03]  &  [0.05]  &  [0.05]  &  [0.02]  &  [0.00]  &  [0.03] \\
High school edu dad  &  0.40  &  0.29  &  0.47  &  0.62  &  0.38  &  0.27***  &  0.31  &  0.26***  &  0.33  &  0.34  &  0.33  &  0.50  &  0.24***  &  0.33  &  0.22*** \\
  &  [0.03]  &  [0.06]  &  [0.05]  &  [0.18]  &  [0.03]  &  [0.04]  &  [0.05]  &  [0.04]  &  [0.21]  &  [0.03]  &  [0.06]  &  [0.07]  &  [0.02]  &  [0.33]  &  [0.03] \\
University edu dad  &  0.42  &  0.43  &  0.31**  &  0.12  &  0.45  &  0.55**  &  0.44  &  0.62***  &  0.67  &  0.34*  &  0.42  &  0.35  &  0.53***  &  0.33  &  0.56*** \\
  &  [0.03]  &  [0.07]  &  [0.04]  &  [0.12]  &  [0.03]  &  [0.04]  &  [0.05]  &  [0.05]  &  [0.21]  &  [0.03]  &  [0.06]  &  [0.07]  &  [0.03]  &  [0.33]  &  [0.04] \\
Mom born in the province  &  0.88  &  0.86  &  0.84  &  0.75  &  0.71***  &  0.69***  &  0.81*  &  0.76***  &  0.67  &  0.72***  &  0.63***  &  0.73***  &  0.67***  &  0.67  &  0.72*** \\
  &  [0.02]  &  [0.05]  &  [0.03]  &  [0.16]  &  [0.03]  &  [0.04]  &  [0.04]  &  [0.04]  &  [0.21]  &  [0.03]  &  [0.06]  &  [0.06]  &  [0.03]  &  [0.33]  &  [0.03] \\
Dad born in the province  &  0.88  &  0.83  &  0.82  &  0.62*  &  0.79***  &  0.79**  &  0.88  &  0.78**  &  0.67  &  0.76***  &  0.74***  &  0.77**  &  0.74***  &  0.67  &  0.80** \\
  &  [0.02]  &  [0.05]  &  [0.04]  &  [0.18]  &  [0.02]  &  [0.03]  &  [0.04]  &  [0.04]  &  [0.21]  &  [0.03]  &  [0.05]  &  [0.06]  &  [0.02]  &  [0.33]  &  [0.03] \\
%Caregiver is religious  &  0.00  &  0.00  &  0.00  &  0.00  &  0.00  &  0.00  &  0.00  &  0.00  &  0.00  &  0.00  &  0.00  &  0.00  &  0.00  &  0.00  &  0.00 \\
%  &  [0.00]  &  [0.00]  &  [0.00]  &  [0.00]  &  [0.00]  &  [0.00]  &  [0.00]  &  [0.00]  &  [0.00]  &  [0.00]  &  [0.00]  &  [0.00]  &  [0.00]  &  [0.00]  &  [0.00] \\
%Own Home  &  0.42  &  0.62***  &  0.47  &  0.50  &  0.69***  &  0.64***  &  0.58***  &  0.61***  &  0.67  &  0.63***  &  0.70***  &  0.77***  &  0.66***  &  1.00  &  0.66*** \\
%  &  [0.03]  &  [0.06]  &  [0.05]  &  [0.19]  &  [0.03]  &  [0.04]  &  [0.05]  &  [0.05]  &  [0.21]  &  [0.03]  &  [0.05]  &  [0.06]  &  [0.03]  &  [0.00]  &  [0.04] \\
%Income 5k-10k eur  &  0.00  &  0.00  &  0.00  &  0.00  &  0.00  &  0.00  &  0.00  &  0.00  &  0.00  &  0.00  &  0.00  &  0.00  &  0.00  &  0.00  &  0.00 \\
%  &  [0.00]  &  [0.00]  &  [0.00]  &  [0.00]  &  [0.00]  &  [0.00]  &  [0.00]  &  [0.00]  &  [0.00]  &  [0.00]  &  [0.00]  &  [0.00]  &  [0.00]  &  [0.00]  &  [0.00] \\
%Income 10k-25k eur  &  0.00  &  0.00  &  0.00  &  0.00  &  0.00  &  0.00  &  0.00  &  0.00  &  0.00  &  0.00  &  0.00  &  0.00  &  0.00  &  0.00  &  0.00 \\
%  &  [0.00]  &  [0.00]  &  [0.00]  &  [0.00]  &  [0.00]  &  [0.00]  &  [0.00]  &  [0.00]  &  [0.00]  &  [0.00]  &  [0.00]  &  [0.00]  &  [0.00]  &  [0.00]  &  [0.00] \\
%Income 25k-50k eur  &  0.00  &  0.00  &  0.00  &  0.00  &  0.00  &  0.00  &  0.00  &  0.00  &  0.00  &  0.00  &  0.00  &  0.00  &  0.00  &  0.00  &  0.00 \\
%  &  [0.00]  &  [0.00]  &  [0.00]  &  [0.00]  &  [0.00]  &  [0.00]  &  [0.00]  &  [0.00]  &  [0.00]  &  [0.00]  &  [0.00]  &  [0.00]  &  [0.00]  &  [0.00]  &  [0.00] \\
%Income 50k-100k eur  &  0.00  &  0.00  &  0.00  &  0.00  &  0.00  &  0.00  &  0.00  &  0.00  &  0.00  &  0.00  &  0.00  &  0.00  &  0.00  &  0.00  &  0.00 \\
%  &  [0.00]  &  [0.00]  &  [0.00]  &  [0.00]  &  [0.00]  &  [0.00]  &  [0.00]  &  [0.00]  &  [0.00]  &  [0.00]  &  [0.00]  &  [0.00]  &  [0.00]  &  [0.00]  &  [0.00] \\
%Income 100k-250k eur  &  0.00  &  0.00  &  0.00  &  0.00  &  0.00  &  0.00  &  0.00  &  0.00  &  0.00  &  0.00  &  0.00  &  0.00  &  0.00  &  0.00  &  0.00 \\
%  &  [0.00]  &  [0.00]  &  [0.00]  &  [0.00]  &  [0.00]  &  [0.00]  &  [0.00]  &  [0.00]  &  [0.00]  &  [0.00]  &  [0.00]  &  [0.00]  &  [0.00]  &  [0.00]  &  [0.00] \\
%%Income more 250k eur  &  0.00  &  0.00  &  0.00  &  0.00  &  0.00  &  0.00  &  0.00  &  0.00  &  0.00  &  0.00  &  0.00  &  0.00  &  0.00  &  0.00  &  0.00 \\
%  &  [0.00]  &  [0.00]  &  [0.00]  &  [0.00]  &  [0.00]  &  [0.00]  &  [0.00]  &  [0.00]  &  [0.00]  &  [0.00]  &  [0.00]  &  [0.00]  &  [0.00]  &  [0.00]  &  [0.00] \\
%Income below 5k eur  &  0.00  &  0.00  &  0.00  &  0.00  &  0.00  &  0.00  &  0.00  &  0.00  &  0.00  &  0.00  &  0.00  &  0.00  &  0.00  &  0.00  &  0.00 \\
%  &  [0.00]  &  [0.00]  &  [0.00]  &  [0.00]  &  [0.00]  &  [0.00]  &  [0.00]  &  [0.00]  &  [0.00]  &  [0.00]  &  [0.00]  &  [0.00]  &  [0.00]  &  [0.00]  &  [0.00] \\
%Low birthweight  &  0.00  &  0.00  &  0.00  &  0.00  &  0.00  &  0.00  &  0.00  &  0.00  &  0.00  &  0.00  &  0.00  &  0.00  &  0.00  &  0.00  &  0.00 \\
%  &  [0.00]  &  [0.00]  &  [0.00]  &  [0.00]  &  [0.00]  &  [0.00]  &  [0.00]  &  [0.00]  &  [0.00]  &  [0.00]  &  [0.00]  &  [0.00]  &  [0.00]  &  [0.00]  &  [0.00] \\

% it contains the notes, assuming they are the same for all the tables.
\end{tabular}

\end{adjustbox}
\raggedright{
\footnotesize{Average of baseline characteristcs, by city and type of child-care attended. Standard errors of means in brackets. Test for difference in means between each column and the first column (Reggio Municipal, the treatment group) was performed; *** significant difference at 1\%, ** significant difference at 5\%, * significant difference at 10\%. Source: authors calculation using survey data.}
}
\end{table}
  
%\end{table}

\doublespacing

\subsection{Regression Results}
\label{sec:OLS}

In this section, we systematically display the relation between participation in different types of early education and four important outcomes of interest: a measure of behavioral problems from the Strength and Difficulties Questionnaire (SDQ), a measure of depression from the CESD scale, the probability of reporting being in good or excellent health, and a measure of negative attitude towards migration. \todo[backgroundcolor=orange!30,size=\tiny]{Describe more in detail these outcomes. Raw differences?}

Separately for each city and age group we run the following regression:
\[ 
y = \alpha + \sum_{sc} \delta_{sc} D_{sc} + \beta_{X}X + \varepsilon
\]

where $D_{sc}$ are dummies for city-specific school type, and they include municipal schools in Reggio, Parma, and Padova and not-attended in Reggio, Parma, and Padova. The treated group are the children who attended a municipal preschool in Reggio. The omitted category of comparison for every city is any other type of school: private, state or religious, pooled together.\footnote{Note that there are no state infant-toddler centers, and virtually every child and adolescent attended some form of preschool. See results in the appendix for the disaggregated effect of each type of preschool.} As mentioned before, the age groups considered are 0-3, regarding attendance to infant-toddler center, and age 3-6, for attendance to preschool. 

The following controls are considered incrementally through all the analysis: the first column of each table only controls for school types $D_{sc}$ and city dummies; the second column introduces interviewer fixed effects and interview mode (computer or paper); the third column introduces demographic controls such as gender, age at interview, age$^2$, dummies for parental education, dummies for parents born in the region (province), dummies for family owns home, family income brackets; the fourth column includes also dummies for low birthweight, and premature birth (when present)\footnote{Recall that adults did not report information on prematurity}; the fifth column interacts all of the controls with city-dummies, allowing the effect of controls to vary by city; the sixth column ran the regression only for the sample of respondents from Reggio Emilia; the final column does not control for interviewer fixed effects.


\singlespacing
\setlength\tabcolsep{0.25em}
\subsubsection{Children Results: OLS}
\begin{small}
\begin{table}[H]
\caption{Pool Regression: Child Health - Infant-toddler center, Child}
\begin{tabular}{lccccccc} \hline
 & (1) & (2) & (3) & (4) & (5) & (6) & (7) \\
VARIABLES &  &  &  &  &  &  &  \\ \hline
 &  &  &  &  &  &  &  \\
RCH infant-toddler & -0.026 & -0.086 & -0.092 & -0.089 & -0.065 & -0.065 & -0.023 \\
 & [0.058] & [0.090] & [0.087] & [0.086] & [0.089] & [0.087] & [0.059] \\
Reggio None ITC & 0.022 & -0.055 & -0.055 & -0.057 & -0.035 & -0.035 & 0.017 \\
 & [0.061] & [0.096] & [0.093] & [0.092] & [0.095] & [0.094] & [0.062] \\
Parma None ITC & -0.066 & 0.003 & 0.009 & 0.008 & 0.032 &  & -0.049 \\
 & [0.067] & [0.102] & [0.103] & [0.103] & [0.105] &  & [0.068] \\
Parma Muni ITC & -0.096 & -0.028 & -0.025 & -0.028 & -0.022 &  & -0.081 \\
 & [0.061] & [0.096] & [0.096] & [0.097] & [0.098] &  & [0.060] \\
Padova None ITC & 0.042 & -0.030 & -0.009 & -0.009 & -0.024 &  & 0.032 \\
 & [0.058] & [0.078] & [0.082] & [0.082] & [0.086] &  & [0.062] \\
Padova Muni ITC & 0.120* & 0.058 & 0.062 & 0.069 & 0.054 &  & 0.109 \\
 & [0.068] & [0.088] & [0.089] & [0.090] & [0.087] &  & [0.069] \\
Male dummy &  &  & -0.063* & -0.063* & -0.116** & -0.116** & -0.073** \\
 &  &  & [0.035] & [0.035] & [0.059] & [0.057] & [0.034] \\
Constant & 0.695*** & 1.034*** & 7.824 & 8.185 & 5.086 & 19.670** & 8.415 \\
 & [0.043] & [0.102] & [5.402] & [5.406] & [12.912] & [8.773] & [5.333] \\
 &  &  &  &  &  &  &  \\
Observations & 765 & 765 & 765 & 765 & 765 & 280 & 765 \\
R-squared & 0.016 & 0.059 & 0.083 & 0.085 & 0.181 & 0.142 & 0.047 \\
 Controls & None & F.E. & Family & All & Inter & Reggio & no FE \\ \hline
\multicolumn{8}{c}{ Robust standard errors in brackets} \\
\multicolumn{8}{c}{ *** p$<$0.01, ** p$<$0.05, * p$<$0.1} \\
\multicolumn{8}{c}{ Dependent variable: Child health is good (\%) - mom report.} \\
\end{tabular}
  
\end{table}
\begin{table}[H]
\caption{Pool Regression: Child Health - Preschool, Child}
\begin{tabular}{lccccccc} \hline
 & (1) & (2) & (3) & (4) & (5) & (6) & (7) \\
VARIABLES &  &  &  &  &  &  &  \\ \hline
 &  &  &  &  &  &  &  \\
RCH preschool & -0.016 & -0.014 & -0.023 & -0.021 & -0.009 & -0.009 & -0.019 \\
 & [0.045] & [0.056] & [0.057] & [0.057] & [0.058] & [0.057] & [0.046] \\
Reggio None Preschool & -0.203 & -0.122 & -0.186 & -0.191 & -0.289 & -0.289 & -0.297 \\
 & [0.356] & [0.375] & [0.385] & [0.390] & [0.384] & [0.377] & [0.376] \\
Parma None Preschool & -0.203 & -0.135 & -0.156 & -0.168 & -0.189 &  & -0.202 \\
 & [0.206] & [0.228] & [0.247] & [0.248] & [0.219] &  & [0.229] \\
Parma Muni Preschool & -0.117** & -0.040 & -0.054 & -0.061 & -0.056 &  & -0.108** \\
 & [0.049] & [0.064] & [0.065] & [0.065] & [0.069] &  & [0.050] \\
Padova None Preschool & 0.297*** & 0.329*** & 0.263* & 0.254* & 0.151 &  & 0.227*** \\
 & [0.023] & [0.109] & [0.135] & [0.134] & [0.144] &  & [0.083] \\
Padova Muni Preschool & 0.072 & 0.011 & 0.027 & 0.029 & 0.056 &  & 0.067 \\
 & [0.055] & [0.071] & [0.071] & [0.071] & [0.069] &  & [0.055] \\
Male dummy &  &  & -0.065* & -0.064* & -0.119** & -0.119** & -0.078** \\
 &  &  & [0.035] & [0.035] & [0.058] & [0.057] & [0.034] \\
Constant & 0.703*** & 0.994*** & 7.895 & 8.274 & 5.720 & 19.583** & 8.891* \\
 & [0.023] & [0.067] & [5.446] & [5.450] & [12.866] & [8.785] & [5.311] \\
 &  &  &  &  &  &  &  \\
Observations & 765 & 765 & 765 & 765 & 765 & 280 & 765 \\
R-squared & 0.015 & 0.058 & 0.083 & 0.085 & 0.181 & 0.143 & 0.049 \\
 Controls & None & F.E. & Family & All & Inter & Reggio & no FE \\ \hline
\multicolumn{8}{c}{ Robust standard errors in brackets} \\
\multicolumn{8}{c}{ *** p$<$0.01, ** p$<$0.05, * p$<$0.1} \\
\multicolumn{8}{c}{ Dependent variable: Child health is good (\%) - mom report.} \\
\end{tabular}

\end{table}

\begin{table}[H]
\caption{Pool Regression: Child SDQ Score - Infant-toddler center, Child}
\begin{tabular}{lccccccc} \hline
 & (1) & (2) & (3) & (4) & (5) & (6) & (7) \\
VARIABLES &  &  &  &  &  &  &  \\ \hline
 &  &  &  &  &  &  &  \\
RCH infant-toddler & -0.192 & -0.105 & -0.431 & -0.411 & -0.739 & -0.739 & -0.772 \\
 & [0.506] & [0.780] & [0.742] & [0.739] & [0.730] & [0.717] & [0.483] \\
Reggio None ITC & 1.162* & 1.378 & 0.670 & 0.710 & 0.436 & 0.436 & 0.353 \\
 & [0.606] & [0.879] & [0.850] & [0.849] & [0.856] & [0.841] & [0.612] \\
Parma None ITC & 0.119 & 0.836 & 0.722 & 0.809 & 0.737 &  & 0.074 \\
 & [0.587] & [0.673] & [0.660] & [0.660] & [0.662] &  & [0.572] \\
Parma Muni ITC & -0.082 & 0.721 & 0.622 & 0.678 & 0.666 &  & -0.054 \\
 & [0.537] & [0.595] & [0.587] & [0.586] & [0.583] &  & [0.522] \\
Padova None ITC & -0.009 & -1.079 & -1.069 & -1.105 & -1.485 &  & -0.272 \\
 & [0.524] & [0.890] & [0.897] & [0.896] & [0.965] &  & [0.556] \\
Padova Muni ITC & 0.830 & 0.109 & 0.116 & 0.045 & 0.029 &  & 0.778 \\
 & [0.740] & [1.014] & [1.017] & [1.025] & [1.077] &  & [0.777] \\
Male dummy &  &  & 0.590* & 0.566* & -0.055 & -0.055 & 0.558* \\
 &  &  & [0.309] & [0.310] & [0.522] & [0.513] & [0.306] \\
Constant & 8.263*** & 6.266*** & 113.151** & 112.210** & 81.045 & 87.021 & 131.384** \\
 & [0.396] & [1.261] & [49.901] & [50.017] & [89.787] & [88.170] & [51.138] \\
 &  &  &  &  &  &  &  \\
Observations & 766 & 766 & 766 & 766 & 766 & 280 & 766 \\
R-squared & 0.012 & 0.086 & 0.160 & 0.161 & 0.226 & 0.218 & 0.095 \\
 Controls & None & F.E. & Family & All & Inter & Reggio & no FE \\ \hline
\multicolumn{8}{c}{ Robust standard errors in brackets} \\
\multicolumn{8}{c}{ *** p$<$0.01, ** p$<$0.05, * p$<$0.1} \\
\multicolumn{8}{c}{ Dependent variable: SDQ score (mom rep.).} \\
\end{tabular}
  
\end{table}
\begin{table}[H]
\caption{Pool Regression: Child SDQ Score - Preschool, Child}
\begin{tabular}{lccccccc} \hline
 & (1) & (2) & (3) & (4) & (5) & (6) & (7) \\
VARIABLES &  &  &  &  &  &  &  \\ \hline
 &  &  &  &  &  &  &  \\
RCH preschool & -0.494 & -1.113** & -1.102** & -1.091** & -1.266** & -1.266** & -0.913** \\
 & [0.378] & [0.520] & [0.499] & [0.498] & [0.504] & [0.495] & [0.370] \\
Reggio None Preschool & 1.444 & 0.906 & 0.279 & 0.330 & -0.160 & -0.160 & 0.164 \\
 & [2.849] & [2.729] & [2.182] & [2.182] & [1.712] & [1.682] & [2.068] \\
Parma None Preschool & 0.111 & 0.219 & 0.726 & 0.741 & 0.481 &  & 0.884 \\
 & [1.390] & [1.416] & [1.332] & [1.326] & [1.149] &  & [1.283] \\
Parma Muni Preschool & -0.488 & 0.003 & -0.017 & 0.008 & 0.019 &  & -0.191 \\
 & [0.406] & [0.505] & [0.490] & [0.493] & [0.499] &  & [0.405] \\
Padova None Preschool & 3.944*** & 4.981*** & 4.100** & 4.090** & 4.497** &  & 3.395 \\
 & [1.087] & [1.277] & [1.838] & [1.843] & [1.846] &  & [2.066] \\
Padova Muni Preschool & 0.585 & 1.108 & 1.097 & 1.131 & 1.093 &  & 0.744 \\
 & [0.597] & [0.718] & [0.739] & [0.740] & [0.750] &  & [0.623] \\
Male dummy &  &  & 0.591* & 0.570* & -0.123 & -0.123 & 0.556* \\
 &  &  & [0.306] & [0.307] & [0.517] & [0.507] & [0.301] \\
Constant & 8.556*** & 7.138*** & 118.700** & 118.137** & 112.357 & 82.419 & 134.677*** \\
 & [0.213] & [0.908] & [49.942] & [49.987] & [93.269] & [86.865] & [50.077] \\
 &  &  &  &  &  &  &  \\
Observations & 766 & 766 & 766 & 766 & 766 & 280 & 766 \\
R-squared & 0.009 & 0.083 & 0.161 & 0.162 & 0.227 & 0.225 & 0.098 \\
 Controls & None & F.E. & Family & All & Inter & Reggio & no FE \\ \hline
\multicolumn{8}{c}{ Robust standard errors in brackets} \\
\multicolumn{8}{c}{ *** p$<$0.01, ** p$<$0.05, * p$<$0.1} \\
\multicolumn{8}{c}{ Dependent variable: SDQ score (mom rep.).} \\
\end{tabular}

\end{table}
\end{small}

\pagebreak
\subsubsection{Children Results: Propensity Score Matching}
These are the results created by Chiara using different types of propensity score matching
\begin{small}
\begin{table}[H]
\caption{Propensity Score Matching, Children (Age 6)}
\label{tab:PSM_children}
\vspace{-5mm}
\begin{center}
\begin{tabular}{ c c c c }
\hline\hline
                       & & \textbf{Child} & \textbf{Child} \\
\textbf{Specification} & & \textbf{SDQ}   & \textbf{Health} \\
\hline 
\multicolumn{4}{c}{\emph{Infant Toddler Center }}\\ [0.2em]
\hline 
OLS   & coeff.  & -0.919 & -0.328\\ [0.2em]
      & s.e.    & (0.637)  & (0.070) \\ [0.2em]
      & obs.    &  \emph{824} & \emph{824} \\ [0.2em]
PSM 1 & coeff.  & -3.304{*} & 0.012\\ [0.2em]
      & s.e.    & (1.696)  & (0.149) \\ [0.2em]
      & obs.    &  \emph{274} & \emph{274} \\ [0.2em]
 PSM 2 & coeff.  & -0.520 & 0.053\\ [0.2em]
  & s.e.  & (0.731)  & (0.093) \\ [0.2em]
  & obs.  &  \emph{277} & \emph{277} \\ [0.2em]
 PSM 3 & coeff.  & 2.006{*}{*} & -0.079\\ [0.2em]
  & s.e.  & (0.816)  & (0.082) \\ [0.2em]
  & obs.  &  \emph{277} & \emph{277} \\ [0.2em]
 PSM 4 & coeff.  & 1.094 & 0.087\\ [0.2em]
  & s.e.  & (0.758)  & (0.091) \\ [0.2em]
  & obs.  &  \emph{277} & \emph{277} \\ [0.2em]
\hline 
\multicolumn{4}{c}{\emph{Preschool}}\\ [0.2em]
\hline 
OLS  & coeff.  & -1.478{*}{*}{*} & 0.026\\ [0.2em]
  & s.e.  & (0.529)  & (0.059) \\ [0.2em]
  & obs.  &  \emph{822} & \emph{821} \\ [0.2em]
 PSM 1 & coeff.  & -0.375 & -0.054\\ [0.2em]
  & s.e.  & (1.172)  & (0.101) \\ [0.2em]
  & obs.  &  288 & \emph{288} \\ [0.2em]
 PSM 2 & coeff.  & -2.113{*}{*}{*} & 0.106\\ [0.2em]
  & s.e.  & (0.647)  & (0.080) \\ [0.2em]
  & obs.  &  \emph{304} & \emph{304} \\ [0.2em]
 PSM 3 & coeff.  & -2.407{*}{*}{*} & 0.063\\ [0.2em]
  & s.e.  & (0.719)  & (0.074) \\ [0.2em]
  & obs.  &  \emph{304} & \emph{304} \\ [0.2em]
 PSM 4 & coeff.  & -2.490{*}{*}{*} & 0.084\\ [0.2em]
  & s.e.  & (0.761)  & (0.079) \\ [0.2em]
  & obs.  &  \emph{304} & \emph{304} \\ [0.2em]
\hline 
\end{tabular}
\end{center}
\footnotesize{{\bfseries Notes:} Source: authors' calculations from the administrative data on the universe of applications to the municipal preschools of Reggio Emilia. The cells display the percentages of applicants who were 3, 4 and 5 years-old when applying to each school year. The age is calculated based on the year of birth of the child, so that for example children born any time in 2005 are considered 3-year-old if applying to the 2008-2009 School Year.}
\end{table}

\end{small}

\pagebreak
\subsubsection{Adolescents Results}
\begin{table}[H]
\caption{Pool Regression: Child Health - Infant-toddler center, Adolescent}
\begin{tabular}{lccccccc} \hline
 & (1) & (2) & (3) & (4) & (5) & (6) & (7) \\
VARIABLES &  &  &  &  &  &  &  \\ \hline
 &  &  &  &  &  &  &  \\
RCH infant-toddler & -0.060 & -0.128 & -0.102 & -0.105 & -0.123 & -0.123 & -0.012 \\
 & [0.086] & [0.119] & [0.125] & [0.126] & [0.141] & [0.138] & [0.089] \\
Reggio None ITC & -0.119 & -0.206* & -0.171 & -0.174 & -0.170 & -0.170 & -0.062 \\
 & [0.088] & [0.120] & [0.124] & [0.125] & [0.139] & [0.136] & [0.090] \\
Parma None ITC & -0.145 & 0.023 & 0.026 & 0.034 & 0.080 &  & -0.112 \\
 & [0.090] & [0.126] & [0.129] & [0.129] & [0.124] &  & [0.092] \\
Parma Muni ITC & -0.312*** & -0.149 & -0.163 & -0.153 & -0.134 &  & -0.281*** \\
 & [0.095] & [0.131] & [0.134] & [0.134] & [0.134] &  & [0.097] \\
Padova None ITC & -0.010 & -0.221*** & -0.150 & -0.148 & -0.156 &  & -0.011 \\
 & [0.082] & [0.075] & [0.091] & [0.092] & [0.136] &  & [0.083] \\
Padova Muni ITC & -0.147 & -0.314*** & -0.261** & -0.266** & -0.284* &  & -0.142 \\
 & [0.103] & [0.105] & [0.115] & [0.116] & [0.153] &  & [0.105] \\
Male dummy &  &  & 0.055 & 0.054 & 0.086 & 0.086 & 0.055 \\
 &  &  & [0.038] & [0.038] & [0.067] & [0.066] & [0.038] \\
Constant & 0.743*** & 0.721* & -12.976 & -11.758 & 51.942 & -44.741 & -16.853 \\
 & [0.074] & [0.372] & [51.301] & [51.306] & [99.563] & [87.736] & [49.792] \\
 &  &  &  &  &  &  &  \\
Observations & 652 & 652 & 652 & 652 & 652 & 240 & 652 \\
R-squared & 0.037 & 0.111 & 0.135 & 0.139 & 0.207 & 0.110 & 0.075 \\
 Controls & None & F.E. & Family & All & Inter & Reggio & no FE \\ \hline
\multicolumn{8}{c}{ Robust standard errors in brackets} \\
\multicolumn{8}{c}{ *** p$<$0.01, ** p$<$0.05, * p$<$0.1} \\
\multicolumn{8}{c}{ Dependent variable: Child health is good (\%) - mom report.} \\
\end{tabular}
  
\end{table}
\begin{table}[H]
\caption{Pool Regression: Child Health - Preschool, Adolescent}
\begin{tabular}{lccccccc} \hline
 & (1) & (2) & (3) & (4) & (5) & (6) & (7) \\
VARIABLES &  &  &  &  &  &  &  \\ \hline
 &  &  &  &  &  &  &  \\
RCH preschool & 0.037 & 0.056 & 0.051 & 0.041 & 0.048 & 0.048 & 0.054 \\
 & [0.048] & [0.065] & [0.067] & [0.067] & [0.075] & [0.073] & [0.050] \\
Reggio None Preschool & 0.059 & 0.097 & 0.077 & 0.082 & 0.136 & 0.136 & 0.073 \\
 & [0.174] & [0.188] & [0.189] & [0.182] & [0.210] & [0.206] & [0.152] \\
Parma None Preschool & 0.345*** & 0.490*** & 0.450*** & 0.500*** & 0.642*** &  & 0.429*** \\
 & [0.025] & [0.103] & [0.115] & [0.142] & [0.174] &  & [0.086] \\
Parma Muni Preschool & -0.150** & -0.048 & -0.053 & -0.053 & -0.066 &  & -0.145** \\
 & [0.060] & [0.075] & [0.077] & [0.077] & [0.081] &  & [0.061] \\
Padova None Preschool & -0.655*** & -1.000*** & -1.057*** & -1.055*** & -1.040*** &  & -0.687*** \\
 & [0.025] & [0.000] & [0.038] & [0.038] & [0.062] &  & [0.121] \\
Padova Muni Preschool & -0.007 & -0.038 & -0.042 & -0.041 & -0.037 &  & -0.030 \\
 & [0.062] & [0.071] & [0.072] & [0.071] & [0.071] &  & [0.063] \\
Male dummy &  &  & 0.057 & 0.055 & 0.082 & 0.082 & 0.057 \\
 &  &  & [0.038] & [0.038] & [0.068] & [0.066] & [0.038] \\
Constant & 0.655*** & 1.000 & -16.490 & -15.810 & 92.198 & -67.610 & -11.472 \\
 & [0.025] & [.] & [52.018] & [51.969] & [99.962] & [87.411] & [50.738] \\
 &  &  &  &  &  &  &  \\
Observations & 652 & 652 & 652 & 652 & 652 & 240 & 652 \\
R-squared & 0.018 & 0.104 & 0.128 & 0.132 & 0.202 & 0.107 & 0.065 \\
 Controls & None & F.E. & Family & All & Inter & Reggio & no FE \\ \hline
\multicolumn{8}{c}{ Robust standard errors in brackets} \\
\multicolumn{8}{c}{ *** p$<$0.01, ** p$<$0.05, * p$<$0.1} \\
\multicolumn{8}{c}{ Dependent variable: Child health is good (\%) - mom report.} \\
\end{tabular}

\end{table}

\begin{table}[H]
\caption{Pool Regression: Health - Infant-toddler center, Adolescent}
\begin{tabular}{lccccccc} \hline
 & (1) & (2) & (3) & (4) & (5) & (6) & (7) \\
VARIABLES &  &  &  &  &  &  &  \\ \hline
 &  &  &  &  &  &  &  \\
RCH infant-toddler & 0.133 & 0.074 & 0.143 & 0.142 & 0.088 & 0.088 & 0.188** \\
 & [0.089] & [0.151] & [0.156] & [0.157] & [0.159] & [0.155] & [0.090] \\
Reggio None ITC & -0.003 & -0.060 & 0.011 & 0.011 & -0.027 & -0.027 & 0.061 \\
 & [0.093] & [0.152] & [0.157] & [0.158] & [0.158] & [0.154] & [0.094] \\
Parma None ITC & -0.011 & 0.085 & 0.053 & 0.054 & 0.075 &  & 0.012 \\
 & [0.094] & [0.122] & [0.122] & [0.123] & [0.133] &  & [0.094] \\
Parma Muni ITC & -0.171* & -0.068 & -0.083 & -0.081 & -0.069 &  & -0.128 \\
 & [0.100] & [0.129] & [0.129] & [0.129] & [0.141] &  & [0.100] \\
Padova None ITC & 0.122 & 0.270 & 0.339* & 0.345* & 0.369* &  & 0.119 \\
 & [0.087] & [0.195] & [0.183] & [0.184] & [0.193] &  & [0.086] \\
Padova Muni ITC & 0.066 & 0.285 & 0.336* & 0.341* & 0.334 &  & 0.080 \\
 & [0.104] & [0.205] & [0.191] & [0.192] & [0.203] &  & [0.103] \\
Male dummy &  &  & 0.075** & 0.074** & 0.078 & 0.078 & 0.071** \\
 &  &  & [0.036] & [0.036] & [0.065] & [0.064] & [0.035] \\
Constant & 0.657*** & 0.681*** & -13.497 & -14.443 & 31.032 & 10.073 & 1.818 \\
 & [0.081] & [0.202] & [48.569] & [48.530] & [94.148] & [84.270] & [47.034] \\
 &  &  &  &  &  &  &  \\
Observations & 651 & 651 & 651 & 651 & 651 & 240 & 651 \\
R-squared & 0.043 & 0.114 & 0.143 & 0.144 & 0.211 & 0.098 & 0.098 \\
 Controls & None & F.E. & Family & All & Inter & Reggio & no FE \\ \hline
\multicolumn{8}{c}{ Robust standard errors in brackets} \\
\multicolumn{8}{c}{ *** p$<$0.01, ** p$<$0.05, * p$<$0.1} \\
\multicolumn{8}{c}{ Dependent variable: Respondent health is good (\%).} \\
\end{tabular}

\end{table}
\begin{table}[H]
\caption{Pool Regression: Health - Preschool, Adolescent}
\begin{tabular}{lccccccc} \hline
 & (1) & (2) & (3) & (4) & (5) & (6) & (7) \\
VARIABLES &  &  &  &  &  &  &  \\ \hline
 &  &  &  &  &  &  &  \\
RCH preschool & 0.075* & 0.087 & 0.088 & 0.089 & 0.088 & 0.088 & 0.105** \\
 & [0.045] & [0.062] & [0.063] & [0.063] & [0.069] & [0.068] & [0.047] \\
Reggio None Preschool & 0.022 & 0.038 & 0.077 & 0.081 & 0.091 & 0.091 & 0.092 \\
 & [0.173] & [0.174] & [0.158] & [0.154] & [0.178] & [0.175] & [0.138] \\
Parma None Preschool & 0.308*** & 0.454*** & 0.419*** & 0.445*** & 0.439*** &  & 0.385*** \\
 & [0.025] & [0.100] & [0.127] & [0.145] & [0.160] &  & [0.099] \\
Parma Muni Preschool & -0.060 & 0.109 & 0.102 & 0.102 & 0.088 &  & -0.054 \\
 & [0.058] & [0.074] & [0.076] & [0.076] & [0.081] &  & [0.059] \\
Padova None Preschool & -0.692*** & -1.000*** & -1.078*** & -1.077*** & -1.064*** &  & -0.721*** \\
 & [0.025] & [0.000] & [0.037] & [0.037] & [0.057] &  & [0.120] \\
Padova Muni Preschool & -0.016 & -0.037 & -0.036 & -0.036 & -0.053 &  & -0.024 \\
 & [0.061] & [0.069] & [0.069] & [0.069] & [0.071] &  & [0.060] \\
Male dummy &  &  & 0.078** & 0.077** & 0.074 & 0.074 & 0.073** \\
 &  &  & [0.037] & [0.037] & [0.066] & [0.065] & [0.036] \\
Constant & 0.692*** & 0.976*** & -12.635 & -13.082 & 53.721 & -11.184 & 7.263 \\
 & [0.025] & [0.060] & [48.361] & [48.278] & [94.130] & [83.008] & [47.103] \\
 &  &  &  &  &  &  &  \\
Observations & 651 & 651 & 651 & 651 & 651 & 240 & 651 \\
R-squared & 0.012 & 0.111 & 0.140 & 0.141 & 0.209 & 0.093 & 0.078 \\
 Controls & None & F.E. & Family & All & Inter & Reggio & no FE \\ \hline
\multicolumn{8}{c}{ Robust standard errors in brackets} \\
\multicolumn{8}{c}{ *** p$<$0.01, ** p$<$0.05, * p$<$0.1} \\
\multicolumn{8}{c}{ Dependent variable: Respondent health is good (\%).} \\
\end{tabular}

\end{table}

\begin{table}[H]
\caption{Pool Regression: Child SDQ Score - Infant-toddler center, Adolescent}
\begin{tabular}{lccccccc} \hline
 & (1) & (2) & (3) & (4) & (5) & (6) & (7) \\
VARIABLES &  &  &  &  &  &  &  \\ \hline
 &  &  &  &  &  &  &  \\
RCH infant-toddler & -1.804 & -1.442 & -2.227 & -2.212 & -1.378 & -1.378 & -2.049* \\
 & [1.106] & [2.148] & [2.191] & [2.206] & [2.148] & [2.099] & [1.120] \\
Reggio None ITC & -1.129 & -0.894 & -1.662 & -1.657 & -0.620 & -0.620 & -1.424 \\
 & [1.117] & [2.144] & [2.197] & [2.212] & [2.140] & [2.092] & [1.146] \\
Parma None ITC & -1.934* & -1.913 & -1.725 & -1.740 & -2.249 &  & -1.972* \\
 & [1.138] & [1.506] & [1.485] & [1.487] & [1.488] &  & [1.159] \\
Parma Muni ITC & -1.692 & -1.882 & -1.550 & -1.589 & -1.955 &  & -1.643 \\
 & [1.160] & [1.533] & [1.476] & [1.483] & [1.436] &  & [1.154] \\
Padova None ITC & -1.239 & 0.915 & 1.402 & 1.303 & 1.516 &  & -1.618 \\
 & [1.087] & [1.706] & [1.477] & [1.471] & [1.628] &  & [1.091] \\
Padova Muni ITC & -1.206 & 0.916 & 1.656 & 1.571 & 2.118 &  & -1.224 \\
 & [1.243] & [1.854] & [1.644] & [1.647] & [1.869] &  & [1.228] \\
Male dummy &  &  & 0.512 & 0.520 & -0.319 & -0.319 & 0.443 \\
 &  &  & [0.340] & [0.339] & [0.629] & [0.615] & [0.344] \\
Constant & 9.429*** & 4.651*** & -127.870 & -117.715 & -137.777 & -621.006 & -128.480 \\
 & [1.049] & [1.773] & [401.602] & [403.696] & [692.925] & [622.039] & [397.359] \\
 &  &  &  &  &  &  &  \\
Observations & 648 & 648 & 648 & 648 & 648 & 241 & 648 \\
R-squared & 0.012 & 0.087 & 0.135 & 0.136 & 0.208 & 0.117 & 0.068 \\
 Controls & None & F.E. & Family & All & Inter & Reggio & no FE \\ \hline
\multicolumn{8}{c}{ Robust standard errors in brackets} \\
\multicolumn{8}{c}{ *** p$<$0.01, ** p$<$0.05, * p$<$0.1} \\
\multicolumn{8}{c}{ Dependent variable: SDQ score (mom rep.).} \\
\end{tabular}

\end{table}
\begin{table}[H]
\caption{Pool Regression: Child SDQ Score - Preschool, Adolescent}
\begin{tabular}{lccccccc} \hline
 & (1) & (2) & (3) & (4) & (5) & (6) & (7) \\
VARIABLES &  &  &  &  &  &  &  \\ \hline
 &  &  &  &  &  &  &  \\
RCH preschool & -0.621 & -0.724 & -0.970 & -0.987 & -0.952 & -0.952 & -0.751* \\
 & [0.390] & [0.575] & [0.599] & [0.602] & [0.660] & [0.645] & [0.404] \\
Reggio None Preschool & 0.567 & 0.429 & 0.068 & 0.030 & 0.665 & 0.665 & 0.483 \\
 & [1.329] & [1.429] & [1.225] & [1.204] & [1.547] & [1.512] & [1.091] \\
Parma None Preschool & 0.210 & 1.442 & 2.481 & 2.238 & 1.728 &  & 1.316 \\
 & [3.208] & [3.363] & [3.351] & [3.191] & [2.789] &  & [3.225] \\
Parma Muni Preschool & -0.819 & -0.634 & -0.401 & -0.399 & -0.323 &  & -0.519 \\
 & [0.532] & [0.679] & [0.674] & [0.672] & [0.694] &  & [0.531] \\
Padova None Preschool & 5.710*** & 10.000 & 9.533*** & 9.525*** & 8.469*** &  & 5.055*** \\
 & [0.239] & [.] & [0.334] & [0.333] & [0.533] &  & [1.053] \\
Padova Muni Preschool & -0.493 & -1.189* & -0.962 & -0.969 & -0.860 &  & -0.490 \\
 & [0.469] & [0.629] & [0.632] & [0.633] & [0.652] &  & [0.470] \\
Male dummy &  &  & 0.467 & 0.475 & -0.329 & -0.329 & 0.423 \\
 &  &  & [0.334] & [0.333] & [0.626] & [0.612] & [0.336] \\
Constant & 8.290*** & 4.000 & -52.894 & -36.750 & -6.489 & -555.895 & -86.265 \\
 & [0.239] & [.] & [402.796] & [404.632] & [673.447] & [627.546] & [397.301] \\
 &  &  &  &  &  &  &  \\
Observations & 648 & 648 & 648 & 648 & 648 & 241 & 648 \\
R-squared & 0.010 & 0.093 & 0.139 & 0.140 & 0.207 & 0.121 & 0.065 \\
 Controls & None & F.E. & Family & All & Inter & Reggio & no FE \\ \hline
\multicolumn{8}{c}{ Robust standard errors in brackets} \\
\multicolumn{8}{c}{ *** p$<$0.01, ** p$<$0.05, * p$<$0.1} \\
\multicolumn{8}{c}{ Dependent variable: SDQ score (mom rep.).} \\
\end{tabular}

\end{table}

\begin{table}[H]
\caption{Pool Regression: SDQ Score - Infant-toddler center, Adolescent}
\begin{tabular}{lccccccc} \hline
 & (1) & (2) & (3) & (4) & (5) & (6) & (7) \\
VARIABLES &  &  &  &  &  &  &  \\ \hline
 &  &  &  &  &  &  &  \\
RCH infant-toddler & -0.242 & -1.559 & -2.092 & -2.072 & -2.322 & -2.322 & -0.563 \\
 & [0.966] & [1.967] & [1.984] & [2.003] & [2.144] & [2.097] & [0.991] \\
Reggio None ITC & 0.265 & -1.152 & -1.716 & -1.702 & -1.667 & -1.667 & -0.161 \\
 & [0.955] & [1.949] & [1.961] & [1.981] & [2.109] & [2.063] & [0.991] \\
Parma None ITC & -1.313 & -0.751 & -0.559 & -0.601 & -0.964 &  & -1.422 \\
 & [0.953] & [1.135] & [1.147] & [1.147] & [1.086] &  & [0.981] \\
Parma Muni ITC & -0.966 & -0.568 & -0.622 & -0.690 & -0.949 &  & -1.321 \\
 & [1.009] & [1.187] & [1.200] & [1.202] & [1.113] &  & [1.034] \\
Padova None ITC & -0.508 & 1.073 & 0.897 & 0.843 & 1.592 &  & -0.595 \\
 & [0.918] & [1.772] & [1.863] & [1.877] & [2.066] &  & [0.952] \\
Padova Muni ITC & 0.931 & 1.985 & 1.837 & 1.820 & 2.354 &  & 0.886 \\
 & [1.128] & [1.888] & [1.947] & [1.959] & [2.068] &  & [1.112] \\
Male dummy &  &  & -0.330 & -0.320 & -0.408 & -0.408 & -0.384 \\
 &  &  & [0.354] & [0.355] & [0.672] & [0.658] & [0.355] \\
Constant & 9.771*** & 7.147*** & 238.185 & 234.114 & -101.356 & 658.996 & 268.282 \\
 & [0.861] & [2.705] & [441.034] & [440.974] & [712.576] & [839.410] & [438.700] \\
 &  &  &  &  &  &  &  \\
Observations & 649 & 649 & 649 & 649 & 649 & 239 & 649 \\
R-squared & 0.019 & 0.088 & 0.112 & 0.113 & 0.229 & 0.129 & 0.051 \\
 Controls & None & F.E. & Family & All & Inter & Reggio & no FE \\ \hline
\multicolumn{8}{c}{ Robust standard errors in brackets} \\
\multicolumn{8}{c}{ *** p$<$0.01, ** p$<$0.05, * p$<$0.1} \\
\multicolumn{8}{c}{ Dependent variable: SDQ score (self rep.).} \\
\end{tabular}

\end{table}
\begin{table}[H]
\caption{Pool Regression: SDQ Score - Preschool, Adolescent}
\begin{tabular}{lccccccc} \hline
 & (1) & (2) & (3) & (4) & (5) & (6) & (7) \\
VARIABLES &  &  &  &  &  &  &  \\ \hline
 &  &  &  &  &  &  &  \\
RCH preschool & -0.187 & -0.942 & -1.006 & -0.978 & -0.911 & -0.911 & -0.331 \\
 & [0.483] & [0.627] & [0.651] & [0.657] & [0.719] & [0.703] & [0.505] \\
Reggio None Preschool & -2.235 & -2.872* & -2.843* & -2.876* & -1.433 & -1.433 & -2.346 \\
 & [1.537] & [1.515] & [1.626] & [1.554] & [1.874] & [1.833] & [1.517] \\
Parma None Preschool & -0.164 & 2.380* & 3.280** & 3.015** & 3.130** &  & 0.622 \\
 & [1.093] & [1.300] & [1.340] & [1.200] & [1.237] &  & [1.208] \\
Parma Muni Preschool & -1.158** & -0.205 & -0.197 & -0.195 & 0.095 &  & -1.175** \\
 & [0.495] & [0.630] & [0.657] & [0.658] & [0.672] &  & [0.519] \\
Padova None Preschool & 3.336*** & 3.000*** & 3.337*** & 3.328*** & 2.995*** &  & 2.877*** \\
 & [0.239] & [0.000] & [0.354] & [0.354] & [0.631] &  & [1.028] \\
Padova Muni Preschool & -0.450 & -0.973 & -0.905 & -0.911 & -0.665 &  & -0.371 \\
 & [0.559] & [0.656] & [0.663] & [0.667] & [0.668] &  & [0.550] \\
Male dummy &  &  & -0.337 & -0.328 & -0.388 & -0.388 & -0.414 \\
 &  &  & [0.354] & [0.354] & [0.678] & [0.663] & [0.355] \\
Constant & 9.664*** & 10.000*** & 329.538 & 328.281 & 91.956 & 752.472 & 355.514 \\
 & [0.239] & [0.000] & [440.046] & [440.059] & [715.537] & [851.861] & [439.911] \\
 &  &  &  &  &  &  &  \\
Observations & 649 & 649 & 649 & 649 & 649 & 239 & 649 \\
R-squared & 0.011 & 0.094 & 0.117 & 0.118 & 0.227 & 0.126 & 0.044 \\
 Controls & None & F.E. & Family & All & Inter & Reggio & no FE \\ \hline
\multicolumn{8}{c}{ Robust standard errors in brackets} \\
\multicolumn{8}{c}{ *** p$<$0.01, ** p$<$0.05, * p$<$0.1} \\
\multicolumn{8}{c}{ Dependent variable: SDQ score (self rep.).} \\
\end{tabular}

\end{table}

\begin{table}[H]
\caption{Pool Regression: Depression - Infant-toddler center, Adolescent}
\begin{tabular}{lccccccc} \hline
 & (1) & (2) & (3) & (4) & (5) & (6) & (7) \\
VARIABLES &  &  &  &  &  &  &  \\ \hline
 &  &  &  &  &  &  &  \\
RCH infant-toddler & 0.650 & -1.009 & -1.550 & -1.509 & -1.610 & -1.610 & 0.091 \\
 & [1.198] & [2.690] & [2.679] & [2.708] & [2.718] & [2.656] & [1.237] \\
Reggio None ITC & 0.737 & -0.916 & -1.473 & -1.447 & -0.869 & -0.869 & 0.131 \\
 & [1.221] & [2.705] & [2.712] & [2.740] & [2.755] & [2.693] & [1.282] \\
Parma None ITC & -0.299 & 0.825 & 1.065 & 0.965 & -0.174 &  & -0.612 \\
 & [1.177] & [1.201] & [1.238] & [1.235] & [1.190] &  & [1.229] \\
Parma Muni ITC & -0.129 & 0.732 & 0.596 & 0.440 & -0.332 &  & -0.834 \\
 & [1.209] & [1.218] & [1.250] & [1.254] & [1.258] &  & [1.264] \\
Padova None ITC & -1.343 & -1.251 & -2.006 & -2.133 & -1.257 &  & -1.077 \\
 & [1.132] & [3.262] & [3.502] & [3.526] & [4.002] &  & [1.175] \\
Padova Muni ITC & 0.852 & 0.131 & -0.482 & -0.521 & 0.074 &  & 0.980 \\
 & [1.464] & [3.429] & [3.645] & [3.668] & [4.084] &  & [1.495] \\
Male dummy &  &  & -1.350*** & -1.329*** & -1.574* & -1.574* & -1.377*** \\
 &  &  & [0.470] & [0.470] & [0.897] & [0.877] & [0.468] \\
Constant & 21.914*** & 23.751*** & 238.386 & 233.573 & 715.885 & -32.344 & 324.889 \\
 & [1.050] & [6.368] & [619.414] & [615.529] & [1,005.865] & [1,221.122] & [609.264] \\
 &  &  &  &  &  &  &  \\
Observations & 634 & 634 & 634 & 634 & 634 & 234 & 634 \\
R-squared & 0.020 & 0.097 & 0.135 & 0.139 & 0.237 & 0.139 & 0.077 \\
 Controls & None & F.E. & Family & All & Inter & Reggio & no FE \\ \hline
\multicolumn{8}{c}{ Robust standard errors in brackets} \\
\multicolumn{8}{c}{ *** p$<$0.01, ** p$<$0.05, * p$<$0.1} \\
\multicolumn{8}{c}{ Dependent variable: Depression score (CESD).} \\
\end{tabular}

\end{table}
\begin{table}[H]
\caption{Pool Regression: Depression - Preschool, Adolescent}
\begin{tabular}{lccccccc} \hline
 & (1) & (2) & (3) & (4) & (5) & (6) & (7) \\
VARIABLES &  &  &  &  &  &  &  \\ \hline
 &  &  &  &  &  &  &  \\
RCH preschool & 0.088 & -1.581* & -1.516 & -1.432 & -1.635 & -1.635 & -0.156 \\
 & [0.646] & [0.892] & [0.929] & [0.934] & [1.022] & [0.999] & [0.655] \\
Reggio None Preschool & -2.793* & -4.338** & -4.030*** & -4.112*** & -2.788* & -2.788* & -2.857** \\
 & [1.437] & [1.690] & [1.486] & [1.469] & [1.599] & [1.563] & [1.208] \\
Parma None Preschool & -2.436 & -2.665 & -2.071 & -2.786 & -3.004 &  & -2.742 \\
 & [3.925] & [4.136] & [3.789] & [3.239] & [2.076] &  & [3.666] \\
Parma Muni Preschool & -0.377 & -0.485 & -0.307 & -0.302 & -0.566 &  & -0.544 \\
 & [0.617] & [0.761] & [0.797] & [0.797] & [0.795] &  & [0.672] \\
Padova None Preschool & 8.064*** & 15.000 & 16.377*** & 16.351*** & 15.456*** &  & 9.257*** \\
 & [0.329] & [.] & [0.465] & [0.467] & [0.829] &  & [1.290] \\
Padova Muni Preschool & -0.636 & 0.515 & 0.391 & 0.374 & 0.563 &  & -0.275 \\
 & [0.760] & [0.890] & [0.912] & [0.918] & [0.965] &  & [0.770] \\
Male dummy &  &  & -1.377*** & -1.351*** & -1.520* & -1.520* & -1.438*** \\
 &  &  & [0.465] & [0.467] & [0.888] & [0.868] & [0.467] \\
Constant & 21.936*** & 15.000 & 231.059 & 233.376 & 619.995 & 228.471 & 347.025 \\
 & [0.329] & [.] & [627.165] & [623.754] & [1,018.712] & [1,246.574] & [614.309] \\
 &  &  &  &  &  &  &  \\
Observations & 634 & 634 & 634 & 634 & 634 & 234 & 634 \\
R-squared & 0.007 & 0.109 & 0.145 & 0.148 & 0.246 & 0.149 & 0.075 \\
 Controls & None & F.E. & Family & All & Inter & Reggio & no FE \\ \hline
\multicolumn{8}{c}{ Robust standard errors in brackets} \\
\multicolumn{8}{c}{ *** p$<$0.01, ** p$<$0.05, * p$<$0.1} \\
\multicolumn{8}{c}{ Dependent variable: Depression score (CESD).} \\
\end{tabular}

\end{table}

\begin{table}[H]
\caption{Pool Regression: Migration Taste - Infant-toddler center, Adolescent}
\begin{tabular}{lccccccc} \hline
 & (1) & (2) & (3) & (4) & (5) & (6) & (7) \\
VARIABLES &  &  &  &  &  &  &  \\ \hline
 &  &  &  &  &  &  &  \\
RCH infant-toddler & 0.202*** & 0.324*** & 0.290*** & 0.296*** & 0.325*** & 0.325*** & 0.157** \\
 & [0.070] & [0.056] & [0.067] & [0.069] & [0.094] & [0.092] & [0.072] \\
Reggio None ITC & 0.169** & 0.294*** & 0.256*** & 0.260*** & 0.301*** & 0.301*** & 0.113 \\
 & [0.071] & [0.054] & [0.068] & [0.070] & [0.096] & [0.094] & [0.077] \\
Parma None ITC & 0.065 & 0.012 & -0.009 & -0.021 & -0.058 &  & 0.024 \\
 & [0.069] & [0.104] & [0.103] & [0.103] & [0.112] &  & [0.069] \\
Parma Muni ITC & 0.108 & 0.044 & 0.029 & 0.010 & -0.060 &  & 0.058 \\
 & [0.075] & [0.108] & [0.106] & [0.106] & [0.112] &  & [0.077] \\
Padova None ITC & 0.132** & 0.120 & 0.062 & 0.047 & -0.078 &  & 0.099 \\
 & [0.065] & [0.148] & [0.160] & [0.160] & [0.163] &  & [0.069] \\
Padova Muni ITC & 0.143* & 0.093 & 0.080 & 0.076 & -0.021 &  & 0.160* \\
 & [0.086] & [0.161] & [0.170] & [0.170] & [0.174] &  & [0.086] \\
Male dummy &  &  & 0.017 & 0.020 & 0.019 & 0.019 & 0.005 \\
 &  &  & [0.035] & [0.034] & [0.066] & [0.064] & [0.034] \\
Constant & 0.118** & 0.411 & -10.023 & -10.084 & 63.795 & -42.060 & 27.568 \\
 & [0.056] & [0.416] & [46.941] & [46.632] & [90.484] & [87.114] & [45.942] \\
 &  &  &  &  &  &  &  \\
Observations & 635 & 635 & 635 & 635 & 635 & 238 & 635 \\
R-squared & 0.015 & 0.105 & 0.150 & 0.160 & 0.237 & 0.142 & 0.086 \\
 Controls & None & F.E. & Family & All & Inter & Reggio & no FE \\ \hline
\multicolumn{8}{c}{ Robust standard errors in brackets} \\
\multicolumn{8}{c}{ *** p$<$0.01, ** p$<$0.05, * p$<$0.1} \\
\multicolumn{8}{c}{ Dependent variable: Bothered by migrants (\%).} \\
\end{tabular}

\end{table}
\begin{table}[H]
\caption{Pool Regression: Migration Taste - Preschool, Adolescent}
\begin{tabular}{lccccccc} \hline
 & (1) & (2) & (3) & (4) & (5) & (6) & (7) \\
VARIABLES &  &  &  &  &  &  &  \\ \hline
 &  &  &  &  &  &  &  \\
RCH preschool & -0.053 & -0.140** & -0.120* & -0.113* & -0.100 & -0.100 & -0.043 \\
 & [0.044] & [0.063] & [0.064] & [0.063] & [0.068] & [0.067] & [0.045] \\
Reggio None Preschool & -0.136 & -0.231* & -0.230 & -0.241* & -0.198 & -0.198 & -0.160 \\
 & [0.135] & [0.132] & [0.140] & [0.143] & [0.151] & [0.147] & [0.154] \\
Parma None Preschool & 0.221 & 0.387 & 0.434 & 0.345 & 0.292 &  & 0.224 \\
 & [0.356] & [0.371] & [0.346] & [0.277] & [0.247] &  & [0.293] \\
Parma Muni Preschool & -0.065 & 0.041 & 0.047 & 0.047 & 0.042 &  & -0.059 \\
 & [0.051] & [0.061] & [0.062] & [0.062] & [0.061] &  & [0.052] \\
Padova None Preschool & -0.279*** & 0.000 & -0.019 & -0.022 & 0.035 &  & -0.474*** \\
 & [0.024] & [0.000] & [0.034] & [0.034] & [0.060] &  & [0.114] \\
Padova Muni Preschool & -0.076 & -0.165** & -0.160** & -0.161** & -0.151** &  & -0.055 \\
 & [0.054] & [0.065] & [0.063] & [0.063] & [0.067] &  & [0.054] \\
Male dummy &  &  & 0.019 & 0.022 & 0.023 & 0.023 & 0.002 \\
 &  &  & [0.034] & [0.034] & [0.066] & [0.065] & [0.035] \\
Constant & 0.279*** & 0.097 & 9.531 & 9.170 & 78.909 & -11.197 & 34.903 \\
 & [0.024] & [0.062] & [46.014] & [45.753] & [89.130] & [86.032] & [46.203] \\
 &  &  &  &  &  &  &  \\
Observations & 635 & 635 & 635 & 635 & 635 & 238 & 635 \\
R-squared & 0.007 & 0.117 & 0.162 & 0.170 & 0.241 & 0.136 & 0.081 \\
 Controls & None & F.E. & Family & All & Inter & Reggio & no FE \\ \hline
\multicolumn{8}{c}{ Robust standard errors in brackets} \\
\multicolumn{8}{c}{ *** p$<$0.01, ** p$<$0.05, * p$<$0.1} \\
\multicolumn{8}{c}{ Dependent variable: Bothered by migrants (\%).} \\
\end{tabular}

\end{table}

\pagebreak
\subsubsection{Adults Results}
\begin{table}[H]
\caption{Pool Regression: Depression - Infant-toddler center, Adult}
\begin{tabular}{lccccccc} \hline
 & (1) & (2) & (3) & (4) & (5) & (6) & (7) \\
VARIABLES &  &  &  &  &  &  &  \\ \hline
 &  &  &  &  &  &  &  \\
RCH infant-toddler & 0.538 & 2.422 & 2.361 & 2.361 & 2.753 & 2.753 & 2.334* \\
 & [0.977] & [1.865] & [2.079] & [2.079] & [1.943] & [1.926] & [1.239] \\
Reggio None ITC & 1.346 & 2.781 & 2.086 & 2.086 & 2.270 & 2.270 & 2.176* \\
 & [0.839] & [1.792] & [2.025] & [2.025] & [1.874] & [1.858] & [1.148] \\
Parma None ITC & 0.662 & 0.892 & 1.287 & 1.287 & 1.624 &  & 0.845 \\
 & [0.846] & [1.208] & [1.670] & [1.670] & [1.754] &  & [1.164] \\
Parma Muni ITC & -2.900*** & -0.504 & -1.147 & -1.147 & -1.025 &  & -2.772** \\
 & [0.997] & [1.316] & [1.835] & [1.835] & [1.898] &  & [1.407] \\
Padova None ITC & 0.448 & -1.881 & -1.562 & -1.562 & -1.779 &  & 0.152 \\
 & [0.843] & [1.413] & [2.580] & [2.580] & [2.512] &  & [1.141] \\
Padova Muni ITC & 5.882*** & 2.258 & 2.035 & 2.035 & 1.808 &  & 4.139** \\
 & [1.240] & [1.799] & [2.907] & [2.907] & [2.832] &  & [1.636] \\
Reggio None * Age 40-50 &  &  & 5.884*** & 5.884*** & 5.813*** & 5.813*** & 4.875*** \\
 &  &  & [2.145] & [2.145] & [2.062] & [2.044] & [1.327] \\
Reggio Muni * Age 40-50 &  &  & 4.863* & 4.863* & 4.424* & 4.424* & 4.274** \\
 &  &  & [2.492] & [2.492] & [2.430] & [2.409] & [1.779] \\
Reggio None * Age 40-50 &  &  & -8.264*** & -8.264*** & -2.482 & 1.148 & -10.644*** \\
 &  &  & [3.001] & [3.001] & [9.826] & [1.283] & [1.405] \\
Reggio Muni * Age 40-50 &  &  & -8.853*** & -8.853*** & -4.037 & -0.407 & -11.358*** \\
 &  &  & [3.037] & [3.037] & [9.859] & [1.294] & [1.503] \\
Reggio Stat * Age 40-50 &  &  & -8.924*** & -8.924*** & -3.312 & 0.319 & -11.326*** \\
 &  &  & [3.303] & [3.303] & [9.831] & [1.710] & [1.969] \\
Reggio Reli * Age 40-50 &  &  & -8.870*** & -8.870*** & -3.396 & 0.234 & -11.661*** \\
 &  &  & [3.050] & [3.050] & [9.805] & [1.299] & [1.530] \\
Reggio Priv * Age 40-50 &  &  & -9.252*** & -9.252*** & -3.631 &  & -12.449*** \\
 &  &  & [3.294] & [3.294] & [9.865] &  & [2.061] \\
Male dummy &  &  & -0.690*** & -0.690*** & -0.680 & -0.680 & -0.768*** \\
 &  &  & [0.255] & [0.255] & [0.419] & [0.416] & [0.259] \\
Constant & 20.694*** & 23.219*** & 44.410** & 44.410** & 66.728** & 34.716 & 48.121*** \\
 & [0.807] & [1.792] & [17.279] & [17.279] & [30.315] & [28.205] & [17.337] \\
 &  &  &  &  &  &  &  \\
Observations & 1,863 & 1,863 & 1,863 & 1,863 & 1,863 & 735 & 1,863 \\
R-squared & 0.034 & 0.148 & 0.183 & 0.183 & 0.219 & 0.171 & 0.104 \\
 Controls & None & F.E. & Family & All & Inter & Reggio & no FE \\ \hline
\multicolumn{8}{c}{ Robust standard errors in brackets} \\
\multicolumn{8}{c}{ *** p$<$0.01, ** p$<$0.05, * p$<$0.1} \\
\multicolumn{8}{c}{ Dependent variable: Depression score (CESD).} \\
\end{tabular}

\end{table}
\begin{table}[H]
\caption{Pool Regression: Depression - Preschool, Adult}
\begin{tabular}{lccccccc} \hline
 & (1) & (2) & (3) & (4) & (5) & (6) & (7) \\
VARIABLES &  &  &  &  &  &  &  \\ \hline
 &  &  &  &  &  &  &  \\
RCH preschool & 0.575 & 0.202 & 0.848 & 0.848 & 1.058* & 0.683 & 2.015*** \\
 & [0.387] & [0.500] & [0.675] & [0.675] & [0.635] & [0.718] & [0.498] \\
Reggio None Preschool & 2.166*** & 1.148** & 1.371 & 1.371 & 1.213 & 0.957 & 2.757*** \\
 & [0.394] & [0.539] & [0.967] & [0.967] & [0.969] & [0.993] & [0.835] \\
Parma None Preschool & 1.194*** & 1.337*** & 1.682* & 1.682* & 1.648* &  & 0.956 \\
 & [0.393] & [0.506] & [0.935] & [0.935] & [0.947] &  & [0.930] \\
Parma Muni Preschool & -0.586 & 0.038 & 0.245 & 0.245 & 0.425 &  & -0.406 \\
 & [0.517] & [0.606] & [0.834] & [0.834] & [0.843] &  & [0.711] \\
Padova None Preschool & 0.937* & 0.375 & 0.229 & 0.229 & -0.083 &  & 1.214 \\
 & [0.501] & [0.571] & [0.937] & [0.937] & [0.920] &  & [0.918] \\
Padova Muni Preschool & 1.021 & -0.648 & 0.092 & 0.092 & 0.181 &  & 1.163 \\
 & [0.738] & [0.775] & [1.128] & [1.128] & [1.112] &  & [0.987] \\
Reggio None * Age 40-50 &  &  & 7.968*** & 7.968*** & 8.072*** & 8.069*** & 6.996*** \\
 &  &  & [0.720] & [0.720] & [0.926] & [0.918] & [0.638] \\
Reggio Muni * Age 40-50 &  &  & 7.224*** & 7.224*** & 7.184*** & 7.176*** & 6.551*** \\
 &  &  & [1.367] & [1.367] & [1.490] & [1.477] & [1.270] \\
Reggio None * Age 40-50 &  &  & -10.790*** & -10.790*** & -6.658 & 0.202 & -13.341*** \\
 &  &  & [1.816] & [1.816] & [9.981] & [1.610] & [1.625] \\
Reggio Muni * Age 40-50 &  &  & -10.873*** & -10.873*** & -8.047 & -1.070 & -13.307*** \\
 &  &  & [1.786] & [1.786] & [9.921] & [1.479] & [1.580] \\
Reggio Stat * Age 40-50 &  &  & -10.072*** & -10.072*** & -6.240 & 0.353 & -11.233*** \\
 &  &  & [2.279] & [2.279] & [9.846] & [1.714] & [1.972] \\
Reggio Reli * Age 40-50 &  &  & -10.037*** & -10.037*** & -6.351 & 0.252 & -11.599*** \\
 &  &  & [1.897] & [1.897] & [9.821] & [1.302] & [1.529] \\
Reggio Priv * Age 40-50 &  &  & -10.425*** & -10.425*** & -6.601 &  & -12.381*** \\
 &  &  & [2.263] & [2.263] & [9.884] &  & [2.074] \\
Male dummy &  &  & -0.681*** & -0.681*** & -0.696* & -0.687* & -0.770*** \\
 &  &  & [0.255] & [0.255] & [0.419] & [0.415] & [0.260] \\
Constant & 20.845*** & 33.852*** & 61.866*** & 61.866*** & 79.039*** & 36.845 & 56.093*** \\
 & [0.214] & [0.539] & [17.542] & [17.542] & [30.513] & [28.227] & [17.353] \\
 &  &  &  &  &  &  &  \\
Observations & 1,863 & 1,863 & 1,863 & 1,863 & 1,863 & 735 & 1,863 \\
R-squared & 0.020 & 0.144 & 0.179 & 0.179 & 0.214 & 0.170 & 0.097 \\
 Controls & None & F.E. & Family & All & Inter & Reggio & no FE \\ \hline
\multicolumn{8}{c}{ Robust standard errors in brackets} \\
\multicolumn{8}{c}{ *** p$<$0.01, ** p$<$0.05, * p$<$0.1} \\
\multicolumn{8}{c}{ Dependent variable: Depression score (CESD).} \\
\end{tabular}

\end{table}

\begin{table}[H]
\caption{Pool Regression: Health - Infant-toddler center, Adult}
\begin{tabular}{lccccccc} \hline
 & (1) & (2) & (3) & (4) & (5) & (6) & (7) \\
VARIABLES &  &  &  &  &  &  &  \\ \hline
 &  &  &  &  &  &  &  \\
RCH infant-toddler & 0.416*** & -0.100*** & -0.042 & -0.042 & -0.033 & -0.033 & 0.350*** \\
 & [0.076] & [0.032] & [0.034] & [0.034] & [0.036] & [0.036] & [0.088] \\
Reggio None ITC & 0.195*** & -0.290*** & -0.094** & -0.094** & -0.084** & -0.084** & 0.296*** \\
 & [0.074] & [0.025] & [0.037] & [0.037] & [0.039] & [0.039] & [0.089] \\
Parma None ITC & 0.070 & 0.183* & 0.367*** & 0.367*** & 0.387*** &  & 0.182** \\
 & [0.075] & [0.098] & [0.123] & [0.123] & [0.126] &  & [0.092] \\
Parma Muni ITC & -0.260*** & 0.018 & 0.070 & 0.070 & 0.091 &  & -0.201* \\
 & [0.089] & [0.109] & [0.145] & [0.145] & [0.147] &  & [0.116] \\
Padova None ITC & 0.037 & 0.101 & 0.048 & 0.048 & 0.026 &  & 0.137 \\
 & [0.075] & [0.132] & [0.167] & [0.167] & [0.172] &  & [0.093] \\
Padova Muni ITC & 0.187* & 0.213 & -0.157 & -0.157 & -0.180 &  & -0.147 \\
 & [0.107] & [0.152] & [0.196] & [0.196] & [0.197] &  & [0.147] \\
Reggio None * Age 40-50 &  &  & -0.605*** & -0.605*** & -0.666*** & -0.666*** & -1.100*** \\
 &  &  & [0.066] & [0.066] & [0.078] & [0.078] & [0.104] \\
Reggio Muni * Age 40-50 &  &  & -0.650*** & -0.650*** & -0.747*** & -0.747*** & -1.135*** \\
 &  &  & [0.100] & [0.100] & [0.107] & [0.106] & [0.127] \\
Reggio None * Age 40-50 &  &  & -0.145 & -0.145 & -0.028 & -0.068 & -0.231 \\
 &  &  & [0.182] & [0.182] & [0.781] & [0.197] & [0.204] \\
Reggio Muni * Age 40-50 &  &  & -0.016 & -0.016 & -0.003 & -0.044 & -0.070 \\
 &  &  & [0.182] & [0.182] & [0.785] & [0.194] & [0.212] \\
Reggio Stat * Age 40-50 &  &  & -0.003 & -0.003 & 0.034 & -0.007 & -0.059 \\
 &  &  & [0.195] & [0.195] & [0.786] & [0.206] & [0.224] \\
Reggio Reli * Age 40-50 &  &  & -0.085 & -0.085 & -0.038 & -0.078 & -0.156 \\
 &  &  & [0.183] & [0.183] & [0.780] & [0.195] & [0.212] \\
Reggio Priv * Age 40-50 &  &  &  &  & 0.040 &  & -0.072 \\
 &  &  &  &  & [0.802] &  & [0.270] \\
Male dummy &  &  & 0.036* & 0.036* & 0.030 & 0.030 & 0.033* \\
 &  &  & [0.019] & [0.019] & [0.028] & [0.027] & [0.020] \\
Constant & 0.510*** & 1.290*** & 3.493*** & 3.493*** & 4.013 & 0.891 & 1.907 \\
 & [0.072] & [0.025] & [1.289] & [1.289] & [2.673] & [1.520] & [1.289] \\
 &  &  &  &  &  &  &  \\
Observations & 1,874 & 1,874 & 1,874 & 1,874 & 1,874 & 744 & 1,874 \\
R-squared & 0.062 & 0.163 & 0.323 & 0.323 & 0.343 & 0.357 & 0.275 \\
 Controls & None & F.E. & Family & All & Inter & Reggio & no FE \\ \hline
\multicolumn{8}{c}{ Robust standard errors in brackets} \\
\multicolumn{8}{c}{ *** p$<$0.01, ** p$<$0.05, * p$<$0.1} \\
\multicolumn{8}{c}{ Dependent variable: Respondent health is good (\%).} \\
\end{tabular}

\end{table}
\begin{table}[H]
\caption{Pool Regression: Health - Preschool, Adult}
\begin{tabular}{lccccccc} \hline
 & (1) & (2) & (3) & (4) & (5) & (6) & (7) \\
VARIABLES &  &  &  &  &  &  &  \\ \hline
 &  &  &  &  &  &  &  \\
RCH preschool & 0.282*** & 0.144*** & 0.071** & 0.071** & 0.069** & -0.000 & 0.203*** \\
 & [0.025] & [0.034] & [0.032] & [0.032] & [0.033] & [0.033] & [0.028] \\
Reggio None Preschool & -0.055 & -0.169*** & 0.007 & 0.007 & 0.012 & -0.035 & 0.131*** \\
 & [0.035] & [0.043] & [0.049] & [0.049] & [0.049] & [0.049] & [0.045] \\
Parma None Preschool & -0.130*** & -0.074 & 0.134** & 0.134** & 0.146** &  & 0.046 \\
 & [0.039] & [0.046] & [0.067] & [0.067] & [0.070] &  & [0.067] \\
Parma Muni Preschool & -0.120*** & -0.048 & -0.019 & -0.019 & -0.019 &  & -0.102** \\
 & [0.042] & [0.051] & [0.067] & [0.067] & [0.069] &  & [0.052] \\
Padova None Preschool & -0.138*** & -0.109** & 0.134* & 0.134* & 0.125* &  & 0.071 \\
 & [0.042] & [0.048] & [0.074] & [0.074] & [0.075] &  & [0.063] \\
Padova Muni Preschool & -0.029 & 0.041 & -0.003 & -0.003 & -0.022 &  & -0.091 \\
 & [0.063] & [0.065] & [0.096] & [0.096] & [0.096] &  & [0.090] \\
Reggio None * Age 40-50 &  &  & -0.693*** & -0.693*** & -0.748*** & -0.750*** & -0.806*** \\
 &  &  & [0.059] & [0.059] & [0.077] & [0.076] & [0.054] \\
Reggio Muni * Age 40-50 &  &  & -0.688*** & -0.688*** & -0.778*** & -0.779*** & -0.785*** \\
 &  &  & [0.095] & [0.095] & [0.104] & [0.103] & [0.092] \\
Reggio None * Age 40-50 &  &  & 0.271 & 0.271 & -0.427 & -0.034 & -0.291 \\
 &  &  & [0.201] & [0.201] & [0.725] & [0.203] & [0.182] \\
Reggio Muni * Age 40-50 &  &  & 0.333 & 0.333 & -0.459 & -0.043 & -0.200 \\
 &  &  & [0.206] & [0.206] & [0.729] & [0.198] & [0.179] \\
Reggio Stat * Age 40-50 &  &  & 0.417* & 0.417* & -0.352 & -0.007 & 0.015 \\
 &  &  & [0.217] & [0.217] & [0.729] & [0.206] & [0.190] \\
Reggio Reli * Age 40-50 &  &  & 0.336 & 0.336 & -0.424 & -0.078 & -0.085 \\
 &  &  & [0.206] & [0.206] & [0.722] & [0.195] & [0.178] \\
Reggio Priv * Age 40-50 &  &  & 0.421 & 0.421 & -0.346 &  &  \\
 &  &  & [0.269] & [0.269] & [0.746] &  &  \\
Male dummy &  &  & 0.034* & 0.034* & 0.027 & 0.029 & 0.030 \\
 &  &  & [0.020] & [0.020] & [0.027] & [0.027] & [0.020] \\
Constant & 0.620*** & 1.169*** & 3.425*** & 3.425*** & 3.427 & 0.845 & 2.142* \\
 & [0.018] & [0.043] & [1.268] & [1.268] & [2.582] & [1.513] & [1.296] \\
 &  &  &  &  &  &  &  \\
Observations & 1,874 & 1,874 & 1,874 & 1,874 & 1,874 & 744 & 1,874 \\
R-squared & 0.073 & 0.184 & 0.316 & 0.316 & 0.336 & 0.356 & 0.262 \\
 Controls & None & F.E. & Family & All & Inter & Reggio & no FE \\ \hline
\multicolumn{8}{c}{ Robust standard errors in brackets} \\
\multicolumn{8}{c}{ *** p$<$0.01, ** p$<$0.05, * p$<$0.1} \\
\multicolumn{8}{c}{ Dependent variable: Respondent health is good (\%).} \\
\end{tabular}

\end{table}

\begin{table}[H]
\caption{Pool Regression: Migration Taste - Infant-toddler center, Adult}
\begin{tabular}{lccccccc} \hline
 & (1) & (2) & (3) & (4) & (5) & (6) & (7) \\
VARIABLES &  &  &  &  &  &  &  \\ \hline
 &  &  &  &  &  &  &  \\
RCH infant-toddler & -0.337*** & -0.069 & 0.092** & 0.092** & 0.091** & 0.091** & -0.280*** \\
 & [0.074] & [0.135] & [0.036] & [0.036] & [0.044] & [0.044] & [0.092] \\
Reggio None ITC & -0.197*** & 0.065 & 0.154*** & 0.154*** & 0.156*** & 0.156*** & -0.216** \\
 & [0.071] & [0.134] & [0.029] & [0.029] & [0.037] & [0.037] & [0.089] \\
Parma None ITC & -0.175** & -0.113 & -0.231* & -0.231* & -0.196 &  & -0.135 \\
 & [0.072] & [0.092] & [0.120] & [0.120] & [0.120] &  & [0.091] \\
Parma Muni ITC & -0.209** & -0.113 & -0.159 & -0.159 & -0.138 &  & -0.095 \\
 & [0.084] & [0.102] & [0.140] & [0.140] & [0.139] &  & [0.115] \\
Padova None ITC & -0.051 & -0.258** & -0.097 & -0.097 & -0.058 &  & -0.029 \\
 & [0.072] & [0.114] & [0.159] & [0.159] & [0.163] &  & [0.092] \\
Padova Muni ITC & 0.069 & -0.216 & -0.137 & -0.137 & -0.124 &  & -0.017 \\
 & [0.112] & [0.148] & [0.202] & [0.202] & [0.201] &  & [0.150] \\
Reggio None * Age 40-50 &  &  & -0.878*** & -0.878*** & -0.750 & -0.304 & 0.185 \\
 &  &  & [0.060] & [0.060] & [0.821] & [0.463] & [0.254] \\
Reggio Muni * Age 40-50 &  &  & -0.931*** & -0.931*** & -0.795 & -0.349 & 0.131 \\
 &  &  & [0.084] & [0.084] & [0.821] & [0.462] & [0.261] \\
Reggio Reli * Age 40-50 &  &  &  &  & 0.159 & 0.605 & 0.694*** \\
 &  &  &  &  & [0.828] & [0.474] & [0.268] \\
Reggio None * Age 40-50 &  &  & 0.101 & 0.101 & 0.093 & 0.093 & 0.088 \\
 &  &  & [0.155] & [0.155] & [0.149] & [0.148] & [0.147] \\
Reggio Muni * Age 40-50 &  &  & 0.041 & 0.041 & 0.061 & 0.061 & 0.035 \\
 &  &  & [0.154] & [0.154] & [0.146] & [0.145] & [0.146] \\
Reggio Stat * Age 40-50 &  &  & 0.097 & 0.097 & 0.112 & 0.112 & 0.085 \\
 &  &  & [0.170] & [0.170] & [0.163] & [0.162] & [0.163] \\
Reggio Reli * Age 40-50 &  &  & 0.122 & 0.122 & 0.131 & 0.131 & 0.119 \\
 &  &  & [0.157] & [0.157] & [0.149] & [0.148] & [0.150] \\
Male dummy &  &  & 0.031 & 0.031 & 0.048 & 0.048* & 0.037* \\
 &  &  & [0.020] & [0.020] & [0.029] & [0.029] & [0.020] \\
Constant & 0.400*** & -0.065 & -0.539 & -0.539 & 4.385 & -2.210 & 0.480 \\
 & [0.069] & [0.134] & [1.385] & [1.385] & [2.679] & [1.944] & [1.365] \\
 &  &  &  &  &  &  &  \\
Observations & 1,866 & 1,866 & 1,866 & 1,866 & 1,866 & 739 & 1,866 \\
R-squared & 0.037 & 0.075 & 0.111 & 0.111 & 0.132 & 0.066 & 0.080 \\
 Controls & None & F.E. & Family & All & Inter & Reggio & no FE \\ \hline
\multicolumn{8}{c}{ Robust standard errors in brackets} \\
\multicolumn{8}{c}{ *** p$<$0.01, ** p$<$0.05, * p$<$0.1} \\
\multicolumn{8}{c}{ Dependent variable: Bothered by migrants (\%).} \\
\end{tabular}

\end{table}
\begin{table}[H]
\caption{Pool Regression: Migration Taste - Preschool, Adult}
\begin{tabular}{lccccccc} \hline
 & (1) & (2) & (3) & (4) & (5) & (6) & (7) \\
VARIABLES &  &  &  &  &  &  &  \\ \hline
 &  &  &  &  &  &  &  \\
RCH preschool & -0.179*** & -0.115*** & -0.104** & -0.104** & -0.101** & -0.088 & -0.163*** \\
 & [0.025] & [0.035] & [0.048] & [0.048] & [0.049] & [0.055] & [0.033] \\
Reggio None Preschool & -0.062** & 0.010 & -0.055 & -0.055 & -0.049 & -0.040 & -0.119** \\
 & [0.030] & [0.041] & [0.066] & [0.066] & [0.067] & [0.069] & [0.053] \\
Parma None Preschool & -0.135*** & -0.079* & -0.090 & -0.090 & -0.073 &  & -0.089 \\
 & [0.031] & [0.042] & [0.076] & [0.076] & [0.077] &  & [0.065] \\
Parma Muni Preschool & 0.012 & 0.067 & -0.014 & -0.014 & -0.011 &  & 0.007 \\
 & [0.040] & [0.050] & [0.068] & [0.068] & [0.069] &  & [0.051] \\
Padova None Preschool & 0.040 & -0.009 & 0.088 & 0.088 & 0.104 &  & 0.126 \\
 & [0.040] & [0.047] & [0.083] & [0.083] & [0.083] &  & [0.077] \\
Padova Muni Preschool & 0.207*** & 0.096 & 0.206** & 0.206** & 0.215** &  & 0.273*** \\
 & [0.064] & [0.066] & [0.099] & [0.099] & [0.095] &  & [0.092] \\
Reggio None * Age 40-50 &  &  & -0.634*** & -0.634*** & -0.437 & -0.279 & -0.726*** \\
 &  &  & [0.202] & [0.202] & [0.774] & [0.462] & [0.053] \\
Reggio Muni * Age 40-50 &  &  & -0.749*** & -0.749*** & -0.549 & -0.391 & -0.846*** \\
 &  &  & [0.208] & [0.208] & [0.774] & [0.462] & [0.072] \\
Reggio Reli * Age 40-50 &  &  & 0.094 & 0.094 & 0.318 & 0.476 &  \\
 &  &  & [0.208] & [0.208] & [0.788] & [0.485] &  \\
Reggio None * Age 40-50 &  &  & 0.155 & 0.155 & 0.143 & 0.134 & 0.205 \\
 &  &  & [0.169] & [0.169] & [0.164] & [0.164] & [0.156] \\
Reggio Muni * Age 40-50 &  &  & 0.144 & 0.144 & 0.163 & 0.150 & 0.197 \\
 &  &  & [0.162] & [0.162] & [0.154] & [0.155] & [0.150] \\
Reggio Stat * Age 40-50 &  &  & 0.095 & 0.095 & 0.111 & 0.111 & 0.082 \\
 &  &  & [0.171] & [0.171] & [0.163] & [0.162] & [0.164] \\
Reggio Reli * Age 40-50 &  &  & 0.121 & 0.121 & 0.132 & 0.132 & 0.117 \\
 &  &  & [0.157] & [0.157] & [0.149] & [0.148] & [0.151] \\
Male dummy &  &  & 0.030 & 0.030 & 0.051* & 0.051* & 0.037* \\
 &  &  & [0.020] & [0.020] & [0.030] & [0.029] & [0.020] \\
Constant & 0.293*** & -0.010 & -0.293 & -0.293 & 4.158 & -1.761 & 0.313 \\
 & [0.017] & [0.041] & [1.365] & [1.365] & [2.570] & [1.939] & [1.325] \\
 &  &  &  &  &  &  &  \\
Observations & 1,866 & 1,866 & 1,866 & 1,866 & 1,866 & 739 & 1,866 \\
R-squared & 0.041 & 0.080 & 0.113 & 0.113 & 0.135 & 0.067 & 0.086 \\
 Controls & None & F.E. & Family & All & Inter & Reggio & no FE \\ \hline
\multicolumn{8}{c}{ Robust standard errors in brackets} \\
\multicolumn{8}{c}{ *** p$<$0.01, ** p$<$0.05, * p$<$0.1} \\
\multicolumn{8}{c}{ Dependent variable: Bothered by migrants (\%).} \\
\end{tabular}

\end{table}

\todo[backgroundcolor=orange!30,size=\tiny]{Insert here the p-values of the differences in coefficients}

\normalsize

\subsection{Propensity Score Matching Results}
\label{sec:PSM}
Another approach considered was to do a propensity score matching.




%\subsection{Instrumental Variable Approach}
%\label{sec:IV}
