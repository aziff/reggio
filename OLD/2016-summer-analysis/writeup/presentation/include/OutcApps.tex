%% This table can be obtained using the dataset: klmReggio/ADMINISTRATIVE DATA COLLECTION/DATI/ReggioAdminData.dta
%% tab anno_scolastico bam_scol, row

\begin{table}[ht!]
\caption{\textbf{\small Outcome of Applications to Municipal Preschools}}
\label{tab:OutcApps}
%\vspace{-5mm}
\begin{center}
\begin{tabular}{c c c c c }
\hline\hline
\textbf{School Year} & \textbf{Attended} & \textbf{Withdrew} & \textbf{Waitlist} & \textbf{Refused} \\
\hline
        2003-2004  &   48.3  &    42.8  &    2.9  &    6.0   \\[0.2em]
        2004-2005  &   48.3  &    36.9  &    6.2  &    8.6   \\[0.2em]
        2005-2006  &   60.1  &    20.0  &    8.1  &   11.9   \\[0.2em]
        2006-2007  &   52.8  &    24.8  &   13.4  &    8.9   \\[0.2em]
        2007-2008  &   48.5  &    20.9  &   23.4  &    7.1   \\[0.2em]
        2008-2009  &   53.4  &     1.3  &   35.6  &    9.8   \\[0.2em]
        2009-2010  &   52.2  &     2.1  &   36.2  &    9.6   \\[0.2em]
        2010-2011  &   53.5  &     1.9  &   37.1  &    7.5   \\[0.2em]
\hline
\textit{Total} &   \textit{52.2 \%} &   \textit{18.4 \%} &   \textit{20.6 \%} &    \textit{8.7 \%}  \\[0.2em]
\hline
\end{tabular}
\end{center}
\begin{flushleft}
\tiny{{\bfseries Notes:} Source: authors' calculations from the administrative data on the universe of applications to the municipal preschools of Reggio Emilia. The cells display the percentages of applications for that particular school year which had that particular outcome: the child attended the assigned preschool (column 1); the child withdrew from the municipal preschool system (either because she went to another preschool or stayed home) after being accepted but before being assigned a specific preschool (column 2); the child was on waiting list (column three); the child dropped out from the municipal school system after a specific preschool was assigned (column 4).}
\end{flushleft}
\end{table}
