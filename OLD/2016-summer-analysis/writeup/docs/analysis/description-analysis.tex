\documentclass[12pt]{article}

\usepackage{amsmath}
\usepackage{amssymb}
\usepackage{booktabs}
\usepackage{datetime}
\usepackage{float}
\usepackage[margin=0.5in,includefoot]{geometry}
\usepackage{hyperref}
\usepackage{morefloats}
\usepackage{rotating}

\usepackage{setspace}

\usepackage{caption} 
\captionsetup[table]{skip=2pt}

\settimeformat{hhmmsstime}

\begin{document}

\title{Reggio Description and Analysis}
\author{Reggio Team}
\date{Original version: Monday  21$^{\text{st}}$ June, 2016 \\ Current version: \today \\ \vspace{1em} Time: \currenttime}
\maketitle

\tableofcontents
\listoftables

\doublespacing

\section{Overview}
The purpose of this document is to record information and analysis amassed during analysis of the Reggio Children Approach to Early Childhood Education (referred to as RA). Although this approach, implemented beginning in 196X in Reggio Emilia, Italy, has been emulated widely outside of Reggio Emilia, no official evaluation exists.\footnote{The official \href{http://www.reggiochildren.it/network/?lang=en}{Reggio Children International Network} is present in 33 countries worldwide. Many other preschools around the world are ``inspired'' by the Reggio Children Approach but they are not officially part of these network.} This document presents information about the sample, the cities in which they live, and the early childhood education centers that they attended. Any intermediary analyses to help better understand the data are recorded here, as well as an explanation of the desired parameters to evaluate RA, and the empirical strategy to estimate those parameters. 


\section{Context of Sample}
The sample consists of six cohorts of individuals living in three different cities, with data collected on family members when applicable. Individuals in the sample were chosen from a reference sample Table \ref{tab:basic-cohorts} lists the cohorts, Table \ref{tab:basic}
\subsection{Reggio Emilia, Parma, and Padova}
\subsection{Types of Early Childhood Centers}
\subsection{Comparison of Pedagogies}

\section{Descriptive Overview}
\subsection{Description of Baseline Measures}
This section presents a description of the baseline variables. The stars indicate significant differences from the RA group in the same cohort. The tables are disaggregated by cohort and city.

\begin{table}[htbp]\centering
\def\sym#1{\ifmmode^{#1}\else\(^{#1}\)\fi}
\caption{Child, Reggio, Baseline Characteristics}
\begin{tabular}{l*{1}{ccc}}
\toprule
                    &\multicolumn{3}{c}{}                           \\
                    &        Mean         &  Difference&           N\\
\midrule
Male indicator      &                     &            &            \\
Age                 &                     &            &            \\
Low birthweight     &                     &            &            \\
Premature birth     &                     &            &            \\
CAPI                &                     &            &            \\
Mother: age at birth&                     &            &            \\
Mother: born in province&                     &            &            \\
Mother max. edu.: less than middle school&                     &            &            \\
Mother max. edu.: middle school&                     &            &            \\
Mother max. edu.: high school&                     &            &            \\
Mother max. edu.: university&                     &            &            \\
Father: age at birth&                     &            &            \\
Father: born in province&                     &            &            \\
Father max. edu.: less than middle school&                     &            &            \\
Father max. edu.: middle school&                     &            &            \\
Father max. edu.: high school&                     &            &            \\
Father max. edu.: university&                     &            &            \\
Number of siblings  &                     &            &            \\
Religious caregiver indicator&                     &            &            \\
Home ownership indicator&                     &            &            \\
Mother: born outside of Italy&                     &            &            \\
Income: 5,000 euros or less&                     &            &            \\
Income: 5,001-10,000 euros&                     &            &            \\
Income: 10,001-25,000 euros&                     &            &            \\
Income: 25,001-50,000 euros&                     &            &            \\
Income: 50,001-100,000 euros&                     &            &            \\
Income: 100,001-250,000 euros&                     &            &            \\
Income: more than 250,000 euros&                     &            &            \\
\bottomrule
\end{tabular}
\end{table}

\begin{table}[htbp]\centering
\def\sym#1{\ifmmode^{#1}\else\(^{#1}\)\fi}
\caption{Child, Parma, Baseline Characteristics}
\begin{tabular}{l*{1}{ccc}}
\toprule
                    &\multicolumn{3}{c}{}                           \\
                    &        mu\_1         &        diff&         N\_1\\
\midrule
Male indicator      &       0.667         &       0.109&           6\\
Age                 &       6.803         &       0.009&           6\\
Low birthweight     &       0.000         &      -0.082&           6\\
Premature birth     &       0.000         &      -0.088&           6\\
CAPI                &       0.500         &      -0.085&           6\\
Mother: age at birth&      35.301         &       2.199&           6\\
Mother: born in province&       0.500         &      -0.003&           6\\
Mother max. edu.: less than middle school&       0.000         &      -0.136&           6\\
Mother max. edu.: middle school&       0.000         &      -0.082&           6\\
Mother max. edu.: high school&       0.167         &      -0.323&           6\\
Mother max. edu.: university&       0.833\sym{**} &       0.554&           6\\
Father: age at birth&      38.011         &       2.513&           6\\
Father: born in province&       0.167         &      -0.405&           6\\
Father max. edu.: less than middle school&       0.333         &       0.088&           6\\
Father max. edu.: middle school&       0.000         &      -0.088&           6\\
Father max. edu.: high school&       0.333         &      -0.054&           6\\
Father max. edu.: university&       0.333         &       0.068&           6\\
Number of siblings  &       0.833         &      -0.194&           6\\
Religious caregiver indicator&       0.833         &       0.024&           6\\
Home ownership indicator&       0.833         &       0.276&           6\\
Mother: born outside of Italy&       0.000         &      -0.082&           6\\
Income: 5,000 euros or less&       0.000         &       0.000&           6\\
Income: 5,001-10,000 euros&       0.000         &      -0.014&           6\\
Income: 10,001-25,000 euros&       0.333         &       0.184&           6\\
Income: 25,001-50,000 euros&       0.667         &       0.320&           6\\
Income: 50,001-100,000 euros&       0.000         &      -0.211&           6\\
Income: 100,001-250,000 euros&       0.000         &      -0.041&           6\\
Income: more than 250,000 euros&       0.000         &       0.000&           6\\
\bottomrule
\end{tabular}
\end{table}

\begin{table}[htbp]\centering
\def\sym#1{\ifmmode^{#1}\else\(^{#1}\)\fi}
\caption{Child, Padova, Baseline Characteristics}
\begin{tabular}{l*{1}{ccc}}
\toprule
                    &\multicolumn{3}{c}{}                           \\
                    &        mu\_1         &        diff&         N\_1\\
\midrule
Male indicator      &       0.500         &      -0.058&           2\\
Age                 &       6.386         &      -0.408&           2\\
Low birthweight     &       0.000         &      -0.082&           2\\
Premature birth     &       0.000         &      -0.088&           2\\
CAPI                &       0.000         &      -0.585&           2\\
Mother: age at birth&      34.920         &       1.818&           2\\
Mother: born in province&       0.000         &      -0.503&           2\\
Mother max. edu.: less than middle school&       0.500         &       0.364&           2\\
Mother max. edu.: middle school&       0.000         &      -0.082&           2\\
Mother max. edu.: high school&       0.000         &      -0.490&           2\\
Mother max. edu.: university&       0.500         &       0.221&           2\\
Father: age at birth&      40.285         &       4.787&           2\\
Father: born in province&       0.500         &      -0.071&           2\\
Father max. edu.: less than middle school&       0.500         &       0.255&           2\\
Father max. edu.: middle school&       0.000         &      -0.088&           2\\
Father max. edu.: high school&       0.000         &      -0.388&           2\\
Father max. edu.: university&       0.500         &       0.235&           2\\
Number of siblings  &       2.000         &       0.973&           2\\
Religious caregiver indicator&       1.000         &       0.190&           2\\
Home ownership indicator&       0.500         &      -0.058&           2\\
Mother: born outside of Italy&       0.000         &      -0.082&           2\\
Income: 5,000 euros or less&       0.000         &       0.000&           2\\
Income: 5,001-10,000 euros&       0.000         &      -0.014&           2\\
Income: 10,001-25,000 euros&       0.000         &      -0.150&           2\\
Income: 25,001-50,000 euros&       0.000         &      -0.347&           2\\
Income: 50,001-100,000 euros&       0.000         &      -0.211&           2\\
Income: 100,001-250,000 euros&       0.500\sym{**} &       0.459&           2\\
Income: more than 250,000 euros&       0.000         &       0.000&           2\\
\bottomrule
\end{tabular}
\end{table}

\begin{table}[htbp]\centering
\def\sym#1{\ifmmode^{#1}\else\(^{#1}\)\fi}
\caption{Migrant, Reggio, Baseline Characteristics}
\begin{tabular}{l*{1}{ccc}}
\toprule
                    &\multicolumn{3}{c}{}                           \\
                    &        mu\_1         &        diff&         N\_1\\
\midrule
Male indicator      &       0.500         &       0.000&           4\\
Age                 &       6.533         &      -0.278&           4\\
Low birthweight     &       0.000         &      -0.143&           4\\
Premature birth     &       0.000         &      -0.119&           4\\
CAPI                &       0.750         &      -0.036&           4\\
Mother: age at birth&      27.214         &      -1.886&           4\\
Mother: born in province&       0.000         &       0.000&           4\\
Mother max. edu.: less than middle school&       0.500         &       0.190&           4\\
Mother max. edu.: middle school&       0.000         &       0.000&           4\\
Mother max. edu.: high school&       0.500         &       0.095&           4\\
Mother max. edu.: university&       0.000         &      -0.095&           4\\
Father: age at birth&      29.433\sym{*}  &      -6.936&           4\\
Father: born in province&       0.000         &       0.000&           4\\
Father max. edu.: less than middle school&       0.750         &       0.464&           4\\
Father max. edu.: middle school&       0.000         &       0.000&           4\\
Father max. edu.: high school&       0.250         &      -0.155&           4\\
Father max. edu.: university&       0.000         &      -0.071&           4\\
Number of siblings  &       2.250         &       0.821&           4\\
Religious caregiver indicator&       0.250\sym{**} &      -0.583&           4\\
Home ownership indicator&       0.500         &       0.310&           4\\
Mother: born outside of Italy&       1.000         &       0.000&           4\\
Income: 5,000 euros or less&       0.000         &      -0.024&           4\\
Income: 5,001-10,000 euros&       0.000         &      -0.024&           4\\
Income: 10,001-25,000 euros&       0.000         &      -0.405&           4\\
Income: 25,001-50,000 euros&       0.500         &       0.238&           4\\
Income: 50,001-100,000 euros&       0.000         &       0.000&           4\\
Income: 100,001-250,000 euros&       0.000         &       0.000&           4\\
Income: more than 250,000 euros&       0.000         &       0.000&           4\\
Migrants: year entered city&    2002.250         &       0.012&           4\\
Migrants: age entered city&      23.500         &      -1.881&           4\\
\bottomrule
\end{tabular}
\end{table}

\begin{table}[htbp]\centering
\def\sym#1{\ifmmode^{#1}\else\(^{#1}\)\fi}
\caption{Migrant, Parma, Baseline Characteristics}
\begin{tabular}{l*{1}{ccc}}
\toprule
                    &\multicolumn{3}{c}{}                           \\
                    &        mu\_1         &        diff&         N\_1\\
\midrule
Male indicator      &       0.750         &       0.250&           4\\
Age                 &       6.967         &       0.156&           4\\
Low birthweight     &       0.000         &      -0.143&           4\\
Premature birth     &       0.000         &      -0.119&           4\\
CAPI                &       0.750         &      -0.036&           4\\
Mother: age at birth&      25.828         &      -3.273&           4\\
Mother: born in province&       0.000         &       0.000&           4\\
Mother max. edu.: less than middle school&       0.500         &       0.190&           4\\
Mother max. edu.: middle school&       0.000         &       0.000&           4\\
Mother max. edu.: high school&       0.250         &      -0.155&           4\\
Mother max. edu.: university&       0.000         &      -0.095&           4\\
Father: age at birth&      38.581         &       2.212&           4\\
Father: born in province&       0.000         &       0.000&           4\\
Father max. edu.: less than middle school&       0.250         &      -0.036&           4\\
Father max. edu.: middle school&       0.000         &       0.000&           4\\
Father max. edu.: high school&       0.500         &       0.095&           4\\
Father max. edu.: university&       0.000         &      -0.071&           4\\
Number of siblings  &       1.250         &      -0.179&           4\\
Religious caregiver indicator&       1.000         &       0.167&           4\\
Home ownership indicator&       0.000         &      -0.190&           4\\
Mother: born outside of Italy&       1.000         &       0.000&           4\\
Income: 5,000 euros or less&       0.000         &      -0.024&           4\\
Income: 5,001-10,000 euros&       0.750\sym{***}&       0.726&           4\\
Income: 10,001-25,000 euros&       0.250         &      -0.155&           4\\
Income: 25,001-50,000 euros&       0.000         &      -0.262&           4\\
Income: 50,001-100,000 euros&       0.000         &       0.000&           4\\
Income: 100,001-250,000 euros&       0.000         &       0.000&           4\\
Income: more than 250,000 euros&       0.000         &       0.000&           4\\
Migrants: year entered city&    1996.500\sym{***}&      -5.738&           4\\
Migrants: age entered city&      16.250\sym{**} &      -9.131&           4\\
\bottomrule
\end{tabular}
\end{table}

\begin{table}[htbp]\centering
\def\sym#1{\ifmmode^{#1}\else\(^{#1}\)\fi}
\caption{Migrant, Padova, Baseline Characteristics}
\begin{tabular}{l*{1}{ccc}}
\toprule
                    &\multicolumn{3}{c}{}                           \\
                    &        mu\_1         &        diff&         N\_1\\
\midrule
Male indicator      &       0.667         &       0.167&           3\\
Age                 &       6.465         &      -0.346&           3\\
Low birthweight     &       0.000         &      -0.143&           3\\
Premature birth     &       0.000         &      -0.119&           3\\
CAPI                &       0.333         &      -0.452&           3\\
Mother: age at birth&      29.519         &       0.418&           3\\
Mother: born in province&       0.000         &       0.000&           3\\
Mother max. edu.: less than middle school&       0.667         &       0.357&           3\\
Mother max. edu.: middle school&       0.000         &       0.000&           3\\
Mother max. edu.: high school&       0.333         &      -0.071&           3\\
Mother max. edu.: university&       0.000         &      -0.095&           3\\
Father: age at birth&      33.147         &      -3.222&           3\\
Father: born in province&       0.000         &       0.000&           3\\
Father max. edu.: less than middle school&       0.667         &       0.381&           3\\
Father max. edu.: middle school&       0.000         &       0.000&           3\\
Father max. edu.: high school&       0.000         &      -0.405&           3\\
Father max. edu.: university&       0.333         &       0.262&           3\\
Number of siblings  &       1.333         &      -0.095&           3\\
Religious caregiver indicator&       1.000         &       0.167&           3\\
Home ownership indicator&       0.667         &       0.476&           3\\
Mother: born outside of Italy&       1.000         &       0.000&           3\\
Income: 5,000 euros or less&       0.000         &      -0.024&           3\\
Income: 5,001-10,000 euros&       0.000         &      -0.024&           3\\
Income: 10,001-25,000 euros&       1.000\sym{*}  &       0.595&           3\\
Income: 25,001-50,000 euros&       0.000         &      -0.262&           3\\
Income: 50,001-100,000 euros&       0.000         &       0.000&           3\\
Income: 100,001-250,000 euros&       0.000         &       0.000&           3\\
Income: more than 250,000 euros&       0.000         &       0.000&           3\\
Migrants: year entered city&    1999.000\sym{*}  &      -3.238&           3\\
Migrants: age entered city&      22.333         &      -3.048&           3\\
\bottomrule
\end{tabular}
\end{table}

\begin{table}[htbp]\centering
\def\sym#1{\ifmmode^{#1}\else\(^{#1}\)\fi}
\caption{Adolescent, Padova, Baseline Characteristics}
\begin{tabular}{l*{1}{ccc}}
\toprule
                    &\multicolumn{3}{c}{}                           \\
                    &        mu\_1         &        diff&         N\_1\\
\midrule
Male indicator      &       1.000         &       0.547&           1\\
Age                 &      18.557         &      -0.135&           1\\
Low birthweight     &       1.000         &       0.949&           1\\
Premature birth     &       1.000         &       0.971&           1\\
CAPI                &       0.000         &      -0.460&           1\\
Mother: age at birth&      33.029         &       2.695&           1\\
Mother: born in province&       1.000         &       0.241&           1\\
Mother max. edu.: less than middle school&       0.000         &      -0.131&           1\\
Mother max. edu.: middle school&       0.000         &      -0.095&           1\\
Mother max. edu.: high school&       1.000         &       0.467&           1\\
Mother max. edu.: university&       0.000         &      -0.241&           1\\
Father: age at birth&      35.414         &       2.648&           1\\
Father: born in province&       1.000         &       0.277&           1\\
Father max. edu.: less than middle school&       0.000         &      -0.219&           1\\
Father max. edu.: middle school&       0.000         &      -0.088&           1\\
Father max. edu.: high school&       1.000         &       0.518&           1\\
Father max. edu.: university&       0.000         &      -0.212&           1\\
Number of siblings  &       1.000         &      -0.204&           1\\
Religious caregiver indicator&       1.000         &       0.307&           1\\
Home ownership indicator&       0.000         &      -0.869&           1\\
Mother: born outside of Italy&       0.000         &       0.000&           1\\
Income: 5,000 euros or less&       0.000         &       0.000&           1\\
Income: 5,001-10,000 euros&       0.000         &      -0.015&           1\\
Income: 10,001-25,000 euros&       0.000         &      -0.131&           1\\
Income: 25,001-50,000 euros&       0.000         &      -0.307&           1\\
Income: 50,001-100,000 euros&       0.000         &      -0.292&           1\\
Income: 100,001-250,000 euros&       0.000         &      -0.044&           1\\
Income: more than 250,000 euros&       0.000         &       0.000&           1\\
\bottomrule
\end{tabular}
\end{table}

\begin{table}[htbp]\centering
\def\sym#1{\ifmmode^{#1}\else\(^{#1}\)\fi}
\caption{Adolescent, Parma, Baseline Characteristics}
\begin{tabular}{l*{1}{ccc}}
\toprule
                    &\multicolumn{3}{c}{}                           \\
                    &        mu\_1         &        diff&         N\_1\\
\midrule
Male indicator      &       0.500         &       0.047&           2\\
Age                 &      18.736         &       0.044&           2\\
Low birthweight     &       0.500\sym{**} &       0.449&           2\\
Premature birth     &       0.500\sym{***}&       0.471&           2\\
CAPI                &       0.000         &      -0.460&           2\\
Mother: age at birth&      29.336         &      -0.999&           2\\
Mother: born in province&       0.000\sym{*}  &      -0.759&           2\\
Mother max. edu.: less than middle school&       0.000         &      -0.131&           2\\
Mother max. edu.: middle school&       0.000         &      -0.095&           2\\
Mother max. edu.: high school&       0.500         &      -0.033&           2\\
Mother max. edu.: university&       0.500         &       0.259&           2\\
Father: age at birth&      32.424         &      -0.342&           2\\
Father: born in province&       0.000\sym{*}  &      -0.723&           2\\
Father max. edu.: less than middle school&       0.000         &      -0.219&           2\\
Father max. edu.: middle school&       0.000         &      -0.088&           2\\
Father max. edu.: high school&       0.500         &       0.018&           2\\
Father max. edu.: university&       0.500         &       0.288&           2\\
Number of siblings  &       1.000         &      -0.204&           2\\
Religious caregiver indicator&       1.000         &       0.307&           2\\
Home ownership indicator&       1.000         &       0.131&           2\\
Mother: born outside of Italy&       0.000         &       0.000&           2\\
Income: 5,000 euros or less&       0.000         &       0.000&           2\\
Income: 5,001-10,000 euros&       0.000         &      -0.015&           2\\
Income: 10,001-25,000 euros&       0.000         &      -0.131&           2\\
Income: 25,001-50,000 euros&       0.500         &       0.193&           2\\
Income: 50,001-100,000 euros&       0.000         &      -0.292&           2\\
Income: 100,001-250,000 euros&       0.000         &      -0.044&           2\\
Income: more than 250,000 euros&       0.000         &       0.000&           2\\
\bottomrule
\end{tabular}
\end{table}

\begin{table}[htbp]\centering
\def\sym#1{\ifmmode^{#1}\else\(^{#1}\)\fi}
\caption{Adolescent, Reggio, Baseline Characteristics}
\begin{tabular}{l*{1}{ccc}}
\toprule
                    &\multicolumn{3}{c}{}                           \\
                    &        mu\_1         &        diff&         N\_1\\
\midrule
Male indicator      &       0.571         &       0.119&           7\\
Age                 &      18.671         &      -0.022&           7\\
Low birthweight     &       0.143         &       0.092&           7\\
Premature birth     &       0.143         &       0.114&           7\\
CAPI                &       0.429         &      -0.031&           7\\
Mother: age at birth&      29.957         &      -0.377&           7\\
Mother: born in province&       0.571         &      -0.188&           7\\
Mother max. edu.: less than middle school&       0.286         &       0.154&           7\\
Mother max. edu.: middle school&       0.000         &      -0.095&           7\\
Mother max. edu.: high school&       0.714         &       0.181&           7\\
Mother max. edu.: university&       0.000         &      -0.241&           7\\
Father: age at birth&      33.705         &       0.938&           7\\
Father: born in province&       0.429         &      -0.294&           7\\
Father max. edu.: less than middle school&       0.429         &       0.210&           7\\
Father max. edu.: middle school&       0.000         &      -0.088&           7\\
Father max. edu.: high school&       0.143         &      -0.339&           7\\
Father max. edu.: university&       0.429         &       0.217&           7\\
Number of siblings  &       1.714         &       0.510&           7\\
Religious caregiver indicator&       0.857         &       0.164&           7\\
Home ownership indicator&       0.429\sym{**} &      -0.440&           7\\
Mother: born outside of Italy&       0.286\sym{***}&       0.286&           7\\
Income: 5,000 euros or less&       0.000         &       0.000&           7\\
Income: 5,001-10,000 euros&       0.000         &      -0.015&           7\\
Income: 10,001-25,000 euros&       0.286         &       0.154&           7\\
Income: 25,001-50,000 euros&       0.429         &       0.122&           7\\
Income: 50,001-100,000 euros&       0.000         &      -0.292&           7\\
Income: 100,001-250,000 euros&       0.000         &      -0.044&           7\\
Income: more than 250,000 euros&       0.000         &       0.000&           7\\
\bottomrule
\end{tabular}
\end{table}

\begin{table}[htbp]\centering
\def\sym#1{\ifmmode^{#1}\else\(^{#1}\)\fi}
\caption{Adult30, Padova, Baseline Characteristics}
\begin{tabular}{l*{1}{ccc}}
\toprule
                    &\multicolumn{3}{c}{}                           \\
                    &        mu\_1         &        diff&         N\_1\\
\midrule
Male indicator      &       0.702         &       0.038&          47\\
Age                 &      33.116\sym{***}&       0.360&          47\\
CAPI                &       0.468\sym{*}  &      -0.183&          47\\
Mother: born in province&       0.702\sym{*}  &      -0.144&          47\\
Mother max. edu.: less than middle school&       0.000         &       0.000&          47\\
Mother max. edu.: middle school&       0.106         &       0.066&          47\\
Mother max. edu.: high school&       0.404         &      -0.032&          47\\
Mother max. edu.: university&       0.489         &      -0.034&          47\\
Father: born in province&       0.851         &      -0.035&          47\\
Father max. edu.: less than middle school&       0.000         &       0.000&          47\\
Father max. edu.: middle school&       0.043         &       0.016&          47\\
Father max. edu.: high school&       0.340         &      -0.069&          47\\
Father max. edu.: university&       0.596         &       0.032&          47\\
Religious caregiver indicator&       0.787\sym{***}&       0.378&          47\\
\bottomrule
\end{tabular}
\end{table}

\begin{table}[htbp]\centering
\def\sym#1{\ifmmode^{#1}\else\(^{#1}\)\fi}
\caption{Adult30, Parma, Baseline Characteristics}
\begin{tabular}{l*{1}{ccc}}
\toprule
                    &\multicolumn{3}{c}{}                           \\
                    &        mu\_1         &        diff&         N\_1\\
\midrule
Male indicator      &       0.568         &      -0.096&          44\\
Age                 &      32.837         &       0.081&          44\\
CAPI                &       0.409\sym{**} &      -0.242&          44\\
Mother: born in province&       0.773         &      -0.073&          44\\
Mother max. edu.: less than middle school&       0.000         &       0.000&          44\\
Mother max. edu.: middle school&       0.068         &       0.028&          44\\
Mother max. edu.: high school&       0.455         &       0.018&          44\\
Mother max. edu.: university&       0.477         &      -0.046&          44\\
Father: born in province&       0.818         &      -0.068&          44\\
Father max. edu.: less than middle school&       0.000         &       0.000&          44\\
Father max. edu.: middle school&       0.136\sym{**} &       0.110&          44\\
Father max. edu.: high school&       0.409         &      -0.000&          44\\
Father max. edu.: university&       0.455         &      -0.109&          44\\
Religious caregiver indicator&       0.841\sym{***}&       0.432&          44\\
\bottomrule
\end{tabular}
\end{table}

\begin{table}[htbp]\centering
\def\sym#1{\ifmmode^{#1}\else\(^{#1}\)\fi}
\caption{Adult30, Reggio, Baseline Characteristics}
\begin{tabular}{l*{1}{ccc}}
\toprule
                    &\multicolumn{3}{c}{}                           \\
                    &        mu\_1         &        diff&         N\_1\\
\midrule
Male indicator      &       0.474\sym{*}  &      -0.191&          57\\
Age                 &      32.781         &       0.025&          57\\
CAPI                &       0.474\sym{*}  &      -0.177&          57\\
Mother: born in province&       0.860         &       0.014&          57\\
Mother max. edu.: less than middle school&       0.000         &       0.000&          57\\
Mother max. edu.: middle school&       0.000         &      -0.040&          57\\
Mother max. edu.: high school&       0.246\sym{*}  &      -0.191&          57\\
Mother max. edu.: university&       0.754\sym{**} &       0.231&          57\\
Father: born in province&       0.860         &      -0.026&          57\\
Father max. edu.: less than middle school&       0.000         &       0.000&          57\\
Father max. edu.: middle school&       0.000         &      -0.027&          57\\
Father max. edu.: high school&       0.281         &      -0.129&          57\\
Father max. edu.: university&       0.719\sym{*}  &       0.156&          57\\
Religious caregiver indicator&       0.702\sym{***}&       0.292&          57\\
\bottomrule
\end{tabular}
\end{table}

\begin{table}[htbp]\centering
\def\sym#1{\ifmmode^{#1}\else\(^{#1}\)\fi}
\caption{Adult40, Padova, Baseline Characteristics}
\begin{tabular}{l*{1}{ccc}}
\toprule
                    &\multicolumn{3}{c}{}                           \\
                    &        mu\_1         &        diff&         N\_1\\
\midrule
Male indicator      &       0.533         &      -0.045&          75\\
Age                 &      44.386\sym{***}&       0.727&          75\\
CAPI                &       0.307\sym{***}&      -0.279&          75\\
Mother: born in province&       0.640\sym{***}&      -0.282&          75\\
Mother max. edu.: less than middle school&       0.000         &      -0.016&          75\\
Mother max. edu.: middle school&       0.120\sym{**} &      -0.161&          75\\
Mother max. edu.: high school&       0.320\sym{*}  &      -0.149&          75\\
Mother max. edu.: university&       0.560\sym{***}&       0.333&          75\\
Father: born in province&       0.733\sym{*}  &      -0.134&          75\\
Father max. edu.: less than middle school&       0.000         &      -0.023&          75\\
Father max. edu.: middle school&       0.080\sym{***}&      -0.201&          75\\
Father max. edu.: high school&       0.187\sym{**} &      -0.220&          75\\
Father max. edu.: university&       0.733\sym{***}&       0.452&          75\\
Religious caregiver indicator&       0.680\sym{***}&       0.289&          75\\
\bottomrule
\end{tabular}
\end{table}

\begin{table}[htbp]\centering
\def\sym#1{\ifmmode^{#1}\else\(^{#1}\)\fi}
\caption{Adult40, Parma, Baseline Characteristics}
\begin{tabular}{l*{1}{ccc}}
\toprule
                    &\multicolumn{3}{c}{}                           \\
                    &        mu\_1         &        diff&         N\_1\\
\midrule
Male indicator      &       0.500         &      -0.078&         116\\
Age                 &      43.622         &      -0.037&         116\\
CAPI                &       0.250\sym{***}&      -0.336&         116\\
Mother: born in province&       0.698\sym{***}&      -0.224&         116\\
Mother max. edu.: less than middle school&       0.000         &      -0.016&         116\\
Mother max. edu.: middle school&       0.233         &      -0.048&         116\\
Mother max. edu.: high school&       0.362         &      -0.107&         116\\
Mother max. edu.: university&       0.397\sym{**} &       0.170&         116\\
Father: born in province&       0.879         &       0.012&         116\\
Father max. edu.: less than middle school&       0.009         &      -0.015&         116\\
Father max. edu.: middle school&       0.216         &      -0.066&         116\\
Father max. edu.: high school&       0.397         &      -0.010&         116\\
Father max. edu.: university&       0.371         &       0.089&         116\\
Religious caregiver indicator&       0.793\sym{***}&       0.402&         116\\
\bottomrule
\end{tabular}
\end{table}

\begin{table}[htbp]\centering
\def\sym#1{\ifmmode^{#1}\else\(^{#1}\)\fi}
\caption{Adult40, Reggio, Baseline Characteristics}
\begin{tabular}{l*{1}{ccc}}
\toprule
                    &\multicolumn{3}{c}{}                           \\
                    &        mu\_1         &        diff&         N\_1\\
\midrule
Male indicator      &       0.463         &      -0.116&          80\\
Age                 &      43.751         &       0.092&          80\\
CAPI                &       0.613         &       0.027&          80\\
Mother: born in province&       0.562\sym{***}&      -0.359&          80\\
Mother max. edu.: less than middle school&       0.037         &       0.022&          80\\
Mother max. edu.: middle school&       0.037\sym{***}&      -0.244&          80\\
Mother max. edu.: high school&       0.450         &      -0.019&          80\\
Mother max. edu.: university&       0.450\sym{***}&       0.223&          80\\
Father: born in province&       0.650\sym{***}&      -0.217&          80\\
Father max. edu.: less than middle school&       0.037         &       0.014&          80\\
Father max. edu.: middle school&       0.050\sym{***}&      -0.231&          80\\
Father max. edu.: high school&       0.400         &      -0.006&          80\\
Father max. edu.: university&       0.475\sym{**} &       0.194&          80\\
Religious caregiver indicator&       0.650\sym{***}&       0.259&          80\\
\bottomrule
\end{tabular}
\end{table}

\begin{table}[htbp]\centering
\def\sym#1{\ifmmode^{#1}\else\(^{#1}\)\fi}
\caption{Adult50, Padova, Baseline Characteristics}
\begin{tabular}{l*{1}{ccc}}
\toprule
                    &\multicolumn{3}{c}{}                           \\
                    &        mu\_1         &        diff&         N\_1\\
\midrule
Male indicator      &       0.491         &      -0.175&          57\\
Age                 &      57.065         &       0.254&          57\\
CAPI                &       0.316\sym{**} &      -0.462&          57\\
Mother: born in province&       0.842         &      -0.158&          57\\
Mother max. edu.: less than middle school&       0.035         &       0.035&          57\\
Mother max. edu.: middle school&       0.649         &      -0.129&          57\\
Mother max. edu.: high school&       0.105         &      -0.117&          57\\
Mother max. edu.: university&       0.193         &       0.193&          57\\
Father: born in province&       0.860         &      -0.029&          57\\
Father max. edu.: less than middle school&       0.035         &       0.035&          57\\
Father max. edu.: middle school&       0.491\sym{*}  &      -0.398&          57\\
Father max. edu.: high school&       0.158         &       0.047&          57\\
Father max. edu.: university&       0.298         &       0.298&          57\\
Religious caregiver indicator&       0.754\sym{***}&       0.643&          57\\
\bottomrule
\end{tabular}
\end{table}

\begin{table}[htbp]\centering
\def\sym#1{\ifmmode^{#1}\else\(^{#1}\)\fi}
\caption{Adult50, Parma, Baseline Characteristics}
\begin{tabular}{l*{1}{ccc}}
\toprule
                    &\multicolumn{3}{c}{}                           \\
                    &        mu\_1         &        diff&         N\_1\\
\midrule
Male indicator      &       0.389         &      -0.278&          72\\
Age                 &      56.487         &      -0.324&          72\\
CAPI                &       0.431\sym{*}  &      -0.347&          72\\
Mother: born in province&       0.736         &      -0.264&          72\\
Mother max. edu.: less than middle school&       0.042         &       0.042&          72\\
Mother max. edu.: middle school&       0.486         &      -0.292&          72\\
Mother max. edu.: high school&       0.306         &       0.083&          72\\
Mother max. edu.: university&       0.167         &       0.167&          72\\
Father: born in province&       0.528\sym{*}  &      -0.361&          72\\
Father max. edu.: less than middle school&       0.056         &       0.056&          72\\
Father max. edu.: middle school&       0.500\sym{*}  &      -0.389&          72\\
Father max. edu.: high school&       0.222         &       0.111&          72\\
Father max. edu.: university&       0.208         &       0.208&          72\\
Religious caregiver indicator&       0.708\sym{***}&       0.597&          72\\
\bottomrule
\end{tabular}
\end{table}

\begin{table}[htbp]\centering
\def\sym#1{\ifmmode^{#1}\else\(^{#1}\)\fi}
\caption{Adult50, Reggio, Baseline Characteristics}
\begin{tabular}{l*{1}{ccc}}
\toprule
                    &\multicolumn{3}{c}{}                           \\
                    &        mu\_1         &        diff&         N\_1\\
\midrule
Male indicator      &       0.469         &      -0.197&         147\\
Age                 &      56.533         &      -0.277&         147\\
CAPI                &       0.395\sym{*}  &      -0.383&         147\\
Mother: born in province&       0.728         &      -0.272&         147\\
Mother max. edu.: less than middle school&       0.014         &       0.014&         147\\
Mother max. edu.: middle school&       0.306\sym{**} &      -0.472&         147\\
Mother max. edu.: high school&       0.395         &       0.172&         147\\
Mother max. edu.: university&       0.286         &       0.286&         147\\
Father: born in province&       0.830         &      -0.059&         147\\
Father max. edu.: less than middle school&       0.007         &       0.007&         147\\
Father max. edu.: middle school&       0.245\sym{***}&      -0.644&         147\\
Father max. edu.: high school&       0.415         &       0.304&         147\\
Father max. edu.: university&       0.333\sym{*}  &       0.333&         147\\
Religious caregiver indicator&       0.714\sym{***}&       0.603&         147\\
\bottomrule
\end{tabular}
\end{table}











\subsection{Description of Outcomes}

\section{Constructing a Comparison Group} %LPM
\subsection{Selection into Different Preschool Types}


\end{document}