\documentclass[11pt]{article}

\usepackage{amsmath}
\usepackage{amssymb}
\usepackage{booktabs}
\usepackage{datetime}
\usepackage{float}
\usepackage[margin=0.5in,includefoot]{geometry}
\usepackage{graphics}
\usepackage{hyperref}
\usepackage{morefloats}

\settimeformat{hhmmsstime}

\begin{document}

\title{Disaggregated Description of Baseline Variables}
\author{Reggio Team}
\date{Original version: Thursday  9$^{\text{th}}$ June, 2016 \\ Current version: \today \\ \vspace{1em} Time: \currenttime}
\maketitle

\listoftables

\section*{Overview}

This document presents tables that describe the baseline characteristics of each preschool group (municipal, state, religious, private, and no preschool). These tables are presented by cohort and city. Numbers that are bolded are significantly different from the municipal group in Reggio for that cohort at the 10\% level for a two-sided test. 

%Children
\begin{table}[htbp]
\begin{center}
	\caption{Baseline: Materna, Reggio, Children}
	\begin{tabular}{l c c c c c c }
\toprule
& \textbf{Municipal} & \textbf{State} & \textbf{Religious} & \textbf{Private} & \textbf{None} \\
\midrule
Male indicator &      0.55 &      0.53 &      0.53 &      0.40 &      0.50 \\
\midrule
Observations &       166 &        45 &        92 &         5 &         2
Age &      6.80 & \textbf{     6.92} &      6.74 &      6.69 &      6.93 \\
\midrule
Observations &       166 &        45 &        92 &         5 &         2
Low birthweight &      0.10 &      0.07 &      0.05 & \textbf{     0.00} & \textbf{     0.00} \\
\midrule
Observations &       166 &        45 &        92 &         5 &         2
Premature birth &      0.10 &      0.09 &      0.10 & \textbf{     0.00} & \textbf{     0.00} \\
\midrule
Observations &       166 &        45 &        92 &         5 &         2
CAPI &      0.60 & \textbf{     0.40} &      0.55 &      0.40 &      0.50 \\
\midrule
Observations &       166 &        45 &        92 &         5 &         2
Mother: age at birth &     33.13 &     32.97 &     32.47 &     37.28 &     34.65 \\
\midrule
Observations &       165 &        44 &        92 &         5 &         2
Mother: born in province &      0.51 &      0.38 &      0.59 &      0.40 & \textbf{     1.00} \\
\midrule
Observations &       166 &        45 &        92 &         5 &         2
Mother max. edu.: less than middle school &      0.15 & \textbf{     0.29} &      0.14 & \textbf{     0.00} & \textbf{     1.00} \\
\midrule
Observations &       166 &        45 &        92 &         5 &         2
Mother max. edu.: middle school &      0.08 & \textbf{     0.02} &      0.11 & \textbf{     0.00} & \textbf{     0.00} \\
\midrule
Observations &       166 &        45 &        92 &         5 &         2
Mother max. edu.: high school &      0.46 &      0.44 &      0.43 &      0.60 & \textbf{     0.00} \\
\midrule
Observations &       166 &        45 &        92 &         5 &         2
Mother max. edu.: university &      0.29 & \textbf{     0.18} &      0.32 &      0.40 & \textbf{     0.00} \\
\midrule
Observations &       166 &        45 &        92 &         5 &         2
Father: age at birth &     35.48 &     36.75 &     35.55 &     35.95 &     36.54 \\
\midrule
Observations &       148 &        41 &        89 &         4 &         2
Father: born in province &      0.51 &      0.44 &      0.57 &      0.40 & \textbf{     1.00} \\
\midrule
Observations &       166 &        45 &        92 &         5 &         2
Father max. edu.: less than middle school &      0.22 &      0.29 &      0.24 & \textbf{     0.00} & \textbf{     0.00} \\
\midrule
Observations &       166 &        45 &        92 &         5 &         2
Father max. edu.: middle school &      0.08 &      0.11 &      0.05 & \textbf{     0.00} &      0.50 \\
\midrule
Observations &       166 &        45 &        92 &         5 &         2
Father max. edu.: high school &      0.34 &      0.31 &      0.38 &      0.40 &      0.50 \\
\midrule
Observations &       166 &        45 &        92 &         5 &         2
Father max. edu.: university &      0.23 &      0.18 &      0.29 &      0.40 & \textbf{     0.00} \\
\midrule
Observations &       166 &        45 &        92 &         5 &         2
Number of siblings &      1.01 &      1.02 &      1.04 &      0.60 &      0.50 \\
\midrule
Observations &       166 &        45 &        92 &         5 &         2
Religious caregiver indicator &      0.81 &      0.89 & \textbf{     0.90} &      0.80 &      0.50 \\
\midrule
Observations &       166 &        45 &        92 &         5 &         2
Mother: born outside of Italy &      0.08 &      0.16 & \textbf{     0.02} & \textbf{     0.00} & \textbf{     0.00} \\
\midrule
Observations &       166 &        45 &        92 &         5 &         2
Income: 5,000 euros or less &      0.00 &      0.02 & \textbf{     0.03} &      0.00 &      0.00 \\
\midrule
Observations &       166 &        45 &        92 &         5 &         2
Income: 5,001-10,000 euros &      0.01 &      0.00 &      0.02 &      0.00 &      0.00 \\
\midrule
Observations &       166 &        45 &        92 &         5 &         2
Income: 10,001-25,000 euros &      0.17 &      0.22 &      0.16 & \textbf{     0.00} & \textbf{     0.00} \\
\midrule
Observations &       166 &        45 &        92 &         5 &         2
Income: 25,001-50,000 euros &      0.34 &      0.29 &      0.30 &      0.40 & \textbf{     0.00} \\
\midrule
Observations &       166 &        45 &        92 &         5 &         2
Income: 50,001-100,000 euros &      0.19 & \textbf{     0.07} &      0.27 &      0.20 & \textbf{     0.00} \\
\midrule
Observations &       166 &        45 &        92 &         5 &         2
Income: 100,001-250,000 euros &      0.04 & \textbf{     0.00} &      0.01 & \textbf{     0.00} & \textbf{     0.00} \\
\midrule
Observations &       166 &        45 &        92 &         5 &         2
Income: more than 250,000 euros &      0.00 &      0.00 &      0.00 &      0.00 &      0.00 \\
\midrule
Observations &       166 &        45 &        92 &         5 &         2
\bottomrule
\end{tabular}

\end{center}
\end{table}

\begin{table}[htbp]
\begin{center}
	\caption{Baseline: Materna, Parma, Children}
	\begin{tabular}{l c c c c c c }
\toprule
& \textbf{Municipal} & \textbf{State} & \textbf{Religious} & \textbf{Private} & \textbf{None} \\
\midrule
Male indicator &      0.54 &      0.56 &      0.57 &      0.67 &      0.67 \\
\midrule
Observations &       154 &        43 &        77 &         9 &         6
Age & \textbf{     6.70} &      6.71 & \textbf{     6.70} &      6.83 &      6.80 \\
\midrule
Observations &       154 &        43 &        77 &         9 &         6
Low birthweight & \textbf{     0.04} &      0.14 &      0.08 &      0.11 & \textbf{     0.00} \\
\midrule
Observations &       154 &        43 &        77 &         9 &         6
Premature birth & \textbf{     0.03} &      0.14 &      0.12 &      0.22 & \textbf{     0.00} \\
\midrule
Observations &       154 &        43 &        77 &         9 &         6
CAPI & \textbf{     0.42} & \textbf{     0.37} & \textbf{     0.44} &      0.78 &      0.50 \\
\midrule
Observations &       154 &        43 &        77 &         9 &         6
Mother: age at birth & \textbf{    34.27} &     33.96 &     33.21 &     33.11 & \textbf{    35.30} \\
\midrule
Observations &       153 &        43 &        76 &         9 &         6
Mother: born in province & \textbf{     0.60} & \textbf{     0.72} &      0.56 &      0.67 &      0.50 \\
\midrule
Observations &       154 &        43 &        77 &         9 &         6
Mother max. edu.: less than middle school & \textbf{     0.08} & \textbf{     0.00} &      0.09 & \textbf{     0.00} & \textbf{     0.00} \\
\midrule
Observations &       154 &        43 &        77 &         9 &         6
Mother max. edu.: middle school &      0.06 & \textbf{     0.02} &      0.08 & \textbf{     0.00} & \textbf{     0.00} \\
\midrule
Observations &       154 &        43 &        77 &         9 &         6
Mother max. edu.: high school &      0.38 &      0.44 &      0.44 &      0.56 &      0.17 \\
\midrule
Observations &       154 &        43 &        77 &         9 &         6
Mother max. edu.: university & \textbf{     0.46} & \textbf{     0.53} &      0.39 &      0.44 & \textbf{     0.83} \\
\midrule
Observations &       154 &        43 &        77 &         9 &         6
Father: age at birth & \textbf{    36.63} &     35.30 &     35.61 &     35.40 &     38.01 \\
\midrule
Observations &       141 &        42 &        73 &         9 &         6
Father: born in province &      0.58 & \textbf{     0.70} & \textbf{     0.62} &      0.33 & \textbf{     0.17} \\
\midrule
Observations &       154 &        43 &        77 &         9 &         6
Father max. edu.: less than middle school &      0.16 & \textbf{     0.05} & \textbf{     0.08} &      0.11 &      0.33 \\
\midrule
Observations &       154 &        43 &        77 &         9 &         6
Father max. edu.: middle school &      0.08 &      0.12 &      0.13 &      0.11 & \textbf{     0.00} \\
\midrule
Observations &       154 &        43 &        77 &         9 &         6
Father max. edu.: high school &      0.34 &      0.37 &      0.39 &      0.44 &      0.33 \\
\midrule
Observations &       154 &        43 &        77 &         9 &         6
Father max. edu.: university & \textbf{     0.33} & \textbf{     0.44} & \textbf{     0.35} &      0.33 &      0.33 \\
\midrule
Observations &       154 &        43 &        77 &         9 &         6
Number of siblings &      1.11 &      1.02 & \textbf{     0.77} &      1.33 &      0.83 \\
\midrule
Observations &       154 &        43 &        77 &         9 &         6
Religious caregiver indicator &      0.86 & \textbf{     0.91} &      0.86 & \textbf{     1.00} &      0.83 \\
\midrule
Observations &       154 &        43 &        77 &         9 &         6
Mother: born outside of Italy & \textbf{     0.03} & \textbf{     0.02} & \textbf{     0.00} & \textbf{     0.00} & \textbf{     0.00} \\
\midrule
Observations &       154 &        43 &        77 &         9 &         6
Income: 5,000 euros or less & \textbf{     0.02} &      0.02 &      0.03 &      0.11 &      0.00 \\
\midrule
Observations &       154 &        43 &        77 &         9 &         6
Income: 5,001-10,000 euros &      0.01 &      0.00 &      0.04 &      0.00 &      0.00 \\
\midrule
Observations &       154 &        43 &        77 &         9 &         6
Income: 10,001-25,000 euros &      0.21 &      0.21 &      0.13 & \textbf{     0.00} &      0.33 \\
\midrule
Observations &       154 &        43 &        77 &         9 &         6
Income: 25,001-50,000 euros & \textbf{     0.45} &      0.37 &      0.31 &      0.44 &      0.67 \\
\midrule
Observations &       154 &        43 &        77 &         9 &         6
Income: 50,001-100,000 euros &      0.17 &      0.26 &      0.23 &      0.11 & \textbf{     0.00} \\
\midrule
Observations &       154 &        43 &        77 &         9 &         6
Income: 100,001-250,000 euros &      0.02 &      0.02 &      0.03 & \textbf{     0.00} & \textbf{     0.00} \\
\midrule
Observations &       154 &        43 &        77 &         9 &         6
Income: more than 250,000 euros &      0.00 &      0.00 &      0.00 &      0.00 &      0.00 \\
\midrule
Observations &       154 &        43 &        77 &         9 &         6
\bottomrule
\end{tabular}

\end{center}
\end{table}

\begin{table}[htbp]
\begin{center}
	\caption{Baseline: Materna, Padova, Children}
	\begin{tabular}{l c c c c c c }
\toprule
& \textbf{Municipal} & \textbf{State} & \textbf{Religious} & \textbf{Private} & \textbf{None} \\
\midrule
Male indicator &      0.59 &      0.63 &      0.48 & \textbf{     0.25} &      0.50 \\
\midrule
Observations &        82 &        40 &       141 &        12 &         2
Age & \textbf{     6.66} & \textbf{     6.66} & \textbf{     6.67} &      6.74 & \textbf{     6.39} \\
\midrule
Observations &        82 &        40 &       141 &        12 &         2
Low birthweight &      0.07 &      0.05 & \textbf{     0.03} &      0.08 & \textbf{     0.00} \\
\midrule
Observations &        82 &        40 &       141 &        12 &         2
Premature birth &      0.06 &      0.07 &      0.08 & \textbf{     0.00} & \textbf{     0.00} \\
\midrule
Observations &        82 &        40 &       141 &        12 &         2
CAPI & \textbf{     0.45} &      0.55 & \textbf{     0.48} &      0.42 & \textbf{     0.00} \\
\midrule
Observations &        82 &        40 &       141 &        12 &         2
Mother: age at birth & \textbf{    34.60} & \textbf{    34.50} & \textbf{    34.05} & \textbf{    36.41} &     34.92 \\
\midrule
Observations &        82 &        39 &       141 &        12 &         2
Mother: born in province & \textbf{     0.62} & \textbf{     0.70} & \textbf{     0.74} &      0.58 & \textbf{     0.00} \\
\midrule
Observations &        82 &        40 &       141 &        12 &         2
Mother max. edu.: less than middle school & \textbf{     0.07} &      0.17 &      0.09 & \textbf{     0.00} &      0.50 \\
\midrule
Observations &        82 &        40 &       141 &        12 &         2
Mother max. edu.: middle school &      0.06 &      0.05 &      0.12 &      0.08 & \textbf{     0.00} \\
\midrule
Observations &        82 &        40 &       141 &        12 &         2
Mother max. edu.: high school &      0.43 &      0.40 &      0.47 &      0.58 & \textbf{     0.00} \\
\midrule
Observations &        82 &        40 &       141 &        12 &         2
Mother max. edu.: university & \textbf{     0.44} &      0.35 &      0.32 &      0.33 &      0.50 \\
\midrule
Observations &        82 &        40 &       141 &        12 &         2
Father: age at birth & \textbf{    38.08} & \textbf{    37.43} & \textbf{    37.20} & \textbf{    39.22} &     40.28 \\
\midrule
Observations &        73 &        34 &       130 &        11 &         2
Father: born in province &      0.61 &      0.50 & \textbf{     0.70} &      0.67 &      0.50 \\
\midrule
Observations &        82 &        40 &       141 &        12 &         2
Father max. edu.: less than middle school & \textbf{     0.11} & \textbf{     0.10} & \textbf{     0.09} & \textbf{     0.00} &      0.50 \\
\midrule
Observations &        82 &        40 &       141 &        12 &         2
Father max. edu.: middle school &      0.09 &      0.05 &      0.11 & \textbf{     0.00} & \textbf{     0.00} \\
\midrule
Observations &        82 &        40 &       141 &        12 &         2
Father max. edu.: high school &      0.40 &      0.40 &      0.43 &      0.50 & \textbf{     0.00} \\
\midrule
Observations &        82 &        40 &       141 &        12 &         2
Father max. edu.: university &      0.29 &      0.30 &      0.30 &      0.42 &      0.50 \\
\midrule
Observations &        82 &        40 &       141 &        12 &         2
Number of siblings & \textbf{     1.26} &      0.93 & \textbf{     0.82} &      0.75 & \textbf{     2.00} \\
\midrule
Observations &        82 &        40 &       141 &        12 &         2
Religious caregiver indicator &      0.76 &      0.75 &      0.83 &      0.92 & \textbf{     1.00} \\
\midrule
Observations &        82 &        40 &       141 &        12 &         2
Mother: born outside of Italy & \textbf{     0.02} & \textbf{     0.03} & \textbf{     0.01} &      0.08 & \textbf{     0.00} \\
\midrule
Observations &        82 &        40 &       141 &        12 &         2
Income: 5,000 euros or less &      0.01 &      0.03 & \textbf{     0.05} &      0.00 &      0.00 \\
\midrule
Observations &        82 &        40 &       141 &        12 &         2
Income: 5,001-10,000 euros &      0.05 &      0.00 &      0.00 &      0.00 &      0.00 \\
\midrule
Observations &        82 &        40 &       141 &        12 &         2
Income: 10,001-25,000 euros &      0.13 &      0.25 &      0.16 & \textbf{     0.00} & \textbf{     0.00} \\
\midrule
Observations &        82 &        40 &       141 &        12 &         2
Income: 25,001-50,000 euros &      0.34 &      0.25 &      0.33 &      0.17 & \textbf{     0.00} \\
\midrule
Observations &        82 &        40 &       141 &        12 &         2
Income: 50,001-100,000 euros &      0.17 &      0.10 &      0.13 & \textbf{     0.00} & \textbf{     0.00} \\
\midrule
Observations &        82 &        40 &       141 &        12 &         2
Income: 100,001-250,000 euros &      0.01 &      0.03 &      0.03 &      0.08 &      0.50 \\
\midrule
Observations &        82 &        40 &       141 &        12 &         2
Income: more than 250,000 euros &      0.00 &      0.00 &      0.00 &      0.00 &      0.00 \\
\midrule
Observations &        82 &        40 &       141 &        12 &         2
\bottomrule
\end{tabular}

\end{center}
\end{table}

% Migrants

\begin{table}[htbp]
\begin{center}
	\caption{Baseline: Materna, Reggio, Migrants}
	\begin{tabular}{l c c c c c c c c c c c c c c c c c c}
\toprule
& \multicolumn{3}{c}{Municipal} & \multicolumn{3}{c}{State} & \multicolumn{3}{c}{Religious} & \multicolumn{3}{c}{Private} & \multicolumn{3}{c}{None} & Unconditional R^2 & Conditional R^2 & Conditional N\\
& \scriptsize Mean & \scriptsize C. Mean & \scriptsize N & \scriptsize Mean & \scriptsize C. Mean & \scriptsize N & \scriptsize Mean & \scriptsize C. Mean & \scriptsize C. Mean & \scriptsize C. Mean & \scriptsize N & \scriptsize Mean & \scriptsize C. Mean & \scriptsize N & \scriptsize Mean & & \scriptsize C. Mean \scriptsize N & & & \\
\midrule
Male indicator &         . &         . & &         . &         . & &         . &         . & &         . &         . & &         . &         . & &      0.00 &      0.00 &      3974 \\
Age &         . &         . & &         . &         . & &         . &         . & &         . &         . & &         . &         . & &      0.26 &      0.26 &      3974 \\
Low birthweight &         . &         . & & \textbf{        .} &         . & &         . &         . & &         . &         . & & \textbf{        .} &         . & &      0.01 &      0.01 &      3974 \\
Premature birth &         . &         . & & \textbf{        .} &         . & &         . &         . & &         . &         . & & \textbf{        .} &         . & &      0.01 &      0.01 &      3974 \\
CAPI &         . &         . & & \textbf{        .} &         . & &         . &         . & &         . &         . & &         . &         . & &      0.01 &      0.01 &      3974 \\
Mother: age at birth &         . &         . & &         . &         . & &         . &         . & &         . &         . & &         . &         . & &      0.18 &      0.18 &      3947 \\
Mother: born in province &         . &         . & &         . &         . & &         . &         . & &         . &         . & &         . &         . & &      0.01 &      0.01 &      3974 \\
Mother max. edu.: less than middle school &         . &         . & &         . &         . & &         . &         . & &         . &         . & &         . &         . & &      0.02 &      0.02 &      3974 \\
Mother max. edu.: middle school &         . &         . & &         . &         . & &         . &         . & &         . &         . & &         . &         . & &      0.01 &      0.01 &      3974 \\
Mother max. edu.: high school &         . &         . & &         . &         . & &         . &         . & &         . &         . & &         . &         . & &      0.00 &      0.00 &      3974 \\
Mother max. edu.: university &         . &         . & &         . &         . & &         . &         . & &         . &         . & & \textbf{        .} &         . & &      0.00 &      0.00 &      3974 \\
Father: age at birth &         . &         . & &         . &         . & &         . &         . & &         . &         . & & \textbf{        .} &         . & &      0.17 &      0.17 &      3764 \\
Father: born in province &         . &         . & &         . &         . & &         . &         . & &         . &         . & &         . &         . & &      0.02 &      0.02 &      3974 \\
Father max. edu.: less than middle school &         . &         . & &         . &         . & &         . &         . & &         . &         . & &         . &         . & &      0.01 &      0.01 &      3974 \\
Father max. edu.: middle school &         . &         . & &         . &         . & &         . &         . & &         . &         . & &         . &         . & &      0.01 &      0.01 &      3974 \\
Father max. edu.: high school &         . &         . & &         . &         . & &         . &         . & &         . &         . & &         . &         . & &      0.00 &      0.00 &      3974 \\
Father max. edu.: university &         . &         . & &         . &         . & &         . &         . & &         . &         . & & \textbf{        .} &         . & &      0.01 &      0.01 &      3974 \\
Number of siblings &         . &         . & &         . &         . & & \textbf{        .} &         . & &         . &         . & &         . &         . & &      0.01 &      0.01 &      1980 \\
Religious caregiver indicator &         . &         . & &         . &         . & &         . &         . & &         . &         . & & \textbf{        .} &         . & &      0.01 &      0.01 &      3974 \\
Home ownership indicator &         . &         . & &         . &         . & &         . &         . & &         . &         . & &         . &         . & &      0.10 &      0.10 &      3974 \\
Mother: born outside of Italy &         . &         . & &         . &         . & &         . &         . & &         . &         . & &         . &         . & &      0.06 &      0.06 &      1980 \\
Income: 5,000 euros or less &         . &         . & &         . &         . & &         . &         . & &         . &         . & &         . &         . & &      0.01 &      0.01 &      3974 \\
Income: 5,001-10,000 euros &         . &         . & &         . &         . & &         . &         . & &         . &         . & &         . &         . & &      0.00 &      0.00 &      3974 \\
Income: 10,001-25,000 euros &         . &         . & & \textbf{        .} &         . & &         . &         . & &         . &         . & & \textbf{        .} &         . & &      0.04 &      0.04 &      3974 \\
Income: 25,001-50,000 euros &         . &         . & & \textbf{        .} &         . & &         . &         . & &         . &         . & &         . &         . & &      0.03 &      0.03 &      3974 \\
Income: 50,001-100,000 euros &         . &         . & &         . &         . & &         . &         . & &         . &         . & &         . &         . & &      0.02 &      0.02 &      3974 \\
Income: 100,001-250,000 euros &         . &         . & &         . &         . & &         . &         . & &         . &         . & &         . &         . & &      0.00 &      0.00 &      3974 \\
Income: more than 250,000 euros &         . &         . & &         . &         . & &         . &         . & &         . &         . & &         . &         . & &      0.00 &      0.00 &      3974 \\
Migrants: year entered city &         . &         . & &         . &         . & &         . &         . & &         . &         . & &         . &         . & &      0.22 &      0.22 &      2113 \\
Migrants: age entered city &         . &         . & &         . &         . & &         . &         . & &         . &         . & &         . &         . & &      0.03 &      0.03 &      1814 \\
\bottomrule
\end{tabular}

\end{center}
\end{table}

\begin{table}[htbp]
\begin{center}
	\caption{Baseline: Materna, Parma, Migrants}
	\scalebox{0.9}{\begin{tabular}{l c c c c c c c c c c c c c c c c c c}
\toprule
& \multicolumn{3}{c}{Municipal} & \multicolumn{3}{c}{State} & \multicolumn{3}{c}{Religious} & \multicolumn{3}{c}{Private} & \multicolumn{3}{c}{None} & Unconditional R^2 & Conditional R^2 & Conditional N\\
& \scriptsize Mean & \scriptsize C. Mean & \scriptsize N & \scriptsize Mean & \scriptsize C. Mean & \scriptsize N & \scriptsize Mean & \scriptsize C. Mean & \scriptsize C. Mean & \scriptsize C. Mean & \scriptsize N & \scriptsize Mean & \scriptsize C. Mean & \scriptsize N & \scriptsize Mean & & \scriptsize C. Mean \scriptsize N & & & \\
\midrule
Male indicator &         . &         . & & \textbf{        .} &         . & &         . &         . & &         . &         . & &         . &         . & &      0.00 &      0.00 &      3974 \\
Age &         . &         . & &         . &         . & &         . &         . & &         . &         . & &         . &         . & &      0.26 &      0.26 &      3974 \\
Low birthweight &         . &         . & &         . &         . & & \textbf{        .} &         . & &         . &         . & & \textbf{        .} &         . & &      0.01 &      0.01 &      3974 \\
Premature birth &         . &         . & & \textbf{        .} &         . & & \textbf{        .} &         . & &         . &         . & & \textbf{        .} &         . & &      0.01 &      0.01 &      3974 \\
CAPI &         . &         . & &         . &         . & & \textbf{        .} &         . & &         . &         . & &         . &         . & &      0.01 &      0.01 &      3974 \\
Mother: age at birth &         . &         . & &         . &         . & &         . &         . & &         . &         . & & \textbf{        .} &         . & &      0.18 &      0.18 &      3947 \\
Mother: born in province &         . &         . & &         . &         . & &         . &         . & &         . &         . & &         . &         . & &      0.01 &      0.01 &      3974 \\
Mother max. edu.: less than middle school &         . &         . & & \textbf{        .} &         . & & \textbf{        .} &         . & &         . &         . & &         . &         . & &      0.02 &      0.02 &      3974 \\
Mother max. edu.: middle school &         . &         . & &         . &         . & &         . &         . & &         . &         . & &         . &         . & &      0.01 &      0.01 &      3974 \\
Mother max. edu.: high school &         . &         . & &         . &         . & &         . &         . & & \textbf{        .} &         . & &         . &         . & &      0.00 &      0.00 &      3974 \\
Mother max. edu.: university &         . &         . & & \textbf{        .} &         . & &         . &         . & &         . &         . & & \textbf{        .} &         . & &      0.00 &      0.00 &      3974 \\
Father: age at birth & \textbf{        .} &         . & & \textbf{        .} &         . & &         . &         . & & \textbf{        .} &         . & &         . &         . & &      0.17 &      0.17 &      3764 \\
Father: born in province &         . &         . & &         . &         . & &         . &         . & &         . &         . & &         . &         . & &      0.02 &      0.02 &      3974 \\
Father max. edu.: less than middle school &         . &         . & &         . &         . & & \textbf{        .} &         . & & \textbf{        .} &         . & &         . &         . & &      0.01 &      0.01 &      3974 \\
Father max. edu.: middle school &         . &         . & &         . &         . & &         . &         . & &         . &         . & &         . &         . & &      0.01 &      0.01 &      3974 \\
Father max. edu.: high school &         . &         . & &         . &         . & &         . &         . & &         . &         . & &         . &         . & &      0.00 &      0.00 &      3974 \\
Father max. edu.: university &         . &         . & &         . &         . & & \textbf{        .} &         . & &         . &         . & & \textbf{        .} &         . & &      0.01 &      0.01 &      3974 \\
Number of siblings &         . &         . & &         . &         . & &         . &         . & &         . &         . & &         . &         . & &      0.01 &      0.01 &      1980 \\
Religious caregiver indicator & \textbf{        .} &         . & &         . &         . & & \textbf{        .} &         . & & \textbf{        .} &         . & & \textbf{        .} &         . & &      0.01 &      0.01 &      3974 \\
Home ownership indicator &         . &         . & &         . &         . & &         . &         . & & \textbf{        .} &         . & & \textbf{        .} &         . & &      0.10 &      0.10 &      3974 \\
Mother: born outside of Italy &         . &         . & &         . &         . & &         . &         . & &         . &         . & &         . &         . & &      0.06 &      0.06 &      1980 \\
Income: 5,000 euros or less &         . &         . & &         . &         . & &         . &         . & &         . &         . & &         . &         . & &      0.01 &      0.01 &      3974 \\
Income: 5,001-10,000 euros &         . &         . & &         . &         . & &         . &         . & &         . &         . & & \textbf{        .} &         . & &      0.00 &      0.00 &      3974 \\
Income: 10,001-25,000 euros &         . &         . & &         . &         . & &         . &         . & &         . &         . & &         . &         . & &      0.04 &      0.04 &      3974 \\
Income: 25,001-50,000 euros &         . &         . & & \textbf{        .} &         . & & \textbf{        .} &         . & & \textbf{        .} &         . & & \textbf{        .} &         . & &      0.03 &      0.03 &      3974 \\
Income: 50,001-100,000 euros &         . &         . & &         . &         . & &         . &         . & &         . &         . & &         . &         . & &      0.02 &      0.02 &      3974 \\
Income: 100,001-250,000 euros &         . &         . & &         . &         . & &         . &         . & &         . &         . & &         . &         . & &      0.00 &      0.00 &      3974 \\
Income: more than 250,000 euros &         . &         . & &         . &         . & &         . &         . & &         . &         . & &         . &         . & &      0.00 &      0.00 &      3974 \\
Migrants: year entered city &         . &         . & & \textbf{        .} &         . & &         . &         . & &         . &         . & & \textbf{        .} &         . & &      0.22 &      0.22 &      2113 \\
Migrants: age entered city &         . &         . & &         . &         . & &         . &         . & &         . &         . & & \textbf{        .} &         . & &      0.03 &      0.03 &      1814 \\
\bottomrule
\end{tabular}
}
\end{center}
\end{table}

\begin{table}[htbp]
\begin{center}
	\caption{Baseline: Materna, Padova, Migrants}
	\scalebox{0.9}{\begin{tabular}{l c c c c c c c c c c c c c c c c c c}
\toprule
& \multicolumn{3}{c}{Municipal} & \multicolumn{3}{c}{State} & \multicolumn{3}{c}{Religious} & \multicolumn{3}{c}{Private} & \multicolumn{3}{c}{None} & Unconditional R^2 & Conditional R^2 & Conditional N\\
& \scriptsize Mean & \scriptsize C. Mean & \scriptsize N & \scriptsize Mean & \scriptsize C. Mean & \scriptsize N & \scriptsize Mean & \scriptsize C. Mean & \scriptsize C. Mean & \scriptsize C. Mean & \scriptsize N & \scriptsize Mean & \scriptsize C. Mean & \scriptsize N & \scriptsize Mean & & \scriptsize C. Mean \scriptsize N & & & \\
\midrule
Male indicator &         . &         . & &         . &         . & &         . &         . & &         . &         . & &         . &         . & &      0.00 &      0.00 &      3974 \\
Age &         . &         . & & \textbf{        .} &         . & & \textbf{        .} &         . & &         . &         . & & \textbf{        .} &         . & &      0.26 &      0.26 &      3974 \\
Low birthweight &         . &         . & &         . &         . & &         . &         . & &         . &         . & & \textbf{        .} &         . & &      0.01 &      0.01 &      3974 \\
Premature birth &         . &         . & &         . &         . & & \textbf{        .} &         . & &         . &         . & & \textbf{        .} &         . & &      0.01 &      0.01 &      3974 \\
CAPI & \textbf{        .} &         . & & \textbf{        .} &         . & &         . &         . & &         . &         . & & \textbf{        .} &         . & &      0.01 &      0.01 &      3974 \\
Mother: age at birth &         . &         . & &         . &         . & &         . &         . & &         . &         . & &         . &         . & &      0.18 &      0.18 &      3947 \\
Mother: born in province &         . &         . & &         . &         . & &         . &         . & &         . &         . & &         . &         . & &      0.01 &      0.01 &      3974 \\
Mother max. edu.: less than middle school &         . &         . & &         . &         . & & \textbf{        .} &         . & &         . &         . & &         . &         . & &      0.02 &      0.02 &      3974 \\
Mother max. edu.: middle school &         . &         . & &         . &         . & &         . &         . & &         . &         . & &         . &         . & &      0.01 &      0.01 &      3974 \\
Mother max. edu.: high school &         . &         . & &         . &         . & & \textbf{        .} &         . & &         . &         . & &         . &         . & &      0.00 &      0.00 &      3974 \\
Mother max. edu.: university &         . &         . & &         . &         . & &         . &         . & &         . &         . & & \textbf{        .} &         . & &      0.00 &      0.00 &      3974 \\
Father: age at birth & \textbf{        .} &         . & &         . &         . & & \textbf{        .} &         . & &         . &         . & & \textbf{        .} &         . & &      0.17 &      0.17 &      3764 \\
Father: born in province &         . &         . & &         . &         . & &         . &         . & &         . &         . & &         . &         . & &      0.02 &      0.02 &      3974 \\
Father max. edu.: less than middle school & \textbf{        .} &         . & &         . &         . & &         . &         . & &         . &         . & &         . &         . & &      0.01 &      0.01 &      3974 \\
Father max. edu.: middle school &         . &         . & &         . &         . & &         . &         . & &         . &         . & &         . &         . & &      0.01 &      0.01 &      3974 \\
Father max. edu.: high school &         . &         . & &         . &         . & & \textbf{        .} &         . & &         . &         . & & \textbf{        .} &         . & &      0.00 &      0.00 &      3974 \\
Father max. edu.: university &         . &         . & & \textbf{        .} &         . & &         . &         . & &         . &         . & &         . &         . & &      0.01 &      0.01 &      3974 \\
Number of siblings &         . &         . & &         . &         . & & \textbf{        .} &         . & &         . &         . & &         . &         . & &      0.01 &      0.01 &      1980 \\
Religious caregiver indicator & \textbf{        .} &         . & & \textbf{        .} &         . & & \textbf{        .} &         . & &         . &         . & & \textbf{        .} &         . & &      0.01 &      0.01 &      3974 \\
Home ownership indicator &         . &         . & &         . &         . & & \textbf{        .} &         . & &         . &         . & &         . &         . & &      0.10 &      0.10 &      3974 \\
Mother: born outside of Italy &         . &         . & &         . &         . & &         . &         . & &         . &         . & &         . &         . & &      0.06 &      0.06 &      1980 \\
Income: 5,000 euros or less & \textbf{        .} &         . & & \textbf{        .} &         . & &         . &         . & &         . &         . & &         . &         . & &      0.01 &      0.01 &      3974 \\
Income: 5,001-10,000 euros & \textbf{        .} &         . & & \textbf{        .} &         . & & \textbf{        .} &         . & &         . &         . & &         . &         . & &      0.00 &      0.00 &      3974 \\
Income: 10,001-25,000 euros & \textbf{        .} &         . & & \textbf{        .} &         . & &         . &         . & &         . &         . & & \textbf{        .} &         . & &      0.04 &      0.04 &      3974 \\
Income: 25,001-50,000 euros & \textbf{        .} &         . & & \textbf{        .} &         . & & \textbf{        .} &         . & &         . &         . & & \textbf{        .} &         . & &      0.03 &      0.03 &      3974 \\
Income: 50,001-100,000 euros &         . &         . & &         . &         . & &         . &         . & &         . &         . & &         . &         . & &      0.02 &      0.02 &      3974 \\
Income: 100,001-250,000 euros &         . &         . & &         . &         . & &         . &         . & &         . &         . & &         . &         . & &      0.00 &      0.00 &      3974 \\
Income: more than 250,000 euros &         . &         . & &         . &         . & &         . &         . & &         . &         . & &         . &         . & &      0.00 &      0.00 &      3974 \\
Migrants: year entered city & \textbf{        .} &         . & & \textbf{        .} &         . & & \textbf{        .} &         . & &         . &         . & &         . &         . & &      0.22 &      0.22 &      2113 \\
Migrants: age entered city &         . &         . & & \textbf{        .} &         . & & \textbf{        .} &         . & &         . &         . & &         . &         . & &      0.03 &      0.03 &      1814 \\
\bottomrule
\end{tabular}
}
\end{center}
\end{table}

% Adolesecent
\begin{table}[htbp]
\begin{center}
	\caption{Baseline: Materna, Reggio, Adolescents}
	\begin{tabular}{l c c c c c c }
\toprule
& \textbf{Municipal} & \textbf{State} & \textbf{Religious} & \textbf{Private} & \textbf{None} \\
\midrule
Male indicator &      0.42 &      0.55 &      0.40 &      0.50 &      0.57 \\
\midrule
Observations &       166 &        22 &        96 &         6 &         7
Age &     18.70 &     18.75 &     18.73 &     18.64 &     18.67 \\
\midrule
Observations &       166 &        22 &        96 &         6 &         7
Low birthweight &      0.05 & \textbf{     0.00} &      0.05 &      0.17 &      0.14 \\
\midrule
Observations &       166 &        22 &        96 &         6 &         7
Premature birth &      0.04 &      0.09 &      0.08 &      0.17 &      0.14 \\
\midrule
Observations &       166 &        22 &        96 &         6 &         7
CAPI &      0.47 &      0.41 &      0.38 &      0.33 &      0.43 \\
\midrule
Observations &       166 &        22 &        96 &         6 &         7
Mother: age at birth &     30.26 &     31.53 &     30.15 &     30.62 &     29.96 \\
\midrule
Observations &       161 &        22 &        94 &         6 &         7
Mother: born in province &      0.72 & \textbf{     0.50} &      0.64 &      0.83 &      0.57 \\
\midrule
Observations &       166 &        22 &        96 &         6 &         7
Mother max. edu.: less than middle school &      0.14 &      0.27 &      0.14 &      0.17 &      0.29 \\
\midrule
Observations &       166 &        22 &        96 &         6 &         7
Mother max. edu.: middle school &      0.09 &      0.09 &      0.11 & \textbf{     0.00} & \textbf{     0.00} \\
\midrule
Observations &       166 &        22 &        96 &         6 &         7
Mother max. edu.: high school &      0.51 &      0.36 &      0.42 &      0.67 &      0.71 \\
\midrule
Observations &       166 &        22 &        96 &         6 &         7
Mother max. edu.: university &      0.22 &      0.27 &      0.31 &      0.17 & \textbf{     0.00} \\
\midrule
Observations &       166 &        22 &        96 &         6 &         7
Father: age at birth &     32.83 &     32.92 &     33.50 &     35.73 &     33.70 \\
\midrule
Observations &       141 &        18 &        85 &         5 &         7
Father: born in province &      0.61 &      0.45 &      0.57 &      0.67 &      0.43 \\
\midrule
Observations &       166 &        22 &        96 &         6 &         7
Father max. edu.: less than middle school &      0.19 &      0.23 &      0.18 & \textbf{     0.00} &      0.43 \\
\midrule
Observations &       166 &        22 &        96 &         6 &         7
Father max. edu.: middle school &      0.07 &      0.09 &      0.10 &      0.33 & \textbf{     0.00} \\
\midrule
Observations &       166 &        22 &        96 &         6 &         7
Father max. edu.: high school &      0.40 &      0.41 &      0.40 &      0.50 &      0.14 \\
\midrule
Observations &       166 &        22 &        96 &         6 &         7
Father max. edu.: university &      0.19 &      0.09 &      0.21 & \textbf{     0.00} &      0.43 \\
\midrule
Observations &       166 &        22 &        96 &         6 &         7
Number of siblings &      1.14 &      1.27 &      1.31 &      0.83 &      1.71 \\
\midrule
Observations &       166 &        22 &        96 &         6 &         7
Religious caregiver indicator &      0.69 &      0.73 & \textbf{     0.91} & \textbf{     1.00} &      0.86 \\
\midrule
Observations &       166 &        22 &        96 &         6 &         7
Mother: born outside of Italy &      0.00 &      0.00 &      0.01 &      0.17 &      0.29 \\
\midrule
Observations &       166 &        22 &        96 &         6 &         7
Income: 5,000 euros or less &      0.00 &      0.00 &      0.01 &      0.00 &      0.00 \\
\midrule
Observations &       166 &        22 &        96 &         6 &         7
Income: 5,001-10,000 euros &      0.01 &      0.00 &      0.01 &      0.00 &      0.00 \\
\midrule
Observations &       166 &        22 &        96 &         6 &         7
Income: 10,001-25,000 euros &      0.17 & \textbf{     0.36} &      0.17 & \textbf{     0.00} &      0.29 \\
\midrule
Observations &       166 &        22 &        96 &         6 &         7
Income: 25,001-50,000 euros &      0.32 &      0.27 &      0.34 &      0.17 &      0.43 \\
\midrule
Observations &       166 &        22 &        96 &         6 &         7
Income: 50,001-100,000 euros &      0.25 & \textbf{     0.09} &      0.25 &      0.33 & \textbf{     0.00} \\
\midrule
Observations &       166 &        22 &        96 &         6 &         7
Income: 100,001-250,000 euros &      0.04 & \textbf{     0.00} &      0.06 & \textbf{     0.00} & \textbf{     0.00} \\
\midrule
Observations &       166 &        22 &        96 &         6 &         7
Income: more than 250,000 euros &      0.00 &      0.00 &      0.01 &      0.00 &      0.00 \\
\midrule
Observations &       166 &        22 &        96 &         6 &         7
\bottomrule
\end{tabular}

\end{center}
\end{table}

\begin{table}[htbp]
\begin{center}
	\caption{Baseline: Materna, Parma, Adolescents}
	\begin{tabular}{l c c c c c c }
\toprule
& \textbf{Municipal} & \textbf{State} & \textbf{Religious} & \textbf{Private} & \textbf{None} \\
\midrule
Male indicator &      0.40 &      0.42 &      0.52 &      0.67 &      0.50 \\
\midrule
Observations &       116 &        43 &        82 &         6 &         4
Age &     18.77 &     18.75 & \textbf{    18.80} &     18.81 &     18.59 \\
\midrule
Observations &       116 &        43 &        82 &         6 &         4
Low birthweight &      0.06 &      0.05 &      0.07 & \textbf{     0.00} &      0.25 \\
\midrule
Observations &       116 &        43 &        82 &         6 &         4
Premature birth & \textbf{     0.09} &      0.07 & \textbf{     0.12} & \textbf{     0.00} &      0.25 \\
\midrule
Observations &       116 &        43 &        82 &         6 &         4
CAPI &      0.53 &      0.47 & \textbf{     0.59} & \textbf{     1.00} &      0.50 \\
\midrule
Observations &       116 &        43 &        82 &         6 &         4
Mother: age at birth &     30.84 &     30.05 &     30.95 & \textbf{    27.15} &     26.80 \\
\midrule
Observations &       115 &        41 &        81 &         6 &         4
Mother: born in province &      0.68 &      0.74 &      0.66 &      0.67 &      0.25 \\
\midrule
Observations &       116 &        43 &        82 &         6 &         4
Mother max. edu.: less than middle school &      0.09 &      0.16 &      0.12 & \textbf{     0.00} & \textbf{     0.00} \\
\midrule
Observations &       116 &        43 &        82 &         6 &         4
Mother max. edu.: middle school &      0.09 &      0.16 &      0.10 & \textbf{     0.00} & \textbf{     0.00} \\
\midrule
Observations &       116 &        43 &        82 &         6 &         4
Mother max. edu.: high school &      0.44 &      0.47 &      0.45 & \textbf{     0.00} &      0.75 \\
\midrule
Observations &       116 &        43 &        82 &         6 &         4
Mother max. edu.: university & \textbf{     0.37} &      0.16 & \textbf{     0.33} & \textbf{     1.00} &      0.25 \\
\midrule
Observations &       116 &        43 &        82 &         6 &         4
Father: age at birth &     33.77 &     32.01 & \textbf{    34.34} &     31.43 &     32.42 \\
\midrule
Observations &        98 &        38 &        70 &         4 &         2
Father: born in province &      0.58 &      0.70 &      0.63 &      0.50 & \textbf{     0.00} \\
\midrule
Observations &       116 &        43 &        82 &         6 &         4
Father max. edu.: less than middle school &      0.16 &      0.21 &      0.15 & \textbf{     0.00} & \textbf{     0.00} \\
\midrule
Observations &       116 &        43 &        82 &         6 &         4
Father max. edu.: middle school &      0.09 &      0.12 & \textbf{     0.02} & \textbf{     0.00} & \textbf{     0.00} \\
\midrule
Observations &       116 &        43 &        82 &         6 &         4
Father max. edu.: high school &      0.35 &      0.40 &      0.38 &      0.17 &      0.25 \\
\midrule
Observations &       116 &        43 &        82 &         6 &         4
Father max. edu.: university &      0.24 &      0.16 & \textbf{     0.32} &      0.50 &      0.25 \\
\midrule
Observations &       116 &        43 &        82 &         6 &         4
Number of siblings &      1.07 &      1.21 &      1.01 &      1.17 &      1.00 \\
\midrule
Observations &       116 &        43 &        82 &         6 &         4
Religious caregiver indicator & \textbf{     0.88} & \textbf{     0.86} & \textbf{     0.87} &      0.83 &      0.75 \\
\midrule
Observations &       116 &        43 &        82 &         6 &         4
Mother: born outside of Italy &      0.00 &      0.02 & \textbf{     0.04} &      0.00 &      0.00 \\
\midrule
Observations &       116 &        43 &        82 &         6 &         4
Income: 5,000 euros or less & \textbf{     0.03} &      0.02 &      0.02 &      0.00 &      0.00 \\
\midrule
Observations &       116 &        43 &        82 &         6 &         4
Income: 5,001-10,000 euros &      0.01 &      0.00 &      0.00 &      0.00 &      0.00 \\
\midrule
Observations &       116 &        43 &        82 &         6 &         4
Income: 10,001-25,000 euros &      0.16 &      0.28 &      0.12 &      0.50 &      0.25 \\
\midrule
Observations &       116 &        43 &        82 &         6 &         4
Income: 25,001-50,000 euros &      0.25 &      0.37 &      0.30 &      0.50 &      0.25 \\
\midrule
Observations &       116 &        43 &        82 &         6 &         4
Income: 50,001-100,000 euros &      0.26 &      0.23 &      0.26 & \textbf{     0.00} & \textbf{     0.00} \\
\midrule
Observations &       116 &        43 &        82 &         6 &         4
Income: 100,001-250,000 euros &      0.03 & \textbf{     0.00} &      0.05 & \textbf{     0.00} & \textbf{     0.00} \\
\midrule
Observations &       116 &        43 &        82 &         6 &         4
Income: more than 250,000 euros &      0.00 &      0.00 &      0.00 &      0.00 &      0.00 \\
\midrule
Observations &       116 &        43 &        82 &         6 &         4
\bottomrule
\end{tabular}

\end{center}
\end{table}

\begin{table}[htbp]
\begin{center}
	\caption{Baseline: Materna, Padova, Adolescents}
	\begin{tabular}{l c c c c c c }
\toprule
& \textbf{Municipal} & \textbf{State} & \textbf{Religious} & \textbf{Private} & \textbf{None} \\
\midrule
Male indicator &      0.44 &      0.45 &      0.50 &      0.50 &      1.00 \\
\midrule
Observations &        93 &        47 &       131 &         6 &         1
Age &     18.75 & \textbf{    18.82} &     18.64 &     18.74 &     18.56 \\
\midrule
Observations &        93 &        47 &       131 &         6 &         1
Low birthweight &      0.06 &      0.06 &      0.02 & \textbf{     0.00} &      1.00 \\
\midrule
Observations &        93 &        47 &       131 &         6 &         1
Premature birth &      0.10 &      0.09 &      0.04 & \textbf{     0.00} &      1.00 \\
\midrule
Observations &        93 &        47 &       131 &         6 &         1
CAPI &      0.43 &      0.55 &      0.53 &      0.33 &      0.00 \\
\midrule
Observations &        93 &        47 &       131 &         6 &         1
Mother: age at birth & \textbf{    32.02} & \textbf{    31.89} & \textbf{    32.06} &     32.37 &     33.03 \\
\midrule
Observations &        89 &        46 &       128 &         5 &         1
Mother: born in province &      0.71 &      0.77 & \textbf{     0.82} & \textbf{     1.00} &      1.00 \\
\midrule
Observations &        93 &        47 &       131 &         6 &         1
Mother max. edu.: less than middle school &      0.11 &      0.11 &      0.18 & \textbf{     0.00} &      0.00 \\
\midrule
Observations &        93 &        47 &       131 &         6 &         1
Mother max. edu.: middle school &      0.15 &      0.11 &      0.08 &      0.17 &      0.00 \\
\midrule
Observations &        93 &        47 &       131 &         6 &         1
Mother max. edu.: high school & \textbf{     0.39} &      0.43 &      0.45 &      0.50 &      1.00 \\
\midrule
Observations &        93 &        47 &       131 &         6 &         1
Mother max. edu.: university &      0.31 &      0.34 &      0.27 &      0.33 &      0.00 \\
\midrule
Observations &        93 &        47 &       131 &         6 &         1
Father: age at birth & \textbf{    34.98} &     34.05 & \textbf{    34.90} &     33.77 &     35.41 \\
\midrule
Observations &        81 &        43 &       121 &         4 &         1
Father: born in province &      0.65 &      0.72 & \textbf{     0.79} &      0.67 &      1.00 \\
\midrule
Observations &        93 &        47 &       131 &         6 &         1
Father max. edu.: less than middle school &      0.13 & \textbf{     0.09} &      0.17 & \textbf{     0.00} &      0.00 \\
\midrule
Observations &        93 &        47 &       131 &         6 &         1
Father max. edu.: middle school &      0.11 &      0.13 &      0.08 &      0.17 &      0.00 \\
\midrule
Observations &        93 &        47 &       131 &         6 &         1
Father max. edu.: high school &      0.32 &      0.47 &      0.41 &      0.17 &      1.00 \\
\midrule
Observations &        93 &        47 &       131 &         6 &         1
Father max. edu.: university & \textbf{     0.31} &      0.23 & \textbf{     0.27} &      0.50 &      0.00 \\
\midrule
Observations &        93 &        47 &       131 &         6 &         1
Number of siblings &      1.01 & \textbf{     0.72} &      1.05 &      0.83 &      1.00 \\
\midrule
Observations &        93 &        47 &       131 &         6 &         1
Religious caregiver indicator &      0.77 &      0.62 & \textbf{     0.78} &      0.50 &      1.00 \\
\midrule
Observations &        93 &        47 &       131 &         6 &         1
Mother: born outside of Italy &      0.00 &      0.00 &      0.00 &      0.00 &      0.00 \\
\midrule
Observations &        93 &        47 &       131 &         6 &         1
Income: 5,000 euros or less & \textbf{     0.03} &      0.02 & \textbf{     0.05} &      0.00 &      0.00 \\
\midrule
Observations &        93 &        47 &       131 &         6 &         1
Income: 5,001-10,000 euros &      0.00 &      0.02 &      0.01 &      0.00 &      0.00 \\
\midrule
Observations &        93 &        47 &       131 &         6 &         1
Income: 10,001-25,000 euros &      0.12 & \textbf{     0.09} & \textbf{     0.11} & \textbf{     0.00} &      0.00 \\
\midrule
Observations &        93 &        47 &       131 &         6 &         1
Income: 25,001-50,000 euros &      0.30 & \textbf{     0.09} &      0.27 &      0.17 &      0.00 \\
\midrule
Observations &        93 &        47 &       131 &         6 &         1
Income: 50,001-100,000 euros & \textbf{     0.10} & \textbf{     0.13} & \textbf{     0.11} &      0.17 &      0.00 \\
\midrule
Observations &        93 &        47 &       131 &         6 &         1
Income: 100,001-250,000 euros & \textbf{     0.01} & \textbf{     0.00} &      0.05 & \textbf{     0.00} &      0.00 \\
\midrule
Observations &        93 &        47 &       131 &         6 &         1
Income: more than 250,000 euros &      0.00 &      0.00 &      0.00 &      0.00 &      0.00 \\
\midrule
Observations &        93 &        47 &       131 &         6 &         1
\bottomrule
\end{tabular}

\end{center}
\end{table}


% Adult 30s

\begin{table}[htbp]
\begin{center}
	\caption{Baseline: Materna, Reggio, Adults (30s)}
	\begin{tabular}{l c c c c c c c c c c c c c c c c c c}
\toprule
& \multicolumn{3}{c}{Municipal} & \multicolumn{3}{c}{State} & \multicolumn{3}{c}{Religious} & \multicolumn{3}{c}{Private} & \multicolumn{3}{c}{None} & Unconditional R^2 & Conditional R^2 & Conditional N\\
& \scriptsize Mean & \scriptsize C. Mean & \scriptsize N & \scriptsize Mean & \scriptsize C. Mean & \scriptsize N & \scriptsize Mean & \scriptsize C. Mean & \scriptsize C. Mean & \scriptsize C. Mean & \scriptsize N & \scriptsize Mean & \scriptsize C. Mean & \scriptsize N & \scriptsize Mean & & \scriptsize C. Mean \scriptsize N & & & \\
\midrule
Male indicator &         . &         . & & \textbf{        .} &         . & &         . &         . & &         . &         . & & \textbf{        .} &         . & &      0.00 &      0.00 &      3974 \\
Age &         . &         . & &         . &         . & & \textbf{        .} &         . & &         . &         . & &         . &         . & &      0.26 &      0.26 &      3974 \\
CAPI &         . &         . & & \textbf{        .} &         . & & \textbf{        .} &         . & &         . &         . & & \textbf{        .} &         . & &      0.01 &      0.01 &      3974 \\
Mother: born in province &         . &         . & &         . &         . & &         . &         . & &         . &         . & &         . &         . & &      0.01 &      0.01 &      3974 \\
Mother max. edu.: less than middle school &         . &         . & &         . &         . & &         . &         . & &         . &         . & &         . &         . & &      0.02 &      0.02 &      3974 \\
Mother max. edu.: middle school &         . &         . & & \textbf{        .} &         . & &         . &         . & &         . &         . & & \textbf{        .} &         . & &      0.01 &      0.01 &      3974 \\
Mother max. edu.: high school &         . &         . & &         . &         . & &         . &         . & &         . &         . & & \textbf{        .} &         . & &      0.00 &      0.00 &      3974 \\
Mother max. edu.: university &         . &         . & &         . &         . & & \textbf{        .} &         . & &         . &         . & & \textbf{        .} &         . & &      0.00 &      0.00 &      3974 \\
Father: born in province &         . &         . & &         . &         . & &         . &         . & &         . &         . & &         . &         . & &      0.02 &      0.02 &      3974 \\
Father max. edu.: less than middle school &         . &         . & &         . &         . & &         . &         . & &         . &         . & &         . &         . & &      0.01 &      0.01 &      3974 \\
Father max. edu.: middle school &         . &         . & &         . &         . & &         . &         . & &         . &         . & & \textbf{        .} &         . & &      0.01 &      0.01 &      3974 \\
Father max. edu.: high school &         . &         . & &         . &         . & &         . &         . & &         . &         . & & \textbf{        .} &         . & &      0.00 &      0.00 &      3974 \\
Father max. edu.: university &         . &         . & &         . &         . & &         . &         . & &         . &         . & & \textbf{        .} &         . & &      0.01 &      0.01 &      3974 \\
Religious caregiver indicator &         . &         . & &         . &         . & & \textbf{        .} &         . & &         . &         . & & \textbf{        .} &         . & &      0.01 &      0.01 &      3974 \\
\bottomrule
\end{tabular}

\end{center}
\end{table}

\begin{table}[htbp]
\begin{center}
	\caption{Baseline: Materna, Parma, Adults (30s)}
	\begin{tabular}{l c c c c c c }
\toprule
& \textbf{Municipal} & \textbf{State} & \textbf{Religious} & \textbf{Private} & \textbf{None} \\
\midrule
Male indicator & \textbf{     0.54} & \textbf{     0.51} & \textbf{     0.50} &      0.40 &      0.57 \\
\midrule
Observations &        98 &        51 &        50 &         5 &        44
Age &     32.69 &     32.69 &     32.78 & \textbf{    32.43} &     32.84 \\
\midrule
Observations &        98 &        51 &        50 &         5 &        44
CAPI & \textbf{     0.27} & \textbf{     0.41} &      0.64 &      0.40 & \textbf{     0.41} \\
\midrule
Observations &        98 &        51 &        50 &         5 &        44
Mother: born in province & \textbf{     0.62} &      0.75 &      0.76 &      0.60 &      0.77 \\
\midrule
Observations &        98 &        51 &        50 &         5 &        44
Mother max. edu.: less than middle school &      0.00 &      0.00 &      0.00 &      0.00 &      0.00 \\
\midrule
Observations &        98 &        51 &        50 &         5 &        44
Mother max. edu.: middle school &      0.09 &      0.08 &      0.02 & \textbf{     0.00} &      0.07 \\
\midrule
Observations &        98 &        51 &        50 &         5 &        44
Mother max. edu.: high school & \textbf{     0.24} & \textbf{     0.29} & \textbf{     0.24} &      0.60 &      0.45 \\
\midrule
Observations &        98 &        51 &        50 &         5 &        44
Mother max. edu.: university & \textbf{     0.66} &      0.63 & \textbf{     0.74} &      0.40 &      0.48 \\
\midrule
Observations &        98 &        51 &        50 &         5 &        44
Father: born in province & \textbf{     0.74} &      0.88 & \textbf{     0.76} &      0.60 &      0.82 \\
\midrule
Observations &        98 &        51 &        50 &         5 &        44
Father max. edu.: less than middle school &      0.00 &      0.00 &      0.00 &      0.00 &      0.00 \\
\midrule
Observations &        98 &        51 &        50 &         5 &        44
Father max. edu.: middle school & \textbf{     0.08} &      0.06 &      0.06 & \textbf{     0.00} & \textbf{     0.14} \\
\midrule
Observations &        98 &        51 &        50 &         5 &        44
Father max. edu.: high school &      0.34 &      0.33 & \textbf{     0.20} &      0.40 &      0.41 \\
\midrule
Observations &        98 &        51 &        50 &         5 &        44
Father max. edu.: university &      0.58 &      0.61 & \textbf{     0.74} &      0.60 &      0.45 \\
\midrule
Observations &        98 &        51 &        50 &         5 &        44
Religious caregiver indicator & \textbf{     0.74} &      0.51 & \textbf{     0.80} &      0.80 & \textbf{     0.84} \\
\midrule
Observations &        98 &        51 &        50 &         5 &        44
Caregiver is Catholic &      0.00 &      0.00 &      0.00 &      0.00 &      0.00 \\
\midrule
Observations &        98 &        51 &        50 &         5 &        44
Caregiver is Catholic AND more faithful than the average. &      0.00 &      0.00 &      0.00 &      0.00 &      0.00 \\
\midrule
Observations &        98 &        51 &        50 &         5 &        44
\bottomrule
\end{tabular}

\end{center}
\end{table}

\begin{table}[htbp]
\begin{center}
	\caption{Baseline: Materna, Padova, Adults (30s)}
	\begin{tabular}{l c c c c c c }
\toprule
& \textbf{Municipal} & \textbf{State} & \textbf{Religious} & \textbf{Private} & \textbf{None} \\
\midrule
Male indicator &      0.51 & \textbf{     0.38} & \textbf{     0.53} &      0.00 &      0.70 \\
\midrule
Observations &        35 &        26 &       140 &         1 &        47
Age &     32.96 &     32.89 & \textbf{    33.04} &     32.07 & \textbf{    33.12} \\
\midrule
Observations &        35 &        26 &       140 &         1 &        47
CAPI & \textbf{     0.20} & \textbf{     0.31} & \textbf{     0.36} &      0.00 & \textbf{     0.47} \\
\midrule
Observations &        35 &        26 &       140 &         1 &        47
Mother: born in province & \textbf{     0.69} &      0.73 & \textbf{     0.71} &      0.00 & \textbf{     0.70} \\
\midrule
Observations &        35 &        26 &       140 &         1 &        47
Mother max. edu.: less than middle school &      0.00 &      0.00 &      0.00 &      0.00 &      0.00 \\
\midrule
Observations &        35 &        26 &       140 &         1 &        47
Mother max. edu.: middle school &      0.11 &      0.12 & \textbf{     0.09} &      0.00 &      0.11 \\
\midrule
Observations &        35 &        26 &       140 &         1 &        47
Mother max. edu.: high school &      0.37 &      0.35 & \textbf{     0.33} &      0.00 &      0.40 \\
\midrule
Observations &        35 &        26 &       140 &         1 &        47
Mother max. edu.: university &      0.49 &      0.50 &      0.58 &      1.00 &      0.49 \\
\midrule
Observations &        35 &        26 &       140 &         1 &        47
Father: born in province & \textbf{     0.71} &      0.77 & \textbf{     0.74} &      0.00 &      0.85 \\
\midrule
Observations &        35 &        26 &       140 &         1 &        47
Father max. edu.: less than middle school &      0.00 &      0.00 &      0.00 &      0.00 &      0.00 \\
\midrule
Observations &        35 &        26 &       140 &         1 &        47
Father max. edu.: middle school &      0.09 & \textbf{     0.15} & \textbf{     0.11} &      0.00 &      0.04 \\
\midrule
Observations &        35 &        26 &       140 &         1 &        47
Father max. edu.: high school &      0.34 &      0.42 & \textbf{     0.29} &      0.00 &      0.34 \\
\midrule
Observations &        35 &        26 &       140 &         1 &        47
Father max. edu.: university &      0.57 &      0.42 &      0.58 &      1.00 &      0.60 \\
\midrule
Observations &        35 &        26 &       140 &         1 &        47
Religious caregiver indicator & \textbf{     0.57} &      0.54 & \textbf{     0.77} &      1.00 & \textbf{     0.79} \\
\midrule
Observations &        35 &        26 &       140 &         1 &        47
Caregiver is Catholic &      0.00 &      0.00 &      0.00 &      0.00 &      0.00 \\
\midrule
Observations &        35 &        26 &       140 &         1 &        47
Caregiver is Catholic AND more faithful than the average. &      0.00 &      0.00 &      0.00 &      0.00 &      0.00 \\
\midrule
Observations &        35 &        26 &       140 &         1 &        47
\bottomrule
\end{tabular}

\end{center}
\end{table}

% Adults 40s

\begin{table}[htbp]
\begin{center}
	\caption{Baseline: Materna, Reggio, Adults (40s)}
	\begin{tabular}{l c c c c c c }
\toprule
& \textbf{Municipal} & \textbf{State} & \textbf{Religious} & \textbf{Private} & \textbf{None} \\
\midrule
Male indicator &      0.58 &      0.53 &      0.56 &      0.60 &      0.46 \\
\midrule
Observations &       128 &        17 &        52 &         5 &        80
Age &     43.66 &     43.73 &     43.56 &     43.52 &     43.75 \\
\midrule
Observations &       128 &        17 &        52 &         5 &        80
CAPI &      0.59 & \textbf{     0.88} &      0.69 &      0.60 &      0.61 \\
\midrule
Observations &       128 &        17 &        52 &         5 &        80
Mother: born in province &      0.92 &      0.82 &      0.87 &      0.60 & \textbf{     0.56} \\
\midrule
Observations &       128 &        17 &        52 &         5 &        80
Mother max. edu.: less than middle school &      0.02 &      0.00 &      0.00 &      0.00 &      0.04 \\
\midrule
Observations &       128 &        17 &        52 &         5 &        80
Mother max. edu.: middle school &      0.28 &      0.29 & \textbf{     0.13} &      0.40 & \textbf{     0.04} \\
\midrule
Observations &       128 &        17 &        52 &         5 &        80
Mother max. edu.: high school &      0.47 & \textbf{     0.24} & \textbf{     0.62} &      0.40 &      0.45 \\
\midrule
Observations &       128 &        17 &        52 &         5 &        80
Mother max. edu.: university &      0.23 & \textbf{     0.47} &      0.25 &      0.20 & \textbf{     0.45} \\
\midrule
Observations &       128 &        17 &        52 &         5 &        80
Father: born in province &      0.87 &      0.76 &      0.79 &      0.40 & \textbf{     0.65} \\
\midrule
Observations &       128 &        17 &        52 &         5 &        80
Father max. edu.: less than middle school &      0.02 & \textbf{     0.00} & \textbf{     0.00} & \textbf{     0.00} &      0.04 \\
\midrule
Observations &       128 &        17 &        52 &         5 &        80
Father max. edu.: middle school &      0.28 &      0.29 & \textbf{     0.13} &      0.20 & \textbf{     0.05} \\
\midrule
Observations &       128 &        17 &        52 &         5 &        80
Father max. edu.: high school &      0.41 &      0.41 & \textbf{     0.63} &      0.60 &      0.40 \\
\midrule
Observations &       128 &        17 &        52 &         5 &        80
Father max. edu.: university &      0.28 &      0.29 &      0.23 &      0.20 & \textbf{     0.47} \\
\midrule
Observations &       128 &        17 &        52 &         5 &        80
Religious caregiver indicator &      0.39 &      0.47 & \textbf{     0.54} &      0.60 & \textbf{     0.65} \\
\midrule
Observations &       128 &        17 &        52 &         5 &        80
Caregiver is Catholic &      0.00 &      0.00 &      0.00 &      0.00 &      0.00 \\
\midrule
Observations &       128 &        17 &        52 &         5 &        80
Caregiver is Catholic AND more faithful than the average. &      0.00 &      0.00 &      0.00 &      0.00 &      0.00 \\
\midrule
Observations &       128 &        17 &        52 &         5 &        80
\bottomrule
\end{tabular}

\end{center}
\end{table}

\begin{table}[htbp]
\begin{center}
	\caption{Baseline: Materna, Parma, Adults (40s)}
	\begin{tabular}{l c c c c c c c c c c c c c c c c c c}
\toprule
& \multicolumn{3}{c}{Municipal} & \multicolumn{3}{c}{State} & \multicolumn{3}{c}{Religious} & \multicolumn{3}{c}{Private} & \multicolumn{3}{c}{None} & Unconditional R^2 & Conditional R^2 & Conditional N\\
& \scriptsize Mean & \scriptsize C. Mean & \scriptsize N & \scriptsize Mean & \scriptsize C. Mean & \scriptsize N & \scriptsize Mean & \scriptsize C. Mean & \scriptsize C. Mean & \scriptsize C. Mean & \scriptsize N & \scriptsize Mean & \scriptsize C. Mean & \scriptsize N & \scriptsize Mean & & \scriptsize C. Mean \scriptsize N & & & \\
\midrule
Male indicator &         . &         . & &         . &         . & &         . &         . & &         . &         . & &         . &         . & &      0.00 &      0.00 &      3974 \\
Age &         . &         . & & \textbf{        .} &         . & &         . &         . & &         . &         . & &         . &         . & &      0.26 &      0.26 &      3974 \\
CAPI & \textbf{        .} &         . & & \textbf{        .} &         . & &         . &         . & &         . &         . & & \textbf{        .} &         . & &      0.01 &      0.01 &      3974 \\
Mother: born in province & \textbf{        .} &         . & &         . &         . & & \textbf{        .} &         . & &         . &         . & & \textbf{        .} &         . & &      0.01 &      0.01 &      3974 \\
Mother max. edu.: less than middle school &         . &         . & &         . &         . & &         . &         . & &         . &         . & &         . &         . & &      0.02 &      0.02 &      3974 \\
Mother max. edu.: middle school &         . &         . & & \textbf{        .} &         . & &         . &         . & &         . &         . & &         . &         . & &      0.01 &      0.01 &      3974 \\
Mother max. edu.: high school & \textbf{        .} &         . & &         . &         . & &         . &         . & &         . &         . & & \textbf{        .} &         . & &      0.00 &      0.00 &      3974 \\
Mother max. edu.: university & \textbf{        .} &         . & &         . &         . & & \textbf{        .} &         . & &         . &         . & & \textbf{        .} &         . & &      0.00 &      0.00 &      3974 \\
Father: born in province &         . &         . & &         . &         . & &         . &         . & &         . &         . & &         . &         . & &      0.02 &      0.02 &      3974 \\
Father max. edu.: less than middle school & \textbf{        .} &         . & & \textbf{        .} &         . & & \textbf{        .} &         . & &         . &         . & &         . &         . & &      0.01 &      0.01 &      3974 \\
Father max. edu.: middle school &         . &         . & & \textbf{        .} &         . & & \textbf{        .} &         . & &         . &         . & &         . &         . & &      0.01 &      0.01 &      3974 \\
Father max. edu.: high school & \textbf{        .} &         . & &         . &         . & &         . &         . & &         . &         . & &         . &         . & &      0.00 &      0.00 &      3974 \\
Father max. edu.: university & \textbf{        .} &         . & &         . &         . & & \textbf{        .} &         . & &         . &         . & &         . &         . & &      0.01 &      0.01 &      3974 \\
Religious caregiver indicator & \textbf{        .} &         . & & \textbf{        .} &         . & & \textbf{        .} &         . & &         . &         . & & \textbf{        .} &         . & &      0.01 &      0.01 &      3974 \\
\bottomrule
\end{tabular}

\end{center}
\end{table}

\begin{table}[htbp]
\begin{center}
	\caption{Baseline: Materna, Padova, Adults (40s)}
	\begin{tabular}{l c c c c c c c c c c c c c c c c c c}
\toprule
& \multicolumn{3}{c}{Municipal} & \multicolumn{3}{c}{State} & \multicolumn{3}{c}{Religious} & \multicolumn{3}{c}{Private} & \multicolumn{3}{c}{None} & Unconditional R^2 & Conditional R^2 & Conditional N\\
& \scriptsize Mean & \scriptsize C. Mean & \scriptsize N & \scriptsize Mean & \scriptsize C. Mean & \scriptsize N & \scriptsize Mean & \scriptsize C. Mean & \scriptsize C. Mean & \scriptsize C. Mean & \scriptsize N & \scriptsize Mean & \scriptsize C. Mean & \scriptsize N & \scriptsize Mean & & \scriptsize C. Mean \scriptsize N & & & \\
\midrule
Male indicator & \textbf{        .} &         . & &         . &         . & & \textbf{        .} &         . & &         . &         . & &         . &         . & &      0.00 &      0.00 &      3974 \\
Age & \textbf{        .} &         . & & \textbf{        .} &         . & & \textbf{        .} &         . & &         . &         . & & \textbf{        .} &         . & &      0.26 &      0.26 &      3974 \\
CAPI &         . &         . & & \textbf{        .} &         . & & \textbf{        .} &         . & &         . &         . & & \textbf{        .} &         . & &      0.01 &      0.01 &      3974 \\
Mother: born in province & \textbf{        .} &         . & & \textbf{        .} &         . & & \textbf{        .} &         . & &         . &         . & & \textbf{        .} &         . & &      0.01 &      0.01 &      3974 \\
Mother max. edu.: less than middle school &         . &         . & &         . &         . & &         . &         . & &         . &         . & &         . &         . & &      0.02 &      0.02 &      3974 \\
Mother max. edu.: middle school & \textbf{        .} &         . & & \textbf{        .} &         . & &         . &         . & &         . &         . & & \textbf{        .} &         . & &      0.01 &      0.01 &      3974 \\
Mother max. edu.: high school & \textbf{        .} &         . & &         . &         . & & \textbf{        .} &         . & &         . &         . & & \textbf{        .} &         . & &      0.00 &      0.00 &      3974 \\
Mother max. edu.: university &         . &         . & &         . &         . & & \textbf{        .} &         . & &         . &         . & & \textbf{        .} &         . & &      0.00 &      0.00 &      3974 \\
Father: born in province &         . &         . & &         . &         . & & \textbf{        .} &         . & &         . &         . & & \textbf{        .} &         . & &      0.02 &      0.02 &      3974 \\
Father max. edu.: less than middle school & \textbf{        .} &         . & &         . &         . & & \textbf{        .} &         . & &         . &         . & & \textbf{        .} &         . & &      0.01 &      0.01 &      3974 \\
Father max. edu.: middle school &         . &         . & & \textbf{        .} &         . & & \textbf{        .} &         . & &         . &         . & & \textbf{        .} &         . & &      0.01 &      0.01 &      3974 \\
Father max. edu.: high school &         . &         . & & \textbf{        .} &         . & & \textbf{        .} &         . & &         . &         . & & \textbf{        .} &         . & &      0.00 &      0.00 &      3974 \\
Father max. edu.: university &         . &         . & &         . &         . & & \textbf{        .} &         . & &         . &         . & & \textbf{        .} &         . & &      0.01 &      0.01 &      3974 \\
Religious caregiver indicator & \textbf{        .} &         . & & \textbf{        .} &         . & & \textbf{        .} &         . & &         . &         . & & \textbf{        .} &         . & &      0.01 &      0.01 &      3974 \\
\bottomrule
\end{tabular}

\end{center}
\end{table}

% Adults 50s

\begin{table}[htbp]
\begin{center}
	\caption{Baseline: Materna, Reggio, Adults (50s)}
	\begin{tabular}{l c c c c c c }
\toprule
& \textbf{Municipal} & \textbf{State} & \textbf{Religious} & \textbf{Private} & \textbf{None} \\
\midrule
Male indicator &      0.67 &      0.60 &      0.36 &      0.50 &      0.47 \\
\midrule
Observations &         9 &        10 &        28 &         2 &       147
Age &     56.81 &     56.44 &     56.12 & \textbf{    55.67} &     56.53 \\
\midrule
Observations &         9 &        10 &        28 &         2 &       147
CAPI &      0.78 &      0.50 &      0.54 &      0.50 & \textbf{     0.39} \\
\midrule
Observations &         9 &        10 &        28 &         2 &       147
Mother: born in province &      1.00 &      1.00 & \textbf{     0.86} &      1.00 & \textbf{     0.73} \\
\midrule
Observations &         9 &        10 &        28 &         2 &       147
Mother max. edu.: less than middle school &      0.00 &      0.00 &      0.00 &      0.00 &      0.01 \\
\midrule
Observations &         9 &        10 &        28 &         2 &       147
Mother max. edu.: middle school &      0.78 &      1.00 &      0.68 & \textbf{     0.00} & \textbf{     0.31} \\
\midrule
Observations &         9 &        10 &        28 &         2 &       147
Mother max. edu.: high school &      0.22 &      0.00 &      0.25 & \textbf{     1.00} &      0.39 \\
\midrule
Observations &         9 &        10 &        28 &         2 &       147
Mother max. edu.: university &      0.00 &      0.00 &      0.07 &      0.00 & \textbf{     0.29} \\
\midrule
Observations &         9 &        10 &        28 &         2 &       147
Father: born in province &      0.89 &      0.90 &      0.86 &      1.00 &      0.83 \\
\midrule
Observations &         9 &        10 &        28 &         2 &       147
Father max. edu.: less than middle school &      0.00 &      0.00 &      0.00 &      0.00 &      0.01 \\
\midrule
Observations &         9 &        10 &        28 &         2 &       147
Father max. edu.: middle school &      0.89 &      0.80 & \textbf{     0.61} & \textbf{     0.00} & \textbf{     0.24} \\
\midrule
Observations &         9 &        10 &        28 &         2 &       147
Father max. edu.: high school &      0.11 &      0.10 &      0.25 & \textbf{     1.00} & \textbf{     0.41} \\
\midrule
Observations &         9 &        10 &        28 &         2 &       147
Father max. edu.: university &      0.00 &      0.00 & \textbf{     0.14} &      0.00 & \textbf{     0.33} \\
\midrule
Observations &         9 &        10 &        28 &         2 &       147
Religious caregiver indicator &      0.11 &      0.10 & \textbf{     0.61} &      0.00 & \textbf{     0.71} \\
\midrule
Observations &         9 &        10 &        28 &         2 &       147
Caregiver is Catholic &      0.00 &      0.00 &      0.00 &      0.00 &      0.00 \\
\midrule
Observations &         9 &        10 &        28 &         2 &       147
Caregiver is Catholic AND more faithful than the average. &      0.00 &      0.00 &      0.00 &      0.00 &      0.00 \\
\midrule
Observations &         9 &        10 &        28 &         2 &       147
\bottomrule
\end{tabular}

\end{center}
\end{table}

\begin{table}[htbp]
\begin{center}
	\caption{Baseline: Materna, Parma, Adults (50s)}
	\begin{tabular}{l c c c c c c }
\toprule
& \textbf{Municipal} & \textbf{State} & \textbf{Religious} & \textbf{Private} & \textbf{None} \\
\midrule
Male indicator & \textbf{     0.17} &      0.29 &      0.55 &         . &      0.39 \\
\midrule
Observations &        12 &         7 &        11 & . &        72
Age &     56.08 &     56.67 &     56.49 &         . &     56.49 \\
\midrule
Observations &        12 &         7 &        11 & . &        72
CAPI & \textbf{     0.00} & \textbf{     0.14} & \textbf{     0.27} &         . & \textbf{     0.43} \\
\midrule
Observations &        12 &         7 &        11 & . &        72
Mother: born in province &      0.92 &      1.00 &      0.91 &         . & \textbf{     0.74} \\
\midrule
Observations &        12 &         7 &        11 & . &        72
Mother max. edu.: less than middle school &      0.08 &      0.00 &      0.00 &         . & \textbf{     0.04} \\
\midrule
Observations &        12 &         7 &        11 & . &        72
Mother max. edu.: middle school &      0.83 &      0.86 &      0.55 &         . & \textbf{     0.49} \\
\midrule
Observations &        12 &         7 &        11 & . &        72
Mother max. edu.: high school &      0.00 &      0.14 &      0.27 &         . &      0.31 \\
\midrule
Observations &        12 &         7 &        11 & . &        72
Mother max. edu.: university &      0.08 &      0.00 &      0.18 &         . & \textbf{     0.17} \\
\midrule
Observations &        12 &         7 &        11 & . &        72
Father: born in province &      1.00 &      1.00 &      0.82 &         . & \textbf{     0.53} \\
\midrule
Observations &        12 &         7 &        11 & . &        72
Father max. edu.: less than middle school &      0.00 &      0.00 &      0.00 &         . & \textbf{     0.06} \\
\midrule
Observations &        12 &         7 &        11 & . &        72
Father max. edu.: middle school &      0.83 &      0.86 & \textbf{     0.45} &         . & \textbf{     0.50} \\
\midrule
Observations &        12 &         7 &        11 & . &        72
Father max. edu.: high school &      0.08 &      0.14 &      0.18 &         . &      0.22 \\
\midrule
Observations &        12 &         7 &        11 & . &        72
Father max. edu.: university &      0.08 &      0.00 & \textbf{     0.36} &         . & \textbf{     0.21} \\
\midrule
Observations &        12 &         7 &        11 & . &        72
Religious caregiver indicator & \textbf{     0.67} & \textbf{     0.71} & \textbf{     0.73} &         . & \textbf{     0.71} \\
\midrule
Observations &        12 &         7 &        11 & . &        72
Caregiver is Catholic &      0.00 &      0.00 &      0.00 &         . &      0.00 \\
\midrule
Observations &        12 &         7 &        11 & . &        72
Caregiver is Catholic AND more faithful than the average. &      0.00 &      0.00 &      0.00 &         . &      0.00 \\
\midrule
Observations &        12 &         7 &        11 & . &        72
\bottomrule
\end{tabular}

\end{center}
\end{table}

\begin{table}[htbp]
\begin{center}
	\caption{Baseline: Materna, Padova, Adults (50s)}
	\begin{tabular}{l c c c c c c c c c c c c c c c c c c}
\toprule
& \multicolumn{3}{c}{Municipal} & \multicolumn{3}{c}{State} & \multicolumn{3}{c}{Religious} & \multicolumn{3}{c}{Private} & \multicolumn{3}{c}{None} & Unconditional R^2 & Conditional R^2 & Conditional N\\
& \scriptsize Mean & \scriptsize C. Mean & \scriptsize N & \scriptsize Mean & \scriptsize C. Mean & \scriptsize N & \scriptsize Mean & \scriptsize C. Mean & \scriptsize C. Mean & \scriptsize C. Mean & \scriptsize N & \scriptsize Mean & \scriptsize C. Mean & \scriptsize N & \scriptsize Mean & & \scriptsize C. Mean \scriptsize N & & & \\
\midrule
Male indicator &         . &         . & & \textbf{        .} &         . & &         . &         . & & \textbf{        .} &         . & &         . &         . & &      0.00 &      0.00 &      3974 \\
Age &         . &         . & & \textbf{        .} &         . & &         . &         . & &         . &         . & &         . &         . & &      0.26 &      0.26 &      3974 \\
CAPI & \textbf{        .} &         . & &         . &         . & & \textbf{        .} &         . & & \textbf{        .} &         . & & \textbf{        .} &         . & &      0.01 &      0.01 &      3974 \\
Mother: born in province & \textbf{        .} &         . & &         . &         . & & \textbf{        .} &         . & &         . &         . & & \textbf{        .} &         . & &      0.01 &      0.01 &      3974 \\
Mother max. edu.: less than middle school &         . &         . & &         . &         . & &         . &         . & &         . &         . & &         . &         . & &      0.02 &      0.02 &      3974 \\
Mother max. edu.: middle school &         . &         . & &         . &         . & &         . &         . & &         . &         . & &         . &         . & &      0.01 &      0.01 &      3974 \\
Mother max. edu.: high school &         . &         . & &         . &         . & &         . &         . & &         . &         . & &         . &         . & &      0.00 &      0.00 &      3974 \\
Mother max. edu.: university &         . &         . & &         . &         . & & \textbf{        .} &         . & &         . &         . & & \textbf{        .} &         . & &      0.00 &      0.00 &      3974 \\
Father: born in province &         . &         . & &         . &         . & &         . &         . & &         . &         . & &         . &         . & &      0.02 &      0.02 &      3974 \\
Father max. edu.: less than middle school &         . &         . & &         . &         . & &         . &         . & &         . &         . & &         . &         . & &      0.01 &      0.01 &      3974 \\
Father max. edu.: middle school &         . &         . & &         . &         . & & \textbf{        .} &         . & & \textbf{        .} &         . & & \textbf{        .} &         . & &      0.01 &      0.01 &      3974 \\
Father max. edu.: high school &         . &         . & &         . &         . & &         . &         . & &         . &         . & &         . &         . & &      0.00 &      0.00 &      3974 \\
Father max. edu.: university &         . &         . & &         . &         . & & \textbf{        .} &         . & &         . &         . & & \textbf{        .} &         . & &      0.01 &      0.01 &      3974 \\
Religious caregiver indicator & \textbf{        .} &         . & &         . &         . & & \textbf{        .} &         . & & \textbf{        .} &         . & & \textbf{        .} &         . & &      0.01 &      0.01 &      3974 \\
\bottomrule
\end{tabular}

\end{center}
\end{table}




\end{document}