\documentclass[12pt]{article}
\usepackage[top=1in, bottom=1in, left=1in, right=1in]{geometry}
\parindent 22pt

\newcommand\independent{\protect\mathpalette{\protect\independenT}{\perp}}
\def\independenT#1#2{\mathrel{\rlap{$#1#2$}\mkern2mu{#1#2}}}

\usepackage{adjustbox}
\usepackage{amsmath}
\usepackage{amssymb}
\usepackage{appendix}
\usepackage{array}
\usepackage{authblk}
\usepackage{booktabs}
\usepackage{caption} 
\usepackage{datetime}
\usepackage{enumerate}
\usepackage{fancyhdr}
\usepackage{float}
\usepackage{graphicx}
\usepackage[colorlinks=true,linkcolor=blue,urlcolor=blue,anchorcolor=blue,citecolor=blue]{hyperref}
\usepackage{lscape}
\usepackage{epstopdf}
\usepackage{mathtools}
\usepackage{multirow}
\usepackage{natbib}
\usepackage{pgffor}
\usepackage{setspace}
\usepackage{tabularx}
\usepackage{threeparttable}
\usepackage[colorinlistoftodos,linecolor=black]{todonotes}

\captionsetup[table]{skip = 2pt}

\newcolumntype{L}[1]{>{\raggedright\arraybackslash}p{#1}}
\newcolumntype{C}[1]{>{\centering\arraybackslash}p{#1}}
\newcolumntype{R}[1]{>{\raggedleft\arraybackslash}p{#1}}


\settimeformat{hhmmsstime}

\begin{document}

\title{Reggio Variables}
\author{Reggio Team}
\date{Original version: Friday 19$^{\text{nd}}$ August, 2016 \\ Current version: \today \\ \vspace{1em} Time: \currenttime}
\maketitle

\doublespacing

\section{Overview}

This document presents the main outcome variables and baseline variables available for each age cohort. Note that variables differ across age cohorts, but not within age cohorts. Table \ref{tab:baseline} shows the availability of baseline characteristics. 
\begin{landscape}
\begin{table}[H] \caption{Availability of Outcome Variables} \label{tab:outcome}
\begin{center}
\begin{tabular}{L{4.4cm} L{6cm} L{6cm} L{6cm}}
\toprule
\textbf{Category} & \textbf{Children} & \textbf{Adolescents} & \textbf{Adults} \\ \midrule
\textbf{Cognitive} & Raven's IQ & Raven's IQ & Raven's IQ  \\ \midrule
\textbf{Education} & & High school grade, Graduated from high school  & High school grade, Graduated from high school, University grade, Maximum education \\ \midrule
\textbf{Employment} &  & & Employment status, Occupation, Work hours, Wage, Household income \\ \midrule
\textbf{Living Environment} & & & Marital status, Living with parents, Children  \\ \midrule
\textbf{Noncognitive} & Difficulties (sitting, learning, obeying, eating), SDQ and subscores, Like school, Like subject,    & & \\ \midrule
\textbf{Health} & General Health, Weight and height, Diseases, Sleep \\ \midrule
\textbf{Social} & Bullied, \\ \bottomrule

\end{tabular}
\end{center}
\end{table}
\end{landscape}

\begin{table}[H] \caption{Availability of Baseline Variables} 
\label{tab:baseline}
\begin{center}
\begin{tabular}{lccc}
\toprule
\textbf{Variable} & \textbf{Children} & \textbf{Adolescents} & \textbf{Adults} \\ \midrule
\textbf{Parent Information} \\
\quad Age at Birth & \checkmark & \checkmark \\
\quad Born in Province & \checkmark & \checkmark & \checkmark \\
\quad Maximum Education & \checkmark & \checkmark & \checkmark \\
\quad Marital Status & \checkmark \\ 
\textbf{Caregiver Information} \\
\quad Preschool Experience & \checkmark \\ 
\quad Marital Status & \checkmark \\ 
\quad Religious or Not & \checkmark & \checkmark & \checkmark \\
\quad Specific Religion & \checkmark & \checkmark \\
\quad Degree of Faithfulness & \checkmark & \checkmark \\
\quad Migration Status & \checkmark & \checkmark \\
\textbf{Household Information} \\
\quad Number of Siblings & \checkmark & \checkmark & \checkmark \\
\quad Family Income* & \checkmark & \checkmark \\
\quad House Ownership* & \checkmark & \checkmark \\
\textbf{Child Information} \\
\quad Low Birth Weight & \checkmark & \checkmark \\
\quad Premature at Birth & \checkmark & \checkmark \\
\bottomrule
\end{tabular}
\end{center}
\raggedright \footnotesize Note: This table shows the availability of baseline characteristics for each age cohort. ``Children" column includes both Italian and migrant children cohorts. ``Adult" column includes all age-30, age-40, and age-50 cohorts. * indicates that the variable shows the information during the time of the survey, and hence variables with * might not be appropriate for controlling baseline characteristics. 
\end{table}


\end{document}
