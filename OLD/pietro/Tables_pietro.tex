\documentclass[12pt,english]{article}
\usepackage[latin9]{inputenc}
\usepackage{geometry}
\geometry{verbose,tmargin=3cm,bmargin=3cm,lmargin=2cm,rmargin=2cm}
\usepackage{color}
\usepackage{babel}
\usepackage{array}
\usepackage{verbatim}
\usepackage{rotating}
\usepackage{multirow}
\usepackage{setspace}
\usepackage[authoryear]{natbib}
\onehalfspacing
\usepackage[unicode=true,pdfusetitle,
 bookmarks=true,bookmarksnumbered=false,bookmarksopen=false,
 breaklinks=false,pdfborder={0 0 0},backref=false,colorlinks=false]
 {hyperref}

\usepackage{underscore}
\usepackage{fullpage}
\usepackage{placeins}
\usepackage{rotating}
\usepackage[flushleft]{threeparttable}
\usepackage[T1]{fontenc}
\usepackage{titling}
\usepackage{forloop}

\makeatletter

%%%%%%%%%%%%%%%%%%%%%%%%%%%%%% LyX specific LaTeX commands.
%% Because html converters don't know tabularnewline
\providecommand{\tabularnewline}{\\}

%%%%%%%%%%%%%%%%%%%%%%%%%%%%%% User specified LaTeX commands.
%\usepackage[english]{babel}
%%Language for the Table of Contents etc..

%%% AMS: American Mathematic Society 
%% Math Package (equation, eqnarray, \leq, integrals, parenthesis, etc..)
%% The amssymb package provides various useful mathematical symbols
\usepackage{amsmath}
\usepackage{amsthm}%% The amsthm package provides extended theorem environments
\usepackage{mathrsfs}%\usepackage{breqn}    %% Breaks the equations over multiple lines

%%Tables and figures packages
\usepackage{graphicx}
\graphicspath{{../include/}}% where figures are saved  %%must \usepackage{graphicx}
\usepackage{bigstrut}%creates some space next to lines in table to improve appearance
%\usepackage{longtable}
%\usepackage{multirow}
\usepackage{booktabs}
\usepackage{dcolumn}
%\usepackage{ctable} %%centered tables with footnotes
\usepackage{caption}%%customize captions in floating environments
%\usepackage{subcaption} %%typesetting of sub-captions (by using the sub-caption feature of the caption package).
%\usepackage{subfigure}
%\newcommand{\otoprule}{\midrule[\heavyrulewidth]} %the line below the heading of the tables has always thickness equal to \toprule and \bottomrule but it is vertically centered with respect to the row above and below (which is typical of the lines \midrule)
%\allowdisplaybreaks %%Correcting underfull vboxes and bad page breaks in multiline formulas
%\setcounter{MaxMatrixCols}{10} %By default, LaTeX only allows 10 columns or less in matrices. This limit can be changed

%%Page setting, margins, spacing options
%\usepackage{setspace}%%\begin{spacing}{1.0}
%\onehalfspacing
%\usepackage{geometry}
%\geometry{verbose,tmargin=1.5cm,bmargin=1.5cm}
\usepackage{pdflscape}
%\renewcommand{\baselinestretch}{1.5} %%single-spaced, double-spaced, etc..
%%\usepackage{fullpage} %pre-defined full page margins
%%\usepackage{gmeometric} %supposedly allows changing the margins of only one page
%\usepackage{times} %- this makes the formatting different!!!
%\usepackage{rotating}
%\usepackage{scalefnt}  %%To make text twice the base size (10, 11 or 12 pt) add \scalefont{2}. To change it back, add \scalefont{.5}

% BIBLIOGRAPHY
%Creates click-links for the tableofcontents, figures, references
%\hypersetup{colorlinks=false} %related to hyperref
\usepackage{natbib}
%\usepackage[colorlinks=true,citecolor=blue,linkcolor=black,urlcolor=blue]{hyperref}


\setlength{\footnotesep}{12pt}

\usepackage{booktabs}

%Other Packages
\usepackage{appendix}
\usepackage{comment}%lyx uses this type of comments

%\DeclareMathOperator{\Lagr}{\mathcal{L}} %Lagrangian (needs amsmath)
%\DeclareMathOperator{\Likl}{\mathscr{L}} %Likelihood (needs mathrsfs)

%%% Macro for multiple references to same footnote
%\newcommand{\footnoteremember}[2]{\footnote{#2}\newcounter{#1}\setcounter{#1}{\value{footnote}}}
%\newcommand{\footnoterecall}[1]{\footnotemark[\value{#1}]}
\setlength{\footnotesep}{12pt}

%%%% User defined math symbols
%\newcommand{\Nor}[2]{N\!\left(#1, #2\right)}         % normal distribution
%\newcommand{\indp}{\perp\!\!\!\perp}                     % independence
%\newcommand{\1}[1]{\mathrm{1\hspace*{-2.5pt}l}[#1]} %indicator

%%% Use the title, date, and author inside the text
%\usepackage{titling}%\makeatletter
%\let\currentTitle\@title
%\let\currentAuthor\@author
%\let\currentDate\@date

\makeatother

\title{Reggio Evaluation, Preliminary Regression Tables}
\author{Pietro Biroli}
\date{\today}

\begin{document}
\maketitle

%\pagebreak
In order to estimate the relationship between preschool attendance and later life outcomes, I run the following regressions separately for children, adolescents, and adults:

\begin{align}
y = & \alpha D + \hspace{17ex}  \beta_c City + \beta_a Cohort + \beta_x X + \gamma_t + \varepsilon \\
y = & \alpha D + \beta_s SchoolType + \beta_c City + \beta_a Cohort + \beta_x X + \gamma_t + \varepsilon \\
y = & \alpha D + \hspace{26ex}  \beta_a Cohort + \beta_x X + \gamma_t + \varepsilon \hspace{2ex}  \text{ if City=Reggio}
\end{align}

Where we have the following:
\begin{itemize}
	\item $D$ = treatment: RCH infant-toddler center, RCH preschool (run separately)
	\item $SchoolType$
	\begin{itemize}
		\item Baseline: all other school types pooled together (Non-Reggio municipal, 	State, Religious, Private, None)
		\item Fully interacted: (Municipal, State, Private) X (City); omitted	category: Religious in Reggio
	\end{itemize}
	\item $City$ = dummies for Parma and Reggio
	\item $Cohort$ = dummies for 30-year-olds and 40 year-old cohort (only for adults)
	\item $X$: controls for 
	\begin{itemize}
		\item \textit{Demographics and parental characteristics (for all)}: CAPI; age; age$^2$; gender; dummy for mother at home 0-6; dummies for family size (4, 5 or more); dummies for mother and father education (middle school, high school, university or more); dummies for grandparents living nearby at age 4;
		\item \textit{Caregiver SES (only for children and adolescents)}: dummy for one-parent household; number of other siblings in the house; dummy for owning the house; dummies for caregiver marital status (married, separated or divorced, cohabit); dummies for income categories; dummies for caregiver principal activity and father principal activity: unemployed or looking for first job, out-labor-force, housewife; dummies for caregiver occupation and household-head occupation: teacher, professional, self-employed;
		\item \textit{Own SES (only for adults)}: own home; dummies for own education (middle school, high school, university or more); dummies for marital status (married, separated or divorced, cohabit); dummies for income categories; dummies for unemployed or looking for first job, out-labor-force, housewife, teacher, professional, self-employed;
		\item $\gamma_t$: dummies for interview month; 
	\end{itemize}
\item Omitted categories: city of Reggio, Religious school in Reggio (if $SchoolType$), family size of 3, elementary education or lower, grandparents in same house or as neighbors, single and never-married, working (employee or self-employed or employed in family business or intern), occupation as farmer or worker or employee or atypical worker or "other", 50-year-old (only for adults)
\end{itemize}

\bigskip
\textbf{QUESTIONS}
\begin{itemize}
	\item Combining asilo + materna attendance into ``treatment'' ?
	\item Show just some `raw' correlations, i.e. controlling only for city, time and CAPI dummies?
	\item Interviewer fixed effects?
	\item Other controls that are more outcome specific, like caregiver attitudes towards migration?
	\vspace{3ex}
	\item IV: the ones compiled by Chiara B?
	\item interaction between the distance and the availability?
\end{itemize}

%----- CHILD -----------%
\newcounter{num}
\section{Italian Children}

\forloop{num}{1}{\value{num} < 2} {

\FloatBarrier
\begin{table}[htb!]
%\caption{Bothered by Migrants}
\begin{small}
\input{ItaChild-\arabic{num}.tex}
\end{small}
   \begin{tablenotes}
      \footnotesize
      \item \emph{Notes:} ReggioMaterna equal to 1 if attended Reggio Children preschool; ReggioAsilo equal to 1 if attended Reggio Children infant-toddler center (1) The dependent variables are binary variables (2) Controls added to columns 3-7: gender of the respondent; age and age-squared; time trend (date of the interview) and dummy for second wave of interviews; dummies for: family size; owning home; whether the respondent principal activity was unemployed, out of the labor force, or a housewife; whether the respondent is married, divorced, or cohabits; respondent education at the level of middle school, high school, and university; number of siblings while growing up; mother and father education at the level of middle school, high school, and university; 
    \end{tablenotes}
\end{table}

}

%----- Migrant CHILD -----------%
\pagebreak

\section{Migrant Children}

\forloop{num}{1}{\value{num} < 2} {

\FloatBarrier
\begin{table}[htb!]
%\caption{Bothered by Migrants}
\begin{small}
\input{MigrChild-\arabic{num}.tex}
\end{small}
   \begin{tablenotes}
      \footnotesize
      \item \emph{Notes:} ReggioMaterna equal to 1 if attended Reggio Children preschool; ReggioAsilo equal to 1 if attended Reggio Children infant-toddler center (1) The dependent variables are binary variables (2) Controls added to columns 3-7: gender of the respondent; age and age-squared; time trend (date of the interview) and dummy for second wave of interviews; dummies for: family size; owning home; whether the respondent principal activity was unemployed, out of the labor force, or a housewife; whether the respondent is married, divorced, or cohabits; respondent education at the level of middle school, high school, and university; number of siblings while growing up; mother and father education at the level of middle school, high school, and university; 
    \end{tablenotes}
\end{table}

}

%----- ADO -----------%
\pagebreak

\section{Italian Adolescents}

\forloop{num}{1}{\value{num} < 3} {

\FloatBarrier
\begin{table}[htb!]
%\caption{Bothered by Migrants}
\begin{small}
\input{ItaAdo-\arabic{num}.tex}
\end{small}
   \begin{tablenotes}
      \footnotesize
      \item \emph{Notes:} ReggioMaterna equal to 1 if attended Reggio Children preschool; ReggioAsilo equal to 1 if attended Reggio Children infant-toddler center (1) The dependent variables are binary variables (2) Controls added to columns 3-7: gender of the respondent; age and age-squared; time trend (date of the interview); dummies for: family size; owning home; whether the caregiver principal activity was unemployed, out of the labor force, or a housewife; whether the caregiver is married, divorced, or cohabits; number of siblings while growing up; mother and father education at the level of middle school, high school, and university; foreign language or dialect spoken in the household; 
    \end{tablenotes}
\end{table}

}

%----- ADULTS -----------%
\section{Italian Adults}
\forloop{num}{1}{\value{num} < 2} {

\FloatBarrier
\begin{table}[htb!]
%\caption{Bothered by Migrants}
\begin{small}
\input{ItaAdult-\arabic{num}.tex}
\end{small}
   \begin{tablenotes}
      \footnotesize
      \item \emph{Notes:} Reggio Preschool equal to 1 if attended Reggio Children preschool; Reggio infant-toddler center equal to 1 if attended Reggio Children infant-toddler center (1) The dependent variables are binary variables (2) Controls added to columns 3-7: gender of the respondent; age and age-squared; time trend (date of the interview) and dummy for second wave of interviews; dummies for: family size; owning home; whether the respondent principal activity was unemployed, out of the labor force, or a housewife; whether the respondent is married, divorced, or cohabits; respondent education at the level of middle school, high school, and university; number of siblings while growing up; mother and father education at the level of middle school, high school, and university; 
    \end{tablenotes}
\end{table}

}

\end{document}