\section{Background}
\begin{frame}
\frametitle{Early Childhood Education in Italy}\label{frame:ECE_IT}
%\usebeamerfont{table}
\begin{itemize}
	\item 4 institutional types
	\begin{itemize}
		\item Municipal
		\item State
		\item Religious
		\item Private
	\end{itemize}
	\vspace{1ex}
	\item 2 age groups:
	\begin{itemize}
		\item Age 0-3: 13\% attendance rate (\textit{asilo nido})
		\item Age 3-6: 94\% attendance rate (\textit{scuola materna})
	\end{itemize}
	
	\vspace{3ex}
	
	\item[] Reggio Children Approach: \textbf{municipal} infant-toddler centers and preschools \textbf{(age 0-6)} in \textbf{Reggio Emilia}
\end{itemize} 
\end{frame} 

\begin{frame}
\frametitle{The Reggio Children Approach (RCA)}\label{frame:reggiostyle}
\usebeamerfont{table}
\begin{itemize}
	\item Educational philosophy started by \textbf{Loris Malaguzzi} after World War II.
	\item First preschool 1963; first infant-toddler center 1972 
	\item Famous and replicated all over the world
	\item Salient features:

		\begin{itemize}
		\usebeamerfont{table}
		\item Child-centered philosophy: child guides the learning
		\item Co-teaching
		\item Atelieristas and cooks are also teachers
		\item Family and Community are integrated in learning process
		\item Longer daily hours
		\item Children as a group direct the learning process
		\item Teacher and staff continuosly training
		\end{itemize} 
\end{itemize} 
\end{frame} 

