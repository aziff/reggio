\documentclass{beamer}

\usepackage{amssymb} 
\usepackage{amsmath} % This is the file main.tex \usepackage{multimedia} 
\usepackage{graphicx}
\hypersetup{colorlinks,linkcolor=blue,urlcolor=blue} 
\usepackage{hyperref}
\usepackage{caption}
\usepackage{adjustbox}
\usepackage{epstopdf}
%\usepackage[outdir= /include]{epstopdf}

\usepackage[para, flushleft]{threeparttablex}

%%%%%%%%% NEW COMMAND MACRO %%%%%%%%%%%%%%%%%%%%%%%%%%%%%%

\newcommand{\notorth}{\ensuremath{\perp\!\!\!\!\!\!\diagup\!\!\!\!\!\!\perp}}
\newcommand{\orth}{\ensuremath{\perp\!\!\!\perp}}
\newcommand{\mr}{\multirow}
\newcommand{\mc}{\multicolumn}

%%%%%%%%% Theme %%%%%%%%%%%%%%%%%%%%%%%%%%%%%%%%%%%%%%%%%%%%%%%%%

%\usetheme{Warsaw}
\usetheme{Boadilla}
% or ...

\usecolortheme{orchid}
%\usecolortheme[Blue]{structure}
% Alternatively, you can try Copenhagen; its borders are slightly less dark, but its the same style (rounded edged rectangle on the first %slide)
%\usetheme{Copenhagen}

%% Whatever is on the top of the slide
%\useinnertheme{rounded} %%rectangles, circles, rounded, inmargin, 
%\useoutertheme{split} %%puts sections on the upper-left hand side and subsections in upper-rhs; name and title on 

\setbeamerfont{table}{size=\footnotesize} 
\setbeamerfont{smalltable}{size=\scriptsize} 
\setbeamerfont{microtable}{size=\tiny}
\setbeamertemplate{navigation symbols}{}  %%Get rid of the navigation symbols at the bottom
\setbeamertemplate{footline}[text line]{\parbox{\linewidth}{\hfill  \insertpagenumber / \insertpresentationendpage}} %\insertshortauthor \hspace{2ex} \insertshorttitle
%\setbeamertemplate{footline}[text line]{\parbox{\linewidth}{\vspace*{-8pt}\insertshortauthor  \hfill  \insertpagenumber }}  %\insertpresentationendpage

\usepackage{booktabs}
\usepackage[flushleft]{threeparttable}


%Here is alternative table of contents pointers% 
%\setbeamertemplate{sections/subsections in toc}[square] 
\setbeamertemplate{sections/subsections in toc}[circle] 
%\setbeamertemplate{sections/subsections in toc}[ball] 
%\setbeamertemplate{sections/subsections in toc}[ball unnumbered]
%\setbeamertemplate{sections/subsections in toc}[default]

%The next line is the background. Right now we have it set to a vertical shading, with top white bottom 20% blue. 
%You can alter this however you want. We were using red!20 before. You can use standard names for colors, or if I understand correctly 
%you can also use RGB (red green blue) code, for example 30,30,30) 
%\setbeamertemplate{background canvas}[vertical shading][top=white,bottom=blue!10]

%Alternatively, you can make the background just a solid color. 
%\setbeamercolor{background canvas}{bg=red!5}

%Below is the amount of opaqueness in terms of percent.  You can remove this line or set to 0 and it will make future lines unable to be seen.
\setbeamercovered{transparent=15}

%%%%%%%%%%%%%%%%%%%%%%%%%%%%%%%%%%%%%%%%%%%%%%%%%%%%%%%%%%%%%%%%%%%%%%%%%%%%%%%

\graphicspath{{include/}}
