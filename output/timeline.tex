\def\arraystretch{1.5}
\begin{tabular}{r L{45em}}
\toprule
  1831 & Italy's first private care institutions for abandoned children under 3 years of age were charitable shelters and orphanages provided by the Catholic Church. \\
  1907 & Maria Montessori establishes the \textit{Casa dei bambini} in Rome. \\
  1925 & Charitable organization Opera Nazionale Maternita e Infanzia (ONMI) funds health and childcare initiatives for poor mothers and abandoned children. \\
  1950's & The Catholic Church provides 52.7\% of early childhood programs in Italy \citep{Hohnerlein_2009_Paradox-Public-Preschools}. \\
  1957-64 & The municipalities of Reggio Emilia, Parma, Bologna and Modena experiment with progressive models of early childhood education.  \\
  1963 & The Robinson Crusoe school is built by the municipality of Reggio Emilia, directed by Loris Malaguzzi \citep{Cagliari-etal-eds_2016_BOOK_Loris-Malaguzzi}. \\
  1968 &  \textbf{Law 444} mandates preschools serving children aged 3-6 years with an all-female staff possessing a high school education \citep{OECD_2001_Italy-Country-Note}. \\
  1969 & Malaguzzi directs the municipal school systems in Reggio Emilia and Modena \citep{Cagliari-etal-eds_2016_BOOK_Loris-Malaguzzi}. \\
  1969 & \textit{Orientamenti} outlines non-binding standards and guidelines for preschools, including religious pedagogy \citep{Corsaro_1996_Early-Edu}. \\
  1971 & \textbf{Law 1044} requires regional governments to provide enough infant-toddler centers for children aged 3 months to 3 years to meet local demand, with priority given to working mothers \citep{Saraceno_1984_Soc-Probs}. \\
  1972  & \textbf{Law 1073} provides financial support for non-state schools providing full tuition to children of poor families \citep{Corsaro_1996_Early-Edu}. \\
  1977 & Act 517 legislates inclusion for disabled children \citep{Hohnerlein_2009_Paradox-Public-Preschools}. \\
  1978 & \textbf{Law 463} mandates equivalent pay for preschool and elementary school teachers, enables males to work as early childhood teachers, reduces work hours to 30 hours per week, and requires 50\% of teaching staff to have a diploma from a teacher training school \citep{Hohnerlein_2015_Development-and-Diffusion}. \\
  1986-1987 & The State runs more than 50\% of preschools, as many municipal and some private preschools are transformed for financial reason \citep{Del-Boca-etal_2016_CESifo-ES}. \\
  1987 & Guidelines for working hours and a maximum teacher-child ratio of 1:25 established forpreschools. \\
  1991 & Revised guidelines for preschools emphasize social, affective and cognitive development within the context of the community and family. Play, meals, and collaborative skills are key tasks of early childhood development \citep{Corsaro_1996_Early-Edu}. \\
  1994 & Reggio Emilia first allocates funds to private religious schools to enable inclusion of children with special needs, improve school environments, and train staff. \\
  1997-98 & University degrees and supervised experience are required for teachers in infant-toddler centers and preschools, expanding teacher training from Catholic institutions to secular higher education \citep{Ghedini_2001_Ital-Natl-Policy}.  \\
  2000 & Less than 20\% of preschools are provided by the Catholic Church \citep{Hohnerlein_2009_Paradox-Public-Preschools}.  \\
  2001 & Over 95\% of all Italian children aged 3 to 6 years are enrolled in preschools \citep{OECD_2001_Italy-Country-Note}. \\
  2010 & Only 13\% of Italian infants and toddlers attend infant-toddler centers \citep{Del-Boca-etal_2016_CESifo-ES}. \\
\bottomrule
\end{tabular}
