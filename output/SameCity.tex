\begin{table}[ht!]
\caption{\textbf{Percentage of people living in the same city since birth, by cohort}}
\label{tab:SameCity}
\vspace{-5mm}
\begin{center}
\begin{tabular}{ l c c c c }
\hline\hline
\textbf{Cohort} & \textbf{Reggio (\%)} & \textbf{Parma (\%)} & \textbf{Padova (\%)} & \textbf{Total (\%)}\\
\hline
Italian Children born in 2006 (Cohort V)   & 61.3  & 70.2  & 65.1  & 65.2 \\[0.2em]
Adolescents born in 1994 (Cohort IV)       & 58.1  & 63.0  & 64.4  & 61.9 \\[0.2em]
Adults born in 1980-81 (Cohort III)        & 26.5  & 27.5  & 32.6  & 29.0 \\[0.2em]
Adults born in 1969-70 (Cohort II)         & 27.9  & 31.6  & 31.9  & 30.6 \\[0.2em]
Adults born in 1954-59 (Cohort I)          & 28.8  & 27.9  & 31.4  & 29.5 \\[0.2em]
\hline
\textit{Total}         & \textit{32.3\%}  & \textit{32.5\%}  & \textit{35.2\%} & \textit{33.5\%} \\
\hline
\end{tabular}
\end{center}
\footnotesize{{\bfseries Notes:} Reference sample who satisfied the selection criteria (born in the city of residence and of Italian citizenship) as a percentage of the total number of names given by the population registries, broken down by City and Cohort. Source: authors calculations on data provided by the population registries.}
\end{table}
