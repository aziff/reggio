\section{Summary}
This document describes the evaluation of the Reggio Children Approach to Early Childhood Education (for convenience, referred to as RA). This is a unique natural experiment which has been in place for fifty years in the city of
Reggio Emilia, Italy. Here a universal high-quality early child care system has developed a different
vision of the child – as an individual with rights and potential. The Reggio Children Approach has received
world-wide recognition and has been emulated in different countries and in a variety of settings,\footnote{The official \href{http://www.reggiochildren.it/network/?lang=en}{Reggio Children International Network} is present in 33 countries worldwide. Many other preschools around the world are ``inspired'' by the Reggio Children Approach but they are not officially part of these network.} but it has never been evaluated. 

The RA preschool system was introduced and grew non-randomly over the course of several decades. Therefore, the evaluation strategy has to deal with the lack of a well-defined control group, variation of treatment over time, and potential spillover effects. 

This round of analysis evaluates RA tackling these challenges. We consider several groupings of controls to account for differences by region, data collection, socioeconomic factors, and most importantly type of preschool chosen by the caregiver. %Observable parental characteristics, such as family religiosity and proximity to grandparents, are used to understand and control for selection into different types of early education. 
Different model specification allow for comparison with various control groups, allowing for a more nuanced understanding of the effects of RA, including the effects both within and between regions and in relation to other types of early education.

In general, we find \ldots \todo[backgroundcolor=orange!30,size=\tiny]{Summary of results}

Section \ref{sec:background} gives a description of RA and the cities involved with the data collection. More detail about this data collection, including sample structure and survey design, is given in Section \ref{sec:data}. The analysis is discussed in Section \ref{sec:analysis} with results presented in Section \ref{sec:OLS}.