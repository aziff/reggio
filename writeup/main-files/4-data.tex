\section{Data}
\label{sec:data}

This section discusses the survey data which has been used for the analysis in this report.\footnote{Also administrative data from the RA preschool system was collected. Its description and more details about the survey data are contained in \citet{biroli2015evaluating}} To evaluate the impact of Reggio, individuals living in Reggio Emilia, Parma, and Padova since their first year of life were interviewed. Data were collected on five age cohorts, including three cohorts of adults, one cohort of adolescents, and one cohort of children in their first year of elementary school. 

The structure of the cohorts is described in Table \ref{tab:cohorts}.

\begin{table}[htbp]
\begin{center}
\caption{Cohort Structure}\label{tab:cohorts}
\begin{tabular}{ccc}
\toprule
Cohort & Years of Birth & Age at Interview \\
\midrule
I & 1954--1959 & 54--60 \\
II & 1969--1970 & 43 \\
III & 1980--1981 & 32 \\
IV & 1994 & 18 \\
V & 2006 & 6 \\
\bottomrule  
\end{tabular}
\end{center}
\end{table}

\subsection{Outcomes considered}\label{sec:outcomes}
A rich set of outcomes were collected during the survey. Individuals at different stages of their lifecycle were asked about family composition, fertility, labor force participation, income, schooling, cognitive ability, social and emotional skills, health and healthy habits, social capital, interpersonal ties, as well as attitudes on immigration and integration. 

We focus our attention on four different outcomes, in four different domains: the mother reports to the Strength and Difficulties Questionnaire (SDQ), a widely used and validated scale from 0 to 40 measuring behavioral problems for children and adolescents; the percentage of respondents reporting to have good or excellent health;\footnote{For children and adolescent, the respondent is again the mother} the raw score of the Center for Epidemiological Studies Depression Scale (CESD), a scale ranging from 10 to 50 and measuring self-reported depression for adolescents and adults; and finally the share of respondents reporting to be `very' or  `quite' bothered by the immigration into the city. Tables (\ref{tab:outcomes-child-asilo}) to (\ref{tab:outcomes-adult-asilo}) report the summary statistics of these outcome variables, disaggregated by city and type of \textit{infant-toddler center} attended. Similarly, Tables (\ref{tab:outcomes-child-materna}) to (\ref{tab:outcomes-adult-materna}) report the average of these same variables, this time focusing on type of \textit{preschool} attended. % We see that


%% children, infant-toddler
\begin{table}[H] \label{tab:outcomes-child-asilo}
\caption{Outcomes of interest by infant-toddler-center type, children (age 6)}
% this is the top part of the tables that display the summary of the baseline characteristics, by city and child-care type
\centering
\begin{adjustbox}{width=1.2\textwidth,center=\textwidth}
\small
\begin{tabular}{m{4.0cm} cccccccccccc}
\hline \hline 
 & Reggio & Reggio & Reggio & Reggio & Parma & Parma & Parma & Parma & Padova & Padova & Padova & Padova \\
 & Municipal & Religious & Private & Not Attended & Municipal & Religious & Private & Not Attended & Municipal & Religious & Private & Not Attended \\

\hline 

SDQ score (mom rep.)   &   8.36   &   8.19   &   7.60   &   9.53**   &   7.93   &   8.29   &   7.11*   &   8.30   &   9.07   &   9.08   &   8.56   &   8.49 \\
   &   [0.32]   &   [0.79]   &   [1.78]   &   [0.43]   &   [0.33]   &   [1.44]   &   [0.52]   &   [0.43]   &   [0.54]   &   [0.98]   &   [0.75]   &   [0.33] \\
Child health is good (\%) - mom report   &   0.65   &   0.74   &   0.80   &   0.71   &   0.60   &   0.86   &   0.67   &   0.64   &   0.79**   &   0.77   &   0.71   &   0.73 \\
   &   [0.04]   &   [0.09]   &   [0.20]   &   [0.04]   &   [0.04]   &   [0.14]   &   [0.08]   &   [0.05]   &   [0.05]   &   [0.08]   &   [0.07]   &   [0.04] \\
%Child likes school (\%)   &   0.71   &   0.70   &   0.80   &   0.59**   &   0.71   &   0.71   &   0.78   &   0.74   &   0.46***   &   0.46**   &   0.59   &   0.54*** \\
%   &   [0.04]   &   [0.09]   &   [0.20]   &   [0.05]   &   [0.04]   &   [0.18]   &   [0.07]   &   [0.04]   &   [0.06]   &   [0.10]   &   [0.08]   &   [0.04] \\
%Child likes math (\%)   &   0.61   &   0.81*   &   1.00   &   0.62   &   0.68   &   0.86   &   0.69   &   0.58   &   0.68   &   0.62   &   0.65   &   0.62 \\
%   &   [0.04]   &   [0.08]   &   [0.00]   &   [0.05]   &   [0.04]   &   [0.14]   &   [0.08]   &   [0.05]   &   [0.06]   &   [0.10]   &   [0.08]   &   [0.04] \\
%Child likes reading/italian (\%)   &   0.59   &   0.59   &   0.80   &   0.49   &   0.67   &   0.86   &   0.78*   &   0.62   &   0.60   &   0.50   &   0.57   &   0.44*** \\
%   &   [0.04]   &   [0.10]   &   [0.20]   &   [0.05]   &   [0.04]   &   [0.14]   &   [0.07]   &   [0.05]   &   [0.06]   &   [0.10]   &   [0.08]   &   [0.04] \\
%Ability to sit still in a group when asked (difficulties in primary school)   &   0.12   &   0.07   &   0.00   &   0.19   &   0.15   &   0.14   &   0.06   &   0.14   &   0.19   &   0.15   &   0.10   &   0.06* \\
%   &   [0.03]   &   [0.05]   &   [0.00]   &   [0.04]   &   [0.03]   &   [0.14]   &   [0.04]   &   [0.04]   &   [0.05]   &   [0.07]   &   [0.05]   &   [0.02] \\
%Lack of excitement to learn (difficulties in primary school)   &   0.02   &   0.04   &   0.00   &   0.05   &   0.05   &   0.00   &   0.00   &   0.06*   &   0.07*   &   0.04   &   0.02   &   0.06 \\
%   &   [0.01]   &   [0.04]   &   [0.00]   &   [0.02]   &   [0.02]   &   [0.00]   &   [0.00]   &   [0.02]   &   [0.03]   &   [0.04]   &   [0.02]   &   [0.02] \\
%Ability to obey rules and directions (difficulties in primary school)   &   0.10   &   0.04   &   0.00   &   0.11   &   0.15   &   0.00   &   0.06   &   0.09   &   0.13   &   0.08   &   0.05   &   0.05 \\
%   &   [0.02]   &   [0.04]   &   [0.00]   &   [0.03]   &   [0.03]   &   [0.00]   &   [0.04]   &   [0.03]   &   [0.04]   &   [0.05]   &   [0.03]   &   [0.02] \\
%Fussy eater (difficulties in primary school)   &   0.11   &   0.19   &   0.17   &   0.03**   &   0.05*   &   0.14   &   0.03   &   0.18*   &   0.09   &   0.12   &   0.02   &   0.08 \\
%   &   [0.02]   &   [0.08]   &   [0.17]   &   [0.02]   &   [0.02]   &   [0.14]   &   [0.03]   &   [0.04]   &   [0.03]   &   [0.06]   &   [0.02]   &   [0.02] \\
% Difficulties encountered when starting primary school   &   4.29   &   4.44   &   4.83   &   3.92   &   4.17   &   4.43   &   4.64   &   3.98***   &   3.93   &   4.23   &   4.44   &   4.49* \\
%   &   [0.11]   &   [0.22]   &   [0.17]   &   [0.15]   &   [0.12]   &   [0.57]   &   [0.17]   &   [0.15]   &   [0.20]   &   [0.30]   &   [0.20]   &   [0.10] \\
\hline

% it contains the notes, assuming they are the same for all the tables.
\end{tabular}

\end{adjustbox}
\raggedright{
\footnotesize{Average of baseline characteristcs, by city and type of child-care attended. Standard errors of means in brackets. Test for difference in means between each column and the first column (Reggio Municipal, the treatment group) was performed; *** significant difference at 1\%, ** significant difference at 5\%, * significant difference at 10\%. Source: authors calculation using survey data.}
}
\end{table}
  
%\end{table}

%% adolescents, infant-toddler
\begin{table}[H] \label{tab:outcomes-adol-asilo}
\caption{Outcomes of interest by infant-toddler-center type, adolescents (age 18)}
% this is the top part of the tables that display the summary of the baseline characteristics, by city and child-care type
\centering
\begin{adjustbox}{width=1.2\textwidth,center=\textwidth}
\small
\begin{tabular}{m{4.0cm} cccccccccccc}
\hline \hline 
 & Reggio & Reggio & Reggio & Reggio & Parma & Parma & Parma & Parma & Padova & Padova & Padova & Padova \\
 & Municipal & Religious & Private & Not Attended & Municipal & Religious & Private & Not Attended & Municipal & Religious & Private & Not Attended \\

\hline 
  
%SDQ score (mom rep.)  &  7.95  &  8.89  &  9.67  &  8.47  &  8.02  &  9.70  &  9.38  &  8.01  &  8.15  &  6.38  & . &  8.32 \\
%  &  [0.36]  &  [2.64]  &  [2.33]  &  [0.37]  &  [0.42]  &  [2.39]  &  [1.45]  &  [0.38]  &  [0.57]  &  [0.91]  & .&  [0.26] \\
SDQ score &  9.96  &  11.89  &  7.67  &  10.04  &  9.11  &  9.10  &  9.46  &  8.95  &  10.52  &  7.75 & . &  9.54 \\
(self rep.)  &  [0.40]  &  [2.19]  &  [0.88]  &  [0.39]  &  [0.44]  &  [1.89]  &  [1.07]  &  [0.37]  &  [0.67]  &  [1.46]  & .&  [0.30] \\
Depression score &  23.08  &  23.78  &  20.00  &  22.75  &  22.08  &  21.20  &  21.15  &  22.35  &  22.87  &  20.88  & . &  20.90*** \\
(CESD)    &  [0.51]  &  [2.96]  &  [4.58]  &  [0.58]  &  [0.54]  &  [1.31]  &  [1.35]  &  [0.45]  &  [0.85]  &  [2.06]   & . &  [0.40] \\
Respondent health &  0.76  &  0.67  &  1.00  &  0.66*  &  0.54***  &  0.50  &  0.85  &  0.60***  &  0.70  &  0.57   & . &  0.76 \\
is good (\%)    &  [0.03]  &  [0.17]  &  [0.00]  &  [0.04]  &  [0.05]  &  [0.17]  &  [0.10]  &  [0.04]  &  [0.06]  &  [0.20]   & . &  [0.03] \\
%Child health is good (\%) - mom report  &  0.70  &  0.78  &  1.00  &  0.62  &  0.45***  &  0.50  &  0.77  &  0.57**  &  0.62  &  0.75   & . &  0.72 \\
%  &  [0.04]  &  [0.15]  &  [0.00]  &  [0.04]  &  [0.05]  &  [0.17]  &  [0.12]  &  [0.04]  &  [0.06]  &  [0.16]   & . &  [0.03] \\
%Bothered by migrants (\%)  &  0.32  &  0.11  &  0.00  &  0.29  &  0.22  &  0.00  &  0.31  &  0.24  &  0.28  &  0.25   & . &  0.26 \\
%  &  [0.04]  &  [0.11]  &  [0.00]  &  [0.04]  &  [0.04]  &  [0.00]  &  [0.13]  &  [0.04]  &  [0.06]  &  [0.16]  & .  &  [0.03] \\
%Child likes school (\%)  &  0.74  &  0.56  &  0.67  &  0.72  &  0.69  &  0.44  &  0.62  &  0.70  &  0.70  &  0.62   & . &  0.70 \\
%  &  [0.04]  &  [0.18]  &  [0.33]  &  [0.04]  &  [0.05]  &  [0.18]  &  [0.14]  &  [0.04]  &  [0.06]  &  [0.18]   & . &  [0.03] \\
%Child likes math (\%)  &  0.51  &  0.44  &  0.67  &  0.53  &  0.69***  &  0.56  &  0.54  &  0.57  &  0.52  &  0.62   & . &  0.58 \\
%  &  [0.04]  &  [0.18]  &  [0.33]  &  [0.05]  &  [0.05]  &  [0.18]  &  [0.14]  &  [0.04]  &  [0.07]  &  [0.18]   & . &  [0.03] \\
%Child likes reading/italian (\%)  &  0.67  &  0.56  &  0.33  &  0.72  &  0.78  &  0.44  &  0.62  &  0.81**  &  0.60  &  0.25**   & . &  0.65 \\
%  &  [0.04]  &  [0.18]  &  [0.33]  &  [0.04]  &  [0.04]  &  [0.18]  &  [0.14]  &  [0.03]  &  [0.06]  &  [0.16]   & . &  [0.03] \\
%Ability to sit still in a group when asked (difficulties in primary school)  &  0.06  &  0.00  &  0.00  &  0.08  &  0.10  &  0.10  &  0.15  &  0.11  &  0.10  &  0.12   & . &  0.09 \\
%  &  [0.02]  &  [0.00]  &  [0.00]  &  [0.02]  &  [0.03]  &  [0.10]  &  [0.10]  &  [0.03]  &  [0.04]  &  [0.12]   & . &  [0.02] \\
%Lack of excitement to learn (difficulties in primary school)  &  0.03  &  0.00  &  0.00  &  0.05  &  0.04  &  0.00  &  0.00  &  0.02  &  0.07  &  0.00   & . &  0.04 \\
%  &  [0.01]  &  [0.00]  &  [0.00]  &  [0.02]  &  [0.02]  &  [0.00]  &  [0.00]  &  [0.01]  &  [0.03]  &  [0.00]   & . &  [0.01] \\
%Ability to obey rules and directions (difficulties in primary school)  &  0.04  &  0.00  &  0.00  &  0.06  &  0.07  &  0.00  &  0.15  &  0.07  &  0.08  &  0.00   & . &  0.05 \\
%  &  [0.02]  &  [0.00]  &  [0.00]  &  [0.02]  &  [0.03]  &  [0.00]  &  [0.10]  &  [0.02]  &  [0.04]  &  [0.00]   & . &  [0.02] \\
%Fussy eater (difficulties in primary school)  &  0.09  &  0.00  &  0.00  &  0.06  &  0.09  &  0.20  &  0.00  &  0.13  &  0.05  &  0.00   & . &  0.10 \\
%  &  [0.02]  &  [0.00]  &  [0.00]  &  [0.02]  &  [0.03]  &  [0.13]  &  [0.00]  &  [0.03]  &  [0.03]  &  [0.00]   & . &  [0.02] \\
% Difficulties encountered when starting primary school  &  4.55  &  5.00  &  5.00  &  4.46  &  4.35  &  4.40  &  4.38  &  4.34  &  4.39  &  4.50   & . &  4.42 \\
%  &  [0.09]  &  [0.00]  &  [0.00]  &  [0.11]  &  [0.13]  &  [0.40]  &  [0.42]  &  [0.11]  &  [0.17]  &  [0.50]   & . &  [0.09] \\
\hline

% it contains the notes, assuming they are the same for all the tables.
\end{tabular}

\end{adjustbox}
\raggedright{
\footnotesize{Average of baseline characteristcs, by city and type of child-care attended. Standard errors of means in brackets. Test for difference in means between each column and the first column (Reggio Municipal, the treatment group) was performed; *** significant difference at 1\%, ** significant difference at 5\%, * significant difference at 10\%. Source: authors calculation using survey data.}
}
\end{table}
  
%\end{table}

%% adults, infant-toddler
\begin{table}[H] \label{tab:outcomes-adult-asilo}
\caption{Outcomes of interest by infant-toddler-center type, adults (age 30-50)}
% this is the top part of the tables that display the summary of the baseline characteristics, by city and child-care type
\centering
\begin{adjustbox}{width=1.2\textwidth,center=\textwidth}
\small
\begin{tabular}{m{4.0cm} cccccccccccc}
\hline \hline 
 & Reggio & Reggio & Reggio & Reggio & Parma & Parma & Parma & Parma & Padova & Padova & Padova & Padova \\
 & Municipal & Religious & Private & Not Attended & Municipal & Religious & Private & Not Attended & Municipal & Religious & Private & Not Attended \\

\hline 
  
Depression score (CESD)  &  21.28  &  17.75  &  22.33  &  21.99  &  17.91***  &  19.73  &  19.82  &  21.35  &  26.57***  &  22.71  &  20.33  &  21.32 \\
  &  [0.55]  &  [0.75]  &  [4.33]  &  [0.23]  &  [0.55]  &  [1.59]  &  [1.59]  &  [0.24]  &  [0.88]  &  [1.44]  &  [2.33]  &  [0.24] \\
Respondent health is good (\%)  &  0.93  &  1.00  &  1.00  &  0.71***  &  0.29***  &  0.53***  &  0.27***  &  0.60***  &  0.63***  &  0.24***  &  1.00  &  0.54*** \\
  &  [0.03]  &  [0.00]  &  [0.00]  &  [0.02]  &  [0.05]  &  [0.13]  &  [0.14]  &  [0.02]  &  [0.08]  &  [0.11]  &  [0.00]  &  [0.02] \\
Bothered by migrants (\%)  &  0.07  &  0.25  &  0.00  &  0.21***  &  0.21**  &  0.33**  &  0.36**  &  0.23***  &  0.47***  &  0.61***  &  0.67**  &  0.35*** \\
  &  [0.03]  &  [0.25]  &  [0.00]  &  [0.02]  &  [0.05]  &  [0.13]  &  [0.15]  &  [0.02]  &  [0.08]  &  [0.12]  &  [0.33]  &  [0.02] \\
\hline

% it contains the notes, assuming they are the same for all the tables.
\end{tabular}

\end{adjustbox}
\raggedright{
\footnotesize{Average of baseline characteristcs, by city and type of child-care attended. Standard errors of means in brackets. Test for difference in means between each column and the first column (Reggio Municipal, the treatment group) was performed; *** significant difference at 1\%, ** significant difference at 5\%, * significant difference at 10\%. Source: authors calculation using survey data.}
}
\end{table}
  
%\end{table}

%%% children, preschool
\begin{table}[H] \label{tab:outcomes-child-materna}
\caption{Outcomes of interest by preschool type, children (age 6)}
% this is the top part of the tables that display the summary of the baseline characteristics, by city and child-care type
\centering
\begin{adjustbox}{width=1.2\textwidth,center=\textwidth}
\small
\begin{tabular}{m{4.0cm} ccccccccccccccc}
\hline \hline 
 & Reggio & Reggio & Reggio & Reggio & Reggio & Parma & Parma & Parma & Parma & Parma & Padova & Padova & Padova & Padova & Padova \\
 & Municipal & State & Religious & Private & Not Attended & Municipal & State & Religious & Private & Not Attended & Municipal & State & Religious & Private & Not Attended \\

\hline 
  
SDQ score (mom rep.) & 8.33 & 10.02* & 9.00 & 9.00 & 10.00 & 7.80 & 7.19* & 8.66 & 8.56 & 8.67 & 9.07 & 8.85 & 8.23 & 10.00 & 12.50* \\
 & [0.32] & [0.82] & [0.42] & [2.71] & [4.00] & [0.32] & [0.58] & [0.48] & [0.96] & [1.50] & [0.50] & [0.62] & [0.34] & [1.69] & [1.50] \\
Child health is good (\%) - mom report & 0.67 & 0.71 & 0.68 & 0.75 & 0.50 & 0.60 & 0.63 & 0.68 & 0.78 & 0.50 & 0.78 & 0.80 & 0.71 & 0.75 & 1.00 \\
 & [0.04] & [0.07] & [0.05] & [0.25] & [0.50] & [0.04] & [0.07] & [0.05] & [0.15] & [0.22] & [0.05] & [0.06] & [0.04] & [0.13] & [0.00] \\
%Child likes school (\%) & 0.69 & 0.60 & 0.64 & 0.75 & 0.50 & 0.72 & 0.77 & 0.77 & 0.67 & 0.50 & 0.39*** & 0.50** & 0.58* & 0.58 & 1.00 \\
% & [0.04] & [0.07] & [0.05] & [0.25] & [0.50] & [0.04] & [0.07] & [0.05] & [0.17] & [0.22] & [0.05] & [0.08] & [0.04] & [0.15] & [0.00] \\
%Child likes math (\%) & 0.61 & 0.71 & 0.67 & 0.75 & 0.00 & 0.68 & 0.65 & 0.68 & 0.56 & 0.00 & 0.69 & 0.60 & 0.62 & 0.55 & 1.00 \\
% & [0.04] & [0.07] & [0.05] & [0.25] & [0.00] & [0.04] & [0.07] & [0.05] & [0.18] & [0.00] & [0.05] & [0.08] & [0.04] & [0.16] & [0.00] \\
%Child likes reading/italian (\%) & 0.57 & 0.56 & 0.53 & 1.00 & 0.00 & 0.67* & 0.72* & 0.64 & 0.67 & 0.80 & 0.57 & 0.47 & 0.50 & 0.27* & 0.50 \\
% & [0.04] & [0.07] & [0.05] & [0.00] & [0.00] & [0.04] & [0.07] & [0.06] & [0.17] & [0.20] & [0.06] & [0.08] & [0.04] & [0.14] & [0.50] \\
%Ability to sit still in a group when asked (difficulties in primary school) & 0.13 & 0.09 & 0.17 & 0.00 & 0.00 & 0.14 & 0.09 & 0.16 & 0.00 & 0.17 & 0.20 & 0.03* & 0.08 & 0.08 & 0.00 \\
% & [0.03] & [0.04] & [0.04] & [0.00] & [0.00] & [0.03] & [0.04] & [0.04] & [0.00] & [0.17] & [0.04] & [0.03] & [0.02] & [0.08] & [0.00] \\
%Lack of excitement to learn (difficulties in primary school) & 0.02 & 0.07 & 0.02 & 0.00 & 0.00 & 0.03 & 0.05 & 0.06 & 0.00 & 0.33** & 0.09** & 0.05 & 0.04 & 0.08 & 0.00 \\
% & [0.01] & [0.04] & [0.02] & [0.00] & [0.00] & [0.01] & [0.03] & [0.03] & [0.00] & [0.21] & [0.03] & [0.03] & [0.02] & [0.08] & [0.00] \\
%Ability to obey rules and directions (difficulties in primary school) & 0.10 & 0.11 & 0.08 & 0.00 & 0.50 & 0.10 & 0.14 & 0.16 & 0.11 & 0.00 & 0.10 & 0.12 & 0.05* & 0.00 & 0.00 \\
% & [0.02] & [0.05] & [0.03] & [0.00] & [0.50] & [0.02] & [0.05] & [0.04] & [0.11] & [0.00] & [0.03] & [0.05] & [0.02] & [0.00] & [0.00] \\
%Fussy eater (difficulties in primary school) & 0.07 & 0.09 & 0.11 & 0.40** & 0.00 & 0.06 & 0.07 & 0.16** & 0.11 & 0.33* & 0.09 & 0.07 & 0.09 & 0.00 & 0.00 \\
% & [0.02] & [0.04] & [0.03] & [0.24] & [0.00] & [0.02] & [0.04] & [0.04] & [0.11] & [0.21] & [0.03] & [0.04] & [0.02] & [0.00] & [0.00] \\
%Difficulties encountered when starting primary school & 4.25 & 4.22 & 4.04* & 4.60** & 4.00 & 4.25 & 4.30 & 3.95** & 4.67 & 3.00*** & 3.89** & 4.53 & 4.49* & 4.42 & 5.00 \\
% & [0.11] & [0.20] & [0.16] & [0.24] & [1.00] & [0.12] & [0.20] & [0.18] & [0.24] & [0.63] & [0.18] & [0.16] & [0.10] & [0.40] & [0.00] \\
\hline

% it contains the notes, assuming they are the same for all the tables.
\end{tabular}

\end{adjustbox}
\raggedright{
\footnotesize{Average of baseline characteristcs, by city and type of child-care attended. Standard errors of means in brackets. Test for difference in means between each column and the first column (Reggio Municipal, the treatment group) was performed; *** significant difference at 1\%, ** significant difference at 5\%, * significant difference at 10\%. Source: authors calculation using survey data.}
}
\end{table}
  
%\end{table}
%
%%% adolescents, preschool
\begin{table}[H] \label{tab:outcomes-adol-materna}
\caption{Outcomes of interest by preschool type, adolescents (age 18)}
% this is the top part of the tables that display the summary of the baseline characteristics, by city and child-care type
\centering
\begin{adjustbox}{width=1.2\textwidth,center=\textwidth}
\small
\begin{tabular}{m{4.0cm} ccccccccccccccc}
\hline \hline 
 & Reggio & Reggio & Reggio & Reggio & Reggio & Parma & Parma & Parma & Parma & Parma & Padova & Padova & Padova & Padova & Padova \\
 & Municipal & State & Religious & Private & Not Attended & Municipal & State & Religious & Private & Not Attended & Municipal & State & Religious & Private & Not Attended \\

\hline 
  
%SDQ score (mom rep.) & 8.16 & 6.64* & 8.70 & 9.33 & 8.86 & 7.86 & 7.95 & 8.73 & 8.17 & 9.75 & 7.68 & 9.02 & 8.32 & 8.17 & . \\
% & [0.33] & [0.58] & [0.53] & [1.43] & [1.40] & [0.39] & [0.60] & [0.54] & [2.27] & [2.32] & [0.34] & [0.60] & [0.37] & [1.45] & . \\
SDQ score & 9.82 & 7.73** & 10.77* & 11.00 & 7.43 & 9.11 & 8.00** & 9.53 & 8.67 & 8.75 & 9.56 & 9.00 & 10.18 & 7.50 & . \\
(self rep.)  & [0.37] & [0.79] & [0.47] & [1.32] & [1.63] & [0.39] & [0.68] & [0.47] & [1.50] & [0.85] & [0.47] & [0.55] & [0.43] & [1.26] & . \\
Depression score & 22.46 & 20.71 & 24.15** & 23.33 & 19.14 & 22.29 & 20.60 & 22.81 & 20.00 & 20.75 & 21.75 & 19.32*** & 21.77 & 19.67 & . \\
(CESD)  & [0.51] & [1.23] & [0.69] & [1.58] & [1.50] & [0.48] & [0.74] & [0.58] & [1.55] & [2.66] & [0.61] & [0.78] & [0.55] & [1.98] & . \\
Respondent health & 0.74 & 0.82 & 0.66 & 0.83 & 0.71 & 0.62** & 0.37*** & 0.63 & 1.00 & 0.75 & 0.68 & 0.85 & 0.75 & 0.67 & . \\
is good (\%)  & [0.03] & [0.08] & [0.05] & [0.17] & [0.18] & [0.05] & [0.07] & [0.05] & [0.00] & [0.25] & [0.05] & [0.05] & [0.04] & [0.21] & . \\
%Child health is good (\%) - mom report & 0.69 & 0.64 & 0.66 & 0.33* & 0.71 & 0.49*** & 0.40*** & 0.63 & 0.83 & 1.00 & 0.65 & 0.83* & 0.69 & 0.67 & . \\
% & [0.04] & [0.10] & [0.05] & [0.21] & [0.18] & [0.05] & [0.08] & [0.05] & [0.17] & [0.00] & [0.05] & [0.06] & [0.04] & [0.21] & . \\
%Bothered by migrants (\%) & 0.26 & 0.23 & 0.37* & 0.67** & 0.14 & 0.23 & 0.14 & 0.28 & 0.17 & 0.75* & 0.24 & 0.18 & 0.32 & 0.33 & 0.00 \\
% & [0.03] & [0.09] & [0.05] & [0.21] & [0.14] & [0.04] & [0.05] & [0.05] & [0.17] & [0.25] & [0.05] & [0.06] & [0.04] & [0.21] & . \\
%Child likes school (\%) & 0.73 & 0.85 & 0.72 & 0.33* & 0.60 & 0.68 & 0.79 & 0.62 & 0.50 & 1.00 & 0.69 & 0.68 & 0.71 & 0.83 & 0.00 \\
% & [0.03] & [0.08] & [0.05] & [0.21] & [0.24] & [0.04] & [0.06] & [0.05] & [0.22] & [0.00] & [0.05] & [0.07] & [0.04] & [0.17] & . \\
%Child likes math (\%) & 0.51 & 0.80** & 0.46 & 0.67 & 0.40 & 0.62* & 0.67* & 0.59 & 0.33 & 0.50 & 0.57 & 0.42 & 0.58 & 1.00 & 0.00 \\
% & [0.04] & [0.09] & [0.05] & [0.21] & [0.24] & [0.05] & [0.07] & [0.06] & [0.21] & [0.29] & [0.05] & [0.08] & [0.04] & [0.00] & . \\
%Child likes reading/italian (\%) & 0.71 & 0.65 & 0.66 & 0.50 & 0.80 & 0.80* & 0.81 & 0.73 & 0.67 & 0.50 & 0.63 & 0.67 & 0.62 & 0.50 & 1.00 \\
% & [0.04] & [0.11] & [0.05] & [0.22] & [0.20] & [0.04] & [0.06] & [0.05] & [0.21] & [0.29] & [0.05] & [0.07] & [0.04] & [0.22] & . \\
%Ability to sit still in a group when asked (difficulties in primary school) & 0.07 & 0.00 & 0.06 & 0.17 & 0.14 & 0.11 & 0.02 & 0.13* & 0.17 & 0.25 & 0.08 & 0.17** & 0.08 & 0.00 & 0.00 \\
% & [0.02] & [0.00] & [0.02] & [0.17] & [0.14] & [0.03] & [0.02] & [0.04] & [0.17] & [0.25] & [0.03] & [0.06] & [0.02] & [0.00] & . \\
%Lack of excitement to learn (difficulties in primary school) & 0.03 & 0.05 & 0.04 & 0.00 & 0.29** & 0.05 & 0.00 & 0.00 & 0.00 & 0.25 & 0.03 & 0.09 & 0.04 & 0.17 & 0.00 \\
% & [0.01] & [0.05] & [0.02] & [0.00] & [0.18] & [0.02] & [0.00] & [0.00] & [0.00] & [0.25] & [0.02] & [0.04] & [0.02] & [0.17] & . \\
%Ability to obey rules and directions (difficulties in primary school) & 0.04 & 0.05 & 0.04 & 0.00 & 0.14 & 0.08 & 0.09 & 0.09 & 0.00 & 0.00 & 0.06 & 0.11 & 0.05 & 0.00 & 0.00 \\
% & [0.02] & [0.05] & [0.02] & [0.00] & [0.14] & [0.02] & [0.04] & [0.03] & [0.00] & [0.00] & [0.03] & [0.05] & [0.02] & [0.00] & . \\
%Fussy eater (difficulties in primary school) & 0.07 & 0.05 & 0.08 & 0.33* & 0.00 & 0.15** & 0.05 & 0.11 & 0.00 & 0.00 & 0.05 & 0.17** & 0.07 & 0.17 & 0.00 \\
% & [0.02] & [0.05] & [0.03] & [0.21] & [0.00] & [0.03] & [0.03] & [0.03] & [0.00] & [0.00] & [0.02] & [0.06] & [0.02] & [0.17] & . \\
%Difficulties encountered when starting primary school & 4.57 & 4.82 & 4.52 & 4.00* & 3.57** & 4.24* & 4.67 & 4.34 & 4.33 & 3.25* & 4.57 & 3.98** & 4.46 & 4.33 & 5.00 \\
% & [0.09] & [0.14] & [0.11] & [0.63] & [0.69] & [0.13] & [0.13] & [0.15] & [0.67] & [1.03] & [0.12] & [0.23] & [0.11] & [0.49] & . \\
\hline

% it contains the notes, assuming they are the same for all the tables.
\end{tabular}

\end{adjustbox}
\raggedright{
\footnotesize{Average of baseline characteristcs, by city and type of child-care attended. Standard errors of means in brackets. Test for difference in means between each column and the first column (Reggio Municipal, the treatment group) was performed; *** significant difference at 1\%, ** significant difference at 5\%, * significant difference at 10\%. Source: authors calculation using survey data.}
}
\end{table}
  
%\end{table}
%
%%% adults, preschool
\begin{table}[H] \label{tab:outcomes-adult-materna}
\caption{Outcomes of interest by preschool type, adults (age 30-50)}
% this is the top part of the tables that display the summary of the baseline characteristics, by city and child-care type
\centering
\begin{adjustbox}{width=1.2\textwidth,center=\textwidth}
\small
\begin{tabular}{m{4.0cm} ccccccccccccccc}
\hline \hline 
 & Reggio & Reggio & Reggio & Reggio & Reggio & Parma & Parma & Parma & Parma & Parma & Padova & Padova & Padova & Padova & Padova \\
 & Municipal & State & Religious & Private & Not Attended & Municipal & State & Religious & Private & Not Attended & Municipal & State & Religious & Private & Not Attended \\

\hline 
  
Depression score (CESD) & 21.25 & 22.50 & 20.77 & 19.29 & 22.96*** & 20.50 & 18.96*** & 20.04 & 20.83 & 22.21*** & 22.11 & 25.20*** & 20.79 & 20.67 & 22.07* \\
 & [0.34] & [0.88] & [0.50] & [1.49] & [0.33] & [0.47] & [0.66] & [0.48] & [2.99] & [0.31] & [0.70] & [0.81] & [0.31] & [2.96] & [0.46] \\
Respondent health is good (\%) & 0.88 & 0.81 & 0.75*** & 0.75 & 0.57*** & 0.54*** & 0.48*** & 0.68*** & 0.33*** & 0.50*** & 0.59*** & 0.80 & 0.53*** & 0.50 & 0.47*** \\
 & [0.02] & [0.05] & [0.04] & [0.16] & [0.03] & [0.04] & [0.05] & [0.04] & [0.21] & [0.03] & [0.06] & [0.06] & [0.03] & [0.50] & [0.04] \\
Bothered by migrants (\%) & 0.12 & 0.16 & 0.25*** & 0.14 & 0.23*** & 0.30*** & 0.24*** & 0.30*** & 0.17 & 0.16 & 0.50*** & 0.32*** & 0.37*** & 0.33 & 0.34*** \\
 & [0.02] & [0.05] & [0.04] & [0.14] & [0.03] & [0.04] & [0.05] & [0.04] & [0.17] & [0.02] & [0.06] & [0.07] & [0.03] & [0.33] & [0.04] \\
\hline

% it contains the notes, assuming they are the same for all the tables.
\end{tabular}

\end{adjustbox}
\raggedright{
\footnotesize{Average of baseline characteristcs, by city and type of child-care attended. Standard errors of means in brackets. Test for difference in means between each column and the first column (Reggio Municipal, the treatment group) was performed; *** significant difference at 1\%, ** significant difference at 5\%, * significant difference at 10\%. Source: authors calculation using survey data.}
}
\end{table}
  
%\end{table}

%%%%%%%%%%%%
Finally, table (\ref{tab:outcomes-list}) compares the outcomes available in the our survey data with the ones available in ABC and Perry.
\begin{table}[htbp]\label{tab:outcomes-list}
\begin{center}

	\caption{Outcome Variables in Reggio and Other Early Childhood Studies}
\begin{tabular}{lllll}
\toprule
Outcome & Reggio & ABC & CARE & Perry \\
\midrule
\textbf{Cognitive} & & & & \\
\quad IQ & 6, 18, 32, 43, 54-60 & 6, 15 & 6 & 6, 14 \\
\textbf{Schooling} & & & & \\
\quad HS Graduation & 32, 43, 54-60 & 30 & 30 & 27, 40 \\
\quad Ever Suspended & 18 & & & 27 \\
\textbf{Socio-emotional} & & & & \\
\quad Strengths and Difficulties & 6, 18, 32, 43, 54-60 & & & \\
\quad Depression &18, 32, 43, 54-60 & 15, 21, 34 & 21, 34 & \\
\textbf{Health} & & & & \\
\quad Number of Cigarettes (per day) & 18, 32, 43, 54-60 & 21 & 21 & 27 \\
\quad BMI & 6, 32, 43, 54-60 & 34 & 34 & 40 \\
\quad Health Problems & 32, 43, 54-60 & 34 & 34 & 27 \\
\quad Ever Tried Drugs & 32, 43, 54-60 & 12, 15, 21, 30 & 12, 15, 21, 30 & 27 \\
\bottomrule
\end{tabular}
%
%
\end{center}
\footnotesize 
Notes: The columns for each dataset specify the ages in years at which the variables are available. IQ is measured by Raven's Progressive Matrices for Reggio, the Wechsler Intelligence Scale for Children (WISC) for ABC and CARE, and WISC and Stanford Binet for Perry. The ``Health Problems" item in Reggio, ABC, and CARE corresponds to the number of days in the past 30 days the subject was sick. In Perry, this variable is the number of days sick in the past year. ``Strengths and Difficulties" is the Strengths and Difficulties Questionnaire administered to adolescent and adult subjects in Reggio. ``Depression" is the Depression Scale in Reggio. In ABC, depression is measured from the Achenbach Youth Report at age 15. In ABC and CARE, depression is measured using the Brief Symptom Inventory at age 21 and is self-reported at age 34.
\end{table}

%\todo[backgroundcolor=orange!30,size=\tiny]{Tables with raw differences across cities and preschool types}
