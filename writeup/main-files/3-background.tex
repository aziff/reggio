\section{Background}
\label{sec:background}

% Quick description of Reggio Children Approach history
This section briefly describes the program and the cities in which the data were collected. For a more complete discussion of the Italian early childhood landscape, the educational philosophy and the research design, see \citet{biroli2015evaluating}. The Reggio Children Approach to early childhood education is a community effort of public investment in high-quality early childhood education spearheaded and led by the pedagogist Loris Malaguzzi (1920-1994), whose foresight has been the main influence on the approach. The system evolved slowly over the years from a parent cooperative in the outskirts of Reggio Emilia into a structured municipal system of 26 infant-toddler centers and 30 preschools. Building on existing pedagogical models, this educational project is centered around the interaction between children, families, teachers, and the community (\cite{Malaguzzi1993}).

The Reggio Children Approach includes infant-toddler centers (ages 0-2) and preschools (ages 3-5), both of which are for children before they enter primary school at age 6. The first Reggio infant-toddler center was established in 1971, while the first Reggio preschool was established even earlier, in 1963. 

% Reggio Emilia vs. Parma vs. Padova
In addition to collecting data from individuals in Reggio Emilia, data were collected from individuals in Parma and Padova, two cities that share several features with Reggio Emilia. Table (\ref{tab:comparison}) lists some characteristics of the three cities. More information on the data collection is given in Section (\ref{sec:data}).

\begin{table}[htbp]
\begin{center}
\caption{Demographic Comparison of Reggio Emilia, Parma, and Padova}
\label{tab:comparison}
\begin{tabular}{lccc}
\toprule
& \multicolumn{3}{c}{City} \\
\cmidrule{2-4}
& Reggio Emilia & Parma & Padova \\
\midrule
Population (2013)* & 172,525 &  187,938 & 209,678 \\
Average per-capita income (2011 euros)**  & 25,226 & 28,437 & 29,915 \\
\bottomrule
\end{tabular}
\end{center}
\footnotesize Notes: *ISTAT, \url{http://www.demo.istat.it/}; **Finance Minister, taxable income for 2011.
\end{table}
\todo[backgroundcolor=orange!30,size=\tiny]{Add information about (updated) italian and immigrant births to table \ref{tab:comparison}. Look for values in 1960 (start of the program, baseline)}

% Preschool availability and take-up

% Different types of preschool
As in every Italian city, early educational experiences are divided into two main age categories. The first, infant-toddler centers (\textit{asilo}), is available for children aged 0 through 3 while the second, preschool (\textit{materna}), is for children aged 3 through 6. 

For each of these age groups, there are schools established and managed by different entities: municipal, state, private, and religious. Table \ref{tab:types} summarizes the types of schools available for each age group, highlighting that there are no state-run infant-toddler centers.

\begin{table}[htbp]
\begin{center}
\caption{Types of Schools}\label{tab:types}
\begin{tabular}{ccc}
\toprule
& Infant-Toddler & Preschool \\
\midrule
Municipal & \checkmark & \checkmark \\
State & & \checkmark \\
Private & \checkmark & \checkmark \\
Religious & \checkmark & \checkmark \\
\bottomrule
\end{tabular}
\end{center}
\end{table}

The schools that follow the Reggio Children Approach are the \textbf{municipal} infant-toddler centers and preschools in Reggio Emilia. All of the children who attended a municipal infant-toddler center or preschool in Reggio Emilia are considered part of the treated group, since they received the RA intervention. The other types of schools in Reggio Emilia, and all the schools in Parma and Padova (including the municipal schools) did not receive the RA intervention. However, we cannot rule out that also these other school types borrowed aspects from the RA philosophy of early childhood education, especially in the city of Reggio Emilia. A detailed interview with the managers of the religious preschools in Reggio Emilia showed that [...] \todo[backgroundcolor=orange!30,size=\tiny]{Together with Claudia and Moira, who run the RA schools, we are collecting more specific information available about the religious schools in Reggio. No information is available for state and private}
In Padova, influences from Guerra Frabboni's philosophy prevail \citep{Frabboni1999}, while in Parma, there is a mixture of influences.
