\section{Analysis}
\label{sec:analysis}
This section reports the preliminary results of the analysis. 
Section \ref{sec:balance} provides information across cities and school types about the balance of characteristics that can be observed at baseline, before the child attends any type of child care.
Section \ref{sec:OLS} reports evidence of the difference in the relevant outcomes once these initial characteristics are controlled for.
%Section \ref{sec:PSM} takes a different approach and compares the children who had a similar probability of attending the RA school system. 
%Finally, section \ref{sec:IV} leverages information about distance from the different centers as an instrument for participation in the RA.
\todo[backgroundcolor=orange!30,size=\tiny]{Propensity score matching and IV}
\subsection{Balance of observable characteristics}
\label{sec:balance}
Given the non experimental nature of the analysis, selection of the proper control group is crucial. The following tables describe the characteristics of the family and child before going to the different types of child care. 

%\todo[backgroundcolor=orange!30,size=\tiny]{Insert here all the tables of balance of background characteristics.}
\singlespacing

%% children, infant-toddler
\begin{table}[H]
\caption{Baseline characteristics by infant-toddler-center type, children (age 6)}
% this is the top part of the tables that display the summary of the baseline characteristics, by city and child-care type
\centering
\begin{adjustbox}{width=1.2\textwidth,center=\textwidth}
\small
\begin{tabular}{m{4.0cm} cccccccccccc}
\hline \hline 
 & Reggio & Reggio & Reggio & Reggio & Parma & Parma & Parma & Parma & Padova & Padova & Padova & Padova \\
 & Municipal & Religious & Private & Not Attended & Municipal & Religious & Private & Not Attended & Municipal & Religious & Private & Not Attended \\

\hline 

CAPI & 0.60 & 0.56 & 0.17{*} & 0.51 & 0.44{***} & 0.43 & 0.36{**} & 0.46{**} & 0.49 & 0.31{***} & 0.32{***} & 0.55 \\
 & [0.04] & [0.10] & [0.17] & [0.05] & [0.04] & [0.20] & [0.08] & [0.05] & [0.06] & [0.09] & [0.07] & [0.04] \\
Male dummy & 0.57 & 0.52 & 0.67 & 0.49 & 0.58 & 0.43 & 0.56 & 0.54 & 0.56 & 0.62 & 0.59 & 0.47{*} \\
 & [0.04] & [0.10] & [0.21] & [0.05] & [0.04] & [0.20] & [0.08] & [0.05] & [0.06] & [0.10] & [0.08] & [0.04] \\
Age & 6.77 & 6.76 & 6.83 & 6.84 & 6.70{*} & 6.7 & 6.77 & 6.70{*} & 6.77 & 6.74 & 6.71 & 6.59{***} \\
 & [0.03] & [0.07] & [0.15] & [0.03] & [0.03] & [0.11] & [0.05] & [0.03] & [0.04] & [0.08] & [0.05] & [0.03] \\
%Age sq. & 46 & 45.77 & 46.76 & 46.95 & 44.97{*} & 44.9 & 46 & 44.97{*} & 45.89 & 45.64 & 45.14 & 43.52{***} \\
% & [0.38] & [0.97] & [2.12] & [0.47] & [0.35] & [1.47] & [0.75] & [0.45] & [0.56] & [1.04] & [0.66] & [0.37] \\
Middle school edu mom & 0.07 & 0.04 & 0.00 & 0.1 & 0.05 & 0.00 & 0.06 & 0.06 & 0.09 & 0.04 & 0.1 & 0.1 \\
 & [0.02] & [0.04] & [0.00] & [0.03] & [0.02] & [0.00] & [0.04] & [0.02] & [0.03] & [0.04] & [0.05] & [0.02] \\
High school edu mom & 0.47 & 0.41 & 0.5 & 0.42 & 0.37{*} & 0.57 & 0.42 & 0.46 & 0.31{**} & 0.35 & 0.34 & 0.57 \\
 & [0.04] & [0.10] & [0.22] & [0.05] & [0.04] & [0.20] & [0.08] & [0.05] & [0.06] & [0.10] & [0.07] & [0.04] \\
University edu mom & 0.33 & 0.37 & 0.5 & 0.17{***} & 0.51{***} & 0.43 & 0.47 & 0.38 & 0.50{**} & 0.42 & 0.54{**} & 0.23{*} \\
 & [0.04] & [0.09] & [0.22] & [0.03] & [0.04] & [0.20] & [0.08] & [0.05] & [0.06] & [0.10] & [0.08] & [0.04] \\
Middle school edu dad & 0.09 & 0.04 & 0.00 & 0.08 & 0.07 & 0.00 & 0.14 & 0.12 & 0.06 & 0.12 & 0.1 & 0.09 \\
 & [0.02] & [0.04] & [0.00] & [0.02] & [0.02] & [0.00] & [0.06] & [0.03] & [0.03] & [0.06] & [0.05] & [0.02] \\
High school edu dad & 0.37 & 0.22 & 0.67 & 0.33 & 0.34 & 0.14 & 0.36 & 0.41 & 0.37 & 0.23 & 0.44 & 0.48{*} \\
 & [0.04] & [0.08] & [0.21] & [0.04] & [0.04] & [0.14] & [0.08] & [0.05] & [0.06] & [0.08] & [0.08] & [0.04] \\
University edu dad & 0.23 & 0.52{***} & 0.33 & 0.21 & 0.37{***} & 0.57{*} & 0.33 & 0.32 & 0.32 & 0.38{*} & 0.29 & 0.28 \\
 & [0.03] & [0.10] & [0.21] & [0.04] & [0.04] & [0.20] & [0.08] & [0.05] & [0.06] & [0.10] & [0.07] & [0.04] \\
Mom born in province & 0.44 & 0.56 & 0.67 & 0.60{**} & 0.58{**} & 0.29 & 0.75{***} & 0.61{**} & 0.56 & 0.54 & 0.54 & 0.82{***} \\
 & [0.04] & [0.10] & [0.21] & [0.05] & [0.04] & [0.18] & [0.07] & [0.05] & [0.06] & [0.10] & [0.08] & [0.03] \\
Dad born in province & 0.45 & 0.70{**} & 0.67 & 0.57{*} & 0.56{*} & 0.57 & 0.67{**} & 0.60{**} & 0.49 & 0.5 & 0.54 & 0.77{***} \\
 & [0.04] & [0.09] & [0.21] & [0.05] & [0.04] & [0.20] & [0.08] & [0.05] & [0.06] & [0.10] & [0.08] & [0.04] \\
Caregiver is religious & 0.82 & 0.81 & 0.67 & 0.9 & 0.85 & 0.57 & 0.89 & 0.91{*} & 0.72 & 0.92 & 0.88 & 0.79 \\
 & [0.03] & [0.08] & [0.21] & [0.03] & [0.03] & [0.20] & [0.05] & [0.03] & [0.05] & [0.05] & [0.05] & [0.03] \\
Own Home & 0.55 & 0.74{*} & 0.5 & 0.58 & 0.70{***} & 0.43 & 0.67 & 0.76{***} & 0.66 & 0.69 & 0.61 & 0.66{**} \\
 & [0.04] & [0.09] & [0.22] & [0.05] & [0.04] & [0.20] & [0.08] & [0.04] & [0.06] & [0.09] & [0.08] & [0.04] \\
Income 5k-10k eur & 0.01 & 0.00 & 0.00 & 0.02 & 0.03 & 0.00 & 0.03 & 0.00 & 0.00 & 0.00 & 0.02 & 0.02 \\
 & [0.01] & [0.00] & [0.00] & [0.01] & [0.01] & [0.00] & [0.03] & [0.00] & [0.00] & [0.00] & [0.02] & [0.01] \\
Income 10k-25k eur & 0.21 & 0.11 & 0.17 & 0.15 & 0.17 & 0.00 & 0.19 & 0.22 & 0.22 & 0.19 & 0.2 & 0.10{**} \\
 & [0.03] & [0.06] & [0.17] & [0.03] & [0.03] & [0.00] & [0.07] & [0.04] & [0.05] & [0.08] & [0.06] & [0.03] \\
Income 25k-50k eur & 0.38 & 0.33 & 0.00 & 0.25{**} & 0.44 & 0.43 & 0.36 & 0.36 & 0.43 & 0.38 & 0.39 & 0.23{***} \\
 & [0.04] & [0.09] & [0.00] & [0.04] & [0.04] & [0.20] & [0.08] & [0.05] & [0.06] & [0.10] & [0.08] & [0.04] \\
Income 50k-100k eur & 0.21 & 0.3 & 0.33 & 0.15 & 0.21 & 0.00 & 0.19 & 0.17 & 0.16 & 0.23 & 0.24 & 0.06{***} \\
 & [0.03] & [0.09] & [0.21] & [0.03] & [0.03] & [0.00] & [0.07] & [0.04] & [0.04] & [0.08] & [0.07] & [0.02] \\
Income 100k-250k eur & 0.03 & 0.04 & 0.00 & 0.02 & 0.03 & 0.00 & 0.03 & 0.01 & 0.04 & 0.04 & 0.00 & 0.03 \\
 & [0.01] & [0.04] & [0.00] & [0.01] & [0.01] & [0.00] & [0.03] & [0.01] & [0.03] & [0.04] & [0.00] & [0.01] \\
%Income more 250k eur & 0.00 & 0.00 & 0.00 & 0.00 & 0.00 & 0.00 & 0.00 & 0.00 & 0.00 & 0.00 & 0.00 & 0 \\
% & [0.00] & [0.00] & [0.00] & [0.00] & [0.00] & [0.00] & [0.00] & [0.00] & [0.00] & [0.00] & [0.00] & [0.00] \\
%Income below 5k eur & 0.01 & 0.04 & 0.00 & 0.01 & 0.01 & 0.00 & 0.03 & 0.04 & 0.01 & 0.00 & 0.00 & 0.06{*} \\
% & [0.01] & [0.04] & [0.00] & [0.01] & [0.01] & [0.00] & [0.03] & [0.02] & [0.01] & [0.00] & [0.00] & [0.02] \\
Low birthweight & 0.12 & 0.00 & 0.17 & 0.04{**} & 0.05{**} & 0.00 & 0.03 & 0.11 & 0.09 & 0.04 & 0.05 & 0.03{***} \\
 & [0.03] & [0.00] & [0.17] & [0.02] & [0.02] & [0.00] & [0.03] & [0.03] & [0.03] & [0.04] & [0.03] & [0.01] \\
Premature & 0.14 & 0.07 & 0.17 & 0.04{**} & 0.06{**} & 0.00 & 0.11 & 0.09 & 0.15 & 0.04 & 0.02{*} & 0.05{**} \\
 & [0.03] & [0.05] & [0.17] & [0.02] & [0.02] & [0.00] & [0.05] & [0.03] & [0.04] & [0.04] & [0.02] & [0.02]\\ \hline


% it contains the notes, assuming they are the same for all the tables.
\end{tabular}

\end{adjustbox}
\raggedright{
\footnotesize{Average of baseline characteristcs, by city and type of child-care attended. Standard errors of means in brackets. Test for difference in means between each column and the first column (Reggio Municipal, the treatment group) was performed; *** significant difference at 1\%, ** significant difference at 5\%, * significant difference at 10\%. Source: authors calculation using survey data.}
}
\end{table}
  
%\end{table}

%% adolescents, infant-toddler
\begin{table}[H]
\caption{Baseline characteristics by infant-toddler-center type, adolescents (age 18)}
% this is the top part of the tables that display the summary of the baseline characteristics, by city and child-care type
\centering
\begin{adjustbox}{width=1.2\textwidth,center=\textwidth}
\small
\begin{tabular}{m{4.0cm} cccccccccccc}
\hline \hline 
 & Reggio & Reggio & Reggio & Reggio & Parma & Parma & Parma & Parma & Padova & Padova & Padova & Padova \\
 & Municipal & Religious & Private & Not Attended & Municipal & Religious & Private & Not Attended & Municipal & Religious & Private & Not Attended \\

\hline 
  
CAPI & 0.44 & 0.56 & 0.33 & 0.39 & 0.53 & 0.70 & 0.85*** & 0.52 & 0.43 & 0.50 & & 0.50\\
  &  [0.04]  &  [0.18]  &  [0.33]  &  [0.04]  &  [0.05]  &  [0.15]  &  [0.10]  &  [0.04]  &  [0.06]  &  [0.19]  & . & [0.03] \\
Male dummy  &  0.42  &  0.44  &  0.67  &  0.41  &  0.46  &  0.60  &  0.54  &  0.41  &  0.49  &  0.62  &  . & 0.47 \\
  &  [0.04]  &  [0.18]  &  [0.33]  &  [0.04]  &  [0.05]  &  [0.16]  &  [0.14]  &  [0.04]  &  [0.06]  &  [0.18]  & . & [0.03] \\
Age  &  18.71  &  18.83  &  18.96  &  18.69  &  18.84***  &  18.86  &  18.79  &  18.71  &  18.69  &  18.74  &  . & 18.72 \\
  &  [0.03]  &  [0.06]  &  [0.05]  &  [0.03]  &  [0.03]  &  [0.13]  &  [0.11]  &  [0.03]  &  [0.04]  &  [0.11]  & . & [0.02] \\
%Age sq.  &  350.15  &  354.75  &  359.37  &  349.53  &  355.15***  &  355.92  &  353.15  &  350.11  &  349.45  &  351.26  & . & 350.42 \\
%  &  [0.99]  &  [2.16]  &  [1.90]  &  [1.18]  &  [1.23]  &  [4.83]  &  [3.96]  &  [1.12]  &  [1.67]  &  [4.11]  & . & [0.90] \\
Middle school edu mom  &  0.08  &  0.11  &  0.00  &  0.10  &  0.15*  &  0.00  &  0.00  &  0.08  &  0.08  &  0.00  & . & 0.12 \\
  &  [0.02]  &  [0.11]  &  [0.00]  &  [0.03]  &  [0.04]  &  [0.00]  &  [0.00]  &  [0.02]  &  [0.04]  &  [0.00]  & . & [0.02] \\
High school edu mom  &  0.51  &  0.44  &  0.67  &  0.45  &  0.38*  &  0.60  &  0.23*  &  0.50  &  0.39  &  0.62  & . & 0.44 \\
  &  [0.04]  &  [0.18]  &  [0.33]  &  [0.04]  &  [0.05]  &  [0.16]  &  [0.12]  &  [0.04]  &  [0.06]  &  [0.18]  & . & [0.03] \\
University edu mom  &  0.29  &  0.44  &  0.33  &  0.20*  &  0.35  &  0.20  &  0.69***  &  0.30  &  0.46**  &  0.12  & . & 0.25 \\
  &  [0.04]  &  [0.18]  &  [0.33]  &  [0.04]  &  [0.05]  &  [0.13]  &  [0.13]  &  [0.04]  &  [0.06]  &  [0.12]  & . & [0.03] \\
Middle school edu dad  &  0.05  &  0.11  &  0.00  &  0.12*  &  0.10  &  0.00  &  0.00  &  0.06  &  0.13*  &  0.00  & . & 0.10 \\
  &  [0.02]  &  [0.11]  &  [0.00]  &  [0.03]  &  [0.03]  &  [0.00]  &  [0.00]  &  [0.02]  &  [0.04]  &  [0.00]  & . & [0.02] \\
High school edu dad  &  0.42  &  0.33  &  0.67  &  0.38  &  0.35  &  0.60  &  0.15*  &  0.38  &  0.36  &  0.38  & . & 0.40 \\
  &  [0.04]  &  [0.17]  &  [0.33]  &  [0.04]  &  [0.05]  &  [0.16]  &  [0.10]  &  [0.04]  &  [0.06]  &  [0.18]  & . & [0.03] \\
University edu dad  &  0.22  &  0.44  &  0.33  &  0.13*  &  0.25  &  0.10  &  0.54**  &  0.25  &  0.30  &  0.25  & . & 0.28 \\
  &  [0.03]  &  [0.18]  &  [0.33]  &  [0.03]  &  [0.04]  &  [0.10]  &  [0.14]  &  [0.04]  &  [0.06]  &  [0.16]  & . & [0.03] \\
Mom born in province  &  0.75  &  0.67  &  0.33  &  0.61**  &  0.61**  &  0.70  &  0.92  &  0.71  &  0.67  &  0.75  & . & 0.81 \\
  &  [0.04]  &  [0.17]  &  [0.33]  &  [0.04]  &  [0.05]  &  [0.15]  &  [0.08]  &  [0.04]  &  [0.06]  &  [0.16]  & . & [0.03] \\
Dad born in province  &  0.62  &  0.78  &  0.33  &  0.55  &  0.53  &  0.80  &  0.69  &  0.64  &  0.62  &  0.50  & . & 0.78*** \\
  &  [0.04]  &  [0.15]  &  [0.33]  &  [0.04]  &  [0.05]  &  [0.13]  &  [0.13]  &  [0.04]  &  [0.06]  &  [0.19]  & . & [0.03] \\
Caregiver is religious  &  0.74  &  1.00  &  1.00  &  0.79  &  0.86**  &  0.80  &  0.85  &  0.89***  &  0.72  &  0.38**  & . & 0.76 \\
  &  [0.04]  &  [0.00]  &  [0.00]  &  [0.04]  &  [0.04]  &  [0.13]  &  [0.10]  &  [0.03]  &  [0.06]  &  [0.18]  & . & [0.03] \\
Own Home  &  0.89  &  1.00  &  1.00  &  0.78**  &  0.77**  &  0.80  &  1.00  &  0.84  &  0.80  &  0.75  & . & 0.76*** \\
  &  [0.03]  &  [0.00]  &  [0.00]  &  [0.04]  &  [0.04]  &  [0.13]  &  [0.00]  &  [0.03]  &  [0.05]  &  [0.16]  & . & [0.03] \\
Income 5k-10k eur  &  0.01  &  0.00  &  0.00  &  0.01  &  0.00  &  0.00  &  0.00  &  0.02  &  0.00  &  0.00  & . & 0.01 \\
  &  [0.01]  &  [0.00]  &  [0.00]  &  [0.01]  &  [0.00]  &  [0.00]  &  [0.00]  &  [0.01]  &  [0.00]  &  [0.00]  & . & [0.01] \\
Income 10k-25k eur  &  0.16  &  0.22  &  0.33  &  0.20  &  0.19  &  0.00  &  0.23  &  0.16  &  0.11  &  0.00  &  . & 0.10* \\
  &  [0.03]  &  [0.15]  &  [0.33]  &  [0.04]  &  [0.04]  &  [0.00]  &  [0.12]  &  [0.03]  &  [0.04]  &  [0.00]  & . & [0.02] \\
Income 25k-50k eur  &  0.33  &  0.33  &  0.33  &  0.32  &  0.31  &  0.80***  &  0.23  &  0.25  &  0.33  &  0.12  & . & 0.22** \\
  &  [0.04]  &  [0.17]  &  [0.33]  &  [0.04]  &  [0.05]  &  [0.13]  &  [0.12]  &  [0.04]  &  [0.06]  &  [0.12]  & . & [0.03] \\
Income 50k-100k eur  &  0.29  &  0.00  &  0.00  &  0.18*  &  0.28  &  0.20  &  0.38  &  0.21  &  0.15**  &  0.12  & . & 0.10*** \\
  &  [0.04]  &  [0.00]  &  [0.00]  &  [0.03]  &  [0.05]  &  [0.13]  &  [0.14]  &  [0.04]  &  [0.05]  &  [0.12]  & . & [0.02] \\
Income 100k-250k eur  &  0.05  &  0.22*  &  0.00  &  0.03  &  0.02  &  0.00  &  0.00  &  0.05  &  0.00  &  0.00  & . & 0.03 \\
  &  [0.02]  &  [0.15]  &  [0.00]  &  [0.02]  &  [0.01]  &  [0.00]  &  [0.00]  &  [0.02]  &  [0.00]  &  [0.00]  & . & [0.01] \\
%Income more 250k eur  &  0.01  &  0.00  &  0.00  &  0.00  &  0.00  &  0.00  &  0.00  &  0.00  &  0.00  &  0.00  & . & 0.00 \\
%  &  [0.01]  &  [0.00]  &  [0.00]  &  [0.00]  &  [0.00]  &  [0.00]  &  [0.00]  &  [0.00]  &  [0.00]  &  [0.00]  & . & [0.00] \\
%Income below 5k eur  &  0.00  &  0.00  &  0.00  &  0.01  &  0.03*  &  0.00  &  0.00  &  0.02*  &  0.03*  &  0.00  & . & 0.04** \\
%  &  [0.00]  &  [0.00]  &  [0.00]  &  [0.01]  &  [0.02]  &  [0.00]  &  [0.00]  &  [0.01]  &  [0.02]  &  [0.00]  & . & [0.01] \\
Low birthweight  &  0.05  &  0.11  &  0.00  &  0.05  &  0.08  &  0.00  &  0.00  &  0.06  &  0.07  &  0.00  & . & 0.04 \\
  &  [0.02]  &  [0.11]  &  [0.00]  &  [0.02]  &  [0.03]  &  [0.00]  &  [0.00]  &  [0.02]  &  [0.03]  &  [0.00]  & . & [0.01] \\
Premature  &  0.06  &  0.11  &  0.00  &  0.07  &  0.13*  &  0.00  &  0.00  &  0.09  &  0.10  &  0.12  & . & 0.06 \\
 & [0.02] & [0.11] & [0.00] & [0.02] & [0.03] & [0.00] & [0.00] & [0.03] & [0.04] & [0.12] & & [0.02]\\ \hline


% it contains the notes, assuming they are the same for all the tables.
\end{tabular}

\end{adjustbox}
\raggedright{
\footnotesize{Average of baseline characteristcs, by city and type of child-care attended. Standard errors of means in brackets. Test for difference in means between each column and the first column (Reggio Municipal, the treatment group) was performed; *** significant difference at 1\%, ** significant difference at 5\%, * significant difference at 10\%. Source: authors calculation using survey data.}
}
\end{table}
  
%\end{table}

%% adults, infant-toddler
\begin{table}[H]
\caption{Baseline characteristics by infant-toddler-center type, adults (age 30-50)}
% this is the top part of the tables that display the summary of the baseline characteristics, by city and child-care type
\centering
\begin{adjustbox}{width=1.2\textwidth,center=\textwidth}
\small
\begin{tabular}{m{4.0cm} cccccccccccc}
\hline \hline 
 & Reggio & Reggio & Reggio & Reggio & Parma & Parma & Parma & Parma & Padova & Padova & Padova & Padova \\
 & Municipal & Religious & Private & Not Attended & Municipal & Religious & Private & Not Attended & Municipal & Religious & Private & Not Attended \\

\hline 
  
CAPI & 0.69 & 0.25 & 0.33 & 0.54*** & 0.30*** & 0.53 & 0.36** & 0.37*** & 0.50** & 0.28*** & 0.00 & 0.33***\\
  &  [0.05]  &  [0.25]  &  [0.33]  &  [0.02]  &  [0.05]  &  [0.13]  &  [0.15]  &  [0.02]  &  [0.08]  &  [0.11]  &  [0.00]  &  [0.02] \\
Male dummy  &  0.62  &  0.25  &  1.00  &  0.53  &  0.44**  &  0.53  &  0.36  &  0.49**  &  0.63  &  0.50  &  0.33  &  0.49** \\
  &  [0.05]  &  [0.25]  &  [0.00]  &  [0.02]  &  [0.06]  &  [0.13]  &  [0.15]  &  [0.02]  &  [0.08]  &  [0.12]  &  [0.33]  &  [0.02] \\
Age  &  36.81  &  39.56  &  32.92  &  44.03***  &  40.21**  &  41.39  &  33.91  &  41.65***  &  38.14  &  44.54***  &  32.77  &  43.01*** \\
  &  [0.54]  &  [6.44]  &  [0.20]  &  [0.37]  &  [1.00]  &  [2.80]  &  [1.08]  &  [0.37]  &  [0.91]  &  [2.19]  &  [0.39]  &  [0.38] \\
%Age sq.  &  1382.43  &  1689.38  &  1084.10  &  2027.19***  &  1693.41**  &  1823.06  &  1161.36  &  1804.89***  &  1485.28  &  2065.43***  &  1074.13  &  1935.41*** \\
%  &  [41.05]  &  [592.43]  &  [13.06]  &  [33.21]  &  [88.03]  &  [250.51]  &  [83.17]  &  [32.88]  &  [69.12]  &  [195.13]  &  [25.30]  &  [34.33] \\
Middle school edu mom  &  0.19  &  0.25  &  0.33  &  0.19  &  0.39***  &  0.27  &  0.00  &  0.20  &  0.11  &  0.28  &  0.00  &  0.28* \\
  &  [0.04]  &  [0.25]  &  [0.33]  &  [0.02]  &  [0.06]  &  [0.12]  &  [0.00]  &  [0.02]  &  [0.05]  &  [0.11]  &  [0.00]  &  [0.02] \\
High school edu mom  &  0.43  &  0.75  &  0.00  &  0.43  &  0.21***  &  0.20  &  0.36  &  0.33*  &  0.42  &  0.22  &  0.33  &  0.31** \\
  &  [0.05]  &  [0.25]  &  [0.00]  &  [0.02]  &  [0.05]  &  [0.11]  &  [0.15]  &  [0.02]  &  [0.08]  &  [0.10]  &  [0.33]  &  [0.02] \\
University edu mom  &  0.39  &  0.00  &  0.67  &  0.37  &  0.40  &  0.47  &  0.64  &  0.46  &  0.39  &  0.50  &  0.67  &  0.40 \\
  &  [0.05]  &  [0.00]  &  [0.33]  &  [0.02]  &  [0.06]  &  [0.13]  &  [0.15]  &  [0.02]  &  [0.08]  &  [0.12]  &  [0.33]  &  [0.02] \\
Middle school edu dad  &  0.20  &  0.25  &  0.33  &  0.17  &  0.38**  &  0.27  &  0.00  &  0.20  &  0.11  &  0.28  &  0.00  &  0.22 \\
  &  [0.04]  &  [0.25]  &  [0.33]  &  [0.01]  &  [0.06]  &  [0.12]  &  [0.00]  &  [0.02]  &  [0.05]  &  [0.11]  &  [0.00]  &  [0.02] \\
High school edu dad  &  0.34  &  0.50  &  0.00  &  0.41  &  0.23  &  0.07**  &  0.18  &  0.32  &  0.34  &  0.11*  &  0.00  &  0.27 \\
  &  [0.05]  &  [0.29]  &  [0.00]  &  [0.02]  &  [0.05]  &  [0.07]  &  [0.12]  &  [0.02]  &  [0.08]  &  [0.08]  &  [0.00]  &  [0.02] \\
University edu dad  &  0.45  &  0.25  &  0.67  &  0.41  &  0.39  &  0.60  &  0.82**  &  0.47  &  0.45  &  0.61  &  1.00  &  0.51 \\
  &  [0.05]  &  [0.25]  &  [0.33]  &  [0.02]  &  [0.06]  &  [0.13]  &  [0.12]  &  [0.02]  &  [0.08]  &  [0.12]  &  [0.00]  &  [0.02] \\
Mom born in province  &  0.82  &  1.00  &  1.00  &  0.81  &  0.78  &  0.80  &  0.55**  &  0.73*  &  0.66*  &  0.56**  &  1.00  &  0.69** \\
  &  [0.04]  &  [0.00]  &  [0.00]  &  [0.02]  &  [0.05]  &  [0.11]  &  [0.16]  &  [0.02]  &  [0.08]  &  [0.12]  &  [0.00]  &  [0.02] \\
Dad born in province  &  0.89  &  1.00  &  1.00  &  0.82  &  0.84  &  0.73  &  0.55**  &  0.79**  &  0.66***  &  0.78  &  0.33**  &  0.77** \\
  &  [0.03]  &  [0.00]  &  [0.00]  &  [0.02]  &  [0.04]  &  [0.12]  &  [0.16]  &  [0.02]  &  [0.08]  &  [0.10]  &  [0.33]  &  [0.02] \\
Caregiver is religious  &  0.00  &  0.00  &  0.00  &  0.00  &  0.00  &  0.00  &  0.00  &  0.00  &  0.00  &  0.00  &  0.00  &  0.00 \\
  &  [0.00]  &  [0.00]  &  [0.00]  &  [0.00]  &  [0.00]  &  [0.00]  &  [0.00]  &  [0.00]  &  [0.00]  &  [0.00]  &  [0.00]  &  [0.00] \\
%Own Home  &  0.31  &  0.50  &  0.33  &  0.59***  &  0.68***  &  0.73***  &  0.64**  &  0.61***  &  0.87***  &  0.78***  &  0.67  &  0.66*** \\
%  &  [0.05]  &  [0.29]  &  [0.33]  &  [0.02]  &  [0.05]  &  [0.12]  &  [0.15]  &  [0.02]  &  [0.06]  &  [0.10]  &  [0.33]  &  [0.02] \\
%Income 5k-10k eur  &  0.00  &  0.00  &  0.00  &  0.00  &  0.00  &  0.00  &  0.00  &  0.00  &  0.00  &  0.00  &  0.00  &  0.00 \\
%  &  [0.00]  &  [0.00]  &  [0.00]  &  [0.00]  &  [0.00]  &  [0.00]  &  [0.00]  &  [0.00]  &  [0.00]  &  [0.00]  &  [0.00]  &  [0.00] \\
%Income 10k-25k eur  &  0.00  &  0.00  &  0.00  &  0.00  &  0.00  &  0.00  &  0.00  &  0.00  &  0.00  &  0.00  &  0.00  &  0.00 \\
%  &  [0.00]  &  [0.00]  &  [0.00]  &  [0.00]  &  [0.00]  &  [0.00]  &  [0.00]  &  [0.00]  &  [0.00]  &  [0.00]  &  [0.00]  &  [0.00] \\
%Income 25k-50k eur  &  0.00  &  0.00  &  0.00  &  0.00  &  0.00  &  0.00  &  0.00  &  0.00  &  0.00  &  0.00  &  0.00  &  0.00 \\
%  &  [0.00]  &  [0.00]  &  [0.00]  &  [0.00]  &  [0.00]  &  [0.00]  &  [0.00]  &  [0.00]  &  [0.00]  &  [0.00]  &  [0.00]  &  [0.00] \\
%Income 50k-100k eur  &  0.00  &  0.00  &  0.00  &  0.00  &  0.00  &  0.00  &  0.00  &  0.00  &  0.00  &  0.00  &  0.00  &  0.00 \\
%  &  [0.00]  &  [0.00]  &  [0.00]  &  [0.00]  &  [0.00]  &  [0.00]  &  [0.00]  &  [0.00]  &  [0.00]  &  [0.00]  &  [0.00]  &  [0.00] \\
%Income 100k-250k eur  &  0.00  &  0.00  &  0.00  &  0.00  &  0.00  &  0.00  &  0.00  &  0.00  &  0.00  &  0.00  &  0.00  &  0.00 \\
%  &  [0.00]  &  [0.00]  &  [0.00]  &  [0.00]  &  [0.00]  &  [0.00]  &  [0.00]  &  [0.00]  &  [0.00]  &  [0.00]  &  [0.00]  &  [0.00] \\
%%Income more 250k eur  &  0.00  &  0.00  &  0.00  &  0.00  &  0.00  &  0.00  &  0.00  &  0.00  &  0.00  &  0.00  &  0.00  &  0.00 \\
%%  &  [0.00]  &  [0.00]  &  [0.00]  &  [0.00]  &  [0.00]  &  [0.00]  &  [0.00]  &  [0.00]  &  [0.00]  &  [0.00]  &  [0.00]  &  [0.00] \\
%%Income below 5k eur  &  0.00  &  0.00  &  0.00  &  0.00  &  0.00  &  0.00  &  0.00  &  0.00  &  0.00  &  0.00  &  0.00  &  0.00 \\
%%  &  [0.00]  &  [0.00]  &  [0.00]  &  [0.00]  &  [0.00]  &  [0.00]  &  [0.00]  &  [0.00]  &  [0.00]  &  [0.00]  &  [0.00]  &  [0.00] \\
%Low birthweight  &  0.00  &  0.00  &  0.00  &  0.00  &  0.00  &  0.00  &  0.00  &  0.00  &  0.00  &  0.00  &  0.00  &  0.00 \\
% & [0.00] & [0.00] & [0.00] & [0.00] & [0.00] & [0.00] & [0.00] & [0.00] & [0.00] & [0.00] & [0.00] & [0.00]\\ 
\hline


% it contains the notes, assuming they are the same for all the tables.
\end{tabular}

\end{adjustbox}
\raggedright{
\footnotesize{Average of baseline characteristcs, by city and type of child-care attended. Standard errors of means in brackets. Test for difference in means between each column and the first column (Reggio Municipal, the treatment group) was performed; *** significant difference at 1\%, ** significant difference at 5\%, * significant difference at 10\%. Source: authors calculation using survey data.}
}
\end{table}
  
%\end{table}

%%% children, preschool
\begin{table}[H]
\caption{Baseline characteristics by preschool type, children (age 6)}
% this is the top part of the tables that display the summary of the baseline characteristics, by city and child-care type
\centering
\begin{adjustbox}{width=1.2\textwidth,center=\textwidth}
\small
\begin{tabular}{m{4.0cm} ccccccccccccccc}
\hline \hline 
 & Reggio & Reggio & Reggio & Reggio & Reggio & Parma & Parma & Parma & Parma & Parma & Padova & Padova & Padova & Padova & Padova \\
 & Municipal & State & Religious & Private & Not Attended & Municipal & State & Religious & Private & Not Attended & Municipal & State & Religious & Private & Not Attended \\

\hline 
  
CAPI  &  0.60  &  0.40**  &  0.55  &  0.40  &  0.50  &  0.42***  &  0.37***  &  0.44**  &  0.78  &  0.50  &  0.45**  &  0.55  &  0.48**  &  0.42  &  0.00 \\
  &  [0.04]  &  [0.07]  &  [0.05]  &  [0.24]  &  [0.50]  &  [0.04]  &  [0.07]  &  [0.06]  &  [0.15]  &  [0.22]  &  [0.06]  &  [0.08]  &  [0.04]  &  [0.15]  &  [0.00] \\
Male dummy  &  0.55  &  0.53  &  0.53  &  0.40  &  0.50  &  0.54  &  0.56  &  0.57  &  0.67  &  0.67  &  0.59  &  0.62  &  0.48  &  0.25*  &  0.50 \\
  &  [0.04]  &  [0.08]  &  [0.05]  &  [0.24]  &  [0.50]  &  [0.04]  &  [0.08]  &  [0.06]  &  [0.17]  &  [0.21]  &  [0.05]  &  [0.08]  &  [0.04]  &  [0.13]  &  [0.50] \\
Age  &  6.80  &  6.92*  &  6.74  &  6.69  &  6.93  &  6.70***  &  6.71  &  6.70**  &  6.83  &  6.80  &  6.66***  &  6.66**  &  6.67***  &  6.74  &  6.39* \\
  &  [0.03]  &  [0.05]  &  [0.04]  &  [0.18]  &  [0.16]  &  [0.03]  &  [0.05]  &  [0.04]  &  [0.09]  &  [0.14]  &  [0.04]  &  [0.06]  &  [0.03]  &  [0.10]  &  [0.04] \\
%Age sq.  &  46.39  &  47.93*  &  45.58  &  44.92  &  48.01  &  44.99***  &  45.17  &  44.96**  &  46.76  &  46.38  &  44.40***  &  44.46**  &  44.60***  &  45.51  &  40.78* \\
%  &  [0.38]  &  [0.68]  &  [0.54]  &  [2.48]  &  [2.28]  &  [0.35]  &  [0.63]  &  [0.51]  &  [1.18]  &  [1.91]  &  [0.48]  &  [0.83]  &  [0.39]  &  [1.32]  &  [0.54] \\
Middle school edu mom  &  0.08  &  0.02  &  0.11  &  0.00  &  0.00  &  0.06  &  0.02  &  0.08  &  0.00  &  0.00  &  0.06  &  0.05  &  0.12  &  0.08  &  0.00 \\
  &  [0.02]  &  [0.02]  &  [0.03]  &  [0.00]  &  [0.00]  &  [0.02]  &  [0.02]  &  [0.03]  &  [0.00]  &  [0.00]  &  [0.03]  &  [0.03]  &  [0.03]  &  [0.08]  &  [0.00] \\
High school edu mom  &  0.46  &  0.44  &  0.43  &  0.60  &  0.00  &  0.38  &  0.44  &  0.44  &  0.56  &  0.17  &  0.43  &  0.40  &  0.47  &  0.58  &  0.00 \\
  &  [0.04]  &  [0.07]  &  [0.05]  &  [0.24]  &  [0.00]  &  [0.04]  &  [0.08]  &  [0.06]  &  [0.18]  &  [0.17]  &  [0.05]  &  [0.08]  &  [0.04]  &  [0.15]  &  [0.00] \\
University edu mom  &  0.29  &  0.18  &  0.32  &  0.40  &  0.00  &  0.46***  &  0.53***  &  0.39  &  0.44  &  0.83**  &  0.44**  &  0.35  &  0.32  &  0.33  &  0.50 \\
  &  [0.04]  &  [0.06]  &  [0.05]  &  [0.24]  &  [0.00]  &  [0.04]  &  [0.08]  &  [0.06]  &  [0.18]  &  [0.17]  &  [0.06]  &  [0.08]  &  [0.04]  &  [0.14]  &  [0.50] \\
Middle school edu dad  &  0.08  &  0.11  &  0.05  &  0.00  &  0.50  &  0.08  &  0.12  &  0.13  &  0.11  &  0.00  &  0.09  &  0.05  &  0.11  &  0.00  &  0.00 \\
  &  [0.02]  &  [0.05]  &  [0.02]  &  [0.00]  &  [0.50]  &  [0.02]  &  [0.05]  &  [0.04]  &  [0.11]  &  [0.00]  &  [0.03]  &  [0.03]  &  [0.03]  &  [0.00]  &  [0.00] \\
High school edu dad  &  0.34  &  0.31  &  0.38  &  0.40  &  0.50  &  0.34  &  0.37  &  0.39  &  0.44  &  0.33  &  0.40  &  0.40  &  0.43  &  0.50  &  0.00 \\
  &  [0.04]  &  [0.07]  &  [0.05]  &  [0.24]  &  [0.50]  &  [0.04]  &  [0.07]  &  [0.06]  &  [0.18]  &  [0.21]  &  [0.05]  &  [0.08]  &  [0.04]  &  [0.15]  &  [0.00] \\
University edu dad  &  0.23  &  0.18  &  0.29  &  0.40  &  0.00  &  0.33*  &  0.44**  &  0.35*  &  0.33  &  0.33  &  0.29  &  0.30  &  0.30  &  0.42  &  0.50 \\
  &  [0.03]  &  [0.06]  &  [0.05]  &  [0.24]  &  [0.00]  &  [0.04]  &  [0.08]  &  [0.05]  &  [0.17]  &  [0.21]  &  [0.05]  &  [0.07]  &  [0.04]  &  [0.15]  &  [0.50] \\
Mom born in province  &  0.51  &  0.38  &  0.59  &  0.40  &  1.00  &  0.60  &  0.72**  &  0.56  &  0.67  &  0.50  &  0.62  &  0.70**  &  0.74***  &  0.58  &  0.00 \\
  &  [0.04]  &  [0.07]  &  [0.05]  &  [0.24]  &  [0.00]  &  [0.04]  &  [0.07]  &  [0.06]  &  [0.17]  &  [0.22]  &  [0.05]  &  [0.07]  &  [0.04]  &  [0.15]  &  [0.00] \\
Dad born in province  &  0.51  &  0.44  &  0.57  &  0.40  &  1.00  &  0.58  &  0.70**  &  0.62  &  0.33  &  0.17  &  0.61  &  0.50  &  0.70***  &  0.67  &  0.50 \\
  &  [0.04]  &  [0.07]  &  [0.05]  &  [0.24]  &  [0.00]  &  [0.04]  &  [0.07]  &  [0.06]  &  [0.17]  &  [0.17]  &  [0.05]  &  [0.08]  &  [0.04]  &  [0.14]  &  [0.50] \\
Caregiver is religious  &  0.81  &  0.89  &  0.90*  &  0.80  &  0.50  &  0.86  &  0.91  &  0.86  &  1.00  &  0.83  &  0.76  &  0.75  &  0.83  &  0.92  &  1.00 \\
  &  [0.03]  &  [0.05]  &  [0.03]  &  [0.20]  &  [0.50]  &  [0.03]  &  [0.04]  &  [0.04]  &  [0.00]  &  [0.17]  &  [0.05]  &  [0.07]  &  [0.03]  &  [0.08]  &  [0.00] \\
Own Home  &  0.54  &  0.44  &  0.73***  &  0.40  &  0.50  &  0.69***  &  0.74**  &  0.69**  &  0.78  &  0.83  &  0.63  &  0.60  &  0.68**  &  0.75  &  0.50 \\
  &  [0.04]  &  [0.07]  &  [0.05]  &  [0.24]  &  [0.50]  &  [0.04]  &  [0.07]  &  [0.05]  &  [0.15]  &  [0.17]  &  [0.05]  &  [0.08]  &  [0.04]  &  [0.13]  &  [0.50] \\
Income 5k-10k eur  &  0.01  &  0.00  &  0.02  &  0.00  &  0.00  &  0.01  &  0.00  &  0.04  &  0.00  &  0.00  &  0.05*  &  0.00  &  0.00  &  0.00  &  0.00 \\
  &  [0.01]  &  [0.00]  &  [0.02]  &  [0.00]  &  [0.00]  &  [0.01]  &  [0.00]  &  [0.02]  &  [0.00]  &  [0.00]  &  [0.02]  &  [0.00]  &  [0.00]  &  [0.00]  &  [0.00] \\
Income 10k-25k eur  &  0.17  &  0.22  &  0.16  &  0.00  &  0.00  &  0.21  &  0.21  &  0.13  &  0.00  &  0.33  &  0.13  &  0.25  &  0.16  &  0.00  &  0.00 \\
  &  [0.03]  &  [0.06]  &  [0.04]  &  [0.00]  &  [0.00]  &  [0.03]  &  [0.06]  &  [0.04]  &  [0.00]  &  [0.21]  &  [0.04]  &  [0.07]  &  [0.03]  &  [0.00]  &  [0.00] \\
Income 25k-50k eur  &  0.34  &  0.29  &  0.30  &  0.40  &  0.00  &  0.45*  &  0.37  &  0.31  &  0.44  &  0.67  &  0.34  &  0.25  &  0.33  &  0.17  &  0.00 \\
  &  [0.04]  &  [0.07]  &  [0.05]  &  [0.24]  &  [0.00]  &  [0.04]  &  [0.07]  &  [0.05]  &  [0.18]  &  [0.21]  &  [0.05]  &  [0.07]  &  [0.04]  &  [0.11]  &  [0.00] \\
Income 50k-100k eur  &  0.19  &  0.07*  &  0.27  &  0.20  &  0.00  &  0.17  &  0.26  &  0.23  &  0.11  &  0.00  &  0.17  &  0.10  &  0.13  &  0.00  &  0.00 \\
  &  [0.03]  &  [0.04]  &  [0.05]  &  [0.20]  &  [0.00]  &  [0.03]  &  [0.07]  &  [0.05]  &  [0.11]  &  [0.00]  &  [0.04]  &  [0.05]  &  [0.03]  &  [0.00]  &  [0.00] \\
Income 100k-250k eur  &  0.04  &  0.00  &  0.01  &  0.00  &  0.00  &  0.02  &  0.02  &  0.03  &  0.00  &  0.00  &  0.01  &  0.03  &  0.03  &  0.08  &  0.50* \\
  &  [0.01]  &  [0.00]  &  [0.01]  &  [0.00]  &  [0.00]  &  [0.01]  &  [0.02]  &  [0.02]  &  [0.00]  &  [0.00]  &  [0.01]  &  [0.03]  &  [0.01]  &  [0.08]  &  [0.50] \\
%Income more 250k eur  &  0.00  &  0.00  &  0.00  &  0.00  &  0.00  &  0.00  &  0.00  &  0.00  &  0.00  &  0.00  &  0.00  &  0.00  &  0.00  &  0.00  &  0.00 \\
%  &  [0.00]  &  [0.00]  &  [0.00]  &  [0.00]  &  [0.00]  &  [0.00]  &  [0.00]  &  [0.00]  &  [0.00]  &  [0.00]  &  [0.00]  &  [0.00]  &  [0.00]  &  [0.00]  &  [0.00] \\
%Income below 5k eur  &  0.00  &  0.02  &  0.03**  &  0.00  &  0.00  &  0.02  &  0.02  &  0.03*  &  0.11*  &  0.00  &  0.01  &  0.03  &  0.05***  &  0.00  &  0.00 \\
%  &  [0.00]  &  [0.02]  &  [0.02]  &  [0.00]  &  [0.00]  &  [0.01]  &  [0.02]  &  [0.02]  &  [0.11]  &  [0.00]  &  [0.01]  &  [0.03]  &  [0.02]  &  [0.00]  &  [0.00] \\
Low birthweight  &  0.10  &  0.07  &  0.05  &  0.00  &  0.00  &  0.04**  &  0.14  &  0.08  &  0.11  &  0.00  &  0.07  &  0.05  &  0.03**  &  0.08  &  0.00 \\
  &  [0.02]  &  [0.04]  &  [0.02]  &  [0.00]  &  [0.00]  &  [0.02]  &  [0.05]  &  [0.03]  &  [0.11]  &  [0.00]  &  [0.03]  &  [0.03]  &  [0.01]  &  [0.08]  &  [0.00] \\
Premature  &  0.10  &  0.09  &  0.10  &  0.00  &  0.00  &  0.03**  &  0.14  &  0.12  &  0.22  &  0.00  &  0.06  &  0.07  &  0.08  &  0.00  &  0.00 \\
  &  [0.02]  &  [0.04]  &  [0.03]  &  [0.00]  &  [0.00]  &  [0.01]  &  [0.05]  &  [0.04]  &  [0.15]  &  [0.00]  &  [0.03]  &  [0.04]  &  [0.02]  &  [0.00]  &  [0.00] \\

% it contains the notes, assuming they are the same for all the tables.
\end{tabular}

\end{adjustbox}
\raggedright{
\footnotesize{Average of baseline characteristcs, by city and type of child-care attended. Standard errors of means in brackets. Test for difference in means between each column and the first column (Reggio Municipal, the treatment group) was performed; *** significant difference at 1\%, ** significant difference at 5\%, * significant difference at 10\%. Source: authors calculation using survey data.}
}
\end{table}
  
%\end{table}
%
%%% adolescents, preschool
\begin{table}[H]
\caption{Baseline characteristics by preschool type, adolescents (age 18)}
% this is the top part of the tables that display the summary of the baseline characteristics, by city and child-care type
\centering
\begin{adjustbox}{width=1.2\textwidth,center=\textwidth}
\small
\begin{tabular}{m{4.0cm} ccccccccccccccc}
\hline \hline 
 & Reggio & Reggio & Reggio & Reggio & Reggio & Parma & Parma & Parma & Parma & Parma & Padova & Padova & Padova & Padova & Padova \\
 & Municipal & State & Religious & Private & Not Attended & Municipal & State & Religious & Private & Not Attended & Municipal & State & Religious & Private & Not Attended \\

\hline 
  
CAPI  &  0.47  &  0.41  &  0.38  &  0.33  &  0.43  &  0.53  &  0.47  &  0.59  &  1.00  &  0.50  &  0.43  &  0.55  &  0.53  &  0.33  &  . \\
  &  [0.04]  &  [0.11]  &  [0.05]  &  [0.21]  &  [0.20]  &  [0.05]  &  [0.08]  &  [0.05]  &  [0.00]  &  [0.29]  &  [0.05]  &  [0.07]  &  [0.04]  &  [0.21]  &  . \\
Male dummy  &  0.42  &  0.55  &  0.40  &  0.50  &  0.57  &  0.40  &  0.42  &  0.52  &  0.67  &  0.50  &  0.44  &  0.45  &  0.50  &  0.50  &  . \\
  &  [0.04]  &  [0.11]  &  [0.05]  &  [0.22]  &  [0.20]  &  [0.05]  &  [0.08]  &  [0.06]  &  [0.21]  &  [0.29]  &  [0.05]  &  [0.07]  &  [0.04]  &  [0.22]  &  . \\
Age  &  18.70  &  18.75  &  18.73  &  18.64  &  18.67  &  18.77  &  18.75  &  18.80*  &  18.81  &  18.59  &  18.75  &  18.82**  &  18.64*  &  18.74  &  . \\
  &  [0.03]  &  [0.07]  &  [0.03]  &  [0.14]  &  [0.17]  &  [0.03]  &  [0.05]  &  [0.04]  &  [0.17]  &  [0.08]  &  [0.04]  &  [0.05]  &  [0.03]  &  [0.18]  &  . \\
%Age sq.  &  349.95  &  351.75  &  350.76  &  347.57  &  348.77  &  352.39  &  351.55  &  353.45*  &  354.07  &  345.74  &  351.60  &  354.38**  &  347.68*  &  351.29  &  . \\
%  &  [0.99]  &  [2.69]  &  [1.27]  &  [5.27]  &  [6.49]  &  [1.21]  &  [2.04]  &  [1.36]  &  [6.50]  &  [3.14]  &  [1.39]  &  [1.88]  &  [1.07]  &  [6.84]  &  . \\
Middle school edu mom  &  0.09  &  0.09  &  0.11  &  0.00  &  0.00  &  0.09  &  0.16  &  0.10  &  0.00  &  0.00  &  0.15  &  0.11  &  0.08  &  0.17  &  . \\
  &  [0.02]  &  [0.06]  &  [0.03]  &  [0.00]  &  [0.00]  &  [0.03]  &  [0.06]  &  [0.03]  &  [0.00]  &  [0.00]  &  [0.04]  &  [0.05]  &  [0.02]  &  [0.17]  &  . \\
High school edu mom  &  0.51  &  0.36  &  0.42  &  0.67  &  0.71  &  0.44  &  0.47  &  0.45  &  0.00  &  0.75  &  0.39*  &  0.43  &  0.45  &  0.50  &  . \\
  &  [0.04]  &  [0.10]  &  [0.05]  &  [0.21]  &  [0.18]  &  [0.05]  &  [0.08]  &  [0.06]  &  [0.00]  &  [0.25]  &  [0.05]  &  [0.07]  &  [0.04]  &  [0.22]  &  . \\
University edu mom  &  0.22  &  0.27  &  0.31  &  0.17  &  0.00  &  0.37***  &  0.16  &  0.33*  &  1.00  &  0.25  &  0.31  &  0.34  &  0.27  &  0.33  &  . \\
  &  [0.03]  &  [0.10]  &  [0.05]  &  [0.17]  &  [0.00]  &  [0.05]  &  [0.06]  &  [0.05]  &  [0.00]  &  [0.25]  &  [0.05]  &  [0.07]  &  [0.04]  &  [0.21]  &  . \\
Middle school edu dad  &  0.07  &  0.09  &  0.10  &  0.33*  &  0.00  &  0.09  &  0.12  &  0.02  &  0.00  &  0.00  &  0.11  &  0.13  &  0.08  &  0.17  &  . \\
  &  [0.02]  &  [0.06]  &  [0.03]  &  [0.21]  &  [0.00]  &  [0.03]  &  [0.05]  &  [0.02]  &  [0.00]  &  [0.00]  &  [0.03]  &  [0.05]  &  [0.02]  &  [0.17]  &  . \\
High school edu dad  &  0.40  &  0.41  &  0.40  &  0.50  &  0.14  &  0.35  &  0.40  &  0.38  &  0.17  &  0.25  &  0.32  &  0.47  &  0.41  &  0.17  &  . \\
  &  [0.04]  &  [0.11]  &  [0.05]  &  [0.22]  &  [0.14]  &  [0.04]  &  [0.08]  &  [0.05]  &  [0.17]  &  [0.25]  &  [0.05]  &  [0.07]  &  [0.04]  &  [0.17]  &  . \\
University edu dad  &  0.19  &  0.09  &  0.21  &  0.00  &  0.43  &  0.24  &  0.16  &  0.32**  &  0.50*  &  0.25  &  0.31**  &  0.23  &  0.27  &  0.50*  &  . \\
  &  [0.03]  &  [0.06]  &  [0.04]  &  [0.00]  &  [0.20]  &  [0.04]  &  [0.06]  &  [0.05]  &  [0.22]  &  [0.25]  &  [0.05]  &  [0.06]  &  [0.04]  &  [0.22]  &  . \\
Mom born in province&  0.72  &  0.50**  &  0.64  &  0.83  &  0.57  &  0.68  &  0.74  &  0.66  &  0.67  &  0.25*  &  0.71  &  0.77  &  0.82*  &  1.00  & . \\
  &  [0.03]  &  [0.11]  &  [0.05]  &  [0.17]  &  [0.20]  &  [0.04]  &  [0.07]  &  [0.05]  &  [0.21]  &  [0.25]  &  [0.05]  &  [0.06]  &  [0.03]  &  [0.00]  &  . \\
Dad born in province&  0.61  &  0.45  &  0.57  &  0.67  &  0.43  &  0.58  &  0.70  &  0.63  &  0.50  &  0.00  &  0.65  &  0.72  &  0.79***  &  0.67  &  . \\
  &  [0.04]  &  [0.11]  &  [0.05]  &  [0.21]  &  [0.20]  &  [0.05]  &  [0.07]  &  [0.05]  &  [0.22]  &  [0.00]  &  [0.05]  &  [0.07]  &  [0.04]  &  [0.21]  &  . \\
Caregiver is religious  &  0.69  &  0.73  &  0.91***  &  1.00  &  0.86  &  0.88***  &  0.86**  &  0.87***  &  0.83  &  0.75  &  0.77  &  0.62  &  0.78*  &  0.50  &  . \\
  &  [0.04]  &  [0.10]  &  [0.03]  &  [0.00]  &  [0.14]  &  [0.03]  &  [0.05]  &  [0.04]  &  [0.17]  &  [0.25]  &  [0.04]  &  [0.07]  &  [0.04]  &  [0.22]  &  . \\
Own Home  &  0.86  &  0.77  &  0.85  &  0.83  &  0.43**  &  0.79  &  0.81  &  0.83  &  0.83  &  1.00  &  0.75**  &  0.64***  &  0.82  &  0.83  &  . \\
  &  [0.03]  &  [0.09]  &  [0.04]  &  [0.17]  &  [0.20]  &  [0.04]  &  [0.06]  &  [0.04]  &  [0.17]  &  [0.00]  &  [0.04]  &  [0.07]  &  [0.03]  &  [0.17]  &  . \\
Income 5k-10k eur  &  0.01  &  0.00  &  0.01  &  0.00  &  0.00  &  0.01  &  0.00  &  0.00  &  0.00  &  0.00  &  0.00  &  0.02  &  0.01  &  0.00  &  . \\
  &  [0.01]  &  [0.00]  &  [0.01]  &  [0.00]  &  [0.00]  &  [0.01]  &  [0.00]  &  [0.00]  &  [0.00]  &  [0.00]  &  [0.00]  &  [0.02]  &  [0.01]  &  [0.00]  &  . \\
Income 10k-25k eur  &  0.17  &  0.36**  &  0.17  &  0.00  &  0.29  &  0.16  &  0.28  &  0.12  &  0.50*  &  0.25  &  0.12  &  0.09  &  0.11  &  0.00  &  . \\
  &  [0.03]  &  [0.10]  &  [0.04]  &  [0.00]  &  [0.18]  &  [0.03]  &  [0.07]  &  [0.04]  &  [0.22]  &  [0.25]  &  [0.03]  &  [0.04]  &  [0.03]  &  [0.00]  &  . \\
Income 25k-50k eur  &  0.32  &  0.27  &  0.34  &  0.17  &  0.43  &  0.25  &  0.37  &  0.30  &  0.50  &  0.25  &  0.30  &  0.09***  &  0.27  &  0.17  &  . \\
  &  [0.04]  &  [0.10]  &  [0.05]  &  [0.17]  &  [0.20]  &  [0.04]  &  [0.07]  &  [0.05]  &  [0.22]  &  [0.25]  &  [0.05]  &  [0.04]  &  [0.04]  &  [0.17]  &  . \\
Income 50k-100k eur  &  0.25  &  0.09  &  0.25  &  0.33  &  0.00  &  0.26  &  0.23  &  0.26  &  0.00  &  0.00  &  0.10***  &  0.13*  &  0.11***  &  0.17  &  . \\
  &  [0.03]  &  [0.06]  &  [0.04]  &  [0.21]  &  [0.00]  &  [0.04]  &  [0.07]  &  [0.05]  &  [0.00]  &  [0.00]  &  [0.03]  &  [0.05]  &  [0.03]  &  [0.17]  &  . \\
Income 100k-250k eur  &  0.04  &  0.00  &  0.06  &  0.00  &  0.00  &  0.03  &  0.00  &  0.05  &  0.00  &  0.00  &  0.01  &  0.00  &  0.05  &  0.00  &  . \\
  &  [0.02]  &  [0.00]  &  [0.02]  &  [0.00]  &  [0.00]  &  [0.01]  &  [0.00]  &  [0.02]  &  [0.00]  &  [0.00]  &  [0.01]  &  [0.00]  &  [0.02]  &  [0.00]  &  . \\
%Income more 250k eur  &  0.00  &  0.00  &  0.01  &  0.00  &  0.00  &  0.00  &  0.00  &  0.00  &  0.00  &  0.00  &  0.00  &  0.00  &  0.00  &  0.00  &  . \\
%  &  [0.00]  &  [0.00]  &  [0.01]  &  [0.00]  &  [0.00]  &  [0.00]  &  [0.00]  &  [0.00]  &  [0.00]  &  [0.00]  &  [0.00]  &  [0.00]  &  [0.00]  &  [0.00]  &  . \\
%Income below 5k eur  &  0.00  &  0.00  &  0.01  &  0.00  &  0.00  &  0.03*  &  0.02  &  0.02  &  0.00  &  0.00  &  0.03**  &  0.02  &  0.05***  &  0.00  &  . \\
%  &  [0.00]  &  [0.00]  &  [0.01]  &  [0.00]  &  [0.00]  &  [0.01]  &  [0.02]  &  [0.02]  &  [0.00]  &  [0.00]  &  [0.02]  &  [0.02]  &  [0.02]  &  [0.00]  &  . \\
Low birthweight  &  0.05  &  0.00  &  0.05  &  0.17  &  0.14  &  0.06  &  0.05  &  0.07  &  0.00  &  0.25  &  0.06  &  0.06  &  0.02  &  0.00  &  . \\
  &  [0.02]  &  [0.00]  &  [0.02]  &  [0.17]  &  [0.14]  &  [0.02]  &  [0.03]  &  [0.03]  &  [0.00]  &  [0.25]  &  [0.03]  &  [0.04]  &  [0.01]  &  [0.00]  &  . \\
Premature  &  0.04  &  0.09  &  0.08  &  0.17  &  0.14  &  0.09*  &  0.07  &  0.12**  &  0.00  &  0.25  &  0.10  &  0.09  &  0.04  &  0.00  &  . \\
  &  [0.02]  &  [0.06]  &  [0.03]  &  [0.17]  &  [0.14]  &  [0.03]  &  [0.04]  &  [0.04]  &  [0.00]  &  [0.25]  &  [0.03]  &  [0.04]  &  [0.02]  &  [0.00]  &  . \\

\hline
% it contains the notes, assuming they are the same for all the tables.
\end{tabular}

\end{adjustbox}
\raggedright{
\footnotesize{Average of baseline characteristcs, by city and type of child-care attended. Standard errors of means in brackets. Test for difference in means between each column and the first column (Reggio Municipal, the treatment group) was performed; *** significant difference at 1\%, ** significant difference at 5\%, * significant difference at 10\%. Source: authors calculation using survey data.}
}
\end{table}
  
%\end{table}
%
%%% adults, preschool
\begin{table}[H]
\caption{Baseline characteristics by preschool type, adults (age 30-50)}
% this is the top part of the tables that display the summary of the baseline characteristics, by city and child-care type
\centering
\begin{adjustbox}{width=1.2\textwidth,center=\textwidth}
\small
\begin{tabular}{m{4.0cm} ccccccccccccccc}
\hline \hline 
 & Reggio & Reggio & Reggio & Reggio & Reggio & Parma & Parma & Parma & Parma & Parma & Padova & Padova & Padova & Padova & Padova \\
 & Municipal & State & Religious & Private & Not Attended & Municipal & State & Religious & Private & Not Attended & Municipal & State & Religious & Private & Not Attended \\

\hline 
  
CAPI  &  0.63  &  0.59  &  0.59  &  0.62  &  0.47***  &  0.25***  &  0.38***  &  0.54  &  0.50  &  0.34***  &  0.32***  &  0.56  &  0.29***  &  0.00  &  0.35*** \\
  &  [0.03]  &  [0.07]  &  [0.05]  &  [0.18]  &  [0.03]  &  [0.03]  &  [0.05]  &  [0.05]  &  [0.22]  &  [0.03]  &  [0.05]  &  [0.07]  &  [0.03]  &  [0.00]  &  [0.04] \\
Male dummy  &  0.63  &  0.48*  &  0.55  &  0.62  &  0.47***  &  0.52**  &  0.46**  &  0.48**  &  0.50  &  0.48***  &  0.45***  &  0.46**  &  0.48***  &  0.00  &  0.56 \\
  &  [0.03]  &  [0.07]  &  [0.05]  &  [0.18]  &  [0.03]  &  [0.04]  &  [0.05]  &  [0.05]  &  [0.22]  &  [0.03]  &  [0.06]  &  [0.07]  &  [0.03]  &  [0.00]  &  [0.04] \\
Age  &  38.39  &  40.01  &  42.95***  &  45.28**  &  48.17***  &  37.91*  &  38.19  &  40.17*  &  34.35*  &  45.57***  &  40.66**  &  38.86  &  42.04***  &  49.37  &  45.46*** \\
  &  [0.37]  &  [1.19]  &  [0.79]  &  [2.59]  &  [0.57]  &  [0.57]  &  [0.83]  &  [0.69]  &  [1.92]  &  [0.55]  &  [1.00]  &  [0.90]  &  [0.50]  &  [8.69]  &  [0.69] \\
%Age sq.  &  1513.95  &  1681.25  &  1919.39***  &  2097.11**  &  2411.03***  &  1489.30*  &  1516.04  &  1668.54*  &  1198.50*  &  2147.34***  &  1725.28**  &  1551.57  &  1849.30***  &  2588.00  &  2151.50*** \\
%  &  [30.48]  &  [103.92]  &  [70.72]  &  [239.00]  &  [52.25]  &  [47.83]  &  [71.17]  &  [59.28]  &  [146.55]  &  [50.96]  &  [87.94]  &  [73.52]  &  [44.44]  &  [786.18]  &  [63.12] \\
Middle school edu mom  &  0.17  &  0.26  &  0.23  &  0.38  &  0.17  &  0.20  &  0.26*  &  0.15  &  0.00  &  0.28***  &  0.33***  &  0.13  &  0.27***  &  0.33  &  0.28*** \\
  &  [0.02]  &  [0.06]  &  [0.04]  &  [0.18]  &  [0.02]  &  [0.03]  &  [0.05]  &  [0.03]  &  [0.00]  &  [0.03]  &  [0.06]  &  [0.05]  &  [0.02]  &  [0.33]  &  [0.03] \\
High school edu mom  &  0.44  &  0.34  &  0.51  &  0.50  &  0.38  &  0.25***  &  0.29**  &  0.31**  &  0.50  &  0.36*  &  0.32**  &  0.46  &  0.30***  &  0.33  &  0.27*** \\
  &  [0.03]  &  [0.06]  &  [0.05]  &  [0.19]  &  [0.03]  &  [0.03]  &  [0.05]  &  [0.04]  &  [0.22]  &  [0.03]  &  [0.05]  &  [0.07]  &  [0.03]  &  [0.33]  &  [0.03] \\
University edu mom  &  0.37  &  0.40  &  0.25**  &  0.12  &  0.43  &  0.55***  &  0.45  &  0.54***  &  0.50  &  0.34  &  0.33  &  0.37  &  0.42  &  0.33  &  0.42 \\
  &  [0.03]  &  [0.06]  &  [0.04]  &  [0.12]  &  [0.03]  &  [0.04]  &  [0.05]  &  [0.05]  &  [0.22]  &  [0.03]  &  [0.06]  &  [0.07]  &  [0.03]  &  [0.33]  &  [0.04] \\
Middle school edu dad  &  0.17  &  0.24  &  0.22  &  0.25  &  0.14  &  0.18  &  0.25  &  0.12  &  0.00  &  0.29***  &  0.25  &  0.13  &  0.22  &  0.00  &  0.20 \\
  &  [0.02]  &  [0.06]  &  [0.04]  &  [0.16]  &  [0.02]  &  [0.03]  &  [0.05]  &  [0.03]  &  [0.00]  &  [0.03]  &  [0.05]  &  [0.05]  &  [0.02]  &  [0.00]  &  [0.03] \\
High school edu dad  &  0.40  &  0.29  &  0.47  &  0.62  &  0.38  &  0.27***  &  0.31  &  0.26***  &  0.33  &  0.34  &  0.33  &  0.50  &  0.24***  &  0.33  &  0.22*** \\
  &  [0.03]  &  [0.06]  &  [0.05]  &  [0.18]  &  [0.03]  &  [0.04]  &  [0.05]  &  [0.04]  &  [0.21]  &  [0.03]  &  [0.06]  &  [0.07]  &  [0.02]  &  [0.33]  &  [0.03] \\
University edu dad  &  0.42  &  0.43  &  0.31**  &  0.12  &  0.45  &  0.55**  &  0.44  &  0.62***  &  0.67  &  0.34*  &  0.42  &  0.35  &  0.53***  &  0.33  &  0.56*** \\
  &  [0.03]  &  [0.07]  &  [0.04]  &  [0.12]  &  [0.03]  &  [0.04]  &  [0.05]  &  [0.05]  &  [0.21]  &  [0.03]  &  [0.06]  &  [0.07]  &  [0.03]  &  [0.33]  &  [0.04] \\
Mom born in the province  &  0.88  &  0.86  &  0.84  &  0.75  &  0.71***  &  0.69***  &  0.81*  &  0.76***  &  0.67  &  0.72***  &  0.63***  &  0.73***  &  0.67***  &  0.67  &  0.72*** \\
  &  [0.02]  &  [0.05]  &  [0.03]  &  [0.16]  &  [0.03]  &  [0.04]  &  [0.04]  &  [0.04]  &  [0.21]  &  [0.03]  &  [0.06]  &  [0.06]  &  [0.03]  &  [0.33]  &  [0.03] \\
Dad born in the province  &  0.88  &  0.83  &  0.82  &  0.62*  &  0.79***  &  0.79**  &  0.88  &  0.78**  &  0.67  &  0.76***  &  0.74***  &  0.77**  &  0.74***  &  0.67  &  0.80** \\
  &  [0.02]  &  [0.05]  &  [0.04]  &  [0.18]  &  [0.02]  &  [0.03]  &  [0.04]  &  [0.04]  &  [0.21]  &  [0.03]  &  [0.05]  &  [0.06]  &  [0.02]  &  [0.33]  &  [0.03] \\
%Caregiver is religious  &  0.00  &  0.00  &  0.00  &  0.00  &  0.00  &  0.00  &  0.00  &  0.00  &  0.00  &  0.00  &  0.00  &  0.00  &  0.00  &  0.00  &  0.00 \\
%  &  [0.00]  &  [0.00]  &  [0.00]  &  [0.00]  &  [0.00]  &  [0.00]  &  [0.00]  &  [0.00]  &  [0.00]  &  [0.00]  &  [0.00]  &  [0.00]  &  [0.00]  &  [0.00]  &  [0.00] \\
%Own Home  &  0.42  &  0.62***  &  0.47  &  0.50  &  0.69***  &  0.64***  &  0.58***  &  0.61***  &  0.67  &  0.63***  &  0.70***  &  0.77***  &  0.66***  &  1.00  &  0.66*** \\
%  &  [0.03]  &  [0.06]  &  [0.05]  &  [0.19]  &  [0.03]  &  [0.04]  &  [0.05]  &  [0.05]  &  [0.21]  &  [0.03]  &  [0.05]  &  [0.06]  &  [0.03]  &  [0.00]  &  [0.04] \\
%Income 5k-10k eur  &  0.00  &  0.00  &  0.00  &  0.00  &  0.00  &  0.00  &  0.00  &  0.00  &  0.00  &  0.00  &  0.00  &  0.00  &  0.00  &  0.00  &  0.00 \\
%  &  [0.00]  &  [0.00]  &  [0.00]  &  [0.00]  &  [0.00]  &  [0.00]  &  [0.00]  &  [0.00]  &  [0.00]  &  [0.00]  &  [0.00]  &  [0.00]  &  [0.00]  &  [0.00]  &  [0.00] \\
%Income 10k-25k eur  &  0.00  &  0.00  &  0.00  &  0.00  &  0.00  &  0.00  &  0.00  &  0.00  &  0.00  &  0.00  &  0.00  &  0.00  &  0.00  &  0.00  &  0.00 \\
%  &  [0.00]  &  [0.00]  &  [0.00]  &  [0.00]  &  [0.00]  &  [0.00]  &  [0.00]  &  [0.00]  &  [0.00]  &  [0.00]  &  [0.00]  &  [0.00]  &  [0.00]  &  [0.00]  &  [0.00] \\
%Income 25k-50k eur  &  0.00  &  0.00  &  0.00  &  0.00  &  0.00  &  0.00  &  0.00  &  0.00  &  0.00  &  0.00  &  0.00  &  0.00  &  0.00  &  0.00  &  0.00 \\
%  &  [0.00]  &  [0.00]  &  [0.00]  &  [0.00]  &  [0.00]  &  [0.00]  &  [0.00]  &  [0.00]  &  [0.00]  &  [0.00]  &  [0.00]  &  [0.00]  &  [0.00]  &  [0.00]  &  [0.00] \\
%Income 50k-100k eur  &  0.00  &  0.00  &  0.00  &  0.00  &  0.00  &  0.00  &  0.00  &  0.00  &  0.00  &  0.00  &  0.00  &  0.00  &  0.00  &  0.00  &  0.00 \\
%  &  [0.00]  &  [0.00]  &  [0.00]  &  [0.00]  &  [0.00]  &  [0.00]  &  [0.00]  &  [0.00]  &  [0.00]  &  [0.00]  &  [0.00]  &  [0.00]  &  [0.00]  &  [0.00]  &  [0.00] \\
%Income 100k-250k eur  &  0.00  &  0.00  &  0.00  &  0.00  &  0.00  &  0.00  &  0.00  &  0.00  &  0.00  &  0.00  &  0.00  &  0.00  &  0.00  &  0.00  &  0.00 \\
%  &  [0.00]  &  [0.00]  &  [0.00]  &  [0.00]  &  [0.00]  &  [0.00]  &  [0.00]  &  [0.00]  &  [0.00]  &  [0.00]  &  [0.00]  &  [0.00]  &  [0.00]  &  [0.00]  &  [0.00] \\
%%Income more 250k eur  &  0.00  &  0.00  &  0.00  &  0.00  &  0.00  &  0.00  &  0.00  &  0.00  &  0.00  &  0.00  &  0.00  &  0.00  &  0.00  &  0.00  &  0.00 \\
%  &  [0.00]  &  [0.00]  &  [0.00]  &  [0.00]  &  [0.00]  &  [0.00]  &  [0.00]  &  [0.00]  &  [0.00]  &  [0.00]  &  [0.00]  &  [0.00]  &  [0.00]  &  [0.00]  &  [0.00] \\
%Income below 5k eur  &  0.00  &  0.00  &  0.00  &  0.00  &  0.00  &  0.00  &  0.00  &  0.00  &  0.00  &  0.00  &  0.00  &  0.00  &  0.00  &  0.00  &  0.00 \\
%  &  [0.00]  &  [0.00]  &  [0.00]  &  [0.00]  &  [0.00]  &  [0.00]  &  [0.00]  &  [0.00]  &  [0.00]  &  [0.00]  &  [0.00]  &  [0.00]  &  [0.00]  &  [0.00]  &  [0.00] \\
%Low birthweight  &  0.00  &  0.00  &  0.00  &  0.00  &  0.00  &  0.00  &  0.00  &  0.00  &  0.00  &  0.00  &  0.00  &  0.00  &  0.00  &  0.00  &  0.00 \\
%  &  [0.00]  &  [0.00]  &  [0.00]  &  [0.00]  &  [0.00]  &  [0.00]  &  [0.00]  &  [0.00]  &  [0.00]  &  [0.00]  &  [0.00]  &  [0.00]  &  [0.00]  &  [0.00]  &  [0.00] \\

% it contains the notes, assuming they are the same for all the tables.
\end{tabular}

\end{adjustbox}
\raggedright{
\footnotesize{Average of baseline characteristcs, by city and type of child-care attended. Standard errors of means in brackets. Test for difference in means between each column and the first column (Reggio Municipal, the treatment group) was performed; *** significant difference at 1\%, ** significant difference at 5\%, * significant difference at 10\%. Source: authors calculation using survey data.}
}
\end{table}
  
%\end{table}

\doublespacing

\subsection{Regression Results}
\label{sec:OLS}

In this section, we systematically display the relation between participation in different types of early education and four important outcomes of interest: a measure of behavioral problems from the Strength and Difficulties Questionnaire (SDQ), a measure of depression from the CESD scale, the probability of reporting being in good or excellent health, and a measure of negative attitude towards migration.

Separately for each city and age group we run the following regression:

Pooling all three cities, and running the model separately for infant-toddler centers (age 0-3) and preschools (age 3-6), we consider the following:
\[ Y_{iac} = \sum_{s} \delta_{c,s} D^{s}_{iac} + \beta_{X}X_{iac} + \alpha_{c} + \varepsilon_{iac} \]

where $Y_{iac}$ is the outcome of individual $i$, of age $a$ in city $c$; 
$D^{s}_{iac}$ is a dummy for attendance of school type $s$, $s \in Mun, Rel, Priv, Stat, None$. In some specifications, private, state and religious child-cares were pooled together in a single group, defined ``other''.\footnote{Note that there are no state infant-toddler centers, and virtually every child and adolescent attended some form of preschool. See results in the appendix for the disaggregated effect of each type of preschool.};
$X_{iac}$ are baseline family characteristics of individual $i$, of age $a$ in city $c$ 
$\alpha_{c}$ are city zip-code dummies.

The treated group are the children who attended a municipal child-care in Reggio.
As mentioned before, the age groups considered are 0-3, regarding attendance to infant-toddler center, and age 3-6, for attendance to preschool. 

The following tables show the results of the estimated $\delta_{c,s}$, using first Ordinary Least Squares with an incremental set of controls (columns 1 to 3), and then using an instrumental variable approach (column 4 to 6). The first column of each table only controls for school types $D_{sc}$ and city dummies, effectively showing raw differences in means; the second column introduces controls for baseline characteristics\footnote{Zip-code fixed effects, interview mode (computer or paper), gender, age and age$^2$ of respondent, dummies for mother and father maximum level of education (middle school, high school, university), mother and father age at birth, dummies for mother and father born in the region (province), dummy for religious mother, dummy for house ownership (only for the families of children and adolescents), dummies for family income bracket of children and adolescents, dummies for low-birth weight or premature birth of children and adolescents.}; the third column interacts all of these controls with city dummies, allowing the effect of controls to vary by city.
The fourth column uses the distances to the closest municipal child-care center, the closest religious center, the closest state center, and the closest private center as instruments for participation in the Reggio Children Approach child-care; the fifth column interacts these distances with the number of siblings; the sixth column uses as IV both the distances and the estimated score from the application to the Reggio Children Approach child-care.\todo{describe score}

The coefficients in column 1 to 3 report the difference between the child-care type (state, religious, private, and none in the three cities) and the Reggio Children Approach, which is the omitted category. The coefficients in column 4 to 6 instead report the difference with respect to other child-care (state, religious or private) in Reggio Emilia.\footnote{This is done in order to instrument directly for attending the Reggio Children Approach child-care.}

\singlespacing
\setlength\tabcolsep{0.25em}
\subsubsection{Children Results: OLS and IV}
%------------------------------------------------------%
%-------------------   Children  ----------------------%
%------------------------------------------------------%
\begin{small}
%------------------- Health ----------------------%
\begin{table}[H]
\caption{Pooled: Child Health - Infant-toddler center, Child}
\begin{tabular}{lcccccc} \hline
 & (1) & (2) & (3) & (4) & (5) & (6) \\
VARIABLES &  &  &  &  &  &  \\ \hline
 &  &  &  &  &  &  \\
RCH infant-toddler &  &  &  & 0.334 & 0.407 & 0.425 \\
 &  &  &  & [0.671] & [0.479] & [0.689] \\
Reggio None ITC & 0.051 & 0.036 & 0.032 & 0.288 & 0.346 & 0.361 \\
 & [0.059] & [0.066] & [0.070] & [0.548] & [0.394] & [0.564] \\
Reggio other ITC & 0.081 & 0.088 & 0.063 &  &  &  \\
 & [0.087] & [0.086] & [0.088] &  &  &  \\
Parma Muni ITC & -0.067 & -0.129 & -13.648 & -14.408 & -14.539 & -14.569 \\
 & [0.059] & [0.087] & [15.451] & [14.614] & [14.658] & [14.743] \\
Padova Muni ITC & 0.146** & 0.187 & -21.370* & -22.132* & -22.262* & -22.292* \\
 & [0.066] & [0.125] & [12.214] & [11.678] & [11.731] & [11.832] \\
Male dummy &  & -0.061* & -0.113* & -0.118** & -0.118** & -0.119** \\
 &  & [0.035] & [0.058] & [0.056] & [0.057] & [0.057] \\
Parma other ITC & -0.051 & -0.105 & -13.627 & -14.388 & -14.519 & -14.549 \\
 & [0.093] & [0.111] & [15.458] & [14.619] & [14.664] & [14.749] \\
Padova other ITC & 0.048 & 0.119 & -21.421* & -22.183* & -22.312* & -22.343* \\
 & [0.074] & [0.121] & [12.220] & [11.684] & [11.737] & [11.838] \\
Parma None ITC & -0.040 & -0.093 & -13.593 & -14.353 & -14.484 & -14.515 \\
 & [0.065] & [0.093] & [15.460] & [14.622] & [14.666] & [14.751] \\
Padova None ITC & 0.067 & 0.109 & -21.449* & -22.211* & -22.341* & -22.371* \\
 & [0.056] & [0.109] & [12.214] & [11.679] & [11.732] & [11.833] \\
Constant & 0.669*** & 7.884 & 19.206** & 19.967** & 20.097** & 20.128** \\
 & [0.040] & [5.382] & [8.897] & [8.719] & [8.790] & [8.924] \\
 &  &  &  &  &  &  \\
Observations & 768 & 768 & 768 & 768 & 768 & 768 \\
R-squared & 0.018 & 0.085 & 0.181 & 0.158 & 0.149 & 0.146 \\
Controls & None & All & Inter &  &  &  \\
 IV &  &  &  & distance & distXsib & dist score \\ \hline
\multicolumn{7}{c}{ Robust standard errors in brackets} \\
\multicolumn{7}{c}{ *** p$<$0.01, ** p$<$0.05, * p$<$0.1} \\
\multicolumn{7}{c}{ Dependent variable: Child health is good (\%) - mom report.} \\
\end{tabular}
  
\end{table}
\begin{table}[H]
\caption{Pooled: Child Health - Preschool, Child}
\begin{tabular}{lcccccc} \hline
 & (1) & (2) & (3) & (4) & (5) & (6) \\
VARIABLES &  &  &  &  &  &  \\ \hline
 &  &  &  &  &  &  \\
RCH preschool &  &  &  & 0.567 & 0.043 & 0.470 \\
 &  &  &  & [0.348] & [0.211] & [0.329] \\
Reggio None Preschool & -0.187 & -0.171 & -0.279 & 0.078 & -0.255 & 0.017 \\
 & [0.358] & [0.390] & [0.383] & [0.419] & [0.375] & [0.410] \\
Reggio other preschool & 0.025 & 0.024 & 0.009 &  &  &  \\
 & [0.055] & [0.057] & [0.058] &  &  &  \\
Parma Muni Preschool & -0.101* & -0.148* & -16.778 & -24.512 & -17.495 & -23.223 \\
 & [0.058] & [0.081] & [15.780] & [15.978] & [14.687] & [15.624] \\
Padova Muni Preschool & 0.088 & 0.123 & -20.261 & -27.995** & -20.977* & -26.706** \\
 & [0.063] & [0.107] & [12.421] & [13.179] & [11.581] & [12.744] \\
Male dummy &  & -0.061* & -0.119** & -0.140** & -0.121** & -0.136** \\
 &  & [0.035] & [0.058] & [0.066] & [0.055] & [0.063] \\
Parma other preschool & -0.044 & -0.095 & -16.734 & -24.468 & -17.451 & -23.179 \\
 & [0.060] & [0.086] & [15.786] & [15.982] & [14.692] & [15.629] \\
Padova other preschool & 0.047 & 0.095 & -20.316 & -28.051** & -21.033* & -26.761** \\
 & [0.052] & [0.105] & [12.412] & [13.171] & [11.572] & [12.736] \\
Parma None Preschool & -0.187 & -0.256 & -16.907 & -24.641 & -17.624 & -23.352 \\
 & [0.209] & [0.251] & [15.781] & [15.979] & [14.688] & [15.625] \\
Padova None Preschool & 0.313*** & 0.351** & -20.165 & -27.900** & -20.882* & -26.610** \\
 & [0.038] & [0.164] & [12.428] & [13.184] & [11.587] & [12.750] \\
Constant & 0.687*** & 7.293 & 19.394** & 27.129*** & 20.110** & 25.838*** \\
 & [0.038] & [5.475] & [8.933] & [10.463] & [8.358] & [9.888] \\
 &  &  &  &  &  &  \\
Observations & 767 & 767 & 767 & 767 & 767 & 767 \\
R-squared & 0.018 & 0.085 & 0.180 & 0.055 & 0.179 & 0.093 \\
Controls & None & All & Inter &  &  &  \\
 IV &  &  &  & distance & distXsib & dist score \\ \hline
\multicolumn{7}{c}{ Robust standard errors in brackets} \\
\multicolumn{7}{c}{ *** p$<$0.01, ** p$<$0.05, * p$<$0.1} \\
\multicolumn{7}{c}{ Dependent variable: Child health is good (\%) - mom report.} \\
\end{tabular}

\end{table}

%------------------- SDQ ----------------------%
\begin{table}[H]
\caption{Pooled: Child SDQ Score - Infant-toddler center, Child}
\begin{tabular}{lcccccc} \hline
 & (1) & (2) & (3) & (4) & (5) & (6) \\
VARIABLES &  &  &  &  &  &  \\ \hline
 &  &  &  &  &  &  \\
RCH infant-toddler &  &  &  & -3.749 & -0.821 & -5.672 \\
 &  &  &  & [5.252] & [3.395] & [5.540] \\
Reggio None ITC & 1.350** & 1.125* & 1.179* & -1.972 & 0.376 & -3.513 \\
 & [0.554] & [0.581] & [0.637] & [4.236] & [2.758] & [4.477] \\
Reggio other ITC & 0.023 & 0.414 & 0.735 &  &  &  \\
 & [0.775] & [0.737] & [0.729] &  &  &  \\
Parma Muni ITC & 0.101 & 0.331 & 124.088 & 128.649 & 123.511 & 132.018 \\
 & [0.479] & [0.682] & [129.980] & [123.265] & [120.875] & [126.654] \\
Padova Muni ITC & 1.022 & 3.566*** & -4.330 & 0.226 & -4.908 & 3.604 \\
 & [0.701] & [1.313] & [126.610] & [120.197] & [117.735] & [123.719] \\
Male dummy &  & 0.572* & -0.049 & -0.015 & -0.048 & 0.007 \\
 &  & [0.309] & [0.521] & [0.493] & [0.484] & [0.513] \\
Parma other ITC & -0.570 & -0.290 & 123.440 & 128.001 & 122.863 & 131.370 \\
 & [0.631] & [0.736] & [129.992] & [123.275] & [120.886] & [126.664] \\
Padova other ITC & 0.911 & 3.637*** & -4.285 & 0.270 & -4.863 & 3.649 \\
 & [0.756] & [1.343] & [126.606] & [120.193] & [117.731] & [123.716] \\
Parma None ITC & 0.312 & 0.473 & 124.150 & 128.712 & 123.574 & 132.081 \\
 & [0.537] & [0.786] & [130.000] & [123.283] & [120.894] & [126.672] \\
Padova None ITC & 0.183 & 2.381** & -5.852 & -1.297 & -6.430 & 2.082 \\
 & [0.467] & [1.175] & [126.636] & [120.220] & [117.759] & [123.741] \\
Constant & 8.070*** & 109.182** & 87.202 & 82.644 & 87.782 & 79.268 \\
 & [0.316] & [49.819] & [89.550] & [86.844] & [83.417] & [91.605] \\
 &  &  &  &  &  &  \\
Observations & 769 & 769 & 769 & 769 & 769 & 769 \\
R-squared & 0.016 & 0.160 & 0.227 & 0.211 & 0.227 & 0.184 \\
Controls & None & All & Inter &  &  &  \\
 IV &  &  &  & distance & distXsib & dist score \\ \hline
\multicolumn{7}{c}{ Robust standard errors in brackets} \\
\multicolumn{7}{c}{ *** p$<$0.01, ** p$<$0.05, * p$<$0.1} \\
\multicolumn{7}{c}{ Dependent variable: SDQ score (mom rep.).} \\
\end{tabular}
  
\end{table}
\begin{table}[H]
\caption{Pooled: Child SDQ Score - Preschool, Child}
\input{../Output/iv_texMaternaChildCS_short.tex}
\end{table}

%%------------------- Like School ----------------------%
%\begin{table}[H]
%\caption{Pooled: Child likes school - Infant-toddler center, Child}
%\input{../Output/iv_texAsiloChildLS_short.tex}  
%\end{table}
%\begin{table}[H]
%\caption{Pooled: Child likes school - Preschool, Child}
%\begin{tabular}{lcccccc} \hline
 & (1) & (2) & (3) & (4) & (5) & (6) \\
VARIABLES &  &  &  &  &  &  \\ \hline
 &  &  &  &  &  &  \\
RCH preschool &  &  &  & -0.224 & -0.195 & -0.036 \\
 &  &  &  & [0.255] & [0.213] & [0.249] \\
Reggio None Preschool & -0.194 & -0.259 & -0.128 & -0.250 & -0.231 & -0.127 \\
 & [0.358] & [0.320] & [0.305] & [0.326] & [0.314] & [0.324] \\
Reggio other preschool & -0.060 & -0.069 & -0.071 &  &  &  \\
 & [0.057] & [0.059] & [0.063] &  &  &  \\
Parma Muni Preschool & 0.001 & -0.030 & 9.259 & 13.213 & 12.831 & 10.730 \\
 & [0.056] & [0.080] & [14.226] & [13.754] & [13.532] & [13.582] \\
Padova Muni Preschool & -0.314*** & -0.468*** & 12.556 & 16.509 & 16.128 & 14.028 \\
 & [0.069] & [0.114] & [14.442] & [13.946] & [13.728] & [13.760] \\
Male dummy &  & -0.227*** & -0.172*** & -0.163*** & -0.164*** & -0.169*** \\
 &  & [0.034] & [0.061] & [0.060] & [0.059] & [0.058] \\
Parma other preschool & 0.083 & 0.038 & 9.324 & 13.278 & 12.896 & 10.796 \\
 & [0.055] & [0.083] & [14.233] & [13.761] & [13.539] & [13.589] \\
Padova other preschool & -0.117** & -0.302*** & 12.697 & 16.650 & 16.269 & 14.169 \\
 & [0.055] & [0.108] & [14.434] & [13.940] & [13.721] & [13.753] \\
Parma None Preschool & -0.194 & -0.214 & 9.079 & 13.033 & 12.651 & 10.550 \\
 & [0.209] & [0.237] & [14.234] & [13.762] & [13.540] & [13.589] \\
Padova None Preschool & 0.306*** & -0.019 & 12.957 & 16.911 & 16.529 & 14.429 \\
 & [0.038] & [0.164] & [14.459] & [13.961] & [13.743] & [13.775] \\
Constant & 0.694*** & 3.653 & -3.252 & -7.205 & -6.824 & -4.724 \\
 & [0.038] & [5.667] & [9.884] & [9.949] & [9.643] & [9.697] \\
 &  &  &  &  &  &  \\
Observations & 765 & 765 & 765 & 765 & 765 & 765 \\
R-squared & 0.050 & 0.150 & 0.187 & 0.156 & 0.162 & 0.183 \\
Controls & None & All & Inter &  &  &  \\
 IV &  &  &  & distance & distXsib & dist score \\ \hline
\multicolumn{7}{c}{ Robust standard errors in brackets} \\
\multicolumn{7}{c}{ *** p$<$0.01, ** p$<$0.05, * p$<$0.1} \\
\multicolumn{7}{c}{ Dependent variable: Child likes school (\%).} \\
\end{tabular}

%\end{table}
%%------------------- Like Math ----------------------%
%\begin{table}[H]
%\caption{Pooled: Child likes math - Infant-toddler center, Child}
%\input{../Output/iv_texAsiloChildLM_short.tex}  
%\end{table}
%\begin{table}[H]
%\caption{Pooled: Child likes math - Preschool, Child}
%\input{../Output/iv_texMaternaChildLM_short.tex}
%\end{table}
%%------------------- Like Reading  ----------------------%
%\begin{table}[H]
%\caption{Pooled: Child likes reading - Infant-toddler center, Child}
%\input{../Output/iv_texAsiloChildLL_short.tex}  
%\end{table}
%\begin{table}[H]
%\caption{Pooled: Child likes reading - Preschool, Child}
%\input{../Output/iv_texMaternaChildLL_short.tex}
%\end{table}
%%------------------- Difficulties ----------------------%
%\begin{table}[H]
%\caption{Pooled: difficulties at school entrance - Infant-toddler center, Child}
%\input{../Output/iv_texAsiloChildDIFF_short.tex}  
%\end{table}
%\begin{table}[H]
%\caption{Pooled: difficulties at school entrance - Preschool, Child}
%\begin{tabular}{lcccccc} \hline
 & (1) & (2) & (3) & (4) & (5) & (6) \\
VARIABLES &  &  &  &  &  &  \\ \hline
 &  &  &  &  &  &  \\
RCH preschool &  &  &  & 1.702* & 0.053 & 1.854* \\
 &  &  &  & [0.979] & [0.601] & [0.949] \\
Reggio None Preschool & -0.299 & -0.094 & 0.352 & 1.465 & 0.439 & 1.560 \\
 & [0.720] & [0.955] & [0.917] & [1.099] & [0.930] & [1.090] \\
Reggio other preschool & -0.179 & -0.117 & -0.142 &  &  &  \\
 & [0.170] & [0.169] & [0.173] &  &  &  \\
Parma Muni Preschool & -0.051 & -0.174 & -72.473 & -94.719* & -71.055 & -96.896* \\
 & [0.169] & [0.213] & [54.256] & [55.582] & [51.024] & [55.861] \\
Padova Muni Preschool & -0.454** & -0.415 & -70.762 & -93.010* & -69.344 & -95.185** \\
 & [0.229] & [0.355] & [45.096] & [48.009] & [42.604] & [48.380] \\
Male dummy &  & -0.219** & -0.116 & -0.169 & -0.113 & -0.175 \\
 &  & [0.104] & [0.181] & [0.194] & [0.170] & [0.199] \\
Parma other preschool & -0.192 & -0.277 & -72.588 & -94.834* & -71.170 & -97.011* \\
 & [0.175] & [0.249] & [54.255] & [55.582] & [51.024] & [55.861] \\
Padova other preschool & 0.265* & 0.175 & -70.191 & -92.439* & -68.773 & -94.614* \\
 & [0.142] & [0.290] & [45.058] & [47.978] & [42.569] & [48.349] \\
Parma None Preschool & -1.299** & -1.503** & -73.765 & -96.011* & -72.347 & -98.188* \\
 & [0.592] & [0.660] & [54.241] & [55.570] & [51.011] & [55.849] \\
Padova None Preschool & 0.701*** & 1.053** & -69.407 & -91.655* & -67.989 & -93.830* \\
 & [0.113] & [0.499] & [45.075] & [47.992] & [42.585] & [48.362] \\
Constant & 4.299*** & -30.794 & 5.300 & 27.549 & 3.884 & 29.723 \\
 & [0.113] & [19.275] & [32.589] & [38.304] & [31.264] & [38.763] \\
 &  &  &  &  &  &  \\
Observations & 769 & 769 & 769 & 769 & 769 & 769 \\
R-squared & 0.029 & 0.120 & 0.181 & 0.078 & 0.180 & 0.057 \\
Controls & None & All & Inter &  &  &  \\
 IV &  &  &  & distance & distXsib & dist score \\ \hline
\multicolumn{7}{c}{ Robust standard errors in brackets} \\
\multicolumn{7}{c}{ *** p$<$0.01, ** p$<$0.05, * p$<$0.1} \\
\multicolumn{7}{c}{ Dependent variable: dv: Difficulties encountered when starting primary school.} \\
\end{tabular}

%\end{table}
%%------------------- Difficulties Seated ----------------------%
%\begin{table}[H]
%\caption{Pooled: difficulties staying seated - Infant-toddler center, Child}
%\input{../Output/iv_texAsiloChildDS_short.tex}  
%\end{table}
%\begin{table}[H]
%\caption{Pooled: difficulties staying seated - Preschool, Child}
%\begin{tabular}{lcccccc} \hline
 & (1) & (2) & (3) & (4) & (5) & (6) \\
VARIABLES &  &  &  &  &  &  \\ \hline
 &  &  &  &  &  &  \\
RCH preschool &  &  &  & -0.339 & -0.034 & -0.369 \\
 &  &  &  & [0.242] & [0.145] & [0.234] \\
Reggio None Preschool & -0.116*** & -0.121 & -0.234** & -0.455** & -0.265* & -0.473** \\
 & [0.027] & [0.090] & [0.118] & [0.202] & [0.141] & [0.198] \\
Reggio other preschool & 0.027 & 0.013 & 0.025 &  &  &  \\
 & [0.040] & [0.041] & [0.041] &  &  &  \\
Parma Muni Preschool & 0.027 & 0.017 & 12.501 & 16.978 & 12.595 & 17.402 \\
 & [0.040] & [0.052] & [13.885] & [14.058] & [13.080] & [14.070] \\
Padova Muni Preschool & 0.110* & 0.036 & 15.404 & 19.881* & 15.498 & 20.306* \\
 & [0.056] & [0.084] & [11.003] & [11.653] & [10.436] & [11.679] \\
Male dummy &  & 0.030 & 0.024 & 0.035 & 0.024 & 0.036 \\
 &  & [0.025] & [0.044] & [0.046] & [0.041] & [0.046] \\
Parma other preschool & 0.000 & -0.012 & 12.471 & 16.948 & 12.565 & 17.373 \\
 & [0.040] & [0.060] & [13.887] & [14.059] & [13.081] & [14.072] \\
Padova other preschool & -0.067** & -0.123** & 15.241 & 19.719* & 15.336 & 20.143* \\
 & [0.032] & [0.062] & [10.995] & [11.646] & [10.428] & [11.672] \\
Parma None Preschool & 0.051 & 0.061 & 12.513 & 16.990 & 12.607 & 17.415 \\
 & [0.155] & [0.171] & [13.887] & [14.059] & [13.081] & [14.072] \\
Padova None Preschool & -0.116*** & -0.259** & 15.104 & 19.582* & 15.199 & 20.006* \\
 & [0.027] & [0.113] & [11.001] & [11.651] & [10.434] & [11.677] \\
Constant & 0.116*** & 9.881** & 3.283 & -1.195 & 3.188 & -1.619 \\
 & [0.027] & [4.717] & [8.286] & [9.519] & [7.984] & [9.557] \\
 &  &  &  &  &  &  \\
Observations & 769 & 769 & 769 & 769 & 769 & 769 \\
R-squared & 0.022 & 0.098 & 0.173 & 0.096 & 0.173 & 0.081 \\
Controls & None & All & Inter &  &  &  \\
 IV &  &  &  & distance & distXsib & dist score \\ \hline
\multicolumn{7}{c}{ Robust standard errors in brackets} \\
\multicolumn{7}{c}{ *** p$<$0.01, ** p$<$0.05, * p$<$0.1} \\
\multicolumn{7}{c}{ Dependent variable: Ability to sit still in a group when asked (difficulties in primary school).} \\
\end{tabular}

%\end{table}
%%------------------- Difficulties Obey ----------------------%
%\begin{table}[H]
%\caption{Pooled: difficulties obeying - Infant-toddler center, Child}
%\input{../Output/iv_texAsiloChildDO_short.tex}  
%\end{table}
%\begin{table}[H]
%\caption{Pooled: difficulties obeying - Preschool, Child}
%\input{../Output/iv_texMaternaChildDO_short.tex}
%\end{table}
%%------------------- Difficulties Eat ----------------------%
%\begin{table}[H]
%\caption{Pooled: difficulties eating - Infant-toddler center, Child}
%\input{../Output/iv_texAsiloChildDE_short.tex}  
%\end{table}
%\begin{table}[H]
%\caption{Pooled: difficulties eating - Preschool, Child}
%\begin{tabular}{lcccccc} \hline
 & (1) & (2) & (3) & (4) & (5) & (6) \\
VARIABLES &  &  &  &  &  &  \\ \hline
 &  &  &  &  &  &  \\
RCH preschool &  &  &  & -0.389* & -0.160 & -0.390* \\
 &  &  &  & [0.229] & [0.113] & [0.222] \\
Reggio None Preschool & -0.068*** & -0.075 & -0.017 & -0.270 & -0.127 & -0.271 \\
 & [0.021] & [0.055] & [0.090] & [0.183] & [0.122] & [0.180] \\
Reggio other preschool & 0.037 & 0.025 & 0.029 &  &  &  \\
 & [0.034] & [0.036] & [0.036] &  &  &  \\
Parma Muni Preschool & -0.008 & -0.014 & 13.451 & 18.586* & 15.306 & 18.609* \\
 & [0.029] & [0.045] & [9.985] & [10.645] & [9.598] & [10.640] \\
Padova Muni Preschool & 0.031 & -0.045 & -2.367 & 2.767 & -0.513 & 2.791 \\
 & [0.041] & [0.046] & [7.057] & [8.384] & [7.006] & [8.351] \\
Male dummy &  & -0.006 & -0.038 & -0.025 & -0.033 & -0.025 \\
 &  & [0.021] & [0.035] & [0.040] & [0.034] & [0.040] \\
Parma other preschool & 0.057 & 0.025 & 13.488 & 18.622* & 15.342 & 18.646* \\
 & [0.038] & [0.049] & [9.988] & [10.647] & [9.600] & [10.643] \\
Padova other preschool & -0.007 & -0.093** & -2.399 & 2.735 & -0.545 & 2.759 \\
 & [0.028] & [0.041] & [7.063] & [8.388] & [7.011] & [8.355] \\
Parma None Preschool & 0.265 & 0.276 & 13.757 & 18.891* & 15.611 & 18.915* \\
 & [0.195] & [0.173] & [9.980] & [10.640] & [9.592] & [10.635] \\
Padova None Preschool & -0.068*** & -0.190* & -2.596 & 2.539 & -0.742 & 2.562 \\
 & [0.021] & [0.114] & [7.049] & [8.378] & [6.999] & [8.344] \\
Constant & 0.068*** & 3.946 & 0.238 & -4.897 & -1.617 & -4.920 \\
 & [0.021] & [3.576] & [4.685] & [6.803] & [5.005] & [6.770] \\
 &  &  &  &  &  &  \\
Observations & 769 & 769 & 769 & 769 & 769 & 769 \\
R-squared & 0.014 & 0.097 & 0.154 & 0.016 & 0.136 & 0.015 \\
Controls & None & All & Inter &  &  &  \\
 IV &  &  &  & distance & distXsib & dist score \\ \hline
\multicolumn{7}{c}{ Robust standard errors in brackets} \\
\multicolumn{7}{c}{ *** p$<$0.01, ** p$<$0.05, * p$<$0.1} \\
\multicolumn{7}{c}{ Dependent variable: Fussy eater (difficulties in primary school).} \\
\end{tabular}

%\end{table}
%%------------------- Difficulties Interest ----------------------%
%\begin{table}[H]
%\caption{Pooled: difficulties staying interested - Infant-toddler center, Child}
%\begin{tabular}{lcccccc} \hline
 & (1) & (2) & (3) & (4) & (5) & (6) \\
VARIABLES &  &  &  &  &  &  \\ \hline
 &  &  &  &  &  &  \\
RCH infant-toddler &  &  &  & -0.035 & -0.111 & -0.123 \\
 &  &  &  & [0.218] & [0.138] & [0.235] \\
Reggio None ITC & 0.042* & 0.044* & 0.030 & 0.003 & -0.057 & -0.067 \\
 & [0.024] & [0.026] & [0.024] & [0.165] & [0.101] & [0.180] \\
Reggio other ITC & 0.016 & 0.009 & -0.005 &  &  &  \\
 & [0.032] & [0.030] & [0.029] &  &  &  \\
Parma Muni ITC & 0.048** & 0.089** & -0.954 & -0.831 & -0.611 & -0.574 \\
 & [0.024] & [0.039] & [5.111] & [4.874] & [4.890] & [4.951] \\
Padova Muni ITC & 0.079* & 0.121** & 4.478 & 4.601 & 4.821 & 4.858 \\
 & [0.041] & [0.058] & [6.509] & [6.145] & [6.155] & [6.227] \\
Male dummy &  & 0.029* & 0.005 & 0.006 & 0.007 & 0.007 \\
 &  & [0.015] & [0.021] & [0.019] & [0.020] & [0.020] \\
Parma other ITC & -0.014 & 0.040 & -1.002 & -0.880 & -0.659 & -0.622 \\
 & [0.010] & [0.025] & [5.116] & [4.878] & [4.894] & [4.955] \\
Padova other ITC & 0.005 & 0.032 & 4.390 & 4.512 & 4.733 & 4.770 \\
 & [0.021] & [0.050] & [6.502] & [6.139] & [6.149] & [6.221] \\
Parma None ITC & 0.053* & 0.093** & -0.949 & -0.827 & -0.606 & -0.569 \\
 & [0.029] & [0.038] & [5.111] & [4.874] & [4.890] & [4.951] \\
Padova None ITC & 0.032 & 0.086* & 4.454 & 4.576 & 4.797 & 4.834 \\
 & [0.021] & [0.051] & [6.502] & [6.139] & [6.149] & [6.221] \\
Constant & 0.014 & -1.848 & -2.312 & -2.434 & -2.654 & -2.691 \\
 & [0.010] & [2.519] & [2.604] & [2.659] & [2.692] & [2.812] \\
 &  &  &  &  &  &  \\
Observations & 770 & 770 & 770 & 770 & 770 & 770 \\
R-squared & 0.014 & 0.094 & 0.134 & 0.133 & 0.125 & 0.122 \\
Controls & None & All & Inter &  &  &  \\
 IV &  &  &  & distance & distXsib & dist score \\ \hline
\multicolumn{7}{c}{ Robust standard errors in brackets} \\
\multicolumn{7}{c}{ *** p$<$0.01, ** p$<$0.05, * p$<$0.1} \\
\multicolumn{7}{c}{ Dependent variable: Lack of excitement to learn (difficulties in primary school).} \\
\end{tabular}
  
%\end{table}
%\begin{table}[H]
%\caption{Pooled: difficulties staying interested - Preschool, Child}
%\begin{tabular}{lcccccc} \hline
 & (1) & (2) & (3) & (4) & (5) & (6) \\
VARIABLES &  &  &  &  &  &  \\ \hline
 &  &  &  &  &  &  \\
RCH preschool &  &  &  & -0.009 & -0.011 & -0.034 \\
 &  &  &  & [0.076] & [0.046] & [0.081] \\
Reggio None Preschool & -0.020* & -0.063 & -0.083* & -0.093 & -0.094* & -0.109 \\
 & [0.012] & [0.039] & [0.050] & [0.066] & [0.055] & [0.069] \\
Reggio other preschool & 0.017 & 0.013 & 0.012 &  &  &  \\
 & [0.020] & [0.021] & [0.021] &  &  &  \\
Parma Muni Preschool & 0.017 & 0.048 & -0.964 & -1.027 & -1.000 & -0.668 \\
 & [0.020] & [0.031] & [4.806] & [4.629] & [4.495] & [4.652] \\
Padova Muni Preschool & 0.036 & 0.068 & 3.651 & 3.588 & 3.614 & 3.947 \\
 & [0.030] & [0.048] & [6.763] & [6.399] & [6.306] & [6.464] \\
Male dummy &  & 0.024* & -0.004 & -0.004 & -0.004 & -0.003 \\
 &  & [0.014] & [0.021] & [0.018] & [0.019] & [0.019] \\
Parma other preschool & 0.042 & 0.083** & -0.930 & -0.993 & -0.967 & -0.634 \\
 & [0.026] & [0.039] & [4.803] & [4.626] & [4.492] & [4.649] \\
Padova other preschool & 0.029 & 0.058 & 3.641 & 3.579 & 3.605 & 3.937 \\
 & [0.021] & [0.055] & [6.749] & [6.386] & [6.293] & [6.451] \\
Parma None Preschool & 0.313 & 0.345* & -0.664 & -0.727 & -0.701 & -0.368 \\
 & [0.194] & [0.209] & [4.790] & [4.615] & [4.480] & [4.639] \\
Padova None Preschool & -0.020* & 0.013 & 3.628 & 3.565 & 3.592 & 3.924 \\
 & [0.012] & [0.071] & [6.755] & [6.392] & [6.299] & [6.456] \\
Constant & 0.020* & -1.479 & -1.916 & -1.853 & -1.879 & -2.212 \\
 & [0.012] & [2.547] & [2.493] & [2.622] & [2.370] & [2.673] \\
 &  &  &  &  &  &  \\
Observations & 769 & 769 & 769 & 769 & 769 & 769 \\
R-squared & 0.020 & 0.100 & 0.141 & 0.141 & 0.141 & 0.140 \\
Controls & None & All & Inter &  &  &  \\
 IV &  &  &  & distance & distXsib & dist score \\ \hline
\multicolumn{7}{c}{ Robust standard errors in brackets} \\
\multicolumn{7}{c}{ *** p$<$0.01, ** p$<$0.05, * p$<$0.1} \\
\multicolumn{7}{c}{ Dependent variable: Lack of excitement to learn (difficulties in primary school).} \\
\end{tabular}

%\end{table}

\end{small}

\pagebreak
\subsubsection{Children Results: Propensity Score Matching}
These are the results created by Chiara using different types of propensity score matching
\begin{small}
\begin{table}[H]
\caption{Propensity Score Matching, Children (Age 6)}
\label{tab:PSM_children}
\vspace{-5mm}
\begin{center}
\begin{tabular}{ c c c c }
\hline\hline
                       & & \textbf{Child} & \textbf{Child} \\
\textbf{Specification} & & \textbf{SDQ}   & \textbf{Health} \\
\hline 
\multicolumn{4}{c}{\emph{Infant Toddler Center }}\\ [0.2em]
\hline 
OLS   & coeff.  & -0.919 & -0.328\\ [0.2em]
      & s.e.    & (0.637)  & (0.070) \\ [0.2em]
      & obs.    &  \emph{824} & \emph{824} \\ [0.2em]
PSM 1 & coeff.  & -3.304{*} & 0.012\\ [0.2em]
      & s.e.    & (1.696)  & (0.149) \\ [0.2em]
      & obs.    &  \emph{274} & \emph{274} \\ [0.2em]
 PSM 2 & coeff.  & -0.520 & 0.053\\ [0.2em]
  & s.e.  & (0.731)  & (0.093) \\ [0.2em]
  & obs.  &  \emph{277} & \emph{277} \\ [0.2em]
 PSM 3 & coeff.  & 2.006{*}{*} & -0.079\\ [0.2em]
  & s.e.  & (0.816)  & (0.082) \\ [0.2em]
  & obs.  &  \emph{277} & \emph{277} \\ [0.2em]
 PSM 4 & coeff.  & 1.094 & 0.087\\ [0.2em]
  & s.e.  & (0.758)  & (0.091) \\ [0.2em]
  & obs.  &  \emph{277} & \emph{277} \\ [0.2em]
\hline 
\multicolumn{4}{c}{\emph{Preschool}}\\ [0.2em]
\hline 
OLS  & coeff.  & -1.478{*}{*}{*} & 0.026\\ [0.2em]
  & s.e.  & (0.529)  & (0.059) \\ [0.2em]
  & obs.  &  \emph{822} & \emph{821} \\ [0.2em]
 PSM 1 & coeff.  & -0.375 & -0.054\\ [0.2em]
  & s.e.  & (1.172)  & (0.101) \\ [0.2em]
  & obs.  &  288 & \emph{288} \\ [0.2em]
 PSM 2 & coeff.  & -2.113{*}{*}{*} & 0.106\\ [0.2em]
  & s.e.  & (0.647)  & (0.080) \\ [0.2em]
  & obs.  &  \emph{304} & \emph{304} \\ [0.2em]
 PSM 3 & coeff.  & -2.407{*}{*}{*} & 0.063\\ [0.2em]
  & s.e.  & (0.719)  & (0.074) \\ [0.2em]
  & obs.  &  \emph{304} & \emph{304} \\ [0.2em]
 PSM 4 & coeff.  & -2.490{*}{*}{*} & 0.084\\ [0.2em]
  & s.e.  & (0.761)  & (0.079) \\ [0.2em]
  & obs.  &  \emph{304} & \emph{304} \\ [0.2em]
\hline 
\end{tabular}
\end{center}
\footnotesize{{\bfseries Notes:} Source: authors' calculations from the administrative data on the universe of applications to the municipal preschools of Reggio Emilia. The cells display the percentages of applicants who were 3, 4 and 5 years-old when applying to each school year. The age is calculated based on the year of birth of the child, so that for example children born any time in 2005 are considered 3-year-old if applying to the 2008-2009 School Year.}
\end{table}

\end{small}

%------------------------------------------------------% 
%------------------- Adolescents ----------------------%
%------------------------------------------------------% 
\pagebreak
\subsubsection{Adolescents Results}
%------------------- Health ----------------------%
\begin{table}[H]
\caption{Pooled: Child Health - Infant-toddler center, Adolescent}
\begin{tabular}{lcccccc} \hline
 & (1) & (2) & (3) & (4) & (5) & (6) \\
VARIABLES &  &  &  &  &  &  \\ \hline
 &  &  &  &  &  &  \\
RCH infant-toddler &  &  &  & -0.765 & -1.442* & -0.793 \\
 &  &  &  & [1.110] & [0.849] & [0.967] \\
Reggio None ITC & -0.068 & -0.077 & -0.054 & -0.765 & -1.384* & -0.790 \\
 & [0.063] & [0.068] & [0.072] & [1.021] & [0.791] & [0.890] \\
Reggio other ITC & 0.132 & 0.101 & 0.124 &  &  &  \\
 & [0.125] & [0.125] & [0.141] &  &  &  \\
Parma Muni ITC & -0.255*** & -0.242** & 92.566 & 67.471 & 40.590 & 66.695 \\
 & [0.073] & [0.099] & [133.495] & [140.716] & [128.656] & [147.854] \\
Padova Muni ITC & -0.074 & -0.077 & -29.659 & -72.886 & -87.207 & -46.942 \\
 & [0.082] & [0.124] & [122.844] & [152.250] & [123.703] & [129.691] \\
Male dummy &  & 0.059 & 0.090 & 0.090 & 0.090 & 0.090 \\
 &  & [0.038] & [0.066] & [0.063] & [0.070] & [0.063] \\
Parma other ITC & -0.054 & -0.090 & 92.700 & 67.604 & 40.724 & 66.830 \\
 & [0.119] & [0.145] & [133.479] & [140.703] & [128.643] & [147.840] \\
Padova other ITC & 0.314*** & 0.163 & -29.383 & -72.611 & -86.931 & -46.665 \\
 & [0.042] & [0.112] & [122.847] & [152.256] & [123.707] & [129.694] \\
Parma None ITC & -0.084 & -0.050 & 92.781 & 67.686 & 40.806 & 66.911 \\
 & [0.065] & [0.092] & [133.490] & [140.713] & [128.653] & [147.851] \\
Padova None ITC & 0.042 & 0.007 & -29.569 & -72.795 & -87.116 & -46.851 \\
 & [0.054] & [0.109] & [122.846] & [152.252] & [123.705] & [129.692] \\
Constant & 0.686*** & -13.042 & -42.833 & -11.938 & 11.239 & -19.108 \\
 & [0.042] & [51.359] & [89.421] & [105.029] & [91.970] & [102.109] \\
 &  &  &  &  &  &  \\
Observations & 658 & 658 & 658 & 658 & 658 & 658 \\
R-squared & 0.040 & 0.132 & 0.201 & 0.178 & 0.102 & 0.175 \\
Controls & None & All & Inter &  &  &  \\
 IV &  &  &  & distance & distXsib & dist score \\ \hline
\multicolumn{7}{c}{ Robust standard errors in brackets} \\
\multicolumn{7}{c}{ *** p$<$0.01, ** p$<$0.05, * p$<$0.1} \\
\multicolumn{7}{c}{ Dependent variable: Child health is good (\%) - mom report.} \\
\end{tabular}
  
\end{table}
\begin{table}[H]
\caption{Pooled: Child Health - Preschool, Adolescent}
\begin{tabular}{lcccccc} \hline
 & (1) & (2) & (3) & (4) & (5) & (6) \\
VARIABLES &  &  &  &  &  &  \\ \hline
 &  &  &  &  &  &  \\
RCH preschool &  &  &  & 0.325 & 0.164 & 0.219 \\
 &  &  &  & [0.313] & [0.290] & [0.290] \\
Reggio None Preschool & 0.028 & 0.035 & 0.071 & 0.258 & 0.170 & 0.204 \\
 & [0.177] & [0.180] & [0.205] & [0.256] & [0.247] & [0.251] \\
Reggio other preschool & -0.055 & -0.026 & -0.018 &  &  &  \\
 & [0.063] & [0.066] & [0.074] &  &  &  \\
Parma Muni Preschool & -0.175*** & -0.127 & 137.843 & 163.799 & 150.059 & 168.940 \\
 & [0.067] & [0.096] & [134.209] & [118.470] & [129.447] & [161.756] \\
Padova Muni Preschool & -0.039 & -0.020 & -18.708 & -8.919 & -3.235 & 33.046 \\
 & [0.070] & [0.110] & [124.709] & [103.914] & [126.337] & [172.131] \\
Male dummy &  & 0.054 & 0.078 & 0.076 & 0.076 & 0.075 \\
 &  & [0.038] & [0.067] & [0.065] & [0.062] & [0.063] \\
Parma other preschool & -0.127** & -0.080 & 137.901 & 163.856 & 150.117 & 168.998 \\
 & [0.064] & [0.090] & [134.209] & [118.470] & [129.447] & [161.756] \\
Padova other preschool & 0.047 & 0.002 & -18.691 & -8.904 & -3.218 & 33.064 \\
 & [0.054] & [0.106] & [124.712] & [103.916] & [126.340] & [172.135] \\
Parma None Preschool & 0.313*** & 0.433*** & 138.557 & 164.515 & 150.773 & 169.654 \\
 & [0.040] & [0.163] & [134.243] & [118.503] & [129.477] & [161.789] \\
Padova None Preschool & -0.687*** & -1.052*** & -19.737 & -9.950 & -4.264 & 32.019 \\
 & [0.040] & [0.111] & [124.713] & [103.917] & [126.341] & [172.136] \\
Constant & 0.687*** & -8.438 & -46.675 & -68.378 & -60.243 & -78.828 \\
 & [0.040] & [51.775] & [89.347] & [78.890] & [91.080] & [114.418] \\
 &  &  &  &  &  &  \\
Observations & 660 & 660 & 660 & 660 & 660 & 660 \\
R-squared & 0.029 & 0.126 & 0.185 & 0.156 & 0.179 & 0.173 \\
Controls & None & All & Inter &  &  &  \\
 IV &  &  &  & distance & distXsib & dist score \\ \hline
\multicolumn{7}{c}{ Robust standard errors in brackets} \\
\multicolumn{7}{c}{ *** p$<$0.01, ** p$<$0.05, * p$<$0.1} \\
\multicolumn{7}{c}{ Dependent variable: Child health is good (\%) - mom report.} \\
\end{tabular}

\end{table}

%\begin{table}[H]
%\caption{Pooled: Health - Infant-toddler center, Adolescent}
%\input{../Output/iv_texAsiloAdolH_short.tex}
%\end{table}
%\begin{table}[H]
%\caption{Pooled: Health - Preschool, Adolescent}
%\begin{tabular}{lcccccc} \hline
 & (1) & (2) & (3) & (4) & (5) & (6) \\
VARIABLES &  &  &  &  &  &  \\ \hline
 &  &  &  &  &  &  \\
RCH preschool &  &  &  & 0.049 & 0.241 & 0.026 \\
 &  &  &  & [0.274] & [0.270] & [0.240] \\
Reggio None Preschool & -0.060 & -0.020 & -0.013 & 0.054 & 0.157 & 0.042 \\
 & [0.176] & [0.151] & [0.174] & [0.211] & [0.217] & [0.197] \\
Reggio other preschool & -0.101* & -0.094 & -0.088 &  &  &  \\
 & [0.059] & [0.063] & [0.069] &  &  &  \\
Parma Muni Preschool & -0.149** & -0.129 & 51.040 & 48.548 & 60.655 & 51.737 \\
 & [0.063] & [0.090] & [126.967] & [144.205] & [138.946] & [136.193] \\
Padova Muni Preschool & -0.098 & 0.009 & -48.993 & -48.733 & -40.183 & -51.646 \\
 & [0.067] & [0.101] & [120.155] & [130.523] & [152.764] & [117.038] \\
Male dummy &  & 0.086** & 0.082 & 0.083 & 0.082 & 0.083 \\
 &  & [0.036] & [0.064] & [0.059] & [0.060] & [0.059] \\
Parma other preschool & -0.239*** & -0.221** & 50.957 & 48.465 & 60.572 & 51.654 \\
 & [0.062] & [0.089] & [126.965] & [144.203] & [138.944] & [136.191] \\
Padova other preschool & 0.031 & 0.045 & -48.939 & -48.679 & -40.129 & -51.592 \\
 & [0.049] & [0.097] & [120.161] & [130.528] & [152.770] & [117.042] \\
Parma None Preschool & 0.226*** & 0.227 & 51.391 & 48.898 & 61.005 & 52.090 \\
 & [0.036] & [0.167] & [126.984] & [144.225] & [138.963] & [136.214] \\
Padova None Preschool & -0.774*** & -1.041*** & -50.021 & -49.761 & -41.211 & -52.674 \\
 & [0.036] & [0.102] & [120.154] & [130.523] & [152.765] & [117.037] \\
Constant & 0.774*** & -6.326 & -6.400 & -4.368 & -17.153 & -3.591 \\
 & [0.036] & [48.157] & [84.431] & [94.760] & [98.631] & [87.339] \\
 &  &  &  &  &  &  \\
Observations & 659 & 659 & 659 & 659 & 659 & 659 \\
R-squared & 0.046 & 0.137 & 0.198 & 0.197 & 0.188 & 0.196 \\
Controls & None & All & Inter &  &  &  \\
 IV &  &  &  & distance & distXsib & dist score \\ \hline
\multicolumn{7}{c}{ Robust standard errors in brackets} \\
\multicolumn{7}{c}{ *** p$<$0.01, ** p$<$0.05, * p$<$0.1} \\
\multicolumn{7}{c}{ Dependent variable: Respondent health is good (\%).} \\
\end{tabular}

%\end{table}

%------------------- SDQ ----------------------%
\begin{table}[H]
\caption{Pooled: Child SDQ Score - Infant-toddler center, Adolescent}
\input{../Output/iv_texAsiloAdolCS_short.tex}
\end{table}
\begin{table}[H]
\caption{Pooled: Child SDQ Score - Preschool, Adolescent}
\input{../Output/iv_texMaternaAdolCS_short.tex}
\end{table}

%\begin{table}[H]
%\caption{Pooled: SDQ Score - Infant-toddler center, Adolescent}
%\input{../Output/iv_texAsiloAdolS_short.tex}
%\end{table}
%\begin{table}[H]
%\caption{Pooled: SDQ Score - Preschool, Adolescent}
%\begin{tabular}{lcccccc} \hline
 & (1) & (2) & (3) & (4) & (5) & (6) \\
VARIABLES &  &  &  &  &  &  \\ \hline
 &  &  &  &  &  &  \\
RCH preschool &  &  &  & 2.603 & 3.155 & 1.881 \\
 &  &  &  & [2.607] & [2.225] & [2.237] \\
Reggio None Preschool & -2.087 & -1.828 & -0.451 & 0.603 & 0.918 & 0.192 \\
 & [1.575] & [1.537] & [1.838] & [2.197] & [2.091] & [2.073] \\
Reggio other preschool & 0.802 & 0.806 & 0.761 &  &  &  \\
 & [0.602] & [0.648] & [0.706] &  &  &  \\
Parma Muni Preschool & -0.981* & -0.495 & -439.485 & -282.814 & 12.971 & -375.667 \\
 & [0.594] & [0.867] & [1,113.273] & [1,393.090] & [817.120] & [1,286.175] \\
Padova Muni Preschool & -0.301 & 1.550 & -935.656 & -628.010 & -606.389 & -921.281 \\
 & [0.651] & [1.156] & [1,185.003] & [1,181.004] & [1,027.920] & [1,459.399] \\
Male dummy &  & -0.398 & -0.519 & -0.540 & -0.546 & -0.532 \\
 &  & [0.353] & [0.660] & [0.636] & [0.644] & [0.625] \\
Parma other preschool & -0.707 & -0.365 & -439.606 & -282.935 & 12.851 & -375.788 \\
 & [0.612] & [0.837] & [1,113.256] & [1,393.076] & [817.102] & [1,286.160] \\
Padova other preschool & 0.165 & 2.475** & -935.021 & -627.372 & -605.757 & -920.653 \\
 & [0.543] & [1.115] & [1,184.994] & [1,180.995] & [1,027.914] & [1,459.404] \\
Parma None Preschool & -0.015 & 2.589* & -436.473 & -279.861 & 16.032 & -372.685 \\
 & [1.144] & [1.429] & [1,113.483] & [1,393.380] & [817.299] & [1,286.401] \\
Padova None Preschool & 3.485*** & 5.873*** & -932.010 & -624.359 & -602.747 & -917.645 \\
 & [0.409] & [1.152] & [1,185.005] & [1,181.003] & [1,027.927] & [1,459.417] \\
Constant & 9.515*** & 306.594 & 631.508 & 376.093 & 262.909 & 516.846 \\
 & [0.409] & [438.655] & [850.759] & [934.207] & [668.409] & [957.091] \\
 &  &  &  &  &  &  \\
Observations & 656 & 656 & 656 & 656 & 656 & 656 \\
R-squared & 0.019 & 0.112 & 0.219 & 0.175 & 0.159 & 0.191 \\
Controls & None & All & Inter &  &  &  \\
 IV &  &  &  & distance & distXsib & dist score \\ \hline
\multicolumn{7}{c}{ Robust standard errors in brackets} \\
\multicolumn{7}{c}{ *** p$<$0.01, ** p$<$0.05, * p$<$0.1} \\
\multicolumn{7}{c}{ Dependent variable: SDQ score (self rep.).} \\
\end{tabular}

%\end{table}
%
%------------------- Depression ----------------------%
\begin{table}[H]
\caption{Pooled: Depression - Infant-toddler center, Adolescent}
\begin{tabular}{lcccccc} \hline
 & (1) & (2) & (3) & (4) & (5) & (6) \\
VARIABLES &  &  &  &  &  &  \\ \hline
 &  &  &  &  &  &  \\
RCH infant-toddler &  &  &  & 9.947 & 0.464 & 8.825 \\
 &  &  &  & [15.607] & [12.638] & [13.395] \\
Reggio None ITC & 0.042 & 0.025 & 0.735 & 9.650 & 1.018 & 8.627 \\
 & [0.845] & [0.891] & [0.928] & [14.158] & [11.367] & [12.112] \\
Reggio other ITC & 0.498 & 1.459 & 1.568 &  &  &  \\
 & [2.622] & [2.705] & [2.718] &  &  &  \\
Parma Muni ITC & -0.808 & -0.623 & 782.210 & 1,204.802 & 848.186 & 1,094.628 \\
 & [0.832] & [1.196] & [1,601.582] & [1,482.196] & [1,517.434] & [1,374.599] \\
Padova Muni ITC & 0.141 & 0.663 & -298.870 & 91.911 & -170.668 & 36.060 \\
 & [1.137] & [1.730] & [1,619.737] & [1,489.248] & [1,444.625] & [1,482.176] \\
Male dummy &  & -1.339*** & -1.602* & -1.587* & -1.600** & -1.587* \\
 &  & [0.464] & [0.880] & [0.854] & [0.803] & [0.842] \\
Parma other ITC & -1.225 & -1.061 & 782.538 & 1,205.132 & 848.519 & 1,094.965 \\
 & [1.130] & [1.440] & [1,601.363] & [1,482.013] & [1,517.246] & [1,374.431] \\
Padova other ITC & -1.193 & 1.187 & -298.949 & 91.827 & -170.742 & 35.978 \\
 & [2.872] & [3.719] & [1,619.701] & [1,489.213] & [1,444.585] & [1,482.141] \\
Parma None ITC & -0.963 & -0.115 & 782.360 & 1,204.952 & 848.337 & 1,094.780 \\
 & [0.780] & [1.145] & [1,601.548] & [1,482.169] & [1,517.406] & [1,374.575] \\
Padova None ITC & -1.996*** & -0.922 & -300.178 & 90.603 & -171.975 & 34.753 \\
 & [0.714] & [1.512] & [1,619.817] & [1,489.318] & [1,444.693] & [1,482.249] \\
Constant & 22.593*** & 221.455 & -54.620 & -486.641 & -145.530 & -419.085 \\
 & [0.575] & [614.809] & [1,246.085] & [1,242.056] & [1,214.800] & [1,197.592] \\
 &  &  &  &  &  &  \\
Observations & 640 & 640 & 640 & 640 & 640 & 640 \\
R-squared & 0.021 & 0.139 & 0.238 & 0.186 & 0.236 & 0.196 \\
Controls & None & All & Inter &  &  &  \\
 IV &  &  &  & distance & distXsib & dist score \\ \hline
\multicolumn{7}{c}{ Robust standard errors in brackets} \\
\multicolumn{7}{c}{ *** p$<$0.01, ** p$<$0.05, * p$<$0.1} \\
\multicolumn{7}{c}{ Dependent variable: Depression score (CESD).} \\
\end{tabular}

\end{table}
\begin{table}[H]
\caption{Pooled: Depression - Preschool, Adolescent}
\input{../Output/iv_texMaternaAdolD_short.tex}
\end{table}

%------------------- Migrant Taste ----------------------%
\begin{table}[H]
\caption{Pooled: Bothered by immigrants - Infant-toddler center, Adolescent}
\input{../Output/iv_texAsiloAdolM_short.tex}
\end{table}
\begin{table}[H]
\caption{Pooled: Bothered by immigrants - Preschool, Adolescent}
\begin{tabular}{lcccccc} \hline
 & (1) & (2) & (3) & (4) & (5) & (6) \\
VARIABLES &  &  &  &  &  &  \\ \hline
 &  &  &  &  &  &  \\
RCH preschool &  &  &  & 0.023 & 0.299 & -0.009 \\
 &  &  &  & [0.293] & [0.243] & [0.277] \\
Reggio None Preschool & -0.092 & -0.129 & -0.089 & -0.114 & 0.044 & -0.122 \\
 & [0.138] & [0.136] & [0.142] & [0.209] & [0.190] & [0.206] \\
Reggio other preschool & 0.134** & 0.098 & 0.086 &  &  &  \\
 & [0.061] & [0.063] & [0.067] &  &  &  \\
Parma Muni Preschool & -0.011 & 0.021 & -91.629 & -71.702 & -34.328 & -17.279 \\
 & [0.059] & [0.079] & [108.175] & [141.585] & [114.977] & [177.324] \\
Padova Muni Preschool & -0.032 & 0.139 & 116.521 & 124.349 & 200.977 & 188.730 \\
 & [0.061] & [0.102] & [124.742] & [149.015] & [158.511] & [234.020] \\
Male dummy &  & 0.023 & 0.026 & 0.024 & 0.021 & 0.023 \\
 &  & [0.034] & [0.064] & [0.059] & [0.063] & [0.059] \\
Parma other preschool & -0.060 & -0.040 & -91.683 & -71.756 & -34.383 & -17.334 \\
 & [0.054] & [0.072] & [108.180] & [141.589] & [114.981] & [177.326] \\
Padova other preschool & 0.044 & 0.303*** & 116.672 & 124.499 & 201.130 & 188.883 \\
 & [0.053] & [0.105] & [124.747] & [149.021] & [158.517] & [234.029] \\
Parma None Preschool & 0.265 & 0.311 & -91.375 & -71.444 & -34.074 & -17.010 \\
 & [0.358] & [0.285] & [108.205] & [141.618] & [114.999] & [177.354] \\
Padova None Preschool & -0.235*** & 0.279** & 116.699 & 124.526 & 201.157 & 188.910 \\
 & [0.037] & [0.110] & [124.751] & [149.024] & [158.521] & [234.033] \\
Constant & 0.235*** & 16.863 & -22.255 & -34.308 & -78.389 & -68.083 \\
 & [0.037] & [45.618] & [87.081] & [104.232] & [99.413] & [139.106] \\
 &  &  &  &  &  &  \\
Observations & 643 & 643 & 643 & 643 & 643 & 643 \\
R-squared & 0.022 & 0.166 & 0.241 & 0.236 & 0.181 & 0.238 \\
Controls & None & All & Inter &  &  &  \\
 IV &  &  &  & distance & distXsib & dist score \\ \hline
\multicolumn{7}{c}{ Robust standard errors in brackets} \\
\multicolumn{7}{c}{ *** p$<$0.01, ** p$<$0.05, * p$<$0.1} \\
\multicolumn{7}{c}{ Dependent variable: Bothered by migrants (\%).} \\
\end{tabular}

\end{table}

%%------------------- Like School ----------------------%
%\begin{table}[H]
%\caption{Pooled: Adol likes school - Infant-toddler center, Adolescent}
%\input{../Output/iv_texAsiloAdolLS_short.tex}  
%\end{table}
%\begin{table}[H]
%\caption{Pooled: Adol likes school - Preschool, Adolescent}
%\input{../Output/iv_texMaternaAdolLS_short.tex}
%\end{table}
%%------------------- Like Math ----------------------%
%\begin{table}[H]
%\caption{Pooled: Adol likes math - Infant-toddler center, Adolescent}
%\begin{tabular}{lcccccc} \hline
 & (1) & (2) & (3) & (4) & (5) & (6) \\
VARIABLES &  &  &  &  &  &  \\ \hline
 &  &  &  &  &  &  \\
RCH infant-toddler &  &  &  & 1.972 & 1.214 & 1.222 \\
 &  &  &  & [1.248] & [0.798] & [0.968] \\
Reggio None ITC & -0.012 & 0.004 & -0.001 & 1.806 & 1.114 & 1.123 \\
 & [0.068] & [0.070] & [0.075] & [1.153] & [0.734] & [0.889] \\
Reggio other ITC & -0.001 & -0.082 & -0.102 &  &  &  \\
 & [0.158] & [0.177] & [0.185] &  &  &  \\
Parma Muni ITC & 0.189*** & 0.218** & 39.518 & 103.401 & 70.268 & 107.376 \\
 & [0.071] & [0.094] & [132.555] & [117.223] & [105.412] & [162.718] \\
Padova Muni ITC & -0.005 & 0.007 & 48.982 & 117.284 & 75.769 & 95.557 \\
 & [0.086] & [0.134] & [138.448] & [127.867] & [102.389] & [123.374] \\
Male dummy &  & 0.078* & 0.101 & 0.100 & 0.101 & 0.100 \\
 &  & [0.041] & [0.073] & [0.088] & [0.075] & [0.075] \\
Parma other ITC & -0.046 & -0.036 & 39.190 & 103.074 & 69.941 & 107.045 \\
 & [0.127] & [0.146] & [132.547] & [117.216] & [105.405] & [162.707] \\
Padova other ITC & 0.254 & 0.176 & 49.226 & 117.528 & 76.012 & 95.801 \\
 & [0.186] & [0.194] & [138.458] & [127.876] & [102.397] & [123.380] \\
Parma None ITC & 0.041 & 0.066 & 39.336 & 103.219 & 70.087 & 107.194 \\
 & [0.068] & [0.094] & [132.556] & [117.223] & [105.412] & [162.718] \\
Padova None ITC & 0.024 & 0.061 & 48.971 & 117.273 & 75.758 & 95.547 \\
 & [0.060] & [0.119] & [138.440] & [127.860] & [102.381] & [123.368] \\
Constant & 0.546*** & -56.622 & -47.359 & -119.001 & -83.731 & -100.763 \\
 & [0.046] & [53.232] & [90.272] & [87.474] & [76.060] & [99.779] \\
 &  &  &  &  &  &  \\
Observations & 639 & 639 & 639 & 639 & 639 & 639 \\
R-squared & 0.016 & 0.119 & 0.179 &  & 0.111 & 0.110 \\
Controls & None & All & Inter &  &  &  \\
 IV &  &  &  & distance & distXsib & dist score \\ \hline
\multicolumn{7}{c}{ Robust standard errors in brackets} \\
\multicolumn{7}{c}{ *** p$<$0.01, ** p$<$0.05, * p$<$0.1} \\
\multicolumn{7}{c}{ Dependent variable: Child likes math (\%).} \\
\end{tabular}
  
%\end{table}
%\begin{table}[H]
%\caption{Pooled: Adol likes math - Preschool, Adolescent}
%\input{../Output/iv_texMaternaAdolLM_short.tex}
%\end{table}
%%------------------- Like Reading  ----------------------%
%\begin{table}[H]
%\caption{Pooled: Adol likes reading - Infant-toddler center, Adolescent}
%\begin{tabular}{lcccccc} \hline
 & (1) & (2) & (3) & (4) & (5) & (6) \\
VARIABLES &  &  &  &  &  &  \\ \hline
 &  &  &  &  &  &  \\
RCH infant-toddler &  &  &  & 0.117 & 0.565 & -1.092 \\
 &  &  &  & [0.798] & [0.613] & [0.965] \\
Reggio None ITC & 0.057 & 0.062 & 0.053 & 0.175 & 0.584 & -0.931 \\
 & [0.061] & [0.064] & [0.067] & [0.738] & [0.572] & [0.892] \\
Reggio other ITC & -0.226 & -0.176 & -0.160 &  &  &  \\
 & [0.157] & [0.179] & [0.199] &  &  &  \\
Parma Muni ITC & 0.128** & 0.077 & -140.116 & -121.713 & -97.924 & -240.544* \\
 & [0.064] & [0.087] & [108.054] & [94.665] & [86.634] & [138.332] \\
Padova Muni ITC & -0.097 & -0.121 & -154.777 & -141.755 & -106.343 & -206.620* \\
 & [0.084] & [0.122] & [123.881] & [113.313] & [92.971] & [113.480] \\
Male dummy &  & -0.212*** & -0.174** & -0.174*** & -0.175*** & -0.173** \\
 &  & [0.038] & [0.069] & [0.063] & [0.064] & [0.075] \\
Parma other ITC & -0.125 & -0.166 & -140.351 & -121.950 & -98.161 & -240.773* \\
 & [0.126] & [0.139] & [108.037] & [94.650] & [86.619] & [138.313] \\
Padova other ITC & -0.681*** & -0.719*** & -155.360 & -142.337 & -106.924 & -207.201* \\
 & [0.043] & [0.125] & [123.882] & [113.314] & [92.971] & [113.481] \\
Parma None ITC & 0.093 & 0.024 & -140.188 & -121.786 & -97.997 & -240.616* \\
 & [0.061] & [0.081] & [108.049] & [94.660] & [86.629] & [138.327] \\
Padova None ITC & -0.016 & -0.055 & -154.742 & -141.719 & -106.308 & -206.585* \\
 & [0.056] & [0.108] & [123.877] & [113.309] & [92.968] & [113.478] \\
Constant & 0.681*** & -1.886 & 82.013 & 73.366 & 48.051 & 148.857 \\
 & [0.043] & [45.848] & [79.224] & [75.176] & [68.147] & [92.391] \\
 &  &  &  &  &  &  \\
Observations & 639 & 639 & 639 & 639 & 639 & 639 \\
R-squared & 0.043 & 0.168 & 0.222 & 0.222 & 0.212 & 0.124 \\
Controls & None & All & Inter &  &  &  \\
 IV &  &  &  & distance & distXsib & dist score \\ \hline
\multicolumn{7}{c}{ Robust standard errors in brackets} \\
\multicolumn{7}{c}{ *** p$<$0.01, ** p$<$0.05, * p$<$0.1} \\
\multicolumn{7}{c}{ Dependent variable: Child likes reading/italian (\%).} \\
\end{tabular}
  
%\end{table}
%\begin{table}[H]
%\caption{Pooled: Adol likes reading - Preschool, Adolescent}
%\begin{tabular}{lcccccc} \hline
 & (1) & (2) & (3) & (4) & (5) & (6) \\
VARIABLES &  &  &  &  &  &  \\ \hline
 &  &  &  &  &  &  \\
RCH preschool &  &  &  & -0.146 & -0.135 & -0.336 \\
 &  &  &  & [0.237] & [0.210] & [0.231] \\
Reggio None Preschool & 0.088 & 0.079 & 0.060 & 0.043 & 0.042 & -0.057 \\
 & [0.184] & [0.211] & [0.194] & [0.209] & [0.204] & [0.211] \\
Reggio other preschool & -0.049 & -0.091 & -0.103 &  &  &  \\
 & [0.062] & [0.064] & [0.070] &  &  &  \\
Parma Muni Preschool & 0.100* & -0.000 & -175.125 & -172.000* & -194.438** & -230.448** \\
 & [0.058] & [0.085] & [107.335] & [91.949] & [93.665] & [109.737] \\
Padova Muni Preschool & -0.089 & -0.145 & -148.308 & -128.136 & -231.351 & -216.451 \\
 & [0.071] & [0.117] & [122.186] & [92.518] & [157.649] & [138.095] \\
Male dummy &  & -0.217*** & -0.176*** & -0.169*** & -0.167*** & -0.161** \\
 &  & [0.038] & [0.067] & [0.065] & [0.064] & [0.069] \\
Parma other preschool & 0.020 & -0.064 & -175.200 & -172.075* & -194.513** & -230.523** \\
 & [0.060] & [0.082] & [107.331] & [91.946] & [93.662] & [109.733] \\
Padova other preschool & -0.076 & -0.176 & -148.368 & -128.194 & -231.414 & -216.512 \\
 & [0.056] & [0.111] & [122.197] & [92.528] & [157.660] & [138.106] \\
Parma None Preschool & -0.212 & -0.214 & -175.332 & -172.204* & -194.635** & -230.657** \\
 & [0.358] & [0.358] & [107.368] & [91.983] & [93.686] & [109.767] \\
Padova None Preschool & 0.288*** & 0.040 & -148.179 & -128.004 & -231.228 & -216.324 \\
 & [0.040] & [0.117] & [122.196] & [92.526] & [157.661] & [138.106] \\
Constant & 0.712*** & 1.702 & 87.111 & 88.739 & 125.242 & 138.378* \\
 & [0.040] & [45.206] & [78.271] & [63.698] & [82.642] & [82.110] \\
 &  &  &  &  &  &  \\
Observations & 640 & 640 & 640 & 640 & 640 & 640 \\
R-squared & 0.019 & 0.149 & 0.209 & 0.190 & 0.191 & 0.149 \\
Controls & None & All & Inter &  &  &  \\
 IV &  &  &  & distance & distXsib & dist score \\ \hline
\multicolumn{7}{c}{ Robust standard errors in brackets} \\
\multicolumn{7}{c}{ *** p$<$0.01, ** p$<$0.05, * p$<$0.1} \\
\multicolumn{7}{c}{ Dependent variable: Child likes reading/italian (\%).} \\
\end{tabular}

%\end{table}
%%------------------- Difficulties ----------------------%
%\begin{table}[H]
%\caption{Pooled: difficulties at school entrance - Infant-toddler center, Adolescent}
%\input{../Output/iv_texAsiloAdolDIFF_short.tex}  
%\end{table}
%\begin{table}[H]
%\caption{Pooled: difficulties at school entrance - Preschool, Adolescent}
%\begin{tabular}{lcccccc} \hline
 & (1) & (2) & (3) & (4) & (5) & (6) \\
VARIABLES &  &  &  &  &  &  \\ \hline
 &  &  &  &  &  &  \\
RCH preschool &  &  &  & 0.292 & 0.433 & 0.308 \\
 &  &  &  & [0.500] & [0.492] & [0.439] \\
Reggio None Preschool & -1.152* & -1.097* & -1.417** & -1.124* & -1.036 & -1.110* \\
 & [0.643] & [0.655] & [0.647] & [0.655] & [0.645] & [0.644] \\
Reggio other preschool & -0.128 & -0.211 & -0.299** &  &  &  \\
 & [0.124] & [0.130] & [0.131] &  &  &  \\
Parma Muni Preschool & -0.474*** & -0.314 & 130.139 & 138.267 & 204.275 & 166.500 \\
 & [0.164] & [0.206] & [303.736] & [319.605] & [378.743] & [347.021] \\
Padova Muni Preschool & -0.104 & -0.335 & 124.495 & 106.581 & 207.233 & 151.633 \\
 & [0.149] & [0.281] & [308.586] & [228.363] & [411.028] & [325.662] \\
Male dummy &  & -0.212** & 0.006 & 0.007 & 0.004 & 0.005 \\
 &  & [0.093] & [0.137] & [0.125] & [0.125] & [0.125] \\
Parma other preschool & -0.164 & -0.001 & 130.411 & 138.539 & 204.548 & 166.772 \\
 & [0.131] & [0.163] & [303.723] & [319.593] & [378.733] & [347.010] \\
Padova other preschool & -0.419*** & -0.582** & 124.350 & 106.435 & 207.091 & 151.489 \\
 & [0.130] & [0.295] & [308.581] & [228.357] & [411.028] & [325.659] \\
Parma None Preschool & -1.224 & -1.050 & 129.893 & 138.027 & 204.043 & 166.264 \\
 & [1.070] & [0.980] & [303.739] & [319.623] & [378.768] & [347.040] \\
Padova None Preschool & 0.276*** & -0.370 & 125.014 & 107.099 & 207.756 & 152.154 \\
 & [0.072] & [0.310] & [308.602] & [228.375] & [411.048] & [325.676] \\
Constant & 4.724*** & -64.634 & -101.247 & -97.775 & -152.704 & -120.355 \\
 & [0.072] & [133.666] & [165.229] & [156.874] & [217.072] & [189.413] \\
 &  &  &  &  &  &  \\
Observations & 661 & 661 & 661 & 661 & 661 & 661 \\
R-squared & 0.032 & 0.166 & 0.236 & 0.239 & 0.238 & 0.239 \\
Controls & None & All & Inter &  &  &  \\
 IV &  &  &  & distance & distXsib & dist score \\ \hline
\multicolumn{7}{c}{ Robust standard errors in brackets} \\
\multicolumn{7}{c}{ *** p$<$0.01, ** p$<$0.05, * p$<$0.1} \\
\multicolumn{7}{c}{ Dependent variable: dv: Difficulties encountered when starting primary school.} \\
\end{tabular}

%\end{table}
%%------------------- Difficulties Seated ----------------------%
%\begin{table}[H]
%\caption{Pooled: difficulties staying seated - Infant-toddler center, Adolescent}
%\begin{tabular}{lcccccc} \hline
 & (1) & (2) & (3) & (4) & (5) & (6) \\
VARIABLES &  &  &  &  &  &  \\ \hline
 &  &  &  &  &  &  \\
RCH infant-toddler &  &  &  & -1.527 & -0.852 & -0.961 \\
 &  &  &  & [1.057] & [0.556] & [0.740] \\
Reggio None ITC & 0.029 & 0.026 & 0.048 & -1.348 & -0.730 & -0.829 \\
 & [0.026] & [0.028] & [0.030] & [0.974] & [0.511] & [0.680] \\
Reggio other ITC & -0.025* & 0.001 & -0.010 &  &  &  \\
 & [0.014] & [0.031] & [0.023] &  &  &  \\
Parma Muni ITC & 0.045 & 0.004 & 53.691 & -10.047 & 15.012 & 17.765 \\
 & [0.033] & [0.044] & [78.477] & [76.218] & [62.754] & [78.887] \\
Padova Muni ITC & 0.077* & 0.090 & 13.908 & -46.752 & -21.981 & -21.274 \\
 & [0.046] & [0.071] & [68.960] & [74.593] & [70.726] & [58.440] \\
Male dummy &  & 0.038* & 0.006 & 0.005 & 0.005 & 0.005 \\
 &  & [0.021] & [0.025] & [0.048] & [0.033] & [0.035] \\
Parma other ITC & 0.133 & 0.079 & 53.798 & -9.938 & 15.121 & 17.873 \\
 & [0.085] & [0.098] & [78.457] & [76.203] & [62.738] & [78.868] \\
Padova other ITC & 0.175 & 0.177 & 13.963 & -46.697 & -21.926 & -21.218 \\
 & [0.181] & [0.144] & [68.951] & [74.586] & [70.718] & [58.432] \\
Parma None ITC & 0.077** & 0.038 & 53.705 & -10.032 & 15.027 & 17.779 \\
 & [0.034] & [0.039] & [78.480] & [76.220] & [62.757] & [78.889] \\
Padova None ITC & 0.056** & 0.068 & 13.887 & -46.772 & -22.001 & -21.294 \\
 & [0.025] & [0.065] & [68.957] & [74.591] & [70.724] & [58.438] \\
Constant & 0.025* & -1.783 & -27.729 & 33.735 & 8.172 & 8.844 \\
 & [0.014] & [31.537] & [32.805] & [47.097] & [34.397] & [36.085] \\
 &  &  &  &  &  &  \\
Observations & 659 & 659 & 659 & 659 & 659 & 659 \\
R-squared & 0.017 & 0.115 & 0.172 &  & 0.026 &  \\
Controls & None & All & Inter &  &  &  \\
 IV &  &  &  & distance & distXsib & dist score \\ \hline
\multicolumn{7}{c}{ Robust standard errors in brackets} \\
\multicolumn{7}{c}{ *** p$<$0.01, ** p$<$0.05, * p$<$0.1} \\
\multicolumn{7}{c}{ Dependent variable: Ability to sit still in a group when asked (difficulties in primary school).} \\
\end{tabular}
  
%\end{table}
%\begin{table}[H]
%\caption{Pooled: difficulties staying seated - Preschool, Adolescent}
%\begin{tabular}{lcccccc} \hline
 & (1) & (2) & (3) & (4) & (5) & (6) \\
VARIABLES &  &  &  &  &  &  \\ \hline
 &  &  &  &  &  &  \\
RCH preschool &  &  &  & -0.149 & -0.095 & -0.071 \\
 &  &  &  & [0.093] & [0.095] & [0.070] \\
Reggio None Preschool & 0.120 & 0.134 & 0.174 & 0.072 & 0.102 & 0.115 \\
 & [0.134] & [0.143] & [0.146] & [0.148] & [0.145] & [0.145] \\
Reggio other preschool & 0.026 & 0.032 & 0.048* &  &  &  \\
 & [0.025] & [0.026] & [0.026] &  &  &  \\
Parma Muni Preschool & 0.103*** & 0.061 & 47.775 & 45.520 & 52.472 & 53.477 \\
 & [0.038] & [0.049] & [73.683] & [77.967] & [87.756] & [81.185] \\
Padova Muni Preschool & 0.048 & 0.091 & -0.270 & -5.291 & 1.644 & 4.830 \\
 & [0.033] & [0.062] & [64.605] & [47.239] & [84.302] & [66.341] \\
Male dummy &  & 0.039* & 0.002 & 0.002 & 0.002 & 0.001 \\
 &  & [0.021] & [0.026] & [0.024] & [0.023] & [0.024] \\
Parma other preschool & 0.048* & 0.015 & 47.733 & 45.478 & 52.430 & 53.435 \\
 & [0.029] & [0.034] & [73.677] & [77.962] & [87.751] & [81.180] \\
Padova other preschool & 0.075*** & 0.098 & -0.284 & -5.305 & 1.630 & 4.816 \\
 & [0.027] & [0.066] & [64.601] & [47.235] & [84.299] & [66.337] \\
Parma None Preschool & -0.022* & -0.079 & 47.584 & 45.332 & 52.284 & 53.289 \\
 & [0.013] & [0.093] & [73.685] & [77.973] & [87.763] & [81.190] \\
Padova None Preschool & -0.022* & 0.059 & -0.400 & -5.421 & 1.514 & 4.700 \\
 & [0.013] & [0.072] & [64.607] & [47.241] & [84.306] & [66.343] \\
Constant & 0.022* & -2.949 & -22.818 & -17.169 & -23.175 & -25.305 \\
 & [0.013] & [30.188] & [27.903] & [27.006] & [37.154] & [31.214] \\
 &  &  &  &  &  &  \\
Observations & 661 & 661 & 661 & 661 & 661 & 661 \\
R-squared & 0.018 & 0.119 & 0.180 & 0.175 & 0.182 & 0.182 \\
Controls & None & All & Inter &  &  &  \\
 IV &  &  &  & distance & distXsib & dist score \\ \hline
\multicolumn{7}{c}{ Robust standard errors in brackets} \\
\multicolumn{7}{c}{ *** p$<$0.01, ** p$<$0.05, * p$<$0.1} \\
\multicolumn{7}{c}{ Dependent variable: Ability to sit still in a group when asked (difficulties in primary school).} \\
\end{tabular}

%\end{table}
%%------------------- Difficulties Obey ----------------------%
%\begin{table}[H]
%\caption{Pooled: difficulties obeying - Infant-toddler center, Adolescent}
%\begin{tabular}{lcccccc} \hline
 & (1) & (2) & (3) & (4) & (5) & (6) \\
VARIABLES &  &  &  &  &  &  \\ \hline
 &  &  &  &  &  &  \\
RCH infant-toddler &  &  &  & 0.214 & 0.061 & -0.343 \\
 &  &  &  & [0.438] & [0.213] & [0.496] \\
Reggio None ITC & 0.021 & 0.010 & 0.021 & 0.222 & 0.082 & -0.287 \\
 & [0.027] & [0.028] & [0.028] & [0.405] & [0.209] & [0.454] \\
Reggio other ITC & -0.033** & -0.037 & -0.064* &  &  &  \\
 & [0.016] & [0.035] & [0.038] &  &  &  \\
Parma Muni ITC & 0.050 & 0.051 & -74.764 & -66.574 & -65.997 & -93.283 \\
 & [0.037] & [0.043] & [58.823] & [55.837] & [48.508] & [61.560] \\
Padova Muni ITC & 0.028 & 0.112 & -42.630 & -40.336 & -44.506 & -53.840 \\
 & [0.038] & [0.077] & [60.151] & [60.628] & [60.062] & [54.395] \\
Male dummy &  & 0.031 & 0.021 & 0.021 & 0.020 & 0.020 \\
 &  & [0.020] & [0.031] & [0.029] & [0.029] & [0.031] \\
Parma other ITC & 0.072 & 0.074 & -74.695 & -66.506 & -65.929 & -93.214 \\
 & [0.073] & [0.092] & [58.796] & [55.816] & [48.485] & [61.537] \\
Padova other ITC & -0.033** & 0.013 & -42.710 & -40.416 & -44.586 & -53.920 \\
 & [0.016] & [0.056] & [60.168] & [60.643] & [60.078] & [54.408] \\
Parma None ITC & 0.038 & 0.047 & -74.775 & -66.586 & -66.008 & -93.294 \\
 & [0.031] & [0.043] & [58.824] & [55.838] & [48.508] & [61.562] \\
Padova None ITC & 0.019 & 0.105 & -42.655 & -40.362 & -44.532 & -53.866 \\
 & [0.024] & [0.067] & [60.164] & [60.639] & [60.074] & [54.404] \\
Constant & 0.033** & 3.018 & 39.643 & 34.123 & 37.613 & 54.889 \\
 & [0.016] & [24.692] & [44.033] & [45.474] & [41.248] & [46.796] \\
 &  &  &  &  &  &  \\
Observations & 659 & 659 & 659 & 659 & 659 & 659 \\
R-squared & 0.007 & 0.101 & 0.210 & 0.204 & 0.210 & 0.169 \\
Controls & None & All & Inter &  &  &  \\
 IV &  &  &  & distance & distXsib & dist score \\ \hline
\multicolumn{7}{c}{ Robust standard errors in brackets} \\
\multicolumn{7}{c}{ *** p$<$0.01, ** p$<$0.05, * p$<$0.1} \\
\multicolumn{7}{c}{ Dependent variable: Ability to obey rules and directions (difficulties in primary school).} \\
\end{tabular}
  
%\end{table}
%\begin{table}[H]
%\caption{Pooled: difficulties obeying - Preschool, Adolescent}
%\begin{tabular}{lcccccc} \hline
 & (1) & (2) & (3) & (4) & (5) & (6) \\
VARIABLES &  &  &  &  &  &  \\ \hline
 &  &  &  &  &  &  \\
RCH preschool &  &  &  & 0.176 & 0.065 & 0.031 \\
 &  &  &  & [0.137] & [0.110] & [0.117] \\
Reggio None Preschool & 0.113 & 0.087 & 0.103 & 0.190 & 0.124 & 0.108 \\
 & [0.134] & [0.142] & [0.139] & [0.145] & [0.141] & [0.136] \\
Reggio other preschool & 0.009 & 0.015 & 0.019 &  &  &  \\
 & [0.024] & [0.027] & [0.030] &  &  &  \\
Parma Muni Preschool & 0.072** & 0.072 & -95.062 & -84.959 & -122.589 & -113.898 \\
 & [0.036] & [0.048] & [58.707] & [61.024] & [75.268] & [70.291] \\
Padova Muni Preschool & 0.026 & 0.105 & -72.883 & -48.846 & -103.475 & -86.535 \\
 & [0.031] & [0.069] & [61.766] & [46.909] & [83.461] & [68.733] \\
Male dummy &  & 0.034* & 0.026 & 0.025 & 0.026 & 0.026 \\
 &  & [0.020] & [0.027] & [0.028] & [0.026] & [0.025] \\
Parma other preschool & 0.040 & 0.047 & -95.069 & -84.967 & -122.597 & -113.906 \\
 & [0.030] & [0.043] & [58.705] & [61.021] & [75.266] & [70.289] \\
Padova other preschool & 0.029 & 0.115* & -72.873 & -48.836 & -103.467 & -86.526 \\
 & [0.024] & [0.069] & [61.776] & [46.917] & [83.471] & [68.742] \\
Parma None Preschool & -0.030** & -0.046 & -95.211 & -85.112 & -122.746 & -114.053 \\
 & [0.015] & [0.107] & [58.738] & [61.056] & [75.298] & [70.321] \\
Padova None Preschool & -0.030** & 0.081 & -72.978 & -48.940 & -103.572 & -86.631 \\
 & [0.015] & [0.066] & [61.776] & [46.917] & [83.472] & [68.742] \\
Constant & 0.030** & -3.987 & 48.092 & 31.036 & 62.576 & 56.071 \\
 & [0.015] & [25.485] & [42.992] & [40.502] & [53.478] & [49.988] \\
 &  &  &  &  &  &  \\
Observations & 661 & 661 & 661 & 661 & 661 & 661 \\
R-squared & 0.011 & 0.103 & 0.198 & 0.148 & 0.188 & 0.195 \\
Controls & None & All & Inter &  &  &  \\
 IV &  &  &  & distance & distXsib & dist score \\ \hline
\multicolumn{7}{c}{ Robust standard errors in brackets} \\
\multicolumn{7}{c}{ *** p$<$0.01, ** p$<$0.05, * p$<$0.1} \\
\multicolumn{7}{c}{ Dependent variable: Ability to obey rules and directions (difficulties in primary school).} \\
\end{tabular}

%\end{table}
%%------------------- Difficulties Eat ----------------------%
%\begin{table}[H]
%\caption{Pooled: difficulties eating - Infant-toddler center, Adolescent}
%\input{../Output/iv_texAsiloAdolDE_short.tex}  
%\end{table}
%\begin{table}[H]
%\caption{Pooled: difficulties eating - Preschool, Adolescent}
%\begin{tabular}{lcccccc} \hline
 & (1) & (2) & (3) & (4) & (5) & (6) \\
VARIABLES &  &  &  &  &  &  \\ \hline
 &  &  &  &  &  &  \\
RCH preschool &  &  &  & -0.078 & -0.161 & -0.047 \\
 &  &  &  & [0.204] & [0.227] & [0.183] \\
Reggio None Preschool & -0.067*** & -0.075* & -0.030 & -0.079 & -0.131 & -0.066 \\
 & [0.022] & [0.043] & [0.063] & [0.131] & [0.150] & [0.123] \\
Reggio other preschool & 0.000 & 0.007 & 0.016 &  &  &  \\
 & [0.033] & [0.035] & [0.034] &  &  &  \\
Parma Muni Preschool & 0.069 & 0.017 & -84.231 & -95.199 & -139.119 & -110.569 \\
 & [0.043] & [0.050] & [85.239] & [93.734] & [116.233] & [103.162] \\
Padova Muni Preschool & -0.039 & -0.118*** & -78.527 & -71.480 & -138.900 & -99.266 \\
 & [0.029] & [0.043] & [68.599] & [57.434] & [106.671] & [81.898] \\
Male dummy &  & -0.026 & -0.006 & -0.006 & -0.004 & -0.005 \\
 &  & [0.022] & [0.036] & [0.033] & [0.034] & [0.033] \\
Parma other preschool & 0.023 & -0.016 & -84.251 & -95.219 & -139.139 & -110.589 \\
 & [0.036] & [0.050] & [85.229] & [93.725] & [116.225] & [103.153] \\
Padova other preschool & 0.030 & -0.070* & -78.503 & -71.456 & -138.878 & -99.243 \\
 & [0.032] & [0.042] & [68.605] & [57.438] & [106.677] & [81.903] \\
Parma None Preschool & -0.067*** & -0.135* & -84.546 & -95.519 & -139.443 & -110.891 \\
 & [0.022] & [0.077] & [85.269] & [93.766] & [116.264] & [103.191] \\
Padova None Preschool & -0.067*** & -0.044 & -78.463 & -71.415 & -138.839 & -99.203 \\
 & [0.022] & [0.048] & [68.610] & [57.441] & [106.681] & [81.907] \\
Constant & 0.067*** & 26.836 & 67.355 & 70.603 & 106.923 & 82.508 \\
 & [0.022] & [30.482] & [51.711] & [55.118] & [77.302] & [64.636] \\
 &  &  &  &  &  &  \\
Observations & 661 & 661 & 661 & 661 & 661 & 661 \\
R-squared & 0.013 & 0.097 & 0.190 & 0.186 & 0.167 & 0.189 \\
Controls & None & All & Inter &  &  &  \\
 IV &  &  &  & distance & distXsib & dist score \\ \hline
\multicolumn{7}{c}{ Robust standard errors in brackets} \\
\multicolumn{7}{c}{ *** p$<$0.01, ** p$<$0.05, * p$<$0.1} \\
\multicolumn{7}{c}{ Dependent variable: Fussy eater (difficulties in primary school).} \\
\end{tabular}

%\end{table}
%%------------------- Difficulties Interest ----------------------%
%\begin{table}[H]
%\caption{Pooled: difficulties staying interested - Infant-toddler center, Adolescent}
%\input{../Output/iv_texAsiloAdolDI_short.tex}  
%\end{table}
%\begin{table}[H]
%\caption{Pooled: difficulties staying interested - Preschool, Adolescent}
%\begin{tabular}{lcccccc} \hline
 & (1) & (2) & (3) & (4) & (5) & (6) \\
VARIABLES &  &  &  &  &  &  \\ \hline
 &  &  &  &  &  &  \\
RCH preschool &  &  &  & -0.017 & -0.068 & -0.034 \\
 &  &  &  & [0.070] & [0.074] & [0.075] \\
Reggio None Preschool & 0.256 & 0.246 & 0.281 & 0.253 & 0.226 & 0.244 \\
 & [0.173] & [0.173] & [0.171] & [0.166] & [0.165] & [0.162] \\
Reggio other preschool & 0.009 & 0.026 & 0.039 &  &  &  \\
 & [0.024] & [0.026] & [0.029] &  &  &  \\
Parma Muni Preschool & 0.004 & 0.016 & -32.602 & -37.681 & -38.084 & -34.975 \\
 & [0.024] & [0.035] & [37.201] & [37.072] & [51.018] & [44.520] \\
Padova Muni Preschool & -0.016 & -0.015 & 23.288 & 18.101 & 21.977 & 20.432 \\
 & [0.020] & [0.045] & [52.873] & [38.971] & [70.639] & [55.981] \\
Male dummy &  & 0.024 & 0.002 & 0.002 & 0.002 & 0.002 \\
 &  & [0.015] & [0.030] & [0.027] & [0.027] & [0.027] \\
Parma other preschool & -0.030** & -0.034 & -32.649 & -37.728 & -38.131 & -35.022 \\
 & [0.015] & [0.030] & [37.207] & [37.078] & [51.023] & [44.525] \\
Padova other preschool & 0.029 & 0.043 & 23.347 & 18.160 & 22.036 & 20.491 \\
 & [0.024] & [0.057] & [52.878] & [38.975] & [70.644] & [55.985] \\
Parma None Preschool & 0.470 & 0.497 & -32.163 & -37.244 & -37.646 & -34.537 \\
 & [0.356] & [0.366] & [37.215] & [37.087] & [51.032] & [44.534] \\
Padova None Preschool & -0.030** & 0.019 & 23.280 & 18.093 & 21.969 & 20.423 \\
 & [0.015] & [0.054] & [52.884] & [38.980] & [70.649] & [55.989] \\
Constant & 0.030** & 9.515 & 10.611 & 12.914 & 13.599 & 12.000 \\
 & [0.015] & [18.402] & [33.327] & [30.034] & [41.980] & [36.516] \\
 &  &  &  &  &  &  \\
Observations & 661 & 661 & 661 & 661 & 661 & 661 \\
R-squared & 0.048 & 0.135 & 0.216 & 0.215 & 0.213 & 0.215 \\
Controls & None & All & Inter &  &  &  \\
 IV &  &  &  & distance & distXsib & dist score \\ \hline
\multicolumn{7}{c}{ Robust standard errors in brackets} \\
\multicolumn{7}{c}{ *** p$<$0.01, ** p$<$0.05, * p$<$0.1} \\
\multicolumn{7}{c}{ Dependent variable: Lack of excitement to learn (difficulties in primary school).} \\
\end{tabular}

%\end{table}

%------------------------------------------------------% 
%-------------------   Adults    ----------------------%
%------------------------------------------------------% 
\pagebreak
\subsubsection{Adults Results}
%------------------- Health ----------------------%
\begin{table}[H]
\caption{Pooled: Health - Infant-toddler center, Adult}
\begin{tabular}{lcccccc} \hline
 & (1) & (2) & (3) & (4) & (5) & (6) \\
VARIABLES &  &  &  &  &  &  \\ \hline
 &  &  &  &  &  &  \\
RCH infant-toddler &  &  &  & -4.607 & -2.541 & -9.087 \\
 &  &  &  & [4.977] & [2.370] & [6.877] \\
Reggio None ITC & -0.222*** & -0.053* & -0.026 & -4.336 & -2.401 & -8.518 \\
 & [0.032] & [0.031] & [0.032] & [4.682] & [2.232] & [6.518] \\
Reggio other ITC & 0.073*** & 0.111 & 0.151 &  &  &  \\
 & [0.027] & [0.090] & [0.103] &  &  &  \\
Parma Muni ITC & -0.677*** & -0.303*** & 1.850*** & -1.658 & -0.051 & -5.034 \\
 & [0.059] & [0.076] & [0.632] & [3.960] & [1.952] & [5.638] \\
Padova Muni ITC & -0.251*** & -0.162 & 2.140*** & -1.368 & 0.375 & -4.744 \\
 & [0.085] & [0.110] & [0.701] & [3.971] & [1.981] & [5.651] \\
Male dummy &  & 0.039** & 0.028 & 0.043 & 0.040 & 0.058 \\
 &  & [0.019] & [0.027] & [0.043] & [0.031] & [0.069] \\
Parma other ITC & -0.504*** & -0.337*** & 1.783*** & -1.726 & -0.119 & -5.102 \\
 & [0.101] & [0.112] & [0.633] & [3.960] & [1.952] & [5.638] \\
Padova other ITC & -0.490*** & -0.356** & 1.912*** & -1.597 & 0.134 & -4.973 \\
 & [0.127] & [0.146] & [0.706] & [3.972] & [1.981] & [5.652] \\
Parma None ITC & -0.342*** & -0.133** & 2.026*** & -1.483 & 0.124 & -4.859 \\
 & [0.035] & [0.051] & [0.631] & [3.959] & [1.951] & [5.638] \\
Padova None ITC & -0.380*** & -0.222*** & 2.038*** & -1.471 & 0.259 & -4.847 \\
 & [0.034] & [0.080] & [0.696] & [3.970] & [1.979] & [5.650] \\
Constant & 0.927*** & 0.760*** & -0.491 & 3.018 & 1.411 & 6.394 \\
 & [0.027] & [0.284] & [0.396] & [3.930] & [1.891] & [5.619] \\
 &  &  &  &  &  &  \\
Observations & 1,899 & 1,899 & 1,899 & 1,899 & 1,865 & 1,899 \\
R-squared & 0.065 & 0.302 & 0.323 & 0.038 & 0.245 &  \\
Controls & None & All & Inter &  &  &  \\
 IV &  &  &  & distance & distXsib & dist score \\ \hline
\multicolumn{7}{c}{ Robust standard errors in brackets} \\
\multicolumn{7}{c}{ *** p$<$0.01, ** p$<$0.05, * p$<$0.1} \\
\multicolumn{7}{c}{ Dependent variable: Respondent health is good (\%).} \\
\end{tabular}

\end{table}
\begin{table}[H]
\caption{Pooled: Health - Preschool, Adult}
\input{../Output/iv_texMaternaAdultH_short.tex}
\end{table}

%------------------- Depression ----------------------%
\begin{table}[H]
\caption{Pooled: Depression - Infant-toddler center, Adult}
\begin{tabular}{lcccccc} \hline
 & (1) & (2) & (3) & (4) & (5) & (6) \\
VARIABLES &  &  &  &  &  &  \\ \hline
 &  &  &  &  &  &  \\
RCH infant-toddler &  &  &  & -37.069 & 40.511 & 6.682 \\
 &  &  &  & [57.562] & [35.943] & [46.571] \\
Reggio None ITC & 0.725 & -0.074 & 0.205 & -34.201 & 38.084 & 6.637 \\
 & [0.594] & [0.614] & [0.633] & [53.841] & [33.865] & [43.405] \\
Reggio other ITC & -1.567 & -2.778 & -2.928 &  &  &  \\
 & [1.868] & [2.006] & [1.812] &  &  &  \\
Parma Muni ITC & -3.487*** & -0.963 & 1.007 & -26.168 & 34.495 & 6.760 \\
 & [0.803] & [1.119] & [8.286] & [45.023] & [29.079] & [35.803] \\
Padova Muni ITC & 5.307*** & 3.733** & 14.080 & -13.094 & 47.494 & 19.833 \\
 & [1.066] & [1.594] & [8.809] & [45.118] & [29.290] & [36.120] \\
Male dummy &  & -0.678*** & -0.666 & -0.537 & -0.801* & -0.678 \\
 &  & [0.252] & [0.411] & [0.526] & [0.476] & [0.423] \\
Parma other ITC & -1.512 & -0.428 & 1.414 & -25.761 & 34.902 & 7.167 \\
 & [1.227] & [1.451] & [8.314] & [45.027] & [29.086] & [35.811] \\
Padova other ITC & 1.344 & 1.125 & 11.777 & -15.397 & 45.178 & 17.531 \\
 & [1.507] & [1.907] & [8.955] & [45.145] & [29.329] & [36.143] \\
Parma None ITC & 0.089 & 0.617 & 2.615 & -24.559 & 36.103 & 8.369 \\
 & [0.603] & [0.869] & [8.330] & [45.030] & [29.090] & [35.826] \\
Padova None ITC & -0.108 & -0.727 & 9.615 & -17.559 & 42.989 & 15.369 \\
 & [0.600] & [1.302] & [8.741] & [45.105] & [29.266] & [36.104] \\
Constant & 21.281*** & 33.911*** & 28.358*** & 55.533 & -5.130 & 22.605 \\
 & [0.548] & [3.547] & [6.060] & [44.681] & [28.546] & [35.388] \\
 &  &  &  &  &  &  \\
Observations & 1,888 & 1,888 & 1,888 & 1,888 & 1,855 & 1,888 \\
R-squared & 0.036 & 0.169 & 0.205 & 0.036 & 0.052 & 0.204 \\
Controls & None & All & Inter &  &  &  \\
 IV &  &  &  & distance & distXsib & dist score \\ \hline
\multicolumn{7}{c}{ Robust standard errors in brackets} \\
\multicolumn{7}{c}{ *** p$<$0.01, ** p$<$0.05, * p$<$0.1} \\
\multicolumn{7}{c}{ Dependent variable: Depression score (CESD).} \\
\end{tabular}

\end{table}
\begin{table}[H]
\caption{Pooled: Depression - Preschool, Adult}
\begin{tabular}{lcccccc} \hline
 & (1) & (2) & (3) & (4) & (5) & (6) \\
VARIABLES &  &  &  &  &  &  \\ \hline
 &  &  &  &  &  &  \\
RCH preschool &  &  &  & 1.007 & -4.122 & -6.355 \\
 &  &  &  & [6.349] & [5.118] & [5.533] \\
Reggio None Preschool & 1.765*** & 0.455 & 0.984* & 1.457 & -1.263 & -2.318 \\
 & [0.472] & [0.527] & [0.594] & [3.305] & [2.623] & [2.906] \\
Reggio other preschool & 0.018 & -0.195 & 0.088 &  &  &  \\
 & [0.549] & [0.528] & [0.522] &  &  &  \\
Parma Muni Preschool & -0.878 & 0.267 & 2.703 & 2.272 & 6.076 & 4.884 \\
 & [0.588] & [0.836] & [8.377] & [8.376] & [8.372] & [8.592] \\
Padova Muni Preschool & 0.620 & -0.772 & 9.867 & 9.955 & 12.292 & 8.774 \\
 & [0.782] & [1.313] & [8.992] & [8.970] & [9.256] & [9.113] \\
Male dummy &  & -0.645** & -0.685* & -0.733 & -0.539 & -0.407 \\
 &  & [0.254] & [0.415] & [0.509] & [0.458] & [0.502] \\
Parma other preschool & -1.325** & 0.171 & 2.418 & 2.086 & 5.385 & 3.973 \\
 & [0.531] & [0.822] & [8.413] & [8.311] & [8.356] & [8.554] \\
Padova other preschool & 0.059 & -0.213 & 10.170 & 10.257 & 12.608 & 9.076 \\
 & [0.457] & [1.266] & [8.938] & [8.918] & [9.199] & [9.062] \\
Parma None Preschool & 0.794* & 1.547** & 3.969 & 3.622 & 7.001 & 5.624 \\
 & [0.471] & [0.744] & [8.443] & [8.353] & [8.391] & [8.591] \\
Padova None Preschool & 0.537 & 0.151 & 10.048 & 10.135 & 12.507 & 8.955 \\
 & [0.565] & [1.304] & [8.995] & [8.973] & [9.264] & [9.115] \\
Constant & 21.246*** & 33.274*** & 27.316*** & 27.228*** & 26.066*** & 28.409*** \\
 & [0.337] & [3.669] & [6.112] & [6.229] & [6.294] & [6.470] \\
 &  &  &  &  &  &  \\
Observations & 1,875 & 1,875 & 1,875 & 1,875 & 1,847 & 1,875 \\
R-squared & 0.023 & 0.161 & 0.197 & 0.196 & 0.162 & 0.115 \\
Controls & None & All & Inter &  &  &  \\
 IV &  &  &  & distance & distXsib & dist score \\ \hline
\multicolumn{7}{c}{ Robust standard errors in brackets} \\
\multicolumn{7}{c}{ *** p$<$0.01, ** p$<$0.05, * p$<$0.1} \\
\multicolumn{7}{c}{ Dependent variable: Depression score (CESD).} \\
\end{tabular}

\end{table}


%------------------- Migrant Taste ----------------------%
\begin{table}[H]
\caption{Pooled: Bothered by immigrants - Infant-toddler center, Adult}
\begin{tabular}{lcccccc} \hline
 & (1) & (2) & (3) & (4) & (5) & (6) \\
VARIABLES &  &  &  &  &  &  \\ \hline
 &  &  &  &  &  &  \\
RCH infant-toddler &  &  &  & 6.438 & 2.069 & 10.367 \\
 &  &  &  & [6.099] & [2.122] & [7.740] \\
Reggio None ITC & 0.135*** & 0.099*** & 0.083** & 6.089 & 2.007 & 9.757 \\
 & [0.031] & [0.033] & [0.035] & [5.749] & [2.000] & [7.341] \\
Reggio other ITC & 0.070 & 0.051 & 0.053 &  &  &  \\
 & [0.135] & [0.126] & [0.120] &  &  &  \\
Parma Muni ITC & 0.118** & 0.153** & 0.725 & 5.537 & 2.199 & 8.482 \\
 & [0.055] & [0.075] & [0.676] & [4.853] & [1.800] & [6.332] \\
Padova Muni ITC & 0.412*** & 0.549*** & 0.488 & 5.300 & 1.940 & 8.245 \\
 & [0.091] & [0.118] & [0.709] & [4.858] & [1.820] & [6.332] \\
Male dummy &  & 0.024 & 0.034 & 0.013 & 0.029 & -0.000 \\
 &  & [0.020] & [0.030] & [0.057] & [0.033] & [0.080] \\
Parma other ITC & 0.273*** & 0.261** & 0.798 & 5.609 & 2.271 & 8.555 \\
 & [0.097] & [0.104] & [0.678] & [4.853] & [1.801] & [6.333] \\
Padova other ITC & 0.515*** & 0.773*** & 0.716 & 5.528 & 2.155 & 8.473 \\
 & [0.123] & [0.141] & [0.715] & [4.858] & [1.823] & [6.333] \\
Parma None ITC & 0.151*** & 0.145*** & 0.722 & 5.533 & 2.195 & 8.478 \\
 & [0.033] & [0.054] & [0.677] & [4.853] & [1.801] & [6.333] \\
Padova None ITC & 0.276*** & 0.496*** & 0.464 & 5.276 & 1.903 & 8.221 \\
 & [0.034] & [0.082] & [0.705] & [4.857] & [1.819] & [6.332] \\
Constant & 0.073*** & 0.197 & -0.042 & -4.854 & -1.516 & -7.799 \\
 & [0.027] & [0.288] & [0.439] & [4.827] & [1.728] & [6.310] \\
 &  &  &  &  &  &  \\
Observations & 1,891 & 1,891 & 1,891 & 1,891 & 1,859 & 1,891 \\
R-squared & 0.039 & 0.091 & 0.113 &  & 0.029 &  \\
Controls & None & All & Inter &  &  &  \\
 IV &  &  &  & distance & distXsib & dist score \\ \hline
\multicolumn{7}{c}{ Robust standard errors in brackets} \\
\multicolumn{7}{c}{ *** p$<$0.01, ** p$<$0.05, * p$<$0.1} \\
\multicolumn{7}{c}{ Dependent variable: Bothered by migrants (\%).} \\
\end{tabular}

\end{table}
\begin{table}[H]
\caption{Pooled: Bothered by immigrants - Preschool, Adult}
\begin{tabular}{lcccccc} \hline
 & (1) & (2) & (3) & (4) & (5) & (6) \\
VARIABLES &  &  &  &  &  &  \\ \hline
 &  &  &  &  &  &  \\
RCH preschool &  &  &  & 0.392 & 0.379 & -0.497 \\
 &  &  &  & [0.450] & [0.393] & [0.399] \\
Reggio None Preschool & 0.116*** & 0.082** & 0.055 & 0.219 & 0.200 & -0.237 \\
 & [0.032] & [0.036] & [0.040] & [0.234] & [0.201] & [0.208] \\
Reggio other preschool & 0.100*** & 0.082** & 0.075** &  &  &  \\
 & [0.036] & [0.037] & [0.037] &  &  &  \\
Parma Muni Preschool & 0.175*** & 0.191*** & 0.683 & 0.515 & 0.486 & 0.766 \\
 & [0.042] & [0.057] & [0.674] & [0.700] & [0.702] & [0.698] \\
Padova Muni Preschool & 0.384*** & 0.533*** & 0.503 & 0.538 & 0.474 & 0.329 \\
 & [0.065] & [0.088] & [0.710] & [0.729] & [0.741] & [0.730] \\
Male dummy &  & 0.033 & 0.047 & 0.028 & 0.033 & 0.064* \\
 &  & [0.020] & [0.029] & [0.036] & [0.034] & [0.035] \\
Parma other preschool & 0.138*** & 0.127** & 0.617 & 0.486 & 0.456 & 0.651 \\
 & [0.037] & [0.057] & [0.679] & [0.698] & [0.701] & [0.697] \\
Padova other preschool & 0.238*** & 0.436*** & 0.393 & 0.427 & 0.363 & 0.219 \\
 & [0.032] & [0.084] & [0.706] & [0.726] & [0.737] & [0.727] \\
Parma None Preschool & 0.042 & 0.029 & 0.547 & 0.411 & 0.381 & 0.590 \\
 & [0.032] & [0.056] & [0.685] & [0.705] & [0.707] & [0.703] \\
Padova None Preschool & 0.218*** & 0.393*** & 0.360 & 0.394 & 0.333 & 0.186 \\
 & [0.041] & [0.088] & [0.709] & [0.728] & [0.740] & [0.730] \\
Constant & 0.116*** & 0.114 & -0.046 & -0.080 & -0.045 & 0.129 \\
 & [0.019] & [0.292] & [0.429] & [0.474] & [0.471] & [0.475] \\
 &  &  &  &  &  &  \\
Observations & 1,877 & 1,877 & 1,877 & 1,877 & 1,850 & 1,877 \\
R-squared & 0.047 & 0.095 & 0.116 & 0.041 & 0.046 & 0.062 \\
Controls & None & All & Inter &  &  &  \\
 IV &  &  &  & distance & distXsib & dist score \\ \hline
\multicolumn{7}{c}{ Robust standard errors in brackets} \\
\multicolumn{7}{c}{ *** p$<$0.01, ** p$<$0.05, * p$<$0.1} \\
\multicolumn{7}{c}{ Dependent variable: Bothered by migrants (\%).} \\
\end{tabular}

\end{table}

\todo[backgroundcolor=orange!30,size=\tiny]{Insert here the p-values of the differences in coefficients}

\normalsize

%\subsection{Propensity Score Matching Results}
%\label{sec:PSM}
%Another approach considered was to do a propensity score matching.
