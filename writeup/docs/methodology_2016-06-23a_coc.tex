\documentclass[10pt]{article}
\usepackage[top=1in, bottom=1in, left=1in, right=1in]{geometry}
\parindent 22pt

\newcommand\independent{\protect\mathpalette{\protect\independenT}{\perp}}
\def\independenT#1#2{\mathrel{\rlap{$#1#2$}\mkern2mu{#1#2}}}

\usepackage{adjustbox}
\usepackage{amsmath}
\usepackage{amssymb}
\usepackage{appendix}
\usepackage{array}
\usepackage{authblk}
\usepackage{booktabs}
\usepackage{caption} 
\usepackage{datetime}
\usepackage{enumerate}
\usepackage{fancyhdr}
\usepackage{float}
\usepackage{graphicx}
\usepackage[colorlinks=true,linkcolor=blue,urlcolor=blue,anchorcolor=blue,citecolor=blue]{hyperref}
\usepackage{lscape}
\usepackage{mathtools}
\usepackage{multirow}
\usepackage{natbib}
\usepackage{pgffor}
\usepackage{setspace}
\usepackage{tabularx}
\usepackage{threeparttable}
\usepackage[colorinlistoftodos,linecolor=black]{todonotes}

\captionsetup[table]{skip = 2pt}

\newcolumntype{L}[1]{>{\raggedright\arraybackslash}p{#1}}
\newcolumntype{C}[1]{>{\centering\arraybackslash}p{#1}}
\newcolumntype{R}[1]{>{\raggedleft\arraybackslash}p{#1}}


\settimeformat{hhmmsstime}

\begin{document}

\title{Description of Methodology}
\author{Reggio Team}
\date{Original version: Thursday  16$^{\text{th}}$ June, 2016 \\ Current version: \today \\ \vspace{1em} Time: \currenttime}
\maketitle

\doublespacing

\section*{Notes:}

\begin{itemize}

\item Catholic School Comparison=
\begin{itemize}
Sylvi has located a detailed secription of catholic preschool and other parocial schools in Italy as well as contacted Notre Dame. A (no more than 3 paragraph) description of this document is due to me tomorrow.
\begin{itemize}
\item Daniel (her summer assistant) is preparing 3 tables detailing these differences. These are also due to me tomorrow.
\end{itemize}
\item Anna and Jessica are synthesizing the (huge) set of tables and including interpretations of the results for:
\begin{itemize}
\item The linear probability results analyzing school selection and changes in market share over the cohorts.
\item The interpretation of the baseline OLS characteristics with a more detailed description.
\end{itemize}
\item Finally, the three of us our meeting tomorrow to review our diff-in-diff results and develop their interpretation.
\item We think it would be beneficial to meet to discuss our progress either later this weekend if your are feeling better from your travels. 
]\end{itemize}
\end{itemize}

\section{Methodology}
We consider the determinants for individual, $i$'s, life cycle outcomes which might include labor market participation, social participation, civic engagement, health and mental health, and non-cognitive skills. Each outcomes is considered to be determined by the individual's baseline characteristics, $X$, which includes their family background, household income, ethnic and religious identity, and other individual characteristics. For this simplest  case, considering only one possible outcomes, $y_1$ we specify the following relationship  where $\varepsilon_{i,1}$ is an individual specific disturbance that we assume to be independent from the outcome variable after conditioning on the control variables, and $c_{i,1} = \mathbf{1}(C_i = c_1)$ is a dummy indicator for person, $i$, living in city $1$. \\

\begin{equation} \label{ols}
y_{i,1} =  X_i \beta_{i,1}' + \gamma_1 c_{i} + \varepsilon_{i}
\end{equation}

Consider the simple example of two cities, cohorts, and preschool types. The baseline OLS model ignores effects from cohort or type of preschool. We define $C_i = 0$ as the reference city, and estimate the effect of not attending preschool in that city, which assumes that all preschool types are perfect substitutes within the respective cities.  The estimated $\tilde{\beta_{i}}$ can be interpreted as the differences in $Y_i$ that are explained by observed characteristics, and  $\tilde{\gamma_1}$ is the estimated effect of not living in the reference city. 


\subsection{Preschool's effect on life-cycle outcomes -- Fixed Effect model}
Starting from this baseline specification, we consider the set of $j$ possible outcomes, $\{y_1, \dots, y_j\}$, and account for the possibility of city, $c$, cohort, $k$, and preschool-type, $s$ fixed-effects, along with all possible two-way\footnote{i.e. The difference-in-difference effects} and all possible three way effects\footnote{i.e. -- The triple difference (or diff-i-diff-i-diff effects}. This system is represented below:

\begin{subequations}
\begin{align}
y_{i,1} & = \beta_{1}x_i  + \gamma^c_{i,1} c_i + \gamma^k_{i,1} k_i + \gamma^s_{i,1} s_i + \gamma^k_{i,1} k_i + \gamma^{c,k}_{i,1} (c_i \times k_i) + \gamma^{k,s}_{i,1} (k_i \times s_i) + \gamma^{c,s}_{i,1} (c_i \times s_i) + \\
&\qquad \qquad \qquad \qquad \qquad \qquad \qquad \qquad \qquad \qquad \qquad \qquad +\gamma^{c,k,s}_{i,1} (c_i \times k_i \times s_i) + \varepsilon_{i,c,k,s,1}  \nonumber \\
y_{i,2} & = \beta_{2}x_i  + \gamma^c_{i,2} c_i + \gamma^k_{i,2} k_i + \gamma^s_{i,2} s_i + \gamma^k_{i,2} k_i + \gamma^{c,k}_{i,2} (c_i \times k_i) + \gamma^{k,s}_{i,2} (k_i \times s_i) + \gamma^{c,s}_{i,2} (c_i \times s_i) + \\
\vdots & \qquad \qquad \qquad \qquad \qquad \qquad \vdots \qquad \qquad \qquad \qquad \qquad \qquad +\gamma^{c,k,s}_{i,2} (c_i \times k_i \times s_i) + \varepsilon_{i,c,k,s,2}  \nonumber \\
\vdots & \qquad \qquad \qquad \qquad \qquad \qquad \vdots \qquad \qquad \qquad \qquad \qquad \qquad \qquad \qquad \vdots  \nonumber \\
y_{i,j} & = \beta_{j}x_i + \gamma^c_{i,j} c_i + \gamma^k_{i,j} k_i + \gamma^s_{i,j} s_i + \gamma^k_{i,j} k_i + \gamma^{c,k}_{i,j} (c_i \times k_i) + \gamma^{k,s}_{i,j} (k_i \times s_i) + \gamma^{c,s}_{i,j} (c_i \times s_i) + \\
& \qquad \qquad \qquad \qquad \qquad \qquad \qquad \qquad \qquad \qquad \qquad \qquad +\gamma^{c,k,s}_{i,j} (c_i \times k_i \times s_i) + \varepsilon_{i,c,k,s,j}  \nonumber
\end{align}
\end{subequations}

Stacking the outcomes equations we define the outcome vector $Y_i = [y_1, \dots , y_j]'$, $C_i$ as a $\ell \times 1$ vector indicating city, where $c_{a} = \mathbf{1}(C_i = c_a\; \forall a \in \{c_1, \dots, c_{\ell} \}$. $K_i$ is a $m \times 1$ vector indicating the individual's cohort $k_a = \mathbf{1}(K_i = k_a\; \forall a \in \{k_1, \dots, k_{m} \}$, and  $S_i$ is a $n \times 1$ vector indicating school type where $s_a = \mathbf{1}(S_i = s_a\; \forall a \in \{s_1, \dots, s_{n} \}$.  Ignoring the interaction terms for now, we can rewrite the system of equations for $\{y_{i,1}, \dots, y_{i,j}\}$ in terms of $Y_i,\ \mathbf{c},\ \mathbf{k}, \text{ and, } \mathbf{s}$ as: 

\begin{equation} \label{fixed}
Y_i = X_i \boldsymbol{\beta_i}' + C_i \boldsymbol{\gamma}_c' + K_i \boldsymbol{\gamma}_k' + S_i \boldsymbol{\gamma}_s'  + \boldsymbol{\varepsilon}_{i}
\end{equation}

Note that this is a standard OLS regression model after including city-specific fixed effects. Likewise  $K_i \boldsymbol{\gamma}_k$ and $S_i \boldsymbol{\gamma}_s$ correspond to cohort-specific and preschool-type specific fixed effects. Again for the case where $\ell = m = n = 2$, the coeffcients on $C, K, S$ can be interpreted as the specific effects from an individual not being part of the respective reference group. Consider for example, an individual who does live in the reference city, but is not in the reference cohort or preschool type, which implies that  $\tilde{\gamma}_c = 0$, or no city effect for the reference city, while $\tilde{\gamma}_k, \tilde{\gamma}_c \neq 0$ can be interpreted as the effect for the individual from being different from the reference school and cohort. 

\subsection{Preschool's effect on life-cycle outcomes -- Difference-in-difference model}

We also expand this model to include two way interactions between, $C_i, K_i, \& S_i$ and define the respective coefficients as the difference-in-difference estimator, and finally, the three-way interaction, $C_i \times K_i \times S_i$, and define the  We now define the two-- and three-way interaction terms. Let $M_i = C_i \otimes K_i$,  $N_i = C_i \otimes S_i$, $Q_i = K_i \otimes S_i$, and let $T_i = C_i \otimes K_i \otimes S_i$:\footnote{The kronecker product of three $\ell \times 1$, $m \times 1$, and $n \times 1$ yields a $(\ell \times m \times n) \times 1$ long column vector of the desired interactions. (Turkington 2002, 2013)}  : where the vectors of interaction terms are given as:

\begin{equation*}
\underset{\begin{bmatrix} m_1 = c_1 k_1 \\
\vdots  \\
m_m = c_1 k_m  \\
\vdots  \\
m_{\ell m} = c_{\ell} k_m 
\end{bmatrix}}{M_i = C_i \otimes K_i  } \qquad \qquad
 \underset{\begin{bmatrix} n_1 = c_1 s_1 \\
\vdots  \\
s_n = c_1 s_n  \\
\vdots  \\
s_{\ell n} = c_{\ell} s_n 
\end{bmatrix}}{N_i = C_i \otimes S_i} \qquad \qquad
\underset{\begin{bmatrix} q_1 = k_1 s_1 \\
\vdots  \\
q_m = k_1 s_n  \\
\vdots  \\
q_{m n} = k_{m} s_n 
\end{bmatrix}}{Q_i = K_i \otimes S_i } \qquad \qquad
\underset{\begin{bmatrix} q_1 = c_1 k_1 s_1  \\
\vdots  \\
q_{\ell n} = c_{\ell} k_1 s_n  \\
\vdots  \\
q_{\ell n m} = c_{\ell} k_m s_n  \\
\end{bmatrix}}{T_i = C_i \otimes Q_i \\
= C_i \otimes K_i \otimes S_i }
\end{equation*}


We can now specify the system of equations given in ( equation 2) in terms of the $j \times 1$ vector of outcomes, $Y_i$, the vector of City types, the vector of Cohort identifiers, the vector of preschool types, the $M,N,Q$ vectors of two way interaction terms, and the $T$ vector of three way interactions, as follows

\begin{equation} \label{diffs}
Y_i = \underbrace{X_i \boldsymbol{\beta_i}'}_{\text{Individual Covariates}} + \underbrace{C_i \boldsymbol{\gamma}_c' + K_i \boldsymbol{\gamma}_k' + S_i \boldsymbol{\gamma}_s'}_{\text{City, Cohort, School F.E.}}  +\; \underbrace{M_i \boldsymbol{\gamma}_m' + N_i \boldsymbol{\gamma}_n' + Q_i \boldsymbol{\gamma}_q'}_{\text{Diff in Diff}} + \underbrace{T_i \boldsymbol{\gamma}_t'}_{\text{Triple difference}} \; + \; \boldsymbol{\varepsilon}_{i}
\end{equation}

For notational convenience, we stack the vectors $C_i, K_i, S_i, M_i, N_i, Q_i, \text{ and } T_i$ into the long column vector, $D_i$, and similarly stack $\gamma_c, \gamma_k, \gamma_s, \gamma_m, \gamma_n, \gamma_q, \gamma_t $ into a vector of equal length composed of all the dummy variable coefficients, which is defined  as $\gamma_d$. Rewriting (equation 3) in this simplified notation gives:

\begin{equation}
Y_i = X_i \boldsymbol{\beta_i}' + D_i \boldsymbol{\gamma}_d' + \boldsymbol{\varepsilon}_{i}
\end{equation}

The full specification which includes the difference-in-differences (and difference-in-differences) estimators addreses concerns that \ref{eq:fixed} does not reflect the difference in cohort structure or in preschool categories that may exist in different cities, or that $(S_i = 1 \mid C_i = 0) \neq (S_i = 1 \mid C_1 =1)$ Again considering the simple case of $\ell = m = n = 2$. The estimated model can be written as:
\begin{eqnarray*}  \label{eq:specific2}
Y_i & = & \alpha_0  + \gamma_c (C_i = 1) + \gamma_k (K_i = 1) + \gamma_s (S_i = 1) \nonumber \\
& &\ + \gamma_{c,k} (C_i = 1 \times K_i = 1) + \gamma_{k,s} (K_i = 1 \times S_i = 1)  + \gamma_{c,s} (C_i = 1 \times S_i = 1)   \nonumber \\
 & &\ + \gamma_{c,k,s}(C_i = 1 \times K_i = 1 \times S_i = 1) + X_i \boldsymbol{\beta} + \varepsilon_i  
\end{eqnarray*}

Where $\alpha_0$ is the intercept corresponding to being a member of the reference group, i.e. -- $(C_i = 0, K_i = 0, S_i = 0)$   The interpretation for the coefficients on each indicator is best understood in terms of the expected outcomes implied by \ref{eq:diffs}. We consider the interpretation of $\gamma_c, \gamma_{c,k}, \text{ and }, \gamma_{c,k,s}$ explicitly. 
\begin{eqnarray*}  
    \mathbb{E}[Y_i \mid C_i = 0, K_i = 0, S_i = 0] & = & \alpha_0 + X_i \boldsymbol{\beta} + \varepsilon_i \\
    \mathbb{E}[Y_i \mid C_i = 1, K_i = 0, S_i = 0] & = & \alpha_0 +  \gamma_c + X_i \boldsymbol{\beta} + \varepsilon_i \\
    \mathbb{E}[Y_i \mid C_i = 1, K_i = 1, S_i = 0] & = & \alpha_0 + \gamma_c + \gamma_k + \gamma_{c,k} + X_i \boldsymbol{\beta} + \varepsilon_i \\
    \mathbb{E}[Y_i \mid C_i = 1, K_i = 1, S_i = 1] & = & \alpha_0 + \gamma_c + \gamma_k + \gamma_s + \gamma_{c,k} + \gamma_{c,s} + \gamma_{k,s} + \gamma_{c,k,s} + X_i \boldsymbol{\beta} + \varepsilon_i
\end{eqnarray*}
Thus, $\gamma_c$ can be interpreted as ``the mean difference in outcomes for individuals attending a comparable reference preschool type in the `treatment' city compared to those in the reference city.  Interpretation for coefficients on other city and school type dummies are analogous. Likewise, $\gamma_{c,k}$ compares how effects on outcomes are different between the two city categories, when moving across cohorts. $\gamma_{c,k} > 0$ would imply that the improvement in outcomes between the two cohorts increased, on average, for the `treatment city' more so than it did for the reference category. Consider an example where, $C_i = 0$ if the individual attended no preschool and  $C_i = 1$ if they attended any preschool. Then $\gamma_{c,k,s} > 0$ can be understood as the comparison city, $C_i = 1$, having larger growth in the average difference in outcomes between younger and older cohorts, for those attending preschool compared to those with none.

\begin{eqnarray*}  
\boldsymbol{\gamma_c} & = &  \Big(\overline{Y}_i \mid C_i = 1\Big) - \Big(\overline{Y}_i \mid C_i = 0\Big) & \\ [0.4em]
%& = & (\mathbb{E}[Y_i \mid C_i = 1, K_i = 0, S_i = 0]  - \mathbb{E}[Y_i \mid C_i = 0, K_i = 0, S_i = 0]) & \\
\boldsymbol{\gamma_{c,k}} & = & \Bigg[\Big(\overline{Y_i} \mid K_i = 1\Big) - \Big(\overline{Y_i} \mid K_i = 0\Big) \Big| C_i =1 \Bigg] - \Bigg[\Big(\overline{Y_i} \mid K_i = 1) - \Big(\overline{Y_i} \mid K_i = 0\Big) \Big| C_i = 0 \Bigg] & \\[0.6em]
% & = & (\mathbb{E}[Y_i \mid C_i = 1, K_i = 1, S_i = 0]  - \mathbb{E}[Y_i \mid C_i = 1, K_i = 0, S_i = 0] ) & \\
%& & - (\mathbb{E}[Y_i \mid C_i = 0, K_i = 1, S_i = 0]  -  \mathbb{E}[Y_i \mid C_i = 0, K_i = 0, S_i = 0]) & \\
\boldsymbol{\gamma_{c,k,s}} & = & \Bigg(\Big[\Big(\overline{Y_i} \mid S_i = 1\Big) - \Big(\overline{Y_i} \mid S_i = 0\Big)\Big| C_i =1, K_i = 1 \Big] - \Big[\Big(\overline{Y_i} \mid S_i = 1\Big) - \Big(\overline{Y_i} \mid S_i = 0\Big)\Big| C_i =1, K_i = 0 \Big]\Bigg)  &\\
&  - & \Bigg(\Big[\Big(\overline{Y_i} \mid S_i = 1\Big) - \Big(\overline{Y_i} \mid S_i = 0\Big)\Big| C_i =0, K_i = 1 \Big] - \Big[\Big(\overline{Y_i} \mid S_i = 1\Big) - \Big(\overline{Y_i} \mid S_i = 0\Big)\Big| C_i =0, K_i = 0 \Big]\Bigg) &
 %& = &(\mathbb{E}[Y_i \mid C_i = 1, K_i = 1, S_i = 1]  - \mathbb{E}[Y_i \mid C_i = 1, K_i = 1, S_i = 0]) & \\
%& & - (\mathbb{E}[Y_i \mid C_i = 0, K_i = 1, S_i = 1]  - \mathbb{E}[Y_i \mid C_i = 0, K_i = 1, S_i = 0]) & \\
%& & - (\mathbb{E}[Y_i \mid C_i = 1, K_i = 0, S_i = 1]  - \mathbb{E}[Y_i \mid C_i = 1, K_i = 0, S_i = 0]) &
\end{eqnarray*}

\subsection{Selection into preschool - Linear Probaility Model}

We consider the possibility of selection into preschool based on observable characteristics influencing parental choice of preschool type.  Define a set of observables, $Z$, assumed to be the latent variable determining the choice of preschool type, $s^{*}$, and let $\eta_i \independent \varepsilon_{i} \mid Z_i$ be the unobservable components of choice, which are assumed to be independent of choice of preschool type, after conditioning on individual characteristics, that is: $\varepsilon_{i} \independent S \mid X$.  where:

\begin{equation}
s^{*} = \Psi_i(Z_i) + \eta_i
\end{equation}

\begin{equation}
S_{i} = 1(s^{*}_{i} > 0) = 1(\mathrm{argmax} s^{*} = S_i) 
\end{equation}


Relating this selection problem to our evaluation of outcomes is the concern that higher IQ or wealthier parents, for example, may select into different preschool types than more disadvantaged parents, where we think that parental IQ and wealth also has impacts outcomes for the child. Any observed changes in patterns of selection over different cohorts and / or in different cities can also be interpreted as the growth in market share for the given preschool type. We estimate a system of linear equations for the probability that individual $i$ attended preschool type $c_a$, conditional on their observed city and cohort.

\begin{equation}
\mathbb{P}(S_i = s^*\mid C_i, K_i) = X_i\beta_i' + Z_i \gamma_i' + \eta_i
\end{equation}


\end{document}
