\documentclass{article}
\usepackage{verbatim}
\usepackage{geometry}
\usepackage{booktabs}
\usepackage{amsmath}
\usepackage{amsthm}
\usepackage{bm}
\usepackage[sort]{natbib}
\usepackage[colorlinks=true,linkcolor=blue,urlcolor=blue,anchorcolor=blue,citecolor=blue]{}
\usepackage{multirow}
\usepackage{multicol}
\usepackage{url,hyperref}
\usepackage{fmtcount}
\newlength{\templength}
\usepackage{adjustbox}
\usepackage{tabularx}
\usepackage{ragged2e}
\usepackage{booktabs}
\usepackage{caption}
\usepackage{threeparttablex}
\usepackage{enumerate}
\usepackage{rotating}
\graphicspath{{images/}}
\usepackage{pdflscape}
\usepackage{setspace}

\usepackage{graphicx}
\newcommand{\foo}{\hspace{-2.3pt}$\bullet$ \hspace{5pt}}

%\newcommand{\singlespace}{\relax{}}
%\newcommand{\doublespace}{\relax{}}

% BEGIN MATH COMMANDS
	\newcommand{\phat}{\ensuremath{\hat{p}}}
	\newcommand{\tildep}{\ensuremath{\tilde{p}}}
	%%%%%%%%%%%%%%%%%%%%%%%%%
	% these lines come from the jjhbymms.sty style file
	% in "revisions/templates/style-experiment  on athens
	\newcommand\statindep{\protect\mathpalette{\protect\overlapper}{\perp}}
	\newcommand{\overlapper}[2]{\mathrel{\rlap{\ensuremath{#1#2}}\mkern6mu{#1#2}}}
	\makeatletter
		\newcommand{\moverlay}{\mathpalette\mov@rlay}
		\newcommand{\mov@rlay}[2]{\leavevmode\vtop{\baselineskip\z@skip{}
			\lineskiplimit-\maxdimen\ialign{\hfil\ensuremath{#1##}\hfil\cr#2\cr\cr}}}
	\makeatother
	\newcommand{\notstatindep}{\mathrel{\moverlay{\diagup\cr\statindep}}}
	%%%%%%%%%%%%%%%%%%%%%%%%%
	\newcommand{\setG}{\ensuremath{\mathcal{G}}}
	\newcommand{\setI}{\ensuremath{\mathcal{I}}}
	\newcommand{\setL}{\ensuremath{\mathcal{L}}}
	\newcommand{\equald}{\ensuremath{\overset{d}{=}}}

	\DeclareMathOperator{\Prob}{P}
	\DeclareMathOperator{\supp}{supp}

%\theoremstyle{remark}
\theoremstyle{definition}

	% amsthm commands
		%\newtheorem{theorem}{Theorem}[section]
		%\newtheorem{lemma}{Lemma}[section]
		%\newtheorem{definition}{Definition}[section]
		%\newtheorem{corollary}{Corollary}[section]
	  %\theoremstyle{definition}
		%\newtheorem{example}{Example}[section]
    %\newtheorem{hypothesis}{Hypothesis}
    %\renewcommand{\thehypothesis}{H-\arabic{hypothesis}}
		%\newtheorem{assumption}{Assumption}

    %\renewcommand{\theassumption}{A-\arabic{assumption}}
		%\newtheorem{algorithm}{Algorithm}

	  \newtheorem{remark}{Remark}[section]
    \newtheorem{theorem}{Theorem}
    \renewcommand{\thetheorem}{\textbf{T-\arabic{theorem}}}
    \newtheorem{definition}{Definition}
    \renewcommand{\thedefinition}{\textbf{D-\arabic{definition}}}
    \newtheorem{example}{Example}[section]
    \newtheorem{model}{Model}
    \renewcommand{\themodel}{\textbf{M-\arabic{model}}}
    \newtheorem{surr}{Model}
    \renewcommand{\thesurr}{\textbf{S-\arabic{model}}}
    \newtheorem{empM}{Model}
    \renewcommand{\theempM}{\textbf{E-\arabic{empM}}}
    \newtheorem{empMprime}{Model}
    \renewcommand{\theempMprime}{\textbf{E-\arabic{empM}$^{\prime}$}}
    \newtheorem{hypM}{Model}
    \renewcommand{\thehypM}{\textbf{H-\arabic{hypM}}}
    \newtheorem{hypMprime}{Model}
    \renewcommand{\thehypMprime}{\textbf{H-\arabic{hypM}$^{\prime}$}}
    \newtheorem{hypothesis}{Hypothesis}
    \renewcommand{\thehypothesis}{\textbf{H-\arabic{hypothesis}}}
    \newtheorem{assumption}{Assumption}
    \renewcommand{\theassumption}{\textbf{A-\arabic{assumption}}}
    \newtheorem{assumptionPrime}{Assumption}
    \renewcommand{\theassumptionPrime}{\textbf{A-\arabic{assumptionPrime}$^{\prime}$}}
    \newtheorem{assumptionPrimeH}{Hypothesis}
    \renewcommand{\theassumptionPrimeH}{\textbf{H-\arabic{assumptionPrime}$^{\prime}$}}
    \newtheorem{algorithm}{Algorithm}
    \newtheorem{axiom}{Axiom}
    \renewcommand{\theaxiom}{\textbf{X-\arabic{axiom}}}
    \newtheorem{lemma}{Lemma}
    \renewcommand{\thelemma}{\textbf{L-\arabic{lemma}}}
    \newtheorem{corollary}{Corollary}
    \renewcommand{\thecorollary}{\textbf{C-\arabic{corollary}}}
    \theoremstyle{remark}
    %\newtheorem{remark}{Remark}[section]
    \renewcommand{\theremark}{\emph{\arabic{section}.\arabic{remark}}}

% END MATH COMMANDS

% BEGIN FORMATTING COMMANDS
	\newcommand{\phantomdots}[1]{\phantom{#1}\!\!\!\!\!\!...}
	\newcommand{\tablefootnotebf}[1]{\ensuremath{^{\mathbf{#1}}}}
	\newcommand{\tablefootnote}[1]{\ensuremath{^{#1}}}
%	\newcommand{\extratext}[1]{{\itshape #1}}
	\newcommand{\extratext}[1]{\relax}
\geometry{margin=1in}

\def\indep{\perp\!\!\!\perp}

% END FORMATTING COMMANDS

%	temporary length object to work with during formatting
%\newlength{\templength}

\setlength{\tabcolsep}{3pt}

% \newcommand{\notesize}{\small}

%\SetWatermarkFontSize{5cm}
%\SetWatermarkScale{5}
%\SetWatermarkLightness{.9}
%\newcommand{\SetWatermarkOn}{\SetWatermarkText{DRAFT}}% use watermarks
%\newcommand{\SetWatermarkOn}{\SetWatermarkText{}}% don't use watermarks
%\newcommand{\SetWatermarkOff}{\SetWatermarkText{}}
\newcommand{\SetWatermarkOn}{\relax}
\newcommand{\SetWatermarkOff}{\relax}

%%%%%%%%%%%%%%%%%%%%%%%%%%%
%		DOCUMENT HEADER
%%%%%%%%%%%%%%%%%%%%%%%%%%%

%	set page margins
%\geometry{width=6in,height=9in}

%	set the text size of float comments
%	this runs after the content but before the comments of every float
\newcommand{\notesize}{\small}

	
%%%%%%%%%%%%%%%%%%%%%%%%%%%
%		THE DOCUMENT:
%%%%%%%%%%%%%%%%%%%%%%%%%%%

%%%	TITLE PAGE
%\input{0_titlepage}
%\input{0_titlepage_buffet}

%%%% TABLE OF CONTENTS
%\singlespace
%\tableofcontents
%\newpage
%\doublespace

%\input{1_}

%\input{2_}

%\input{3_}

%\input{4_}

%\input{5_}

%\input{6_}

%\input{7_}

\newcolumntype{L}[1]{>{\raggedright\let\newline\\\arraybackslash\hspace{0pt}}m{#1}}
\newcolumntype{C}[1]{>{\centering\let\newline\\\arraybackslash\hspace{0pt}}m{#1}}
\newcolumntype{R}[1]{>{\raggedleft\let\newline\\\arraybackslash\hspace{0pt}}m{#1}}

\title{Municipal, Private-Secular, and Private-Religious \\ Early Childhood Programs and Policies\\[15pt] Reggio Emilia, Parma, and Padova}
\author{Reggio Team}
\date{\today}

\begin{document}
\maketitle

\doublespacing

\section{Reggio (Municipal Schools of Reggio Emilia)}

The Reggio model is a form of progressive early childhood education reflecting the theories of Dewey, Piaget, Vygotsky, and Bronfenbrenner. A Reggio child is seen as a ``producer of culture, values, and rights.'' There is an explicit focus on building and maintaining strong parent-teacher relationships, and involving the child's home and community in the learning process. 

Children and teachers negotiate learning in a long-term project-based approach. Knowledge is constructed, emerging from thoughtful conversation and the collaborative reviewing and revisiting of ideas. Teachers scaffold learning by providing instruction in the use of materials and help in finding resources. Children demonstrated their emerging knowledge through visual art, media, and symbolic language, aided by special teachers called atelieristas. Teachers document development and learning by creating portfolios of each child's work. Reviewing one's portfolio over the year is intended to build self-awareness and meta-cognition about the learning process. 

Reggio teachers work in pairs. The school cook may or may not be one of the pair. Atelieriestas are specially trained in visual arts. A pedagogista is assigned to a number of Reggio schools, to serve as a professional mentor for teachers and support reflection. 

%The asili nido are organized in 3 groups by age: 3-12 months, 12-24 months, and 24-36 months \citep{Becchi-Ferrari_1990_Pub-Inf-Centres-Italy,Saraceno_1984_Soc-Probs}. 

This document compiles information about the early childhood programming in Reggio Emilia, Parma, and Padova. We document both asilo serving children aged 3-36 months and materna serving children from ages 3-6 years. We focus our discussion on programs with comprehensive curricula to offer a comparison of program elements, pedagogy, and theories of early childhood learning. 

\section{Montessori}

In 1907, Dr. Maria Montessori established the Casa dei Bambini in Rome for disadvantaged children. Trained as a physician, Montessori carefully recorded observations of her students to improve the design of learning materials still in use today \citep{Schilling2011}. Within 5-10 years, the Montessori method was adopted in Milan and at the University of Virginia \citep{OECD_2001_Italy-Country-Note, Lillard2013}. In 1924, the Rome-based Opera Nazionale Montessori (ONM) was recognized as a legal entity, and continues to support both public and private schools that adopt the Montessori method. In Italy, ONM-affiliated schools are operated at the state, municipal, provincial, private, matching, ``parificata" (official private), and parish level. The ONM promotes research at the national and international levels, organizes and coordinates training for teachers and program administrators, and has published a monthly magazine in press since 1952.

The Montessori method encourages a prepared environment, sensory education and music, the child's physical health, freedom to move independently, and intellectual learning. Children are viewed as inherently motivated learners when provided access to carefully prepared environment \citep{Lillard2013}. Instead of a prescribed curriculum in Montessori, there is a focus on highly individualized, child-initiated learning. Teachers serve as ``unobtrusive directors'' as children spend 80\% of their day engaged in self-directed or small group activity \citep{Edwards_2002_ECRP}. In this setting, with teachers available to guide (or ``scaffold") learning on an as needed basis, children naturally work hard at learning. Padova has two ONM-affiliated Montessori schools. Each Padova Montessori school is parificata, and hosts both a nursery (for ages 3-36 months) and a children's house (for ages 2.5-6 years). There are no ONM-affiliated Montessori schools located in either Reggio Emilia or Parma \citep{ONM-website}. 

Mixed-age classes (from ages 3 to 6) encourage older children to serve as leaders and role models for younger peers. Rooms are deliberately organized into domains for practical life, sensorial, art, math, language, science and geography. Montessori was the first to design and manufacture child-sized furniture, to enable autonomous physical, social-emotional, and cognitive development \citep{Lillard2008, Schilling2011}. Montessori teachers complete a minimum of one full-time academic year of affiliated training in the Montessori Method. In the U.S., a B.A. is required for admission to Montessori training programs. At this time, we do not yet have data on teacher qualifications, teacher-child ratios, and hours of operations for the Montessori schools in Reggio Emilia and Parma.

Maria Montessori developed a unique set of materials that engage sensory learning in the development of mathematical and pre-literacy skills. In traditional Montessori schools, there are specific sequences and purposes for which learning materials are meant to be engaged. Children do not play with toys, but rather engage independently in their ``work." To promote physical health, children play outdoors daily regardless of weather. 

In the 1920s and 1930s, Montessori trained psychologists Erik Erikson and Anna Freud. In Geneva, Jean Piaget was director of a ``modified" Montessori school and president of the Swiss Montessori Society \citep{Haines2000}. 

\section{Waldorf}

In 1919, Rudolf Steiner founded the first free Waldorf school in Germany to provide elementary education for children of parents working at the Waldorf-Astoria Cigarette Factory. Waldorf students---like Reggio students---experience an arts-based curriculum where learning occurs through representation.  
%Steiner 1923 Modern Art Education

The International Association for Steiner-Waldorf Early Childhood Education lists one affiliated scuola materna in Reggio Emilia and one affiliated scuola materna in Padova. Each is a private school offering early childhood education for ages 2.5 to 6, and primary school for kindergarten through 8th grade. \citetext{\citeauthor{WaldorfPadova-website}}

The Waldorf model seeks to instill a holistic manner of integrating knowledge gained from thinking, feeling, and doing. Steiner defined essential experiences for healthy development: love, warmth, creativity and artistry, joy, humor and happiness, meaningful adult activities to be imitated, and interaction with adults who are pursuing their own path of inner development. In Waldorf early childhood programs, children are engaged in physical activity, imitation of adults, language, fantasy, and imaginary play \citep{Edwards_2002_ECRP}.

Waldorf teachers serve as role models, encouraging the child's natural sense of wonder, spirituality, belief in goodness, and love of beauty. Color, carefully chosen props, and the use of natural materials are intrinsic to uncluttered, homelike, and aesthetically pleasing environments. Predictable, regular schedules support children's need for security. 

Learning is experiential and sensory-based. Children act out stories in order to experience life and develop language. Activities include simple crafts, baking, nature walks, puppetry, and fairy tales \citep{Edwards_2002_ECRP}.

In Waldorf schools, teachers govern and administer the program. Waldorf teacher training is 2 years full-time, includes studies in the arts, philosophy, group dynamics, organizational development, human development, pedagogy and teaching, and practical classroom experience. Waldorf teachers commit to ``ongoing self-development'' and collaboration with peers. 

\pagebreak
\setlength{\tabcolsep}{12pt}
\begin{landscape}
\begin{center}
\thispagestyle{empty}
\begin{table}
\caption{Comparison of Montessori, Reggio, and Waldorf Pedagogy and Practices}
\label{tab:ecvars_1}
\small
\begin{tabular}{*4{>{\raggedright\arraybackslash}p{4.5cm}}}
\toprule
\textbf{ECE Variables} 	&	\textbf{Montessori}		&	\textbf{Reggio}		&	\textbf{Waldorf}		\\
\midrule
Child Learning	&	Highly individualized, child-driven	&	Child-initiated, project-based, visual arts  		&	Oral listening, imaginary play, arts-based, imitation, experiential 	\\ \\
Teacher's Role	&	Organizer, Observer, and Scaffolder	&	Negotiator, Socratic Method, Mentor, Documenter		&		Didact, Role Model, Moral Leader, Learner, School Governance	\\ \\
Teacher Training		&	Required Certification	&	In-service Training and Mentoring by Pedagogista		&	Required Certification	\\ \\
Peer Interaction	&	Spontaneous	&	Emphasis on Relationships and Collaboration		&	(unknown)		\\ \\
Learning Materials	&	Sensorial, Sequential, Montessori-Specified	&	Arts-based, Community-based	&	Arts-based, Natural (Pine cones, etc)	\\ \\
Environment	&	Natural, Organized, Child-sized		&	Open Space, Aesthetic		&	Aesthetically-pleasing, garden	\\ \\
Evidence of Learning	&	Progression through Materials	&	Portfolio, Visual Arts, Reflection	&	Imaginary Play and Visual Arts	\\ \\
Academic-Focused Content	&	Math, Science, Literacy, Social Studies	&	Language and Literacy		&	Not the goal of early childhood	\\ \\
Parental Involvement 	&	Encouraged by School	&	Focal Curriculum Component		&	(unknown)		\\
\bottomrule
\end{tabular}
\end{table}
\raisebox{-110pt}{\makebox[\linewidth]{\thepage}}
\end{center}
\end{landscape}

\clearpage

\section{Provision of Asili Nido for Infants and Toddlers (3-36 months)}

In 1989, Italy's Health Ministry engaged University of Pavia to investigate the provision, staffing, admission policies, attendance, and regional funding for construction and operation of public and private centers for infants and toddlers \citep{Becchi-Ferrari_1990_Pub-Inf-Centres-Italy}. The 1990 report finds an unequal provision of day care across Italy's 19 regions, comparing comprehensive expenditure by region to state funding received.\footnote{The report distinguishes institutional responsibilities of each the State, Region, and Township as follows: The State passes laws, defines aims, and provides the majority of funding for day care centers to the Regions via the Health Ministry. Regions pass laws regarding the organization and basic planning of their centers. Townships organize and run the centers.}  

\subsection{Emilia Romagna}

In 1989, Emilia Romagna reported 338 asili nido that offered slots to 16,280 children, for a total provision of daycare for 15.6\% of its infant-toddler population.\footnote{Provision of care is calculated using 1988 population data reported by the National Institute of Statistics (ISTAT) \citep{Becchi-Ferrari_1990_Pub-Inf-Centres-Italy}.} In 1989, asili nido were run by degree-level site coordinators, who also had educational duties. Adult-child ratios in Emilia Romagna were reported as 1:5 for infants and 1:7 for ages 12-36 months. The region reported a long history of planning and financing in-service training and refresher courses for caregivers and auxiliary staff. Overall, the report commends Emilia Romagna for 20 years of regional policies that emphasized the quality and educational significance of their asili nido.

\subsection{Veneto}

In 1989, Veneto reported 121 asili nido that offered slots to 5976 children, with a total provision of care for 3.9\% of its infant-toddler population. Asili nido in Veneto were coordinated by non-degreed caregivers who rotated the extra responsibilities. Veneto reported 10 unused day care centers, and 25 centers that served ``other uses.'' The practice of in-service trainings for caregiver staff began in 1986. The 1990 report suggests that Veneto's regional laws are to blame for the construction of day care centers with state funds that are later re-purposed for other uses \citep{Becchi-Ferrari_1990_Pub-Inf-Centres-Italy}. In 2001, the OECD includes Veneto in a list of exemplary regions and municipalities providing high quality infant-toddler services. 
~\\

\section{Historical Overview of Policies, Reforms, and Provision of Early Childhood Education and Care in Italy}

\begin{itemize}

\item In 1831, Italy's first private care institutions for abandoned children under 3 years of age were charitable shelters and orphanages administered by the Catholic Church. 

\item In 1907, Maria Montessori establishes the Casa dei bambini in Rome. Her approach expands to Milan in 1915 \citep{OECD_2001_Italy-Country-Note}.

\item In 1925, a state law established the Opera Nazionale Maternita e Infanzia (ONMI), funding health and child care for poor mothers and abandoned children.

\item Pre-1950's, the "Church controlled virtually all of the preschools". During the 1957-1964 period of economic growth, the Church expanded its number of private schools to meet the socialization and educational needs of children in working families \citep{Corsaro_1996_Early-nuEdu}.

\item In the 1960's, progressive pre-primary education movements flourished in the North. In Emilia Romagna, the municipalities of Reggio Emilia, Bologna, Parma, and Modena offered experimental, progressive education in schools for children ages 3-6 years \citep{OECD_2001_Italy-Country-Note}.

\item In 1968, Law 444 formally legitimized state involvement in public and private early childhood education for ages 3-6 years. Law 444 mandated public `scuolo materna' (also `scuola per l'infanzia') to be operated by an all-female staff of young teachers\footnote{Teachers were required to be less than 35 years of age.} with a high school diploma \citep{OECD_2001_Italy-Country-Note}.

\item In 1969, `Orientamenti' required adherence by scuola materna to minimum program standards and guidelines for curricula based on then-current psychological and pedagogical research \citep{Corsaro_1996_Early-nuEdu}. 

\item In 1971, Law 1044 defined early childhood education as a social right of children. Supported by state funds, regional governments were required to plan and operate enough day care nurseries (`asili nido') to meet the local needs of mothers with infants and toddlers aged 3 to 36 months. Working mothers received priority enrollment, however, facilities were scarce as demand generally exceeded supply \citep{Saraceno_1984_Soc-Probs}.  

\item In 1972, Law 1073 provided financial support for non-state schools upon request. To be eligible for state funds, non-state school were required to offer free tuition and/or refectory rights to all or part of their students \citep{Corsaro_1996_Early-Edu}.

\item In 1977, Law 616 transferred financial responsibility for the construction of scuola materna to regional governments \citep{OECD_2001_Italy-Country-Note}.

\item In 1987, maximum teacher-child ratios (1:25 max) and working hours were dictated for scuola materna \citep{OECD_2001_Italy-Country-Note}.

\item In 1991, revisions to the Orientamenti curricular guidelines expanded the educational focus on intellectual development to emphasize social, affective and cognitive development within the context of the community and family. In this view, playing, eating, and working together are viewed as important intellectual tasks for early childhood \citep{Corsaro_1996_Early-Edu}.

\item In 1997 and 1998, laws were passed requiring university degrees and supervised professional experience for all early childhood teachers, including those in programs for ages 0-3 years \citep{Ghedini_2001_Ital-Natl-Policy}. Prior to these laws. most teacher training programs were provided by state-recognized Catholic training institutes. 

\item In 2001, the OECD reports over 95\% of all children aged 3 to 6 are enrolled in scuola materna.  
\end{itemize}

\clearpage

\begin{table}
\begin{center}
\begin{threeparttable}
\caption{Timeline of Policies, Reforms, and Provision of Early Childhood Education \\ and Care in Italy}
\label{tbl:1}
\def\arraystretch{2}
\begin{tabular}{r |@{\foo} p{10cm}}
\toprule
  1907 & Maria Montessori establishes the \textit{Casa dei bambini} in Rome.\\
  1915 & Montessori's approach expands to Milan. \\
  1925 & Opera Nazionale Maternita e Infanzia (ONMI) is established. \\
  1957-64 & Church expands number of private schools. \\
  1960's & Reggio,  Parma, Emilia Romagna, Bologna and Modena offer experimental, progressive education. \\
  1968 &  Law 444  mandates public \textit{scuolo materna} be operated by young, high school educated, all-female staff.\\
  1969 & \textit{Orientamenti} requires \textit{scuola materna} to adhere to curricula standards and guidelines. \\
  1971 & Law 1044 defines early childhood as social right of children. Regional governments  required to operate \textit{asili nido} to meet local needs of children aged 3-36 months with priorty given to working mothers. Demand exceeds supply. \\
  1972  & Law 1073 makes non-state schools eligible for financial support if school offers free tuition/refectory rights to all or part of their students. \\
  1977 & Law 616 transfers financial responsibility of construction of \textit{scuola materna} to regional governments. \\
  1987 & Guidelines for working hours and a maximum teacher-child ratio of 1:25 established for \textit{scuola materna}. \\
  1991 & Revised \textit{Orientamenti} expands educational focus to emphasize social, affective and congnitive development.\\
  1997-98 & Early childhood teachers required to possess university degrees and supervised profesional experience. \\
  2001 & Over 95\% of all chidren aged 3-6 are enrolled in \textit{scuola materna}. \\
\bottomrule
\end{tabular}
\begin{tablenotes}
\small
\item Sources: \citet{Corsaro_1996_Early-Edu}, \citet{OECD_2001_Italy-Country-Note} and \citet{Saraceno_1984_Soc-Probs}
\end{tablenotes}
\end{threeparttable}
\end{center}
\end{table}

\clearpage
Information to be investigated:
\begin{itemize}
	\item Municipal Schools of Parma
	\item Municipal Schools of Padova
	\item Catholic Schools
\end{itemize}


%%%	BIBLIOGRAPHY
\clearpage
\addcontentsline{toc}{section}{References}
\bibliographystyle{chicago}
\bibliography{heckman,sid}
\pagebreak

\end{document}  


