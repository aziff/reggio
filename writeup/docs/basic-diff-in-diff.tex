\documentclass[11pt]{article}
\usepackage[top=1in, bottom=1in, left=1in, right=1in]{geometry}
\parindent 22pt

\newcommand\independent{\protect\mathpalette{\protect\independenT}{\perp}}
\def\independenT#1#2{\mathrel{\rlap{$#1#2$}\mkern2mu{#1#2}}}

\usepackage{adjustbox}
\usepackage{amsmath}
\usepackage{amssymb}
\usepackage{appendix}
\usepackage{array}
\usepackage{authblk}
\usepackage{booktabs}
\usepackage{caption} 
\usepackage{datetime}
\usepackage{enumerate}
\usepackage{fancyhdr}
\usepackage{float}
\usepackage{graphicx}
\usepackage[colorlinks=true,linkcolor=blue,urlcolor=blue,anchorcolor=blue,citecolor=blue]{hyperref}
\usepackage{lscape}
\usepackage{mathtools}
\usepackage{multirow}
\usepackage{natbib}
\usepackage{pgffor}
\usepackage{setspace}
\usepackage{tabularx}
\usepackage{threeparttable}
\usepackage[colorinlistoftodos,linecolor=black]{todonotes}

\captionsetup[table]{skip = 2pt}

\newcolumntype{L}[1]{>{\raggedright\arraybackslash}p{#1}}
\newcolumntype{C}[1]{>{\centering\arraybackslash}p{#1}}
\newcolumntype{R}[1]{>{\raggedleft\arraybackslash}p{#1}}


\settimeformat{hhmmsstime}

\begin{document}

\title{Preliminary Difference-in-Difference Model: Adult Cohorts}
\author{Reggio Team}
\date{Original version: Thursday  16$^{\text{th}}$ June, 2016 \\ Current version: \today \\ \vspace{1em} Time: \currenttime}
\maketitle

\doublespacing

\section{Model}


We define $\mathbf{C}$ as a vector of indicators for city types, $\mathbf{K}$ is a vector for indicators for each of 5 cohorts, and  $\mathbf{S}$ is a vector for indicators for school types: 

\begin{equation*}
\mathbf{C} =\begin{bmatrix} c_1 & : \text{Reggio Emilia} \\
c_2 & : \text{Parma} \\
c_3 & : \text{Padova} 
\end{bmatrix} \qquad \mathbf{K} =\begin{bmatrix} k_1 & : \text{Children} \\
k_2 & : \text{Adolescents} \\
k_3 & : \text{Age 30}  \\
k_4 & : \text{Age 40} \\
k_5 & : \text{Age 50} 
\end{bmatrix} \qquad \mathbf{S} =\begin{bmatrix} s_1 & : \text{Municipal} \\
s_2 & : \text{State} \\
s_3 & : \text{Religion}  \\
s_4 & : \text{None} 
\end{bmatrix}
\end{equation*}

For convenience, we define $(\mathbf{A} \otimes \mathbf{B}$) as a vector that captures all possible interactions between elements in vector $\mathbf{A}$ and vector $\mathbf{B}$. Moreover, $(\mathbf{A} \otimes \mathbf{B} \otimes \mathbf{C})$ is a vector that captures all possible triple interactions among elements in vector $\mathbf{A}$, vector $\mathbf{B}$, and vector $\mathbf{C}$. In order to define a vector that includes all indicators for cities, cohorts, and school types, and possible double and triple interactions among them, we define vector $\mathbf{D}$ as following:

\begin{equation*}
\mathbf{D} =\begin{bmatrix} 
\mathbf{C} \\
\mathbf{K} \\
\mathbf{S} \\
(\mathbf{C} \otimes \mathbf{K}) \\
(\mathbf{C} \otimes \mathbf{S}) \\
(\mathbf{K} \otimes \mathbf{S}) \\
(\mathbf{C} \otimes \mathbf{K} \otimes \mathbf{S}) \\
\end{bmatrix}
\end{equation*}


We write a general model for the returns from preschool for outcome $j$, which include labor market outcomes, educational attainment, health, and civic engagement for the individual $i$, first explicitly in terms of $\mathbf{C}$, $\mathbf{K}$, $\mathbf{S}$, and necessary interaction terms. 
\begin{eqnarray} \label{eq:expanded}
y_{i,j} & = \mathbf{C}_{i} \ \mathbf{\gamma}_{i,j}^c\ +\ \mathbf{K}_{i} \mathbf{\gamma}^k_{i,j} \ +\ \mathbf{S}_{i}\  \mathbf{\gamma}^{s}_{i,j} \ +  (\mathbf{C} \otimes \mathbf{K})\mathbf{\gamma}_{i,j}^{ck} +   (\mathbf{C} \otimes \mathbf{S})\mathbf{\gamma}_{i,j}^{cs} +  (\mathbf{K} \otimes \mathbf{S})\mathbf{\gamma}_{i,j}^{ks} \nonumber \\ 
& + (\mathbf{C} \otimes \mathbf{K} \otimes \mathbf{S})\mathbf{\gamma}_{i,j}^{cks} +    \  \mathbf{X}_{i}\  \mathbf{\beta}_{i,j} \ + \ \varepsilon_{i,j}
\end{eqnarray}

\noindent where $\mathbf{X}_i$ is a vector of baseline characteristics for individual $i$. In order for the effects of baseline characteristics to vary across cities, cohorts, and school types, $\mathbf{X}_i$ needs to be interacted with every indicator variable. Moreover, since including all indicator variables results in perfect collinearity, we drop indicators for one city, one cohort, one school type, and all possible interactions with dropped indicators. We express Equation \ref{eq:expanded} in terms of $\mathbf{D}$ as

\begin{equation}
y_{i,j} = \underbrace{\mathbf{D}_{i,j} \ \mathbf{\gamma}{_{i,j}}}_{\mathclap{\text{City, Cohort, Preschool Indicators and Interactions}}} \ + \mathbf{X}_{i}\mathbf{\beta}_{i,j} \ + \ \varepsilon_{i,j}
\end{equation}


We make the assumption that the unobservables are independent of choice of preschool type, after conditioning on individual characteristics, that is: $\varepsilon_{i,j} \independent D \mid X$. Define a set of observables, $Z$, assumed to determine the latent variable determining the choice of preschool type, $D^{*}_j$, and let $\eta_j \independent \varepsilon_{\cdot} \mid Z$ be the unobsevable components of choice, where:

\begin{equation}
D^{*}_j= \Psi_j(Z) + \eta_j.
\end{equation}

\noindent Note that if we wish to use an instrumental variable approach, we would assume $Z \independent \varepsilon_{\cdot} \mid X$. For each of the $j$ possible outcomes: 

\begin{equation}
D_{j} = 1(D^{*}_{j} > 0) = 1(\mathrm{argmax} D^{*}_{j} = j) 
\end{equation}


\section{Estimation Strategy}
For estimation, we consider two routes of analysis. The first is shutting down the city effects, and estimating the difference in outcomes across cohorts for different choices in preschool types after fixing the city. The second is to shut down the effects of different cohorts, and comparing difference in outcomes across cities for different preschool types after fixing the cohort. We begin by focusing only on the adult cohorts because outcomes and baseline control variables are different for younger cohorts (children, migrants, and adolescents). In this section, we present the estimation model for a case where we fix city to explicitly show how our difference-in-difference approach work. The estimation strategy for the case where we fix cohort is analogous. 

\subsection{Estimation Model: Fixing Cohort}


In the context of our analysis, lets consider a case with 3 cities, $c = 1, 2, 3$, and 4 school types $k = 1,2,3,4$. Assuming that we restrict our sample to only age 50 cohort, we can rewrite Equation \ref{eq:expanded} for a certain outcome $y$ for any individual in that cohort as:
\begin{eqnarray}  \label{eq:specific2}
y & = \gamma^0 + \gamma^1 c_2 + \gamma^2 c_3 + \gamma^3 s_2 + \gamma^4 s_3 + \gamma^5 s_4  + \gamma^6 ({c_2}{s_2}) + \gamma^7 ({c_2}{s_3})  \nonumber \\
 & \gamma^8 ({c_2}{s_4}) + \gamma^9 ({c_3}{s_2}) + \gamma^9 ({c_3}{s_3}) +  + \gamma^9 ({c_3}{s_4}) + \mathbf{X}\beta + \varepsilon  
\end{eqnarray}

We drop $c_1$ and $s_1$ from the above equation to avoid perfect multicollinearity. To explicitly show what the interpretation for coefficients on each indicator is, we give an example of how to interpret $\gamma^1$. Assume two cases: (1) an individual lives in Reggio and attended a municipal school and (2) an individual lives in Parma and attended municipal school. The expected outcomes for those individuals are:
\begin{eqnarray*}  
    \mathbb{E}[y \mid c_1 = 1, s_1 = 1] & = & \gamma^0 + \mathbf{X}\beta + \varepsilon \\
    \mathbb{E}[y \mid c_2 = 1, s_1 = 1] & = & \gamma^0 + \gamma^1 + \mathbf{X}\beta + \varepsilon      
\end{eqnarray*}
This shows that $\gamma^1 = \mathbb{E}[y \mid c_2 = 1, s_1 = 1] - \mathbb{E}[y \mid c_1 = 1, s_1 = 1]$, which has an interpretation of ``the mean difference in outcomes between people in Parma who attended municipal schools and people in Reggio who attended municipal schools." Interpretation for coefficients on other city and school type dummies are analogous.

To show what the interpretation for coefficient on each interaction term is, we give an example of how to interpret $\gamma^6$. This can be written as:
\begin{eqnarray*}
\gamma^6 & = & \Big(\mathbb{E}[y \mid c_2 = 1, s_2 = 1] - \mathbb{E}[y \mid c_1 = 1, s_2 = 1] \Big) - \Big(\mathbb{E}[y \mid c_2 = 1, s_1 = 1] - \mathbb{E}[y \mid c_1 = 1, s_1 = 1] \Big) \\
& = & \Big((\gamma^0 + \gamma^1 + \gamma^3 + \gamma^6) - (\gamma^0 + \gamma^3)\Big) - \Big((\gamma^0 + \gamma^1) - (\gamma^0) \Big) \\
& = & (\gamma^1 + \gamma^6) - \gamma^1 \\
& = & \gamma^6
\end{eqnarray*}

Hence, $\gamma^6$ is the difference between \Big((Parma None) - (Reggio None)\Big) and \Big((Parma Muni) - (Reggio Muni)\Big). Hence, the coefficient on the interaction term shows the mean difference-in-difference across city and school type as shown above.   

\end{document}
