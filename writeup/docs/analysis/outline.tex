\documentclass{article}

\usepackage{amsmath}
\usepackage{amssymb}
\usepackage{booktabs}
\usepackage[colorinlistoftodos, linecolor=black]{todonotes}

\begin{document}

% title
\title{Outline and Todo List for Reggio Analysis}
\author{Pietro Biroli \and Jessica Yu Kyung Koh \and Anna Ziff}
\date{Original version: January 22, 2016 \\ Current version: \today}
\maketitle

\listoftodos

% outline
\section{Outline}
\begin{enumerate}

\item Background 
\todo[backgroundcolor=orange!30, size=\footnotesize]{Refer to first paper and presentation; keep it brief for now}
\begin{enumerate}
	\item Cities where the program took place
		\begin{enumerate}
			\item General information about the populations of those cities
			\item Preschool availability take-up in those cities over the years and how this compares with the rest of Italy
		\end{enumerate}
		
	\item Description of different ages preschool offered (infant-toddler care and preschool) and different types of schools (municipal, state, private, religious, none) and how availability of these have changed over the years.
		\begin{enumerate}
			\item Describe the program itself (municipal infant-toddler and preschool in Reggio Emilia) 
			\item Compare with other programs \todo[backgroundcolor=orange!30, size=\footnotesize]{Ask Sylvie for help}
		\end{enumerate}
		\end{enumerate}
	
	\item Data
	\begin{enumerate}	
		\item Cohort design
		\begin{enumerate}
			\item Describe the different cohorts in terms of ages and preschool availability
			\todo[backgroundcolor=green!30, size=\footnotesize]{Mention year of survey collection}
		\end{enumerate}
	\item Survey Data
	\todo[backgroundcolor=red!30, size=\footnotesize]{Refer to data description document}
	\begin{enumerate}
		\item General description of survey data collected, including the different sections collected for the different cohorts
		\item Variables of interest that are used in this first stage of analysis. Include description of the scales (raw or standardized) when appropriate.
		\begin{enumerate}
			\item Include descriptives of these variables \todo[backgroundcolor=orange!30,size=\footnotesize]{Include F-tests}
		\end{enumerate}
		\todo[backgroundcolor=green!30, size=\footnotesize]{Also, include a cognitive variable and variables for earnings and education.}
	\end{enumerate}
	\end{enumerate}
	
	\item Analysis
	\begin{enumerate}
		\item Main summary of findings across different specifications \todo[backgroundcolor=green!30,size=\footnotesize]{Which result is the best?}
		\item OLS 
		\begin{enumerate}
			\item Deeper description of the different results with tables and graphs \textit{with $p$-values}			
			\item List controls for each of the different regressions. 
			\item Simple OLS by asilo/materna with dummies for interaction of school type and city with religious school as the omitted category \todo[backgroundcolor=orange!30, size=\footnotesize]{To avoid collinearity omit Reggio as city}
			\item Pooled across cities \todo[backgroundcolor=red!30,size=\footnotesize]{Check uniformity of $\beta_X$ for the different controls }
			\item Pooled across types of  schools (municipal, religious, private, etc.)
		\end{enumerate}
		\item Instrumental Variable approach \todo[backgroundcolor=orange!30,size=\footnotesize]{Define instruments well}
		\item Propensity Score Matching \todo[backgroundcolor=orange!30,size=\footnotesize]{Chiara Pronzato is working on this}
	\end{enumerate}
	
\end{enumerate}

% quick notes
\section{Data Quick Facts}

\begin{table}[htbp]
\begin{center}
\caption{Values for Core Variables}
\begin{tabular}{cl}
\toprule
Value & Meaning \\
\midrule
& Cohort \\
\midrule
1 & Children \\
2 & Migrants \\
3 & Adolescents \\
4 & Adult 30 \\
5 & Adult 40 \\
6 & Adult 50 \\
\midrule
& City \\
\midrule
1 & Reggio \\
2 & Parma \\
3 & Padova \\
\midrule
& Type \\
\midrule
0 & Not attended \\
1 & Municipal \\
2 & State \\
3 & Religious \\
4 & Private \\
\bottomrule
\end{tabular}
\end{center}
\end{table}


\end{document}