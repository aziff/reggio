\documentclass[11pt]{article}
\usepackage[top=1in, bottom=1in, left=1in, right=1in]{geometry}
\parindent 22pt

\usepackage{adjustbox}
\usepackage{amsmath}
\usepackage{amssymb}
\usepackage{appendix}
\usepackage{array}
\usepackage{authblk}
\usepackage{booktabs}
\usepackage{caption} 
\usepackage{datetime}
\usepackage{enumerate}
\usepackage{fancyhdr}
\usepackage{float}
\usepackage{graphicx}
\usepackage[colorlinks=true,linkcolor=blue,urlcolor=blue,anchorcolor=blue,citecolor=blue]{hyperref}
\usepackage{longtable}
\usepackage{lscape}
\usepackage{pdfpages}
\usepackage{mathtools}
\usepackage{multirow}
\usepackage{natbib}
\usepackage{pdflscape}
\usepackage{pgffor}
\usepackage{setspace}
\usepackage{tabularx}
\usepackage{threeparttable}
\usepackage[colorinlistoftodos,linecolor=black]{todonotes}
\usepackage{enumitem}
\usepackage{xcolor}

\setlist{  
  listparindent=\parindent,
  parsep=0pt,}
  
\captionsetup[table]{skip = 2pt}

\newcolumntype{L}[1]{>{\raggedright\arraybackslash}p{#1}}
\newcolumntype{C}[1]{>{\centering\arraybackslash}p{#1}}
\newcolumntype{R}[1]{>{\raggedleft\arraybackslash}p{#1}}

\newcommand{\highlight}[1]{\colorbox{lightgray!50}{\makebox(20,4){#1}}}

\newcommand{\fnDID}{\underline{Note:} This table shows difference in difference across school types and cities with the sample restricted to a specific age cohort. For convenience, we denote "Reggio None" as individuals in Reggio who did not attend any materna school. Notations are analogous across city and school type. Each column shows the following diff-in-diff estimate. \textbf{(1)} (Reggio Muni - Reggio None) - (Parma Muni - Parma None), \textbf{(2)} (Reggio Muni - Reggio State) - (Parma Muni -  Parma State), \textbf{(3)} (Reggio Muni - Reggio Reli) - (Parma Muni - Parma Reli), \textbf{(4)}(Reggio Muni - Reggio None) - (Padova Muni - Padova None),  \textbf{(5)}  (Reggio Muni - Reggio State) - (Padova Muni - Padova State), \textbf{(6)}  (Reggio Muni - Reggio Reli) - (Padova Muni - Padova Reli). Bold number indicates statistical significance at the 10\% level. Standard errors are reported in parentheses.}


\newcommand{\fnOLS}{This table presents conditional and unconditional OLS results for our outcomes of interest. For each school type, the ``C. Mean" section shows the conditional means and the ``Mean" section shows the unconditional means. Bold numbers indicate statistical significance at the 10\% level.}

\newcommand\independent{\protect\mathpalette{\protect\independenT}{\perp}}
\def\independenT#1#2{\mathrel{\rlap{$#1#2$}\mkern2mu{#1#2}}}


\settimeformat{hhmmsstime}

%--------------------------------------------------------------------------------------
\begin{document}

\title{Effects of the Reggio Approach on Adult Cohorts: Estimation Strategy and Results}
\author{Reggio Team}
\date{Original version: Thursday  16$^{\text{th}}$ June, 2016 \\ Current version: \today \\ \vspace{1em} Time: \currenttime}
\maketitle

\doublespacing

\section{Introduction}

This document presents the estimation strategies and summarizes estimation results across different methodologies. This document focuses on adult cohorts and how their materna decisions might have affected their adult outcomes. We use the following two different approaches in comparing the outcomes across people who attended different types of preschool: (1) OLS with and without controls, (2) difference-in-difference with controls. For the summary of outcomes, we focus on outcomes that show consistently significant differences between the Reggio Approach materna schools and non-Reggio Approach childcare options. Tables with the full estimation results are included in the appendix. 

\section{Methodology}
We begin by specifying a basic single-outcome OLS model. In the following sections, we expand this model to include $j$ possible outcomes, city, cohort and preschool-type fixed effects, and estimators of difference-in-difference effects.

Consider the determinants for individual, $i$'s, lifecycle outcomes which might include labor market participation, social participation, civic engagement, health and mental health, and non-cognitive skills. Each outcome is considered to be determined by the individual's baseline characteristics, $X$, which includes family background, household income, ethnic and religious identity, and other individual characteristics. For this simplest  case, considering only one possible outcome, $y_1$, we specify the following relationship  where $\varepsilon_{i}$ is an individual-specific disturbance that we assume to be independent from the outcome after conditioning on the controls, and $c_{i,1} = \mathbf{1}(C_i = c_1)$ is a dummy indicator for person, $i$, living in city 1.

\begin{equation} \label{ols}
y_{i,1} =  X_i \beta' + \gamma_1 c_{i,1} + \varepsilon_{i}
\end{equation}

Consider the simple example of two cities, cohorts, and preschool types. The baseline OLS model ignores effects from cohort or type of preschool. We define a reference city, and estimate the effect of not attending preschool in that city, which assumes that all preschool types are perfect substitutes within the respective cities.  The estimated $\tilde{\beta}$ can be interpreted as the differences in $Y_i$ that are explained by observed characteristics, and  $\tilde{\gamma_1}$ is the estimated effect of not living in the reference city. 


\subsection{Preschool's Effect on Lifecycle Outcomes -- Fixed Effect Model}
Starting from this baseline specification, we consider the set of $j$ possible outcomes, $\{y_1, \dots, y_j\}$, and account for the possibility of city, $c$, cohort, $k$, and preschool-type, $s$ fixed-effects. %, along with all possible two-way\footnote{i.e., The difference-in-difference effects} and all possible three way effects\footnote{i.e., The triple difference (or diff-in-diff-in-diff effects)}. This system is represented below:

%\begin{subequations}
%\begin{align}
%y_{i,1} & = \beta_{1}x_i  + \gamma^c_{1} c_i + \gamma^k_{1} k_i + \gamma^s_{1} s_i + \gamma^k_{1} k_i + \gamma^{c,k}_{1} (c_i \times k_i) + \gamma^{k,s}_{1} (k_i \times s_i) + \gamma^{c,s}_{1} (c_i \times s_i) + \\
%&\qquad \qquad \qquad \qquad \qquad \qquad \qquad \qquad \qquad \qquad \qquad \qquad +\gamma^{c,k,s}_{1} (c_i \times k_i \times s_i) + \varepsilon_{i,c,k,s,1}  \nonumber \\
%y_{i,2} & = \beta_{2}x_i  + \gamma^c_{2} c_i + \gamma^k_{2} k_i + \gamma^s_{2} s_i + \gamma^k_{2} k_i + \gamma^{c,k}_{2} (c_i \times k_i) + \gamma^{k,s}_{2} (k_i \times s_i) + \gamma^{c,s}_{2} (c_i \times s_i) + \\
%\vdots & \qquad \qquad \qquad \qquad \qquad \qquad \vdots \qquad \qquad \qquad \qquad \qquad \qquad +\gamma^{c,k,s}_{2} (c_i \times k_i \times s_i) + \varepsilon_{i,c,k,s,2}  \nonumber \\
%\vdots & \qquad \qquad \qquad \qquad \qquad \qquad \vdots \qquad \qquad \qquad \qquad \qquad \qquad \qquad \qquad \vdots  \nonumber \\
%y_{i,j} & = \beta_{j}x_i + \gamma^c_{j} c_i + \gamma^k_{j} k_i + \gamma^s_{j} s_i + \gamma^k_{j} k_i + \gamma^{c,k}_{j} (c_i \times k_i) + \gamma^{k,s}_{j} (k_i \times s_i) + \gamma^{c,s}_{j} (c_i \times s_i) + \\
%& \qquad \qquad \qquad \qquad \qquad \qquad \qquad \qquad \qquad \qquad \qquad \qquad +\gamma^{c,k,s}_{j} (c_i \times k_i \times s_i) + \varepsilon_{i,c,k,s,j}  \nonumber
%\end{align}
%\end{subequations}

Defining $Y_i = [y_1, \dots , y_j]'$ as the vector of outcomes, $C_i$ as a $\ell \times 1$ vector indicating city where $c_{a} = \mathbf{1}(C_i = c_a)\; \forall a \in \{1, \dots, {\ell} \}$, $K_i$ as a $m \times 1$ vector indicating the individual's cohort where $k_a = \mathbf{1}(K_i = k_a)\; \forall a \in \{1, \dots, m\}$, and  $S_i$ as a $n \times 1$ vector indicating school type where $s_a = \mathbf{1}(S_i = s_a)\; \forall a \in \{1, \dots, n \}$, we can write a fixed-effects model as follows:

%.  Ignoring the interaction terms for now, we can rewrite the system of equations for $\{y_{i,1}, \dots, y_{i,j}\}$ in terms of $Y_i,\ \mathbf{c},\ \mathbf{k}, \text{ and, } \mathbf{s}$ as: 

\begin{equation} \label{eq:fixed}
Y_i = X_i \boldsymbol{\beta}' + \boldsymbol{C_i} \boldsymbol{\gamma}_c' + \boldsymbol{K_i} \boldsymbol{\gamma}_k' + \boldsymbol{S_i} \boldsymbol{\gamma}_s'  + \boldsymbol{\varepsilon}_{i}
\end{equation}

Note that this is a standard OLS regression model that includes city-specific fixed effects, $ \boldsymbol{C_i \gamma}_c$, cohort-specific fixed effects,  $ \boldsymbol{K_i \gamma}_k$, and preschool-type fixed effects, $ \boldsymbol{S_i \gamma}_s$. Again for the case where $\ell = m = n = 2$, the coeffcients on $C, K, S$ can be interpreted as the specific effects from an individual not being part of the respective reference group. To illustrate, we restrict our sample to only those individuals from city 1. In this case,  $\tilde{\gamma}_c = 0$	 as there is no variation in city when the city is fixed. However, $ \tilde{\gamma}_s, \tilde{\gamma}_k \neq 0$ and can be interpreted as the effect on an individual for being different from the reference school and cohort. 

\subsection{Preschool's Effect on Lifecycle Outcomes -- Difference-in-difference Model}

We also expand this model to include two-way interactions between, $C_i,$ $K_i,$ and $S_i$ and define the respective coefficients as the difference-in-difference estimator, and finally, the three-way interaction, $C_i \times K_i \times S_i$, and define the coefficients as the difference-in-difference-in-differences estimator.  We now define the two-- and three-way interaction terms. Let $M_i = C_i \otimes K_i$,  $N_i = C_i \otimes S_i$, $Q_i = K_i \otimes S_i$, and let $T_i = C_i \otimes K_i \otimes S_i$:\footnote{The kronecker product of three $\ell \times 1$, $m \times 1$, and $n \times 1$ yields a $(\ell \times m \times n) \times 1$ long column vector of the desired interactions. (Turkington 2002, 2013)}  : where the vectors of interaction terms are given as:

\begin{equation*}
\underset{\begin{bmatrix} m_1 = c_1 k_1 \\
\vdots  \\
m_m = c_1 k_m  \\
\vdots  \\
m_{\ell m} = c_{\ell} k_m 
\end{bmatrix}}{M_i = C_i \otimes K_i  } \qquad \qquad
 \underset{\begin{bmatrix} n_1 = c_1 s_1 \\
\vdots  \\
n_n = c_1 s_n  \\
\vdots  \\
n_{\ell n} = c_{\ell} s_n 
\end{bmatrix}}{N_i = C_i \otimes S_i} \qquad \qquad
\underset{\begin{bmatrix} q_1 = k_1 s_1 \\
\vdots  \\
q_m = k_1 s_n  \\
\vdots  \\
q_{m n} = k_{m} s_n 
\end{bmatrix}}{Q_i = K_i \otimes S_i } \qquad \qquad
\underset{\begin{bmatrix} t_1 = c_1 k_1 s_1  \\
\vdots  \\
t_{\ell n} = c_{\ell} k_1 s_n  \\
\vdots  \\
t_{\ell n m} = c_{\ell} k_m s_n  \\
\end{bmatrix}}{T_i = C_i \otimes Q_i \\
= C_i \otimes K_i \otimes S_i }
\end{equation*}


Specifying our model in terms of the $j \times 1$ vector of outcomes, $Y_i$, the vector of City types, the vector of Cohort identifiers, the vector of preschool types, the $M,N,Q$ vectors of two way interaction terms, and the $T$ vector of three way interactions, we get:

\begin{equation} \label{eq:diffs}
Y_i = \underbrace{X_i \boldsymbol{\beta}'}_{\text{Individual Covariates}} + \underbrace{C_i \boldsymbol{\gamma}_c' + K_i \boldsymbol{\gamma}_k' + S_i \boldsymbol{\gamma}_s'}_{\text{City, Cohort, School F.E.}}  +\; \underbrace{M_i \boldsymbol{\gamma}_m' + N_i \boldsymbol{\gamma}_n' + Q_i \boldsymbol{\gamma}_q'}_{\text{Diff-in-Diff}} + \underbrace{T_i \boldsymbol{\gamma}_t'}_{\text{Triple difference}} \; + \; \boldsymbol{\varepsilon}_{i}
\end{equation}

For notational convenience, we stack the vectors $C_i, K_i, S_i, M_i, N_i, Q_i, \text{ and } T_i$ into the long column vector, $D_i$, and similarly stack $\gamma_c, \gamma_k, \gamma_s, \gamma_m, \gamma_n, \gamma_q, \gamma_t $ into a vector of equal length composed of all the dummy variable coefficients, which is defined  as $\gamma_d$. Rewriting \ref{eq:diffs} in this simplified notation gives:

\begin{equation}
Y_i = X_i \boldsymbol{\beta}' + D_i \boldsymbol{\gamma}_d' + \boldsymbol{\varepsilon}_{i}
\end{equation}

The full specification which includes the difference-in-differences (and difference-in-difference-in-differences) estimators addresses concerns that \ref{eq:fixed} does not reflect the difference in cohort structure or in preschool categories that may exist in different cities, or that $(S_i = 1 \mid C_i = 0) \neq (S_i = 1 \mid C_1 =1)$. Again considering the simple case of $\ell = m = n = 2$, the estimated model can be written as:
\begin{eqnarray*}  \label{eq:specific2}
Y_i & = & \alpha_0  + \gamma_c (C_i = 1) + \gamma_k (K_i = 1) + \gamma_s (S_i = 1) \nonumber \\
& &\ + \gamma_{c,k} (C_i = 1 \times K_i = 1) + \gamma_{k,s} (K_i = 1 \times S_i = 1)  + \gamma_{c,s} (C_i = 1 \times S_i = 1)   \nonumber \\
 & &\ + \gamma_{c,k,s}(C_i = 1 \times K_i = 1 \times S_i = 1) + X_i \boldsymbol{\beta} + \varepsilon_i  
\end{eqnarray*}

Where $\alpha_0$ is the intercept corresponding to being a member of the reference group, i.e. -- $(C_i = 0, K_i = 0, S_i = 0)$   The interpretation for the coefficients on each indicator is best understood in terms of the expected outcomes implied by \ref{eq:diffs}. We consider the interpretation of $\gamma_c, \gamma_{c,k}, \text{ and }, \gamma_{c,k,s}$ explicitly. 
\begin{eqnarray*}  
    \mathbb{E}[Y_i \mid C_i = 0, K_i = 0, S_i = 0] & = & \alpha_0 + X_i \boldsymbol{\beta} + \varepsilon_i \\
    \mathbb{E}[Y_i \mid C_i = 1, K_i = 0, S_i = 0] & = & \alpha_0 +  \gamma_c + X_i \boldsymbol{\beta} + \varepsilon_i \\
    \mathbb{E}[Y_i \mid C_i = 1, K_i = 1, S_i = 0] & = & \alpha_0 + \gamma_c + \gamma_k + \gamma_{c,k} + X_i \boldsymbol{\beta} + \varepsilon_i \\
    \mathbb{E}[Y_i \mid C_i = 1, K_i = 1, S_i = 1] & = & \alpha_0 + \gamma_c + \gamma_k + \gamma_s + \gamma_{c,k} + \gamma_{c,s} + \gamma_{k,s} + \gamma_{c,k,s} + X_i \boldsymbol{\beta} + \varepsilon_i
\end{eqnarray*}
Thus, $\gamma_c$ can be interpreted as ``the mean difference in outcomes'' for individuals attending a comparable reference preschool type in the ``treatment" city compared to those in the reference city.  Interpretation for coefficients on other city and school type dummies are analogous. Likewise, $\gamma_{c,k}$ compares how effects on outcomes are different between the two city categories, when moving across cohorts. $\gamma_{c,k} > 0$ would imply that the improvement in outcomes between the two cohorts increased, on average, for the ``treatment city" more so than it did for the reference category. Consider an example where, $C_i = 0$ if the individual attended no preschool and  $C_i = 1$ if they attended any preschool. Then $\gamma_{c,k,s} > 0$ can be understood as the comparison city, $C_i = 1$, having larger growth in the average difference in outcomes between younger and older cohorts, for those attending preschool compared to those with none.

\begin{eqnarray*}  
\boldsymbol{\gamma_c} & = &  \Big(\overline{Y}_i \mid C_i = 1\Big) - \Big(\overline{Y}_i \mid C_i = 0\Big) \\ [0.4em]
%& = & (\mathbb{E}[Y_i \mid C_i = 1, K_i = 0, S_i = 0]  - \mathbb{E}[Y_i \mid C_i = 0, K_i = 0, S_i = 0]) & \\
\boldsymbol{\gamma_{c,k}} & = & \Bigg[\Big(\overline{Y_i} \mid K_i = 1\Big) - \Big(\overline{Y_i} \mid K_i = 0\Big) \Big| C_i =1 \Bigg] - \Bigg[\Big(\overline{Y_i} \mid K_i = 1) - \Big(\overline{Y_i} \mid K_i = 0\Big) \Big| C_i = 0 \Bigg] \\[0.6em]
% & = & (\mathbb{E}[Y_i \mid C_i = 1, K_i = 1, S_i = 0]  - \mathbb{E}[Y_i \mid C_i = 1, K_i = 0, S_i = 0] ) & \\
%& & - (\mathbb{E}[Y_i \mid C_i = 0, K_i = 1, S_i = 0]  -  \mathbb{E}[Y_i \mid C_i = 0, K_i = 0, S_i = 0]) & \\
\boldsymbol{\gamma_{c,k,s}} & = & \Bigg(\Big[\Big(\overline{Y_i} \mid S_i = 1\Big) - \Big(\overline{Y_i} \mid S_i = 0\Big)\Big| C_i =1, K_i = 1 \Big] - \Big[\Big(\overline{Y_i} \mid S_i = 1\Big) - \Big(\overline{Y_i} \mid S_i = 0\Big)\Big| C_i =1, K_i = 0 \Big]\Bigg) \\
&  - & \Bigg(\Big[\Big(\overline{Y_i} \mid S_i = 1\Big) - \Big(\overline{Y_i} \mid S_i = 0\Big)\Big| C_i =0, K_i = 1 \Big] - \Big[\Big(\overline{Y_i} \mid S_i = 1\Big) - \Big(\overline{Y_i} \mid S_i = 0\Big)\Big| C_i =0, K_i = 0 \Big]\Bigg)
 %& = &(\mathbb{E}[Y_i \mid C_i = 1, K_i = 1, S_i = 1]  - \mathbb{E}[Y_i \mid C_i = 1, K_i = 1, S_i = 0]) & \\
%& & - (\mathbb{E}[Y_i \mid C_i = 0, K_i = 1, S_i = 1]  - \mathbb{E}[Y_i \mid C_i = 0, K_i = 1, S_i = 0]) & \\
%& & - (\mathbb{E}[Y_i \mid C_i = 1, K_i = 0, S_i = 1]  - \mathbb{E}[Y_i \mid C_i = 1, K_i = 0, S_i = 0]) &
\end{eqnarray*}


\section{Estimation Strategy}
\subsection{OLS Model}
The purpose of the OLS model is to compare outcomes of individuals in the different cities. We estimate the OLS model in two channels. We first estimate the OLS model without including controls to capture the uncontrolled mean differences in outcomes among groups of people who attended different types of alternative preschools for each city and each age cohort. Formally written,
\begin{equation} \label{OLS-nocontrol}
	y_{i} = \gamma_0 + \gamma_1 s_{i,2} + \gamma_2 s_{i,3} + \gamma_3 s_{i,4} + \gamma_4 s_{i,5} + \varepsilon_{i}, i \in I := \{ \text{individuals in city $j$ and age cohort $h$}\}
\end{equation}
\noindent where $i$ indexes over all individuals in the three cities, $y_{i}$ is an outcome of interest, $s$ is a type of materna school as shown in its subscript, $\varepsilon_{i}$ is an individual disturbance that we assume to be independent from the outcome variable. An indicator for attending municipal school type for individual $i$, $s_{i,1}$, is dropped due to collinearity. 

We also estimate the above OLS model controlling for baseline characteristics to capture the controlled mean differences in outcomes among different groups. It is written as:
\begin{equation} \label{OLS-control}
	y_{i} = \gamma_0 + \gamma_1 s_{i,2} + \gamma_2 s_{i,3} + \gamma_3 s_{i,4} + \gamma_4 s_{i,5} + \mathbf{X}\beta + \varepsilon_{i}, i \in I := \{ \text{individuals in city $j$ and age cohort $h$}\}	
\end{equation}
where $\mathbf{X}$ is a set of five control variables that have the lowest BIC score among all possible sets out of the baseline variables. 


\subsection{Difference in Difference Model}
For difference-in-difference estimation, we consider the following route of analysis. We shut down the effects of different cohorts, and compare differences in outcomes across cities for different preschool types after fixing the cohort. We focus this document only on the adult cohorts because outcomes and baseline control variables are different for younger cohorts (children, migrants, and adolescents). 

\subsubsection{Estimation Model: Fixing Cohort}

Let's consider a case with 3 cities, denoted by the number in subscript of $c$, and 4 school types, denoted by the number in subscript for $s$. Assuming that we restrict our sample to only the age 50 cohort, we can write our model for a certain outcome $y$ as:
\begin{eqnarray}  \label{eq:specific2}
y_i & = \gamma_0 + \gamma_1 c_{i,2} + \gamma_2 c_{i,3} + \gamma_3 s_{i,2} + \gamma_4 s_{i,3} + \gamma_5 s_{i,4}  + \gamma_6 ({c_{i,2}}\cdot{s_{i,2}}) + \gamma_7 ({c_{i,2}}\cdot{s_{i,3}})  \nonumber \\
 & \gamma_8 ({c_{i,2}}\cdot{s_{i,4}}) + \gamma_9 ({c_{i,3}}\cdot{s_{i,2}}) + \gamma_{10} ({c_{i,3}}\cdot{s_{i,3}}) +  + \gamma_{11} ({c_{i,3}}\cdot{s_{i,4}}) + \mathbf{X}\beta + \varepsilon_i  
\end{eqnarray}
We drop $c_1$ and $s_1$ from the above equation to avoid perfect multicollinearity. Our $\mathbf{X}$, which is a vector of controls, is the selected set of 5 variables that has the lowest BIC score. For the employment and income category, however, we additionally control for each person's occupation.

\subsubsection{Interpreting the difference estimator}

We provide an interpretation of $\gamma_1$ to demonstrate how the simple difference estimators in our model should be interpreted. Assume two cases: (1) an individual lives in Reggio and attended a municipal school and (2) an individual lives in Parma and attended municipal school. The expected outcomes for those individuals are:
\begin{eqnarray*}  
    \mathbb{E}[y \mid c_1 = 1, s_1 = 1] & = & \gamma_0 + \mathbf{X}\beta + \varepsilon_i \\
    \mathbb{E}[y \mid c_2 = 1, s_1 = 1] & = & \gamma_0 + \gamma_1 + \mathbf{X}\beta + \varepsilon_i      
\end{eqnarray*}
This shows that $\gamma_1 = \mathbb{E}[y \mid c_2 = 1, s_1 = 1] - \mathbb{E}[y \mid c_1 = 1, s_1 = 1]$, which can be interpreted as ``the mean difference in outcomes between people in Parma who attended municipal schools and people in Reggio who attended municipal schools." While the simple difference estimator is informative, we cannot use it to interpret the treatment effect of attending municipal school in Reggio because it includes confounding effects from permanent average differences in baseline characteristics between individuals of Reggio and Parma. Interpretation for coefficients on other city and school type dummies are analogous.

\subsubsection{Interpreting the difference-in-difference estimator}
We now provide an interpretation of $\gamma_6$ to illustrate how the diff-in-diff estimators should be interpreted in our model. Assume four cases: (1) an individual lives in Reggio and attended a municipal preschool, (2) an individual lives in Reggio and didn't attend preschool, (3) an individual lives in Parma and attended municipal preschool, and (4) an individual lives in Parma and did not attend preschool. The expected outcomes for these individuals can be arranged in the following manner to yield $\gamma_6$:
\begin{eqnarray*}
\gamma_6 & = & \Big( \mathbb{E}[y \mid c_1 = 1, s_1 = 1]  - \mathbb{E}[y \mid c_1 = 1, s_2 = 1] \Big) - \Big(\mathbb{E}[y \mid c_2 = 1, s_1 = 1] - \mathbb{E}[y \mid c_2 = 1, s_2 = 1] \Big) \\
& = & \Big( (\gamma_0)  - (\gamma_0 + \gamma_3)\Big) - \Big((\gamma_0 + \gamma_1) - (\gamma_0 + \gamma_1 + \gamma_3 + \gamma_6)\Big) \\
& = & (\gamma_3) - ( \gamma_3 - \gamma_6) \\
& = & \gamma_6
\end{eqnarray*}
Hence, $\gamma_6$ is the difference between \Big((Reggio Muni) - (Reggio None)\Big) and \Big((Parma Muni) - (Parma None)\Big). The first difference captures the degree by which Municipal educated students underperform or outperform those who did not attend any preschool in Reggio. The second difference captures this same effect in Parma. We can thus, interpret $\gamma_6$ as a comparison of the treatment effects of attending Municipal school over not attending any preschool  between Reggio and Parma.

\section{Summary of Estimation Results}
\subsection{Education}

Compared to other individuals, Municipal educated individuals from Reggio generally have lower IQ, higher high school grades, mixed results for University grades, and lower likelihood of graduation.
\begin{itemize}
\item \textbf{IQ Factor:} \\
Reggio Municipal students consistently underperform those who didn't attend preschool in Reggio, and tend to have lower IQ scores than most individuals from Parma and Padova.
	\begin{itemize}
	\item \textbf{OLS results:} Comparing different materna types within Reggio, the conditional mean (C.Mean) estimates of Table \ref{table:OLS_E} show that the mean IQ for individuals who attended Municipal preschools in Reggio was significantly higher than the mean IQ for those who attended state schools and significantly lower than those who attended Private school or didn't attend preschool in the age 30 cohort. Most of the estimates become insignificant at the age 40 level and we only see that Municipal educated individuals from Reggio have significantly lower mean IQ than those who didn't attend any preschool in Reggio.
	
	Comparing across cities, the C.Mean estimates of Table \ref{table:OLS_E} show that Municipal educated individuals from Reggio have significantly lower mean IQ than individuals from all materna types in Parma and Padova at the age 30 cohort. At the age 40 cohort, Reggio Municipal schools similarly significantly underperform all materna types in Parma except State schools. Reggio Municipal schools only significantly underperform Municipal schools in Padova at this age level.

	\item \textbf{Diff-in-Diff results:} For the age 30 cohort, Tables  \ref{table:OLS_E} and \ref{table:ECh-30} show that Municipal schools underperform Religious schools in all three cities, and that the degree of this underperformance is significantly larger in Reggio than in Parma or Padova. For the age 40 cohort, this diff-in-diff effect persists in Parma and becomes insignificant in Padova. Furthermore, we detect that \textbf{(i)} Municipal educated students attain higher mean IQ scores than state educated students in both Reggio and Padova, and that this difference is significantly smaller in Reggio than in Padova, and \textbf{(ii)} the degree by which Municipal educated individuals underperform those who haven't attended preschool is lower in Reggio than in Padova.

	\end{itemize}
	
\item \textbf{High school grades:} \\
Municipal students from Reggio generally tend to outperform individuals from Parma and Padova. Consistent patterns over the different cohorts aren't detected when looking within Reggio.
	\begin{itemize}
	\item \textbf{OLS results:} Looking within Reggio, Table \ref{table:OLS_E} shows that Municipal students from the age 30 cohort have significantly higher mean high school grades than Reggio individuals who attended State or Religious preschool, and significantly lower grades than those who attended private preschool. At age 40, Municipal students from Reggio only significantly outperform those who didn't attend any preschool, and underperform those who attended State or Private preschool. 

Comparing across cities, at the age 30 cohort, Municipal students from Reggio  have significantly higher mean high school grades than all individuals in Padova and those who attended Municipal, State and Religious schools in Parma. At age 40,  Municipal students from Reggio  have significantly higher mean high school grades than all individuals in Padova except those who attended State schools, and all individuals in Parma except those who attended Religious schools.

	\item \textbf{Diff-in-diff results:}  Significant diff-in-diff effects aren't detected for high school grades at the age 30 cohort.	For the age 40 cohort, Tables \ref{table:OLS_E} and \ref{table:ECh-40} show that Municipal schools outperform religious schools in Reggio while the opposite effect is true in Parma, and that the difference of these differences is significant. Furthermore, we see that Municipal schools underperform state schools in both Reggio and Padova, and that this underperformance is significantly higher in Padova than in Reggio.

\end{itemize}


\item \textbf{University grades:} \\
Municipal students from Reggio generally have significantly higher mean grades than individuals from Parma and Padova in the age 30 cohort, and the reverse is true at the age 40 cohort. Consistent patterns over cohorts aren't detected when looking within Reggio.
	\begin{itemize}
	\item \textbf{OLS results:} Looking within Reggio, Table \ref{table:OLS_E} shows that Municipal students from the age 30 cohort have significantly higher mean university grades than Reggio individuals who attended private or no preschool, and significantly lower grades than those who attended state school. For the age 40 cohort, Municipal students from Reggio have significantly higher mean university grades than Reggio individuals who attended state preschool, and  lower grades than those who attended religious school.

	Comparing across cities, at the age 30 cohort, Municipal students from Reggio have significantly higher mean grades than all individuals in Parma except those who attended private and no preschool, and all individuals in Padova except those who attended state and private schools. At the age 40 cohort, Municipal students from Reggio have significantly lower mean grades than all individuals in Parma except those who attended private school, and all individuals in Padova except those who attended state and private schools. 

	\item \textbf{Diff-in-diff results:} For the age 30 cohort, Tables \ref{table:OLS_E} and \ref{table:ECh-30} show Municipal schools outperform religious schools in Reggio while the opposite is true in Parma and Padova, and that the difference of these differences is significant. For the age 40 cohort, Tables \ref{table:OLS_E} and \ref{table:ECh-40} show that Municipal educated students underperform those who didn't attend preschool in Reggio while the opposite is true in Parma, and that the difference of these differences is significant.

	\end{itemize}

\item \textbf{Graduation from High School:} \\
Within Reggio, Municipal students are less likely to graduate than state students in both age 30 and 40 cohorts. Municipal students from Reggio generally have lower likelihoods of graduation than individuals in Parma and Padova.
	\begin{itemize}
	\item \textbf{OLS results:} Looking within Reggio, Table \ref{table:OLS_E} shows that Municipal students from the age 30 cohort have significantly higher mean likelihood of graduating high school than those who attended no preschool, and lower likelihoods than state and private students. For the age 40 cohort, Municipal students from Reggio have significantly higher likelihood of graduating high school than private students, and lower likelihood than state students. 

	Comparing across cities, at the age 30 cohort, Municipal students from Reggio have significantly lower likelihoods of graduation than all individuals from Parma and Padova. At the age 40 cohort, Municipal students from Reggio have significantly lower likelihoods of graduation than all individuals from Parma and Padova except those who attended state and private schools in these cities. 

	\item \textbf{Diff-in-diff results:} Significant diff-in-diff effects aren't detected for high school graduation at the age 30 cohort. For the age 40 cohort, Tables \ref{table:OLS_E} and \ref{table:ECh-40} shows that
Municipal educated students are less likely to graduate high school than those who didn't attend preschool in both Reggio and Parma, and that this difference in likelihood is significantly greater in Reggio than in Parma. 
Furthermore, Municipal students have lower likelihoods of graduation than State students in Reggio while the opposite is true in Parma, and the difference in these differences is significant.
	\end{itemize}

\item \textbf{Graduation from University:} \\
Within Reggio, Municipal students are more likely to graduate than religious students in both age 30 and 40 cohorts. Municipal students from Reggio generally have lower likelihoods of graduation than individuals in Parma and Padova.
	\begin{itemize}
	\item \textbf{OLS results:}  Looking within Reggio, Table \ref{table:OLS_E} shows that Municipal students from the age 30 cohort have significantly higher mean likelihood of graduating university than those who attended religious preschool, and lower likelihoods than those who attended state or no preschool. For the age 40 cohort, Municipal students from Reggio have significantly higher likelihood of graduation than students from all other materna types in Reggio. 

	Comparing across cities, at the age 30 cohort, Municipal students from Reggio have significantly lower likelihoods of graduation than all individuals from Parma except those who attended Religious schools, and all individuals from Padova except those who didn't attend preschool. At the age 40 cohort, Municipal students from Reggio have significantly lower likelihoods of graduation than all individuals from Parma except those who didn't attend preschool, and all individuals from Padova except those who attended state or private schools. 

	\item \textbf{Diff-in-diff results:} For the age 30 cohort, Tables \ref{table:OLS_E} and \ref{table:ECh-30} show that Municipal schools outperform Religious schools in Reggio while the opposite is true in Parma, and that the difference of these differences is significant. We also find that Municipal students underperform those who didn't attend preschool in Reggio while the opposite is true in Padova, and that the difference of these differences is significant. 

For the age 40 cohort, Tables \ref{table:OLS_E} and \ref{table:ECh-40} show that Municipal students outperform those who didn't attend preschool in both Reggio and Parma, and that this difference is greater in Parma. We also find that Municipal students outperform state students in Reggio while the opposite is true in Padova, and that the difference of these differences is significant. 

	\end{itemize}

\end{itemize}

\subsection{Employment and Earnings}

\begin{itemize}

\item \textbf{Hours Worked:} 
	\begin{itemize}
	\item \textbf{OLS results:}
Comparing materna types within Reggio, Table \ref{table:OLS_W} shows that Municipal students from the age 30 cohort have significantly higher mean hours worked per week than Reggio individuals who attended state and religious schools, and lower hours than those who attended private schools. For the age 40 cohort, Municipal students from Reggio worked significantly lower hours than state and religious students.

	Comparing across cities, at the age 30 cohort, Municipal students from Reggio work significantly longer hours than all individuals in Parma except those who didn't attend preschool, and all individuals in Padova except those who attended state or private school. For the age 40 cohort,  Municipal students from Reggio work significantly shorter hours than those who attended private school in Parma, and significantly longer hours than all other Parma individuals. Municipal students from Reggio work significantly shorter hours than Padova individuals who attended Municipal, Religious or no preschool.

	\item \textbf{Diff-in-diff results:} For the age 30 cohort, Tables \ref{table:OLS_W} and \ref{table:WCh-30} show that Municipal students work longer hours than those who didn't attend preschool in all three cities, and that this effect is greater in Reggio than Parma and Padova. 

For the age 40 cohort, Tables \ref{table:OLS_W} and \ref{table:WCh-40} show that the same diff-in-diff effect persists between Reggio and Parma. We see that a diff-in-diff effect still exists between Reggio and Padova, but at this age level, Municipal students work longer hours than those who didn't attend preschool in Reggio while the opposite is true in Padova. Lastly, we see that Municipal students work shorter hours than state students in Reggio while the opposite is true in Padiva, and that this diff-in-diff effect is significant.

	\end{itemize}
	
\end{itemize}  

\subsection{Household Information}
We find consistent differences between Municipal educated individuals from Reggio and other individuals in terms of house ownership and marriage and cohabitation.

\begin{itemize}

\item \textbf{Own House:} \\
Within Reggio, Municipal students are gernerally more likely to own a house than other Reggio individuals at age 30, and the reverse is true for the age 40 cohort. Municipal students from Reggio are generally less likely to own a house than individuals from Parma and Padova.
	\begin{itemize}
	\item \textbf{OLS results:}
Comparing materna types within Reggio in the age 30 cohort, Table \ref{table:OLS_L} shows that Municipal students have significantly higher mean likelihoods of owning a house than Reggio individuals who attended religious, private or no preschool, and significantly lower likelihood than those who attended state school.  For the age 40 cohort, Municipal students from Reggio were significantly less likely to own a house than students from all other materna types within Reggio.

Comparing across cities in the age 30 cohort, Table \ref{table:OLS_L} shows that Municipal educated individuals from Reggio are less likely to own a house than individuals from all materna types in Parma and Padova. In the age 40 cohort, individuals who attended Municipal school in Reggio are less likely to own a house than Municipal, Religious and Private educated individuals in Parma, and those who attended Municipal, State, Religious or no preschool in Padova.
	
	\item \textbf{Diff-in-diff results:} For the age 40 cohort, Table \ref{table:LCh-40} shows that there is a difference in the likelihood of house ownership between Municipal educated individuals and those who did not attend preschool, and that this difference in likelihood in Reggio is significantly different than the difference in Parma and Padova. Examination of Table \ref{table:OLS_L} shows that Municipal educated individuals are less likely to own a house than those who didn't receive any preschool education in Reggio, and that this relationship is reversed in Parma and Padova.
	\end{itemize}
		
\item \textbf{Married or Cohabitating:} 
	\begin{itemize}
	\item \textbf{OLS results:}
Comparing materna types within Reggio in the age 30 cohort, Table \ref{table:OLS_L} shows that Municipal students have significantly higher mean likelihoods of marriage or cohabitation than Reggio individuals who attended state preschool, and significantly lower likelihoods than those who received Religious or Private education. For the age 40 cohort,  Municipal students have significantly higher mean likelihoods of marriage or cohabitation than  individuals who attended state or religious preschool, and significantly lower likelihoods than those who received a Private education. 

	Comparing across cities in the age 30 cohort, Table \ref{table:OLS_L} shows that Municipal educated individuals from Reggio are less likely to be married or cohabitating than all individuals in Parma, and all individuals in Padova except those who received a Religious education. For the age 40 cohort, Municipal educated individuals from Reggio are more likely to be married than Municipal educated individuals from Parma and Padova as well as those who didn't receive preschool education in Padova, and less likely than privately educated individuals of Parma.

	\item \textbf{Diff-in-diff results:} Table \ref{table:LCh-30} does not show significant diff-in-diff effects in likelihood of marriage and cohabitation in the age 30 cohort. For the age 40 cohort, Table \ref{table:LCh-40} shows that there is a difference in the likelihood of marriage or cohabitation between Municipal educated individuals and those who didn't attend preschool, and that this difference in likelihood in Reggio is significantly different than the difference in Parma. Similarly, we find that such a diff-in-diff effect exists between Reggio and Padova when comparing Municipal schools with State and Religious schools. Table \ref{table:OLS_L} shows that Municipal educated individuals are less likely to be married than those who didn't attend preschool in Reggio, and that the opposite relationship exists in Parma. Comparing Reggio and Padova, we find that Municipal educated individuals are more likely to be married than both State and Religious educated individuals in Reggio, and that these relationships are reversed in Padova.
	\end{itemize}	
\end{itemize}  

\subsection{Health and Risk taking Behaviors}
Compared to other individuals, Municipal educated individuals from Reggio report better health, are less likely to smoke cigarettes, and are generally more likely to have tried marijuana.

\begin{itemize}
\item \textbf{Tried marijuana:} 
	
	\begin{itemize}
	\item \textbf{OLS results:} Comparing materna types within Reggio in the age 30 cohort, Table \ref{table:OLS_H} shows that Municipal students have significantly lower mean likelihoods of having tried marijuana compared to Reggio individuals who attended state and religious preschool, and significantly higher likelihoods than those who received Private education. For the age 40 cohort,  Municipal students have significantly higher mean likelihoods of having tried marijuana than individuals from all other materna types in Reggio. 

The estimates that suggest Municipal students are more likely to have tried marijuana are questionable as they all have a mean likelihood of trying marijuana that is negative. We do not interpret this variable any futher due to this reason. We will need to investigate the data to ensure that this variable is accurate.
		
	\item \textbf{Diff-in-diff results:} Tables  \ref{table:HCh-30} and  \ref{table:HCh-40} shows that Municipal educated individuals are more likely to have tried marijuana than those who did not attend any preschool, and that this likelihood is greater in Reggio than in Padova for the age 30 cohort, and greater in Reggio than in Parma for the age 40 cohort.
	
	Tables  \ref{table:HCh-30} also shows that for the age 30 cohort, there is a difference in the likelihood of having tried marijuana between Municipal educated individuals and Religious educated individuals, and that this difference in likelihood in Reggio is significantly different than the difference in Padova. Table \ref{table:OLS_H} shows that Municipal educated individuals are less likely to have tried marijuana than Religious educated individuals in Reggio, and that this relationship is reversed in Padova.
	\end{itemize}


\item \textbf{Likelihood of smoking:} 
	
	\begin{itemize}
	\item \textbf{OLS results:} Comparing materna types within Reggio in the age 30 cohort, Table \ref{table:OLS_H} shows that Municipal students have significantly lower mean likelihoods of having smoked compared to Reggio individuals who attended state preschool. For the age 40 cohort,  Municipal students have significantly higher mean likelihoods of smoking than individuals who attended any other type of materna in Reggio.

	Comparing across cities in the age 30 cohort, Table \ref{table:OLS_H} shows that Municipal educated individuals from Reggio are less likely to smoke than all individual from Parma, and all individuals from Padova except those from Padova who attended a private preschool. The estimate on individuals from Padova who attended private preschool is significant in the opposite direction and takes on a negative value. For the age 40 cohort, Municipal educated individuals from Reggio are less likely to smoke than all individuals from Parma other than those who attended Religious school, and Municipal and Religiously educated individuals from Padova. Those who attended state or no preschool in Padova are significantly less likely to smoke than those who attended Municipal school in Reggio.

	\item \textbf{Diff-in-diff results:} Table \ref{table:HCh-30} shows that there are no significant diff-in-diff effects in the age 30 cohort. For the age 40 cohort,  Table \ref{table:HCh-40} shows that there is a difference in the likelihood of cigarette consumption between Municipal educated individuals and Religious educated individuals, and that this difference in likelihood in Reggio is significantly different than the difference in Padova. Table \ref{table:OLS_H} shows that in Reggio, Municipal educated individuals are more likely to smoke cigarettes when compared to Religious educated individuals. This relationship is reversed in Padova where Municipal educated individuals are less likely to smoke than their Religious counterparts.
	\end{itemize}


\item \textbf{Number of cigarettes per day:} 
	
	\begin{itemize}
	\item \textbf{OLS results:} Comparing materna types within Reggio in the age 30 cohort, Table \ref{table:OLS_H} shows that among those who smoke, Municipal students smoke more cigarettes per day on average compared to Reggio individuals who attended state and religious preschool. For the age 40 cohort, Municipal students smoke signifcantly lower numbers of cigarettes per day compared to those who attended state or no preschool, and significantly more cigarettes than those who attended private preschool. 

	Comparing across cities in the age 30 cohort, Table \ref{table:OLS_H} shows that Municipal educated individuals from Reggio smoke more cigarattes per day on average than all individuals in Parma and Padova except those in Padova who attended private preschool. In the age 40 cohort, Municipal educated individuals from Reggio smoke more cigarattes per day on average than all individuals in Parma, and smoke significantly less than those in Padova who attended Municipal, State and Religious schools.
	
	\item \textbf{Diff-in-diff results:} Table \ref{table:HCh-30} shows that there are no significant diff-in-diff effects in the age 30 cohort. For the age 40 cohort,  Table \ref{table:HCh-40} shows that there is a difference in the likelihood of cigarette consumption as well as the number of cigarettes smoked per day between Municipal educated individuals and Religious educated individuals, and that this difference in likelihood in Reggio is significantly different than the difference in Padova. Table \ref{table:OLS_H} shows that in Reggio, Municipal educated individuals are more likely to smoke cigarettes and smoke more cigarettes per day when compared to Religious educated individuals. These relationships are reversed in Padova where Municipal educated individuals are less likely to smoke and smoke less cigarettes per day than their Religious counterparts.
	\end{itemize}

\item \textbf{Health:} 
	
	\begin{itemize}
	\item \textbf{OLS results:}  Comparing materna types within Reggio in the age 30 cohort, Table \ref{table:OLS_H} shows that Municipal students have significantly lower health than those who attend State, Religious and private schools, and have significantly better health than those who attend no preschool. For the age 40 cohort, Municipal students have significantly lower health than those who attend Religious schools, and significantly better health than those who attended State and private schools.

	Comparing across cities in the age 30 cohort, Table \ref{table:OLS_H} shows that Municipal educated individuals from Reggio have better health than those who attend municipal, state or no preschool in Parma, and those who attend municipal and religious preschool in Padova. In the age 40 cohort, Municipal educated individuals from Reggio have significantly higher health than all individuals from Parma except those who didn't attend preschool, and have significantly lower health than those in Padova who attended municipal and state schools.

	\item \textbf{Diff-in-diff results:} For the age 30 cohort, Table \ref{table:HCh-30} shows that there is a difference in levels of health between Municipal educated individuals and those who didn't attend preschool, and that this difference in Reggio is significantly different than the difference in Parma and Padova. Table \ref{table:OLS_H} shows that in Reggio, Municipal educated individuals report higher levels of health than those who didn't attend preschool and that this relationship is reversed in both Parma and Padova. 
	
	For the age 40 cohort, we see the same diff-in-diff effect as above between Reggio and Parma. For Padova, we see that a significant diff-in-diff effect exists, however, unlike in the age 30 cohort, Municipal educated individuals report higher health than those who didn't attend preschool in both Reggio and Padova. Table \ref{table:HCh-40} also shows that this effect is larger in Reggio than in Padova. 
	\end{itemize}

\item \textbf{Age at first drink:} 
	
	\begin{itemize}
	\item \textbf{OLS results:} Table  \ref{table:OLS_H}  Comparing materna types within Reggio in the age 30 cohort, Table \ref{table:OLS_H} shows that Municipal students have significantly higher mean age at first drink than those who attended state and private preschools, and lower age than those who received private education. At age 40, Municipal students have significantly higher age at first drink than private school students, and lower age at first drink than those who attended religious or no preschool.

	Comparing across cities in the age 30 cohort, Table \ref{table:OLS_H} shows that Municipal educated individuals from Reggio have significantly lower age at first drink than all individuals in Parma and Padova, except those from Parma who didn't attend preschool. In the age 40 cohort, Municipal educated individuals from Reggio have significantly lower age at first drink than all individuals in Parma and Padova, except those from Parma who attended private school for whom then estimate is significant in the opposite direction. 
	
	\item \textbf{Diff-in-diff results:} Table \ref{table:HCh-40} shows that there are no significant diff-in-diff effects at the age 40 level. At the age 30 cohort, Table \ref{table:HCh-40} shows that a number of significant diff-in-diff effects exist. We explain these significant effects in the following. Looking at Table \ref{table:OLS_H} we see that in Reggio, Municipal educated individuals have higher ages at first drink compared to those who attended state or school or no preschool. In Parma, both of these relationships are reversed, and in Padova, the relationship between Municipal and State are reversed. Table \ref{table:OLS_H} also shows that Municipal educated individuals start drinking at a younger age than Religious educated individuals in Reggio, and that the converse is true in Parma and Padova.
	\end{itemize}
\end{itemize}

\subsection{Noncognitive Outcomes}

\begin{itemize}
\item \textbf{Locus of Control:} 
	
	\begin{itemize}
	\item \textbf{OLS results:} Comparing materna types within Reggio in the age 30 cohort, Table \ref{table:OLS_N} shows that Municipal students tend to believe that they are more in control of events that affect their lives than private students, and that the reverse is true when comparing Municipal students with those who attended Religious or no preschool. For the age 40 cohort, we see that Municipal students believe they are less in control of events that affect their lives than Religious students.

	Comparing across cities in the age 30 cohort, Table \ref{table:OLS_N} shows that Municipal educated individuals from Reggio believe that they are more in control of events that affect their lives than all individuals from Parma and Padova except those who attended Religious school in Parma or Padova. For the age 40 cohort, Municipal educated individuals from Reggio believe that they are more in control of events that affect their lives than all individuals in Parma except those who attended private school, and those individuals from Padova who attended Municipal or State school.

	\item \textbf{Diff-in-diff results:} We do not detect any significant effects at the age 30 cohort. For the age 40 cohort, Table \ref{table:NCh-40} shows that diff-in-diff effects exist between Municipal educated individuals and those who didn't attend preschool when comparing Reggio with Padova. Table \ref{table:OLS_N} shows that Municipal educated individuals believe they are more in control of events affecting them than those who didn't attend preschool in Reggio, while the opposite is true in Padova. Table \ref{table:NCh-40} shows diff-in-diff effects also exist when comparing Municipal and Religious school between Reggio and Padova.  Table \ref{table:OLS_N} shows that in both Reggio and Padova, Municipal educated individuals believe they are less in control of events affecting them than those who attended Religious preschool. The positive value for the diff-in-diff estimator in Table \ref{table:NCh-40} shows that this effect is larger in Padova than in Reggio.
	\end{itemize}

\item \textbf{Depression score:} 
	
	\begin{itemize}
	\item \textbf{OLS results:} Comparing materna types within Reggio in the age 30 cohort, Table \ref{table:OLS_N} shows that Municipal students have significantly lower levels of depression than state students, and significantly higher levels of depression than those who attended private or no preschool. At the age 40 cohort, Municipal students have significantly lower levels of depression than religious students.

	Comparing across cities in the age 30 cohort, Table \ref{table:OLS_N} shows that Municipal students from Reggio are have significantly higher mean levels of depression than all individuals from Parma, and lower mean levels of depression than all individuals from Padova except those who attended religious school. The same pattern persists between Reggio and Parma at the age 40 level. Comparing Reggio and Padova at age 40, Municipal students from Reggio are significantly less depressed on average than Municipal students from Padova, and significantly more depressed than resligious students.

	\item \textbf{Diff-in-diff results:} We do not detect any significant effects at the age 30 cohort. For the age 40 cohort, Table \ref{table:NCh-40} shows that diff-in-diff effects exist between Municipal educated and State educated individuals when comparing Reggio with both Parma and Padova. Table \ref{table:OLS_N} shows that Municipal educated individuals report lower levels of depression than state educated individuals in Reggio and in Padova, and that this relationship is reversed in Parma. The negative score on the diff-in-diff effect between Reggio and Padova shows that the degree by which Municipal educated individuals are less depressed than State educated individuals is larger in Padova than in Reggio. Table \ref{table:NCh-40} also shows that a diff-in-diff effect exists between Reggio and Padova when comparing Municipal educated individuals to those who didn't received any preschool education. Table \ref{table:OLS_N} shows that Municipal educated individuals report lower levels of depression than their State educated counterparts in Reggio, and that the relationship is reversed in Padova. 
	\end{itemize}

\item \textbf{Satisfied with Income:} 
	
	\begin{itemize}
	\item \textbf{OLS results:} Comparing materna types within Reggio in the age 30 cohort, Table \ref{table:OLS_N} shows that Municipal students are significantly more likely to be satisfied with their income than those who attended religious, private or no preschool, and significantly less likely than those who attended state school. For the age 40 cohort, Municipal students are significantly more likely to be satisfied with their income than those who attended religious or private preschool, and significantly less likely than those who attended state school.

	Comparing across cities in the age 30 cohort, Table \ref{table:OLS_N} shows that Municipal students from Reggio are significantly more likely to be satisfied with income than those from Parma who attended Municipal, private or no preschool, and those from PAdova who attended Municipal, state, private or no preschool. For the age 40 cohort, Municipal students from Reggio are significantly more likely to be satisfied with income than those from Parma who attended Municipal, state or religious schools, and significantly less likely than those from Parma who attended private schools. Municipal students from Reggio are also significantly more likely to be satisfied with income than all individuals from Padova, except those who attended private school.
	
	\item \textbf{Diff-in-diff results:} We do not detect significant effects at the age 40 cohort. For the age 30 cohort, Table \ref{table:NCh-30} shows that between Reggio and Parma, diff-in-diff effects exist when comparing Municipal educated individuals with \textbf{(i)} Religious educated individuals and  \textbf{(ii)} those who didn't receive preschool education. Table \ref{table:OLS_N} shows that in Reggio, Municipal educated individuals are more likely to be satisfied with their income than both Religious educated individuals and those who didn't attend preschool, and that both these relationships are reversed in Parma. 
	
	Table \ref{table:OLS_N} also shows that that a significant diff-in-diff effect exists when comparing Municipal and State schools between Reggio and Padova. Table \ref{table:OLS_N} shows that Municipal educated individuals are less likely to be satisfied with their income than State educated individuals in Reggio, and that this relationship is reversed in Padova.
	\end{itemize}


\item \textbf{Satisfied with Health:} 
	
	\begin{itemize}
	\item \textbf{OLS results:} Comparing materna types within Reggio in the age 30 cohort, Table \ref{table:OLS_N} shows that Municipal students are significantly less likely to be satisfied with their health than those who attended state or private preschool. For the age 40 cohort, municipal students are significantly less likely to be satisfied with their health than those who attended state preschool, and significantly more likely than those who attended private preschool.

	Comparing across cities in the age 30 cohort, Table \ref{table:OLS_N} shows that Municipal students from Reggio are significantly less likely to be satisfied with health than those from Parma who attended Municipal, state or no preschool, and more likely to be satisfied than those from Padova who attended state, religious or no preschool. For the age 40 cohort, Municipal students from Reggio are significantly more likely to be satisfied with health than those from Parma who attended Municipal, state, religious, or no preschools, and those from Padova who attended municipal, religious or no preschool.
	
	\item \textbf{Diff-in-diff results:}  Table \ref{table:NCh-30} shows that 5 out of our 6 estimates have significant diff-in-diff effects at age 30. When comparing Reggio and Parma, significant diff-in-diff effects exist between Municipal and None, and between Municipal and Religious. Table \ref{table:OLS_N} shows that Municipal educated individuals are less likely to be satisfied with health compared to Religious educated individuals in Reggio, and that the opposite is true in Parma.  Table \ref{table:OLS_N} also shows that  Municipal educated individuals report are less likely to be satisfied with health compared to those who didn't receive pre school education in both Reggio and Parma. The estimate for this diff-in-diff effect is negative in Table \ref{table:NCh-30}, which implies that the degree by which Municipal educated individuals are less likely to be satisfied with health compared to those who didn't receive preschool education is greater in Reggio than in Parma.
	
	 When looking at Reggio and Padova, significant diff-in-diff effects exist when comparing Municipal educated individuals with those who attended no preschool, State school, and Religious school. Table \ref{table:OLS_N} shows that in Reggio, Municipal educated individuals are less likely to be satisfied with health compared to individuals who attended no preschool, State school, or Religious school. These relationships are all reversed in Padova. 
	
	\end{itemize}

%\item \textbf{Optimistic outlook on life:} 
%	
%	\begin{itemize}
%	\item \textbf{OLS results:} 
%	
%	\item \textbf{Diff-in-diff results:}  Table \ref{table:NCh-30} shows that 5 out of our 6 estimates have significant diff-in-diff effects at age 30. When comparing Reggio and Parma, significant diff-in-diff effects exist between Municipal and None, and between Municipal and Religious. Table \ref{table:OLS_N} shows that Municipal educated individuals report lower satisfaction with health compared to Religious educated individuals in Reggio, and that the opposite is true in Parma.  Table \ref{table:OLS_N} also shows that  Municipal educated individuals report lower satisfaction with health compared to those who didn't receive pre school education in both Reggio and Parma. The estimate for this diff-in-diff effect is negative in Table \ref{table:NCh-30}, which implies that the degree by which Municipal educated individuals report lower health than those who didn't receive preschool education is greater in Reggio than in Parma.
%	
%	 When looking at Reggio and Padova, significant diff-in-diff effects exist when comparing Municipal educated individuals with those who attended no preschool, State school, and Religious school. Table \ref{table:OLS_N} shows that in Reggio, Municipal educated individuals report lower satisfaction with health compared to individuals who attended no preschool, State school, or Religious school. These relationships are all reversed in Padova. 
%	
%	\end{itemize}
%
\end{itemize}

\clearpage

%----------------------------------------
\appendix

\singlespace
\newgeometry{left=.8in,right=.8in,top=1in,bottom=1in}

\begin{landscape}
\section{Appendix: OLS and Difference-in-Difference Results}
\subsection{Education - OLS results}

\begin{center}
		\scriptsize{
			\begin{longtable}{L{3cm} c c c c c c p{.5cm} c c c c c c} 
				\hline \\
				\multicolumn{14}{L{19cm}}{\textbf{Note:} \fnOLS}
				\endfoot
				\caption{Mean outcomes for Education}  \label{table:OLS_E} \\
				\toprule
				\textbf{Outcome} & \multicolumn{6}{c}{\textbf{C. Mean}} & & \multicolumn{6}{c}{\textbf{Mean}} \\
\quad \quad Sample & Muni & State & Reli & Priv & None & $ R^2$ & & Muni & State & Reli & Priv & None & $ R^2$ \\
\quad \quad Restriction & \tiny{$\boldsymbol{\gamma_0}$}& \tiny{$\boldsymbol{\gamma_0+\gamma_1}$}& \tiny{$\boldsymbol{\gamma_0+\gamma_2}$}& \tiny{$\boldsymbol{\gamma_0+\gamma_3}$}& \tiny{$\boldsymbol{\gamma_0+\gamma_4}$} & & & \tiny{$\boldsymbol{\gamma_0}$}& \tiny{$\boldsymbol{\gamma_0+\gamma_1}$}& \tiny{$\boldsymbol{\gamma_0+\gamma_2}$}& \tiny{$\boldsymbol{\gamma_0+\gamma_3}$}& \tiny{$\boldsymbol{\gamma_0+\gamma_4}$} \\
\hline \endhead
~\\*[.05cm]
\textbf{IQ Factor} \\*[.1cm]
\quad \quad \textbf{Adult 30} & & & & & & & & \multicolumn{6}{c}{\highlight{Reference mean = \textbf{    -0.24}}} \\*[.1cm]
\quad \quad \quad Reggio& -0.74 & \textbf{    -0.89} & -0.12 & \textbf{     0.09} & \textbf{    -0.57} &      0.18 & & -0.24 & -0.54 & 0.31 & \textbf{     0.76} & -0.24 & &      0.06 \\*
\quad \quad \quad Parma& \textbf{     0.38} & \textbf{     0.32} & \textbf{     0.51} & \textbf{     0.30} & \textbf{     0.25} &      0.06 & & \textbf{     0.48} & \textbf{     0.40} & \textbf{     0.60} & \textbf{     0.37} & \textbf{     0.38} & &      0.02 \\*
\quad \quad \quad Padova& \textbf{     0.24} & \textbf{     0.03} & \textbf{     0.42} & \textbf{     0.54} & \textbf{     0.14} &      0.17 & & \textbf{     0.42} & 0.10 & \textbf{     0.54} & \textbf{     0.84} & \textbf{     0.28} & &      0.06 \\*
\\
\quad \quad \textbf{Adult 40} & & & & & & & & \multicolumn{6}{c}{\highlight{Reference mean = \textbf{     0.01}}} \\*[.1cm]
\quad \quad \quad Reggio& -0.04 & -0.43 & 0.27 & 0.58 & \textbf{     0.02} &      0.06 & & 0.01 & -0.36 & 0.33 & 0.64 & \textbf{     0.09} & &      0.05 \\*
\quad \quad \quad Parma& \textbf{     0.09} & -0.15 & \textbf{     0.14} & \textbf{     0.29} & \textbf{     0.21} &      0.12 & & \textbf{     0.40} & 0.07 & \textbf{     0.50} & \textbf{     0.71} & \textbf{     0.51} & &      0.05 \\*
\quad \quad \quad Padova& \textbf{     0.05} & -1.07 & 0.32 & . & 0.31 &      0.51 & & \textbf{     0.11} & -1.12 & 0.51 & . & 0.52 & &      0.42 \\*
\\
\quad \quad \textbf{Adult 50} & & & & & & & & \multicolumn{6}{c}{\highlight{Reference mean = \textbf{     0.59}}} \\*[.1cm]
\quad \quad \quad Reggio& 0.61 & 0.12 & \textbf{     0.47} & \textbf{     0.29} & \textbf{     0.54} &      0.12 & & 0.59 & 0.11 & \textbf{     0.52} & \textbf{     0.31} & \textbf{     0.62} & &      0.09 \\*
\quad \quad \quad Parma& \textbf{    -0.22} & \textbf{     0.13} & \textbf{     0.17} & . & 0.22 &      0.14 & & \textbf{0} & \textbf{     0.28} & \textbf{     0.32} & . & 0.38 & &      0.05 \\*
\quad \quad \quad Padova& \textbf{     0.14} & \textbf{     0.69} & 0.53 & \textbf{     0.16} & \textbf{     0.39} &      0.10 & & \textbf{     0.15} & \textbf{     0.73} & 0.55 & \textbf{     0.16} & 0.41 & &      0.06 \\*
\\
~\\*[.05cm]
\textbf{High School Grade} \\*[.1cm]
\quad \quad \textbf{Adult 30} & & & & & & & & \multicolumn{6}{c}{\highlight{Reference mean = \textbf{    83.93}}} \\*[.1cm]
\quad \quad \quad Reggio& 87.40 & \textbf{    86.93} & \textbf{    84.46} & \textbf{    94.84} & 82.40 &      0.08 & & 83.93 & \textbf{    84.96} & \textbf{    81.03} & \textbf{    90.00} & 79.88 & &      0.04 \\*
\quad \quad \quad Parma& \textbf{    60.57} & \textbf{    58.35} & \textbf{    61.63} & 76.59 & 54.77 &      0.32 & & \textbf{    73.22} & \textbf{    73.02} & 80.04 & 90.00 & \textbf{    67.62} & &      0.06 \\*
\quad \quad \quad Padova& \textbf{    77.46} & \textbf{    79.25} & \textbf{    76.88} & \textbf{    78.06} & \textbf{    77.16} &      0.03 & & \textbf{    77.83} & \textbf{    80.63} & \textbf{    77.57} & \textbf{    79.00} & \textbf{    78.11} & &      0.00 \\*
\\
\quad \quad \textbf{Adult 40} & & & & & & & & \multicolumn{6}{c}{\highlight{Reference mean = \textbf{    83.40}}} \\*[.1cm]
\quad \quad \quad Reggio& 82.05 & \textbf{    83.42} & 81.87 & \textbf{    86.19} & \textbf{    80.84} &      0.02 & & 83.40 & \textbf{    85.77} & 83.47 & \textbf{    87.33} & \textbf{    82.71} & &      0.01 \\*
\quad \quad \quad Parma& \textbf{    65.92} & \textbf{    66.55} & 72.58 & \textbf{    79.92} & \textbf{    65.63} &      0.28 & & \textbf{    74.36} & \textbf{    72.65} & 81.98 & \textbf{    97.00} & \textbf{    71.84} & &      0.08 \\*
\quad \quad \quad Padova& \textbf{    75.41} & 87.11 & \textbf{    77.56} & . & \textbf{    79.21} &      0.09 & & \textbf{    75.00} & 86.05 & \textbf{    77.41} & . & \textbf{    79.05} & &      0.06 \\*
\\
\quad \quad \textbf{Adult 50} & & & & & & & & \multicolumn{6}{c}{\highlight{Reference mean = \textbf{    77.00}}} \\*[.1cm]
\quad \quad \quad Reggio& 72.65 & \textbf{    76.36} & \textbf{    76.60} & \textbf{    78.12} & \textbf{    75.81} &      0.06 & & 77.00 & \textbf{    79.14} & \textbf{    80.74} & \textbf{    80.00} & \textbf{    79.72} & &      0.01 \\*
\quad \quad \quad Parma& \textbf{    62.71} & \textbf{    50.90} & \textbf{    64.19} & . & \textbf{    69.04} &      0.27 & & \textbf{    64.20} & \textbf{    50.00} & \textbf{    69.43} & . & \textbf{    74.34} & &      0.09 \\*
\quad \quad \quad Padova& 72.27 & . & \textbf{    71.41} & . & 72.94 &      0.03 & & 76.75 & . & \textbf{    74.97} & . & 77.13 & &      0.01 \\*
\\
~\\*[.05cm]
\textbf{University Grade} \\*[.1cm]
\quad \quad \textbf{Adult 30} & & & & & & & & \multicolumn{6}{c}{\highlight{Reference mean = \textbf{   101.82}}} \\*[.1cm]
\quad \quad \quad Reggio& 107.51 & \textbf{   109.21} & 98.99 & \textbf{   101.56} & \textbf{   106.27} &      0.36 & & 101.82 & \textbf{   104.43} & 93.00 & \textbf{    97.00} & \textbf{    99.43} & &      0.21 \\*
\quad \quad \quad Parma& \textbf{   103.50} & \textbf{   105.97} & \textbf{   105.11} & 115.08 & \textbf{   104.17} &      0.31 & & \textbf{    98.30} & \textbf{   101.25} & \textbf{   100.75} & 110.00 & \textbf{    97.21} & &      0.09 \\*
\quad \quad \quad Padova& \textbf{   101.37} & 107.20 & \textbf{   102.45} & 85.92 & \textbf{   100.40} &      0.11 & & \textbf{    99.24} & 105.75 & \textbf{    99.96} & 85.00 & \textbf{    96.40} & &      0.09 \\*
\\
\quad \quad \textbf{Adult 40} & & & & & & & & \multicolumn{6}{c}{\highlight{Reference mean = \textbf{    96.79}}} \\*[.1cm]
\quad \quad \quad Reggio& 97.52 & \textbf{    91.86} & \textbf{   100.02} & . & 97.94 &      0.21 & & 96.79 & 90.00 & \textbf{   101.00} & . & \textbf{    97.62} & &      0.15 \\*
\quad \quad \quad Parma& \textbf{   102.59} & \textbf{   104.02} & \textbf{   104.86} & 97.79 & 97.28 &      0.30 & & \textbf{   101.65} & \textbf{   102.63} & \textbf{   104.05} & \textbf{   100.00} & 94.94 & &      0.22 \\*
\quad \quad \quad Padova& \textbf{   102.49} & 92.62 & \textbf{   103.11} & . & \textbf{    99.05} &      0.25 & & \textbf{   100.83} & 90.91 & \textbf{   101.08} & . & 97.00 & &      0.22 \\*
\\
\quad \quad \textbf{Adult 50} & & & & & & & & \multicolumn{6}{c}{\highlight{Reference mean = \textbf{        .}}} \\*[.1cm]
\quad \quad \quad Reggio& 93.69 & . & 101.74 & 112.85 & 93.69 &      0.72 & & 94.57 & . & 104.33 & 110.00 & 94.57 & &      0.51 \\*
\quad \quad \quad Parma& \textbf{   111.04} & \textbf{   103.04} & \textbf{   121.05} & . & \textbf{   103.04} &      0.90 & & \textbf{   108.00} & \textbf{   100.00} & \textbf{   110.00} & . & 97.14 & &      0.52 \\*
\quad \quad \quad Padova& \textbf{    99.81} & . & \textbf{    98.33} & . & \textbf{    99.81} &      0.14 & & \textbf{   104.92} & . & \textbf{   103.27} & . & \textbf{   104.92} & &      0.02 \\*
\\
~\\*[.05cm]
\textbf{Graduate from High School} \\*[.1cm]
\quad \quad \textbf{Adult 30} & & & & & & & & \multicolumn{6}{c}{\highlight{Reference mean = \textbf{     0.85}}} \\*[.1cm]
\quad \quad \quad Reggio& 0.73 & \textbf{     0.82} & 0.74 & \textbf{     1.04} & \textbf{     0.72} &      0.17 & & 0.85 & 0.97 & 0.85 & \textbf{     1.00} & \textbf{     0.89} & &      0.01 \\*
\quad \quad \quad Parma& \textbf{     0.79} & \textbf{     0.80} & \textbf{     0.85} & \textbf{     0.91} & \textbf{     0.82} &      0.17 & & \textbf{     0.87} & \textbf{     0.88} & 0.96 & \textbf{     1.00} & \textbf{     0.86} & &      0.02 \\*
\quad \quad \quad Padova& \textbf{     0.94} & \textbf{     0.89} & \textbf{     0.89} & \textbf{     0.89} & \textbf{     0.93} &      0.13 & & \textbf{     0.91} & \textbf{     0.88} & \textbf{     0.88} & \textbf{     1.00} & \textbf{     0.87} & &      0.00 \\*
\\
\quad \quad \textbf{Adult 40} & & & & & & & & \multicolumn{6}{c}{\highlight{Reference mean = \textbf{     0.74}}} \\*[.1cm]
\quad \quad \quad Reggio& 0.68 & \textbf{     0.82} & 0.69 & \textbf{     0.56} & 0.79 &      0.14 & & 0.74 & \textbf{     0.88} & \textbf{     0.73} & \textbf{     0.60} & 0.91 & &      0.04 \\*
\quad \quad \quad Parma& \textbf{     0.84} & 0.66 & \textbf{     0.84} & 0.72 & \textbf{     0.86} &      0.16 & & \textbf{     0.90} & 0.65 & \textbf{     0.87} & \textbf{     1.00} & \textbf{     0.83} & &      0.04 \\*
\quad \quad \quad Padova& \textbf{     0.78} & 0.96 & \textbf{     0.81} & . & \textbf{     0.78} &      0.15 & & \textbf{     0.78} & 1.00 & \textbf{     0.81} & . & \textbf{     0.80} & &      0.02 \\*
\\
\quad \quad \textbf{Adult 50} & & & & & & & & \multicolumn{6}{c}{\highlight{Reference mean = \textbf{     0.67}}} \\*[.1cm]
\quad \quad \quad Reggio& 0.80 & \textbf{     0.83} & \textbf{     0.99} & \textbf{     0.67} & \textbf{     0.84} &      0.16 & & 0.67 & \textbf{     0.70} & \textbf{     0.86} & \textbf{     0.50} & \textbf{     0.73} & &      0.02 \\*
\quad \quad \quad Parma& \textbf{     0.60} & \textbf{     0.53} & 0.96 & . & 0.88 &      0.30 & & \textbf{     0.42} & \textbf{     0.29} & 0.82 & . & \textbf{     0.61} & &      0.07 \\*
\quad \quad \quad Padova& \textbf{     0.28} & \textbf{     0.46} & \textbf{     0.46} & \textbf{     0.04} & \textbf{     0.37} &      0.25 & & \textbf{     0.36} & \textbf{     0.50} & \textbf{     0.62} & \textbf{0} & \textbf{     0.56} & &      0.04 \\*
\\
~\\*[.05cm]
\textbf{Max Edu: University} \\*[.1cm]
\quad \quad \textbf{Adult 30} & & & & & & & & \multicolumn{6}{c}{\highlight{Reference mean = \textbf{     0.16}}} \\*[.1cm]
\quad \quad \quad Reggio& 0.10 & \textbf{     0.14} & \textbf{     0.05} & 1.02 & \textbf{     0.18} &      0.08 & & 0.16 & \textbf{     0.26} & \textbf{     0.10} & 1.00 & 0.28 & &      0.04 \\*
\quad \quad \quad Parma& \textbf{     0.31} & \textbf{     0.24} & 0.44 & \textbf{     0.31} & \textbf{     0.39} &      0.22 & & \textbf{     0.34} & \textbf{     0.31} & 0.58 & \textbf{     0.40} & \textbf{     0.34} & &      0.04 \\*
\quad \quad \quad Padova& \textbf{     0.37} & \textbf{     0.36} & \textbf{     0.39} & \textbf{     0.70} & 0.12 &      0.20 & & \textbf{     0.49} & \textbf{     0.46} & \textbf{     0.51} & \textbf{     1.00} & 0.21 & &      0.06 \\*
\\
\quad \quad \textbf{Adult 40} & & & & & & & & \multicolumn{6}{c}{\highlight{Reference mean = \textbf{     0.16}}} \\*[.1cm]
\quad \quad \quad Reggio& 0.01 & \textbf{    -0.03} & \textbf{    -0.02} & \textbf{    -0.12} & \textbf{    -0.01} &      0.13 & & 0.16 & \textbf{     0.18} & \textbf{     0.12} & \textbf{0} & \textbf{     0.17} & &      0.01 \\*
\quad \quad \quad Parma& \textbf{     0.26} & \textbf{     0.28} & \textbf{     0.18} & \textbf{     0.57} & 0.05 &      0.25 & & \textbf{     0.46} & \textbf{     0.35} & \textbf{     0.35} & \textbf{     1.00} & 0.15 & &      0.09 \\*
\quad \quad \quad Padova& \textbf{     0.30} & 0.60 & \textbf{     0.34} & . & \textbf{     0.25} &      0.11 & & \textbf{     0.26} & 0.58 & \textbf{     0.34} & . & \textbf{     0.28} & &      0.03 \\*
\\
\quad \quad \textbf{Adult 50} & & & & & & & & \multicolumn{6}{c}{\highlight{Reference mean = \textbf{     0.00}}} \\*[.1cm]
\quad \quad \quad Reggio& 0.08 & 0.07 & \textbf{     0.22} & 0.59 & \textbf{     0.14} &      0.11 & & 0 & 0 & \textbf{     0.14} & 0.50 & \textbf{     0.10} & &      0.03 \\*
\quad \quad \quad Parma& \textbf{     0.22} & \textbf{     0.21} & \textbf{     0.02} & . & 0.10 &      0.21 & & \textbf{     0.17} & \textbf{     0.14} & \textbf{     0.09} & . & \textbf{     0.10} & &      0.01 \\*
\quad \quad \quad Padova& \textbf{     0.12} & 0.12 & \textbf{     0.20} & \textbf{     0.03} & \textbf{     0.19} &      0.23 & & \textbf{     0.09} & 0 & \textbf{     0.26} & 0 & \textbf{     0.25} & &      0.02 \\*
\\
~\\*[.05cm]
\textbf{Max Edu: Graduate School} \\*[.1cm]
\quad \quad \textbf{Adult 30} & & & & & & & & \multicolumn{6}{c}{\highlight{Reference mean = \textbf{     0.01}}} \\*[.1cm]
\quad \quad \quad Reggio& \textbf{     0.01} & \textbf{     0.01} & \textbf{     0.01} & \textbf{0} & \textbf{     0.01} &      0.02 & & \textbf{     0.01} & \textbf{0} & \textbf{0} & \textbf{0} & \textbf{0} & &      0.00 \\*
\quad \quad \quad Parma& \textbf{     0.10} & \textbf{     0.09} & \textbf{     0.07} & \textbf{     0.04} & \textbf{     0.07} &      0.06 & & \textbf{     0.04} & \textbf{     0.04} & \textbf{     0.02} & 0 & \textbf{0} & &      0.01 \\*
\quad \quad \quad Padova& \textbf{    -0.04} & 0.13 & 0.09 & \textbf{    -0.11} & \textbf{    -0.03} &      0.11 & & \textbf{0} & 0.15 & 0.13 & 0 & \textbf{0} & &      0.05 \\*
\\
\quad \quad \textbf{Adult 40} & & & & & & & & \multicolumn{6}{c}{\highlight{Reference mean = \textbf{     0.00}}} \\*[.1cm]
\quad \quad \quad Reggio& 0 & 0 & 0 & 0 & 0 &         . & & 0 & 0 & 0 & 0 & 0 & &         . \\*
\quad \quad \quad Parma& \textbf{     0.06} & \textbf{     0.02} & 0 & \textbf{    -0.08} & -0.01 &      0.06 & & \textbf{     0.10} & \textbf{     0.04} & 0.04 & 0 & 0.01 & &      0.03 \\*
\quad \quad \quad Padova& \textbf{     0.06} & \textbf{     0.02} & \textbf{     0.06} & . & \textbf{     0.05} &      0.02 & & \textbf{     0.04} & 0 & \textbf{     0.03} & . & \textbf{     0.01} & &      0.01 \\*
\\
\quad \quad \textbf{Adult 50} & & & & & & & & \multicolumn{6}{c}{\highlight{Reference mean = \textbf{     0.00}}} \\*[.1cm]
\quad \quad \quad Reggio& 0 & 0 & 0 & 0 & 0 &         . & & 0 & 0 & 0 & 0 & 0 & &         . \\*
\quad \quad \quad Parma& 0 & 0 & 0 & . & 0 &         . & & 0 & 0 & 0 & . & 0 & &         . \\*
\quad \quad \quad Padova& \textbf{     0.08} & 0 & 0 & 0.01 & \textbf{     0.01} &      0.11 & & \textbf{     0.09} & 0 & \textbf{     0.04} & 0 & \textbf{     0.05} & &      0.00 \\*
\\

			\end{longtable}
		}		
\end{center}
\end{landscape}

%--------------


\subsection{Education - Difference-in-Difference Results}

\begin{table}[H]
\begin{center}
	\caption{Difference-in-Difference Across School Types and Cities, Restricting to Age-30 Cohort} \label{table:ECh-30}
	\scalebox{0.80}{
		\begin{tabular}{lcccccccc}
\toprule
 \textbf{Outcome} & \textbf{(1)} & \textbf{(2)} & \textbf{(3)} & \textbf{(4)} & \textbf{(5)} & \textbf{(6)} & \textbf{N} & \textbf{$ R^2$} \\
\midrule
IQ Factor &     -0.12 &      0.17 & \textbf{    -0.53} &     -0.21 &     -0.05 & \textbf{    -0.50} & 782 &       0.21 \\ 
 & (     0.17 ) & (     0.19 ) & \textbf{(     0.18 )} & (     0.19 ) & (     0.23 ) & \textbf{(     0.18 )} & \\
High School Grade &     -1.01 &     -2.31 &      6.04 &      4.10 &      1.27 &      1.43 & 627 &       0.17 \\ 
 & (     3.74 ) & (     4.04 ) & (     3.75 ) & (     4.28 ) & (     5.28 ) & (     3.90 ) & \\
University Grade &      1.84 &      1.06 & \textbf{    11.82} &      0.72 &      3.91 & \textbf{     9.77} & 242 &       0.11 \\ 
 & (     3.58 ) & (     4.03 ) & \textbf{(     4.54 )} & (     4.02 ) & (     4.56 ) & \textbf{(     4.51 )} & \\
Graduate from High School &      0.03 &     -0.08 &      0.05 &     -0.04 &     -0.09 &     -0.05 & 782 &       0.13 \\ 
 & (     0.07 ) & (     0.08 ) & (     0.08 ) & (     0.08 ) & (     0.10 ) & (     0.08 ) & \\
Max Edu: University &     -0.04 &     -0.10 & \textbf{     0.26} & \textbf{    -0.37} &     -0.07 &      0.06 & 782 &       0.17 \\ 
 & (     0.10 ) & (     0.11 ) & \textbf{(     0.11 )} & \textbf{(     0.12 )} & (     0.14 ) & (     0.11 ) & \\
Max Edu: Graduate School &     -0.02 &      0.01 &     -0.02 &     -0.01 & \textbf{     0.14} & \textbf{     0.10} & 782 &       0.08 \\ 
 & (     0.05 ) & (     0.05 ) & (     0.05 ) & (     0.05 ) & \textbf{(     0.06 )} & \textbf{(     0.05 )} & \\
\bottomrule
\end{tabular}
}
\end{center}
\footnotesize
\fnDID
\end{table}

\begin{table}[H]
\begin{center}
	\caption{Difference-in-Difference Across School Types and Cities, Restricting to Age-40 Cohort} \label{table:ECh-40}
	\scalebox{0.80}{
		\begin{tabular}{lcccccccc}
\toprule
 \textbf{Outcome} & \textbf{(1)} & \textbf{(2)} & \textbf{(3)} & \textbf{(4)} & \textbf{(5)} & \textbf{(6)} & \textbf{N} & \textbf{$ R^2$} \\
\midrule
IQ Factor &      0.10 &      0.10 &     -0.24 & \textbf{     0.32} & \textbf{    -0.83} &      0.04 & 791 &       0.22 \\ 
 & (     0.14 ) & (     0.23 ) & (     0.16 ) & \textbf{(     0.17 )} & \textbf{(     0.25 )} & (     0.17 ) & \\
High School Grade &      0.56 &     -1.80 & \textbf{     6.95} &      3.70 & \textbf{     8.18} &      0.43 & 622 &       0.20 \\ 
 & (     2.68 ) & (     4.56 ) & \textbf{(     3.11 )} & (     3.34 ) & \textbf{(     4.75 )} & (     3.45 ) & \\
University Grade & \textbf{    -7.01} & \textbf{     8.65} &     -2.10 &     -4.73 &     -2.47 &     -4.54 & 186 &       0.25 \\ 
 & \textbf{(     3.24 )} & \textbf{(     5.05 )} & (     3.81 ) & (     4.05 ) & (     5.45 ) & (     4.37 ) & \\
Graduate from High School & \textbf{    -0.15} & \textbf{    -0.30} &     -0.03 & \textbf{    -0.17} &      0.11 &     -0.00 & 791 &       0.12 \\ 
 & \textbf{(     0.08 )} & \textbf{(     0.13 )} & (     0.09 ) & \textbf{(     0.10 )} & (     0.14 ) & (     0.10 ) & \\
Max Edu: University & \textbf{    -0.22} &     -0.02 &     -0.09 &     -0.05 & \textbf{     0.32} &      0.04 & 791 &       0.18 \\ 
 & \textbf{(     0.09 )} & (     0.14 ) & (     0.10 ) & (     0.10 ) & \textbf{(     0.15 )} & (     0.10 ) & \\
Max Edu: Graduate School & \textbf{    -0.07} &     -0.04 &     -0.05 &     -0.02 &     -0.03 &     -0.01 & 791 &       0.04 \\ 
 & \textbf{(     0.03 )} & (     0.05 ) & (     0.03 ) & (     0.04 ) & (     0.05 ) & (     0.04 ) & \\
\bottomrule
\end{tabular}
}
\end{center}
\footnotesize
\fnDID
\end{table}

%\begin{table}[H]
%\begin{center}
%	\caption{Difference-in-Difference Across School Types and Cities, Restricting to Age-50 Cohort} \label{table:ECh-50}
%	\scalebox{0.80}{
%		\begin{tabular}{lcccccccc}
\toprule
 \textbf{Outcome} & \textbf{(1)} & \textbf{(2)} & \textbf{(3)} & \textbf{(4)} & \textbf{(5)} & \textbf{(6)} & \textbf{N} & \textbf{$ R^2$} \\
\midrule
IQ Factor & \textbf{     0.45} & \textbf{     0.80} & \textbf{     0.42} &      0.25 & \textbf{     1.08} & \textbf{     0.45} & 449 &       0.13 \\ 
 & \textbf{(     0.19 )} & \textbf{(     0.29 )} & \textbf{(     0.24 )} & (     0.18 ) & \textbf{(     0.39 )} & \textbf{(     0.20 )} & \\
High School Grade &      7.41 &    -14.57 &      1.40 &      2.94 &      0.00 &     -0.31 & 280 &       0.15 \\ 
 & (     6.11 ) & (    12.92 ) & (     7.39 ) & (     6.43 ) & (        . ) & (     6.70 ) & \\
University Grade &     -6.11 &      0.00 &     -0.29 &     -1.50 &      0.00 &    -12.34 & 62 &       0.42 \\ 
 & (     6.29 ) & (        . ) & (     9.22 ) & (     7.94 ) & (        . ) & (     8.40 ) & \\
Graduate from High School &      0.19 &     -0.15 &      0.14 &      0.10 &      0.07 &     -0.04 & 449 &       0.19 \\ 
 & (     0.19 ) & (     0.28 ) & (     0.23 ) & (     0.18 ) & (     0.38 ) & (     0.19 ) & \\
Max Edu: University &      0.09 &      0.18 &     -0.09 & \textbf{     0.32} &      0.15 &      0.22 & 449 &       0.18 \\ 
 & (     0.14 ) & (     0.21 ) & (     0.18 ) & \textbf{(     0.13 )} & (     0.28 ) & (     0.14 ) & \\
Max Edu: Graduate School &      0.02 &      0.01 &     -0.01 &      0.01 &     -0.05 &     -0.02 & 449 &       0.07 \\ 
 & (     0.05 ) & (     0.08 ) & (     0.07 ) & (     0.05 ) & (     0.10 ) & (     0.05 ) & \\
\bottomrule
\end{tabular}
}
%\end{center}
%\footnotesize
%\fnDID
%\end{table}

%----------------------------------------

\begin{landscape}
\subsection{Employment and Earnings - OLS results}

\begin{center}
		\scriptsize{
			\begin{longtable}{L{3cm} c c c c c c p{.5cm} c c c c c c}
				\hline \\
				\multicolumn{14}{L{19cm}}{\textbf{Note:} \fnOLS}
				\endfoot
				\caption{Mean outcomes for Employment and Earnings} \label{table:OLS_W} \\
				\toprule \\
				\textbf{Outcome} & \multicolumn{6}{c}{\textbf{C. Mean}} & & \multicolumn{6}{c}{\textbf{Mean}} \\
\quad \quad Sample & Muni & State & Reli & Priv & None & $ R^2$ & & Muni & State & Reli & Priv & None & $ R^2$ \\
\quad \quad Restriction & \tiny{$\boldsymbol{\gamma_0}$}& \tiny{$\boldsymbol{\gamma_0+\gamma_1}$}& \tiny{$\boldsymbol{\gamma_0+\gamma_2}$}& \tiny{$\boldsymbol{\gamma_0+\gamma_3}$}& \tiny{$\boldsymbol{\gamma_0+\gamma_4}$} & & & \tiny{$\boldsymbol{\gamma_0}$}& \tiny{$\boldsymbol{\gamma_0+\gamma_1}$}& \tiny{$\boldsymbol{\gamma_0+\gamma_2}$}& \tiny{$\boldsymbol{\gamma_0+\gamma_3}$}& \tiny{$\boldsymbol{\gamma_0+\gamma_4}$} \\
\hline \endhead
~\\*[.05cm]
\textbf{Employed} \\*[.1cm]
\quad \quad \textbf{Adult 30} & & & & & & & & \multicolumn{6}{c}{\highlight{Reference mean = \textbf{     0.97}}} \\*[.1cm]
\quad \quad \quad Reggio& 0.93 & \textbf{     0.92} & \textbf{     0.96} & 0.92 & \textbf{     0.90} &      0.04 & & \textbf{     0.97} & \textbf{     0.94} & \textbf{     1.00} & \textbf{     1.00} & 0.89 & &      0.03 \\*
\quad \quad \quad Parma& \textbf{     0.81} & \textbf{     0.86} & 0.90 & \textbf{     0.71} & \textbf{     0.88} &      0.09 & & \textbf{     0.85} & \textbf{     0.90} & 0.94 & \textbf{     0.80} & \textbf{     0.93} & &      0.02 \\*
\quad \quad \quad Padova& \textbf{     1.05} & \textbf{     1.04} & 0.94 & 0.16 & \textbf{     0.96} &      0.16 & & \textbf{     0.94} & \textbf{     0.96} & \textbf{     0.87} & 0 & \textbf{     0.91} & &      0.05 \\*
\\
\quad \quad \textbf{Adult 40} & & & & & & & & \multicolumn{6}{c}{\highlight{Reference mean = \textbf{     0.98}}} \\*[.1cm]
\quad \quad \quad Reggio& 1.03 & \textbf{     0.99} & \textbf{     1.02} & 0.81 & 0.96 &      0.19 & & \textbf{     0.98} & \textbf{     0.94} & \textbf{     0.98} & 0.80 & 0.91 & &      0.03 \\*
\quad \quad \quad Parma& \textbf{     0.91} & \textbf{     0.86} & \textbf{     0.90} & \textbf{     0.97} & \textbf{     0.93} &      0.05 & & \textbf{     0.94} & \textbf{     0.88} & \textbf{     0.95} & 1.00 & \textbf{     0.95} & &      0.01 \\*
\quad \quad \quad Padova& \textbf{     0.93} & 1.01 & \textbf{     0.94} & . & \textbf{     0.98} &      0.06 & & \textbf{     0.85} & 1.00 & \textbf{     0.88} & . & \textbf{     0.93} & &      0.02 \\*
\\
\quad \quad \textbf{Adult 50} & & & & & & & & \multicolumn{6}{c}{\highlight{Reference mean = \textbf{     1.00}}} \\*[.1cm]
\quad \quad \quad Reggio& 1.04 & \textbf{     0.97} & \textbf{     0.94} & 0.53 & 1.00 &      0.20 & & 1.00 & \textbf{     0.90} & \textbf{     0.82} & 0.50 & \textbf{     0.91} & &      0.03 \\*
\quad \quad \quad Parma& \textbf{     0.78} & 0.44 & \textbf{     0.72} & . & \textbf{     0.66} &      0.31 & & \textbf{     0.92} & \textbf{     0.71} & 1.00 & . & \textbf{     0.85} & &      0.03 \\*
\quad \quad \quad Padova& \textbf{     0.60} & \textbf{     0.70} & \textbf{     0.71} & 1.22 & \textbf{     0.66} &      0.21 & & \textbf{     0.73} & 1.00 & \textbf{     0.74} & 1.00 & \textbf{     0.73} & &      0.01 \\*
\\
~\\*[.05cm]
\textbf{Self-Employed} \\*[.1cm]
\quad \quad \textbf{Adult 30} & & & & & & & & \multicolumn{6}{c}{\highlight{Reference mean = \textbf{     0.12}}} \\*[.1cm]
\quad \quad \quad Reggio& 0 & 0 & 0 & 0 & 0 &      1.00 & & 0.12 & \textbf{     0.07} & \textbf{     0.13} & 1.00 & \textbf{     0.17} & &      0.03 \\*
\quad \quad \quad Parma& 0 & 0 & 0 & 0 & 0 &      1.00 & & \textbf{     0.06} & \textbf{     0.02} & \textbf{     0.06} & \textbf{0} & 0.14 & &      0.02 \\*
\quad \quad \quad Padova& 0 & 0 & 0 & 0 & 0 &      1.00 & & \textbf{     0.06} & \textbf{     0.08} & \textbf{     0.11} & \textbf{0} & \textbf{     0.06} & &      0.01 \\*
\\
\quad \quad \textbf{Adult 40} & & & & & & & & \multicolumn{6}{c}{\highlight{Reference mean = \textbf{     0.16}}} \\*[.1cm]
\quad \quad \quad Reggio& 0 & 0 & 0 & 0 & 0 &      1.00 & & 0.16 & \textbf{     0.18} & \textbf{     0.15} & \textbf{     0.40} & \textbf{     0.11} & &      0.01 \\*
\quad \quad \quad Parma& 0 & 0 & 0 & 0 & 0 &      1.00 & & \textbf{     0.10} & \textbf{     0.04} & \textbf{     0.15} & \textbf{0} & \textbf{     0.12} & &      0.01 \\*
\quad \quad \quad Padova& 0 & 0 & 0 & . & 0 &      1.00 & & \textbf{0} & \textbf{     0.04} & 0.13 & . & 0.16 & &      0.03 \\*
\\
\quad \quad \textbf{Adult 50} & & & & & & & & \multicolumn{6}{c}{\highlight{Reference mean = \textbf{     0.11}}} \\*[.1cm]
\quad \quad \quad Reggio& 0 & 0 & 0 & 0 & 0 &      1.00 & & 0.11 & \textbf{     0.20} & \textbf{0} & \textbf{0} & 0.12 & &      0.02 \\*
\quad \quad \quad Parma& 0 & 0 & 0 & . & 0 &      1.00 & & \textbf{     0.08} & \textbf{0} & \textbf{     0.27} & . & \textbf{     0.14} & &      0.03 \\*
\quad \quad \quad Padova& 0 & 0 & 0 & 0 & 0 &      1.00 & & \textbf{     0.18} & \textbf{0} & \textbf{     0.09} & \textbf{0} & \textbf{     0.13} & &      0.01 \\*
\\
~\\*[.05cm]
\textbf{Hours Worked Per Week} \\*[.1cm]
\quad \quad \textbf{Adult 30} & & & & & & & & \multicolumn{6}{c}{\highlight{Reference mean = \textbf{    42.24}}} \\*[.1cm]
\quad \quad \quad Reggio& 43.29 & \textbf{    42.94} & \textbf{    41.50} & \textbf{    44.17} & 38.36 &      0.27 & & 42.24 & \textbf{    40.33} & 39.90 & \textbf{    50.00} & 35.77 & &      0.11 \\*
\quad \quad \quad Parma& \textbf{    26.02} & \textbf{    24.71} & \textbf{    24.18} & \textbf{    22.39} & 23.17 &      0.29 & & \textbf{    39.26} & \textbf{    37.41} & \textbf{    37.47} & 31.25 & \textbf{    38.71} & &      0.02 \\*
\quad \quad \quad Padova& \textbf{    38.70} & 35.03 & \textbf{    39.90} & . & \textbf{    38.51} &      0.30 & & \textbf{    39.67} & 34.50 & \textbf{    40.66} & . & \textbf{    39.19} & &      0.05 \\*
\\
\quad \quad \textbf{Adult 40} & & & & & & & & \multicolumn{6}{c}{\highlight{Reference mean = \textbf{    42.71}}} \\*[.1cm]
\quad \quad \quad Reggio& 37.24 & \textbf{    37.67} & \textbf{    38.25} & 29.42 & 34.49 &      0.39 & & 42.71 & \textbf{    43.33} & 42.88 & \textbf{    38.00} & 37.97 & &      0.05 \\*
\quad \quad \quad Parma& \textbf{    36.19} & \textbf{    35.83} & \textbf{    36.27} & \textbf{    39.92} & \textbf{    35.13} &      0.36 & & \textbf{    40.64} & \textbf{    38.35} & \textbf{    40.80} & \textbf{    44.00} & \textbf{    39.31} & &      0.02 \\*
\quad \quad \quad Padova& \textbf{    42.06} & 26.34 & \textbf{    43.43} & . & \textbf{    42.21} &      0.41 & & \textbf{    35.61} & 19.46 & 39.44 & . & \textbf{    39.23} & &      0.28 \\*
\\
\quad \quad \textbf{Adult 50} & & & & & & & & \multicolumn{6}{c}{\highlight{Reference mean = \textbf{    40.50}}} \\*[.1cm]
\quad \quad \quad Reggio& 39.93 & 39.31 & \textbf{    38.56} & 40.26 & 39.57 &      0.39 & & 40.50 & 40.75 & \textbf{    38.91} & 40.00 & 40.61 & &      0.01 \\*
\quad \quad \quad Parma& \textbf{    45.23} & \textbf{    46.36} & \textbf{    44.23} & . & \textbf{    44.26} &      0.50 & & 40.55 & \textbf{    42.00} & \textbf{    41.36} & . & \textbf{    41.27} & &      0.00 \\*
\quad \quad \quad Padova& \textbf{    57.42} & \textbf{    61.92} & \textbf{    62.95} & 71.17 & \textbf{    62.20} &      0.37 & & \textbf{    32.37} & \textbf{    39.00} & \textbf{    37.89} & \textbf{    38.00} & \textbf{    36.54} & &      0.02 \\*
\\
~\\*[.05cm]
\textbf{Monthly Wage} \\*[.1cm]
\quad \quad \textbf{Adult 30} & & & & & & & & \multicolumn{6}{c}{\highlight{Reference mean = \textbf{  1068.74}}} \\*[.1cm]
\quad \quad \quad Reggio& 1234.73 & \textbf{  1285.30} & \textbf{  1288.86} & 2108.66 & 1518.58 &      0.41 & & 1068.74 & \textbf{  1005.56} & \textbf{  1122.67} & 2000.00 & 1257.14 & &      0.04 \\*
\quad \quad \quad Parma& \textbf{   941.91} & \textbf{   915.51} & \textbf{   943.58} & 1089.45 & \textbf{   774.75} &      0.17 & & 1111.11 & \textbf{   969.00} & \textbf{   950.00} & 975.00 & \textbf{   920.00} & &      0.02 \\*
\quad \quad \quad Padova& \textbf{  1480.35} & 1190.74 & \textbf{  1673.16} & . & \textbf{  1468.13} &      0.43 & & \textbf{  1180.80} & \textbf{  1246.18} & \textbf{  1525.22} & . & \textbf{  1150.00} & &      0.05 \\*
\\
\quad \quad \textbf{Adult 40} & & & & & & & & \multicolumn{6}{c}{\highlight{Reference mean = \textbf{  1856.63}}} \\*[.1cm]
\quad \quad \quad Reggio& 587.66 & \textbf{  1909.84} & \textbf{  1177.70} & \textbf{   207.04} & 1732.52 &      0.32 & & 1856.63 & \textbf{  2592.22} & \textbf{  2185.29} & \textbf{  1000.00} & \textbf{  2706.05} & &      0.01 \\*
\quad \quad \quad Parma& \textbf{  1628.46} & 680.29 & \textbf{   803.61} & . & \textbf{   910.24} &      0.35 & & \textbf{  1350.00} & \textbf{  1025.00} & \textbf{  1150.00} & . & \textbf{  1161.54} & &      0.01 \\*
\quad \quad \quad Padova& \textbf{  1376.39} & -548.55 & \textbf{  1452.57} & . & \textbf{  1200.93} &      0.52 & & \textbf{  1700.00} & 123.75 & 1885.71 & . & 1884.92 & &      0.23 \\*
\\
\quad \quad \textbf{Adult 50} & & & & & & & & \multicolumn{6}{c}{\highlight{Reference mean = \textbf{  1300.00}}} \\*[.1cm]
\quad \quad \quad Reggio& 1396.60 & 1467.73 & 1451.06 & \textbf{  2082.38} & 1432.45 &      0.34 & & 1300.00 & 1316.67 & \textbf{  1600.00} & \textbf{  2000.00} & \textbf{  1457.53} & &      0.05 \\*
\quad \quad \quad Parma& \textbf{   810.65} & . & 1827.11 & . & \textbf{   810.65} &      0.47 & & \textbf{  1369.57} & . & 2166.67 & . & \textbf{  1369.57} & &      0.19 \\*
\quad \quad \quad Padova& \textbf{  5344.41} & . & \textbf{  3311.44} & . & 1820.17 &      0.11 & & \textbf{  3500.00} & . & \textbf{  2409.52} & . & 1310.88 & &      0.02 \\*
\\
~\\*[.05cm]
\textbf{H. Income: 5,000 Euros of Less} \\*[.1cm]
\quad \quad \textbf{Adult 30} & & & & & & & & \multicolumn{6}{c}{\highlight{Reference mean = \textbf{     0.15}}} \\*[.1cm]
\quad \quad \quad Reggio& -0.02 & \textbf{    -0.08} & \textbf{    -0.06} & \textbf{    -0.13} & -0.21 &      0.21 & & 0.15 & \textbf{     0.13} & \textbf{     0.10} & \textbf{0} & 0 & &      0.03 \\*
\quad \quad \quad Parma& \textbf{     0.03} & 0.07 & \textbf{     0.05} & \textbf{     0.03} & \textbf{     0.03} &      0.27 & & \textbf{0} & 0.06 & \textbf{     0.02} & \textbf{0} & \textbf{0} & &      0.03 \\*
\quad \quad \quad Padova& \textbf{    -0.01} & 0.03 & \textbf{0} & -0.01 & -0.02 &      0.04 & & \textbf{     0.03} & \textbf{     0.04} & \textbf{     0.01} & \textbf{0} & \textbf{0} & &      0.01 \\*
\\
\quad \quad \textbf{Adult 40} & & & & & & & & \multicolumn{6}{c}{\highlight{Reference mean = \textbf{     0.00}}} \\*[.1cm]
\quad \quad \quad Reggio& \textbf{    -0.02} & \textbf{    -0.03} & \textbf{    -0.02} & \textbf{    -0.02} & \textbf{    -0.01} &      0.08 & & 0 & 0.06 & 0 & 0 & \textbf{     0.01} & &      0.03 \\*
\quad \quad \quad Parma& \textbf{     0.06} & 0.10 & \textbf{     0.07} & \textbf{     0.07} & \textbf{     0.07} &      0.08 & & 0 & 0.04 & 0 & 0 & \textbf{     0.01} & &      0.02 \\*
\quad \quad \quad Padova& \textbf{     0.02} & 0.18 & \textbf{     0.03} & . & \textbf{     0.03} &      0.19 & & 0 & 0.17 & \textbf{     0.01} & . & 0 & &      0.12 \\*
\\
\quad \quad \textbf{Adult 50} & & & & & & & & \multicolumn{6}{c}{\highlight{Reference mean = \textbf{     0.00}}} \\*[.1cm]
\quad \quad \quad Reggio& -0.03 & \textbf{    -0.06} & -0.02 & -0.02 & \textbf{    -0.01} &      0.16 & & 0 & 0 & \textbf{     0.04} & 0 & \textbf{     0.01} & &      0.01 \\*
\quad \quad \quad Parma& -0.02 & \textbf{0} & \textbf{     0.01} & . & \textbf{     0.01} &      0.16 & & 0 & 0 & 0 & . & \textbf{     0.01} & &      0.00 \\*
\quad \quad \quad Padova& \textbf{     0.04} & \textbf{     0.05} & \textbf{     0.03} & -0.01 & \textbf{     0.06} &      0.09 & & 0 & 0 & \textbf{     0.01} & 0 & \textbf{     0.04} & &      0.01 \\*
\\
~\\*[.05cm]
\textbf{H. Income: 5,001-10,000 Euros} \\*[.1cm]
\quad \quad \textbf{Adult 30} & & & & & & & & \multicolumn{6}{c}{\highlight{Reference mean = \textbf{     0.01}}} \\*[.1cm]
\quad \quad \quad Reggio& \textbf{     0.04} & \textbf{     0.03} & \textbf{     0.04} & 0.05 & \textbf{     0.03} &      0.05 & & \textbf{     0.01} & \textbf{0} & \textbf{0} & 0 & 0.04 & &      0.01 \\*
\quad \quad \quad Parma& \textbf{    -0.03} & \textbf{    -0.03} & \textbf{    -0.01} & \textbf{    -0.03} & \textbf{    -0.01} &      0.04 & & \textbf{0} & \textbf{0} & \textbf{     0.02} & \textbf{0} & 0.05 & &      0.03 \\*
\quad \quad \quad Padova& \textbf{    -0.01} & \textbf{    -0.02} & \textbf{     0.01} & \textbf{0} & \textbf{    -0.01} &      0.04 & & \textbf{0} & \textbf{0} & \textbf{     0.02} & 0 & \textbf{0} & &      0.01 \\*
\\
\quad \quad \textbf{Adult 40} & & & & & & & & \multicolumn{6}{c}{\highlight{Reference mean = \textbf{     0.00}}} \\*[.1cm]
\quad \quad \quad Reggio& \textbf{0} & \textbf{0} & \textbf{0} & 0.20 & \textbf{0} &      0.23 & & 0 & 0 & 0 & 0.20 & 0 & &      0.20 \\*
\quad \quad \quad Parma& \textbf{    -0.03} & \textbf{    -0.04} & \textbf{    -0.03} & \textbf{     0.01} & \textbf{    -0.02} &      0.10 & & 0 & 0 & 0 & 0 & \textbf{     0.02} & &      0.01 \\*
\quad \quad \quad Padova& \textbf{     0.11} & 0.02 & 0 & . & 0 &      0.15 & & \textbf{     0.11} & 0.04 & 0.01 & . & 0 & &      0.06 \\*
\\
\quad \quad \textbf{Adult 50} & & & & & & & & \multicolumn{6}{c}{\highlight{Reference mean = \textbf{     0.00}}} \\*[.1cm]
\quad \quad \quad Reggio& 0 & 0 & 0 & 0 & 0 &         . & & 0 & 0 & 0 & 0 & 0 & &         . \\*
\quad \quad \quad Parma& \textbf{     0.54} & \textbf{     0.44} & \textbf{     0.56} & . & \textbf{     0.58} &      0.30 & & 0 & 0 & 0 & . & \textbf{     0.03} & &      0.01 \\*
\quad \quad \quad Padova& 0.03 & 0.02 & \textbf{     0.05} & \textbf{     0.04} & \textbf{     0.14} &      0.23 & & 0 & 0 & \textbf{     0.01} & 0 & \textbf{     0.11} & &      0.04 \\*
\\
~\\*[.05cm]
\textbf{H. Income: 10,001-25,000 Euros} \\*[.1cm]
\quad \quad \textbf{Adult 30} & & & & & & & & \multicolumn{6}{c}{\highlight{Reference mean = \textbf{     0.28}}} \\*[.1cm]
\quad \quad \quad Reggio& 0.39 & \textbf{     0.36} & \textbf{     0.44} & \textbf{     0.27} & 0.59 &      0.09 & & 0.28 & \textbf{     0.23} & \textbf{     0.35} & \textbf{0} & 0.47 & &      0.03 \\*
\quad \quad \quad Parma& \textbf{     0.98} & \textbf{     0.96} & \textbf{     0.89} & 1.39 & \textbf{     0.95} &      0.06 & & \textbf{     0.48} & \textbf{     0.45} & \textbf{     0.36} & \textbf{     0.80} & \textbf{     0.43} & &      0.02 \\*
\quad \quad \quad Padova& \textbf{     0.21} & \textbf{     0.31} & \textbf{     0.34} & \textbf{    -0.03} & 0.40 &      0.12 & & \textbf{     0.26} & \textbf{     0.35} & \textbf{     0.39} & \textbf{0} & 0.53 & &      0.03 \\*
\\
\quad \quad \textbf{Adult 40} & & & & & & & & \multicolumn{6}{c}{\highlight{Reference mean = \textbf{     0.30}}} \\*[.1cm]
\quad \quad \quad Reggio& 0.44 & \textbf{     0.42} & \textbf{     0.40} & \textbf{     0.56} & \textbf{     0.49} &      0.10 & & 0.30 & \textbf{     0.35} & \textbf{     0.23} & \textbf{     0.40} & \textbf{     0.35} & &      0.01 \\*
\quad \quad \quad Parma& 0.44 & 0.77 & \textbf{     0.59} & \textbf{     0.26} & 0.65 &      0.15 & & \textbf{     0.19} & 0.50 & 0.29 & \textbf{0} & 0.41 & &      0.04 \\*
\quad \quad \quad Padova& \textbf{     0.91} & \textbf{     0.82} & \textbf{     0.81} & . & \textbf{     0.93} &      0.13 & & \textbf{     0.37} & \textbf{     0.17} & \textbf{     0.24} & . & \textbf{     0.40} & &      0.03 \\*
\\
\quad \quad \textbf{Adult 50} & & & & & & & & \multicolumn{6}{c}{\highlight{Reference mean = \textbf{     0.33}}} \\*[.1cm]
\quad \quad \quad Reggio& 0.99 & \textbf{     0.85} & \textbf{     0.73} & \textbf{     1.18} & \textbf{     0.90} &      0.14 & & 0.33 & \textbf{     0.20} & \textbf{     0.07} & \textbf{     0.50} & \textbf{     0.29} & &      0.03 \\*
\quad \quad \quad Parma& \textbf{    -0.54} & \textbf{    -0.23} & \textbf{    -0.24} & . & -0.24 &      0.29 & & \textbf{     0.08} & \textbf{     0.29} & 0.36 & . & 0.47 & &      0.07 \\*
\quad \quad \quad Padova& 0.91 & \textbf{     1.10} & 1.02 & \textbf{     0.67} & 0.93 &      0.13 & & \textbf{     0.27} & \textbf{     0.50} & 0.34 & \textbf{0} & \textbf{     0.21} & &      0.03 \\*
\\
~\\*[.05cm]
\textbf{H. Income: 25,001-50,000 Euros} \\*[.1cm]
\quad \quad \textbf{Adult 30} & & & & & & & & \multicolumn{6}{c}{\highlight{Reference mean = \textbf{     0.53}}} \\*[.1cm]
\quad \quad \quad Reggio& 0.58 & \textbf{     0.68} & \textbf{     0.62} & \textbf{     0.86} & 0.56 &      0.09 & & 0.53 & \textbf{     0.61} & \textbf{     0.55} & \textbf{     1.00} & \textbf{     0.46} & &      0.01 \\*
\quad \quad \quad Parma& \textbf{     0.08} & \textbf{     0.05} & \textbf{     0.07} & \textbf{    -0.26} & \textbf{     0.08} &      0.08 & & \textbf{     0.43} & \textbf{     0.41} & \textbf{     0.44} & \textbf{     0.20} & \textbf{     0.45} & &      0.01 \\*
\quad \quad \quad Padova& \textbf{     0.73} & \textbf{     0.62} & 0.58 & \textbf{     1.09} & \textbf{     0.66} &      0.10 & & \textbf{     0.60} & 0.54 & \textbf{     0.46} & \textbf{     1.00} & \textbf{     0.47} & &      0.01 \\*
\\
\quad \quad \textbf{Adult 40} & & & & & & & & \multicolumn{6}{c}{\highlight{Reference mean = \textbf{     0.62}}} \\*[.1cm]
\quad \quad \quad Reggio& 0.52 & \textbf{     0.50} & \textbf{     0.42} & \textbf{     0.28} & \textbf{     0.42} &      0.09 & & 0.62 & \textbf{     0.53} & \textbf{     0.54} & \textbf{     0.40} & \textbf{     0.54} & &      0.01 \\*
\quad \quad \quad Parma& \textbf{     0.46} & 0.17 & \textbf{     0.47} & \textbf{     0.71} & 0.30 &      0.12 & & 0.62 & 0.38 & \textbf{     0.69} & \textbf{     1.00} & \textbf{     0.52} & &      0.04 \\*
\quad \quad \quad Padova& \textbf{    -0.07} & \textbf{     0.03} & \textbf{     0.08} & . & \textbf{     0.06} &      0.05 & & \textbf{     0.44} & \textbf{     0.54} & \textbf{     0.56} & . & \textbf{     0.53} & &      0.00 \\*
\\
\quad \quad \textbf{Adult 50} & & & & & & & & \multicolumn{6}{c}{\highlight{Reference mean = \textbf{     0.67}}} \\*[.1cm]
\quad \quad \quad Reggio& 0.06 & \textbf{     0.25} & \textbf{     0.22} & \textbf{    -0.06} & 0.13 &      0.09 & & 0.67 & \textbf{     0.80} & \textbf{     0.71} & \textbf{     0.50} & \textbf{     0.63} & &      0.01 \\*
\quad \quad \quad Parma& \textbf{     0.87} & \textbf{     0.78} & 0.47 & . & 0.56 &      0.16 & & 0.67 & \textbf{     0.57} & \textbf{     0.45} & . & 0.40 & &      0.03 \\*
\quad \quad \quad Padova& 0.05 & 0.08 & -0.01 & \textbf{     0.42} & \textbf{    -0.06} &      0.09 & & \textbf{     0.55} & \textbf{     0.50} & \textbf{     0.49} & \textbf{     1.00} & \textbf{     0.47} & &      0.02 \\*
\\
~\\*[.05cm]
\textbf{H. Income: 50,001-100,000 Euros} \\*[.1cm]
\quad \quad \textbf{Adult 30} & & & & & & & & \multicolumn{6}{c}{\highlight{Reference mean = \textbf{     0.04}}} \\*[.1cm]
\quad \quad \quad Reggio& 0.01 & \textbf{0} & \textbf{    -0.03} & \textbf{    -0.05} & \textbf{     0.03} &      0.11 & & 0.04 & \textbf{     0.03} & \textbf{0} & \textbf{0} & 0.04 & &      0.01 \\*
\quad \quad \quad Parma& \textbf{    -0.07} & \textbf{    -0.06} & \textbf{    -0.04} & \textbf{    -0.14} & \textbf{    -0.08} &      0.09 & & \textbf{     0.09} & \textbf{     0.08} & \textbf{     0.12} & \textbf{0} & \textbf{     0.07} & &      0.01 \\*
\quad \quad \quad Padova& \textbf{     0.08} & \textbf{     0.06} & \textbf{     0.07} & \textbf{    -0.05} & -0.04 &      0.21 & & \textbf{     0.11} & \textbf{     0.08} & \textbf{     0.11} & \textbf{0} & 0 & &      0.02 \\*
\\
\quad \quad \textbf{Adult 40} & & & & & & & & \multicolumn{6}{c}{\highlight{Reference mean = \textbf{     0.06}}} \\*[.1cm]
\quad \quad \quad Reggio& 0.06 & \textbf{     0.05} & 0.17 & \textbf{     0.04} & \textbf{     0.04} &      0.26 & & 0.06 & \textbf{0} & 0.15 & \textbf{0} & \textbf{     0.04} & &      0.03 \\*
\quad \quad \quad Parma& \textbf{     0.09} & \textbf{     0.02} & -0.02 & \textbf{    -0.01} & \textbf{     0.02} &      0.14 & & \textbf{     0.15} & 0.04 & 0.02 & \textbf{0} & 0.04 & &      0.04 \\*
\quad \quad \quad Padova& \textbf{    -0.01} & \textbf{    -0.03} & 0.04 & . & \textbf{    -0.04} &      0.16 & & \textbf{     0.07} & \textbf{     0.08} & \textbf{     0.15} & . & \textbf{     0.07} & &      0.02 \\*
\\
\quad \quad \textbf{Adult 50} & & & & & & & & \multicolumn{6}{c}{\highlight{Reference mean = \textbf{     0.00}}} \\*[.1cm]
\quad \quad \quad Reggio& -0.01 & -0.05 & \textbf{     0.08} & \textbf{    -0.10} & -0.02 &      0.11 & & 0 & 0 & 0.18 & 0 & \textbf{     0.08} & &      0.03 \\*
\quad \quad \quad Parma& \textbf{     0.13} & 0 & \textbf{     0.12} & . & \textbf{     0.08} &      0.26 & & \textbf{     0.25} & \textbf{     0.14} & \textbf{     0.09} & . & 0.08 & &      0.03 \\*
\quad \quad \quad Padova& \textbf{    -0.09} & \textbf{    -0.21} & \textbf{    -0.08} & \textbf{    -0.12} & -0.04 &      0.14 & & \textbf{     0.09} & 0 & \textbf{     0.13} & 0 & \textbf{     0.18} & &      0.01 \\*
\\
~\\*[.05cm]
\textbf{H. Income: 100,001-250,000 Euros} \\*[.1cm]
\quad \quad \textbf{Adult 30} & & & & & & & & \multicolumn{6}{c}{\highlight{Reference mean = \textbf{     0.00}}} \\*[.1cm]
\quad \quad \quad Reggio& 0 & 0 & 0 & 0 & 0 &         . & & 0 & 0 & 0 & 0 & 0 & &         . \\*
\quad \quad \quad Parma& \textbf{     0.01} & 0.01 & 0.05 & 0.01 & \textbf{     0.02} &      0.06 & & 0 & 0 & 0.04 & 0 & 0 & &      0.03 \\*
\quad \quad \quad Padova& 0 & 0 & 0 & 0 & 0 &      1.00 & & 0 & 0 & \textbf{     0.01} & 0 & 0 & &      0.00 \\*
\\
\quad \quad \textbf{Adult 40} & & & & & & & & \multicolumn{6}{c}{\highlight{Reference mean = \textbf{     0.02}}} \\*[.1cm]
\quad \quad \quad Reggio& -0.01 & \textbf{     0.05} & \textbf{     0.03} & \textbf{    -0.06} & 0.05 &      0.10 & & \textbf{     0.02} & \textbf{     0.06} & \textbf{     0.08} & \textbf{0} & \textbf{     0.06} & &      0.01 \\*
\quad \quad \quad Parma& \textbf{    -0.02} & \textbf{    -0.02} & -0.07 & \textbf{    -0.04} & \textbf{    -0.03} &      0.32 & & \textbf{     0.04} & \textbf{     0.04} & \textbf{0} & \textbf{0} & \textbf{     0.01} & &      0.02 \\*
\quad \quad \quad Padova& \textbf{     0.03} & -0.02 & \textbf{     0.05} & . & \textbf{     0.04} &      0.11 & & \textbf{0} & \textbf{0} & \textbf{     0.02} & . & \textbf{0} & &      0.01 \\*
\\
\quad \quad \textbf{Adult 50} & & & & & & & & \multicolumn{6}{c}{\highlight{Reference mean = \textbf{     0.00}}} \\*[.1cm]
\quad \quad \quad Reggio& 0 & 0 & 0 & 0 & 0 &         . & & 0 & 0 & 0 & 0 & 0 & &         . \\*
\quad \quad \quad Parma& \textbf{     0.02} & 0.01 & \textbf{     0.08} & . & 0.01 &      0.16 & & 0 & 0 & 0.09 & . & 0 & &      0.08 \\*
\quad \quad \quad Padova& \textbf{     0.07} & \textbf{    -0.03} & -0.03 & 0 & -0.02 &      0.18 & & \textbf{     0.09} & 0 & 0 & 0 & 0 & &      0.08 \\*
\\
~\\*[.05cm]
\textbf{H. Income: More than 250,000 Euros} \\*[.1cm]
\quad \quad \textbf{Adult 30} & & & & & & & & \multicolumn{6}{c}{\highlight{Reference mean = \textbf{     0.00}}} \\*[.1cm]
\quad \quad \quad Reggio& 0 & 0 & 0 & 0 & 0 &         . & & 0 & 0 & 0 & 0 & 0 & &         . \\*
\quad \quad \quad Parma& 0 & 0 & 0 & 0 & 0 &         . & & 0 & 0 & 0 & 0 & 0 & &         . \\*
\quad \quad \quad Padova& 0 & 0 & 0 & 0 & 0 &         . & & 0 & 0 & 0 & 0 & 0 & &         . \\*
\\
\quad \quad \textbf{Adult 40} & & & & & & & & \multicolumn{6}{c}{\highlight{Reference mean = \textbf{     0.00}}} \\*[.1cm]
\quad \quad \quad Reggio& 0 & 0 & 0 & 0 & 0 &         . & & 0 & 0 & 0 & 0 & 0 & &         . \\*
\quad \quad \quad Parma& 0 & 0 & 0 & 0 & 0 &         . & & 0 & 0 & 0 & 0 & 0 & &         . \\*
\quad \quad \quad Padova& 0 & 0 & 0 & . & 0 &         . & & 0 & 0 & 0 & . & 0 & &         . \\*
\\
\quad \quad \textbf{Adult 50} & & & & & & & & \multicolumn{6}{c}{\highlight{Reference mean = \textbf{     0.00}}} \\*[.1cm]
\quad \quad \quad Reggio& 0 & 0 & 0 & 0 & 0 &         . & & 0 & 0 & 0 & 0 & 0 & &         . \\*
\quad \quad \quad Parma& 0 & 0 & 0 & . & 0 &         . & & 0 & 0 & 0 & . & 0 & &         . \\*
\quad \quad \quad Padova& 0 & -0.01 & 0.01 & 0 & 0 &      0.03 & & 0 & 0 & \textbf{     0.01} & 0 & 0 & &      0.01 \\*
\\

			\end{longtable}
		}		
\end{center}
\end{landscape}

%--------------

\subsection{Employment and Earnings - Difference-in-Difference Results}


\begin{table}[H]
\begin{center}
	\caption{Difference-in-Difference Across School Types and Cities, Restricting to Age-30 Cohort} \label{table:WCh-30}
	\scalebox{0.80}{
		\begin{tabular}{lcccccccc}
\toprule
 \textbf{Outcome} & \textbf{(1)} & \textbf{(2)} & \textbf{(3)} & \textbf{(4)} & \textbf{(5)} & \textbf{(6)} & \textbf{N} & \textbf{$ R^2$} \\
\midrule
Employed & \textbf{     0.14} &      0.06 &      0.04 &      0.03 &      0.05 &     -0.09 & 782 &       0.06 \\ 
 & \textbf{(     0.07 )} & (     0.07 ) & (     0.07 ) & (     0.08 ) & (     0.09 ) & (     0.07 ) & \\
Self-Employed &      0.00 &     -0.01 &     -0.01 &     -0.08 &      0.05 &      0.04 & 768 &       0.04 \\ 
 & (     0.07 ) & (     0.08 ) & (     0.07 ) & (     0.08 ) & (     0.09 ) & (     0.08 ) & \\
Hours Worked Per Week & \textbf{     5.69} &      0.18 &      0.99 & \textbf{     4.88} &     -3.46 &      3.01 & 655 &       0.09 \\ 
 & \textbf{(     2.12 )} & (     2.53 ) & (     2.14 ) & \textbf{(     2.36 )} & (     2.95 ) & (     2.16 ) & \\
Monthly Wage &   -435.61 &   -125.35 &   -219.65 &   -297.66 &     -9.34 &    164.36 & 285 &       0.14 \\ 
 & (   324.96 ) & (   271.68 ) & (   307.67 ) & (   338.69 ) & (   312.48 ) & (   264.89 ) & \\
H. Income: 5,000 Euros of Less & \textbf{     0.16} &      0.08 &      0.04 & \textbf{     0.12} &      0.04 &      0.02 & 782 &       0.10 \\ 
 & \textbf{(     0.05 )} & (     0.05 ) & (     0.05 ) & \textbf{(     0.06 )} & (     0.07 ) & (     0.05 ) & \\
H. Income: 5,001-10,000 Euros &      0.02 &      0.01 &      0.03 &     -0.03 &      0.01 &      0.03 & 782 &       0.02 \\ 
 & (     0.03 ) & (     0.03 ) & (     0.03 ) & (     0.03 ) & (     0.03 ) & (     0.03 ) & \\
H. Income: 10,001-25,000 Euros & \textbf{    -0.27} &      0.00 & \textbf{    -0.21} &      0.05 &      0.11 &      0.04 & 782 &       0.04 \\ 
 & \textbf{(     0.12 )} & (     0.13 ) & \textbf{(     0.12 )} & (     0.13 ) & (     0.16 ) & (     0.12 ) & \\
H. Income: 25,001-50,000 Euros &      0.10 &     -0.08 &      0.03 &     -0.03 &     -0.14 &     -0.14 & 782 &       0.03 \\ 
 & (     0.12 ) & (     0.13 ) & (     0.13 ) & (     0.14 ) & (     0.16 ) & (     0.13 ) & \\
H. Income: 50,001-100,000 Euros &     -0.01 &     -0.01 &      0.06 & \textbf{    -0.11} &     -0.02 &      0.04 & 782 &       0.03 \\ 
 & (     0.06 ) & (     0.07 ) & (     0.06 ) & \textbf{(     0.07 )} & (     0.08 ) & (     0.06 ) & \\
H. Income: 100,001-250,000 Euros &      0.00 &      0.00 & \textbf{     0.04} &      0.00 &      0.00 &      0.01 & 782 &       0.03 \\ 
 & (     0.01 ) & (     0.02 ) & \textbf{(     0.02 )} & (     0.02 ) & (     0.02 ) & (     0.02 ) & \\
H. Income: More than 250,000 Euros &      0.00 &      0.00 &      0.00 &      0.00 &      0.00 &      0.00 & 782 &          . \\ 
 & (        . ) & (        . ) & (        . ) & (        . ) & (        . ) & (        . ) & \\
\bottomrule
\end{tabular}
}
\end{center}
\footnotesize
\fnDID
\end{table}

\begin{table}[H]
\begin{center}
	\caption{Difference-in-Difference Across School Types and Cities, Restricting to Age-40 Cohort} \label{table:WCh-40}
	\scalebox{0.80}{
		\begin{tabular}{lcccccccc}
\toprule
 \textbf{Outcome} & \textbf{(1)} & \textbf{(2)} & \textbf{(3)} & \textbf{(4)} & \textbf{(5)} & \textbf{(6)} & \textbf{N} & \textbf{$ R^2$} \\
\midrule
Employed &      0.08 &     -0.01 &     -0.00 & \textbf{     0.12} & \textbf{     0.16} &     -0.01 & 790 &       0.03 \\ 
 & (     0.05 ) & (     0.09 ) & (     0.06 ) & \textbf{(     0.06 )} & \textbf{(     0.09 )} & (     0.07 ) & \\
Self-Employed &      0.06 &     -0.09 &      0.03 & \textbf{     0.20} &      0.03 &      0.13 & 782 &       0.02 \\ 
 & (     0.07 ) & (     0.12 ) & (     0.08 ) & \textbf{(     0.09 )} & (     0.13 ) & (     0.09 ) & \\
Hours Worked Per Week & \textbf{     4.28} &     -0.95 &      0.30 & \textbf{     6.23} & \textbf{   -18.14} &      1.71 & 701 &       0.23 \\ 
 & \textbf{(     2.04 )} & (     3.53 ) & (     2.29 ) & \textbf{(     2.43 )} & \textbf{(     3.62 )} & (     2.44 ) & \\
Monthly Wage &   -865.87 &  -1088.48 &   -641.28 &   -498.66 &  -2293.70 &    -76.18 & 264 &       0.07 \\ 
 & (  2048.83 ) & (  2434.85 ) & (  2037.54 ) & (  1312.11 ) & (  1542.64 ) & (  1212.65 ) & \\
H. Income: 5,000 Euros of Less &     -0.01 &     -0.03 &     -0.00 &     -0.01 & \textbf{     0.11} &      0.01 & 791 &       0.08 \\ 
 & (     0.02 ) & (     0.04 ) & (     0.03 ) & (     0.03 ) & \textbf{(     0.04 )} & (     0.03 ) & \\
H. Income: 5,001-10,000 Euros &      0.01 &     -0.01 &      0.00 & \textbf{    -0.10} &     -0.06 & \textbf{    -0.09} & 791 &       0.07 \\ 
 & (     0.02 ) & (     0.03 ) & (     0.02 ) & \textbf{(     0.03 )} & (     0.04 ) & \textbf{(     0.03 )} & \\
H. Income: 10,001-25,000 Euros &      0.15 &      0.24 &      0.16 &     -0.03 &     -0.28 &     -0.07 & 791 &       0.03 \\ 
 & (     0.10 ) & (     0.16 ) & (     0.12 ) & (     0.12 ) & (     0.17 ) & (     0.12 ) & \\
H. Income: 25,001-50,000 Euros &     -0.02 &     -0.14 &      0.17 &      0.18 &      0.19 &      0.21 & 791 &       0.02 \\ 
 & (     0.11 ) & (     0.18 ) & (     0.12 ) & (     0.13 ) & (     0.19 ) & (     0.13 ) & \\
H. Income: 50,001-100,000 Euros &     -0.07 &     -0.04 & \textbf{    -0.23} &     -0.01 &      0.08 &     -0.05 & 791 &       0.06 \\ 
 & (     0.06 ) & (     0.09 ) & \textbf{(     0.07 )} & (     0.07 ) & (     0.10 ) & (     0.07 ) & \\
H. Income: 100,001-250,000 Euros & \textbf{    -0.07} &     -0.03 & \textbf{    -0.09} &     -0.04 &     -0.04 &     -0.02 & 791 &       0.03 \\ 
 & \textbf{(     0.03 )} & (     0.06 ) & \textbf{(     0.04 )} & (     0.04 ) & (     0.06 ) & (     0.04 ) & \\
H. Income: More than 250,000 Euros &      0.00 &      0.00 &      0.00 &      0.00 &      0.00 &      0.00 & 791 &          . \\ 
 & (        . ) & (        . ) & (        . ) & (        . ) & (        . ) & (        . ) & \\
\bottomrule
\end{tabular}
}
\end{center}
\footnotesize
\fnDID
\end{table}

%\begin{table}[H]
%\begin{center}
%	\caption{Difference-in-Difference Across School Types and Cities, Restricting to Age-50 Cohort} \label{table:WCh-50}
%	\scalebox{0.80}{
%		\begin{tabular}{lcccccccc}
\toprule
 \textbf{Outcome} & \textbf{(1)} & \textbf{(2)} & \textbf{(3)} & \textbf{(4)} & \textbf{(5)} & \textbf{(6)} & \textbf{N} & \textbf{$ R^2$} \\
\midrule
Employed &     -0.07 &     -0.14 &      0.13 &      0.09 &      0.30 &      0.19 & 448 &       0.12 \\ 
 & (     0.15 ) & (     0.22 ) & (     0.19 ) & (     0.14 ) & (     0.30 ) & (     0.15 ) & \\
Self-Employed &      0.11 &     -0.12 & \textbf{     0.37} &      0.03 &     -0.17 &      0.12 & 439 &       0.03 \\ 
 & (     0.14 ) & (     0.20 ) & \textbf{(     0.17 )} & (     0.13 ) & (     0.28 ) & (     0.14 ) & \\
Hours Worked Per Week &     -0.84 &      1.63 &     -0.66 &      2.91 &      4.68 &      5.43 & 362 &       0.15 \\ 
 & (     3.20 ) & (     4.99 ) & (     3.97 ) & (     3.23 ) & (     6.30 ) & (     3.46 ) & \\
Monthly Wage &      0.00 &      0.00 &    699.43 &   -474.27 &      0.00 &    499.58 & 129 &       0.03 \\ 
 & (        . ) & (        . ) & (  1578.99 ) & (  1601.11 ) & (        . ) & (  1656.27 ) & \\
H. Income: 5,000 Euros of Less &      0.03 &      0.01 &     -0.01 &      0.04 &      0.01 &     -0.02 & 449 &       0.02 \\ 
 & (     0.05 ) & (     0.07 ) & (     0.06 ) & (     0.05 ) & (     0.10 ) & (     0.05 ) & \\
H. Income: 5,001-10,000 Euros &      0.02 &     -0.01 &      0.00 & \textbf{     0.10} &     -0.00 &      0.01 & 449 &       0.07 \\ 
 & (     0.06 ) & (     0.09 ) & (     0.07 ) & \textbf{(     0.06 )} & (     0.12 ) & (     0.06 ) & \\
H. Income: 10,001-25,000 Euros & \textbf{     0.38} &      0.29 & \textbf{     0.57} &     -0.09 &      0.32 &      0.29 & 449 &       0.09 \\ 
 & \textbf{(     0.19 )} & (     0.28 ) & \textbf{(     0.24 )} & (     0.18 ) & (     0.38 ) & (     0.19 ) & \\
H. Income: 25,001-50,000 Euros &     -0.15 &     -0.15 &     -0.26 &      0.02 &     -0.17 &     -0.08 & 449 &       0.07 \\ 
 & (     0.21 ) & (     0.31 ) & (     0.26 ) & (     0.20 ) & (     0.42 ) & (     0.21 ) & \\
H. Income: 50,001-100,000 Euros & \textbf{    -0.28} &     -0.15 & \textbf{    -0.39} &     -0.01 &     -0.10 &     -0.16 & 449 &       0.04 \\ 
 & \textbf{(     0.13 )} & (     0.20 ) & \textbf{(     0.17 )} & (     0.13 ) & (     0.27 ) & (     0.14 ) & \\
H. Income: 100,001-250,000 Euros &      0.01 &      0.00 & \textbf{     0.09} & \textbf{    -0.05} &     -0.06 & \textbf{    -0.06} & 449 &       0.09 \\ 
 & (     0.03 ) & (     0.04 ) & \textbf{(     0.03 )} & \textbf{(     0.03 )} & (     0.06 ) & \textbf{(     0.03 )} & \\
H. Income: More than 250,000 Euros &      0.00 &      0.00 &      0.00 &      0.00 &      0.00 &      0.02 & 449 &       0.02 \\ 
 & (     0.02 ) & (     0.03 ) & (     0.03 ) & (     0.02 ) & (     0.04 ) & (     0.02 ) & \\
\bottomrule
\end{tabular}
}
%\end{center}
%\footnotesize
%\fnDID
%\end{table}



%----------------------------------------

\begin{landscape}
\subsection{Household Information - OLS results}

\begin{center}
		\scriptsize{
			\begin{longtable}{L{3cm} c c c c c c p{.5cm} c c c c c c}
				\hline \\
				\multicolumn{14}{L{18.5cm}}{\textbf{Note:} \fnOLS}
				\endfoot
				\caption{Mean outcomes for Household Information} \label{table:OLS_L} \\
				\toprule \\
				\textbf{Outcome} & \multicolumn{6}{c}{\textbf{C. Mean}} & & \multicolumn{6}{c}{\textbf{Mean}} \\
\quad \quad Sample & Muni & State & Reli & Priv & None & $ R^2$ & & Muni & State & Reli & Priv & None & $ R^2$ \\
\quad \quad Restriction & \tiny{$\boldsymbol{\gamma_0}$}& \tiny{$\boldsymbol{\gamma_0+\gamma_1}$}& \tiny{$\boldsymbol{\gamma_0+\gamma_2}$}& \tiny{$\boldsymbol{\gamma_0+\gamma_3}$}& \tiny{$\boldsymbol{\gamma_0+\gamma_4}$} & & & \tiny{$\boldsymbol{\gamma_0}$}& \tiny{$\boldsymbol{\gamma_0+\gamma_1}$}& \tiny{$\boldsymbol{\gamma_0+\gamma_2}$}& \tiny{$\boldsymbol{\gamma_0+\gamma_3}$}& \tiny{$\boldsymbol{\gamma_0+\gamma_4}$} \\
\hline \endhead
~\\*[.05cm]
\textbf{Married or Cohabitating} \\*[.1cm]
\quad \quad \textbf{Adult 30} \\*[.1cm]
\quad \quad \quad Reggio& \highlight{0.17} & \textbf{     0.11} & \textbf{     0.19} & \textbf{     0.65} & 0.17 &      0.06 & & \highlight{0.42} & \textbf{     0.29} & \textbf{     0.40} & \textbf{     1.00} & \textbf{     0.35} &      0.01 \\*
\quad \quad \quad Parma& \textbf{     0.43} & \textbf{     0.46} & \textbf{     0.45} & \textbf{     0.65} & \textbf{     0.37} &      0.03 & & \textbf{     0.37} & 0.41 & \textbf{     0.40} & \textbf{     0.60} & \textbf{     0.34} &      0.01 \\*
\quad \quad \quad Padova& \textbf{     0.19} & \textbf{     0.30} & 0.35 & \textbf{     0.87} & \textbf{     0.26} &      0.11 & & \textbf{     0.29} & \textbf{     0.38} & 0.46 & \textbf{     1.00} & \textbf{     0.38} &      0.02 \\*
\\
\quad \quad \textbf{Adult 40} \\*[.1cm]
\quad \quad \quad Reggio& \highlight{0.86} & \textbf{     0.73} & \textbf{     0.84} & \textbf{     0.92} & 0.87 &      0.02 & & \highlight{0.76} & \textbf{     0.65} & \textbf{     0.73} & \textbf{     0.80} & \textbf{     0.75} &      0.00 \\*
\quad \quad \quad Parma& \textbf{     0.85} & 0.57 & 0.66 & \textbf{     1.10} & 0.57 &      0.08 & & \textbf{     0.77} & 0.58 & \textbf{     0.64} & \textbf{     1.00} & 0.54 &      0.03 \\*
\quad \quad \quad Padova& \textbf{     0.51} & 0.91 & 0.71 & . & \textbf{     0.55} &      0.06 & & \textbf{     0.48} & 0.88 & 0.69 & . & \textbf{     0.53} &      0.06 \\*
\\
\quad \quad \textbf{Adult 50} \\*[.1cm]
\quad \quad \quad Reggio& \highlight{0.54} & \textbf{     0.89} & \textbf{     0.81} & \textbf{     1.02} & \textbf{     0.68} &      0.06 & & \highlight{0.56} & \textbf{     0.90} & \textbf{     0.75} & \textbf{     1.00} & \textbf{     0.61} &      0.03 \\*
\quad \quad \quad Parma& \textbf{     1.14} & \textbf{     1.01} & \textbf{     0.96} & . & 0.83 &      0.14 & & \textbf{     1.00} & \textbf{     0.86} & \textbf{     0.82} & . & 0.67 &      0.07 \\*
\quad \quad \quad Padova& \textbf{     0.79} & \textbf{     0.92} & \textbf{     0.69} & \textbf{     0.99} & \textbf{     0.67} &      0.04 & & \textbf{     0.82} & \textbf{     1.00} & \textbf{     0.69} & \textbf{     1.00} & \textbf{     0.68} &      0.02 \\*
\\
~\\*[.05cm]
\textbf{Num. of Children in House} \\*[.1cm]
\quad \quad \textbf{Adult 30} \\*[.1cm]
\quad \quad \quad Reggio& \highlight{0.16} & \textbf{     0.10} & \textbf{     0.13} & \textbf{     0.05} & \textbf{     0.11} &      0.03 & & \highlight{0.13} & \textbf{     0.10} & \textbf{     0.10} & \textbf{0} & \textbf{     0.11} &      0.00 \\*
\quad \quad \quad Parma& \textbf{     0.32} & \textbf{     0.30} & \textbf{     0.20} & \textbf{     0.49} & 0.16 &      0.06 & & \textbf{     0.19} & \textbf{     0.20} & \textbf{     0.10} & \textbf{     0.40} & 0.05 &      0.02 \\*
\quad \quad \quad Padova& \textbf{     0.32} & \textbf{     0.32} & 0.54 & \textbf{     0.04} & \textbf{     0.36} &      0.08 & & \textbf{     0.14} & \textbf{     0.19} & 0.35 & \textbf{0} & 0.13 &      0.03 \\*
\\
\quad \quad \textbf{Adult 40} \\*[.1cm]
\quad \quad \quad Reggio& \highlight{0.99} & \textbf{     0.81} & \textbf{     1.11} & 1.84 & \textbf{     1.04} &      0.12 & & \highlight{0.59} & \textbf{     0.41} & \textbf{     0.71} & 1.40 & \textbf{     0.54} &      0.04 \\*
\quad \quad \quad Parma& \textbf{     1.19} & 0.98 & 0.90 & \textbf{     2.33} & 0.72 &      0.09 & & \textbf{     0.92} & \textbf{     0.81} & 0.62 & \textbf{     2.00} & 0.47 &      0.07 \\*
\quad \quad \quad Padova& \textbf{     0.96} & \textbf{     1.08} & \textbf{     1.11} & . & 0.60 &      0.09 & & \textbf{     0.70} & \textbf{     0.75} & \textbf{     0.83} & . & 0.29 &      0.07 \\*
\\
\quad \quad \textbf{Adult 50} \\*[.1cm]
\quad \quad \quad Reggio& \highlight{0.32} & \textbf{     0.74} & 1.05 & 2.15 & \textbf{     0.39} &      0.22 & & \highlight{0.11} & 0.60 & 0.89 & 2.00 & \textbf{     0.29} &      0.16 \\*
\quad \quad \quad Parma& \textbf{     1.03} & \textbf{     0.55} & 0.48 & . & 0.51 &      0.09 & & \textbf{     0.92} & \textbf{     0.43} & \textbf{     0.45} & . & 0.40 &      0.06 \\*
\quad \quad \quad Padova& \textbf{     0.63} & 2.17 & \textbf{     0.85} & \textbf{     0.50} & \textbf{     0.69} &      0.11 & & \textbf{     0.82} & 2.50 & \textbf{     1.01} & \textbf{     0.50} & \textbf{     0.88} &      0.05 \\*
\\
~\\*[.05cm]
\textbf{Own House} \\*[.1cm]
\quad \quad \textbf{Adult 30} \\*[.1cm]
\quad \quad \quad Reggio& \highlight{0.60} & \textbf{     0.72} & \textbf{     0.47} & \textbf{     0.10} & \textbf{     0.53} &      0.08 & & \highlight{0.58} & 0.74 & \textbf{     0.45} & \textbf{0} & 0.58 &      0.03 \\*
\quad \quad \quad Parma& \textbf{     0.79} & \textbf{     0.67} & \textbf{     0.68} & \textbf{     0.71} & \textbf{     0.72} &      0.02 & & \textbf{     0.67} & \textbf{     0.55} & \textbf{     0.54} & 0.60 & \textbf{     0.59} &      0.01 \\*
\quad \quad \quad Padova& \textbf{     0.72} & \textbf{     0.84} & \textbf{     0.80} & \textbf{     1.07} & \textbf{     0.70} &      0.02 & & \textbf{     0.69} & \textbf{     0.81} & \textbf{     0.75} & \textbf{     1.00} & \textbf{     0.66} &      0.01 \\*
\\
\quad \quad \textbf{Adult 40} \\*[.1cm]
\quad \quad \quad Reggio& \highlight{0.64} & \textbf{     0.72} & \textbf{     0.67} & \textbf{     0.97} & \textbf{     0.68} &      0.06 & & \highlight{0.66} & \textbf{     0.76} & \textbf{     0.69} & \textbf{     1.00} & \textbf{     0.75} &      0.02 \\*
\quad \quad \quad Parma& \textbf{     0.94} & 0.75 & \textbf{     0.86} & \textbf{     1.05} & 0.78 &      0.03 & & \textbf{     0.88} & \textbf{     0.73} & \textbf{     0.82} & \textbf{     1.00} & 0.74 &      0.02 \\*
\quad \quad \quad Padova& \textbf{     0.96} & \textbf{     1.06} & \textbf{     0.92} & . & \textbf{     0.84} &      0.05 & & \textbf{     0.89} & \textbf{     0.96} & \textbf{     0.86} & . & \textbf{     0.77} &      0.02 \\*
\\
\quad \quad \textbf{Adult 50} \\*[.1cm]
\quad \quad \quad Reggio& \highlight{0.32} & 0.75 & 0.86 & 1.03 & 0.87 &      0.22 & & \highlight{0.22} & 0.70 & 0.86 & 1.00 & 0.89 &      0.16 \\*
\quad \quad \quad Parma& \textbf{     0.89} & \textbf{     0.89} & \textbf{     0.69} & . & 0.68 &      0.07 & & \textbf{     1.00} & \textbf{     1.00} & \textbf{     0.82} & . & \textbf{     0.83} &      0.04 \\*
\quad \quad \quad Padova& \textbf{     0.87} & \textbf{     0.55} & \textbf{     0.87} & \textbf{     1.01} & \textbf{     0.85} &      0.03 & & \textbf{     0.82} & \textbf{     0.50} & \textbf{     0.82} & \textbf{     1.00} & \textbf{     0.81} &      0.01 \\*
\\
~\\*[.05cm]
\textbf{Live With Parents} \\*[.1cm]
\quad \quad \textbf{Adult 30} \\*[.1cm]
\quad \quad \quad Reggio& \highlight{0.22} & \textbf{     0.25} & 0.32 & \textbf{    -0.02} & 0.35 &      0.16 & & \highlight{\textbf{     0.11}} & \textbf{     0.10} & 0.23 & \textbf{0} & \textbf{     0.18} &      0.02 \\*
\quad \quad \quad Parma& \textbf{     0.38} & \textbf{     0.30} & \textbf{     0.29} & \textbf{     0.09} & \textbf{     0.28} &      0.05 & & \textbf{     0.30} & \textbf{     0.20} & 0.16 & \textbf{0} & 0.16 &      0.03 \\*
\quad \quad \quad Padova& \textbf{     0.75} & 0.42 & 0.53 & \textbf{     0.27} & \textbf{     0.60} &      0.14 & & \textbf{     0.51} & 0.23 & 0.31 & \textbf{0} & 0.34 &      0.03 \\*
\\
\quad \quad \textbf{Adult 40} \\*[.1cm]
\quad \quad \quad Reggio& \highlight{0.01} & \textbf{    -0.04} & \textbf{     0.03} & \textbf{    -0.02} & \textbf{     0.04} &      0.05 & & \highlight{\textbf{     0.02}} & \textbf{0} & \textbf{     0.06} & \textbf{0} & \textbf{     0.05} &      0.01 \\*
\quad \quad \quad Parma& \textbf{     0.24} & \textbf{     0.23} & \textbf{     0.25} & \textbf{     0.15} & \textbf{     0.21} &      0.06 & & \textbf{     0.10} & \textbf{     0.12} & \textbf{     0.09} & 0 & \textbf{     0.06} &      0.01 \\*
\quad \quad \quad Padova& \textbf{     0.22} & 0 & \textbf{     0.17} & . & 0.10 &      0.08 & & \textbf{     0.22} & 0.04 & \textbf{     0.15} & . & 0.07 &      0.03 \\*
\\
\quad \quad \textbf{Adult 50} \\*[.1cm]
\quad \quad \quad Reggio& \highlight{0.01} & 0.01 & \textbf{     0.08} & 0.01 & \textbf{     0.03} &      0.05 & & \highlight{0} & 0 & \textbf{     0.07} & 0 & \textbf{     0.01} &      0.04 \\*
\quad \quad \quad Parma& \textbf{    -0.04} & 0.11 & \textbf{    -0.03} & . & \textbf{    -0.04} &      0.20 & & 0 & 0.14 & 0 & . & 0 &      0.13 \\*
\quad \quad \quad Padova& \textbf{     0.03} & 0.03 & \textbf{     0.11} & 0.01 & \textbf{     0.06} &      0.03 & & 0 & 0 & \textbf{     0.09} & 0 & \textbf{     0.04} &      0.02 \\*
\\

			\end{longtable}
		}		
\end{center}
\end{landscape}

%--------------

\subsection{Household Information - Difference-in-Difference Results}
\begin{table}[H]
\begin{center}
	\caption{Difference-in-Difference Across School Types and Cities, Restricting to Age-30 Cohort} \label{table:LCh-30}
	\scalebox{0.80}{
		\begin{tabular}{lcccccccc}
\toprule
 \textbf{Outcome} & \textbf{(1)} & \textbf{(2)} & \textbf{(3)} & \textbf{(4)} & \textbf{(5)} & \textbf{(6)} & \textbf{N} & \textbf{$ R^2$} \\
\midrule
Married or Cohabitating &     -0.02 &      0.11 &     -0.04 &      0.12 &      0.19 &      0.16 & 782 &       0.04 \\ 
 & (     0.12 ) & (     0.13 ) & (     0.12 ) & (     0.13 ) & (     0.16 ) & (     0.13 ) & \\
Num. of Children in House &     -0.13 &      0.04 &     -0.09 &      0.05 &      0.09 & \textbf{     0.24} & 782 &       0.06 \\ 
 & (     0.11 ) & (     0.12 ) & (     0.12 ) & (     0.13 ) & (     0.15 ) & \textbf{(     0.12 )} & \\
Own House &     -0.05 & \textbf{    -0.25} &      0.06 &      0.01 &     -0.03 & \textbf{     0.21} & 782 &       0.05 \\ 
 & (     0.11 ) & \textbf{(     0.12 )} & (     0.12 ) & (     0.13 ) & (     0.15 ) & \textbf{(     0.12 )} & \\
Live With Parents & \textbf{    -0.24} &     -0.09 & \textbf{    -0.21} & \textbf{    -0.26} & \textbf{    -0.30} & \textbf{    -0.30} & 782 &       0.09 \\ 
 & \textbf{(     0.10 )} & (     0.11 ) & \textbf{(     0.10 )} & \textbf{(     0.11 )} & \textbf{(     0.13 )} & \textbf{(     0.10 )} & \\
\bottomrule
\end{tabular}
}
\end{center}
\footnotesize
\fnDID
\end{table}

\begin{table}[H]
\begin{center}
	\caption{Difference-in-Difference Across School Types and Cities, Restricting to Age-40 Cohort} \label{table:LCh-40}
	\scalebox{0.80}{
		\begin{tabular}{lcccccccc}
\toprule
 \textbf{Outcome} & \textbf{(1)} & \textbf{(2)} & \textbf{(3)} & \textbf{(4)} & \textbf{(5)} & \textbf{(6)} & \textbf{N} & \textbf{$ R^2$} \\
\midrule
Married or Cohabitating & \textbf{    -0.21} &     -0.08 &     -0.09 &      0.10 & \textbf{     0.52} & \textbf{     0.27} & 791 &       0.05 \\ 
 & \textbf{(     0.10 )} & (     0.16 ) & (     0.12 ) & (     0.12 ) & \textbf{(     0.17 )} & \textbf{(     0.12 )} & \\
Num. of Children in House & \textbf{    -0.44} &      0.00 & \textbf{    -0.40} &     -0.25 &      0.31 &      0.10 & 791 &       0.08 \\ 
 & \textbf{(     0.16 )} & (     0.26 ) & \textbf{(     0.19 )} & (     0.19 ) & (     0.28 ) & (     0.20 ) & \\
Own House & \textbf{    -0.19} &     -0.23 &     -0.08 & \textbf{    -0.18} &      0.01 &     -0.05 & 791 &       0.05 \\ 
 & \textbf{(     0.09 )} & (     0.14 ) & (     0.10 ) & \textbf{(     0.11 )} & (     0.15 ) & (     0.11 ) & \\
Live With Parents & \textbf{    -0.10} &      0.01 &     -0.05 & \textbf{    -0.14} &     -0.14 &     -0.06 & 791 &       0.05 \\ 
 & \textbf{(     0.06 )} & (     0.09 ) & (     0.07 ) & \textbf{(     0.07 )} & (     0.10 ) & (     0.07 ) & \\
\bottomrule
\end{tabular}
}
\end{center}
\footnotesize
\fnDID
\end{table}
%
%\begin{table}[H]
%\begin{center}
%	\caption{Difference-in-Difference Across School Types and Cities, Restricting to Age-50 Cohort} \label{table:LCh-50}
%	\scalebox{0.80}{
%		\begin{tabular}{lcccccccc}
\toprule
 \textbf{Outcome} & \textbf{(1)} & \textbf{(2)} & \textbf{(3)} & \textbf{(4)} & \textbf{(5)} & \textbf{(6)} & \textbf{N} & \textbf{$ R^2$} \\
\midrule
Married or Cohabitating & \textbf{    -0.43} & \textbf{    -0.52} & \textbf{    -0.41} &     -0.22 &     -0.21 & \textbf{    -0.34} & 449 &       0.05 \\ 
 & \textbf{(     0.20 )} & \textbf{(     0.29 )} & \textbf{(     0.25 )} & (     0.19 ) & (     0.40 ) & \textbf{(     0.20 )} & \\
Num. of Children in House &     -0.47 & \textbf{    -0.79} & \textbf{    -1.16} &     -0.17 & \textbf{     1.08} & \textbf{    -0.70} & 449 &       0.20 \\ 
 & (     0.32 ) & \textbf{(     0.47 )} & \textbf{(     0.40 )} & (     0.30 ) & \textbf{(     0.64 )} & \textbf{(     0.32 )} & \\
Own House & \textbf{    -0.56} &     -0.22 & \textbf{    -0.54} & \textbf{    -0.43} & \textbf{    -0.54} & \textbf{    -0.39} & 449 &       0.06 \\ 
 & \textbf{(     0.15 )} & (     0.23 ) & \textbf{(     0.19 )} & \textbf{(     0.14 )} & \textbf{(     0.31 )} & \textbf{(     0.16 )} & \\
Live With Parents &     -0.01 &      0.14 &     -0.07 &     -0.03 &     -0.06 &     -0.04 & 449 &       0.05 \\ 
 & (     0.07 ) & (     0.11 ) & (     0.09 ) & (     0.07 ) & (     0.14 ) & (     0.07 ) & \\
\bottomrule
\end{tabular}
}
%\end{center}
%\footnotesize
%\fnDID
%\end{table}



%----------------------------------------

\begin{landscape}
\subsection{Health and risk taking behavior - OLS results}

\begin{center}
		\scriptsize{
			\begin{longtable}{L{3cm} c c c c c c p{.5cm} c c c c c c}
				\hline \\
				\multicolumn{14}{L{18.5cm}}{\textbf{Note:} \fnOLS}
				\endfoot
				\caption{Mean outcomes for Health and risk taking behavior}  \label{table:OLS_H} \\
				\toprule \\
				\textbf{Outcome} & \multicolumn{6}{c}{\textbf{C. Mean}} & & \multicolumn{6}{c}{\textbf{Mean}} \\
\quad \quad Sample & Muni & State & Reli & Priv & None & $ R^2$ & & Muni & State & Reli & Priv & None & $ R^2$ \\
\quad \quad Restriction & \tiny{$\boldsymbol{\gamma_0}$}& \tiny{$\boldsymbol{\gamma_0+\gamma_1}$}& \tiny{$\boldsymbol{\gamma_0+\gamma_2}$}& \tiny{$\boldsymbol{\gamma_0+\gamma_3}$}& \tiny{$\boldsymbol{\gamma_0+\gamma_4}$} & & & \tiny{$\boldsymbol{\gamma_0}$}& \tiny{$\boldsymbol{\gamma_0+\gamma_1}$}& \tiny{$\boldsymbol{\gamma_0+\gamma_2}$}& \tiny{$\boldsymbol{\gamma_0+\gamma_3}$}& \tiny{$\boldsymbol{\gamma_0+\gamma_4}$} \\
\hline \endhead
~\\*[.05cm]
\textbf{Tried Marijuana} \\*[.1cm]
\quad \quad \textbf{Adult 30} & & & & & & & & \multicolumn{6}{c}{\highlight{Reference mean = \textbf{     0.21}}} \\*[.1cm]
\quad \quad \quad Reggio& 0 & \textbf{     0.06} & \textbf{     0.09} & \textbf{    -0.12} & -0.12 &      0.10 & & 0.21 & \textbf{     0.26} & \textbf{     0.28} & \textbf{0} & \textbf{     0.11} & &      0.02 \\*
\quad \quad \quad Parma& \textbf{     0.25} & \textbf{     0.16} & \textbf{     0.32} & \textbf{     0.44} & \textbf{     0.15} &      0.17 & & \textbf{     0.18} & \textbf{     0.12} & \textbf{     0.28} & \textbf{     0.40} & 0.05 & &      0.05 \\*
\quad \quad \quad Padova& \textbf{     0.36} & \textbf{     0.29} & 0.20 & 0.02 & 0.06 &      0.09 & & \textbf{     0.34} & \textbf{     0.27} & 0.20 & \textbf{0} & 0.04 & &      0.05 \\*
\\
\quad \quad \textbf{Adult 40} & & & & & & & & \multicolumn{6}{c}{\highlight{Reference mean = \textbf{     0.12}}} \\*[.1cm]
\quad \quad \quad Reggio& 0.01 & \textbf{    -0.07} & \textbf{    -0.03} & \textbf{    -0.11} & \textbf{    -0.03} &      0.03 & & \textbf{     0.12} & \textbf{     0.18} & \textbf{     0.08} & \textbf{0} & \textbf{     0.09} & &      0.01 \\*
\quad \quad \quad Parma& \textbf{     0.09} & \textbf{     0.12} & 0.02 & 0.90 & -0.02 &      0.13 & & \textbf{     0.13} & \textbf{     0.15} & \textbf{     0.07} & 1.00 & 0.01 & &      0.11 \\*
\quad \quad \quad Padova& \textbf{     0.14} & -0.01 & \textbf{     0.07} & . & 0.03 &      0.04 & & \textbf{     0.15} & 0 & \textbf{     0.08} & . & 0.04 & &      0.02 \\*
\\
\quad \quad \textbf{Adult 50} & & & & & & & & \multicolumn{6}{c}{\highlight{Reference mean = \textbf{     0.11}}} \\*[.1cm]
\quad \quad \quad Reggio& 0.14 & \textbf{     0.13} & \textbf{     0.11} & \textbf{     0.04} & \textbf{     0.06} &      0.04 & & 0.11 & \textbf{     0.10} & \textbf{     0.07} & \textbf{0} & \textbf{     0.02} & &      0.02 \\*
\quad \quad \quad Parma& \textbf{     0.04} & \textbf{     0.06} & \textbf{     0.16} & . & \textbf{     0.11} &      0.08 & & \textbf{0} & \textbf{0} & \textbf{     0.09} & . & \textbf{     0.03} & &      0.02 \\*
\quad \quad \quad Padova& \textbf{    -0.05} & \textbf{    -0.05} & \textbf{     0.04} & \textbf{0} & \textbf{0} &      0.06 & & \textbf{0} & \textbf{0} & \textbf{     0.09} & \textbf{0} & \textbf{     0.05} & &      0.01 \\*
\\
~\\*[.05cm]
\textbf{Smokes} \\*[.1cm]
\quad \quad \textbf{Adult 30} & & & & & & & & \multicolumn{6}{c}{\highlight{Reference mean = \textbf{     0.21}}} \\*[.1cm]
\quad \quad \quad Reggio& 0.29 & \textbf{     0.34} & 0.28 & . & 0.30 &      0.04 & & 0.21 & \textbf{     0.23} & \textbf{     0.19} & . & \textbf{     0.16} & &      0.00 \\*
\quad \quad \quad Parma& \textbf{     0.41} & \textbf{     0.57} & \textbf{     0.41} & \textbf{     0.44} & \textbf{     0.32} &      0.07 & & \textbf{     0.32} & \textbf{     0.50} & \textbf{     0.31} & \textbf{     0.33} & \textbf{     0.25} & &      0.03 \\*
\quad \quad \quad Padova& \textbf{     0.65} & \textbf{     0.64} & \textbf{     0.66} & \textbf{    -0.07} & \textbf{     0.57} &      0.13 & & \textbf{     0.52} & \textbf{     0.56} & \textbf{     0.51} & \textbf{0} & \textbf{     0.38} & &      0.02 \\*
\\
\quad \quad \textbf{Adult 40} & & & & & & & & \multicolumn{6}{c}{\highlight{Reference mean = \textbf{     0.29}}} \\*[.1cm]
\quad \quad \quad Reggio& 0.37 & \textbf{     0.29} & \textbf{     0.30} & \textbf{     0.10} & \textbf{     0.25} &      0.05 & & 0.29 & 0.29 & \textbf{     0.21} & \textbf{0} & \textbf{     0.18} & &      0.02 \\*
\quad \quad \quad Parma& \textbf{     0.63} & \textbf{     0.64} & 0.39 & \textbf{     0.97} & \textbf{     0.46} &      0.13 & & \textbf{     0.55} & \textbf{     0.53} & 0.21 & \textbf{     1.00} & 0.29 & &      0.09 \\*
\quad \quad \quad Padova& \textbf{     0.44} & \textbf{     0.32} & \textbf{     0.57} & . & \textbf{     0.34} &      0.11 & & \textbf{     0.45} & \textbf{     0.40} & \textbf{     0.62} & . & \textbf{     0.40} & &      0.05 \\*
\\
\quad \quad \textbf{Adult 50} & & & & & & & & \multicolumn{6}{c}{\highlight{Reference mean = \textbf{     0.75}}} \\*[.1cm]
\quad \quad \quad Reggio& 0.82 & \textbf{     0.36} & \textbf{     0.49} & . & \textbf{     0.43} &      0.08 & & 0.75 & \textbf{     0.25} & \textbf{     0.36} & . & 0.32 & &      0.04 \\*
\quad \quad \quad Parma& \textbf{     1.17} & \textbf{     0.63} & \textbf{     0.74} & . & 0.62 &      0.29 & & 0.67 & \textbf{     0.50} & \textbf{     0.50} & . & \textbf{     0.26} & &      0.08 \\*
\quad \quad \quad Padova& \textbf{     1.11} & \textbf{     1.54} & 0.86 & 0.04 & \textbf{     0.75} &      0.15 & & 0.80 & \textbf{     1.00} & 0.71 & \textbf{0} & \textbf{     0.50} & &      0.08 \\*
\\
~\\*[.05cm]
\textbf{Num. of Cigarettes Per Day} \\*[.1cm]
\quad \quad \textbf{Adult 30} & & & & & & & & \multicolumn{6}{c}{\highlight{Reference mean = \textbf{    16.47}}} \\*[.1cm]
\quad \quad \quad Reggio& 18.10 & \textbf{    16.19} & \textbf{    17.30} & . & 15.87 &      0.05 & & 16.47 & \textbf{    15.00} & \textbf{    16.14} & . & \textbf{    15.00} & &      0.02 \\*
\quad \quad \quad Parma& \textbf{    13.03} & \textbf{    15.40} & \textbf{    11.04} & \textbf{     9.44} & \textbf{    12.34} &      0.16 & & \textbf{    12.34} & \textbf{    15.00} & \textbf{     9.39} & \textbf{    11.00} & \textbf{    11.94} & &      0.06 \\*
\quad \quad \quad Padova& \textbf{    13.89} & \textbf{    11.09} & \textbf{    14.69} & 17.76 & \textbf{    13.33} &      0.14 & & \textbf{    10.22} & \textbf{     8.43} & \textbf{    10.89} & \textbf{    15.00} & \textbf{     9.38} & &      0.04 \\*
\\
\quad \quad \textbf{Adult 40} & & & & & & & & \multicolumn{6}{c}{\highlight{Reference mean = \textbf{    16.65}}} \\*[.1cm]
\quad \quad \quad Reggio& 14.68 & \textbf{    16.89} & 10.76 & \textbf{    12.37} & \textbf{    15.48} &      0.15 & & 16.65 & \textbf{    18.00} & 12.64 & \textbf{    15.00} & \textbf{    17.00} & &      0.09 \\*
\quad \quad \quad Parma& \textbf{    11.27} & \textbf{     9.96} & \textbf{     9.88} & . & \textbf{    11.83} &      0.28 & & \textbf{    13.43} & \textbf{    11.78} & \textbf{    11.18} & . & \textbf{    14.74} & &      0.06 \\*
\quad \quad \quad Padova& \textbf{     8.62} & \textbf{     4.83} & \textbf{    10.35} & . & 12.77 &      0.28 & & \textbf{     8.83} & \textbf{     5.50} & \textbf{    10.91} & . & 13.48 & &      0.21 \\*
\\
\quad \quad \textbf{Adult 50} & & & & & & & & \multicolumn{6}{c}{\highlight{Reference mean = \textbf{    25.00}}} \\*[.1cm]
\quad \quad \quad Reggio& 22.32 & \textbf{    18.02} & \textbf{    13.14} & . & \textbf{    11.89} &      0.13 & & 25.00 & \textbf{    21.67} & \textbf{    17.00} & . & \textbf{    15.31} & &      0.06 \\*
\quad \quad \quad Parma& \textbf{     4.50} & 38.62 & \textbf{    14.14} & . & \textbf{     4.72} &      0.55 & & \textbf{    10.00} & 40.00 & 25.00 & . & \textbf{    17.23} & &      0.26 \\*
\quad \quad \quad Padova& \textbf{    12.05} & . & \textbf{     5.15} & \textbf{     8.10} & \textbf{     4.92} &      0.24 & & \textbf{    13.00} & . & \textbf{     7.60} & \textbf{    10.00} & \textbf{     7.69} & &      0.05 \\*
\\
~\\*[.05cm]
\textbf{BMI} \\*[.1cm]
\quad \quad \textbf{Adult 30} & & & & & & & & \multicolumn{6}{c}{\highlight{Reference mean = \textbf{    23.57}}} \\*[.1cm]
\quad \quad \quad Reggio& 22.13 & \textbf{    21.92} & 22.06 & \textbf{    24.50} & 21.51 &      0.26 & & 23.57 & \textbf{    23.07} & \textbf{    23.47} & \textbf{    26.83} & 22.67 & &      0.04 \\*
\quad \quad \quad Parma& \textbf{    21.12} & 20.21 & 19.68 & \textbf{    19.63} & \textbf{    20.66} &      0.45 & & \textbf{    23.87} & 22.87 & 22.34 & \textbf{    21.97} & \textbf{    23.69} & &      0.05 \\*
\quad \quad \quad Padova& \textbf{    21.78} & \textbf{    21.69} & \textbf{    22.32} & . & \textbf{    22.32} &      0.30 & & \textbf{    22.87} & \textbf{    22.58} & \textbf{    23.62} & . & 23.55 & &      0.02 \\*
\\
\quad \quad \textbf{Adult 40} & & & & & & & & \multicolumn{6}{c}{\highlight{Reference mean = \textbf{    24.10}}} \\*[.1cm]
\quad \quad \quad Reggio& 23.55 & \textbf{    21.90} & \textbf{    23.79} & \textbf{    25.19} & \textbf{    24.24} &      0.16 & & 24.10 & 21.98 & \textbf{    24.38} & \textbf{    25.72} & \textbf{    24.51} & &      0.03 \\*
\quad \quad \quad Parma& \textbf{    22.70} & 24.68 & 23.70 & . & \textbf{    23.13} &      0.20 & & \textbf{    23.39} & 25.05 & \textbf{    23.93} & . & \textbf{    23.93} & &      0.02 \\*
\quad \quad \quad Padova& \textbf{    22.92} & \textbf{    22.72} & \textbf{    22.23} & . & \textbf{    22.51} &      0.26 & & 24.18 & \textbf{    23.98} & \textbf{    23.51} & . & \textbf{    23.87} & &      0.01 \\*
\\
\quad \quad \textbf{Adult 50} & & & & & & & & \multicolumn{6}{c}{\highlight{Reference mean = \textbf{    25.22}}} \\*[.1cm]
\quad \quad \quad Reggio& 22.44 & \textbf{    22.80} & \textbf{    23.87} & \textbf{    22.05} & \textbf{    23.10} &      0.31 & & 25.22 & \textbf{    24.72} & \textbf{    24.90} & \textbf{    23.56} & \textbf{    24.30} & &      0.01 \\*
\quad \quad \quad Parma& \textbf{    22.92} & \textbf{    21.84} & \textbf{    23.84} & . & \textbf{    24.35} &      0.23 & & \textbf{    22.78} & \textbf{    22.00} & \textbf{    23.73} & . & \textbf{    24.23} & &      0.07 \\*
\quad \quad \quad Padova& \textbf{    27.65} & \textbf{    32.56} & \textbf{    25.52} & 22.66 & \textbf{    25.78} &      0.12 & & \textbf{    26.99} & \textbf{    31.14} & \textbf{    24.77} & \textbf{    22.43} & 25.10 & &      0.05 \\*
\\
~\\*[.05cm]
\textbf{Good Health} \\*[.1cm]
\quad \quad \textbf{Adult 30} & & & & & & & & \multicolumn{6}{c}{\highlight{Reference mean = \textbf{     4.26}}} \\*[.1cm]
\quad \quad \quad Reggio& 4.19 & \textbf{     4.22} & \textbf{     4.31} & \textbf{     4.90} & \textbf{     4.06} &      0.14 & & 4.26 & 4.26 & \textbf{     4.38} & \textbf{     5.00} & 4.04 & &      0.05 \\*
\quad \quad \quad Parma& \textbf{     3.26} & \textbf{     3.17} & 3.45 & 2.72 & \textbf{     3.35} &      0.21 & & \textbf{     3.78} & \textbf{     3.69} & 4.08 & 3.20 & 3.98 & &      0.07 \\*
\quad \quad \quad Padova& \textbf{     3.54} & 3.84 & \textbf{     3.66} & . & 3.82 &      0.04 & & \textbf{     3.64} & 3.92 & \textbf{     3.77} & . & 3.96 & &      0.03 \\*
\\
\quad \quad \textbf{Adult 40} & & & & & & & & \multicolumn{6}{c}{\highlight{Reference mean = \textbf{     3.91}}} \\*[.1cm]
\quad \quad \quad Reggio& 3.65 & \textbf{     3.54} & \textbf{     3.79} & \textbf{     3.54} & 3.50 &      0.08 & & 3.91 & \textbf{     4.00} & \textbf{     4.06} & \textbf{     3.80} & \textbf{     3.78} & &      0.03 \\*
\quad \quad \quad Parma& \textbf{     2.98} & \textbf{     2.86} & \textbf{     3.11} & \textbf{     3.20} & 3.17 &      0.19 & & \textbf{     3.46} & 3.19 & \textbf{     3.65} & \textbf{     4.00} & \textbf{     3.59} & &      0.04 \\*
\quad \quad \quad Padova& \textbf{     3.74} & \textbf{     3.84} & 3.35 & . & 3.29 &      0.08 & & \textbf{     3.85} & \textbf{     4.00} & 3.47 & . & 3.41 & &      0.07 \\*
\\
\quad \quad \textbf{Adult 50} & & & & & & & & \multicolumn{6}{c}{\highlight{Reference mean = \textbf{     3.11}}} \\*[.1cm]
\quad \quad \quad Reggio& 2.85 & \textbf{     3.11} & \textbf{     2.93} & 4.30 & \textbf{     3.09} &      0.08 & & 3.11 & \textbf{     3.30} & 3.14 & 4.50 & \textbf{     3.30} & &      0.05 \\*
\quad \quad \quad Parma& \textbf{     3.09} & \textbf{     2.75} & \textbf{     3.29} & . & \textbf{     3.09} &      0.14 & & \textbf{     3.00} & \textbf{     2.71} & \textbf{     3.36} & . & 3.10 & &      0.05 \\*
\quad \quad \quad Padova& \textbf{     2.62} & 2.82 & \textbf{     3.07} & \textbf{     2.95} & \textbf{     2.92} &      0.05 & & \textbf{     2.82} & \textbf{     3.00} & \textbf{     3.24} & \textbf{     3.00} & 3.09 & &      0.02 \\*
\\
~\\*[.05cm]
\textbf{Num. of Days Sick Past Month} \\*[.1cm]
\quad \quad \textbf{Adult 30} & & & & & & & & \multicolumn{6}{c}{\highlight{Reference mean = \textbf{     1.41}}} \\*[.1cm]
\quad \quad \quad Reggio& 1.34 & 1.11 & \textbf{     1.21} & \textbf{     0.99} & 1.00 &      0.08 & & 1.41 & 1.17 & \textbf{     1.25} & \textbf{     1.00} & 1.15 & &      0.04 \\*
\quad \quad \quad Parma& 1.35 & \textbf{     1.25} & 1.35 & 2.14 & \textbf{     1.25} &      0.12 & & \textbf{     1.16} & \textbf{     1.08} & \textbf{     1.18} & 2.00 & 1.00 & &      0.08 \\*
\quad \quad \quad Padova& \textbf{     1.38} & 1.35 & \textbf{     1.31} & . & \textbf{     1.16} &      0.04 & & \textbf{     1.24} & \textbf{     1.25} & \textbf{     1.18} & . & 1.00 & &      0.02 \\*
\\
\quad \quad \textbf{Adult 40} & & & & & & & & \multicolumn{6}{c}{\highlight{Reference mean = \textbf{     1.14}}} \\*[.1cm]
\quad \quad \quad Reggio& 1.17 & \textbf{     1.04} & \textbf{     1.09} & \textbf{     1.04} & \textbf{     1.13} &      0.02 & & 1.14 & \textbf{     1.00} & \textbf{     1.06} & \textbf{     1.00} & \textbf{     1.10} & &      0.01 \\*
\quad \quad \quad Parma& \textbf{     1.27} & \textbf{     1.34} & \textbf{     1.26} & \textbf{     1.04} & \textbf{     1.19} &      0.03 & & \textbf{     1.16} & \textbf{     1.27} & \textbf{     1.15} & \textbf{     1.00} & \textbf{     1.06} & &      0.02 \\*
\quad \quad \quad Padova& \textbf{     1.13} & 2.02 & \textbf{     1.12} & . & \textbf{     1.13} &      0.20 & & 1.15 & 2.00 & \textbf{     1.11} & . & \textbf{     1.11} & &      0.17 \\*
\\
\quad \quad \textbf{Adult 50} & & & & & & & & \multicolumn{6}{c}{\highlight{Reference mean = \textbf{     1.38}}} \\*[.1cm]
\quad \quad \quad Reggio& 1.34 & \textbf{     1.22} & 1.36 & \textbf{     1.13} & \textbf{     1.54} &      0.09 & & 1.37 & \textbf{     1.20} & \textbf{     1.22} & \textbf{     1.00} & \textbf{     1.32} & &      0.01 \\*
\quad \quad \quad Parma& \textbf{     1.02} & \textbf{     1.21} & \textbf{     1.22} & . & 1.27 &      0.08 & & \textbf{     1.00} & \textbf{     1.17} & \textbf{     1.09} & . & \textbf{     1.17} & &      0.02 \\*
\quad \quad \quad Padova& \textbf{     1.90} & \textbf{     1.20} & 1.39 & \textbf{     1.01} & \textbf{     1.49} &      0.04 & & \textbf{     1.73} & \textbf{     1.00} & 1.25 & \textbf{     1.00} & \textbf{     1.33} & &      0.02 \\*
\\
~\\*[.05cm]
\textbf{Engaged in A Fight} \\*[.1cm]
\quad \quad \textbf{Adult 30} & & & & & & & & \multicolumn{6}{c}{\highlight{Reference mean = \textbf{     0.00}}} \\*[.1cm]
\quad \quad \quad Reggio& 0 & 0 & 0 & 0 & 0 &         . & & 0 & 0 & 0 & 0 & 0 & &         . \\*
\quad \quad \quad Parma& 0 & 0 & 0 & 0 & 0 &         . & & 0 & 0 & 0 & 0 & 0 & &         . \\*
\quad \quad \quad Padova& 0 & 0 & 0 & 0 & 0 &         . & & 0 & 0 & 0 & 0 & 0 & &         . \\*
\\
\quad \quad \textbf{Adult 40} & & & & & & & & \multicolumn{6}{c}{\highlight{Reference mean = \textbf{     0.00}}} \\*[.1cm]
\quad \quad \quad Reggio& 0 & 0 & 0 & 0 & 0 &         . & & 0 & 0 & 0 & 0 & 0 & &         . \\*
\quad \quad \quad Parma& 0 & 0 & 0 & 0 & 0 &         . & & 0 & 0 & 0 & 0 & 0 & &         . \\*
\quad \quad \quad Padova& 0 & 0 & 0 & . & 0 &         . & & 0 & 0 & 0 & . & 0 & &         . \\*
\\
\quad \quad \textbf{Adult 50} & & & & & & & & \multicolumn{6}{c}{\highlight{Reference mean = \textbf{     0.00}}} \\*[.1cm]
\quad \quad \quad Reggio& 0 & 0 & 0 & 0 & 0 &         . & & 0 & 0 & 0 & 0 & 0 & &         . \\*
\quad \quad \quad Parma& 0 & 0 & 0 & . & 0 &         . & & 0 & 0 & 0 & . & 0 & &         . \\*
\quad \quad \quad Padova& 0 & 0 & 0 & 0 & 0 &         . & & 0 & 0 & 0 & 0 & 0 & &         . \\*
\\
~\\*[.05cm]
\textbf{Drove Under Influence} \\*[.1cm]
\quad \quad \textbf{Adult 30} & & & & & & & & \multicolumn{6}{c}{\highlight{Reference mean = \textbf{     0.00}}} \\*[.1cm]
\quad \quad \quad Reggio& 0 & 0 & 0 & 0 & 0 &         . & & 0 & 0 & 0 & 0 & 0 & &         . \\*
\quad \quad \quad Parma& 0 & 0 & 0 & 0 & 0 &         . & & 0 & 0 & 0 & 0 & 0 & &         . \\*
\quad \quad \quad Padova& 0 & 0 & 0 & 0 & 0 &         . & & 0 & 0 & 0 & 0 & 0 & &         . \\*
\\
\quad \quad \textbf{Adult 40} & & & & & & & & \multicolumn{6}{c}{\highlight{Reference mean = \textbf{     0.00}}} \\*[.1cm]
\quad \quad \quad Reggio& 0 & 0 & 0 & 0 & 0 &         . & & 0 & 0 & 0 & 0 & 0 & &         . \\*
\quad \quad \quad Parma& 0 & 0 & 0 & 0 & 0 &         . & & 0 & 0 & 0 & 0 & 0 & &         . \\*
\quad \quad \quad Padova& 0 & 0 & 0 & . & 0 &         . & & 0 & 0 & 0 & . & 0 & &         . \\*
\\
\quad \quad \textbf{Adult 50} & & & & & & & & \multicolumn{6}{c}{\highlight{Reference mean = \textbf{     0.00}}} \\*[.1cm]
\quad \quad \quad Reggio& 0 & 0 & 0 & 0 & 0 &         . & & 0 & 0 & 0 & 0 & 0 & &         . \\*
\quad \quad \quad Parma& 0 & 0 & 0 & . & 0 &         . & & 0 & 0 & 0 & . & 0 & &         . \\*
\quad \quad \quad Padova& 0 & 0 & 0 & 0 & 0 &         . & & 0 & 0 & 0 & 0 & 0 & &         . \\*
\\
~\\*[.05cm]
\textbf{Ever Suspended from School} \\*[.1cm]
\quad \quad \textbf{Adult 30} & & & & & & & & \multicolumn{6}{c}{\highlight{Reference mean = \textbf{     0.05}}} \\*[.1cm]
\quad \quad \quad Reggio& -0.05 & \textbf{    -0.02} & \textbf{    -0.02} & \textbf{    -0.11} & 0.08 &      0.08 & & 0.05 & \textbf{     0.06} & \textbf{     0.07} & \textbf{0} & 0.16 & &      0.03 \\*
\quad \quad \quad Parma& \textbf{     0.02} & \textbf{     0.06} & \textbf{     0.01} & \textbf{    -0.03} & \textbf{     0.01} &      0.06 & & \textbf{     0.06} & \textbf{     0.10} & \textbf{     0.04} & \textbf{0} & 0.05 & &      0.01 \\*
\quad \quad \quad Padova& \textbf{     0.10} & 0 & 0.03 & \textbf{     0.02} & \textbf{     0.07} &      0.04 & & \textbf{     0.11} & 0 & 0.04 & \textbf{0} & \textbf{     0.09} & &      0.02 \\*
\\
\quad \quad \textbf{Adult 40} & & & & & & & & \multicolumn{6}{c}{\highlight{Reference mean = \textbf{     0.05}}} \\*[.1cm]
\quad \quad \quad Reggio& -0.05 & \textbf{    -0.04} & \textbf{    -0.03} & \textbf{    -0.10} & \textbf{    -0.03} &      0.06 & & \textbf{     0.05} & \textbf{     0.06} & \textbf{     0.06} & \textbf{0} & \textbf{     0.09} & &      0.01 \\*
\quad \quad \quad Parma& \textbf{     0.01} & -0.05 & -0.06 & \textbf{    -0.08} & \textbf{    -0.01} &      0.04 & & \textbf{     0.08} & \textbf{0} & \textbf{     0.02} & \textbf{0} & \textbf{     0.06} & &      0.02 \\*
\quad \quad \quad Padova& \textbf{    -0.02} & \textbf{    -0.09} & \textbf{    -0.01} & . & \textbf{    -0.03} &      0.05 & & \textbf{     0.04} & \textbf{0} & \textbf{     0.07} & . & 0.05 & &      0.01 \\*
\\
\quad \quad \textbf{Adult 50} & & & & & & & & \multicolumn{6}{c}{\highlight{Reference mean = \textbf{     0.00}}} \\*[.1cm]
\quad \quad \quad Reggio& -0.01 & \textbf{     0.08} & \textbf{     0.08} & -0.02 & \textbf{     0.01} &      0.02 & & 0 & \textbf{     0.10} & \textbf{     0.11} & 0 & \textbf{     0.04} & &      0.02 \\*
\quad \quad \quad Parma& \textbf{    -0.04} & \textbf{    -0.03} & 0.14 & . & -0.01 &      0.07 & & 0 & 0 & 0.18 & . & \textbf{     0.04} & &      0.05 \\*
\quad \quad \quad Padova& \textbf{     0.08} & \textbf{    -0.28} & -0.01 & -0.04 & -0.08 &      0.12 & & \textbf{     0.27} & 0 & 0.10 & 0 & 0.05 & &      0.04 \\*
\\
~\\*[.05cm]
\textbf{Age At First Drink} \\*[.1cm]
\quad \quad \textbf{Adult 30} & & & & & & & & \multicolumn{6}{c}{\highlight{Reference mean = \textbf{    12.55}}} \\*[.1cm]
\quad \quad \quad Reggio& 8.31 & \textbf{     6.06} & 11.43 & \textbf{    18.32} & \textbf{     8.02} &      0.23 & & 12.55 & 8.60 & 15.18 & \textbf{    25.00} & 10.44 & &      0.06 \\*
\quad \quad \quad Parma& \textbf{    12.58} & \textbf{    13.94} & \textbf{    12.33} & \textbf{    15.52} & 14.99 &      0.07 & & \textbf{    12.73} & \textbf{    14.29} & \textbf{    12.78} & \textbf{    15.80} & 14.73 & &      0.02 \\*
\quad \quad \quad Padova& \textbf{    15.60} & \textbf{    16.85} & \textbf{    13.61} & \textbf{    14.33} & \textbf{    13.82} &      0.05 & & \textbf{    14.88} & \textbf{    16.24} & \textbf{    12.80} & \textbf{    14.00} & \textbf{    12.94} & &      0.03 \\*
\\
\quad \quad \textbf{Adult 40} & & & & & & & & \multicolumn{6}{c}{\highlight{Reference mean = \textbf{    11.37}}} \\*[.1cm]
\quad \quad \quad Reggio& 10.83 & 6.84 & \textbf{    12.34} & \textbf{    10.04} & \textbf{    11.23} &      0.09 & & 11.37 & 7.65 & \textbf{    12.44} & \textbf{    10.75} & \textbf{    10.41} & &      0.02 \\*
\quad \quad \quad Parma& \textbf{    13.27} & \textbf{    13.29} & \textbf{    12.78} & \textbf{    10.15} & \textbf{    12.88} &      0.11 & & \textbf{    15.14} & \textbf{    14.35} & \textbf{    14.09} & \textbf{    15.00} & \textbf{    13.83} & &      0.01 \\*
\quad \quad \quad Padova& \textbf{    13.51} & \textbf{    13.21} & \textbf{    12.48} & . & \textbf{    11.41} &      0.02 & & \textbf{    14.19} & \textbf{    13.83} & \textbf{    13.20} & . & \textbf{    12.08} & &      0.01 \\*
\\
\quad \quad \textbf{Adult 50} & & & & & & & & \multicolumn{6}{c}{\highlight{Reference mean = \textbf{    16.62}}} \\*[.1cm]
\quad \quad \quad Reggio& 14.48 & \textbf{    13.09} & \textbf{    15.60} & 15.07 & 14.46 &      0.09 & & 16.63 & \textbf{    14.20} & \textbf{    15.39} & \textbf{    15.50} & \textbf{    13.83} & &      0.01 \\*
\quad \quad \quad Parma& \textbf{    18.68} & \textbf{    17.33} & 11.56 & . & 14.71 &      0.13 & & 16.83 & 16.14 & 10.91 & . & \textbf{    13.63} & &      0.05 \\*
\quad \quad \quad Padova& \textbf{    11.29} & 14.18 & \textbf{    13.39} & \textbf{    18.62} & \textbf{    12.38} &      0.06 & & \textbf{    12.91} & 17.00 & \textbf{    14.69} & \textbf{    18.50} & \textbf{    14.00} & &      0.01 \\*
\\

			\end{longtable}
		}		
\end{center}
\end{landscape}

%--------------

\subsection{Health and risk taking behavior - Difference-in-Difference Results}

\begin{table}[H]
\begin{center}
	\caption{Difference-in-Difference Across School Types and Cities, Restricting to Age-30 Cohort} \label{table:HCh-30}
	\scalebox{0.80}{
		\begin{tabular}{lcccccccc}
\toprule
 \textbf{Outcome} & \textbf{(1)} & \textbf{(2)} & \textbf{(3)} & \textbf{(4)} & \textbf{(5)} & \textbf{(6)} & \textbf{N} & \textbf{$ R^2$} \\
\midrule
Tried Marijuana &     -0.04 &     -0.14 &      0.00 & \textbf{    -0.25} &     -0.13 & \textbf{    -0.23} & 782 &       0.07 \\ 
 & (     0.09 ) & (     0.10 ) & (     0.10 ) & \textbf{(     0.10 )} & (     0.12 ) & \textbf{(     0.10 )} & \\
Smokes &     -0.05 &      0.16 &     -0.02 &     -0.06 &      0.05 &      0.01 & 424 &       0.11 \\ 
 & (     0.15 ) & (     0.17 ) & (     0.15 ) & (     0.17 ) & (     0.20 ) & (     0.15 ) & \\
Num. of Cigarettes Per Day &      1.58 &      3.86 &     -1.42 &      1.78 &     -0.78 &      2.11 & 275 &       0.24 \\ 
 & (     2.10 ) & (     2.49 ) & (     2.07 ) & (     2.59 ) & (     3.21 ) & (     2.35 ) & \\
BMI &      0.12 &     -0.85 & \textbf{    -1.47} &      0.72 &      0.09 &      0.49 & 620 &       0.32 \\ 
 & (     0.62 ) & (     0.72 ) & \textbf{(     0.65 )} & (     0.71 ) & (     0.84 ) & (     0.64 ) & \\
Good Health & \textbf{     0.36} &     -0.18 &      0.06 & \textbf{     0.40} &      0.20 &     -0.08 & 775 &       0.17 \\ 
 & \textbf{(     0.14 )} & (     0.16 ) & (     0.15 ) & \textbf{(     0.16 )} & (     0.19 ) & (     0.15 ) & \\
Num. of Days Sick Past Month &      0.12 &      0.16 &      0.14 &      0.04 &      0.26 &      0.09 & 746 &       0.07 \\ 
 & (     0.14 ) & (     0.15 ) & (     0.14 ) & (     0.16 ) & (     0.19 ) & (     0.15 ) & \\
Engaged in A Fight &      0.00 &      0.00 &      0.00 &      0.00 &      0.00 &      0.00 & 782 &          . \\ 
 & (        . ) & (        . ) & (        . ) & (        . ) & (        . ) & (        . ) & \\
Drove Under Influence &      0.00 &      0.00 &      0.00 &      0.00 &      0.00 &      0.00 & 782 &          . \\ 
 & (        . ) & (        . ) & (        . ) & (        . ) & (        . ) & (        . ) & \\
Ever Suspended from School & \textbf{    -0.11} &      0.03 &     -0.01 & \textbf{    -0.13} &     -0.11 &     -0.07 & 782 &       0.05 \\ 
 & \textbf{(     0.06 )} & (     0.06 ) & (     0.06 ) & \textbf{(     0.07 )} & (     0.08 ) & (     0.06 ) & \\
Age At First Drink & \textbf{     2.99} & \textbf{     4.51} & \textbf{    -3.46} &     -1.62 & \textbf{     4.44} & \textbf{    -5.45} & 759 &       0.09 \\ 
 & \textbf{(     1.67 )} & \textbf{(     1.79 )} & \textbf{(     1.71 )} & (     1.88 ) & \textbf{(     2.23 )} & \textbf{(     1.77 )} & \\
\bottomrule
\end{tabular}
}
\end{center}
\footnotesize
\fnDID
\end{table}

\begin{table}[H]
\begin{center}
	\caption{Difference-in-Difference Across School Types and Cities, Restricting to Age-40 Cohort} \label{table:HCh-40}
	\scalebox{0.80}{
		\begin{tabular}{lcccccccc}
\toprule
 \textbf{Outcome} & \textbf{(1)} & \textbf{(2)} & \textbf{(3)} & \textbf{(4)} & \textbf{(5)} & \textbf{(6)} & \textbf{N} & \textbf{$ R^2$} \\
\midrule
Tried Marijuana &     -0.09 &     -0.02 &     -0.03 &     -0.07 & \textbf{    -0.20} &     -0.01 & 791 &       0.05 \\ 
 & (     0.06 ) & (     0.10 ) & (     0.07 ) & (     0.07 ) & \textbf{(     0.10 )} & (     0.07 ) & \\
Smokes &     -0.11 &     -0.00 & \textbf{    -0.26} &      0.07 &     -0.03 &      0.23 & 406 &       0.11 \\ 
 & (     0.14 ) & (     0.23 ) & \textbf{(     0.16 )} & (     0.17 ) & (     0.27 ) & (     0.17 ) & \\
Num. of Cigarettes Per Day &     -1.21 &     -4.71 &      1.87 &      2.28 & \textbf{    -6.56} & \textbf{     5.32} & 254 &       0.27 \\ 
 & (     2.12 ) & (     3.50 ) & (     2.33 ) & (     2.59 ) & \textbf{(     3.94 )} & \textbf{(     2.68 )} & \\
BMI &     -0.07 & \textbf{     3.56} &      0.74 &     -0.87 &      1.48 &     -0.82 & 612 &       0.19 \\ 
 & (     0.67 ) & \textbf{(     1.14 )} & (     0.74 ) & (     0.83 ) & (     1.25 ) & (     0.80 ) & \\
Good Health & \textbf{     0.30} &     -0.27 &     -0.01 & \textbf{    -0.33} &      0.04 & \textbf{    -0.54} & 789 &       0.15 \\ 
 & \textbf{(     0.14 )} & (     0.22 ) & (     0.16 ) & \textbf{(     0.16 )} & (     0.24 ) & \textbf{(     0.17 )} & \\
Num. of Days Sick Past Month &     -0.06 &      0.25 &      0.09 &      0.05 & \textbf{     1.01} &      0.10 & 765 &       0.10 \\ 
 & (     0.11 ) & (     0.19 ) & (     0.13 ) & (     0.13 ) & \textbf{(     0.20 )} & (     0.13 ) & \\
Engaged in A Fight &      0.00 &      0.00 &      0.00 &      0.00 &      0.00 &      0.00 & 791 &          . \\ 
 & (        . ) & (        . ) & (        . ) & (        . ) & (        . ) & (        . ) & \\
Drove Under Influence &      0.00 &      0.00 &      0.00 &      0.00 &      0.00 &      0.00 & 791 &          . \\ 
 & (        . ) & (        . ) & (        . ) & (        . ) & (        . ) & (        . ) & \\
Ever Suspended from School &     -0.03 &     -0.05 &     -0.06 &     -0.04 &     -0.05 &      0.01 & 791 &       0.03 \\ 
 & (     0.05 ) & (     0.08 ) & (     0.06 ) & (     0.06 ) & (     0.08 ) & (     0.06 ) & \\
Age At First Drink &     -0.80 &      2.97 &     -2.09 &     -2.14 &      2.45 &     -2.51 & 769 &       0.08 \\ 
 & (     1.64 ) & (     2.58 ) & (     1.86 ) & (     1.92 ) & (     2.75 ) & (     1.94 ) & \\
\bottomrule
\end{tabular}
}
\end{center}
\footnotesize
\fnDID
\end{table}

%\begin{table}[H]
%\begin{center}
%	\caption{Difference-in-Difference Across School Types and Cities, Restricting to Age-50 Cohort} \label{table:HCh-50}
%	\scalebox{0.80}{
%		\begin{tabular}{lcccccccc}
\toprule
 \textbf{Outcome} & \textbf{(1)} & \textbf{(2)} & \textbf{(3)} & \textbf{(4)} & \textbf{(5)} & \textbf{(6)} & \textbf{N} & \textbf{$ R^2$} \\
\midrule
Tried Marijuana &      0.09 &     -0.02 &      0.09 &      0.10 &     -0.00 &      0.08 & 446 &       0.03 \\ 
 & (     0.09 ) & (     0.13 ) & (     0.11 ) & (     0.08 ) & (     0.18 ) & (     0.09 ) & \\
Smokes &     -0.31 &     -0.20 &     -0.18 &     -0.21 &      0.70 &     -0.12 & 207 &       0.19 \\ 
 & (     0.33 ) & (     0.51 ) & (     0.39 ) & (     0.29 ) & (     0.60 ) & (     0.31 ) & \\
Num. of Cigarettes Per Day &      6.66 & \textbf{    28.53} &     11.93 &     -9.38 &      0.00 &    -10.36 & 114 &       0.37 \\ 
 & (     8.87 ) & \textbf{(    11.82 )} & (     9.97 ) & (     9.01 ) & (        . ) & (     9.22 ) & \\
BMI &      0.67 &     -1.08 &     -0.50 &     -1.50 &      4.53 &     -2.28 & 352 &       0.19 \\ 
 & (     1.37 ) & (     1.96 ) & (     1.73 ) & (     1.45 ) & (     3.56 ) & (     1.52 ) & \\
Good Health &     -0.23 &     -0.57 &      0.14 &     -0.20 &     -0.37 &      0.10 & 444 &       0.06 \\ 
 & (     0.31 ) & (     0.46 ) & (     0.39 ) & (     0.30 ) & (     0.63 ) & (     0.32 ) & \\
Num. of Days Sick Past Month &      0.29 &      0.44 &      0.38 &     -0.41 &     -0.42 &     -0.35 & 433 &       0.06 \\ 
 & (     0.31 ) & (     0.46 ) & (     0.38 ) & (     0.30 ) & (     0.61 ) & (     0.32 ) & \\
Engaged in A Fight &      0.00 &      0.00 &      0.00 &      0.00 &      0.00 &      0.00 & 446 &          . \\ 
 & (        . ) & (        . ) & (        . ) & (        . ) & (        . ) & (        . ) & \\
Drove Under Influence &      0.00 &      0.00 &      0.00 &      0.00 &      0.00 &      0.00 & 446 &          . \\ 
 & (        . ) & (        . ) & (        . ) & (        . ) & (        . ) & (        . ) & \\
Ever Suspended from School &      0.14 &      0.03 & \textbf{     0.22} &      0.02 &     -0.22 &      0.01 & 446 &       0.07 \\ 
 & (     0.11 ) & (     0.16 ) & \textbf{(     0.13 )} & (     0.10 ) & (     0.21 ) & (     0.11 ) & \\
Age At First Drink &     -2.41 &      1.44 & \textbf{    -6.69} &      2.34 &      5.93 &      1.96 & 424 &       0.08 \\ 
 & (     2.83 ) & (     4.09 ) & \textbf{(     3.49 )} & (     2.74 ) & (     5.59 ) & (     2.91 ) & \\
\bottomrule
\end{tabular}
}
%\end{center}
%\footnotesize
%\fnDID
%\end{table}


%
%%----------------------------------------

\begin{landscape}
\subsection{Noncognitive Outcomes - OLS results}

\begin{center}
		\scriptsize{
			\begin{longtable}{L{3cm} c c c c c c p{.5cm} c c c c c c}
				\hline \\
				\multicolumn{14}{L{18.5cm}}{\textbf{Note:} \fnOLS}
				\endfoot
				\caption{Mean outcomes for Noncongnitive measures}  \label{table:OLS_N} \\
				\toprule \\
				\textbf{Outcome} & \multicolumn{6}{c}{\textbf{C. Mean}} & & \multicolumn{6}{c}{\textbf{Mean}} \\
\quad \quad Sample & Muni & State & Reli & Priv & None & $ R^2$ & & Muni & State & Reli & Priv & None & $ R^2$ \\
\quad \quad Restriction & \tiny{$\boldsymbol{\gamma_0}$}& \tiny{$\boldsymbol{\gamma_0+\gamma_1}$}& \tiny{$\boldsymbol{\gamma_0+\gamma_2}$}& \tiny{$\boldsymbol{\gamma_0+\gamma_3}$}& \tiny{$\boldsymbol{\gamma_0+\gamma_4}$} & & & \tiny{$\boldsymbol{\gamma_0}$}& \tiny{$\boldsymbol{\gamma_0+\gamma_1}$}& \tiny{$\boldsymbol{\gamma_0+\gamma_2}$}& \tiny{$\boldsymbol{\gamma_0+\gamma_3}$}& \tiny{$\boldsymbol{\gamma_0+\gamma_4}$} \\
\hline \endhead
~\\*[.05cm]
\textbf{Locus of Control - positive} \\*[.1cm]
\quad \quad \textbf{Adult 30} & & & & & & & & \multicolumn{6}{c}{\highlight{Reference mean = \textbf{     0.13}}} \\*[.1cm]
\quad \quad \quad Reggio& 0.04 & -0.20 & \textbf{     0.12} & \textbf{    -0.07} & \textbf{     0.16} &      0.13 & & 0.13 & -0.19 & \textbf{     0.20} & \textbf{     0.24} & \textbf{     0.06} & &      0.02 \\*
\quad \quad \quad Parma& \textbf{    -1.01} & \textbf{    -0.94} & -0.70 & \textbf{    -0.65} & \textbf{    -1.21} &      0.17 & & \textbf{    -0.35} & \textbf{    -0.23} & 0.18 & \textbf{0} & \textbf{    -0.47} & &      0.06 \\*
\quad \quad \quad Padova& \textbf{    -0.33} & \textbf{    -0.30} & 0.14 & \textbf{    -0.76} & \textbf{    -0.08} &      0.14 & & \textbf{    -0.05} & \textbf{    -0.10} & 0.40 & \textbf{    -0.37} & \textbf{     0.23} & &      0.07 \\*
\\
\quad \quad \textbf{Adult 40} & & & & & & & & \multicolumn{6}{c}{\highlight{Reference mean = \textbf{     0.21}}} \\*[.1cm]
\quad \quad \quad Reggio& 0.07 & -0.47 & \textbf{     0.10} & 0.10 & -0.17 &      0.06 & & 0.21 & -0.15 & \textbf{     0.24} & 0.20 & \textbf{     0.04} & &      0.02 \\*
\quad \quad \quad Parma& \textbf{    -0.70} & \textbf{    -0.69} & \textbf{    -0.64} & 0.16 & \textbf{    -0.62} &      0.15 & & \textbf{    -0.12} & \textbf{    -0.29} & \textbf{    -0.05} & \textbf{     1.22} & \textbf{    -0.14} & &      0.02 \\*
\quad \quad \quad Padova& \textbf{    -0.43} & \textbf{    -0.69} & -0.02 & . & 0.06 &      0.17 & & \textbf{    -0.25} & -0.63 & 0.26 & . & 0.38 & &      0.15 \\*
\\
\quad \quad \textbf{Adult 50} & & & & & & & & \multicolumn{6}{c}{\highlight{Reference mean = \textbf{     0.01}}} \\*[.1cm]
\quad \quad \quad Reggio& 0.14 & \textbf{     0.33} & \textbf{     0.06} & \textbf{     1.03} & \textbf{     0.38} &      0.05 & & 0.01 & \textbf{     0.20} & \textbf{    -0.15} & \textbf{     0.85} & \textbf{     0.15} & &      0.03 \\*
\quad \quad \quad Parma& \textbf{    -1.14} & \textbf{    -1.52} & \textbf{    -0.99} & . & \textbf{    -0.94} &      0.33 & & \textbf{    -0.91} & \textbf{    -1.21} & \textbf{    -0.32} & . & -0.23 & &      0.12 \\*
\quad \quad \quad Padova& \textbf{    -0.16} & -3.19 & \textbf{    -0.14} & 0.09 & \textbf{    -0.37} &      0.12 & & \textbf{     0.18} & -2.29 & 0.03 & \textbf{     0.20} & \textbf{    -0.15} & &      0.06 \\*
\\
~\\*[.05cm]
\textbf{Depression Score - positive} \\*[.1cm]
\quad \quad \textbf{Adult 30} & & & & & & & & \multicolumn{6}{c}{\highlight{Reference mean = \textbf{    38.03}}} \\*[.1cm]
\quad \quad \quad Reggio& 36.39 & \textbf{    36.10} & 38.23 & \textbf{    41.07} & \textbf{    36.94} &      0.34 & & 38.03 & \textbf{    36.16} & \textbf{    39.62} & \textbf{    45.00} & \textbf{    36.70} & &      0.04 \\*
\quad \quad \quad Parma& \textbf{    40.61} & \textbf{    41.88} & \textbf{    41.49} & \textbf{    39.76} & \textbf{    40.62} &      0.09 & & \textbf{    38.69} & 40.41 & \textbf{    40.04} & \textbf{    38.40} & \textbf{    38.30} & &      0.02 \\*
\quad \quad \quad Padova& \textbf{    34.69} & \textbf{    34.07} & 36.81 & \textbf{    35.05} & \textbf{    34.73} &      0.12 & & \textbf{    37.85} & \textbf{    36.36} & 39.88 & 38.00 & \textbf{    38.38} & &      0.05 \\*
\\
\quad \quad \textbf{Adult 40} & & & & & & & & \multicolumn{6}{c}{\highlight{Reference mean = \textbf{    39.62}}} \\*[.1cm]
\quad \quad \quad Reggio& 38.82 & 36.18 & \textbf{    38.25} & 38.82 & 36.54 &      0.09 & & 39.62 & \textbf{    38.06} & \textbf{    39.42} & 39.50 & 37.22 & &      0.03 \\*
\quad \quad \quad Parma& \textbf{    42.43} & \textbf{    43.84} & \textbf{    43.08} & \textbf{    44.34} & 40.81 &      0.15 & & \textbf{    40.40} & \textbf{    41.96} & \textbf{    40.09} & \textbf{    43.00} & 38.19 & &      0.06 \\*
\quad \quad \quad Padova& \textbf{    38.16} & 33.09 & \textbf{    39.32} & . & 38.89 &      0.19 & & \textbf{    38.22} & 32.48 & 40.14 & . & \textbf{    39.88} & &      0.15 \\*
\\
\quad \quad \textbf{Adult 50} & & & & & & & & \multicolumn{6}{c}{\highlight{Reference mean = \textbf{    38.33}}} \\*[.1cm]
\quad \quad \quad Reggio& 37.39 & \textbf{    40.35} & \textbf{    38.96} & \textbf{    41.14} & \textbf{    38.27} &      0.10 & & 38.33 & \textbf{    40.70} & 38.36 & \textbf{    41.00} & \textbf{    37.08} & &      0.03 \\*
\quad \quad \quad Parma& \textbf{    44.24} & \textbf{    43.57} & 40.85 & . & 38.84 &      0.28 & & \textbf{    42.36} & \textbf{    42.14} & 38.91 & . & 36.80 & &      0.20 \\*
\quad \quad \quad Padova& \textbf{    34.99} & \textbf{    38.68} & \textbf{    34.65} & \textbf{    39.83} & \textbf{    33.04} &      0.08 & & \textbf{    37.18} & \textbf{    42.00} & \textbf{    36.09} & \textbf{    40.00} & \textbf{    34.89} & &      0.04 \\*
\\
~\\*[.05cm]
\textbf{Satisfied with Income} \\*[.1cm]
\quad \quad \textbf{Adult 30} & & & & & & & & \multicolumn{6}{c}{\highlight{Reference mean = \textbf{     0.60}}} \\*[.1cm]
\quad \quad \quad Reggio& 0.52 & \textbf{     0.68} & \textbf{     0.50} & \textbf{    -0.14} & \textbf{     0.43} &      0.06 & & 0.60 & \textbf{     0.68} & \textbf{     0.56} & \textbf{0} & \textbf{     0.47} & &      0.02 \\*
\quad \quad \quad Parma& \textbf{     0.34} & 0.52 & 0.60 & \textbf{     0.41} & \textbf{     0.46} &      0.23 & & \textbf{     0.24} & 0.49 & 0.62 & \textbf{     0.40} & \textbf{     0.32} & &      0.09 \\*
\quad \quad \quad Padova& \textbf{     0.40} & \textbf{     0.28} & 0.51 & \textbf{    -0.12} & \textbf{     0.48} &      0.05 & & \textbf{     0.45} & \textbf{     0.36} & \textbf{     0.57} & \textbf{0} & \textbf{     0.53} & &      0.02 \\*
\\
\quad \quad \textbf{Adult 40} & & & & & & & & \multicolumn{6}{c}{\highlight{Reference mean = \textbf{     0.67}}} \\*[.1cm]
\quad \quad \quad Reggio& 0.69 & \textbf{     0.85} & \textbf{     0.58} & \textbf{     0.66} & 0.56 &      0.07 & & 0.67 & \textbf{     0.76} & \textbf{     0.56} & \textbf{     0.60} & 0.54 & &      0.02 \\*
\quad \quad \quad Parma& \textbf{     0.50} & \textbf{     0.46} & \textbf{     0.55} & \textbf{     0.91} & 0.35 &      0.06 & & \textbf{     0.50} & \textbf{     0.42} & \textbf{     0.51} & \textbf{     1.00} & 0.30 & &      0.04 \\*
\quad \quad \quad Padova& \textbf{     0.56} & \textbf{     0.44} & \textbf{     0.58} & . & \textbf{     0.51} &      0.03 & & \textbf{     0.52} & \textbf{     0.42} & \textbf{     0.55} & . & \textbf{     0.49} & &      0.01 \\*
\\
\quad \quad \textbf{Adult 50} & & & & & & & & \multicolumn{6}{c}{\highlight{Reference mean = \textbf{     0.38}}} \\*[.1cm]
\quad \quad \quad Reggio& 0.34 & \textbf{     0.63} & 0.70 & 1.08 & \textbf{     0.52} &      0.11 & & 0.37 & \textbf{     0.60} & \textbf{     0.61} & \textbf{     1.00} & 0.38 & &      0.05 \\*
\quad \quad \quad Parma& \textbf{     0.77} & 0.31 & \textbf{     0.50} & . & \textbf{     0.52} &      0.18 & & \textbf{     0.58} & 0.14 & \textbf{     0.45} & . & 0.36 & &      0.04 \\*
\quad \quad \quad Padova& \textbf{     0.45} & \textbf{     0.51} & \textbf{     0.52} & \textbf{     0.53} & \textbf{     0.52} &      0.03 & & \textbf{     0.45} & \textbf{     0.50} & \textbf{     0.56} & \textbf{     0.50} & \textbf{     0.57} & &      0.00 \\*
\\
~\\*[.05cm]
\textbf{Satisfied with Work} \\*[.1cm]
\quad \quad \textbf{Adult 30} & & & & & & & & \multicolumn{6}{c}{\highlight{Reference mean = \textbf{     0.77}}} \\*[.1cm]
\quad \quad \quad Reggio& 0.73 & \textbf{     0.72} & \textbf{     0.82} & \textbf{     0.90} & \textbf{     0.79} &      0.05 & & \textbf{     0.77} & \textbf{     0.68} & \textbf{     0.85} & \textbf{     1.00} & \textbf{     0.79} & &      0.01 \\*
\quad \quad \quad Parma& \textbf{     0.48} & \textbf{     0.52} & \textbf{     0.47} & \textbf{     0.46} & \textbf{     0.55} &      0.09 & & \textbf{     0.59} & \textbf{     0.67} & \textbf{     0.67} & \textbf{     0.60} & \textbf{     0.64} & &      0.01 \\*
\quad \quad \quad Padova& \textbf{     0.63} & \textbf{     0.58} & \textbf{     0.69} & \textbf{    -0.03} & \textbf{     0.61} &      0.06 & & \textbf{     0.73} & \textbf{     0.68} & \textbf{     0.79} & 0 & \textbf{     0.74} & &      0.02 \\*
\\
\quad \quad \textbf{Adult 40} & & & & & & & & \multicolumn{6}{c}{\highlight{Reference mean = \textbf{     0.88}}} \\*[.1cm]
\quad \quad \quad Reggio& 0.92 & 0.93 & 0.81 & \textbf{     0.87} & \textbf{     0.88} &      0.04 & & \textbf{     0.88} & 0.88 & 0.77 & \textbf{     0.80} & \textbf{     0.80} & &      0.01 \\*
\quad \quad \quad Parma& \textbf{     0.78} & \textbf{     0.78} & \textbf{     0.82} & 0.95 & \textbf{     0.66} &      0.05 & & \textbf{     0.75} & \textbf{     0.73} & \textbf{     0.76} & \textbf{     1.00} & 0.58 & &      0.04 \\*
\quad \quad \quad Padova& \textbf{     0.82} & \textbf{     0.70} & \textbf{     0.76} & . & \textbf{     0.81} &      0.05 & & \textbf{     0.74} & \textbf{     0.58} & \textbf{     0.69} & . & \textbf{     0.74} & &      0.01 \\*
\\
\quad \quad \textbf{Adult 50} & & & & & & & & \multicolumn{6}{c}{\highlight{Reference mean = \textbf{     0.86}}} \\*[.1cm]
\quad \quad \quad Reggio& 0.79 & \textbf{     0.89} & 0.77 & \textbf{     1.01} & \textbf{     0.70} &      0.08 & & 0.86 & \textbf{     0.90} & \textbf{     0.74} & \textbf{     1.00} & \textbf{     0.64} & &      0.03 \\*
\quad \quad \quad Parma& \textbf{     0.61} & \textbf{     0.97} & \textbf{     0.66} & . & \textbf{     0.56} &      0.06 & & \textbf{     0.67} & \textbf{     1.00} & \textbf{     0.73} & . & \textbf{     0.64} & &      0.04 \\*
\quad \quad \quad Padova& \textbf{     0.58} & \textbf{     0.95} & \textbf{     0.65} & \textbf{     0.52} & \textbf{     0.61} &      0.04 & & \textbf{     0.60} & \textbf{     1.00} & \textbf{     0.67} & \textbf{     0.50} & \textbf{     0.64} & &      0.01 \\*
\\
~\\*[.05cm]
\textbf{Satisfied with Health} \\*[.1cm]
\quad \quad \textbf{Adult 30} & & & & & & & & \multicolumn{6}{c}{\highlight{Reference mean = \textbf{     0.83}}} \\*[.1cm]
\quad \quad \quad Reggio& 0.84 & \textbf{     0.90} & 0.98 & \textbf{     0.95} & 1.00 &      0.07 & & 0.83 & \textbf{     0.84} & 0.97 & \textbf{     1.00} & 0.93 & &      0.03 \\*
\quad \quad \quad Parma& \textbf{     0.87} & \textbf{     0.91} & 0.84 & 0.55 & \textbf{     0.88} &      0.05 & & \textbf{     0.93} & \textbf{     0.96} & \textbf{     0.90} & 0.60 & \textbf{     0.95} & &      0.04 \\*
\quad \quad \quad Padova& 0.85 & \textbf{     0.73} & \textbf{     0.82} & -0.07 & \textbf{     0.78} &      0.07 & & \textbf{     0.94} & 0.80 & \textbf{     0.90} & 0 & \textbf{     0.87} & &      0.04 \\*
\\
\quad \quad \textbf{Adult 40} & & & & & & & & \multicolumn{6}{c}{\highlight{Reference mean = \textbf{     0.95}}} \\*[.1cm]
\quad \quad \quad Reggio& 0.94 & \textbf{     0.98} & 0.94 & \textbf{     0.79} & 0.94 &      0.02 & & \textbf{     0.95} & \textbf{     1.00} & \textbf{     0.96} & \textbf{     0.80} & \textbf{     0.95} & &      0.01 \\*
\quad \quad \quad Parma& \textbf{     0.81} & \textbf{     0.78} & \textbf{     0.81} & 0.95 & \textbf{     0.90} &      0.03 & & \textbf{     0.81} & \textbf{     0.77} & \textbf{     0.82} & \textbf{     1.00} & \textbf{     0.88} & &      0.01 \\*
\quad \quad \quad Padova& \textbf{     0.86} & 0.48 & \textbf{     0.84} & . & \textbf{     0.88} &      0.17 & & \textbf{     0.89} & 0.50 & \textbf{     0.90} & . & 0.95 & &      0.14 \\*
\\
\quad \quad \textbf{Adult 50} & & & & & & & & \multicolumn{6}{c}{\highlight{Reference mean = \textbf{     0.75}}} \\*[.1cm]
\quad \quad \quad Reggio& 0.75 & \textbf{     0.82} & \textbf{     0.89} & \textbf{     1.03} & \textbf{     0.91} &      0.05 & & 0.75 & \textbf{     0.80} & \textbf{     0.82} & \textbf{     1.00} & \textbf{     0.80} & &      0.00 \\*
\quad \quad \quad Parma& \textbf{     0.86} & 0.47 & \textbf{     0.61} & . & 0.54 &      0.15 & & 0.75 & \textbf{     0.43} & \textbf{     0.55} & . & 0.49 & &      0.03 \\*
\quad \quad \quad Padova& \textbf{     0.48} & \textbf{     0.91} & \textbf{     0.72} & \textbf{     0.47} & \textbf{     0.59} &      0.05 & & \textbf{     0.55} & \textbf{     1.00} & 0.75 & \textbf{     0.50} & \textbf{     0.62} & &      0.03 \\*
\\
~\\*[.05cm]
\textbf{Satisfied with Family} \\*[.1cm]
\quad \quad \textbf{Adult 30} & & & & & & & & \multicolumn{6}{c}{\highlight{Reference mean = \textbf{     0.67}}} \\*[.1cm]
\quad \quad \quad Reggio& 0.58 & \textbf{     0.72} & \textbf{     0.50} & \textbf{     0.85} & \textbf{     0.64} &      0.04 & & 0.67 & \textbf{     0.77} & \textbf{     0.59} & \textbf{     1.00} & \textbf{     0.70} & &      0.01 \\*
\quad \quad \quad Parma& \textbf{     0.76} & \textbf{     0.84} & \textbf{     0.87} & \textbf{     0.90} & \textbf{     0.75} &      0.14 & & \textbf{     0.61} & \textbf{     0.74} & 0.78 & \textbf{     0.80} & \textbf{     0.57} & &      0.03 \\*
\quad \quad \quad Padova& \textbf{     0.79} & \textbf{     0.71} & \textbf{     0.87} & \textbf{     1.03} & \textbf{     0.70} &      0.05 & & \textbf{     0.73} & 0.68 & \textbf{     0.82} & \textbf{     1.00} & \textbf{     0.62} & &      0.04 \\*
\\
\quad \quad \textbf{Adult 40} & & & & & & & & \multicolumn{6}{c}{\highlight{Reference mean = \textbf{     0.80}}} \\*[.1cm]
\quad \quad \quad Reggio& 0.87 & \textbf{     0.83} & \textbf{     0.89} & \textbf{     1.09} & 0.87 &      0.01 & & 0.80 & \textbf{     0.76} & \textbf{     0.81} & \textbf{     1.00} & 0.79 & &      0.01 \\*
\quad \quad \quad Parma& \textbf{     1.02} & 0.87 & \textbf{     1.07} & \textbf{     1.18} & 0.84 &      0.12 & & \textbf{     0.83} & \textbf{     0.77} & \textbf{     0.89} & \textbf{     1.00} & 0.66 & &      0.05 \\*
\quad \quad \quad Padova& 0.87 & \textbf{     0.68} & \textbf{     0.93} & . & \textbf{     0.72} &      0.07 & & \textbf{     0.78} & \textbf{     0.58} & \textbf{     0.83} & . & 0.61 & &      0.06 \\*
\\
\quad \quad \textbf{Adult 50} & & & & & & & & \multicolumn{6}{c}{\highlight{Reference mean = \textbf{     0.62}}} \\*[.1cm]
\quad \quad \quad Reggio& 0.60 & \textbf{     0.79} & \textbf{     0.81} & \textbf{     0.99} & \textbf{     0.71} &      0.03 & & 0.63 & \textbf{     0.80} & \textbf{     0.82} & \textbf{     1.00} & \textbf{     0.70} & &      0.02 \\*
\quad \quad \quad Parma& \textbf{     1.08} & \textbf{     0.92} & \textbf{     1.00} & . & 0.72 &      0.13 & & \textbf{     1.00} & \textbf{     0.86} & \textbf{     0.90} & . & 0.65 & &      0.09 \\*
\quad \quad \quad Padova& \textbf{     0.76} & \textbf{     0.95} & \textbf{     0.74} & \textbf{     0.97} & \textbf{     0.71} &      0.04 & & \textbf{     0.82} & \textbf{     1.00} & \textbf{     0.78} & \textbf{     1.00} & \textbf{     0.75} & &      0.01 \\*
\\
~\\*[.05cm]
\textbf{Optimistic Look on Life} \\*[.1cm]
\quad \quad \textbf{Adult 30} & & & & & & & & \multicolumn{6}{c}{\highlight{Reference mean = \textbf{     0.54}}} \\*[.1cm]
\quad \quad \quad Reggio& 0.68 & 0.45 & \textbf{     0.64} & \textbf{     1.13} & 0.90 &      0.13 & & 0.54 & 0.32 & \textbf{     0.51} & \textbf{     1.00} & 0.72 & &      0.05 \\*
\quad \quad \quad Parma& \textbf{     0.45} & 0.26 & \textbf{     0.38} & 0.03 & \textbf{     0.33} &      0.08 & & \textbf{     0.62} & 0.44 & \textbf{     0.63} & 0.20 & \textbf{     0.51} & &      0.04 \\*
\quad \quad \quad Padova& \textbf{     1.01} & 0.64 & 0.67 & 0.04 & 0.68 &      0.09 & & \textbf{     0.90} & 0.56 & 0.56 & 0 & 0.56 & &      0.06 \\*
\\
\quad \quad \textbf{Adult 40} & & & & & & & & \multicolumn{6}{c}{\highlight{Reference mean = \textbf{     0.56}}} \\*[.1cm]
\quad \quad \quad Reggio& 0.39 & 0.64 & \textbf{     0.32} & \textbf{     0.52} & \textbf{     0.37} &      0.08 & & 0.56 & 0.88 & \textbf{     0.51} & \textbf{     0.75} & \textbf{     0.63} & &      0.03 \\*
\quad \quad \quad Parma& \textbf{     0.21} & \textbf{     0.07} & \textbf{     0.21} & \textbf{    -0.16} & \textbf{     0.19} &      0.04 & & \textbf{     0.31} & \textbf{     0.16} & \textbf{     0.36} & \textbf{0} & \textbf{     0.30} & &      0.02 \\*
\quad \quad \quad Padova& \textbf{     0.51} & 0.13 & \textbf{     0.51} & . & 0.39 &      0.08 & & \textbf{     0.52} & 0.17 & \textbf{     0.54} & . & \textbf{     0.42} & &      0.06 \\*
\\
\quad \quad \textbf{Adult 50} & & & & & & & & \multicolumn{6}{c}{\highlight{Reference mean = \textbf{     0.11}}} \\*[.1cm]
\quad \quad \quad Reggio& 0.13 & \textbf{     0.27} & \textbf{     0.25} & \textbf{     0.43} & \textbf{     0.05} &      0.13 & & 0.11 & \textbf{     0.30} & \textbf{     0.33} & \textbf{     0.50} & \textbf{     0.18} & &      0.03 \\*
\quad \quad \quad Parma& \textbf{     0.20} & \textbf{     0.24} & \textbf{     0.44} & . & \textbf{     0.20} &      0.18 & & 0.09 & 0.14 & 0.45 & . & \textbf{     0.15} & &      0.08 \\*
\quad \quad \quad Padova& \textbf{     0.03} & \textbf{    -0.22} & \textbf{     0.16} & \textbf{    -0.06} & \textbf{     0.08} &      0.07 & & \textbf{     0.20} & \textbf{0} & \textbf{     0.27} & \textbf{0} & \textbf{     0.20} & &      0.02 \\*
\\
~\\*[.05cm]
\textbf{Return Favor} \\*[.1cm]
\quad \quad \textbf{Adult 30} & & & & & & & & \multicolumn{6}{c}{\highlight{Reference mean = \textbf{     0.89}}} \\*[.1cm]
\quad \quad \quad Reggio& 0.80 & \textbf{     0.71} & 0.93 & \textbf{     0.87} & \textbf{     0.82} &      0.09 & & \textbf{     0.89} & 0.77 & 1.00 & \textbf{     1.00} & \textbf{     0.86} & &      0.03 \\*
\quad \quad \quad Parma& \textbf{     0.92} & \textbf{     0.90} & \textbf{     0.88} & \textbf{     0.95} & \textbf{     0.89} &      0.03 & & \textbf{     0.98} & \textbf{     0.96} & \textbf{     0.94} & \textbf{     1.00} & \textbf{     0.95} & &      0.01 \\*
\quad \quad \quad Padova& \textbf{     0.84} & \textbf{     0.74} & \textbf{     0.91} & \textbf{     0.93} & \textbf{     0.82} &      0.06 & & 0.88 & \textbf{     0.77} & \textbf{     0.94} & \textbf{     1.00} & \textbf{     0.85} & &      0.04 \\*
\\
\quad \quad \textbf{Adult 40} & & & & & & & & \multicolumn{6}{c}{\highlight{Reference mean = \textbf{     0.91}}} \\*[.1cm]
\quad \quad \quad Reggio& 0.85 & 0.72 & 0.94 & \textbf{     0.95} & \textbf{     0.86} &      0.07 & & \textbf{     0.91} & \textbf{     0.82} & 1.00 & \textbf{     1.00} & \textbf{     0.91} & &      0.03 \\*
\quad \quad \quad Parma& \textbf{     0.92} & \textbf{     0.86} & \textbf{     0.88} & \textbf{     0.96} & \textbf{     0.87} &      0.05 & & \textbf{     0.98} & \textbf{     0.92} & \textbf{     0.96} & \textbf{     1.00} & \textbf{     0.96} & &      0.01 \\*
\quad \quad \quad Padova& \textbf{     0.70} & \textbf{     0.56} & 0.90 & . & 0.99 &      0.20 & & \textbf{     0.65} & \textbf{     0.50} & 0.88 & . & 0.97 & &      0.17 \\*
\\
\quad \quad \textbf{Adult 50} & & & & & & & & \multicolumn{6}{c}{\highlight{Reference mean = \textbf{     1.00}}} \\*[.1cm]
\quad \quad \quad Reggio& 1.01 & 1.01 & \textbf{     0.98} & 1.01 & 1.01 &      0.05 & & 1.00 & 1.00 & \textbf{     0.96} & 1.00 & 1.00 & &      0.03 \\*
\quad \quad \quad Parma& 1.00 & 1.01 & 1.02 & . & \textbf{     0.95} &      0.04 & & 1.00 & 1.00 & 1.00 & . & \textbf{     0.93} & &      0.02 \\*
\quad \quad \quad Padova& \textbf{     0.82} & \textbf{     1.11} & \textbf{     0.90} & 0.99 & \textbf{     0.86} &      0.25 & & \textbf{     0.73} & 1.00 & \textbf{     0.85} & 1.00 & \textbf{     0.79} & &      0.02 \\*
\\
~\\*[.05cm]
\textbf{Put Someone in Difficulty} \\*[.1cm]
\quad \quad \textbf{Adult 30} & & & & & & & & \multicolumn{6}{c}{\highlight{Reference mean = \textbf{     0.45}}} \\*[.1cm]
\quad \quad \quad Reggio& 0.54 & 0.54 & 0.35 & \textbf{     1.11} & 0.31 &      0.08 & & 0.45 & \textbf{     0.48} & 0.28 & \textbf{     1.00} & 0.26 & &      0.04 \\*
\quad \quad \quad Parma& \textbf{     0.56} & \textbf{     0.60} & 0.43 & \textbf{     0.46} & \textbf{     0.62} &      0.14 & & \textbf{     0.26} & \textbf{     0.31} & 0.10 & \textbf{     0.20} & \textbf{     0.27} & &      0.03 \\*
\quad \quad \quad Padova& \textbf{     0.49} & \textbf{     0.50} & \textbf{     0.45} & \textbf{     1.17} & 0.29 &      0.10 & & \textbf{     0.33} & \textbf{     0.38} & \textbf{     0.28} & \textbf{     1.00} & 0.09 & &      0.05 \\*
\\
\quad \quad \textbf{Adult 40} & & & & & & & & \multicolumn{6}{c}{\highlight{Reference mean = \textbf{     0.32}}} \\*[.1cm]
\quad \quad \quad Reggio& 0.27 & 0.28 & \textbf{     0.33} & \textbf{     0.40} & 0.28 &      0.03 & & 0.32 & \textbf{     0.24} & \textbf{     0.33} & \textbf{     0.40} & \textbf{     0.29} & &      0.00 \\*
\quad \quad \quad Parma& \textbf{     0.47} & \textbf{     0.54} & \textbf{     0.38} & \textbf{     1.16} & \textbf{     0.39} &      0.09 & & \textbf{     0.33} & \textbf{     0.42} & 0.18 & \textbf{     1.00} & \textbf{     0.23} & &      0.04 \\*
\quad \quad \quad Padova& \textbf{     0.52} & \textbf{     0.65} & \textbf{     0.44} & . & 0.34 &      0.12 & & \textbf{     0.35} & \textbf{     0.46} & \textbf{     0.24} & . & 0.11 & &      0.06 \\*
\\
\quad \quad \textbf{Adult 50} & & & & & & & & \multicolumn{6}{c}{\highlight{Reference mean = \textbf{     0.50}}} \\*[.1cm]
\quad \quad \quad Reggio& 0.26 & \textbf{     0.16} & 0.26 & \textbf{     0.40} & \textbf{     0.15} &      0.15 & & 0.50 & \textbf{     0.30} & \textbf{     0.32} & 0.50 & 0.18 & &      0.04 \\*
\quad \quad \quad Parma& \textbf{     0.74} & \textbf{     0.76} & 0.30 & . & 0.20 &      0.24 & & \textbf{     0.83} & \textbf{     0.86} & 0.45 & . & 0.31 & &      0.17 \\*
\quad \quad \quad Padova& \textbf{     0.32} & \textbf{     0.47} & \textbf{     0.20} & \textbf{     0.49} & \textbf{     0.21} &      0.04 & & \textbf{     0.36} & 0.50 & \textbf{     0.22} & 0.50 & \textbf{     0.23} & &      0.02 \\*
\\
~\\*[.05cm]
\textbf{Help Someone Kind To Me} \\*[.1cm]
\quad \quad \textbf{Adult 30} & & & & & & & & \multicolumn{6}{c}{\highlight{Reference mean = \textbf{     0.92}}} \\*[.1cm]
\quad \quad \quad Reggio& 0.95 & \textbf{     0.98} & \textbf{     1.01} & \textbf{     1.00} & \textbf{     0.97} &      0.02 & & 0.92 & \textbf{     0.94} & \textbf{     0.97} & \textbf{     1.00} & \textbf{     0.91} & &      0.01 \\*
\quad \quad \quad Parma& \textbf{     0.89} & \textbf{     0.88} & \textbf{     0.88} & 0.95 & \textbf{     0.92} &      0.06 & & \textbf{     0.96} & \textbf{     0.94} & \textbf{     0.94} & \textbf{     1.00} & \textbf{     1.00} & &      0.01 \\*
\quad \quad \quad Padova& \textbf{     0.84} & \textbf{     0.82} & \textbf{     0.88} & 0.93 & \textbf{     0.84} &      0.03 & & \textbf{     0.88} & \textbf{     0.85} & \textbf{     0.91} & \textbf{     1.00} & \textbf{     0.87} & &      0.01 \\*
\\
\quad \quad \textbf{Adult 40} & & & & & & & & \multicolumn{6}{c}{\highlight{Reference mean = \textbf{     0.95}}} \\*[.1cm]
\quad \quad \quad Reggio& 0.92 & \textbf{     0.95} & \textbf{     0.96} & \textbf{     0.96} & \textbf{     0.89} &      0.03 & & \textbf{     0.95} & \textbf{     1.00} & \textbf{     1.00} & \textbf{     1.00} & \textbf{     0.94} & &      0.02 \\*
\quad \quad \quad Parma& \textbf{     0.90} & \textbf{     0.84} & \textbf{     0.87} & -0.01 & \textbf{     0.90} &      0.10 & & \textbf{     0.94} & \textbf{     0.88} & \textbf{     0.93} & 0 & \textbf{     0.97} & &      0.07 \\*
\quad \quad \quad Padova& \textbf{     0.72} & 0.55 & 0.88 & . & 0.97 &      0.19 & & \textbf{     0.69} & 0.50 & 0.88 & . & 0.97 & &      0.16 \\*
\\
\quad \quad \textbf{Adult 50} & & & & & & & & \multicolumn{6}{c}{\highlight{Reference mean = \textbf{     1.00}}} \\*[.1cm]
\quad \quad \quad Reggio& 1.03 & \textbf{     1.02} & \textbf{     0.99} & 1.02 & \textbf{     1.01} &      0.03 & & 1.00 & 1.00 & \textbf{     0.96} & 1.00 & \textbf{     0.99} & &      0.01 \\*
\quad \quad \quad Parma& \textbf{     0.96} & \textbf{     0.96} & \textbf{     0.98} & . & \textbf{     0.92} &      0.06 & & 1.00 & 1.00 & 1.00 & . & \textbf{     0.96} & &      0.01 \\*
\quad \quad \quad Padova& \textbf{     0.73} & 1.01 & \textbf{     0.86} & \textbf{     0.97} & \textbf{     0.80} &      0.17 & & \textbf{     0.73} & 1.00 & \textbf{     0.87} & 1.00 & \textbf{     0.79} & &      0.02 \\*
\\
~\\*[.05cm]
\textbf{Insult Back} \\*[.1cm]
\quad \quad \textbf{Adult 30} & & & & & & & & \multicolumn{6}{c}{\highlight{Reference mean = \textbf{     0.26}}} \\*[.1cm]
\quad \quad \quad Reggio& 0.40 & 0.56 & \textbf{     0.29} & \textbf{     0.23} & \textbf{     0.30} &      0.34 & & 0.26 & 0.52 & \textbf{     0.18} & \textbf{0} & \textbf{     0.25} & &      0.04 \\*
\quad \quad \quad Parma& \textbf{     0.33} & \textbf{     0.46} & 0.17 & \textbf{     0.26} & 0.52 &      0.12 & & \textbf{     0.31} & \textbf{     0.41} & 0.08 & \textbf{     0.20} & 0.48 & &      0.08 \\*
\quad \quad \quad Padova& \textbf{     0.41} & 0.69 & \textbf{     0.34} & \textbf{     1.12} & \textbf{     0.31} &      0.11 & & \textbf{     0.30} & 0.62 & \textbf{     0.22} & \textbf{     1.00} & \textbf{     0.17} & &      0.09 \\*
\\
\quad \quad \textbf{Adult 40} & & & & & & & & \multicolumn{6}{c}{\highlight{Reference mean = \textbf{     0.22}}} \\*[.1cm]
\quad \quad \quad Reggio& 0.31 & \textbf{     0.46} & \textbf{     0.29} & 0.71 & 0.42 &      0.19 & & 0.22 & \textbf{     0.24} & \textbf{     0.13} & 0.60 & 0.32 & &      0.04 \\*
\quad \quad \quad Parma& \textbf{     0.34} & \textbf{     0.34} & \textbf{     0.22} & \textbf{     0.17} & \textbf{     0.39} &      0.08 & & \textbf{     0.31} & \textbf{     0.35} & \textbf{     0.20} & \textbf{0} & \textbf{     0.42} & &      0.04 \\*
\quad \quad \quad Padova& \textbf{     0.51} & \textbf{     0.43} & 0.33 & . & \textbf{     0.37} &      0.03 & & \textbf{     0.42} & \textbf{     0.33} & 0.24 & . & \textbf{     0.26} & &      0.02 \\*
\\
\quad \quad \textbf{Adult 50} & & & & & & & & \multicolumn{6}{c}{\highlight{Reference mean = \textbf{     0.38}}} \\*[.1cm]
\quad \quad \quad Reggio& 0.33 & \textbf{     0.14} & \textbf{     0.05} & \textbf{     0.45} & \textbf{     0.23} &      0.10 & & 0.37 & \textbf{     0.20} & 0.07 & \textbf{     0.50} & \textbf{     0.28} & &      0.04 \\*
\quad \quad \quad Parma& \textbf{     0.41} & \textbf{     0.65} & \textbf{     0.24} & . & \textbf{     0.30} &      0.12 & & \textbf{     0.50} & \textbf{     0.71} & 0.36 & . & \textbf{     0.35} & &      0.04 \\*
\quad \quad \quad Padova& \textbf{     0.13} & \textbf{    -0.17} & \textbf{     0.26} & \textbf{    -0.03} & \textbf{     0.11} &      0.07 & & \textbf{     0.27} & \textbf{0} & \textbf{     0.35} & \textbf{0} & \textbf{     0.21} & &      0.03 \\*
\\

			\end{longtable}
		}		
\end{center}
\end{landscape}

%--------------

\subsection{Noncognitive Outcomes - Difference-in-Difference Results}

\begin{table}[H]
\begin{center}
	\caption{Difference-in-Difference Across School Types and Cities, Restricting to Age-30 Cohort} \label{table:NCh-30}
	\scalebox{0.80}{
		\begin{tabular}{lcccccccc}
\toprule
 \textbf{Outcome} & \textbf{(1)} & \textbf{(2)} & \textbf{(3)} & \textbf{(4)} & \textbf{(5)} & \textbf{(6)} & \textbf{N} & \textbf{$ R^2$} \\
\midrule
Locus of Control &      0.17 &     -0.30 &     -0.22 &     -0.10 &     -0.12 &     -0.19 & 747 &       0.15 \\ 
 & (     0.20 ) & (     0.21 ) & (     0.21 ) & (     0.23 ) & (     0.26 ) & (     0.21 ) & \\
Depression Score &      0.08 & \textbf{    -2.48} &      1.53 &     -0.17 &      0.61 &      0.44 & 774 &       0.11 \\ 
 & (     1.34 ) & \textbf{(     1.46 )} & (     1.40 ) & (     1.53 ) & (     1.82 ) & (     1.45 ) & \\
Satisfied with Income &      0.18 &      0.13 & \textbf{     0.33} &      0.11 &     -0.23 &      0.07 & 776 &       0.09 \\ 
 & (     0.12 ) & (     0.13 ) & \textbf{(     0.12 )} & (     0.13 ) & (     0.16 ) & (     0.13 ) & \\
Satisfied with Work &     -0.00 &      0.12 &     -0.06 &     -0.06 &      0.03 &     -0.04 & 771 &       0.06 \\ 
 & (     0.11 ) & (     0.12 ) & (     0.11 ) & (     0.12 ) & (     0.15 ) & (     0.12 ) & \\
Satisfied with Health &     -0.09 &      0.01 & \textbf{    -0.18} & \textbf{    -0.16} &     -0.13 & \textbf{    -0.16} & 778 &       0.04 \\ 
 & (     0.07 ) & (     0.08 ) & \textbf{(     0.08 )} & \textbf{(     0.08 )} & (     0.10 ) & \textbf{(     0.08 )} & \\
Satisfied with Family &     -0.11 &     -0.01 & \textbf{     0.22} &     -0.17 &     -0.16 &      0.17 & 771 &       0.04 \\ 
 & (     0.11 ) & (     0.12 ) & \textbf{(     0.11 )} & (     0.13 ) & (     0.15 ) & (     0.12 ) & \\
Optimistic Look on Life & \textbf{    -0.27} &      0.05 &      0.05 & \textbf{    -0.50} &     -0.11 & \textbf{    -0.31} & 711 &       0.06 \\ 
 & \textbf{(     0.12 )} & (     0.13 ) & (     0.13 ) & \textbf{(     0.14 )} & (     0.16 ) & \textbf{(     0.13 )} & \\
Return Favor &     -0.00 &      0.08 & \textbf{    -0.18} &      0.00 &      0.02 &     -0.04 & 777 &       0.05 \\ 
 & (     0.07 ) & (     0.07 ) & \textbf{(     0.07 )} & (     0.08 ) & (     0.09 ) & (     0.07 ) & \\
Put Someone in Difficulty & \textbf{     0.23} &      0.08 &      0.12 &     -0.04 &      0.01 &      0.13 & 778 &       0.10 \\ 
 & \textbf{(     0.11 )} & (     0.12 ) & (     0.11 ) & (     0.12 ) & (     0.14 ) & (     0.12 ) & \\
Help Someone Kind To Me &      0.05 &     -0.03 &     -0.06 &      0.02 &     -0.02 &      0.01 & 778 &       0.02 \\ 
 & (     0.06 ) & (     0.07 ) & (     0.07 ) & (     0.07 ) & (     0.09 ) & (     0.07 ) & \\
Insult Back & \textbf{     0.24} &     -0.08 &     -0.02 &      0.01 &      0.15 &      0.10 & 778 &       0.14 \\ 
 & \textbf{(     0.10 )} & (     0.11 ) & (     0.11 ) & (     0.12 ) & (     0.14 ) & (     0.11 ) & \\
\bottomrule
\end{tabular}
}
\end{center}
\footnotesize
\fnDID
\end{table}

\begin{table}[H]
\begin{center}
	\caption{Difference-in-Difference Across School Types and Cities, Restricting to Age-40 Cohort} \label{table:NCh-40}
	\scalebox{0.80}{
		\begin{tabular}{lcccccccc}
\toprule
 \textbf{Outcome} & \textbf{(1)} & \textbf{(2)} & \textbf{(3)} & \textbf{(4)} & \textbf{(5)} & \textbf{(6)} & \textbf{N} & \textbf{$ R^2$} \\
\midrule
Locus of Control &     -0.22 &     -0.29 &      0.01 & \textbf{    -0.67} &      0.10 & \textbf{    -0.37} & 759 &       0.11 \\ 
 & (     0.18 ) & (     0.28 ) & (     0.20 ) & \textbf{(     0.21 )} & (     0.30 ) & \textbf{(     0.21 )} & \\
Depression Score &     -0.30 & \textbf{    -3.26} &      0.18 & \textbf{    -4.10} & \textbf{     4.04} &     -2.16 & 784 &       0.08 \\ 
 & (     1.19 ) & \textbf{(     1.91 )} & (     1.37 ) & \textbf{(     1.41 )} & \textbf{(     2.05 )} & (     1.44 ) & \\
Satisfied with Income &     -0.04 &     -0.15 &      0.10 &      0.10 &     -0.17 &      0.12 & 791 &       0.07 \\ 
 & (     0.11 ) & (     0.17 ) & (     0.12 ) & (     0.13 ) & (     0.18 ) & (     0.13 ) & \\
Satisfied with Work &     -0.06 &     -0.00 &      0.12 &      0.09 &     -0.14 &      0.05 & 788 &       0.05 \\ 
 & (     0.09 ) & (     0.15 ) & (     0.11 ) & (     0.11 ) & (     0.16 ) & (     0.11 ) & \\
Satisfied with Health &      0.08 &     -0.08 &     -0.01 &      0.05 & \textbf{    -0.44} &     -0.01 & 790 &       0.09 \\ 
 & (     0.06 ) & (     0.10 ) & (     0.07 ) & (     0.08 ) & \textbf{(     0.11 )} & (     0.08 ) & \\
Satisfied with Family & \textbf{    -0.17} &     -0.05 &      0.06 &     -0.12 &     -0.14 &      0.08 & 786 &       0.05 \\ 
 & \textbf{(     0.09 )} & (     0.15 ) & (     0.10 ) & (     0.11 ) & (     0.16 ) & (     0.11 ) & \\
Optimistic Look on Life &     -0.06 & \textbf{    -0.45} &      0.08 &     -0.13 & \textbf{    -0.66} &      0.10 & 690 &       0.10 \\ 
 & (     0.11 ) & \textbf{(     0.17 )} & (     0.12 ) & (     0.14 ) & \textbf{(     0.18 )} & (     0.13 ) & \\
Return Favor &     -0.01 &      0.04 & \textbf{    -0.11} & \textbf{     0.28} &     -0.10 &      0.10 & 787 &       0.12 \\ 
 & (     0.06 ) & (     0.09 ) & \textbf{(     0.07 )} & \textbf{(     0.07 )} & (     0.10 ) & (     0.07 ) & \\
Put Someone in Difficulty &     -0.09 &      0.15 &     -0.12 & \textbf{    -0.23} &      0.19 &     -0.14 & 787 &       0.06 \\ 
 & (     0.09 ) & (     0.15 ) & (     0.11 ) & \textbf{(     0.11 )} & (     0.16 ) & (     0.12 ) & \\
Help Someone Kind To Me &      0.05 &     -0.10 &     -0.05 & \textbf{     0.25} & \textbf{    -0.27} &      0.10 & 787 &       0.13 \\ 
 & (     0.06 ) & (     0.09 ) & (     0.06 ) & \textbf{(     0.07 )} & \textbf{(     0.10 )} & (     0.07 ) & \\
Insult Back &     -0.06 &     -0.07 &     -0.04 & \textbf{    -0.31} &     -0.11 &     -0.16 & 787 &       0.07 \\ 
 & (     0.10 ) & (     0.15 ) & (     0.11 ) & \textbf{(     0.12 )} & (     0.17 ) & (     0.12 ) & \\
\bottomrule
\end{tabular}
}
\end{center}
\footnotesize
\fnDID
\end{table}

%------------------------------------------------
%
%\begin{table}[H]
%\begin{center}
%	\caption{Difference-in-Difference Across School Types and Cities, Restricting to Age-50 Cohort} \label{table:NCh-50}
%	\scalebox{0.80}{
%		\begin{tabular}{lcccccccc}
\toprule
 \textbf{Outcome} & \textbf{(1)} & \textbf{(2)} & \textbf{(3)} & \textbf{(4)} & \textbf{(5)} & \textbf{(6)} & \textbf{N} & \textbf{$ R^2$} \\
\midrule
Locus of Control - positive &      0.48 &     -0.50 &      0.72 &     -0.06 & \textbf{    -2.68} &      0.46 & 407 &       0.13 \\ 
 & (     0.37 ) & (     0.54 ) & (     0.46 ) & (     0.35 ) & \textbf{(     0.97 )} & (     0.38 ) & \\
Depression Score - positive & \textbf{    -5.08} &     -2.21 &     -3.66 &     -0.52 &      3.66 &      0.02 & 435 &       0.13 \\ 
 & \textbf{(     2.26 )} & (     3.30 ) & (     2.81 ) & (     2.17 ) & (     4.50 ) & (     2.31 ) & \\
Satisfied with Income &     -0.32 & \textbf{    -0.68} & \textbf{    -0.49} &     -0.10 &     -0.17 &     -0.32 & 436 &       0.10 \\ 
 & (     0.21 ) & \textbf{(     0.31 )} & \textbf{(     0.26 )} & (     0.20 ) & (     0.42 ) & (     0.22 ) & \\
Satisfied with Work &      0.12 &      0.26 &      0.12 &      0.18 &      0.33 &      0.13 & 419 &       0.05 \\ 
 & (     0.21 ) & (     0.31 ) & (     0.26 ) & (     0.20 ) & (     0.41 ) & (     0.22 ) & \\
Satisfied with Health & \textbf{    -0.34} &     -0.34 &     -0.26 &     -0.03 &      0.37 &      0.11 & 442 &       0.10 \\ 
 & \textbf{(     0.19 )} & (     0.28 ) & (     0.24 ) & (     0.19 ) & (     0.39 ) & (     0.20 ) & \\
Satisfied with Family & \textbf{    -0.48} &     -0.37 &     -0.33 &     -0.25 &     -0.10 & \textbf{    -0.32} & 429 &       0.04 \\ 
 & \textbf{(     0.19 )} & (     0.28 ) & (     0.24 ) & (     0.18 ) & (     0.38 ) & \textbf{(     0.20 )} & \\
Optimistic Look in Life &     -0.04 &     -0.23 &      0.04 &     -0.03 &     -0.48 &     -0.15 & 374 &       0.07 \\ 
 & (     0.18 ) & (     0.25 ) & (     0.22 ) & (     0.17 ) & (     0.35 ) & (     0.18 ) & \\
Return Favor &      0.04 &      0.06 &      0.09 &      0.08 &      0.31 &      0.15 & 443 &       0.17 \\ 
 & (     0.11 ) & (     0.15 ) & (     0.13 ) & (     0.10 ) & (     0.21 ) & (     0.11 ) & \\
Put Someone in Difficulty & \textbf{    -0.43} &      0.07 & \textbf{    -0.41} &      0.03 &      0.16 &     -0.10 & 443 &       0.13 \\ 
 & \textbf{(     0.19 )} & (     0.27 ) & \textbf{(     0.23 )} & (     0.18 ) & (     0.37 ) & (     0.19 ) & \\
Help Someone Kind To Me &      0.14 &      0.13 &      0.18 &      0.10 &      0.28 & \textbf{     0.19} & 443 &       0.14 \\ 
 & (     0.11 ) & (     0.15 ) & (     0.13 ) & (     0.10 ) & (     0.21 ) & \textbf{(     0.11 )} & \\
Insult Back &     -0.01 &      0.40 &      0.16 &      0.08 &     -0.18 & \textbf{     0.40} & 442 &       0.07 \\ 
 & (     0.20 ) & (     0.29 ) & (     0.24 ) & (     0.19 ) & (     0.39 ) & \textbf{(     0.20 )} & \\
\bottomrule
\end{tabular}
}
%\end{center}
%\footnotesize
%\fnDID
%\end{table}


\end{document}
