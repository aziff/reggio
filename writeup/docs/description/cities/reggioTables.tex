%Category: Applications%%%%%%%%%%%

%%%%%%%%%%%
%% NumApps.tex
%%Numbers of Applications to Municipal Preschools
%%%%%%%%%%%

%% This table can be obtained using the dataset: klmReggio/ADMINISTRATIVE DATA COLLECTION/DATI/ReggioAdminData.dta
%% tab anno_scolastico migr_b, row

\documentclass{article}
\begin{document}

\begin{table}[ht!]
\caption{\textbf{Numbers of Applications to Municipal Preschools, by School Year, Reggio Emilia}}
\label{tab:NumApps}
\vspace{-5mm}
\begin{center}
\begin{tabular}{ccc}
\hline\hline
\textbf{School Year} & \textbf{Applications (\#)} & \textbf{Immigrants (\%)} \\
\hline
2003-2004 & 1,228 & 13.4 \\[0.2em]
2004-2005 & 1,312 & 14.3 \\[0.2em]
2005-2006 & 1,442 & 19.4 \\[0.2em]
2006-2007 & 1,409 & 18.7 \\[0.2em]
2007-2008 & 1,349 & 20.8 \\[0.2em]
2008-2009 & 1,418 & 22.1 \\[0.2em]
2009-2010 & 1,352 & 23.2 \\[0.2em]
2010-2011 & 1,331 & 28.1 \\[0.2em]
\hline
\end{tabular}
\end{center}
\footnotesize{{\bfseries Notes:} Source: authors' calculations from the administrative data on the universe of applications to the municipal preschools of Reggio Emilia. The first column reports the school year the application refers to; the second column reports the number of applications to the municipal preschools received in that school year; the third column reports the percentage of applications made for immigrant children.}
\end{table} 

\begin{table}[ht!]
\caption{\textbf{Outcome of Applications to Municipal Preschools, by School Year, Reggio Emilia}}
\label{tab:OutcApps}
\vspace{-5mm}
\begin{center}
\begin{tabular}{c c c c c }
\hline\hline
\textbf{School Year} & \textbf{Attended (\%)} & \textbf{Withdrew (\%)} & \textbf{Waitlist (\%)} & \textbf{Refused (\%)} \\
\hline
        2003-2004  &   48.3  &    42.8  &    2.9  &    6.0   \\[0.2em]
        2004-2005  &   48.3  &    36.9  &    6.2  &    8.6   \\[0.2em]
        2005-2006  &   60.1  &    20.0  &    8.1  &   11.9   \\[0.2em]
        2006-2007  &   52.8  &    24.8  &   13.4  &    8.9   \\[0.2em]
        2007-2008  &   48.5  &    20.9  &   23.4  &    7.1   \\[0.2em]
        2008-2009  &   53.4  &     1.3  &   35.6  &    9.8   \\[0.2em]
        2009-2010  &   52.2  &     2.1  &   36.2  &    9.6   \\[0.2em]
        2010-2011  &   53.5  &     1.9  &   37.1  &    7.5   \\[0.2em]
\hline
\textit{Total} &   \textit{52.2 \%} &   \textit{18.4 \%} &   \textit{20.6 \%} &    \textit{8.7 \%}  \\[0.2em]
\hline
\end{tabular}
\end{center}
\footnotesize{{\bfseries Notes:} Source: authors' calculations from the administrative data on the universe of applications to the municipal preschools of Reggio Emilia. The cells display the percentages of applications for that particular school year which had that particular outcome: the child attended the assigned preschool (column 1); the child withdrew from the municipal preschool system (either because she went to another preschool or stayed home) after being accepted but before being assigned a specific preschool (column 2); the child was on waiting list (column three); the child dropped out from the municipal school system after a specific preschool was assigned (column 4).}
\end{table}

\begin{table}[ht!]
\caption{\textbf{Total Reference Sample}}
\label{tab:TotRefSample}
\vspace{-5mm}
\begin{center}
\begin{tabular}{ l c c c c }
\hline\hline
\textbf{Cohort} & \textbf{Reggio} & \textbf{Parma} & \textbf{Padova} & \textbf{Total}\\
\hline
Italian Children born in 2006 (Cohort V)   & 1,096 & 1,054 & 1,224 & 3,374\\[0.2em]
Immigrant Children born in 2006 (Cohort V) &   296 &   224 &   264 &   784\\[0.2em]
Adolescents born in 1994 (Cohort IV)       &   865 &   806 & 1,113 & 2,784\\[0.2em]
Adults born in 1980-81 (Cohort III)        & 1,205 & 1,355 & 1,630 & 4,190\\[0.2em]
Adults born in 1969-70 (Cohort II)         & 1,655 & 2,057 & 2,200 & 5,912\\[0.2em]
Adults born in 1954-59 (Cohort I)          & 3,646 & 3,999 & 5,587 & 13,232\\[0.2em]
\hline
Total                                      & 8,763 & 9,495 & 12,018 & 30,276\\
\hline
\end{tabular}
\end{center}
\footnotesize{{\bfseries Notes:} Total number of names provided by the population registries who satisfied the selection criteria (born in the city of residence and of Italian citizenship -- except for Immigrant Children born in 2006), broken down by City and Cohort. Source: authors calculations on data provided by the population registries.}
\end{table} 

\begin{table}[ht!]
\caption{\textbf{Completed Interviews and Response Rates}}
\footnotesize
\label{tab:RespRate}
\vspace{-5mm}
\begin{center}
\begin{tabular}{ l | c | c | c | c | c }
\hline\hline
\textbf{Cohort} & \multicolumn{2}{c}{\textbf{Reggio}} & \textbf{Parma} & \textbf{Padova} & \textbf{Total}\\
\hline
Children (yob 2006)       & Mun. & Rel.+St.+NoP. & Mun.+Rel.+St.+NoP. & Mun.+Rel.+St.+NoP.&\\[0.2em]
Italian (Cohort V)          & 160  & 151            & 291                & 278               & 880\\[0.2em]
\hline
\textit{Nov.2012-Aug.2013} & \multicolumn{2}{c|}{\textit{RR: 50.1\%}} & \textit{RR: 62.7\%} & \textit{RR: 50.1\%} & \textit{RR: 53.6\%}\\[0.2em]
\hline
Children (yob 2006)       & Mun. & Rel.+St.+NoP. & Mun.+Rel.+St.+NoP. & Mun.+Rel.+St.+NoP.&\\[0.2em]
Immigrant (Cohort V)        & 70   & 40             & 58                 & 113               & 281\\[0.2em]
\hline
\textit{Nov.2012-Aug.2013} & \multicolumn{2}{c|}{\textit{RR: 53.1\%}} & \textit{RR: 49.2\%} & \textit{RR: 63.1\%} & \textit{RR: 55.8\%}\\[0.2em]
\hline
Adolescents (yob 1994)    & Mun. & Rel.+St.+NoP. & Mun.+Rel.+St.+NoP. & Mun.+Rel.+St.+NoP.&\\[0.2em]
(Cohort IV)                 & 156  & 144            & 254                & 282               & 836\\[0.2em]
\hline
\textit{Nov.2012-Jul.2013} & \multicolumn{2}{c|}{\textit{RR: 57.1\%}} & \textit{RR: 58.5\%} & \textit{RR: 55.5\%} & \textit{RR: 57.0\%}\\[0.2em]
\hline
Adults (yob 1980-81)       & Mun. & Rel.+St.+NoP. & Mun.+Rel.+St.+NoP. & Mun.+Rel.+St.+NoP.&\\[0.2em]
(Cohort III)                 & 143  & 137            & 251                & 251               & 782\\[0.2em]
\hline
\textit{May 2013-Nov.2013} & \multicolumn{2}{c|}{\textit{RR: 58.3\%}} & \textit{RR: 58.2\%} & \textit{RR: 57.4\%} & \textit{RR: 58.0\%}\\[0.2em]
\hline
Adults (yob 1969-70)        & Mun. & Rel.+St.+NoP. & Mun.+Rel.+St.+NoP. & Mun.+Rel.+St.+NoP.&\\[0.2em]
(Cohort II)                   & 125  & 160            & 254                & 252               & 791\\[0.2em]
\hline
\textit{May 2013-Nov.2013} & \multicolumn{2}{c|}{\textit{RR: 59.3\%}} & \textit{RR: 56.3\%} & \textit{RR: 57.5\%} & \textit{RR: 57.7\%}\\[0.2em]
\hline
Adults (yob 1954-59)     & \multicolumn{2}{c|}{Mun.+Rel.+St.+NoP.} & Mun.+Rel.+St.+NoP. & Mun.+Rel.+St.+NoP.&\\[0.2em]
(Cohort I)                 & \multicolumn{2}{c|}{200} & 103 & 146 & 449\\[0.2em]
\hline
\textit{May 2013-Oct.2013} & \multicolumn{2}{c|}{\textit{RR: 52.2\%}} & \textit{RR: 63.6\%} & \textit{RR: 62.7\%} & \textit{RR: 57.7\%}\\[0.2em]
\hline 
\textbf{Total Interviews} & \multicolumn{2}{c|}{\textbf{1,486}} & \textbf{1,211} & \textbf{1,322} & \textbf{4,019} \\[0.2em]
\textit{Total Response Rates}       & \multicolumn{2}{c|}{\textit{55.1\%}} & \textit{58.8\%} & \textit{56.2\%} & \textit{56.5\%} \\
\hline
\end{tabular}
\end{center}
\footnotesize{{\bfseries Notes:} Mun.=Municipal Preschool; Rel.=Religious Preschool; St.=State Preschool; NoP.=No Preschool attended. yob=year of birth. For each cohort, the third line in the first column reports the interview periods. RR=response rate. The response rates are calculated as the ratio of interviews to total valid contacts. Valid contacts are the sum of: completed interviews, sharp refusal, no person present, talked with a relative, left paper questionnaire but never returned, interview began but not completed.}
\end{table} 

\begin{table}[ht!]
\caption{\textbf{Percentage of people living in the same city since birth, by cohort}}
\label{tab:SameCity}
\vspace{-5mm}
\begin{center}
\begin{tabular}{ l c c c c }
\hline\hline
\textbf{Cohort} & \textbf{Reggio (\%)} & \textbf{Parma (\%)} & \textbf{Padova (\%)} & \textbf{Total (\%)}\\
\hline
Italian Children born in 2006 (Cohort V)   & 61.3  & 70.2  & 65.1  & 65.2 \\[0.2em]
Adolescents born in 1994 (Cohort IV)       & 58.1  & 63.0  & 64.4  & 61.9 \\[0.2em]
Adults born in 1980-81 (Cohort III)        & 26.5  & 27.5  & 32.6  & 29.0 \\[0.2em]
Adults born in 1969-70 (Cohort II)         & 27.9  & 31.6  & 31.9  & 30.6 \\[0.2em]
Adults born in 1954-59 (Cohort I)          & 28.8  & 27.9  & 31.4  & 29.5 \\[0.2em]
\hline
\textit{Total}         & \textit{32.3\%}  & \textit{32.5\%}  & \textit{35.2\%} & \textit{33.5\%} \\
\hline
\end{tabular}
\end{center}
\footnotesize{{\bfseries Notes:} Reference sample who satisfied the selection criteria (born in the city of residence and of Italian citizenship) as a percentage of the total number of names given by the population registries, broken down by City and Cohort. Source: authors calculations on data provided by the population registries.}
\end{table}

\begin{table}[!ht]
\caption{\textbf{Administrative Data Summary Statistics, Italian and Immigrant Children, Reggio Emilia}}
\label{tab:AdminSum}
\begin{center}
%\begin{footnotesize}

\begin{tabular}{l*{3}{c}} \hline\hline
             &     \textbf{Italian}  &     \textbf{Migrant}  &       \textbf{Total}\\
\hline
Female       &       48.6\%  &       44.8\%  &       47.8\%\\
%             &     (50.0)  &     (49.7)  &     (50.0)\\
%Migrant Father  &       0.1  &       0.8  &       0.2\\
%              &     (0.2)  &     (0.4)  &     (0.4)\\
%Migrant Mother  &       0.1  &       0.9  &       0.3\\
%              &     (0.3)  &     (0.4)  &     (0.4)\\
Child born in Reggio  &       67.8\%  &       67.8\%  &       67.8\% \\
%              &     (46.7)  &     (46.7)  &     (46.7) \\
Lives in Reggio  &       98.4\%  &       99.6\%  &       98.6\%\\
%              &     (12.6)  &     (6.1)  &     (11.6)\\
%Father born in Reggio  &       0.3  &       0.1  &       0.3\\
%              &     (0.5)  &     (0.3)  &     (0.5)\\
%Mother born in Reggio  &       0.4  &       0.0  &       0.3\\
%              &     (0.5)  &     (0.2)  &     (0.5)\\
Intact Family  &       94.8\%  &       86.6\%  &       93.6\%\\
%             &     (22.2)  &     (34.2)  &     (24.5)\\
Father low education  &       39.3\%  &       52.1\%  &       41.5\%\\
%              &     (48.8)  &     (50.0)  &     (49.3)\\
%Father high school grad  &       0.4  &       0.4  &       0.4\\
%              &     (0.5)  &     (0.5)  &     (0.5)\\
Father college graduate  &       17.1\%  &      8.8\% &       15.7\%\\
%             &     (37.7) &     (28.3) &     (36.4)\\
Mother low education &       28.5\% &       55.7\% &       33.3\%\\
%             &     (45.2) &     (49.7) &     (47.1)\\
%Mother high school grad &       0.5 &       0.3 &       0.5\\
%             &     (0.5) &     (0.5) &     (0.5)\\
Mother college graduate &       20.8\% &       9.6\% &       18.9\%\\
%             &     (40.6) &     (29.4) &     (39.1)\\
%Asked for priority &       0.5 &       0.5 &       0.5\\
%             &     (0.5) &     (0.5) &     (0.5)\\
%Asked for municipal preschool &       0.9 &       0.8 &       0.9\\
%             &     (0.3) &     (0.4) &     (0.3)\\
%Adopted      &       0.0 &       0.0 &       0.0\\
%             &     (0.1) &     (0.1) &     (0.1)\\
Number of siblings &       1.10 &       1.28 &       1.14 \\
%             &     (0.71) &     (0.81) &     (0.73)\\
Father works &       94.6\% &       90.2\% &       93.8\%\\
%             &     (22.5) &     (29.8) &     (24.2)\\
%Father working hours &      41.2 &      39.9 &      40.9\\
%             &     (7.2) &     (3.9) &     (6.7)\\
Mother works &       72.6\% &       42.2\% &       66.7\%\\
%             &     (44.6) &     (49.4) &     (47.1)\\
%Mother working hours &      33.8 &      33.4 &      33.7\\
%             &     (8.7) &    (10.2) &     (8.9)\\
%ISEE         &     1.3e+04 &    4924.6 &     1.1e+04\\
%             &  (7319.4) &  (4307.7) &  (7532.8)\\
Primary Score &      25.0 &      20.5 &      24.1\\
%             &    (13.9) &    (15.1) &    (14.2)\\
Secondary Score &      29.4 &      30.7 &      29.6\\
%             &    (14.9) &    (18.8) &    (15.7)\\
Child accepted &       82.8\% &       65.7\% &       79.4\%\\
%             &     (37.7) &     (47.5) &     (40.5)\\
\hline
\(N\)        &       8,525 &      2,145  &      10,670 \\
\hline
\end{tabular}
%\end{footnotesize}
\end{center}
\footnotesize{
{\bfseries Notes:} Source: authors' calculations from the administrative data on the universe of applications to the municipal preschools of Reggio Emilia. Standard deviations in parenthesis. ``Low education'' includes no schooling, elementary schooling, and middle-high schooling.}
\end{table}

\begin{table}
\caption{\textbf{Sections of the Questionnaires}}
\scriptsize
\label{tab:SecQuest}
\vspace{-5mm}
\begin{center}
\begin{tabular}{ c c c }
\hline\hline
\textbf{Children} & \textbf{Adolescents} & \textbf{Adults}\\
\hline
\textbf{\textit{(A-P) Caregiver}} & \textbf{\textit{(A-P) Caregiver}} & \textbf{\textit{(A-P) Respondent}}\\
\hline
(A) Family Characteristics & (A) Family Characteristics & (A) Family Characteristics \\
(B) Child Preschool & (B) Adolescent Preschool & (B) Education and Preschool \\
(C) Caregiver Education & (C) Caregiver Education & \\
(D) Caregiver and Household-head & (D) Caregiver and Household-head & (C) Work Experience \\
     Work Experience & Work Experience & \\
(E) Caregiver Marriage and Fertility & (E) Caregiver Marriage and Fertility & (D) Marriage and Fertility \\
 &  & (E) Parents \\
(F) Grandparents & (F) Grandparents & (F) Grandparents \\
(G) Parenting$^{1}$ & (G) Parenting$^{2}$ & (G) Parenting$^{1}$ \\
(H) Child SDQ$^{4}$ & (H) Adolescent SDQ$^{4}$ &  \\
 &  & (H) Child Health$^{3}$ \\
(I) Caregiver and Child Health$^{3}$ & (I) Caregiver and Adolescent Health$^{3}$ & (I) Your Health$^{3}$ \\
(L) Caregiver Social Capital$^{5}$ & (L) Caregiver Social Capital$^{5}$ & (L) Social Capital$^{6}$ \\
(M) Caregiver Time Use$^{7}$ & (M) Caregiver Time Use$^{7}$ & (M) Time Use$^{7}$ \\
(N) Caregiver on Immigration$^{8}$ & (N) Caregiver on Immigration$^{8}$ & (N) Immigration$^{8}$ \\
\textit{(O) Caregiver self-completed} & \textit{(O) Caregiver self-completed} & \textit{(O) Self-completed} \\
\textit{(O-a) Noncognitive$^{9}$} & \textit{(O-a) Noncognitive$^{9}$} & \textit{(O-a) Noncognitive$^{10}$} \\
 &  & \textit{(O-b) Depression$^{11}$} \\
\textit{(O-b) Unhealthy Habits$^{12}$} & \textit{(O-b) Unhealthy Habits$^{12}$} & \textit{(O-c) Risky/unhealthy$^{13}$} \\
\textit{(O-c) Racism$^{14}$} & \textit{(O-c) Racism$^{14}$} & \textit{(O-d) Opinions$^{15}$ and Racism$^{14}$} \\
\textit{(O-d) Income} & \textit{(O-d) Income} & \textit{(O-e) Income} \\
 &  & \textit{(O-f) Weight} \\
(P) Caregiver IQ $^{17}$ & (P) Caregiver IQ$^{17}$ & (P) Respondent IQ$^{17}$ \\
\hline
\textbf{\textit{(A-F) Child}} & \textbf{\textit{(A-G) Adolescent}} & \\
\hline
(A) Child School & (A) Adolescent School & \\
 & (B) Adolescent Health$^{4}$ & \\
(B) Child Social Capital$^{18}$ & (C) Adolescent Social Capital$^{6}$ & \\
(C) Child Time Use$^{7}$ & (D) Adolescent Time Use$^{7}$ & \\
 & (E) Adolescent on Immigration$^{8}$ & \\
 & \textit{(F) Adolescent Self-completed} & \\
 & \textit{(F-a) Relation with Parents$^{2}$} & \\
(D) Child Reciprocity$^{19}$ & \textit{(F-b) Noncognitive$^{10}$} & \\
 & \textit{(F-c) Depression$^{11}$} & \\
 & \textit{(F-d) Sentimental Life} & \\
 & \textit{(F-e) Risky/unhealthy$^{13}$} & \\
(E) Child Racism$^{20}$ & \textit{(F-f) Opinions$^{16}$ and Racism$^{14}$} & \\
 & \textit{(F-g) Weight} & \\
(F) Child IQ$^{17}$ & (G) Adolescent IQ$^{17}$ &  \\
\hline
\end{tabular}
\end{center}
\scriptsize{{\bfseries Notes.} 1: Home Observation Measurement for the Environment (HOME, \cite{Caldwell1984}). 2: Questions on communication, independence, and attachment. From the Avon Longitudinal Study of Parents and Children (ALSPAC). 3: Self-assessed health, height and weight, sick days, visits to the doctor and dentist, eating habits, and level of physical exercise. 4: Strength and Difficulties Questionnaire: a widely-used mental health scale inquiring about emotional symptoms, conduct problems, hyperactivity/inattention, peer relationships problems, and prosocial behavior (SDQ, \cite{Goodman1997}), also used in the ALSPAC and in the MCS. 5: Friendship ties, sociability, political opinions, religiosity. Some questions drawn from \cite{Onyx2000} Social Capital Questionnaire. 6: Volunteering, friendship ties, online social networks, sociability, political opinions, religiosity, discrimination. Some questions drawn from \cite{Onyx2000} Social Capital Questionnaire. 7: Stress and satisfaction with time use (\cite{ISTAT} ISTAT-Indagine Multiscopo 2002-2003). 8: For Italians: opinion about immigration, contact with foreigners. For Immigrants: time needed for integration, friendship with Italians. 9: Short version of the Rotter Locus-of-Control Scale, (\cite{Rotter1966}; National Longitudinal Study of Youth NLSY). 10: Short version of the Rotter Locus-of-Control Scale, (\cite{Rotter1966}). Trust and reciprocity towards strangers (\cite{Dohmen2008}; German Socioeconomic Panel G-SOEP). Health, work or school, family, past- present- and future-life satisfaction (\cite{Cantril1965} Self-Anchoring Ladder; Gallup Survey). 11: Center for Epidemiologic Studies Depression (CES-D) Scale (\cite{Radloff1977}; NLSY). 12: Unhealthy habits: smoking, drinking. Opinions on minimum-age laws. 13: Unhealthy habits: smoking, drinking, drugs. Opinions on minimum-age laws. Risky Behaviors: lying, cheating, stealing, driving under the influence, being suspended from school, and participating in brawls. 14: For Italians: level of annoyance about migrants, willingness to live next door to a foreigner. For Immigrants: ease of integration and feeling of discrimination. 15: Opinion on gender and family issues. 16: Probability of future marriage, children, living a long and successful life. 17: Raven Progressive Matrices: 12-items for adults and adolescents, 18-colored-items for children. Non-verbal test of reasoning ability and pattern completion. 18: Friendship ties, sharing, reciprocity. Happiness and satisfaction using visual scales with emoticons (Child Outcome Rating Scale, \cite{Duncan2003a}). 19: Hypothetical game of sharing candies. Developed by Piovesan and Montinari in Italian elementary schools (\cite{Shaw2014}). 20: Following \cite{Clark1947}, the child was shown a drawing of a 5 boys (or girls) identical in every aspect but for skin and hair color. (S)he was asked to point out which one (s)he would like as playmate, which one looked likeable, which one looked bad, and which one looked like him (her).}
\end{table}
\end{document}



