\documentclass[12pt]{article}

\usepackage{booktabs}
\usepackage{float}
\usepackage[margin=1in]{geometry}
\usepackage{hyperref}
\usepackage{morefloats}

\begin{document}

\title{Comparing Reggio Emilia, Parma, and Padova}
\author{Reggio Team}
\date{Original version: June 10, 2016 \\ Current version: \today}
\maketitle


\begin{table}[htbp]
\begin{center}
\caption{Demographic Comparison of Reggio Emilia, Parma, and Padova}
\label{tab:comparison}
\begin{tabular}{lrrrr}
\toprule
& \multicolumn{3}{c}{City} & Italy \\
\cmidrule{2-4}
& Reggio Emilia & Parma & Padova & \\
\midrule
Population (2013)* & 172,525 &  187,938 & 209,678 & 60,782,668 \\
Average per-capita income (2011 euros)**  & 25,226 & 28,437 & 29,915 & \\
Immigrant Detail (2011)* & & & & \\
\quad Immigrants from Europe & 19,015 & 18,538 & 28,821 & 1,141,540 \\
\quad Immigrants from Africa & 26,106  & 20,878 & 25,183 & 1,096,547 \\
\quad Immigrants from Asia & 25,097 & 9,239 & 17,482 & 903,957 \\
\quad Immigrants from America & 2,306 & 3,305 & 2,661 & 391,189 \\
\quad Immigrants from Oceania & 10 & 18 & 24 & 2,448 \\
\bottomrule
\end{tabular}
\end{center}
\footnotesize Note: *ISTAT, \url{http://www.demo.istat.it/}; **Finance Minister, taxable income for 2011.
\end{table}

\end{document}