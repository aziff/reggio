\section{Non-cognitive Variables}
\label{sec:non-cognitive}

%\subsection{List of variables defined in document} ~\\

\begin{table}[H]
\setlength{\tabcolsep}{20pt}
\rowcolors{3}{gray!7}{white}
	\begin{center}

	\scalebox{0.84}{
	\begin{threeparttable}
	\caption{Availability of variables by cohort}
		\begin{tabular}{L{5.5cm}C{2cm} C{2cm} C{2cm}} \hline
																	& \textbf{Children Cohort} 			& \textbf{Adol Cohort} 		& \textbf{Adult Cohort} 		\\ 
		\hline	
		\textbf{Depression Score}								&										& \checkmark							& \checkmark					\\
		\textbf{Locus of Control}									&										& \checkmark							& \checkmark					\\
		%Caregiver Locus of Control								& \checkmark							&\checkmark															\\
		\textbf{Optimistic outlook on life}						&										& \checkmark							& \checkmark					 \\

		\textbf{SDQ Scores} \footnotemark \footnotemark		&										&										&								\\
		\quad \quad SDQ Composite score						& \checkmark							& \checkmark							&								\\
		\quad \quad SDQ Emotional score						& \checkmark							& \checkmark							&								\\
		\quad \quad SDQ Conduct score							& \checkmark							& \checkmark							&								\\
		\quad \quad SDQ Hyper score							& \checkmark							& \checkmark							&								\\
		\quad \quad SDQ Peer problems score					& \checkmark							& \checkmark							&								\\
		\quad \quad SDQ Pro-social score						&\checkmark							& \checkmark							&								\\

		\textbf{Satisfaction variables}							&										&										&								 \\
		\quad \quad Satisfied with income						& 										&										& \checkmark					\\
		\quad \quad Satisfied with work							& 										&										& \checkmark					\\	
		\quad \quad Satisfied with health						&										& \checkmark							& \checkmark					\\
		\quad \quad Satisfied with family						&										& \checkmark							& \checkmark					\\
		\quad \quad Satisfied with school						&										& \checkmark 							&								\\
		%\quad \quad Satisfied with education					&										& \checkmark 							&								\\
		
		\textbf{Reciprocity variables}								&										&										&								\\
		\quad \quad Return favor									&										& \checkmark							& \checkmark					 \\
		\quad \quad Put someone in difficulty					&										& \checkmark							& \checkmark					 \\	
		\quad \quad Help someone kind to me					&										& \checkmark							& \checkmark					 \\
		\quad \quad Insult back									&										& \checkmark							& \checkmark					 \\

		
		\hline
		\end{tabular}

		\begin{tablenotes}
		\singlespace
		\footnotesize{
			\item [1] The dataset includes the childSDQ score for children. This score is reported by mothers.
			\item [2] The dataset includes data on both childSDQ and SDQ scores for Adolescents. The childSDQ is reported by the mother, and the SDQ is reported by adolescent.	
		}
		\end{tablenotes}
	\end{threeparttable}
	}
	\end{center}
\end{table}
\setcounter{footnote}{0}

 \clearpage

\subsection{Depression score}
\textbf{Variable name in data}: pos\_Depression\_score \\[.3cm]
The variable ranges from 16 to 50 with higher values corresponding with lower levels of depression. The variable is constructed as the sum of the following underlying variables where each variable records respondent's response to the corresponding question on a 1 - 5 scale.

\begin{table}[H]
\setlength{\tabcolsep}{20pt}
\begin{center}

\footnotesize{
	\begin{tabular}{l L{6cm} l}
	\hline
	\textbf{Variable} 	& \textbf{Question} 													& \textbf{Values (1 - 5)} \\
	\hline
	
	Depress01			& I was bothered by things that don’t usually bother me			& 1 = ``Always" and 5 = ``Never"  \\	
	Depress02			& I had trouble keeping my mind on what I was doing				& 1 = ``Always" and 5 = ``Never"  	\\
	Depress03			& I felt depressed														& 1 = ``Always" and 5 = ``Never"	 \\
	Depress04			& I felt everything I did was an effort								& 1 = ``Always" and 5 = ``Never"	 \\
	Depress05			& I felt hopeful about the future										& 1 = ``Never" and 5 = ``Always"	 \\
	Depress06			& I felt fearful 														& 1 = ``Always" and 5 = ``Never"	\\ 
	Depress07			& My sleep was restless												& 1 = ``Always" and 5 = ``Never" \\
	Depress08			& I was happy															& 1 = ``Never" and 5 = ``Always" \\
	Depress09			& I felt lonely															& 1 = ``Always" and 5 = ``Never"  \\
	Depress10			& I could not get going 												& 1 = ``Always" and 5 = ``Never" \\
	
	\hline
	
	\end{tabular}
}

\end{center}
\end{table}
\clearpage

\subsection{Locus of Control}
\textbf{Variable name in data}: pos\_LocusControl \\[.3cm]
This is a factor variable that ranges from -2.42735 to 1.340119 with higher values corresponding with better levels of control. The factor is based on the following underlying variables where each variable records respondent's response to the corresponding questions on a scale of 1 - 5. Respondent is assigned a value of 1 if he/she answers ``Strongly agree" and a value of 5 if respondent answers ``Strongly disagree". \\

\begin{table}[H]
\setlength{\tabcolsep}{20pt}
\begin{center}

\footnotesize{
	\begin{tabular}{l L{10cm}}
	\hline
	\textbf{Variable} & \textbf{Question} \\
	\hline
	
	Locus1		& I feel that I don’t have enough control over the direction my life is taking \\	
	Locus2		& It is not always wise to plan too far ahead, because many things turn out to be a matter of good or bad fortune anyhow\\
	Locus3		& Getting what I want has little or nothing to do with luck \\
	Locus4		& I feel that I have little influence over the things that happen to me \\
	
	\hline
	
	\end{tabular}
}

\end{center}
\end{table}

\subsection{Optimistic outlook on life}
\textbf{Variable name in data}: optimist \\[.3cm]
This is a dummy variable that takes on the value of 1 if the individual feels he/she will be at a better point in his/her ``ladder of life" in the future than today, and 0 otherwise. 

\subsection{SDQ Scores}
\subsubsection{SDQ Emotional score}
\textbf{Variable name in data}: pos\_SDQEmot\_score \\[.3cm]
The variable takes on values from 0 to 10, with higher values corresponding with better outcomes. The variable is constructed as the mean of 5 underlying variables, where each variable measures respondent's response to the corresponding questions in the table below. The underlying variables take on the value of 1 if the respondent answers  ``Completely true" , 2 if the answer is ``Partially true", and 3 if the answer is ``False".

\begin{table}[H]
\setlength{\tabcolsep}{20pt}
\begin{center}
\begin{tabular}{l l}
\hline
\textbf{Variable} & \textbf{Label} \\
\hline
childSDQEmot1 & Often complains of headaches, stomach-aches or sickness\\
childSDQEmot2 & Frequently worried or often seems worried\\
childSDQEmot3 & Often unhappy, depressed or tearful\\
childSDQEmot4 & Nervous or clingy in new situations, easily loses confidence\\
childSDQEmot5 & Many fears, easily scared\\
\hline
\end{tabular}
\end{center}
\end{table}

\subsubsection{SDQ Conduct score}
\textbf{Variable name in data}: pos\_SDQCond\_score \\[.3cm]
The variable takes on values from 2 to 10, with higher values corresponding with better outcomes. The variable is constructed as the mean of 5 underlying variables, where each variable measures respondent's response to the corresponding questions in the table below. The underlying variables \textit{childSDQCond1}, \textit{childSDQCond3}, \textit{childSDQCond4} and \textit{childSDQCond5} take on the value of 1 if the respondent answers  ``Completely true", 2 if the answer is ``Partially true", and 3 if the answer is ``False". For \textit{childSDQCond2}, an answer of ``Completely true" corresponds with 3 and ``False" corresponds with 1. The order is reversed for this variable because the question measures a positive characteristic.

\begin{table}[H]
\setlength{\tabcolsep}{20pt}
\begin{center}
\begin{tabular}{l l}
\hline
\textbf{Variable} & \textbf{Label} \\
\hline
childSDQCond1 & Often loses temper or is in a bad mood\\
childSDQCond2 &  Generally well behaved, usually does what adults request \\
childSDQCond3 & Often fights with other children or bullies them\\
childSDQCond4 & Often lies or cheats\\
childSDQCond5 & Steals from home, school or elsewhere\\
\hline
\end{tabular}
\end{center}
\end{table}

\subsubsection{SDQ Hyper score}
\textbf{Variable name in data}: pos\_SDQHype\_score \\[.3cm]
The variable takes on values from 0 to 10, with higher values corresponding with better outcomes. The variable is constructed as the mean of 5 underlying variables, where each variable measures respondent's response to the corresponding questions in the table below. The underlying variables \textit{childSDQCond1}, \textit{childSDQCond2} and \textit{childSDQCond3} take on the value of 1 if the respondent answers  ``Completely true", 2 if the answer is ``Partially true", and 3 if the answer is ``False". For \textit{childSDQCond4} and \textit{childSDQCond5}, an answer of ``Completely true" corresponds with 3 and ``False" corresponds with 1. The order is reversed for these variables because the question measures a positive characteristic.

\begin{table}[H]
\setlength{\tabcolsep}{20pt}
\begin{center}
\begin{tabular}{l l}
\hline
\textbf{Variable} & \textbf{Label} \\
\hline
childSDQHype1 & Restless, overactive, cannot stay still for long\\
childSDQHype2 & Constantly fidgeting or squirming\\
childSDQHype3 & Easily distracted, concentration wanders\\
childSDQHype4 & Thinks things out before acting \\
childSDQHype5 & Good attention span, sees work through to end\\
\hline
\end{tabular}
\end{center}
\end{table}

\subsubsection{SDQ Hyper score}
\textbf{Variable name in data}: pos\_SDQHype\_score \\[.3cm]
The variable takes on values from 0 to 10, with higher values corresponding with better outcomes. The variable is constructed as the mean of 5 underlying variables, where each variable measures respondent's response to the corresponding questions in the table below. The underlying variables \textit{childSDQCond1}, \textit{childSDQCond2} and \textit{childSDQCond3} take on the value of 1 if the respondent answers  ``Completely true", 2 if the answer is ``Partially true", and 3 if the answer is ``False". For \textit{childSDQCond4} and \textit{childSDQCond5}, an answer of ``Completely true" corresponds with 3 and ``False" corresponds with 1. The order is reversed for these variables because the question measures a positive characteristic.

\begin{table}[H]
\setlength{\tabcolsep}{20pt}
\begin{center}
\begin{tabular}{l l}
\hline
\textbf{Variable} & \textbf{Label} \\
\hline
childSDQHype1 & Restless, overactive, cannot stay still for long\\
childSDQHype2 & Constantly fidgeting or squirming\\
childSDQHype3 & Easily distracted, concentration wanders\\
childSDQHype4 & Thinks things out before acting \\
childSDQHype5 & Good attention span, sees work through to end\\
\hline
\end{tabular}
\end{center}
\end{table}



\subsection{Satisfaction and Reciprocity binary variables}
\setlength{\tabcolsep}{20pt}

\begin{table}[H]
\begin{center}
\scalebox{0.84}{
\begin{threeparttable}
\caption{Definition of Satisfaction and Reciprocity binary variables}

\begin{tabular}{L{5.5cm} L{3cm} L{8cm}}
\hline
	\textbf{Variable} 									& \textbf{Variable name in dataset} 		& \textbf{Question that variable is based on}	\\ \hline
	\textbf{Satisfaction variables} \footnotemark																										\\
	\quad \quad Satisfied with Income				& binSatisIncome							& How satisfied are you today with your income? \\
	\quad \quad Satisfied with work					& binSatisWork							& How satisfied are you today with your work? \\
	\quad \quad Satisfied with health				& binSatisHealth							& How satisfied are you today with your health? \\
	\quad \quad Satisfied with family				& binSatisFamily							& How satisfied are you today with your family? \\[.5cm]

	\textbf{Reciprocity variables}	 \footnotemark																													\\
	\quad \quad Return favor							& reciprocity1bin							& If someone does me a favor, I am prepared to return it  \\
	\quad \quad Put someone in difficulty			& reciprocity2bin							& If someone puts me in a difficult situation, I will do the same to him/her \\
	\quad \quad Help someone kind to me			& reciprocity3bin							& I go out of my way to help somebody who has been kind to me before \\
	\quad \quad Insult back							& reciprocity4bin							& If somebody insulted me, I will insult him/her back \\
\hline
\end{tabular}

\begin{tablenotes}
		\footnotesize{
			\item [1] The satisfaction variables take on the value of 1 if the respondent answered ``very satisfied" or ``quite satisfied", and 0 if they answered ``satisfied", ``somewhat satisfied", or ``not satisfied".
			\item [2] The reciprocity variables take on value of 1 if the individual answers ``I identify very much" or `` I identify quite a lot", and 0 if they answer ``I am neutral", ``I identify little", or ``'I don't identify" to them.
		}
\end{tablenotes}

\end{threeparttable}
}
\end{center}
\end{table}