\section{Health Variables}
\label{sec:health}

\subsection{Body Mass Index (BMI)}
Body Mass Index (BMI) is available for all cohorts in the unit of kilograms of weight over square meters of height. For the cohorts of children, migrants, and adolescents, BMI is reported by the caregivers who also report their own BMI. The z-score of the measure is calculated using the distribution of BMI in the United States. The distributions of BMI are similar across cohorts and cities.

\subsection{Subjective Health Measures}
The variable ``good health" captures a subjective measure of the quality of the subject's health. The possible responses are: (1) excellent, (2) very good, (3) good, (4) moderate, and (5) poor. 

\subsection{Risky Behaviors} 

There are several variables outlining whether and how much the subjects smoke cigarettes. The variables include age of smoking first cigarette, an indicator of smoking, and the number of cigarettes the subject smokes daily. The indicator variable of being a current smoker and the variable of the number of cigarettes per day do not correspond. There are some cases in which someone is not a smoker, but smokes at least one cigarette per day. To keep the variable consistent and conservative, the indicator for smoking is changed to 1 if the number of cigarettes smoked per day is at least 1. Similarly, anybody who does not smoke according to the indicator is coded to smoke 0 cigarettes per day. This issue is also seen in the analogous variables of alcohol consumption.\footnote{The variable documenting the amount of alcohol consumed is on a monthly basis instead of a weekly, as for cigarettes.} The same solution is implemented. 

The only variable related to involvement with drugs is an indicator of having smoked marijuana. It is asked to adolescents and all cohorts of adults.

Finally, there are several variables that capture engagement in violent behavior. There is a question posed to the adolescent and all adult cohorts. Subjects indicate if they have never fought, have fought once or twice in total, fight less than once a month, 1-3 times per month, or 1-2 times per week. These responses are made into indicators for analysis. The same structure is in place for a variable indicating having driven under the influence of alcohol.









