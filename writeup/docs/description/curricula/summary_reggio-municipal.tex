
\subsection{Reggio Approach}

% overview
The Reggio Approach is a form of progressive early childhood education designed by Loris Malaguzzi, an educator influenced by educational practices and psychological theories of Ciari, Dewey, Piaget, Erikson, Vygotsky, Bronfenbrenner, Kagan, and Gardner. Malaguzzi believed that children construct knowledge: ``they develop theories and adapt them, individually and with others, while they interpret and reconstruct the world and their surroundings.'' In RE municipal schools, children are active learners---``researchers"---and valued in their community for their individual identity and unique ways of thinking. Early learning occurs within the context of language development, served by listening and participatory debate---also known as \textit{discussione}---between adults, children, and peers. Receptive and expressive language thus serves as the platform from which social-emotional skills, logic skills, and informational content is acquired. 

% history
\textbf{[AZ: I think it would be helpful to include significant events in the history of the particular school you are summarizing. Significant events (or laws) should include anything that might have contributed to spillover of quality or ideas, or changed the way children were being cared for and educated. Below are some ideas, although please fill in, correct, and/or expand!]}
The first \textit{materna} school was founded in DATE and the first \textit{asilo} school was founded in DATE. From DATE to DATE, Malaguzzi was the director of early childhood eduction in Reggio Emilia and nearby Modena. In DATE, the ORGANIZATION was founded helping early childhood practitioners learn how to implement the Reggio Approach in other areas.

% environmental factors
Reggio educators are assigned distinct specialized roles, including two teachers, a cook, the atelierista, and a pedagogista. Teachers observe, scaffold learning, engage in dialogue and \textit{discussione}.\footnote{Note: \textit{Discussione} is also encouraged by municipal \textit{scuole materne} in Bologna. In Child Care in Context 2014, Corsaro and Emiliani clarify that public debate is an integral part of every day life in Italy, and that children engage in \textit{discussione} with adults and peers from an early age. \textbf{[AZ: Please put in proper citations]}}  Teachers  document learning and development in portfolios, a collection of each child's work. Teachers, families, and children review and discuss the portfolio over the year to build awareness of self-growth and change. A Pedagogista, or educative coordinator, serves as a mentor for teachers in approximately four schools. Debate, reflection, and collaboration amongst teachers is thus embedded in the Reggio Approach.

% curricular factors
There is no set curriculum. That is, there are no institutionally set requirements for educational objectives \textbf{[AZ: Is this a good way to define curriclum?]}. Instead, children and educators collaborate to investigate a child-originated question by constructing projects. Children demonstrate their emerging knowledge of language and ideas as a system of symbols through creative visual media, aided by specially trained art instructors called \textit{atelieristas}. \textbf{[AZ: What do you mean by a ``system of symbols?"]}
