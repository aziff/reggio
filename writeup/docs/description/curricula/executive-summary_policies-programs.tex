\documentclass[11pt]{article}

\usepackage[margin=1in]{geometry}

\usepackage{adjustbox}
\usepackage{amsmath}
\usepackage{amsthm}
\usepackage{bm}
\usepackage{booktabs}
\usepackage{caption}
\usepackage{datetime}
\usepackage{graphicx}
\usepackage{hyperref}
\usepackage{longtable}
\usepackage{multirow}
\usepackage{multicol}
\usepackage[sort]{natbib}
\usepackage{pdflscape}
\usepackage{rotating}
\usepackage{setspace}
\usepackage{tabularx}
\usepackage{threeparttablex}
\usepackage{url}

\newcolumntype{L}[1]{>{\raggedright\let\newline\\\arraybackslash\hspace{0pt}}m{#1}}
\newcolumntype{C}[1]{>{\centering\let\newline\\\arraybackslash\hspace{0pt}}m{#1}}
\newcolumntype{R}[1]{>{\raggedleft\let\newline\\\arraybackslash\hspace{0pt}}m{#1}}

\doublespacing

\begin{document}

\title{Early Childhood Policies and Programs in Italy \\ Reggio Emilia, Parma, and Padova: 1950 -- 2010}
\author{Reggio Team}
\date{Original version: Monday 4$^{\text{th}}$ July, 2016 \\ Current version: \today \\ \vspace{1em} Time: \currenttime}
\maketitle

\tableofcontents

\section{Timeline: Historical Overview of Policies, Reforms, and Provision of Early Childhood Education in Italy}

In Italy, early childhood education is provided at the public level (state and municipal) and by private organizations (religious and secular). Responsibilities for the funding and provision of early childhood care and education are as follows: the state passes laws, defines educational aims, and provides the majority of funding for daycare centers to regions through the Health Ministry. Each region may pass laws regarding the organization and basic planning of centers in that region. Municipalities organize and run the schools \citep{Becchi-Ferrari_1990_Pub-Inf-Centres-Italy}. 

Municipalities are further enabled to set eligibility criteria for public childcare. Selection criteria appear to be similar across municipalities, however, the weighting of distinct family characteristics varies \citep{Del-Boca-etal_2016_CESifo-ES}. State preschools are free to all families, but charge for meals and transportation. Fees to attend municipal schools vary; about half of municipalities provide free early childcare while others are offered on a sliding scale basis.

The Catholic Church offers the majority of private religious early childhood programs; tuition is the family's responsibility, depending on income and amount of state subsidy available as decided by the municipality. Private secular early childhood programs tend to be the sole responsibility of the family \citep{Hohnerlein_2009_Paradox-Public-Preschools}.

\begin{table}[htbp]
\footnotesize
\centering
\caption{Timeline of Policies, Reforms, and Provision of Early Childhood Education and Care in Italy}\label{tab:policies}
\def\arraystretch{1.5}
\begin{longtable}{r L{45em}}
\toprule
  1831 & Italy's first private care institutions for abandoned children under 3 years of age were charitable shelters and orphanages provided by the Catholic Church. \\
  1907 & Maria Montessori establishes the \textit{Casa dei bambini} in Rome. \\
  1925 & Charitable organization Opera Nazionale Maternita e Infanzia (ONMI) funds health and childcare initiatives for poor mothers and abandoned children. \\
  1950's & The Catholic Church provides 52.7\% of early childhood programs in Italy \citep{Hohnerlein_2009_Paradox-Public-Preschools}. \\
  1957-64 & \textit{Comuni} (Municipalities) of Reggio Emilia, Parma, Bologna and Modena experiment with progressive models of early childhood education.  \\
  1963 & The Robinson Crusoe school is built by the textit{Comune} of Reggio Emilia, directed by Loris Malaguzzi \citep{Cagliari-etal-eds_2016_BOOK_Loris-Malaguzzi}. \\
  1968 &  \textbf{Law 444} mandates \textit{scuole materne} serving ages 3-6 years with an all-female staff possessing a high school education \citep{OECD_2001_Italy-Country-Note}. \\
  1969 & Malaguzzi directs the municipal school systems in Reggio Emilia and Modena \citep{Cagliari-etal-eds_2016_BOOK_Loris-Malaguzzi}. \\
  1969 & \textit{Orientamenti} outlines non-binding standards and guidelines for \textit{scuole materne}, including religious pedagogy \citep{Corsaro_1996_Early-Edu}. \\
  1971 & \textbf{Law 1044} requires regional governments to provide enough \textit{asili nido} for ages 3-36 months to meet local demand, with priority given to working mothers \citep{Saraceno_1984_Soc-Probs}. \\
  1972  & \textbf{Law 1073} provides financial support for non-state schools providing full tuition to children of poor families \citep{Corsaro_1996_Early-Edu}. \\
  1977 & Act 517 legislates inclusion for disabled children \citep{Hohnerlein_2009_Paradox-Public-Preschools}. \\
  1978 & \textbf{Law 463} mandates equivalent pay for preschool and elementary school teachers, enables males to work as early childhood teachers, reduces work hours to 30 weekly, and requires 50\% of teaching staff to have a diploma from a teacher training school \citep{Hohnerlein_2015_Development-and-Diffusion}. \\
  1986-1987 & The State runs more than 50\% of preschools, as many municipal and some private preschools are transformed for financial reason (Del Boca Rationing Italy ECE 2016). \\
  1987 & Guidelines for working hours and a maximum teacher-child ratio of 1:25 established for \textit{scuole materne}. \\
  1991 & Revised \textit{Orientamenti} for \textit{scuole materne} emphasize social, affective and cognitive development within the context of the community and family. Play, meals, and collaborative skills are key tasks of early childhood development \citep{Corsaro_1996_Early-Edu}. \\
  1994 & Reggio Emilia first allocates funds to private religious schools to enable inclusion of children with special needs, improve school environments, and train staff. \\
  %\footnote{citation needed from Word doc 2014dic sent by Pietro}  \\
  1997-98 & University degrees and supervised experience are required for teachers in \textit{asili nido} and \textit{scuole materne}, expanding teacher training from Catholic institutions to secular higher education \citep{Ghedini_2001_Ital-Natl-Policy}.  \\
  2000 &  Act n. 20 replaces the name \textit{scuola materna} with \textit{scuola dell'infanzia} to emphasize the child and the educative goal \citep{Hohnerlein_2015_Development-and-Diffusion}. \\
  2000 & Less than 20\% of \textit{scuole dell'infanzia} are provided by the Catholic Church \citep{Hohnerlein_2009_Paradox-Public-Preschools}.  \\
  2001 & Over 95\% of all Italian children aged 3 to 6 years are enrolled in \textit{scuole dell'infanzia} \citep{OECD_2001_Italy-Country-Note}. \\
  2010 & Only 13\% of Italian infants and toddlers attend \textit{asili nido} \citep{Del-Boca-etal_2016_CESifo-ES}. \\
\bottomrule
\end{longtable}
\end{table}

\section{Brief Summary of Programs}

Municipal early childhood education programs in Reggio Emilia, Parma and Padova share many programmatic and pedagogical features that appear unique when compared to American Head Start programs. They do, however, differ in certain aspects of program administration (e.g., age of entry into the asili nido, responsibilities of the pedagogista/educative coordinator, hours of operation, site staffing, eligibility for funding), environmental features (e.g., atelier, in-house kitchen, open-space design, access to outdoor play areas), and pedagogical methods (e.g., the atelierista, flexible vs pre-defined curricula, single-age vs mixed-age classrooms, family engagement, transitions or ``inserimento"). 

\subsection{Reggio Emilia Municipal Schools}

% overview
The Reggio Approach is a form of progressive early childhood education designed by Loris Malaguzzi, an educator influenced by educational practices and psychological theories of Ciari, Dewey, Piaget, Erikson, Vygotsky, Bronfenbrenner, Kagan, and Gardner. Malaguzzi believed that children construct knowledge: ``they develop theories and adapt them, individually and with others, while they interpret and reconstruct the world and their surroundings.'' In the Reggio Approach, children are thus active learners, or ``researchers." Curriculum is viewed as an ongoing, collaborative project (in contrast to a pre-defined set of learning activities) and children's developing knowledge is expressed in creative forms and documented in portfolios. In addition to pedagogy, notable features of the Reggio Approach include: the role of the pedagogista, the role of the atelierista, and an in-house kitchen used both for meal preparation and instruction. 

%Early learning is perceived within the context of receptive and expressive language development; listening and participatory debate---also known as \textit{discussione}---serves as the platform from which social-emotional skills, cognitive skills, and informational content is acquired.\footnote{Note: \textit{Discussione} may not be unique to the Reggio Approach as public debate reflects Italian culture and every day life. According to Corsaro, children engage in \textit{discussione} with both adults and peers from an early age \citep{Corsaro-Molinari_1990_Seggiolini}. 

% history
%\textbf{[AZ: I think it would be helpful to include significant events in the history of the particular school you are summarizing. Significant events (or laws) should include anything that might have contributed to spillover of quality or ideas, or changed the way children were being cared for and educated. Below are some ideas, although please fill in, correct, and/or expand!]}
The early education system in the \textit{Comune} (Municipality) of Reggio Emilia emerged from political conflict between the secular/communist left and the religious/conservative right. Malaguzzi was one of several left-wing educators within the region of Emilia Romagna inspired by Dewey's progressive model of education. Under the guidance of Malaguzzi, Reggio Emilia opened its first \textit{scuola materna} in 1963 for children aged 3-6 years.\footnote{The term ``scuola materna" was later legally replaced by ``scuola dell'infanzia;" the two terms are synonymous. We use the more current term ``scuola dell'infanzia" in this draft.} In 1965, Reggio Emilia opened the first \textit{asilo nido} for children aged 3-36 months. The Malaguzzi-designed RE municipal system thus preceded both Italy's 1968 Law 444 establishing state-run preschools and the 1971 Law 1044 mandating access to educational center-based childcare for children under 3 years of age. 

From 1963 to 1984, Malaguzzi oversaw the early childhood municipal system in Reggio Emilia. He also served as director in nearby Modena until 1974 \citep{Cagliari-etal-eds_2016_BOOK_Loris-Malaguzzi}. While Reggio Emilia's 1963 site was the first to open, the cities of Bologna, Modena, Parma, and Pistoia helped incite a ``municipal school revolution" in northern Italy \citep{Hohnerlein_2015_Development-and-Diffusion}. In 1994, the Reggio Children organization was founded to promote the international implementation of the Reggio Approach. Carla Rinaldi, the first president of Reggio Children, first served in 1970 as a pedagogista, and later as pedagogical director of the municipal early childhood system. 

\subsubsection{Asili Nido in Reggio Emilia (ages 3 months-3 years)} 

In Reggio Emilia the municipal asili nido opens at 8am. Families can choose three options for pick-up: 1 p.m. (part-day), 4 p.m. (full-time), and until 6:20 p.m. (extended day). Teachers in the asili nido generally have lower initial training and receive less pay. The atelierista is not part of the asili nido program, but in Reggio Emilia, teachers are trained by atelieristas from the schools for older children. In contrast to the Reggio Emilia scuola dell'infanza, teachers are assigned to children in single-year increments \citep{Cagliari-etal-eds_2016_BOOK_Loris-Malaguzzi,Giudici-Nicolosi_2014_Reggio-Approach}.

\subsubsection{Scuola dell'Infanzia in Reggio Emilia (ages 3-6 years)} 
% environmental or administrative factors
%KEEP THIS?... In RE, the mayor appoints a director for all municipal programs serving children from 3 months to 6 years. Other leadership includes a Director of Early Childhood Education, a Director of the Pedagogical Team, 7 pedagogisti, a curriculum specialist, and a coordinator of special education.\Citep{Edwards_etal_1993_ReggioEmilia_BOOK} 
Scuola dell'infanzia are open 5 full-time days per week September through July \citep{Giudici-Nicolosi_2014_Reggio-Approach}.\footnote{Other materials suggest Reggio programs end in June.} The educative team is assigned specialized roles. Each class is led by two full-time co-teachers, and each school site has one full-time atelierista, an instructor with a background in visual arts. Auxiliary site staff, such as cooks and janitors, are considered members of the educative staff and are included in all trainings. A \textit{pedagogista}, with a higher degree in psychology or education, is assigned to support professional development for the educative staff of approximately 4-5 schools. Each school is equipped with an in-house kitchen, an atelierista, and is notable for open-space interior planning.  

% curricular factors
Teachers remain with the same group of children for 3 consecutive years, and thus have extended time to know each child and their families. There is no predetermined ``curriculum" enacted by educators in that there are no distinct timelines or institutionally-prescribed content knowledge that educators must convey to achieve ``school readiness." In contrast, educators, children (and sometimes families) collaborate to define a question or topic, plan research, and express growing knowledge in projects that are not restricted by a timeline. Teachers observe, scaffold learning, and engage children in \textit{discussione}. Children demonstrate their emerging knowledge through creative visual media, aided by the atelierista. Teachers document each child's development in a portfolio---a collection of work---which is reviewed and discussed with children and parents over the year. 

\subsection{Parma}
%overview
Published literature documenting the Parma municipal early childhood system in any detail is scarce and varied. Carolyn Pope Edwards suggests that municipal schools in Parma are parallel to those of Reggio Emilia.\footnote{Kuperman, Interview with Carolyn Pope Edwards, 2016. See \citet{edwards1998hundred}.} 

\subsubsection{Asili Nido of Parma (ages 5 months-3 years)} 

%historical and/or administrative factors
As of 2001, 16 asili nido were offered throughout the municipality. The administration of these centers are managed by a director of services for children under 3 years of age. Pedagogical coordinators perform both administrative and professional development roles. Assigned to a specific set of asili nido, they meet twice each month with all site teachers collectively for shared reflection, on-site supervision, and to promote relationships with families. The city director meets biweekly with all pedagogical coordinators for overall planning. University professors or administrators from other municipalities provide professional development in the form of continuing education \citep{Terzi-Cantarelli_2001_Parma}.

% environmental factors
In contrast to pre-fabricated preschool centers, Parma's asili nido are intentionally designed in the context of an apartment. Terzi and Cantarelli report mixed-age classes that include 18 total children from 13-36 months in a single section, led by two teachers (9:1 child-teacher ratio). To accommodate parents, asili nido open at 7:30pm and offer 3 pick-up times: 2 p.m. (short-day), 3:30 p.m.  (normal-day), or 5 p.m. (extended-day) \citep{Terzi-Cantarelli_2001_Parma}. Other published literature suggests age of entry to the asili nido at 5 months. Classrooms can be organized by single-age groups (e.g., 5-12 months, 12-24 months, and 24-36 months) or by mixed-age groups (e.g.,12-36 months).

%\footnote{Majorano, Cigala and Corsano 2009}

\subsubsection{Municipal Scuola dell'Infanzia in Parma (ages 3-6 years)}

This section will be expanded in the next draft. It is anticipated that there is no atelierista in Parma municipal preschools. Pedagogy and administration is likely to be very similar, however, to the Reggio Approach.


%\textit{inserimento}
\subsection{Padova}
Padova is located in the relatively more religious and politically diverse region of Veneto. The University of Padua is well-regarded for its departments of psychology, education, and sociology. Compared to Reggio Emilia and Parma, its municipal early childhood education system is smaller and it has a higher number of private religious programs. 

\subsubsection{Municipal Asili Nido in Padova}
In 1989, the region of Veneto reported a total provision of childcare slots for 3.9\% of its infant-toddler population. In contrast, the region of Emilia Romagna reported a provision of infant-toddler childcare for 15.6\% of its population. The practice of professional development trainings for early childhood staff in Veneto first began in 1986. 

\subsection{Private-Religious schools}
The Catholic Church offers the majority of private religious early childhood programs; tuition is the family's responsibility, depending on income and amount of state subsidy available as decided by the municipality \citep{Hohnerlein_2009_Paradox-Public-Preschools}.  A major concern of the Catholic Church in Italy is equity and parity of state funding for non-state (private) schools.\footnote{In 2005, funding for state and private schools differed greatly. Authorized private schools received a state contribution for pre-primary schools (ages 2-6 years), while funding for private primary schools (ages 6-11 years) was at 15.5\% the rate of public primary schools. Private secondary schools (ages 11-18 years) received no state funding.}

Prior to March 2000, state funding for private schools reflected a 1947 constitutional clause that non-state schools could operate ``without financial burdens on the state.'' Private schools were thus considered options only for affluent families that could afford the tuition expense. 

%history
There is a long history of political---and pedagogical---conflict between the Catholic Church and municipalities in the Emilia Romagna region such as Reggio Emilia and Parma. Following World War II, the prevailing political party in Italy was the centrist \textit{Democrazia Cristiana}. The Catholic Church ``controlled virtually all'' early childcare centers and the institutions that provided teacher training; in 1955, reports suggest 60\% of Italian children under 6 years of age attended ``confessional \textit{scoule materne}'' \citep{Hohnerlein_2015_Development-and-Diffusion}. After 1963, the \textit{Democrazia Cristiana} lost its absolute majority, and an increasingly secular and center-left government began to assume power \citep{Hohnerlein_2009_Paradox-Public-Preschools}. In 1974, following the 1968 and 1971 state laws mandating access to educational childcare for children under age 6, the Catholic Church established the Italian Federation of Catholic Preschools (FISM) to oversee national operations of its own system of early childhood programs. Between 1981 and 1998, the number of municipal and state \textit{scuole dell'infanzia} increased; enrollment in private-religious preschools dropped from 57.7\% to 42.4\%.
 
\section{Glossary}
\begin{itemize}
	\item Private-Religious Programs (Catholic) - The Catholic Church offers the majority of private religious early
	childhood programs; tuition is the family’s responsibility, depending on income and amount
	of state subsidy available as decided by the municipality
	
\item Private-Secular Programs - Privately operated non-religious preschool programs where tuition is typically the sole responsibility of the Family.

\item State Programs - State preschools
are free to all families, but charge for meals and transportation.
\item Municipal Programs - Programs run by the Municipality. Fees to attend municipal
schools vary; about half of municipalities provide free early childcare while others are offered
on a sliding scale basis.
\item Law 444 - Enacted in 1968, mandates public \textit{scuolo materna} be operated by young, high school educated, all-female staff.
\item Law 1044 - Enacted in 1971, defines early childhood as social right of children. Regional governments  required to operate \textit{asili nido} to meet local needs of children aged 3-36 months with priorty given to working mothers. Demand exceeds supply.
\item Law 1073 - Enacted in 1972, makes non-state schools eligible for financial support if school offers free tuition/refectory rights to all or part of their students.
\item Law 463 - Enacted in 1973, mandates equivalent pay for preschool and elementary school teachers,
enables males to work as early childhood teachers, reduces work hours
to 30 weekly, and requires 50\% of teaching staff to have a diploma from a
teacher training school
\item \textit{Orientamenti} - Curriculum that outlines non-binding standards and guidelines for scuole
materne, including religious pedagogy 
\item \textit{Asilo Nido} - Programs typically intended for children ages 3-36 months.
\item \textit{pedagogista} - Educative Coordinator, assigned to 4 schools, to serve as a mentor for
teachers and support professional reflection.
\item \textit{Istituzione del Comune di Reggio Emilia} - Organized in 2003, the \textit{Istituzione} is responsible for relations between the state and Private-Religious preschools 
\item \textit{Discussione} – Participatory dialogue and debate in the classroom.
\item \textit{Atelierista} - Art Instructor with special training in visual arts, to “foster children’s
expressive languages”.

\end{itemize}

 
%%%	BIBLIOGRAPHY
\addcontentsline{toc}{section}{References}
\singlespacing
\bibliographystyle{chicago}
\bibliography{heckman}
\pagebreak

\end{document}  
