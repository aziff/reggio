\documentclass{article}

\usepackage[margin=1in]{geometry}

\begin{document}

\section{Current}

\begin{enumerate}
	\item Childhood Education Experiences in Italy

		\begin{enumerate}
		\item Brief overview of section

		\item The Reggio Approach
			\begin{itemize}
			\item Funding
			\item History
			\item Pedagogical description and general approach
			\item Educative team
			\item Physical characteristics of the schools
			\end{itemize}

		\item State Preschools
			\begin{itemize}
			\item Law 444
			\item History of state schools in Reggio Emilia and Padova \textbf{[AZ: (approximate) start date for Parma?]}
			\item Orientamenti: inspiration and mandates
			\end{itemize}

		\item Religious Preschools
			\begin{itemize}
			\item History of Catholic schools
			\item 2000s
			\item 1990s
			\end{itemize}
		\end{enumerate}

	\item Comparison of Early Childhood Systems in Reggio Emilia, Parma, and Padova
		\begin{enumerate}
		\item Overview of Historical Survey
			\begin{itemize}
			\item Motivation
			\item Who responded to the survey \textbf{[AZ: table here?]}
			\item Survey contents
			\end{itemize}
		\item Results from the Historical Survey
			\begin{itemize}
			\item Spillover evident from survey
			\item Priority for disadvantaged students in municipal programs
			\item Different investments between programs \textbf{[AZ: Is this from the survey?]}
			\end{itemize}
		\end{enumerate}
\end{enumerate}

\section{Proposed}

\begin{enumerate}
\item hi
\begin{itemize}
	\item There are similarities and differences seen between the Reggio Approach and alternatives both in Reggio Emilia and those in Parma and Padova
	\item We build on the foundation of the published literature by administering a survey to school officials and collating historical records
\end{itemize}
\end{enumerate}


\end{document}