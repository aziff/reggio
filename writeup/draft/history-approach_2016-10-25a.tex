%Style
\documentclass[12pt]{article}
\usepackage[top=1in, bottom=1in, left=1in, right=1in]{geometry}
\parindent 22pt
\usepackage{fancyhdr}

%Packages
\usepackage{adjustbox}
\usepackage{amsmath}
\usepackage{amsfonts}
\usepackage{amssymb}
\usepackage{bm}
\usepackage[table]{xcolor}
\usepackage{tabu}
\usepackage{makecell}
\usepackage{longtable}
\usepackage{multirow}
\usepackage[normalem]{ulem}
\usepackage{etoolbox}
\usepackage{graphicx}
\usepackage{tabularx}
\usepackage{ragged2e}
\usepackage{booktabs}
\usepackage{caption}
\usepackage{fixltx2e}
\usepackage[para, flushleft]{threeparttablex}
\usepackage[capposition=top]{floatrow}
\usepackage{subcaption}
\usepackage{pdfpages}
\usepackage{pdflscape}
\usepackage{natbib}
\usepackage{bibunits}
\definecolor{maroon}{HTML}{990012}
\usepackage[colorlinks=true,linkcolor=maroon,citecolor=maroon,urlcolor=maroon,anchorcolor=maroon]{hyperref}
\usepackage{marvosym}
\usepackage{makeidx}
\usepackage{tikz}
\usetikzlibrary{shapes}
\usepackage{setspace}
\usepackage{enumerate}
\usepackage{rotating}
\usepackage{epstopdf}
\usepackage[titletoc]{appendix}
\usepackage{framed}
\usepackage{comment}
\usepackage{xr}
\usepackage{titlesec}
\usepackage{footnote}
\usepackage{longtable}
\newlength{\tablewidth}
\setlength{\tablewidth}{9.3in}
\setcounter{secnumdepth}{4}

\titleformat{\paragraph}
{\normalfont\normalsize\bfseries}{\theparagraph}{1em}{}
\titlespacing*{\paragraph}
{0pt}{3.25ex plus 1ex minus .2ex}{1.5ex plus .2ex}
\makeatletter
\pretocmd\start@align
{%
  \let\everycr\CT@everycr
  \CT@start
}{}{}
\apptocmd{\endalign}{\CT@end}{}{}
\makeatother
%Watermark
\usepackage[printwatermark]{xwatermark}
\usepackage{lipsum}
\definecolor{lightgray}{RGB}{220,220,220}
%\newwatermark[allpages,color=lightgray,angle=45,scale=3,xpos=0,ypos=0]{Preliminary Draft}

%Further subsection level
\usepackage{titlesec}
\setcounter{secnumdepth}{4}
\titleformat{\paragraph}
{\normalfont\normalsize\bfseries}{\theparagraph}{1em}{}
\titlespacing*{\paragraph}
{0pt}{3.25ex plus 1ex minus .2ex}{1.5ex plus .2ex}

\setcounter{secnumdepth}{5}
\titleformat{\subparagraph}
{\normalfont\normalsize\bfseries}{\thesubparagraph}{1em}{}
\titlespacing*{\subparagraph}
{0pt}{3.25ex plus 1ex minus .2ex}{1.5ex plus .2ex}

%Functions
\DeclareMathOperator{\cov}{Cov}
\DeclareMathOperator{\var}{Var}
\DeclareMathOperator{\plim}{plim}
\DeclareMathOperator*{\argmin}{arg\,min}
\DeclareMathOperator*{\argmax}{arg\,max}

%Math Environments
\newtheorem{theorem}{Theorem}[section]
\newtheorem{claim}[theorem]{Claim}
\newtheorem{assumption}[theorem]{Assumption}
\newtheorem{definition}[theorem]{Definition}
\newtheorem{hypothesis}[theorem]{Hypothesis}
\newtheorem{property}[theorem]{Property}
\newtheorem{example}[theorem]{Example}
\newtheorem{condition}[theorem]{Condition}
\newtheorem{result}[theorem]{Result}
\newenvironment{proof}{\paragraph{Proof:}}{\hfill$\square$}

%Commands
\newcommand\independent{\protect\mathpalette{\protect\independenT}{\perp}}
\def\independenT#1#2{\mathrel{\rlap{$#1#2$}\mkern2mu{#1#2}}}
\newcommand{\overbar}[1]{\mkern 1.5mu\overline{\mkern-1.5mu#1\mkern-1.5mu}\mkern 1.5mu}
\newcommand{\equald}{\ensuremath{\overset{d}{=}}}
\captionsetup[table]{skip=10pt}
%\makeindex


\newcolumntype{L}[1]{>{\raggedright\let\newline\\\arraybackslash\hspace{0pt}}m{#1}}
\newcolumntype{C}[1]{>{\centering\let\newline\\\arraybackslash\hspace{0pt}}m{#1}}
\newcolumntype{R}[1]{>{\raggedleft\let\newline\\\arraybackslash\hspace{0pt}}m{#1}}



%Logo
%\AddToShipoutPictureBG{%
%  \AtPageUpperLeft{\raisebox{-\height}{\includegraphics[width=1.5cm]{uchicago.png}}}
%}

\newcolumntype{L}[1]{>{\raggedright\let\newline\\\arraybackslash\hspace{0pt}}m{#1}}
\newcolumntype{C}[1]{>{\centering\let\newline\\\arraybackslash\hspace{0pt}}m{#1}}
\newcolumntype{R}[1]{>{\raggedleft\let\newline\\\arraybackslash\hspace{0pt}}m{#1}} 

\newcommand{\mr}{\multirow}
\newcommand{\mc}{\multicolumn}

%\newcommand{\comment}[1]{}


\begin{document}


\subsection{Types of Early Childhood Education in Italy}

In Italy, the responsibilities for the funding and provision of early childhood care and education are as follows: the state passes laws, defines educational aims, and provides the majority of funding for schools to regions through the Health Ministry. Each region may pass laws regarding the organization and basic planning of centers in that region. Municipalities organize and run the schools \citep{Becchi-Ferrari_1990_Pub-Inf-Centres-Italy}. 

Municipalities are further enabled to set eligibility criteria for public early childhood education. Selection criteria are similar across municipalities, however, the weighting of distinct family characteristics varies \citep{Del-Boca-etal_2016_CESifo-ES}. State preschools are free to all families, but charge for meals and transportation. Fees to attend municipal schools vary; about half of municipalities provide free early childcare while others are offered on a sliding scale.

The Catholic Church offers the majority of private religious early childhood programs. Tuition is the family's responsibility, depending on income and amount of state subsidy available as decided by the municipality. Private secular early childhood programs tend to be the sole responsibility of the family \citep{Hohnerlein_2009_Paradox-Public-Preschools}.

To summarize, early childhood education is publicly provided by the municipality or the state, and privately provided by religious institutions or secular ones. \textbf{[AZ: Maybe include some table or figure tabulating attendance to these types in the data?]}

\subsection{Reggio Approach}
\textbf{[AZ: This section needs more citations. I also moved things around so please double check that you agree with everything!]}

The Reggio Approach is a form of progressive early childhood education designed by Loris Malaguzzi, an educator influenced by the educational practices and psychological theories of Ciari, Dewey, Piaget, Erikson, Vygotsky, Bronfenbrenner, Kagan, and Gardner. \textbf{[AZ: Perhaps we should briefly describe some of the theories and cite these individuals instead of listing the individuals?} The Reggio Approach emerged from political conflict between the secular and communist left and the religious and conservative right. Malaguzzi was one of several left-wing educators within the region of Emilia Romagna. Under the guidance of Malaguzzi, Reggio Emilia opened its first preschool in 1963 for children aged 3-6 years. In 1965 \textbf{[AZ: Should this be 1971?]}, Reggio Emilia opened the first infant-toddler center for children aged 3 months to 3 years. The Reggio Emilia municipal system with this progressive model preceded reforms in the 60s and 70s that established state-run infant-toddler centers and preschools.

In the Reggio Approach, preschool-aged children are active learners, or ``researchers." Curriculum is viewed as an ongoing, collaborative project, in contrast to a pre-defined set of learning activities. Children's developing knowledge is expressed in creative forms and documented in portfolios that are subsequently shared with the parents and children. 

The Reggio Approach infant-toddler center opens at 8am. Families can choose three options for pick-up: 1 p.m. (part-day), 4 p.m. (full-time), and until 6:20 p.m. (extended day). Teachers in the infant-toddler center generally have lower initial training and receive less pay. The atelierista is not part of the infant-toddler center, but in Reggio Emilia, teachers are trained by atelieristas from the preschools. In contrast to the Reggio Approach preschool, teachers are assigned to children in single-year increments \citep{Cagliari-etal-eds_2016_BOOK_Loris-Malaguzzi,Giudici-Nicolosi_2014_Reggio-Approach}.

Preschools are open five full-time days per week from September through July \citep{Giudici-Nicolosi_2014_Reggio-Approach}. The educative team is assigned specialized roles. Each class is led by two full-time co-teachers, and each school site has one full-time atelierista, an instructor with a background in visual arts. Auxiliary site staff, such as cooks and janitors, are considered members of the educative staff and are included in all trainings. A pedagogista, with a higher degree in psychology or education, is assigned to support professional development for the educative staff of approximately 4-5 schools. In addition to having an atelierista, each school is equipped with an in-house kitchen and is has an open interior design.  

Teachers remain with the same group of children for three consecutive years, and thus have extended time to know each child and their families. There is no predetermined curriculum enacted by educators in that there are no distinct timelines or institutionally-prescribed content knowledge that educators must convey to achieve ``school readiness." In contrast, educators, children (and sometimes families) collaborate to define a question or topic and pursue research that is not constrained by a predetermined timeline. Teachers observe, scaffold learning, and engage children in discussion. Children demonstrate their emerging knowledge through creative visual media, aided by the atelierista. Teachers document each child's development in a portfolio---a collection of work---which is reviewed and discussed with children and parents over the year. 

\subsection{Other Early Childhood Education Experiences}

Municipal early childhood education programs in Reggio Emilia, Parma and Padova differ in certain aspects of program administration, environmental features, and pedagogical methods. We discuss these differences below. 

\subsubsection{Parma}

Published literature documenting the Parma municipal early childhood system in detail is scarce. Carolyn Pope Edwards suggests that municipal schools in Parma are parallel to those of Reggio Emilia.\footnote{Kuperman, Interview with Carolyn Pope Edwards, 2016. See \citet{Edwards-etal-eds_1998_Hundred-Languages}.} 

As of 2001, 16 infant-toddler centers were offered throughout the municipality. The administration of these centers are managed by a director of services for children under 3 years of age. Pedagogical coordinators perform both administrative and professional development roles. Assigned to a specific set of infant-toddler centers, they meet twice each month with all site teachers collectively for shared reflection, on-site supervision, and to promote relationships with families. The city director meets biweekly with all pedagogical coordinators for overall planning. University professors or administrators from other municipalities provide professional development in the form of continuing education \citep{Terzi-Cantarelli_2001_Parma}.

In contrast to pre-fabricated preschool centers, Parma's infant-toddler centers are intentionally designed in the context of an apartment. \citet{Terzi-Cantarelli_2001_Parma} report mixed-age classes that include 18 total children from 13 months to 3 years in a single section, led by two teachers (9:1 child-teacher ratio). To accommodate parents, infant-toddler centers open at 7:30pm and offer 3 pick-up times: 2 p.m. (short-day), 3:30 p.m. (normal-day), or 5 p.m. (extended-day). Classrooms can be organized by single-age groups (e.g., 5-12 months, 12-24 months, and 24-36 months) or by mixed-age groups (e.g.,12-36 months) \citep{Majorano-etal_2009_CC-in-P}.

\textbf{[AZ: Information on preschool in Parma]}

\subsubsection{Padova}
Padova is located in the relatively more religious and politically diverse region of Veneto. Compared to Reggio Emilia and Parma, its municipal early childhood education system is smaller and it has a higher number of private religious programs. 

In 1989, the region of Veneto reported a total provision of childcare slots for 3.9\% of its infant-toddler population. In contrast, the region of Emilia Romagna reported a provision of infant-toddler childcare for 15.6\% of its population. The practice of professional development trainings for early childhood staff in Veneto first began in 1986. \textbf{[AZ: citation?]}

The Catholic Church offers the majority of private religious early childhood programs; tuition is the family's responsibility, depending on income and amount of state subsidy available as decided by the municipality \citep{Hohnerlein_2009_Paradox-Public-Preschools}.  A major concern of the Catholic Church in Italy is equity and parity of state funding for private schools.\footnote{In 2005, funding for state and private schools differed greatly. Authorized private schools received a state contribution for pre-primary schools (ages 2-6 years), while funding for private primary schools (ages 6-11 years) was at 15.5\% the rate of public primary schools. Private secondary schools (ages 11-18 years) received no state funding. \textbf{[AZ: citation?]}}

Prior to March 2000, state funding for private schools reflected a 1947 constitutional clause that non-state schools could operate ``without financial burdens on the state.'' Private schools were thus considered options only for affluent families that could afford the tuition expense. \textbf{[AZ: citation?]}

\end{document}
