\textbf{[AZ: The below paragraphs were taken from different areas, but we should develop this into a more cohesive section]}

From 1963 to 1984, Malaguzzi oversaw the early childhood municipal system in Reggio Emilia. He also served as director in nearby Modena until 1974 \citep{Cagliari-etal-eds_2016_BOOK_Loris-Malaguzzi}. In 1994, the Reggio Children organization was founded to promote the international implementation of the Reggio Approach. While Reggio Emilia's 1963 site was the first to open, the cities of Bologna, Modena, Parma, and Pistoia helped incite a ``municipal school revolution" in northern Italy \citep{Hohnerlein_2015_Development-and-Diffusion}. 

There is a long history of political---and pedagogical---conflict between the Catholic Church and municipalities in the Emilia Romagna region such as Reggio Emilia and Parma. Following World War II, the prevailing political party in Italy was the centrist \textit{Democrazia Cristiana}. The Catholic Church ``controlled virtually all'' early childcare centers and the institutions that provided teacher training; in 1955, reports suggest 60\% of Italian children under 6 years of age attended ``confessional \textit{scoule materne}'' \citep{Hohnerlein_2015_Development-and-Diffusion}. After 1963, the \textit{Democrazia Cristiana} lost its absolute majority, and an increasingly secular and center-left government began to assume power \citep{Hohnerlein_2009_Paradox-Public-Preschools}. In 1974, following the 1968 and 1971 state laws mandating access to educational childcare for children under age 6, the Catholic Church established the Italian Federation of Catholic Preschools (FISM) to oversee national operations of its own system of early childhood programs. Between 1981 and 1998, the number of municipal and state \textit{scuole dell'infanzia} increased; enrollment in private-religious preschools dropped from 57.7\% to 42.4\%.

Tables~\ref{tab:comparisonRE} through \ref{tab:comparisonPad} present information related to the curricular and programmatic elements of preschool in Reggio Emilia, Parma, and Padova during the time the cohorts would have attended the schools. Each table represents information for one city. The tables are further subdivided by school type. To understand how these school types relate to the Reggio Approach (Reggio Emilia municipal schools), ten components of the Reggio Approach are listed for each school type by city and cohort. 

The selected components of the Reggio Approach are as follows. See \citet{Rinaldi_2006_ReggioEmilia_BOOK,Giudici-Nicolosi_2014_Reggio-Approach, Cagliari-etal-eds_2016_BOOK_Loris-Malaguzzi} for more information.
\begin{description}
\item[Eligibility priority.] Priority of enrollment is given to single-parent families and children with disabilities.
\item[Teaching Staff hours] Teaching staff work 36 hours per week, of which 3 hours are set aside weekly to interact with families, document children's learning, and participate in professional development.
\item[Two Co-teachers per class of homogenous age.] Two co-teachers per classroom are assigned to incoming 3-year-olds and their families, and remain for the next three years with the same cohort.
\item[Atelierista.] Presence of an on-site, full-time atelierista, with training in visual arts. The Reggio Approach uses visual arts as a creative medium through which children both learn content and demonstrate new knowledge as ``100 Languages."
\item[Pedagogista.] The assignment of a pedagogista, with a higher degree in education or psychology and teaching experience, to mentor the educational and auxiliary staff of 4-5 school sites on a bi-weekly basis, offer ongoing professional development, and improve family-school relations.
\item[Project-based learning.] There is no curriculum with pre-determined expectations for children to acquire specific content knowledge. Instead small groups (4-6 children) engage in research projects to investigate topics that are collaboratively selected by educators, parents, and children. 
\item[Atelier and Environment.] Presence of an atelier (arts-based laboratory/library/classroom), an open floor plan, extensive natural light, and natural furnishings and decorative materials.
\item[Documentation.] Teachers document children's individual and collective growth in portfolios that are shared with parents. Portfolios are reviewed with children to demonstrate growth and development.
\item[Meals prepared in on-site kitchen.] Presence of an internal kitchen in the school site that is used both to prepare meals and for instructional purposes.
\item[Religious teaching.] Religious teaching is \textbf{not} part of the Reggio Approach. This variable is included to compare with other programs that do contain Religious components.
\end{description}

The symbols of the table are as follows: \checkmark\ indicates that the school type for the corresponding city and cohort had that component; $\times$ indicates that the school type did not have that component; $\sim$ indicates that some of the schools within that school type had that component; n/a indicates that the school type was not available for the specified cohort and city; a blank space indicates that we are still compiling information to determine the presence of that component.
\singlespacing

\newgeometry{top=.5in, bottom=.5in, left=.6in, right=.6in}
\begin{table}[htb]
\centering
\scriptsize
\begin{threeparttable}
\caption{Comparison of Different School Types to Reggio Approach, Reggio Emilia}\label{tab:comparisonRE}
	\begin{tabular}{ c l |  c  c  c c c }
\toprule
 & & \rotatebox{90}{Children} & \rotatebox{90}{Adolescents}  & \rotatebox{90}{Adults 30s} & \rotatebox{90}{Adults 40s}  & \rotatebox{90}{Adults 50s} \\
\midrule
\multirow{19}{*}{Municipal$^1$}	&	Eligibility priority	&	\checkmark	&	\checkmark	&	\checkmark	&	\checkmark	&	n/a	\\
	\cmidrule{2-7}												
	&	Staff hours/professional dev.	&	\checkmark	&	\checkmark	&		&		&	n/a	\\
	\cmidrule{2-7}												
	&	2 co-teachers/homogenous age	&	\checkmark	&	\checkmark	&	\checkmark	&	\checkmark	&	n/a	\\
	\cmidrule{2-7}												
	&	Atelierista	&	\checkmark	&	\checkmark	&	\checkmark	&	\checkmark	&	n/a	\\
	\cmidrule{2-7}												
	&	Pedagogista	&	\checkmark	&	\checkmark	&	\checkmark	&	\checkmark	&	n/a	\\
	\cmidrule{2-7}												
	&	Project-based learning	&	\checkmark	&	\checkmark	&	\checkmark	&	\checkmark	&	n/a	\\
	\cmidrule{2-7}												
	&	Atelier/Environment	&	\checkmark	&	\checkmark	&	\checkmark	&	\checkmark	&	n/a	\\
	\cmidrule{2-7}												
	&	Documentation	&	\checkmark	&	\checkmark	&	\checkmark	&	\checkmark	&	n/a	\\
	\cmidrule{2-7}												
	&	Meals prepared on-site/kitchen	&	\checkmark	&	\checkmark	&	\checkmark	&	\checkmark	&	n/a	\\
	\cmidrule{2-7}												
	&	Religious teaching	&	$\times$	&	$\times$	&	$\times$	&	$\times$	&	n/a	\\
	\midrule												
\multirow{19}{*}{Municipal-affiliated$^{2,3}$}	&	Eligibility priority	&	\checkmark	&	n/a	&	n/a	&	n/a	&	n/a	\\
	\cmidrule{2-7}												
	&	Staff hours/professional dev.	&		&	n/a	&	n/a	&	n/a	&	n/a	\\
	\cmidrule{2-7}												
	&	2 co-teachers/homogenous age	&		&	n/a	&	n/a	&	n/a	&	n/a	\\
	\cmidrule{2-7}												
	&	Atelierista	&	\checkmark	&	n/a	&	n/a	&	n/a	&	n/a	\\
	\cmidrule{2-7}												
	&	Pedagogista	&		&	n/a	&	n/a	&	n/a	&	n/a	\\
	\cmidrule{2-7}												
	&	Project-based learning	&		&	n/a	&	n/a	&	n/a	&	n/a	\\
	\cmidrule{2-7}												
	&	Atelier/Environment	&	$\sim$	&	n/a	&	n/a	&	n/a	&	n/a	\\
	\cmidrule{2-7}												
	&	Documentation	&		&	n/a	&	n/a	&	n/a	&	n/a	\\
	\cmidrule{2-7}												
	&	Meals prepared on-site/kitchen	&		&	n/a	&	n/a	&	n/a	&	n/a	\\
	\cmidrule{2-7}												
	&	Religious teaching	&		&		&		&		&	n/a	\\
	\midrule												
\multirow{19}{*}{State}	&	Eligibility priority	&	\checkmark	&		&		&	\checkmark	&	n/a	\\
	\cmidrule{2-7}												
	&	Staff hours/professional dev.	&	$\times$	&		&		&	$\times$	&	n/a	\\
	\cmidrule{2-7}												
	&	2 co-teachers/homogenous age	&	$\sim$	&		&		&		&	n/a	\\
	\cmidrule{2-7}												
	&	Atelierista	&	$\times$	&		&		&	$\times$	&	n/a	\\
	\cmidrule{2-7}												
	&	Pedagogista	&	\checkmark	&		&		&		&	n/a	\\
	\cmidrule{2-7}												
	&	Project-based learning	&	$\times$	&		&		&	$\times$	&	n/a	\\
	\cmidrule{2-7}												
	&	Atelier/Environment	&	$\times$	&		&		&		&	n/a	\\
	\cmidrule{2-7}												
	&	Documentation	&	\checkmark	&		&		&		&	n/a	\\
	\cmidrule{2-7}												
	&	Meals prepared on-site/kitchen	&	$\times$	&		&		&		&	n/a	\\
	\cmidrule{2-7}												
	&	Religious teaching	&	\checkmark	&	\checkmark	&	\checkmark	&	\checkmark	&	n/a	\\
	\midrule												
\multirow{19}{*}{Religious}	&	Eligibility priority	&	$\times$	&		&		&		&		\\
	\cmidrule{2-7}												
	&	Staff hours/professional dev.	&		&		&		&		&	$\times$	\\
	\cmidrule{2-7}												
	&	2 co-teachers/homogenous age	&	$\times$	&		&		&		&	$\times$	\\
	\cmidrule{2-7}												
	&	Atelierista	&	$\times$	&		&		&		&	$\times$	\\
	\cmidrule{2-7}												
	&	Pedagogista	&	\checkmark	&		&		&		&	$\times$	\\
	\cmidrule{2-7}												
	&	Project-based learning	&		&		&		&		&	$\times$	\\
	\cmidrule{2-7}												
	&	Atelier/Environment	&		&		&		&		&	$\times$	\\
	\cmidrule{2-7}												
	&	Documentation	&		&		&		&		&	$\times$	\\
	\cmidrule{2-7}												
	&	Meals prepared on-site/kitchen	&	$\times$	&		&		&		&	\\
	\cmidrule{2-7}												
	&	Religious teaching	&	\checkmark	&	\checkmark	&	\checkmark	&	\checkmark	&	\checkmark	\\
\bottomrule
\end{tabular}
\begin{tablenotes}
\item Note: (1) See \citet{Rinaldi_2006_ReggioEmilia_BOOK,Giudici-Nicolosi_2014_Reggio-Approach, Cagliari-etal-eds_2016_BOOK_Loris-Malaguzzi}. (2) In 1999 and 2001, the municipality of Reggio Emilia integrated two non-profit cooperatives to expand its services. These schools offer a number of childcare slots according to municipal regulations and are overseen by the municipal pedagogical team \citep{Reggio_2008_Brochure}. (3) See \citet{Giudici-Nicolosi_2014_Reggio-Approach}.
\end{tablenotes}
\end{threeparttable}
\end{table}

\begin{table}[htb]
\centering
\scriptsize
\scalebox{.8}[.8]{
\begin{threeparttable}
\caption{Comparison of Different School Types to Reggio Approach, Parma}\label{tab:comparisonPar}
	\begin{tabular}{ c l |  c  c  c c c }
\toprule
 & & \rotatebox{90}{Children} & \rotatebox{90}{Adolescents}  & \rotatebox{90}{Adults 30s} & \rotatebox{90}{Adults 40s}  & \rotatebox{90}{Adults 50s} \\
\midrule
\multirow{19}{*}{Municipal$^4$}	&	Eligibility priority	&	\checkmark	&		&		&		&	n/a	\\
	\cmidrule{2-7}												
	&	Staff hours/professional dev.	&	\checkmark	&		&		&		&	n/a	\\
	\cmidrule{2-7}												
	&	2 co-teachers/homogenous age	&	$\times$	&		&		&		&	n/a	\\
	\cmidrule{2-7}												
	&	Atelierista	&	$\times$	&		&		&		&	n/a	\\
	\cmidrule{2-7}												
	&	Pedagogista	&	\checkmark	&		&		&		&	n/a	\\
	\cmidrule{2-7}												
	&	Project-based learning	&		&		&		&		&	n/a	\\
	\cmidrule{2-7}												
	&	Atelier/Environment	&		&		&		&		&	n/a	\\
	\cmidrule{2-7}												
	&	Documentation	&	\checkmark	&		&		&		&	n/a	\\
	\cmidrule{2-7}												
	&	Meals prepared on-site/kitchen	&	$\times$	&	$\times$	&		&		&	n/a	\\
	\cmidrule{2-7}												
	&	Religious teaching	&		&		&		&		&	n/a	\\
	\midrule												
\multirow{19}{*}{Municipal participatory managed$^5$}	&	Eligibility priority	&		&		&		&		&	n/a	\\
	\cmidrule{2-7}												
													
	&	Staff hours/professional dev.	&		&		&		&		&	n/a	\\
	\cmidrule{2-7}												
	&	2 co-teachers/homogenous age	&	$\times$	&		&		&		&	n/a	\\
	\cmidrule{2-7}												
	&	Atelierista	&		&		&		&		&	n/a	\\
	\cmidrule{2-7}												
	&	Pedagogista	&		&		&		&		&	n/a	\\
	\cmidrule{2-7}												
	&	Project-based learning	&		&		&		&		&	n/a	\\
	\cmidrule{2-7}												
	&	Atelier/Environment	&		&		&		&		&	n/a	\\
	\cmidrule{2-7}												
	&	Documentation	&		&		&		&		&	n/a	\\
	\cmidrule{2-7}												
	&	Meals prepared on-site/kitchen	&	$\times$	&	$\times$	&		&		&	n/a	\\
	\cmidrule{2-7}												
	&	Religious teaching	&		&		&		&		&	n/a	\\
	\midrule												
\multirow{19}{*}{State}	&	Eligibility priority	&	\checkmark	&		&		&		&	n/a	\\
	\cmidrule{2-7}												
	&	Staff hours/professional dev.	&	$\times$	&		&		&		&	n/a	\\
	\cmidrule{2-7}												
	&	2 co-teachers/homogenous age	&		&		&		&		&	n/a	\\
	\cmidrule{2-7}												
	&	Atelierista	&	$\times$	&		&		&		&	n/a	\\
	\cmidrule{2-7}												
	&	Pedagogista	&	\checkmark	&		&		&		&	n/a	\\
	\cmidrule{2-7}												
	&	Project-based learning	&	$\times$	&		&		&		&	n/a	\\
	\cmidrule{2-7}												
	&	Atelier/Environment	&	$\times$	&		&		&		&	n/a	\\
	\cmidrule{2-7}												
	&	Documentation	&	\checkmark	&		&		&		&	n/a	\\
	\cmidrule{2-7}												
	&	Meals prepared on-site/kitchen	&	$\times$	&	$\times$	&		&		&	n/a	\\
	\cmidrule{2-7}												
	&	Religious teaching	&		&		&		&		&	n/a	\\
	\midrule												
\multirow{19}{*}{Municipal-affiliated/Private}	&	Eligibility priority	&		&		&		&		&		\\
	\cmidrule{2-7}												
	&	Staff hours/professional dev.	&		&		&		&		&		\\
	\cmidrule{2-7}												
	&	2 co-teachers/homogenous age	&		&		&		&		&		\\
	\cmidrule{2-7}												
	&	Atelierista	&		&		&		&		&		\\
	\cmidrule{2-7}												
	&	Pedagogista	&		&		&		&		&		\\
	\cmidrule{2-7}												
	&	Project-based learning	&		&		&		&		&		\\
	\cmidrule{2-7}												
	&	Atelier/Environment	&		&		&		&		&		\\
	\cmidrule{2-7}												
	&	Documentation	&		&		&		&		&		\\
	\cmidrule{2-7}												
	&	Meals prepared on-site/kitchen	&		&		&		&		&		\\
	\cmidrule{2-7}												
	&	Religious teaching	&		&		&		&		&		\\
	\midrule												
\multirow{19}{*}{Religious}	&	Eligibility priority	&	$\times$	&		&		&		&		\\
	\cmidrule{2-7}												
	&	Staff hours/professional dev.	&	\checkmark	&		&		&		&		\\
	\cmidrule{2-7}												
	&	2 co-teachers/homogenous age	&		&		&		&		&		\\
	\cmidrule{2-7}												
	&	Atelierista	&		&		&		&		&		\\
	\cmidrule{2-7}												
	&	Pedagogista	&	\checkmark	&		&		&		&		\\
	\cmidrule{2-7}												
	&	Project-based learning	&		&		&		&		&		\\
	\cmidrule{2-7}												
	&	Atelier/Environment	&		&		&		&		&		\\
	\cmidrule{2-7}												
	&	Documentation	&	\checkmark	&		&		&		&		\\
	\cmidrule{2-7}												
	&	Meals prepared on-site/kitchen	&		&		&		&		&		\\
	\cmidrule{2-7}												
	&	Religious teaching	&	\checkmark	&		&		&		&		 \\
\bottomrule
\end{tabular}
\begin{tablenotes}
\item Note: (4) See \citet{Parma_Commune_2014}. (5) Since 2003, ten municipal preschools are managed by ParmaInfanzia and ParmaZeroSei, two public companies funded by a mixture of public and private funds, each with their own team of pedagogical coordinators. Schools managed under this structure are called ``municipal participatory managed." These are different from ``muncipal-affiliated," which offer child slots in programs that align with municipal regulations.
\end{tablenotes}
\end{threeparttable}}
\end{table}
\restoregeometry

\begin{table}[H]
\centering
\scriptsize
\begin{threeparttable}
\caption{Comparison of Different School Types to Reggio Approach, Padova}\label{tab:comparisonPad}
	\begin{tabular}{ c l |  c  c  c c c }
\toprule
 & & \rotatebox{90}{Children} & \rotatebox{90}{Adolescents}  & \rotatebox{90}{Adults 30s} & \rotatebox{90}{Adults 40s}  & \rotatebox{90}{Adults 50s} \\
 \midrule
\multirow{19}{*}{Municipal$^6$}	&	Eligibility priority	&	$\times$	&		&		&		&	n/a	\\
	\cmidrule{2-7}												
	&	Staff hours/professional dev.	&		&		&		&		&	n/a	\\
	\cmidrule{2-7}												
	&	2 co-teachers/homogenous age	&	$\times$	&		&		&		&	n/a	\\
	\cmidrule{2-7}												
	&	Atelierista	&	$\times$	&		&		&		&	n/a	\\
	\cmidrule{2-7}												
	&	Pedagogista	&	$\times$	&		&		&		&	n/a	\\
	\cmidrule{2-7}												
	&	Project-based learning	&	$\times$	&		&		&		&	n/a	\\
	\cmidrule{2-7}												
	&	Atelier/Environment	&	$\times$	&		&		&		&	n/a	\\
	\cmidrule{2-7}												
	&	Documentation	&		&		&		&		&	n/a	\\
	\cmidrule{2-7}												
	&	Meals prepared on-site/kitchen	&		&		&		&		&	n/a	\\
	\cmidrule{2-7}												
	&	Religious teaching	&	\checkmark	&		&		&		&	n/a	\\
	\midrule												
\multirow{19}{*}{State$^7$}	&	Eligibility priority	&	\checkmark	&		&		&		&	n/a	\\
	\cmidrule{2-7}												
	&	Staff hours/professional dev.	&	$\times$	&		&		&		&	n/a	\\
	\cmidrule{2-7}												
	&	2 co-teachers/homogenous age	&	$\times$	&		&		&		&	n/a	\\
	\cmidrule{2-7}												
	&	Atelierista	&	$\times$	&		&		&		&	n/a	\\
	\cmidrule{2-7}												
	&	Pedagogista	&	\checkmark	&		&		&		&	n/a	\\
	\cmidrule{2-7}												
	&	Project-based learning	&	$\times$	&		&		&		&	n/a	\\
	\cmidrule{2-7}												
	&	Atelier/Environment	&	$\times$	&		&		&		&	n/a	\\
	\cmidrule{2-7}												
	&	Documentation	&	\checkmark	&		&		&		&	n/a	\\
	\cmidrule{2-7}												
	&	Meals prepared on-site/kitchen	&	\checkmark	&		&		&		&	n/a	\\
	\cmidrule{2-7}												
	&	Religious teaching	&	\checkmark	&	\checkmark	&	\checkmark	&	\checkmark	&	n/a	\\
	\midrule												
\multirow{19}{*}{Religious}	&	Eligibility priority	&		&		&		&		&		\\
	\cmidrule{2-7}												
	&	Staff hours/professional dev.	&		&		&		&		&		\\
	\cmidrule{2-7}												
	&	2 co-teachers/homogenous age	&		&		&		&		&		\\
	\cmidrule{2-7}												
	&	Atelierista	&	$\times$	&		&		&		&		\\
	\cmidrule{2-7}												
	&	Pedagogista	&		&		&		&		&		\\
	\cmidrule{2-7}												
	&	Project-based learning	&	$\times$	&		&		&		&		\\
	\cmidrule{2-7}												
	&	Atelier/Environment	&		&		&		&		&		\\
	\cmidrule{2-7}												
	&	Documentation	&		&		&		&		&		\\
	\cmidrule{2-7}												
	&	Meals prepared on-site/kitchen	&		&		&		&		&		\\
	\cmidrule{2-7}												
	&	Religious teaching	&	\checkmark	&	\checkmark	&	\checkmark	&	\checkmark	&	\checkmark	\\
\bottomrule
\end{tabular}
\begin{tablenotes}
\item Note: (6) Most of the schools have on-site meal preparation for the youngest cohort \citep{
Padova_Parma-Commune_2016}. (7) See \citet{Padova_Parma-Commune_2016}.
\end{tablenotes}
\end{threeparttable}
\end{table}
