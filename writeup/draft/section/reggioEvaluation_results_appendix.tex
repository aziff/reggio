






\subsection{Estimation Results for Reggio Approach Preschools, Comparison with Other School Types} \label{app:comparison-reli-stat}

\begin{table}[H] \caption{Estimation Results for Main Outcomes, Comparison to Religious Preschools, Child Cohort} \label{ols-M-child-reg-reli}
\scalebox{0.8}{\begin{tabular}{l c c c c c c}
\toprule
 & NoneIt & BICIt & FullIt & DidPmIt & DidPvIt & AIPWIt \\
\midrule
SDQ Composite - Child &      0.64 & \textbf{      1.03 } & \textbf{      1.29 } & \textbf{      1.26 } &     -0.73 & \textbf{     0.84} \\
& (     0.53 ) & (     0.53 ) & (     0.53 ) & (     0.79 ) & (     0.60 ) & (     0.47 ) \\
& \textit{ 315 } & \textit{ 315 } & \textit{ 315 } & \textit{ 442 } & \textit{ 577 } & \textit{ 315 } \\
Obese &      0.08 &      0.07 & \textbf{      0.10 } &      0.00 & \textbf{      0.10 } &      0.00 \\
& (     0.05 ) & (     0.06 ) & (     0.06 ) & (     0.07 ) & (     0.06 ) & (     0.06 ) \\
& \textit{ 315 } & \textit{ 315 } & \textit{ 315 } & \textit{ 442 } & \textit{ 577 } & \textit{ 315 } \\
Overweight &      0.04 &      0.05 &      0.06 &      0.06 &     -0.01 & \textbf{     0.07} \\
& (     0.04 ) & (     0.04 ) & (     0.04 ) & (     0.07 ) & (     0.04 ) & (     0.04 ) \\
& \textit{ 315 } & \textit{ 315 } & \textit{ 315 } & \textit{ 442 } & \textit{ 577 } & \textit{ 315 } \\
Health is Good &     -0.03 &     -0.03 &     -0.00 & \textbf{     -0.15 } &      0.03 &     -0.04 \\
& (     0.06 ) & (     0.06 ) & (     0.06 ) & (     0.09 ) & (     0.05 ) & (     0.06 ) \\
& \textit{ 315 } & \textit{ 315 } & \textit{ 315 } & \textit{ 442 } & \textit{ 576 } & \textit{ 315 } \\
Not Excited to Learn &      0.02 &      0.02 &      0.02 & \textbf{     -0.06 } &      0.04 & \textbf{     0.03} \\
& (     0.02 ) & (     0.02 ) & (     0.02 ) & (     0.04 ) & (     0.03 ) & (     0.01 ) \\
& \textit{ 315 } & \textit{ 315 } & \textit{ 315 } & \textit{ 442 } & \textit{ 577 } & \textit{ 315 } \\
Problems Sitting Still &     -0.01 &      0.02 &     -0.04 &      0.08 & \textbf{      0.07 } &      0.02 \\
& (     0.04 ) & (     0.04 ) & (     0.04 ) & (     0.07 ) & (     0.04 ) & (     0.05 ) \\
& \textit{ 315 } & \textit{ 315 } & \textit{ 315 } & \textit{ 442 } & \textit{ 577 } & \textit{ 315 } \\
How Much Child Likes School &      0.04 &      0.04 &      0.06 &     -0.12 &     -0.07 &      0.05 \\
& (     0.07 ) & (     0.07 ) & (     0.08 ) & (     0.09 ) & (     0.08 ) & (     0.08 ) \\
& \textit{ 314 } & \textit{ 314 } & \textit{ 314 } & \textit{ 441 } & \textit{ 576 } & \textit{ 314 } \\
\bottomrule
\end{tabular}
}
\vspace{1ex} \\
\footnotesize\raggedright{Note: This table shows the estimates of the coefficient for attending Reggio Approach preschools from multiple methods. We compare Reggio Approach people with people who attended religious preschools in Reggio. Column title indicates the corresponding control set and and model. ``None'' refers to the OLS estimate with no control variables. ``BIC'' refers to the OLS estimate with controls selected by Bayesian Information Criterion (BIC) and additional controls for caregiver's religion. ``Full'' refers to the OLS estimate with the full set of controls. ``DidPm'' refers to the difference-in-difference estimate of (Reggio Muni - Parma Muni) - (Reggio Reli - Parma Reli). ``DidPv'' refers to the difference-in-difference estimate of (Reggio Muni - Padova Muni) - (Reggio Reli - Padova Reli). ``PSM" refers to propensity score matching estimation. ``AIPW" refers to AIPW estimate for comparing Reggio Approach children with children in Reggio who attended other type of preschool. Robust standard errors are reported in parentheses. Bold number shows that the estimate is statistically significant at the 10\% level. Number of observations used in estimation is reported in italic.}

\end{table}


\begin{table}[H] \caption{Estimation Results for Main Outcomes, Comparison to State Preschools, Child Cohort} \label{ols-M-child-reg-state}
\scalebox{0.8}{\begin{tabular}{l c c c c c c}
\toprule
 & NoneIt & BICIt & FullIt & DidPmIt & DidPvIt & AIPWIt \\
\midrule
SDQ Composite - Child &      0.79 &      0.67 &      0.95 &      0.16 &      0.30 &      0.63 \\
& (     0.69 ) & (     0.66 ) & (     0.68 ) & (     0.81 ) & (     0.71 ) & (     0.60 ) \\
& \textit{ 290 } & \textit{ 290 } & \textit{ 290 } & \textit{ 392 } & \textit{ 474 } & \textit{ 290 } \\
Obese &     -0.09 &     -0.01 &     -0.05 &      0.01 &     -0.02 &      0.01 \\
& (     0.06 ) & (     0.06 ) & (     0.06 ) & (     0.08 ) & (     0.07 ) & (     0.06 ) \\
& \textit{ 290 } & \textit{ 290 } & \textit{ 290 } & \textit{ 392 } & \textit{ 474 } & \textit{ 290 } \\
Overweight & \textbf{      0.06 } &      0.03 &      0.06 &      0.10 & \textbf{     -0.09 } &      0.03 \\
& (     0.04 ) & (     0.05 ) & (     0.05 ) & (     0.07 ) & (     0.05 ) & (     0.05 ) \\
& \textit{ 290 } & \textit{ 290 } & \textit{ 290 } & \textit{ 392 } & \textit{ 474 } & \textit{ 290 } \\
Health is Good &     -0.07 &     -0.02 &     -0.00 &     -0.11 &     -0.05 &     -0.01 \\
& (     0.06 ) & (     0.07 ) & (     0.07 ) & (     0.09 ) & (     0.06 ) & (     0.07 ) \\
& \textit{ 290 } & \textit{ 290 } & \textit{ 290 } & \textit{ 392 } & \textit{ 474 } & \textit{ 290 } \\
Not Excited to Learn &     -0.04 &     -0.02 &     -0.02 &     -0.02 &      0.03 &     -0.01 \\
& (     0.03 ) & (     0.03 ) & (     0.04 ) & (     0.04 ) & (     0.03 ) & (     0.03 ) \\
& \textit{ 290 } & \textit{ 290 } & \textit{ 290 } & \textit{ 392 } & \textit{ 474 } & \textit{ 290 } \\
Problems Sitting Still &      0.05 &      0.06 &      0.04 & \textbf{      0.13 } & \textbf{      0.09 } &      0.04 \\
& (     0.04 ) & (     0.05 ) & (     0.05 ) & (     0.07 ) & (     0.04 ) & (     0.05 ) \\
& \textit{ 290 } & \textit{ 290 } & \textit{ 290 } & \textit{ 392 } & \textit{ 474 } & \textit{ 290 } \\
How Much Child Likes School &     -0.01 &     -0.08 &     -0.04 &     -0.08 &     -0.10 &     -0.03 \\
& (     0.08 ) & (     0.09 ) & (     0.09 ) & (     0.10 ) & (     0.09 ) & (     0.08 ) \\
& \textit{ 290 } & \textit{ 290 } & \textit{ 290 } & \textit{ 392 } & \textit{ 474 } & \textit{ 290 } \\
\bottomrule
\end{tabular}
}
\vspace{1ex} \\
\footnotesize\raggedright{Note: This table shows the estimates of the coefficient for attending Reggio Approach preschools from multiple methods. We compare Reggio Approach people with people who attended religious preschools in Reggio. Column title indicates the corresponding control set and and model. ``None'' refers to the OLS estimate with no control variables. ``BIC'' refers to the OLS estimate with controls selected by Bayesian Information Criterion (BIC) and additional controls for caregiver's religion. ``Full'' refers to the OLS estimate with the full set of controls. ``DidPm'' refers to the difference-in-difference estimate of (Reggio Muni - Parma Muni) - (Reggio State - Parma State). ``DidPv'' refers to the difference-in-difference estimate of (Reggio Muni - Padova Muni) - (Reggio State - Padova State). ``PSM" refers to propensity score matching estimation. ``AIPW" refers to AIPW estimate for comparing Reggio Approach children with children in Reggio who attended other type of preschool. Robust standard errors are reported in parentheses. Bold number shows that the estimate is statistically significant at the 10\% level. Number of observations used in estimation is reported in italic.}

\end{table}


\begin{table}[H] \caption{Estimation Results for Main Outcomes, Comparison to Religious Preschools, Adolescent Cohort} \label{ols-M-adol-reg-reli}
\scalebox{0.8}{\begin{tabular}{l c c c c c c c c c}
\toprule
 & None & Bic & Full & PSM & AIPW & DidPm & PSMPm & DidPv & PSMPv \\
\midrule
IQ Factor & -0.13 & \textbf{ -0.15 } & -0.07 & \textbf{-0.16} & -0.18 & -0.10 & 0.04 & -0.10 & 0.14 \\
& (0.10) & (0.10) & (0.11) & (0.10) & (0.08) & (0.15) & (0.16) & (0.18) & (0.14) \\
& \textit{ 251 } & \textit{ 251 } & \textit{ 251 } & \textit{ 251 } & \textit{ 251 } & \textit{ 433 } & \textit{ 238 } & \textit{ 467 } & \textit{ 288 } \\
SDQ Composite - Child & 0.55 & 0.82 & \textbf{ 1.25 } & 0.36 & 0.90 & -0.56 & \textbf{2.84} & 0.16 & 1.05 \\
& (0.70) & (0.79) & (0.76) & (1.03) & (0.75) & (1.05) & (1.47) & (0.91) & (0.82) \\
& \textit{ 251 } & \textit{ 251 } & \textit{ 251 } & \textit{ 251 } & \textit{ 251 } & \textit{ 433 } & \textit{ 238 } & \textit{ 463 } & \textit{ 286 } \\
SDQ Composite & \textbf{ 1.44 } & \textbf{ 1.74 } & \textbf{ 1.54 } & \textbf{1.96} & \textbf{1.68} & 0.68 & 0.78 & 0.81 & \textbf{2.16} \\
& (0.70) & (0.78) & (0.82) & (1.04) & (0.76) & (1.01) & (1.10) & (1.03) & (1.15) \\
& \textit{ 249 } & \textit{ 249 } & \textit{ 249 } & \textit{ 249 } & \textit{ 249 } & \textit{ 429 } & \textit{ 235 } & \textit{ 463 } & \textit{ 286 } \\
Depression Score - positive & \textbf{ 1.89 } & \textbf{ 2.97 } & \textbf{ 2.34 } & \textbf{2.84} & \textbf{2.88} & 1.47 & \textbf{1.93} & \textbf{ 1.87 } & 1.84 \\
& (0.88) & (0.96) & (1.02) & (1.12) & (0.87) & (1.17) & (0.93) & (1.24) & (1.15) \\
& \textit{ 245 } & \textit{ 245 } & \textit{ 245 } & \textit{ 245 } & \textit{ 245 } & \textit{ 418 } & \textit{ 230 } & \textit{ 459 } & \textit{ 283 } \\
Locus of Control - positive & 0.08 & \textbf{ 0.16 } & 0.09 & \textbf{0.23} & \textbf{0.13} & -0.04 & \textbf{0.26} & 0.19 & 0.31 \\
& (0.09) & (0.10) & (0.10) & (0.09) & (0.11) & (0.15) & (0.12) & (0.14) & (0.19) \\
& \textit{ 248 } & \textit{ 248 } & \textit{ 248 } & \textit{ 248 } & \textit{ 248 } & \textit{ 426 } & \textit{ 233 } & \textit{ 462 } & \textit{ 285 } \\
Not Obese & \textbf{ -0.11 } & \textbf{ -0.12 } & \textbf{ -0.10 } & \textbf{-0.08} & -0.11 & 0.01 & \textbf{-0.14} & -0.06 & 0.07 \\
& (0.04) & (0.05) & (0.05) & (0.05) & (0.05) & (0.06) & (0.04) & (0.07) & (0.08) \\
& \textit{ 251 } & \textit{ 251 } & \textit{ 251 } & \textit{ 251 } & \textit{ 251 } & \textit{ 433 } & \textit{ 238 } & \textit{ 467 } & \textit{ 288 } \\
Not Overweight & 0.00 & -0.02 & -0.02 & -0.04 & -0.01 & \textbf{ 0.08 } & \textbf{-0.04} & -0.03 & -0.04 \\
& (0.02) & (0.03) & (0.03) & (0.05) & (0.03) & (0.04) & (0.02) & (0.03) & (0.03) \\
& \textit{ 251 } & \textit{ 251 } & \textit{ 251 } & \textit{ 251 } & \textit{ 251 } & \textit{ 433 } & \textit{ 238 } & \textit{ 467 } & \textit{ 288 } \\
Health is Good & 0.04 & 0.05 & 0.06 & -0.03 & 0.04 & \textbf{ 0.18 } & 0.02 & 0.07 & 0.06 \\
& (0.06) & (0.07) & (0.07) & (0.07) & (0.07) & (0.10) & (0.07) & (0.09) & (0.10) \\
& \textit{ 250 } & \textit{ 250 } & \textit{ 250 } & \textit{ 250 } & \textit{ 250 } & \textit{ 432 } & \textit{ 238 } & \textit{ 466 } & \textit{ 288 } \\
Go To School & 0.03 & 0.01 & 0.02 & -0.02 & 0.00 & 0.03 & 0.00 & 0.04 & -0.04 \\
& (0.03) & (0.02) & (0.03) & (0.03) & (0.03) & (0.04) & (0.04) & (0.03) & (0.03) \\
& \textit{ 251 } & \textit{ 251 } & \textit{ 251 } & \textit{ 251 } & \textit{ 251 } & \textit{ 433 } & \textit{ 238 } & \textit{ 467 } & \textit{ 288 } \\
How Much Child Likes School & -0.09 & -0.03 & -0.12 & -0.01 & 0.00 & -0.15 & \textbf{0.44} & -0.07 & 0.05 \\
& (0.12) & (0.13) & (0.13) & (0.14) & (0.15) & (0.18) & (0.25) & (0.17) & (0.17) \\
& \textit{ 240 } & \textit{ 240 } & \textit{ 240 } & \textit{ 240 } & \textit{ 240 } & \textit{ 414 } & \textit{ 230 } & \textit{ 453 } & \textit{ 282 } \\
Days of Sport (Weekly) & -0.25 & -0.33 & -0.07 & -0.41 & -0.32 & -0.35 & \textbf{-1.01} & -0.43 & -0.38 \\
& (0.25) & (0.29) & (0.29) & (0.42) & (0.27) & (0.37) & (0.53) & (0.37) & (0.46) \\
& \textit{ 246 } & \textit{ 246 } & \textit{ 246 } & \textit{ 246 } & \textit{ 246 } & \textit{ 420 } & \textit{ 232 } & \textit{ 449 } & \textit{ 278 } \\
Num. of Friends & -0.06 & 0.02 & 0.29 & 0.08 & 0.44 & -0.77 & -2.19 & -0.28 & 0.05 \\
& (1.39) & (1.11) & (1.30) & (1.30) & (1.08) & (2.13) & (2.10) & (2.26) & (1.62) \\
& \textit{ 245 } & \textit{ 245 } & \textit{ 245 } & \textit{ 245 } & \textit{ 245 } & \textit{ 414 } & \textit{ 227 } & \textit{ 418 } & \textit{ 259 } \\
Volunteers & -0.03 & 0.03 & 0.03 & -0.02 & 0.01 & -0.01 & \textbf{0.26} & -0.03 & 0.06 \\
& (0.06) & (0.07) & (0.07) & (0.08) & (0.08) & (0.09) & (0.07) & (0.09) & (0.09) \\
& \textit{ 251 } & \textit{ 251 } & \textit{ 251 } & \textit{ 251 } & \textit{ 251 } & \textit{ 433 } & \textit{ 238 } & \textit{ 467 } & \textit{ 288 } \\
Trust Score & -0.00 & 0.10 & 0.01 & 0.23 & 0.13 & 0.10 & -0.59 & -0.07 & 0.13 \\
& (0.19) & (0.22) & (0.22) & (0.30) & (0.23) & (0.29) & (0.48) & (0.28) & (0.26) \\
& \textit{ 249 } & \textit{ 249 } & \textit{ 249 } & \textit{ 249 } & \textit{ 249 } & \textit{ 429 } & \textit{ 235 } & \textit{ 460 } & \textit{ 282 } \\
\bottomrule
\end{tabular}
}
\vspace{1ex} \\
\footnotesize\raggedright{Note: This table shows the estimates of the coefficient for attending Reggio Approach preschools from multiple methods. We compare Reggio Approach people with people who attended religious preschools in Reggio. Column title indicates the corresponding control set and and model. ``None'' refers to the OLS estimate with no control variables. ``BIC'' refers to the OLS estimate with controls selected by Bayesian Information Criterion (BIC) and additional controls for caregiver's religion. ``Full'' refers to the OLS estimate with the full set of controls. ``DidPm'' refers to the difference-in-difference estimate of (Reggio Muni - Parma Muni) - (Reggio Reli - Parma Reli). ``DidPv'' refers to the difference-in-difference estimate of (Reggio Muni - Padova Muni) - (Reggio Reli - Padova Reli). ``PSM" refers to propensity score matching estimation. ``AIPW" refers to AIPW estimate for comparing Reggio Approach children with children in Reggio who attended other type of preschool. Robust standard errors are reported in parentheses. Bold number shows that the estimate is statistically significant at the 10\% level. Number of observations used in estimation is reported in italic.}
\end{table}

\begin{table}[H] \caption{Estimation Results for Main Outcomes, Comparison to State Preschools, Adolescent Cohort} \label{ols-M-adol-reg-stat}
\scalebox{0.8}{\begin{tabular}{l c c c c c c}
\toprule
 & None & Bic & Full & AIPW & DidPm & DidPv \\
\midrule
IQ Score &     -0.02 &     -0.05 &      0.01 & \textbf{     0.13} &     -0.07 & \textbf{     -0.21 } \\
& (     0.05 ) & (     0.06 ) & (     0.06 ) & (     0.13 ) & (     0.07 ) & (     0.09 ) \\
& \textit{ 185 } & \textit{ 185 } & \textit{ 185 } & \textit{ 185 } & \textit{ 275 } & \textit{ 316 } \\
IQ Factor &     -0.07 &     -0.20 &      0.02 &      0.42 &     -0.15 & \textbf{     -0.79 } \\
& (     0.18 ) & (     0.22 ) & (     0.23 ) & (     0.48 ) & (     0.25 ) & (     0.30 ) \\
& \textit{ 185 } & \textit{ 185 } & \textit{ 185 } & \textit{ 185 } & \textit{ 275 } & \textit{ 316 } \\
SDQ Composite - Child & \textbf{     -2.09 } & \textbf{     -2.18 } & \textbf{     -1.61 } &     -0.88 &     -1.09 & \textbf{     -3.08 } \\
& (     0.62 ) & (     0.79 ) & (     0.70 ) & (     1.21 ) & (     1.17 ) & (     1.06 ) \\
& \textit{ 185 } & \textit{ 185 } & \textit{ 185 } & \textit{ 185 } & \textit{ 275 } & \textit{ 313 } \\
SDQ Composite &     -1.02 &     -1.15 &     -1.32 &     -3.29 &      0.48 &     -0.15 \\
& (     1.00 ) & (     1.03 ) & (     1.11 ) & (     1.91 ) & (     1.48 ) & (     1.32 ) \\
& \textit{ 183 } & \textit{ 183 } & \textit{ 183 } & \textit{ 183 } & \textit{ 273 } & \textit{ 312 } \\
Depression Score - positive &      0.09 &      0.64 &      0.36 &     -3.32 & \textbf{      2.90 } & \textbf{      2.92 } \\
& (     1.25 ) & (     1.38 ) & (     1.43 ) & (     3.23 ) & (     1.62 ) & (     1.66 ) \\
& \textit{ 180 } & \textit{ 180 } & \textit{ 180 } & \textit{ 180 } & \textit{ 266 } & \textit{ 308 } \\
Locus of Control - positive &     -0.14 &     -0.06 &     -0.07 &     -0.44 & \textbf{     -0.61 } &     -0.08 \\
& (     0.14 ) & (     0.16 ) & (     0.17 ) & (     0.25 ) & (     0.23 ) & (     0.20 ) \\
& \textit{ 182 } & \textit{ 182 } & \textit{ 182 } & \textit{ 182 } & \textit{ 271 } & \textit{ 311 } \\
Obese &     -0.03 &      0.08 &      0.01 &      0.01 &     -0.07 & \textbf{      0.24 } \\
& (     0.08 ) & (     0.08 ) & (     0.08 ) & (     0.18 ) & (     0.10 ) & (     0.12 ) \\
& \textit{ 185 } & \textit{ 185 } & \textit{ 185 } & \textit{ 185 } & \textit{ 275 } & \textit{ 316 } \\
Overweight &     -0.04 &      0.00 &     -0.02 &      0.06 & \textbf{     -0.14 } &     -0.00 \\
& (     0.05 ) & (     0.05 ) & (     0.05 ) & (     0.05 ) & (     0.08 ) & (     0.06 ) \\
& \textit{ 185 } & \textit{ 185 } & \textit{ 185 } & \textit{ 185 } & \textit{ 275 } & \textit{ 316 } \\
Health is Good &      0.05 &      0.07 &      0.10 &      0.03 &      0.01 & \textbf{      0.28 } \\
& (     0.10 ) & (     0.10 ) & (     0.11 ) & (     0.36 ) & (     0.15 ) & (     0.13 ) \\
& \textit{ 185 } & \textit{ 185 } & \textit{ 185 } & \textit{ 185 } & \textit{ 275 } & \textit{ 316 } \\
Go To School &      0.05 &      0.02 &      0.04 &     -0.01 &      0.04 &      0.02 \\
& (     0.05 ) & (     0.05 ) & (     0.06 ) & (     0.21 ) & (     0.05 ) & (     0.06 ) \\
& \textit{ 185 } & \textit{ 185 } & \textit{ 185 } & \textit{ 185 } & \textit{ 275 } & \textit{ 316 } \\
How Much Child Likes School &     -0.24 &     -0.29 & \textbf{     -0.41 } &      0.18 &     -0.08 &     -0.34 \\
& (     0.20 ) & (     0.22 ) & (     0.20 ) & (     0.96 ) & (     0.27 ) & (     0.26 ) \\
& \textit{ 178 } & \textit{ 178 } & \textit{ 178 } & \textit{ 178 } & \textit{ 268 } & \textit{ 305 } \\
Days of Sport (Weekly) & \textbf{     -0.87 } & \textbf{     -1.14 } & \textbf{     -0.95 } &     -1.17 & \textbf{     -1.59 } & \textbf{     -0.95 } \\
& (     0.38 ) & (     0.42 ) & (     0.43 ) & (     1.21 ) & (     0.54 ) & (     0.53 ) \\
& \textit{ 180 } & \textit{ 180 } & \textit{ 180 } & \textit{ 180 } & \textit{ 268 } & \textit{ 299 } \\
\bottomrule
\end{tabular}
}
\vspace{1ex} \\
\footnotesize\raggedright{Note: This table shows the estimates of the coefficient for attending Reggio Approach preschools from multiple methods. We compare Reggio Approach people with people who attended religious preschools in Reggio. Column title indicates the corresponding control set and and model. ``None'' refers to the OLS estimate with no control variables. ``BIC'' refers to the OLS estimate with controls selected by Bayesian Information Criterion (BIC) and additional controls for caregiver's religion. ``Full'' refers to the OLS estimate with the full set of controls. ``DidPm'' refers to the difference-in-difference estimate of (Reggio Muni - Parma Muni) - (Reggio State - Parma State). ``DidPv'' refers to the difference-in-difference estimate of (Reggio Muni - Padova Muni) - (Reggio State - Padova State). ``PSM" refers to propensity score matching estimation. ``AIPW" refers to AIPW estimate for comparing Reggio Approach children with children in Reggio who attended other type of preschool. Robust standard errors are reported in parentheses. Bold number shows that the estimate is statistically significant at the 10\% level. Number of observations used in estimation is reported in italic.}
\end{table}




\begin{table}[H] \caption{Estimation Results for Main Outcomes, Comparison to Religious Preschools, Adult 30s Cohorts} \label{ols-M-adult30-reg-reli}
\scalebox{0.75}{\begin{tabular}{l c c c c c c c c c}
\toprule
 & None & BIC & Full & PSM & AIPW & DidPm & PSMPm & DidPv & PSMPv \\
\midrule
IQ Factor & \textbf{ -0.42 } & \textbf{ -0.41 } & \textbf{ -0.36 } & \textbf{-0.46} & -0.47 & \textbf{ -0.63 } & \textbf{-0.69} & -0.19 & \textbf{-0.84} \\
& (0.15) & (0.17) & (0.18) & (0.17) & (0.17) & (0.20) & (0.12) & (0.23) & (0.12) \\
& \textit{ 144 } & \textit{ 144 } & \textit{ 144 } & \textit{ 144 } & \textit{ 144 } & \textit{ 227 } & \textit{ 151 } & \textit{ 289 } & \textit{ 236 } \\
Graduate from High School & -0.02 & -0.00 & -0.02 & -0.06 & 0.03 & 0.06 & \textbf{-0.10} & -0.10 & 0.03 \\
& (0.06) & (0.06) & (0.07) & (0.12) & (0.08) & (0.09) & (0.03) & (0.08) & (0.05) \\
& \textit{ 144 } & \textit{ 144 } & \textit{ 144 } & \textit{ 144 } & \textit{ 144 } & \textit{ 227 } & \textit{ 151 } & \textit{ 289 } & \textit{ 236 } \\
High School Grade & \textbf{ 3.11 } & 2.44 & 2.37 & \textbf{2.91} & 1.70 & 4.74 & \textbf{7.17} & 2.24 & \textbf{5.61} \\
& (1.61) & (1.69) & (1.82) & (1.42) & (1.95) & (3.47) & (2.00) & (3.84) & (2.59) \\
& \textit{ 112 } & \textit{ 112 } & \textit{ 112 } & \textit{ 112 } & \textit{ 112 } & \textit{ 186 } & \textit{ 121 } & \textit{ 229 } & \textit{ 182 } \\
High School Grade (Standardized) & \textbf{ 4.38 } & 3.29 & \textbf{ 4.05 } & \textbf{4.15} & 2.29 & \textbf{ 5.11 } & \textbf{3.01} & 3.66 & 2.15 \\
& (2.24) & (2.37) & (2.24) & (1.49) & (2.20) & (3.07) & (1.74) & (4.42) & (3.02) \\
& \textit{ 112 } & \textit{ 112 } & \textit{ 112 } & \textit{ 112 } & \textit{ 112 } & \textit{ 184 } & \textit{ 120 } & \textit{ 227 } & \textit{ 181 } \\
Max Edu: University & 0.09 & 0.08 & 0.04 & 0.07 & 0.03 & \textbf{ 0.20 } & \textbf{-0.25} & \textbf{ 0.26 } & \textbf{-0.25} \\
& (0.07) & (0.07) & (0.09) & (0.07) & (0.13) & (0.13) & (0.10) & (0.14) & (0.08) \\
& \textit{ 144 } & \textit{ 144 } & \textit{ 144 } & \textit{ 144 } & \textit{ 144 } & \textit{ 227 } & \textit{ 151 } & \textit{ 289 } & \textit{ 236 } \\
Employed & \textbf{ -0.06 } & \textbf{ -0.06 } & \textbf{ -0.04 } & \textbf{-0.06} & -0.06 & 0.08 & -0.03 & -0.09 & 0.04 \\
& (0.02) & (0.03) & (0.02) & (0.02) & (0.02) & (0.08) & (0.03) & (0.08) & (0.04) \\
& \textit{ 144 } & \textit{ 144 } & \textit{ 144 } & \textit{ 144 } & \textit{ 144 } & \textit{ 227 } & \textit{ 151 } & \textit{ 289 } & \textit{ 236 } \\
Hours Worked Per Week & -2.09 & -2.16 & -2.38 & -2.25 & -2.56 & 1.13 & 1.67 & -2.23 & 1.04 \\
& (1.62) & (1.70) & (1.74) & (1.80) & (1.65) & (4.06) & (3.05) & (3.95) & (2.78) \\
& \textit{ 125 } & \textit{ 125 } & \textit{ 125 } & \textit{ 125 } & \textit{ 125 } & \textit{ 205 } & \textit{ 131 } & \textit{ 267 } & \textit{ 214 } \\
Married or Cohabitating & 0.01 & 0.04 & 0.06 & -0.13 & -0.05 & 0.11 & -0.18 & 0.17 & \textbf{-0.17} \\
& (0.09) & (0.10) & (0.11) & (0.13) & (0.11) & (0.14) & (0.11) & (0.15) & (0.07) \\
& \textit{ 144 } & \textit{ 144 } & \textit{ 144 } & \textit{ 144 } & \textit{ 144 } & \textit{ 227 } & \textit{ 151 } & \textit{ 289 } & \textit{ 236 } \\
Not Obese & \textbf{ -0.12 } & -0.10 & -0.06 & -0.11 & -0.12 & \textbf{ -0.18 } & -0.08 & -0.12 & \textbf{-0.17} \\
& (0.07) & (0.07) & (0.08) & (0.08) & (0.10) & (0.12) & (0.09) & (0.12) & (0.07) \\
& \textit{ 144 } & \textit{ 144 } & \textit{ 144 } & \textit{ 144 } & \textit{ 144 } & \textit{ 227 } & \textit{ 151 } & \textit{ 289 } & \textit{ 236 } \\
Not Overweight & -0.01 & 0.01 & -0.02 & -0.02 & 0.02 & 0.13 & -0.02 & -0.04 & 0.03 \\
& (0.08) & (0.08) & (0.09) & (0.07) & (0.10) & (0.12) & (0.08) & (0.12) & (0.05) \\
& \textit{ 144 } & \textit{ 144 } & \textit{ 144 } & \textit{ 144 } & \textit{ 144 } & \textit{ 227 } & \textit{ 151 } & \textit{ 289 } & \textit{ 236 } \\
Locus of Control - positive & -0.11 & -0.04 & -0.12 & 0.08 & -0.03 & \textbf{ 0.49 } & -0.24 & 0.20 & \textbf{-0.41} \\
& (0.16) & (0.14) & (0.17) & (0.18) & (0.17) & (0.25) & (0.15) & (0.24) & (0.11) \\
& \textit{ 139 } & \textit{ 139 } & \textit{ 139 } & \textit{ 139 } & \textit{ 139 } & \textit{ 216 } & \textit{ 145 } & \textit{ 278 } & \textit{ 229 } \\
Depression Score - positive & \textbf{ -1.83 } & -1.01 & -1.04 & -1.57 & -0.98 & -0.65 & \textbf{-1.88} & -0.14 & \textbf{-3.69} \\
& (1.08) & (0.84) & (0.92) & (1.50) & (1.12) & (1.38) & (0.83) & (1.73) & (0.70) \\
& \textit{ 142 } & \textit{ 142 } & \textit{ 142 } & \textit{ 142 } & \textit{ 142 } & \textit{ 225 } & \textit{ 150 } & \textit{ 285 } & \textit{ 234 } \\
Ever Voted for Municipal & \textbf{ -0.16 } & -0.03 & -0.04 & 0.03 & 0.04 & -0.03 & \textbf{0.16} & \textbf{ 0.21 } & -0.07 \\
& (0.10) & (0.07) & (0.09) & (0.06) & (0.07) & (0.11) & (0.09) & (0.12) & (0.07) \\
& \textit{ 142 } & \textit{ 142 } & \textit{ 142 } & \textit{ 142 } & \textit{ 142 } & \textit{ 224 } & \textit{ 150 } & \textit{ 277 } & \textit{ 228 } \\
Ever Voted for Regional & \textbf{ -0.16 } & -0.04 & -0.06 & 0.00 & -0.00 & -0.03 & \textbf{0.21} & \textbf{ 0.29 } & -0.07 \\
& (0.10) & (0.07) & (0.09) & (0.05) & (0.07) & (0.10) & (0.07) & (0.12) & (0.08) \\
& \textit{ 142 } & \textit{ 142 } & \textit{ 142 } & \textit{ 142 } & \textit{ 142 } & \textit{ 224 } & \textit{ 150 } & \textit{ 277 } & \textit{ 228 } \\
\bottomrule
\end{tabular}
}
\vspace{1ex} \\
\footnotesize\raggedright{Note: This table shows the estimates of the coefficient for attending Reggio Approach preschools from multiple methods. We compare Reggio Approach people with people who attended religious preschools in Reggio. Column title indicates the corresponding control set and and model. ``None'' refers to the OLS estimate with no control variables. ``BIC'' refers to the OLS estimate with controls selected by Bayesian Information Criterion (BIC) and additional controls for caregiver's religion. ``Full'' refers to the OLS estimate with the full set of controls. ``DidPm'' refers to the difference-in-difference estimate of (Reggio Muni - Parma Muni) - (Reggio Reli - Parma Reli). ``DidPv'' refers to the difference-in-difference estimate of (Reggio Muni - Padova Muni) - (Reggio Reli - Padova Reli). ``PSM" refers to propensity score matching estimation. ``AIPW" refers to AIPW estimate for comparing Reggio Approach children with children in Reggio who attended other type of preschool. Robust standard errors are reported in parentheses. Bold number shows that the estimate is statistically significant at the 10\% level. Number of observations used in estimation is reported in italic.}
\end{table}

\begin{table}[H] \caption{Estimation Results for Main Outcomes, Comparison to State Preschools, Adult 30s Cohorts} \label{ols-M-adult30-reg-stat}
\scalebox{0.75}{\begin{tabular}{l c c c c c c c c c}
\toprule
 & None & BIC & Full & PSM & AIPW & DidPm & PSMPm & DidPv & PSMPv \\
\midrule
IQ Factor & \textbf{ 0.64 } & \textbf{ 0.45 } & \textbf{ 0.44 } & 0.29 & \textbf{0.55} & 0.16 & \textbf{-0.49} & 0.38 & \textbf{-0.71} \\
& (0.23) & (0.18) & (0.19) & (0.18) & (0.21) & (0.26) & (0.15) & (0.33) & (0.13) \\
& \textit{ 133 } & \textit{ 133 } & \textit{ 133 } & \textit{ 133 } & \textit{ 133 } & \textit{ 218 } & \textit{ 153 } & \textit{ 173 } & \textit{ 132 } \\
Graduate from High School & \textbf{ -0.14 } & \textbf{ -0.10 } & \textbf{ -0.11 } & \textbf{-0.12} & -0.13 & -0.00 & 0.07 & \textbf{ -0.14 } & -0.10 \\
& (0.03) & (0.04) & (0.05) & (0.03) & (0.03) & (0.08) & (0.09) & (0.09) & (0.06) \\
& \textit{ 133 } & \textit{ 133 } & \textit{ 133 } & \textit{ 133 } & \textit{ 133 } & \textit{ 218 } & \textit{ 153 } & \textit{ 173 } & \textit{ 132 } \\
High School Grade & -2.59 & -2.08 & -1.64 & 1.64 & -0.29 & -1.08 & \textbf{5.76} & -1.40 & \\
& (2.36) & (2.25) & (2.51) & (1.90) & (1.98) & (4.74) & (2.77) & (5.21) & () \\
& \textit{ 99 } & \textit{ 99 } & \textit{ 99 } & \textit{ 99 } & \textit{ 99 } & \textit{ 173 } & \textit{ 121 } & \textit{ 129 } & \\
High School Grade (Standardized) & -0.23 & -0.04 & -1.14 & \textbf{4.54} & 2.05 & 2.06 & 0.47 & -0.60 & \\
& (2.59) & (2.50) & (2.81) & (1.93) & (2.04) & (3.65) & (2.33) & (5.25) & () \\
& \textit{ 98 } & \textit{ 98 } & \textit{ 98 } & \textit{ 98 } & \textit{ 98 } & \textit{ 169 } & \textit{ 119 } & \textit{ 127 } & \\
Max Edu: University & -0.10 & -0.07 & -0.02 & -0.00 & -0.14 & -0.08 & -0.12 & 0.10 & -0.15 \\
& (0.10) & (0.11) & (0.11) & (0.09) & (0.13) & (0.15) & (0.08) & (0.19) & (0.22) \\
& \textit{ 133 } & \textit{ 133 } & \textit{ 133 } & \textit{ 133 } & \textit{ 133 } & \textit{ 218 } & \textit{ 153 } & \textit{ 173 } & \textit{ 132 } \\
Employed & 0.02 & 0.01 & -0.02 & 0.01 & 0.06 & \textbf{ 0.16 } & \textbf{-0.05} & 0.11 & \textbf{-0.06} \\
& (0.06) & (0.07) & (0.07) & (0.08) & (0.10) & (0.09) & (0.03) & (0.10) & (0.02) \\
& \textit{ 133 } & \textit{ 133 } & \textit{ 133 } & \textit{ 133 } & \textit{ 133 } & \textit{ 218 } & \textit{ 153 } & \textit{ 173 } & \textit{ 132 } \\
Hours Worked Per Week & 5.65 & \textbf{ 6.06 } & 5.20 & \textbf{11.40} & \textbf{11.54} & \textbf{ 10.18 } & -2.13 & \textbf{ 8.42 } & -2.31 \\
& (4.47) & (4.16) & (4.13) & (4.38) & (5.23) & (4.96) & (2.28) & (5.29) & (2.99) \\
& \textit{ 103 } & \textit{ 103 } & \textit{ 103 } & \textit{ 103 } & \textit{ 103 } & \textit{ 186 } & \textit{ 134 } & \textit{ 142 } & \textit{ 112 } \\
Married or Cohabitating & \textbf{ 0.16 } & 0.07 & 0.02 & 0.02 & 0.08 & 0.18 & -0.03 & 0.10 & -0.22 \\
& (0.09) & (0.09) & (0.10) & (0.13) & (0.13) & (0.14) & (0.12) & (0.19) & (0.23) \\
& \textit{ 133 } & \textit{ 133 } & \textit{ 133 } & \textit{ 133 } & \textit{ 133 } & \textit{ 218 } & \textit{ 153 } & \textit{ 173 } & \textit{ 132 } \\
Not Obese & \textbf{ 0.21 } & \textbf{ 0.11 } & 0.10 & 0.02 & \textbf{0.13} & 0.15 & \textbf{-0.24} & 0.21 & \textbf{-0.27} \\
& (0.11) & (0.07) & (0.07) & (0.06) & (0.06) & (0.11) & (0.05) & (0.15) & (0.05) \\
& \textit{ 133 } & \textit{ 133 } & \textit{ 133 } & \textit{ 133 } & \textit{ 133 } & \textit{ 218 } & \textit{ 153 } & \textit{ 173 } & \textit{ 132 } \\
Not Overweight & -0.12 & -0.02 & 0.01 & -0.05 & -0.13 & 0.04 & -0.03 & 0.01 & \textbf{-0.12} \\
& (0.08) & (0.09) & (0.09) & (0.11) & (0.09) & (0.13) & (0.10) & (0.14) & (0.07) \\
& \textit{ 133 } & \textit{ 133 } & \textit{ 133 } & \textit{ 133 } & \textit{ 133 } & \textit{ 218 } & \textit{ 153 } & \textit{ 173 } & \textit{ 132 } \\
Locus of Control - positive & \textbf{ 0.40 } & \textbf{ 0.22 } & 0.22 & \textbf{0.28} & \textbf{0.28} & 0.30 & \textbf{0.29} & 0.00 & 0.27 \\
& (0.15) & (0.14) & (0.15) & (0.15) & (0.16) & (0.26) & (0.17) & (0.29) & (0.33) \\
& \textit{ 130 } & \textit{ 130 } & \textit{ 130 } & \textit{ 130 } & \textit{ 130 } & \textit{ 210 } & \textit{ 148 } & \textit{ 168 } & \textit{ 129 } \\
Depression Score - positive & \textbf{ 3.04 } & 1.11 & 1.34 & -0.39 & 0.88 & \textbf{ 2.91 } & \textbf{-4.74} & -1.29 & \textbf{-1.69} \\
& (1.58) & (0.97) & (1.02) & (0.83) & (1.23) & (1.76) & (1.42) & (2.27) & (0.90) \\
& \textit{ 132 } & \textit{ 132 } & \textit{ 132 } & \textit{ 132 } & \textit{ 132 } & \textit{ 217 } & \textit{ 152 } & \textit{ 171 } & \textit{ 131 } \\
Ever Voted for Municipal & 0.07 & -0.01 & 0.04 & -0.08 & -0.01 & -0.07 & 0.12 & \textbf{ 0.43 } & -0.18 \\
& (0.11) & (0.09) & (0.09) & (0.09) & (0.11) & (0.12) & (0.10) & (0.16) & (0.13) \\
& \textit{ 132 } & \textit{ 132 } & \textit{ 132 } & \textit{ 132 } & \textit{ 132 } & \textit{ 215 } & \textit{ 151 } & \textit{ 169 } & \textit{ 131 } \\
Ever Voted for Regional & -0.02 & -0.10 & -0.04 & -0.15 & -0.11 & -0.09 & 0.14 & \textbf{ 0.41 } & -0.16 \\
& (0.12) & (0.10) & (0.10) & (0.11) & (0.12) & (0.13) & (0.11) & (0.16) & (0.14) \\
& \textit{ 132 } & \textit{ 132 } & \textit{ 132 } & \textit{ 132 } & \textit{ 132 } & \textit{ 215 } & \textit{ 151 } & \textit{ 169 } & \textit{ 131 } \\
\bottomrule
\end{tabular}
}
\vspace{1ex} \\
\footnotesize\raggedright{Note: This table shows the estimates of the coefficient for attending Reggio Approach preschools from multiple methods. We compare Reggio Approach people with people who attended religious preschools in Reggio. Column title indicates the corresponding control set and and model. ``None'' refers to the OLS estimate with no control variables. ``BIC'' refers to the OLS estimate with controls selected by Bayesian Information Criterion (BIC) and additional controls for caregiver's religion. ``Full'' refers to the OLS estimate with the full set of controls. ``DidPm'' refers to the difference-in-difference estimate of (Reggio Muni - Parma Muni) - (Reggio State - Parma State). ``DidPv'' refers to the difference-in-difference estimate of (Reggio Muni - Padova Muni) - (Reggio State - Padova State). ``PSM" refers to propensity score matching estimation. ``AIPW" refers to AIPW estimate for comparing Reggio Approach children with children in Reggio who attended other type of preschool. Robust standard errors are reported in parentheses. Bold number shows that the estimate is statistically significant at the 10\% level. Number of observations used in estimation is reported in italic..}
\end{table}




\begin{table}[H] \caption{Estimation Results for Main Outcomes, Comparison to Religious Preschools, Adult 40s Cohorts} \label{ols-M-adult40-reg-reli}
\scalebox{0.75}{\begin{tabular}{l c c c c}
\toprule
 & None & BIC & Full & AIPW \\
\midrule
IQ Score & \textbf{     -0.13 } & \textbf{     -0.12 } & \textbf{     -0.12 } &     -0.13 \\
& (     0.05 ) & (     0.05 ) & (     0.05 ) & (     0.05 ) \\
& \textit{ 135 } & \textit{ 135 } & \textit{ 135 } & \textit{ 135 } \\
IQ Factor & \textbf{     -0.29 } & \textbf{     -0.25 } & \textbf{     -0.25 } &     -0.29 \\
& (     0.13 ) & (     0.12 ) & (     0.13 ) & (     0.14 ) \\
& \textit{ 135 } & \textit{ 135 } & \textit{ 135 } & \textit{ 135 } \\
Graduate from High School &      0.09 &      0.08 &      0.09 &      0.07 \\
& (     0.08 ) & (     0.08 ) & (     0.08 ) & (     0.08 ) \\
& \textit{ 135 } & \textit{ 135 } & \textit{ 135 } & \textit{ 135 } \\
High School Grade &      0.21 &      1.15 &      1.03 &      1.58 \\
& (     1.73 ) & (     1.82 ) & (     1.92 ) & (     1.97 ) \\
& \textit{ 104 } & \textit{ 104 } & \textit{ 104 } & \textit{ 104 } \\
Max Edu: University &      0.08 &      0.06 &      0.04 &      0.04 \\
& (     0.06 ) & (     0.06 ) & (     0.06 ) & (     0.06 ) \\
& \textit{ 135 } & \textit{ 135 } & \textit{ 135 } & \textit{ 135 } \\
Employed &     -0.01 &      0.00 &     -0.01 &      0.02 \\
& (     0.03 ) & (     0.04 ) & (     0.04 ) & (     0.05 ) \\
& \textit{ 135 } & \textit{ 135 } & \textit{ 135 } & \textit{ 135 } \\
Hours Worked Per Week &     -1.89 &     -1.98 &     -2.42 &     -1.51 \\
& (     1.91 ) & (     2.24 ) & (     2.23 ) & (     2.05 ) \\
& \textit{ 123 } & \textit{ 123 } & \textit{ 123 } & \textit{ 123 } \\
Married or Cohabitating &      0.01 &      0.02 &      0.02 &      0.01 \\
& (     0.08 ) & (     0.08 ) & (     0.08 ) & (     0.09 ) \\
& \textit{ 135 } & \textit{ 135 } & \textit{ 135 } & \textit{ 135 } \\
Obese &      0.10 &      0.09 &      0.04 &      0.09 \\
& (     0.08 ) & (     0.07 ) & (     0.08 ) & (     0.09 ) \\
& \textit{ 135 } & \textit{ 135 } & \textit{ 135 } & \textit{ 135 } \\
Overweight &     -0.07 &     -0.06 &     -0.04 &     -0.03 \\
& (     0.09 ) & (     0.08 ) & (     0.08 ) & (     0.09 ) \\
& \textit{ 135 } & \textit{ 135 } & \textit{ 135 } & \textit{ 135 } \\
Locus of Control - positive &      0.01 &      0.01 &     -0.04 &     -0.01 \\
& (     0.14 ) & (     0.15 ) & (     0.16 ) & (     0.17 ) \\
& \textit{ 132 } & \textit{ 132 } & \textit{ 132 } & \textit{ 132 } \\
Depression Score - positive &      0.26 &      1.03 &      0.69 &      1.00 \\
& (     0.93 ) & (     0.90 ) & (     0.97 ) & (     1.01 ) \\
& \textit{ 133 } & \textit{ 133 } & \textit{ 133 } & \textit{ 133 } \\
Ever Voted for Municipal &      0.00 &      0.11 &      0.11 & \textbf{     0.12} \\
& (     0.09 ) & (     0.08 ) & (     0.08 ) & (     0.07 ) \\
& \textit{ 128 } & \textit{ 128 } & \textit{ 128 } & \textit{ 128 } \\
Ever Voted for Regional &      0.04 & \textbf{      0.15 } & \textbf{      0.13 } & \textbf{     0.15} \\
& (     0.09 ) & (     0.08 ) & (     0.08 ) & (     0.08 ) \\
& \textit{ 128 } & \textit{ 128 } & \textit{ 128 } & \textit{ 128 } \\
\bottomrule
\end{tabular}
}
\vspace{1ex} \\
\footnotesize\raggedright{Note: This table shows the estimates of the coefficient for attending Reggio Approach preschools from multiple methods. We compare Reggio Approach people with people who attended religious preschools in Reggio. Column title indicates the corresponding control set and and model. ``None'' refers to the OLS estimate with no control variables. ``BIC'' refers to the OLS estimate with controls selected by Bayesian Information Criterion (BIC) and additional controls for caregiver's religion. ``Full'' refers to the OLS estimate with the full set of controls. ``DidPm'' refers to the difference-in-difference estimate of (Reggio Muni - Parma Muni) - (Reggio Reli - Parma Reli). ``DidPv'' refers to the difference-in-difference estimate of (Reggio Muni - Padova Muni) - (Reggio Reli - Padova Reli). ``PSM" refers to propensity score matching estimation. ``AIPW" refers to AIPW estimate for comparing Reggio Approach children with children in Reggio who attended other type of preschool. Robust standard errors are reported in parentheses. Bold number shows that the estimate is statistically significant at the 10\% level. Number of observations used in estimation is reported in italic.}
\end{table}








\subsection{Estimation Results for Preschools in Parma}

% ========================================================================= %
% CHILD COHORT



\begin{table}[H] \caption{Estimation Results for Main Outcomes, Preschool vs. No Preschool, Adult 30s Cohort in Parma} \label{ols-M-adult30-reg-pres-parma}
\scalebox{0.7}{\begin{tabular}{l c c c c c}
\toprule
 & None & BIC & Full & PSM & AIPW \\
\midrule
IQ Score & \textbf{      0.06 } & \textbf{      0.07 } & \textbf{      0.07 } &      0.02 & \textbf{     0.10} \\
& (     0.04 ) & (     0.04 ) & (     0.03 ) & (     0.03 ) & (     0.04 ) \\
& \textit{ 187 } & \textit{ 187 } & \textit{ 187 } & \textit{ 187 } & \textit{ 187 } \\
IQ Factor & \textbf{      0.16 } & \textbf{      0.20 } & \textbf{      0.18 } &      0.08 & \textbf{     0.24} \\
& (     0.11 ) & (     0.11 ) & (     0.10 ) & (     0.09 ) & (     0.05 ) \\
& \textit{ 187 } & \textit{ 187 } & \textit{ 187 } & \textit{ 187 } & \textit{ 187 } \\
Graduate from High School &      0.04 &     -0.02 &     -0.04 &     -0.05 &     -0.15 \\
& (     0.06 ) & (     0.06 ) & (     0.06 ) & (     0.04 ) & (     0.15 ) \\
& \textit{ 187 } & \textit{ 187 } & \textit{ 187 } & \textit{ 187 } & \textit{ 187 } \\
High School Grade & \textbf{      7.56 } & \textbf{      7.92 } & \textbf{      7.40 } &      4.08 & \textbf{     4.60} \\
& (     3.33 ) & (     2.75 ) & (     2.84 ) & (     3.05 ) & (     2.95 ) \\
& \textit{ 164 } & \textit{ 164 } & \textit{ 164 } & \textit{ 164 } & \textit{ 164 } \\
High School Grade (Standardized) & \textbf{      4.75 } & \textbf{      5.22 } & \textbf{      4.98 } &      2.53 &      2.06 \\
& (     2.55 ) & (     2.14 ) & (     2.22 ) & (     1.66 ) & (     2.39 ) \\
& \textit{ 159 } & \textit{ 159 } & \textit{ 159 } & \textit{ 159 } & \textit{ 159 } \\
Max Edu: University &      0.05 &     -0.04 &     -0.07 &     -0.01 &     -0.48 \\
& (     0.08 ) & (     0.08 ) & (     0.08 ) & (     0.08 ) & (     0.13 ) \\
& \textit{ 187 } & \textit{ 187 } & \textit{ 187 } & \textit{ 187 } & \textit{ 187 } \\
Employed &     -0.03 &     -0.02 &     -0.03 &     -0.04 &      0.04 \\
& (     0.05 ) & (     0.05 ) & (     0.04 ) & (     0.05 ) & (     0.13 ) \\
& \textit{ 187 } & \textit{ 187 } & \textit{ 187 } & \textit{ 187 } & \textit{ 187 } \\
Hours Worked Per Week &     -2.10 &     -1.43 &     -1.91 &     -2.12 &     -5.91 \\
& (     2.33 ) & (     2.31 ) & (     2.26 ) & (     2.44 ) & (     6.85 ) \\
& \textit{ 184 } & \textit{ 184 } & \textit{ 184 } & \textit{ 184 } & \textit{ 184 } \\
Married or Cohabitating &      0.04 &      0.06 &      0.08 &      0.04 &     -0.25 \\
& (     0.08 ) & (     0.09 ) & (     0.09 ) & (     0.11 ) & (     0.18 ) \\
& \textit{ 187 } & \textit{ 187 } & \textit{ 187 } & \textit{ 187 } & \textit{ 187 } \\
Obese & \textbf{      0.08 } & \textbf{      0.12 } & \textbf{      0.12 } & \textbf{     0.11} &      0.01 \\
& (     0.05 ) & (     0.06 ) & (     0.06 ) & (     0.05 ) & (     0.15 ) \\
& \textit{ 187 } & \textit{ 187 } & \textit{ 187 } & \textit{ 187 } & \textit{ 187 } \\
Overweight &     -0.10 &     -0.08 &     -0.08 &     -0.01 &     -0.10 \\
& (     0.08 ) & (     0.08 ) & (     0.08 ) & (     0.08 ) & (     0.13 ) \\
& \textit{ 187 } & \textit{ 187 } & \textit{ 187 } & \textit{ 187 } & \textit{ 187 } \\
Locus of Control - positive & \textbf{      0.36 } & \textbf{      0.34 } & \textbf{      0.30 } & \textbf{     0.33} &     -0.16 \\
& (     0.16 ) & (     0.15 ) & (     0.15 ) & (     0.14 ) & (     0.57 ) \\
& \textit{ 172 } & \textit{ 172 } & \textit{ 172 } & \textit{ 172 } & \textit{ 172 } \\
Depression Score - positive &      0.51 &     -0.13 &     -0.31 &      0.74 &     -3.72 \\
& (     0.98 ) & (     1.03 ) & (     1.02 ) & (     1.10 ) & (     1.62 ) \\
& \textit{ 187 } & \textit{ 187 } & \textit{ 187 } & \textit{ 187 } & \textit{ 187 } \\
Ever Voted for Municipal & \textbf{      0.15 } & \textbf{      0.10 } & \textbf{      0.08 } &      0.06 &     -0.21 \\
& (     0.06 ) & (     0.05 ) & (     0.05 ) & (     0.04 ) & (     0.04 ) \\
& \textit{ 185 } & \textit{ 185 } & \textit{ 185 } & \textit{ 185 } & \textit{ 185 } \\
Ever Voted for Regional & \textbf{      0.13 } & \textbf{      0.08 } &      0.06 &      0.03 &     -0.20 \\
& (     0.06 ) & (     0.05 ) & (     0.05 ) & (     0.04 ) & (     0.05 ) \\
& \textit{ 185 } & \textit{ 185 } & \textit{ 185 } & \textit{ 185 } & \textit{ 185 } \\
\bottomrule
\end{tabular}
}
\vspace{1ex} \\
\footnotesize\raggedright{Note: This table shows the estimates of the coefficient for attending preschools in Parma from multiple methods. We compare people in Parma who attended preschools with people in Parma who attended no preschools. Column title indicates the corresponding control set and and model. ``None'' refers to the OLS estimate with no control variables. ``BIC'' refers to the OLS estimate with controls selected by Bayesian Information Criterion (BIC) and additional controls for caregiver's religion. ``Full'' refers to the OLS estimate with the full set of controls. ``PSM'' refers to propensity score matching estimation. ``AIPW'' refers to the augmented inverse propensity weighting estimation. Robust standard errors are reported in parentheses. Bold number shows that the estimate is statistically significant at the 10\% level. Number of observations used in estimation is reported in italic.}

\end{table}




\begin{table}[H] \caption{Estimation Results for Main Outcomes, Preschool vs. No Preschool, Adult 40s Cohort in Parma} \label{ols-M-adult40-reg-pres-parma}
\scalebox{0.7}{\begin{tabular}{l c c c c c}
\toprule
 & None & BIC & Full & PSM & AIPW \\
\midrule
IQ Score &     -0.03 &     -0.03 &     -0.02 &     -0.02 &      0.00 \\
& (     0.03 ) & (     0.03 ) & (     0.03 ) & (     0.03 ) & (     0.00 ) \\
& \textit{ 222 } & \textit{ 222 } & \textit{ 222 } & \textit{ 222 } & \textit{ 222 } \\
IQ Factor &     -0.10 &     -0.10 &     -0.07 &     -0.05 &      0.00 \\
& (     0.08 ) & (     0.08 ) & (     0.08 ) & (     0.09 ) & (     0.00 ) \\
& \textit{ 222 } & \textit{ 222 } & \textit{ 222 } & \textit{ 222 } & \textit{ 222 } \\
Graduate from High School &      0.02 &      0.02 &     -0.03 &     -0.02 &      0.00 \\
& (     0.05 ) & (     0.05 ) & (     0.05 ) & (     0.05 ) & (     0.00 ) \\
& \textit{ 222 } & \textit{ 222 } & \textit{ 222 } & \textit{ 222 } & \textit{ 222 } \\
High School Grade & \textbf{      7.99 } & \textbf{      7.99 } & \textbf{      4.49 } &      3.32 &      0.00 \\
& (     1.91 ) & (     1.91 ) & (     1.85 ) & (     3.61 ) & (     0.00 ) \\
& \textit{ 186 } & \textit{ 186 } & \textit{ 186 } & \textit{ 186 } & \textit{ 186 } \\
High School Grade (Standardized) & \textbf{      4.48 } & \textbf{      4.48 } & \textbf{      2.70 } & \textbf{     3.68} &      0.00 \\
& (     1.31 ) & (     1.31 ) & (     1.26 ) & (     1.28 ) & (     0.00 ) \\
& \textit{ 180 } & \textit{ 180 } & \textit{ 180 } & \textit{ 180 } & \textit{ 180 } \\
Max Edu: University & \textbf{      0.24 } & \textbf{      0.24 } & \textbf{      0.19 } & \textbf{     0.19} &      0.00 \\
& (     0.06 ) & (     0.06 ) & (     0.06 ) & (     0.06 ) & (     0.00 ) \\
& \textit{ 222 } & \textit{ 222 } & \textit{ 222 } & \textit{ 222 } & \textit{ 222 } \\
Employed &      0.01 &      0.01 &      0.02 &      0.03 &      0.00 \\
& (     0.03 ) & (     0.03 ) & (     0.03 ) & (     0.03 ) & (     0.00 ) \\
& \textit{ 222 } & \textit{ 222 } & \textit{ 222 } & \textit{ 222 } & \textit{ 222 } \\
Hours Worked Per Week &      1.92 &      1.92 &      1.57 &      1.39 &      0.00 \\
& (     1.43 ) & (     1.43 ) & (     1.60 ) & (     1.61 ) & (     0.00 ) \\
& \textit{ 215 } & \textit{ 215 } & \textit{ 215 } & \textit{ 215 } & \textit{ 215 } \\
Married or Cohabitating & \textbf{      0.12 } & \textbf{      0.12 } & \textbf{      0.13 } &      0.11 &      0.00 \\
& (     0.07 ) & (     0.07 ) & (     0.07 ) & (     0.07 ) & (     0.00 ) \\
& \textit{ 222 } & \textit{ 222 } & \textit{ 222 } & \textit{ 222 } & \textit{ 222 } \\
Obese & \textbf{      0.14 } & \textbf{      0.14 } & \textbf{      0.17 } & \textbf{     0.14} &      0.00 \\
& (     0.05 ) & (     0.05 ) & (     0.05 ) & (     0.05 ) & (     0.00 ) \\
& \textit{ 222 } & \textit{ 222 } & \textit{ 222 } & \textit{ 222 } & \textit{ 222 } \\
Overweight &     -0.03 &     -0.03 &     -0.01 &     -0.01 &      0.00 \\
& (     0.06 ) & (     0.06 ) & (     0.06 ) & (     0.07 ) & (     0.00 ) \\
& \textit{ 222 } & \textit{ 222 } & \textit{ 222 } & \textit{ 222 } & \textit{ 222 } \\
Locus of Control - positive &      0.10 &      0.10 &      0.06 &      0.01 &      0.00 \\
& (     0.12 ) & (     0.12 ) & (     0.12 ) & (     0.12 ) & (     0.00 ) \\
& \textit{ 209 } & \textit{ 209 } & \textit{ 209 } & \textit{ 209 } & \textit{ 209 } \\
Depression Score - positive & \textbf{      2.20 } & \textbf{      2.20 } & \textbf{      1.93 } & \textbf{     1.90} &      0.00 \\
& (     0.70 ) & (     0.70 ) & (     0.68 ) & (     0.70 ) & (     0.00 ) \\
& \textit{ 222 } & \textit{ 222 } & \textit{ 222 } & \textit{ 222 } & \textit{ 222 } \\
Ever Voted for Municipal & \textbf{      0.30 } & \textbf{      0.30 } & \textbf{      0.18 } & \textbf{     0.20} &      0.00 \\
& (     0.06 ) & (     0.06 ) & (     0.06 ) & (     0.07 ) & (     0.00 ) \\
& \textit{ 222 } & \textit{ 222 } & \textit{ 222 } & \textit{ 222 } & \textit{ 222 } \\
Ever Voted for Regional & \textbf{      0.23 } & \textbf{      0.23 } & \textbf{      0.14 } & \textbf{     0.13} &      0.00 \\
& (     0.05 ) & (     0.05 ) & (     0.05 ) & (     0.07 ) & (     0.00 ) \\
& \textit{ 222 } & \textit{ 222 } & \textit{ 222 } & \textit{ 222 } & \textit{ 222 } \\
\bottomrule
\end{tabular}
}
\vspace{1ex} \\
\footnotesize\raggedright{Note: This table shows the estimates of the coefficient for attending preschools in Parma from multiple methods. We compare people in Parma who attended preschools with people in Parma who attended no preschools. Column title indicates the corresponding control set and and model. ``None'' refers to the OLS estimate with no control variables. ``BIC'' refers to the OLS estimate with controls selected by Bayesian Information Criterion (BIC) and additional controls for caregiver's religion. ``Full'' refers to the OLS estimate with the full set of controls. ``PSM'' refers to propensity score matching estimation. ``AIPW'' refers to the augmented inverse propensity weighting estimation. Robust standard errors are reported in parentheses. Bold number shows that the estimate is statistically significant at the 10\% level. Number of observations used in estimation is reported in italic.}

\end{table}




\subsection{Estimation Results for Preschools in Padova}

% ========================================================================= %
% CHILD COHORT



\begin{table}[H] \caption{Estimation Results for Main Outcomes, Preschool vs. No Preschool, Adult 30s Cohort in Padova} \label{ols-M-adult30-reg-pres-padova}
\scalebox{0.7}{\begin{tabular}{l c c c c c}
\toprule
 & None & BIC & Full & PSM & AIPW \\
\midrule
IQ Score & \textbf{      0.10 } & \textbf{      0.10 } & \textbf{      0.10 } & \textbf{     0.10} &      0.00 \\
& (     0.04 ) & (     0.04 ) & (     0.03 ) & (     0.03 ) & (     0.00 ) \\
& \textit{ 219 } & \textit{ 219 } & \textit{ 219 } & \textit{ 219 } & \textit{ 219 } \\
IQ Factor & \textbf{      0.25 } & \textbf{      0.28 } & \textbf{      0.25 } & \textbf{     0.27} &      0.00 \\
& (     0.12 ) & (     0.11 ) & (     0.10 ) & (     0.09 ) & (     0.00 ) \\
& \textit{ 219 } & \textit{ 219 } & \textit{ 219 } & \textit{ 219 } & \textit{ 219 } \\
Graduate from High School &      0.01 &      0.00 &     -0.01 &      0.05 &      0.00 \\
& (     0.05 ) & (     0.05 ) & (     0.05 ) & (     0.06 ) & (     0.00 ) \\
& \textit{ 219 } & \textit{ 219 } & \textit{ 219 } & \textit{ 219 } & \textit{ 219 } \\
High School Grade &      0.33 &      0.34 &      1.15 &      2.24 &      0.00 \\
& (     2.02 ) & (     2.07 ) & (     1.98 ) & (     1.54 ) & (     0.00 ) \\
& \textit{ 171 } & \textit{ 171 } & \textit{ 171 } & \textit{ 171 } & \textit{ 171 } \\
High School Grade (Standardized) &     -0.49 &     -0.52 &      0.27 &      0.23 &      0.00 \\
& (     1.97 ) & (     2.11 ) & (     2.07 ) & (     1.97 ) & (     0.00 ) \\
& \textit{ 169 } & \textit{ 169 } & \textit{ 169 } & \textit{ 169 } & \textit{ 169 } \\
Max Edu: University & \textbf{      0.25 } & \textbf{      0.24 } & \textbf{      0.24 } & \textbf{     0.24} &      0.00 \\
& (     0.07 ) & (     0.07 ) & (     0.08 ) & (     0.07 ) & (     0.00 ) \\
& \textit{ 219 } & \textit{ 219 } & \textit{ 219 } & \textit{ 219 } & \textit{ 219 } \\
Employed &     -0.03 &     -0.01 &      0.01 &      0.10 &      0.00 \\
& (     0.05 ) & (     0.05 ) & (     0.05 ) & (     0.11 ) & (     0.00 ) \\
& \textit{ 219 } & \textit{ 219 } & \textit{ 219 } & \textit{ 219 } & \textit{ 219 } \\
Hours Worked Per Week &     -0.16 &      1.33 &      1.74 &      4.28 &      0.00 \\
& (     2.23 ) & (     2.34 ) & (     2.45 ) & (     4.36 ) & (     0.00 ) \\
& \textit{ 215 } & \textit{ 215 } & \textit{ 215 } & \textit{ 215 } & \textit{ 215 } \\
Married or Cohabitating &      0.05 &      0.08 &      0.10 &      0.15 &      0.00 \\
& (     0.08 ) & (     0.08 ) & (     0.08 ) & (     0.13 ) & (     0.00 ) \\
& \textit{ 219 } & \textit{ 219 } & \textit{ 219 } & \textit{ 219 } & \textit{ 219 } \\
Obese & \textbf{     -0.18 } & \textbf{     -0.18 } & \textbf{     -0.17 } & \textbf{    -0.21} &      0.00 \\
& (     0.08 ) & (     0.08 ) & (     0.08 ) & (     0.11 ) & (     0.00 ) \\
& \textit{ 219 } & \textit{ 219 } & \textit{ 219 } & \textit{ 219 } & \textit{ 219 } \\
Overweight &     -0.02 &      0.05 &      0.01 &      0.07 &      0.00 \\
& (     0.07 ) & (     0.07 ) & (     0.07 ) & (     0.11 ) & (     0.00 ) \\
& \textit{ 219 } & \textit{ 219 } & \textit{ 219 } & \textit{ 219 } & \textit{ 219 } \\
Locus of Control - positive &      0.08 &      0.14 &      0.20 &      0.18 &      0.00 \\
& (     0.14 ) & (     0.13 ) & (     0.14 ) & (     0.19 ) & (     0.00 ) \\
& \textit{ 206 } & \textit{ 206 } & \textit{ 206 } & \textit{ 206 } & \textit{ 206 } \\
Depression Score - positive &      1.27 & \textbf{      1.97 } & \textbf{      2.11 } & \textbf{     3.21} &      0.00 \\
& (     0.92 ) & (     0.90 ) & (     0.94 ) & (     1.39 ) & (     0.00 ) \\
& \textit{ 216 } & \textit{ 216 } & \textit{ 216 } & \textit{ 216 } & \textit{ 216 } \\
Ever Voted for Municipal & \textbf{      0.32 } & \textbf{      0.30 } & \textbf{      0.31 } & \textbf{     0.33} &      0.00 \\
& (     0.07 ) & (     0.07 ) & (     0.07 ) & (     0.07 ) & (     0.00 ) \\
& \textit{ 205 } & \textit{ 205 } & \textit{ 205 } & \textit{ 205 } & \textit{ 205 } \\
Ever Voted for Regional & \textbf{      0.29 } & \textbf{      0.27 } & \textbf{      0.27 } & \textbf{     0.32} &      0.00 \\
& (     0.07 ) & (     0.07 ) & (     0.07 ) & (     0.07 ) & (     0.00 ) \\
& \textit{ 205 } & \textit{ 205 } & \textit{ 205 } & \textit{ 205 } & \textit{ 205 } \\
\bottomrule
\end{tabular}
}
\vspace{1ex} \\
\footnotesize\raggedright{Note: This table shows the estimates of the coefficient for attending preschools in Padova from multiple methods. We compare people in Padova who attended preschools with people in Padova who attended no preschools. Column title indicates the corresponding control set and and model. ``None'' refers to the OLS estimate with no control variables. ``BIC'' refers to the OLS estimate with controls selected by Bayesian Information Criterion (BIC) and additional controls for caregiver's religion. ``Full'' refers to the OLS estimate with the full set of controls. ``PSM'' refers to propensity score matching estimation. ``AIPW'' refers to the augmented inverse propensity weighting estimation. Robust standard errors are reported in parentheses. Bold number shows that the estimate is statistically significant at the 10\% level. Number of observations used in estimation is reported in italic.}

\end{table}




\begin{table}[H] \caption{Estimation Results for Main Outcomes, Preschool vs. No Preschool, Adult 40s Cohort in Padova} \label{ols-M-adult40-reg-pres-padova}
\scalebox{0.7}{\begin{tabular}{l c c c c c}
\toprule
 & None & BIC & Full & PSM & AIPW \\
\midrule
IQ Score &     -0.01 &     -0.01 &      0.01 &      0.00 &      0.00 \\
& (     0.03 ) & (     0.03 ) & (     0.02 ) & (     0.02 ) & (     0.00 ) \\
& \textit{ 224 } & \textit{ 224 } & \textit{ 224 } & \textit{ 224 } & \textit{ 224 } \\
IQ Factor & \textbf{     -0.14 } & \textbf{     -0.14 } &     -0.06 &     -0.09 &      0.00 \\
& (     0.07 ) & (     0.07 ) & (     0.07 ) & (     0.06 ) & (     0.00 ) \\
& \textit{ 224 } & \textit{ 224 } & \textit{ 224 } & \textit{ 224 } & \textit{ 224 } \\
Graduate from High School &      0.04 &      0.04 & \textbf{      0.09 } &      0.03 &      0.00 \\
& (     0.06 ) & (     0.06 ) & (     0.05 ) & (     0.07 ) & (     0.00 ) \\
& \textit{ 224 } & \textit{ 224 } & \textit{ 224 } & \textit{ 224 } & \textit{ 224 } \\
High School Grade &     -1.61 &     -1.61 &     -0.93 &     -1.72 &      0.00 \\
& (     1.72 ) & (     1.72 ) & (     1.82 ) & (     1.99 ) & (     0.00 ) \\
& \textit{ 178 } & \textit{ 178 } & \textit{ 178 } & \textit{ 178 } & \textit{ 178 } \\
High School Grade (Standardized) &     -2.03 &     -2.03 &     -1.56 &     -2.15 &      0.00 \\
& (     1.81 ) & (     1.81 ) & (     1.99 ) & (     2.18 ) & (     0.00 ) \\
& \textit{ 178 } & \textit{ 178 } & \textit{ 178 } & \textit{ 178 } & \textit{ 178 } \\
Max Edu: University &      0.06 &      0.06 & \textbf{      0.12 } & \textbf{     0.13} &      0.00 \\
& (     0.07 ) & (     0.07 ) & (     0.06 ) & (     0.06 ) & (     0.00 ) \\
& \textit{ 224 } & \textit{ 224 } & \textit{ 224 } & \textit{ 224 } & \textit{ 224 } \\
Employed &     -0.03 &     -0.03 &     -0.03 &     -0.04 &      0.00 \\
& (     0.04 ) & (     0.04 ) & (     0.04 ) & (     0.04 ) & (     0.00 ) \\
& \textit{ 224 } & \textit{ 224 } & \textit{ 224 } & \textit{ 224 } & \textit{ 224 } \\
Hours Worked Per Week &     -2.19 &     -2.19 &     -0.51 &     -1.32 &      0.00 \\
& (     1.90 ) & (     1.90 ) & (     1.93 ) & (     1.89 ) & (     0.00 ) \\
& \textit{ 223 } & \textit{ 223 } & \textit{ 223 } & \textit{ 223 } & \textit{ 223 } \\
Married or Cohabitating & \textbf{      0.13 } & \textbf{      0.13 } & \textbf{      0.15 } &      0.11 &      0.00 \\
& (     0.07 ) & (     0.07 ) & (     0.07 ) & (     0.08 ) & (     0.00 ) \\
& \textit{ 224 } & \textit{ 224 } & \textit{ 224 } & \textit{ 224 } & \textit{ 224 } \\
Obese &     -0.05 &     -0.05 &     -0.07 &     -0.05 &      0.00 \\
& (     0.06 ) & (     0.06 ) & (     0.06 ) & (     0.07 ) & (     0.00 ) \\
& \textit{ 224 } & \textit{ 224 } & \textit{ 224 } & \textit{ 224 } & \textit{ 224 } \\
Overweight &     -0.08 &     -0.08 &     -0.07 &     -0.04 &      0.00 \\
& (     0.06 ) & (     0.06 ) & (     0.05 ) & (     0.07 ) & (     0.00 ) \\
& \textit{ 224 } & \textit{ 224 } & \textit{ 224 } & \textit{ 224 } & \textit{ 224 } \\
Locus of Control - positive & \textbf{     -0.23 } & \textbf{     -0.23 } &     -0.16 &     -0.16 &      0.00 \\
& (     0.11 ) & (     0.11 ) & (     0.12 ) & (     0.12 ) & (     0.00 ) \\
& \textit{ 211 } & \textit{ 211 } & \textit{ 211 } & \textit{ 211 } & \textit{ 211 } \\
Depression Score - positive &     -0.21 &     -0.21 &      0.16 &     -0.11 &      0.00 \\
& (     0.72 ) & (     0.72 ) & (     0.75 ) & (     0.78 ) & (     0.00 ) \\
& \textit{ 222 } & \textit{ 222 } & \textit{ 222 } & \textit{ 222 } & \textit{ 222 } \\
Ever Voted for Municipal & \textbf{      0.30 } & \textbf{      0.30 } & \textbf{      0.25 } & \textbf{     0.28} &      0.00 \\
& (     0.06 ) & (     0.06 ) & (     0.06 ) & (     0.08 ) & (     0.00 ) \\
& \textit{ 203 } & \textit{ 203 } & \textit{ 203 } & \textit{ 203 } & \textit{ 203 } \\
Ever Voted for Regional & \textbf{      0.29 } & \textbf{      0.29 } & \textbf{      0.26 } & \textbf{     0.24} &      0.00 \\
& (     0.06 ) & (     0.06 ) & (     0.06 ) & (     0.07 ) & (     0.00 ) \\
& \textit{ 203 } & \textit{ 203 } & \textit{ 203 } & \textit{ 203 } & \textit{ 203 } \\
\bottomrule
\end{tabular}
}
\vspace{1ex} \\
\footnotesize\raggedright{Note: This table shows the estimates of the coefficient for attending preschools in Padova from multiple methods. We compare people in Padova who attended preschools with people in Padova who attended no preschools. Column title indicates the corresponding control set and and model. ``None'' refers to the OLS estimate with no control variables. ``BIC'' refers to the OLS estimate with controls selected by Bayesian Information Criterion (BIC) and additional controls for caregiver's religion. ``Full'' refers to the OLS estimate with the full set of controls. ``PSM'' refers to propensity score matching estimation. ``AIPW'' refers to the augmented inverse propensity weighting estimation. Robust standard errors are reported in parentheses. Bold number shows that the estimate is statistically significant at the 10\% level. Number of observations used in estimation is reported in italic.}

\end{table}















\subsection{Dropping Questionable Interviewers}

% ========================================================================= %
% CHILD COHORT


\begin{table}[H] \caption{Estimation Results for Main Outcomes, Comparison to Preschools, Child Cohort, Dropping Questionnable Interviewers} \label{ols-M-child-reg-pres-dropint}
\scalebox{0.8}{\begin{tabular}{l c c c c c c}
\toprule
 & NoneIt & BICIt & FullIt & DidPmIt & DidPvIt \\
\midrule
SDQ Composite - Child & \textbf{      1.08 } & \textbf{      1.17 } & \textbf{      1.47 } &      0.47 &     -0.28 \\
& (     0.50 ) & (     0.49 ) & (     0.50 ) & (     0.90 ) & (     0.58 ) \\
& \textit{ 346 } & \textit{ 346 } & \textit{ 346 } & \textit{ 503 } & \textit{ 726 } \\
Obese &      0.01 &      0.05 &      0.06 &      0.03 &      0.07 \\
& (     0.05 ) & (     0.05 ) & (     0.05 ) & (     0.09 ) & (     0.06 ) \\
& \textit{ 347 } & \textit{ 347 } & \textit{ 347 } & \textit{ 504 } & \textit{ 727 } \\
Overweight & \textbf{      0.06 } &      0.05 & \textbf{      0.07 } &      0.05 &     -0.04 \\
& (     0.04 ) & (     0.04 ) & (     0.04 ) & (     0.08 ) & (     0.04 ) \\
& \textit{ 347 } & \textit{ 347 } & \textit{ 347 } & \textit{ 504 } & \textit{ 727 } \\
Health is Good &     -0.05 &     -0.02 &     -0.01 &     -0.05 &      0.00 \\
& (     0.05 ) & (     0.05 ) & (     0.05 ) & (     0.10 ) & (     0.05 ) \\
& \textit{ 346 } & \textit{ 346 } & \textit{ 346 } & \textit{ 503 } & \textit{ 725 } \\
Not Excited to Learn &     -0.01 &     -0.00 &     -0.01 &     -0.01 &      0.03 \\
& (     0.02 ) & (     0.02 ) & (     0.02 ) & (     0.03 ) & (     0.03 ) \\
& \textit{ 347 } & \textit{ 347 } & \textit{ 347 } & \textit{ 504 } & \textit{ 727 } \\
Problems Sitting Still &      0.04 &      0.05 &      0.03 &      0.08 &      0.05 \\
& (     0.04 ) & (     0.04 ) & (     0.04 ) & (     0.08 ) & (     0.04 ) \\
& \textit{ 347 } & \textit{ 347 } & \textit{ 347 } & \textit{ 504 } & \textit{ 727 } \\
How Much Child Likes School &      0.05 &      0.04 &      0.08 &     -0.03 &     -0.08 \\
& (     0.07 ) & (     0.07 ) & (     0.07 ) & (     0.10 ) & (     0.07 ) \\
& \textit{ 345 } & \textit{ 345 } & \textit{ 345 } & \textit{ 500 } & \textit{ 725 } \\
\bottomrule
\end{tabular}
}
\vspace{1ex} \\
\footnotesize\raggedright{Note: This table shows the estimates of the coefficient for attending Reggio Approach preschools from multiple methods. We compare Reggio Approach people with people who attended other preschools. Column title indicates the corresponding control set and model. ``NoneIt'' refers to the OLS estimate with no control variables. ``BICIt'' refers to the OLS estimate with controls selected by Bayesian Information Criterion (BIC) and additional controls for caregiver's religion. ``FullIt'' refers to the OLS estimate with the full set of controls. ``DidPmIt'' refers to the difference-in-difference estimate of (Reggio Muni - Parma Muni) - (Reggio None - Parma None). ``DidPvIt'' refers to the difference-in-difference estimate of (Reggio Muni - Padova Muni) - (Reggio None - Padova None). Robust standard errors are reported in parentheses. Bold number shows that the estimate is statistically significant at the 10\% level. Number of observations used in estimation is reported in italic.}

\end{table}




\begin{table}[H] \caption{Estimation Results for Main Outcomes, Comparison to Preschools, Adolescent Cohort, Dropping Questionnable Interviewers} \label{ols-M-adol-reg-pres-dropint}
\scalebox{0.8}{\begin{tabular}{l c c c c c}
\toprule
 & None & Bic & Full & DidPm & DidPv \\
\midrule
IQ Score &     -0.03 &     -0.06 &      0.00 &     -0.02 &     -0.06 \\
& (     0.04 ) & (     0.04 ) & (     0.04 ) & (     0.06 ) & (     0.05 ) \\
& \textit{ 225 } & \textit{ 225 } & \textit{ 225 } & \textit{ 362 } & \textit{ 499 } \\
IQ Factor &     -0.11 & \textbf{     -0.22 } &     -0.01 &     -0.02 &     -0.22 \\
& (     0.12 ) & (     0.14 ) & (     0.14 ) & (     0.21 ) & (     0.19 ) \\
& \textit{ 225 } & \textit{ 225 } & \textit{ 225 } & \textit{ 362 } & \textit{ 499 } \\
SDQ Composite - Child &      0.51 &      0.36 &      0.95 &      0.23 &     -0.03 \\
& (     0.71 ) & (     0.88 ) & (     0.76 ) & (     1.34 ) & (     0.91 ) \\
& \textit{ 225 } & \textit{ 225 } & \textit{ 225 } & \textit{ 362 } & \textit{ 494 } \\
SDQ Composite &      0.97 &      1.03 &      0.63 &      1.13 &      0.84 \\
& (     0.69 ) & (     0.81 ) & (     0.79 ) & (     1.23 ) & (     0.99 ) \\
& \textit{ 223 } & \textit{ 223 } & \textit{ 223 } & \textit{ 358 } & \textit{ 495 } \\
Depression Score - positive &      1.23 & \textbf{      1.92 } & \textbf{      1.79 } & \textbf{      2.84 } & \textbf{      1.89 } \\
& (     0.88 ) & (     0.98 ) & (     0.99 ) & (     1.40 ) & (     1.22 ) \\
& \textit{ 218 } & \textit{ 218 } & \textit{ 218 } & \textit{ 344 } & \textit{ 488 } \\
Locus of Control - positive &     -0.02 &      0.01 &     -0.01 &      0.01 &      0.08 \\
& (     0.10 ) & (     0.11 ) & (     0.11 ) & (     0.17 ) & (     0.14 ) \\
& \textit{ 222 } & \textit{ 222 } & \textit{ 222 } & \textit{ 354 } & \textit{ 494 } \\
Obese &      0.06 & \textbf{      0.11 } &      0.06 &      0.00 &      0.08 \\
& (     0.05 ) & (     0.06 ) & (     0.05 ) & (     0.09 ) & (     0.08 ) \\
& \textit{ 225 } & \textit{ 225 } & \textit{ 225 } & \textit{ 362 } & \textit{ 499 } \\
Overweight &      0.01 &      0.02 &      0.01 &     -0.04 &      0.03 \\
& (     0.02 ) & (     0.04 ) & (     0.03 ) & (     0.07 ) & (     0.03 ) \\
& \textit{ 225 } & \textit{ 225 } & \textit{ 225 } & \textit{ 362 } & \textit{ 499 } \\
Health is Good &      0.07 & \textbf{      0.10 } & \textbf{      0.10 } & \textbf{      0.29 } & \textbf{      0.15 } \\
& (     0.07 ) & (     0.07 ) & (     0.07 ) & (     0.12 ) & (     0.09 ) \\
& \textit{ 224 } & \textit{ 224 } & \textit{ 224 } & \textit{ 361 } & \textit{ 498 } \\
Go To School &      0.04 &      0.01 &      0.04 &     -0.01 &      0.05 \\
& (     0.03 ) & (     0.03 ) & (     0.03 ) & (     0.04 ) & (     0.04 ) \\
& \textit{ 225 } & \textit{ 225 } & \textit{ 225 } & \textit{ 362 } & \textit{ 499 } \\
How Much Child Likes School &     -0.05 &     -0.05 &     -0.14 &     -0.06 &     -0.02 \\
& (     0.13 ) & (     0.15 ) & (     0.14 ) & (     0.21 ) & (     0.17 ) \\
& \textit{ 212 } & \textit{ 212 } & \textit{ 212 } & \textit{ 342 } & \textit{ 481 } \\
Days of Sport (Weekly) & \textbf{     -0.48 } & \textbf{     -0.49 } &     -0.34 & \textbf{     -0.93 } & \textbf{     -0.64 } \\
& (     0.25 ) & (     0.30 ) & (     0.29 ) & (     0.45 ) & (     0.36 ) \\
& \textit{ 219 } & \textit{ 219 } & \textit{ 219 } & \textit{ 351 } & \textit{ 474 } \\
\bottomrule
\end{tabular}
}
\vspace{1ex} \\
\footnotesize\raggedright{Note: This table shows the estimates of the coefficient for attending Reggio Approach preschools from multiple methods. We compare Reggio Approach people with people who attended other preschools. Column title indicates the corresponding control set and model. ``None'' refers to the OLS estimate with no control variables. ``BIC'' refers to the OLS estimate with controls selected by Bayesian Information Criterion (BIC) and additional controls for caregiver's religion. ``Full'' refers to the OLS estimate with the full set of controls. ``DidPm'' refers to the difference-in-difference estimate of (Reggio Muni - Parma Muni) - (Reggio None - Parma None). ``DidPv'' refers to the difference-in-difference estimate of (Reggio Muni - Padova Muni) - (Reggio None - Padova None). Robust standard errors are reported in parentheses. Bold number shows that the estimate is statistically significant at the 10\% level. Number of observations used in estimation is reported in italic.}
\end{table}



\begin{table}[H] \caption{Estimation Results for Main Outcomes, Comparison to Preschools, Adult 30s Cohort, Dropping Questionnable Interviewers} \label{ols-M-adult30-reg-pres-dropint}
\scalebox{0.75}{\begin{tabular}{l c c c c c}
\toprule
 & None30 & BIC30 & Full30 & DidPm30 & DidPv30 \\
\midrule
IQ Score &      0.01 &      0.03 &     -0.01 &      0.03 &      0.09 \\
& (     0.06 ) & (     0.05 ) & (     0.06 ) & (     0.07 ) & (     0.08 ) \\
& \textit{ 74 } & \textit{ 74 } & \textit{ 74 } & \textit{ 182 } & \textit{ 240 } \\
IQ Factor &      0.05 &      0.08 &     -0.03 &      0.09 &      0.23 \\
& (     0.14 ) & (     0.13 ) & (     0.14 ) & (     0.17 ) & (     0.19 ) \\
& \textit{ 74 } & \textit{ 74 } & \textit{ 74 } & \textit{ 182 } & \textit{ 240 } \\
Graduate from High School &     -0.02 &      0.06 &      0.05 & \textbf{      0.27 } &     -0.07 \\
& (     0.08 ) & (     0.07 ) & (     0.07 ) & (     0.13 ) & (     0.08 ) \\
& \textit{ 74 } & \textit{ 74 } & \textit{ 74 } & \textit{ 182 } & \textit{ 240 } \\
High School Grade &      1.85 &     -0.02 &     -0.03 &     -0.61 &      0.56 \\
& (     2.17 ) & (     2.42 ) & (     2.18 ) & (     3.59 ) & (     3.99 ) \\
& \textit{ 64 } & \textit{ 64 } & \textit{ 64 } & \textit{ 159 } & \textit{ 195 } \\
High School Grade (Standardized) & \textbf{      4.82 } &      2.78 &     -0.36 &      2.67 &      3.87 \\
& (     2.83 ) & (     3.37 ) & (     2.88 ) & (     3.27 ) & (     4.49 ) \\
& \textit{ 63 } & \textit{ 63 } & \textit{ 63 } & \textit{ 157 } & \textit{ 192 } \\
Max Edu: University &      0.04 &      0.02 &     -0.06 &      0.10 & \textbf{      0.26 } \\
& (     0.08 ) & (     0.11 ) & (     0.08 ) & (     0.15 ) & (     0.15 ) \\
& \textit{ 74 } & \textit{ 74 } & \textit{ 74 } & \textit{ 182 } & \textit{ 240 } \\
Employed & \textbf{     -0.06 } & \textbf{     -0.06 } &     -0.05 &     -0.02 &     -0.08 \\
& (     0.03 ) & (     0.04 ) & (     0.03 ) & (     0.10 ) & (     0.09 ) \\
& \textit{ 74 } & \textit{ 74 } & \textit{ 74 } & \textit{ 182 } & \textit{ 240 } \\
Hours Worked Per Week &     -2.07 &     -2.03 &     -2.78 &     -1.98 &     -1.60 \\
& (     1.87 ) & (     2.21 ) & (     2.52 ) & (     5.08 ) & (     4.00 ) \\
& \textit{ 73 } & \textit{ 73 } & \textit{ 73 } & \textit{ 181 } & \textit{ 235 } \\
Married or Cohabitating &      0.01 &      0.02 &     -0.03 &      0.03 &      0.13 \\
& (     0.12 ) & (     0.12 ) & (     0.13 ) & (     0.17 ) & (     0.17 ) \\
& \textit{ 74 } & \textit{ 74 } & \textit{ 74 } & \textit{ 182 } & \textit{ 240 } \\
Obese &     -0.04 &     -0.02 &     -0.03 &     -0.04 &     -0.02 \\
& (     0.09 ) & (     0.08 ) & (     0.09 ) & (     0.13 ) & (     0.12 ) \\
& \textit{ 74 } & \textit{ 74 } & \textit{ 74 } & \textit{ 182 } & \textit{ 240 } \\
Overweight &      0.03 &     -0.00 &     -0.03 &     -0.08 &      0.02 \\
& (     0.11 ) & (     0.10 ) & (     0.11 ) & (     0.16 ) & (     0.13 ) \\
& \textit{ 74 } & \textit{ 74 } & \textit{ 74 } & \textit{ 182 } & \textit{ 240 } \\
Locus of Control - positive &     -0.01 &     -0.05 &     -0.26 &     -0.03 &      0.18 \\
& (     0.21 ) & (     0.21 ) & (     0.19 ) & (     0.38 ) & (     0.28 ) \\
& \textit{ 69 } & \textit{ 69 } & \textit{ 69 } & \textit{ 169 } & \textit{ 228 } \\
Depression Score - positive &     -0.35 &     -0.59 &     -0.76 &      2.31 &      0.57 \\
& (     1.31 ) & (     1.17 ) & (     1.38 ) & (     1.89 ) & (     1.91 ) \\
& \textit{ 73 } & \textit{ 73 } & \textit{ 73 } & \textit{ 181 } & \textit{ 236 } \\
Ever Voted for Municipal &     -0.17 &     -0.01 &      0.04 &     -0.04 &      0.18 \\
& (     0.13 ) & (     0.10 ) & (     0.11 ) & (     0.13 ) & (     0.13 ) \\
& \textit{ 72 } & \textit{ 72 } & \textit{ 72 } & \textit{ 178 } & \textit{ 228 } \\
Ever Voted for Regional & \textbf{     -0.21 } &     -0.06 &     -0.03 &     -0.09 & \textbf{      0.22 } \\
& (     0.12 ) & (     0.10 ) & (     0.12 ) & (     0.12 ) & (     0.13 ) \\
& \textit{ 72 } & \textit{ 72 } & \textit{ 72 } & \textit{ 178 } & \textit{ 228 } \\
\bottomrule
\end{tabular}
}
\vspace{1ex} \\
\footnotesize\raggedright{Note: This table shows the estimates of the coefficient for attending Reggio Approach preschools from multiple methods. We compare Reggio Approach people with people who attended other preschools.  Column title indicates the corresponding control set and model.  ``None30'' refers to the OLS estimate with no control variables. ``BIC30'' refers to the OLS estimate with controls selected by Bayesian Information Criterion (BIC) and additional controls for caregiver's religion. ``Full30'' refers to the OLS estimate with the full set of controls. ``DidPm30'' refers to the difference-in-difference estimate of (Reggio Muni - Parma Muni) - (Reggio None - Parma None). ``DidPv30'' refers to the difference-in-difference estimate of (Reggio Muni - Padova Muni) - (Reggio None - Padova None).  Robust standard errors are reported in parentheses. Bold number shows that the estimate is statistically significant at the 10\% level. Number of observations used in estimation is reported in italic.}
\end{table}

\begin{table}[H] \caption{Estimation Results for Main Outcomes, Comparison to No Preschools, Adult 30s Cohort, Dropping Questionnable Interviewers} \label{ols-M-adult30-reg-nopres-dropint}
\scalebox{0.75}{\begin{tabular}{l c c c c c}
\toprule
 & None30 & BIC30 & Full30 & DidPm30 & DidPv30 \\
\midrule
IQ Score & \textbf{     -0.07 } & \textbf{     -0.07 } & \textbf{     -0.10 } &     -0.08 &     -0.08 \\
& (     0.04 ) & (     0.05 ) & (     0.05 ) & (     0.07 ) & (     0.08 ) \\
& \textit{ 72 } & \textit{ 72 } & \textit{ 72 } & \textit{ 121 } & \textit{ 138 } \\
IQ Factor & \textbf{     -0.19 } & \textbf{     -0.18 } & \textbf{     -0.28 } &     -0.26 &     -0.14 \\
& (     0.09 ) & (     0.11 ) & (     0.12 ) & (     0.19 ) & (     0.19 ) \\
& \textit{ 72 } & \textit{ 72 } & \textit{ 72 } & \textit{ 121 } & \textit{ 138 } \\
Graduate from High School & \textbf{     -0.10 } &     -0.07 &     -0.04 & \textbf{      0.26 } & \textbf{     -0.15 } \\
& (     0.07 ) & (     0.07 ) & (     0.06 ) & (     0.13 ) & (     0.10 ) \\
& \textit{ 72 } & \textit{ 72 } & \textit{ 72 } & \textit{ 121 } & \textit{ 138 } \\
High School Grade & \textbf{      5.67 } & \textbf{      4.72 } & \textbf{      4.30 } &      1.49 &      2.02 \\
& (     2.84 ) & (     2.65 ) & (     2.51 ) & (     4.20 ) & (     4.55 ) \\
& \textit{ 64 } & \textit{ 64 } & \textit{ 64 } & \textit{ 101 } & \textit{ 117 } \\
High School Grade (Standardized) & \textbf{      7.31 } & \textbf{      6.27 } & \textbf{      5.95 } &      5.68 &      5.03 \\
& (     3.30 ) & (     3.28 ) & (     3.29 ) & (     4.03 ) & (     5.34 ) \\
& \textit{ 64 } & \textit{ 64 } & \textit{ 64 } & \textit{ 101 } & \textit{ 116 } \\
Max Edu: University &     -0.14 &     -0.09 &     -0.14 &     -0.11 &     -0.23 \\
& (     0.11 ) & (     0.11 ) & (     0.11 ) & (     0.16 ) & (     0.17 ) \\
& \textit{ 72 } & \textit{ 72 } & \textit{ 72 } & \textit{ 121 } & \textit{ 138 } \\
Employed &     -0.02 &     -0.04 &     -0.04 &      0.03 &      0.00 \\
& (     0.06 ) & (     0.06 ) & (     0.07 ) & (     0.11 ) & (     0.11 ) \\
& \textit{ 72 } & \textit{ 72 } & \textit{ 72 } & \textit{ 121 } & \textit{ 138 } \\
Hours Worked Per Week &      1.33 &     -0.26 &     -0.30 &      2.39 &      1.94 \\
& (     2.71 ) & (     2.62 ) & (     3.30 ) & (     5.58 ) & (     4.76 ) \\
& \textit{ 72 } & \textit{ 72 } & \textit{ 72 } & \textit{ 121 } & \textit{ 138 } \\
Married or Cohabitating &      0.02 &      0.04 &      0.07 &     -0.08 &      0.12 \\
& (     0.13 ) & (     0.13 ) & (     0.14 ) & (     0.19 ) & (     0.19 ) \\
& \textit{ 72 } & \textit{ 72 } & \textit{ 72 } & \textit{ 121 } & \textit{ 138 } \\
Obese &      0.08 &      0.07 &      0.06 &      0.09 & \textbf{      0.19 } \\
& (     0.06 ) & (     0.06 ) & (     0.08 ) & (     0.13 ) & (     0.13 ) \\
& \textit{ 72 } & \textit{ 72 } & \textit{ 72 } & \textit{ 121 } & \textit{ 138 } \\
Overweight &     -0.04 &     -0.08 &     -0.09 &     -0.13 &     -0.07 \\
& (     0.12 ) & (     0.11 ) & (     0.11 ) & (     0.18 ) & (     0.15 ) \\
& \textit{ 72 } & \textit{ 72 } & \textit{ 72 } & \textit{ 121 } & \textit{ 138 } \\
Locus of Control - positive &      0.06 &      0.05 &     -0.03 &     -0.01 &      0.34 \\
& (     0.20 ) & (     0.24 ) & (     0.24 ) & (     0.41 ) & (     0.32 ) \\
& \textit{ 68 } & \textit{ 68 } & \textit{ 68 } & \textit{ 110 } & \textit{ 126 } \\
Depression Score - positive &     -0.66 & \textbf{     -1.84 } & \textbf{     -1.77 } &      0.44 &     -0.70 \\
& (     1.20 ) & (     1.22 ) & (     1.18 ) & (     2.03 ) & (     2.06 ) \\
& \textit{ 72 } & \textit{ 72 } & \textit{ 72 } & \textit{ 121 } & \textit{ 137 } \\
Ever Voted for Municipal & \textbf{      0.19 } &      0.08 &      0.11 &      0.05 &      0.11 \\
& (     0.11 ) & (     0.09 ) & (     0.09 ) & (     0.12 ) & (     0.14 ) \\
& \textit{ 70 } & \textit{ 70 } & \textit{ 70 } & \textit{ 118 } & \textit{ 129 } \\
Ever Voted for Regional &      0.15 &      0.01 &      0.05 &      0.02 &      0.16 \\
& (     0.11 ) & (     0.09 ) & (     0.09 ) & (     0.12 ) & (     0.14 ) \\
& \textit{ 70 } & \textit{ 70 } & \textit{ 70 } & \textit{ 118 } & \textit{ 129 } \\
\bottomrule
\end{tabular}
}
\vspace{1ex} \\
\footnotesize\raggedright{Note: This table shows the estimates of the coefficient for attending Reggio Approach preschools from multiple methods. We compare Reggio Approach people with people who attended no preschool. Column title indicates the corresponding control set and and model. ``None30'' refers to the OLS estimate with no control variables. ``BIC30'' refers to the OLS estimate with controls selected by Bayesian Information Criterion (BIC) and additional controls for caregiver's religion. ``Full30'' refers to the OLS estimate with the full set of controls. ``DidPm30'' refers to the difference-in-difference estimate of (Reggio Muni - Parma Muni) - (Reggio None - Parma None). ``DidPv30'' refers to the difference-in-difference estimate of (Reggio Muni - Padova Muni) - (Reggio None - Padova None).  Robust standard errors are reported in parentheses. Bold number shows that the estimate is statistically significant at the 10\% level. Number of observations used in estimation is reported in italic.}
\end{table}




\begin{table}[H] \caption{Estimation Results for Main Outcomes, Comparison to Preschools, Adult 40s Cohort, Dropping Questionnable Interviewers} \label{ols-M-adult40-reg-pres-dropint}
\scalebox{0.75}{\begin{tabular}{l c c c}
\toprule
 & None40 & BIC40 & Full40 \\
\midrule
IQ Factor & 0.08 & 0.02 & -0.00 \\
& (0.13) & (0.11) & (0.14) \\
& \textit{ 80 } & \textit{ 80 } & \textit{ 80 } \\
Graduate from High School & 0.05 & 0.08 & 0.09 \\
& (0.09) & (0.09) & (0.09) \\
& \textit{ 80 } & \textit{ 80 } & \textit{ 80 } \\
High School Grade & -0.39 & 0.51 & 0.80 \\
& (2.07) & (2.14) & (2.41) \\
& \textit{ 67 } & \textit{ 67 } & \textit{ 67 } \\
High School Grade (Standardized) & 0.76 & 2.25 & 3.03 \\
& (2.84) & (2.76) & (3.14) \\
& \textit{ 66 } & \textit{ 66 } & \textit{ 66 } \\
Max Edu: University & 0.09 & 0.09 & 0.02 \\
& (0.08) & (0.09) & (0.09) \\
& \textit{ 80 } & \textit{ 80 } & \textit{ 80 } \\
Employed & 0.06 & \textbf{ 0.07 } & \textbf{ 0.08 } \\
& (0.04) & (0.05) & (0.05) \\
& \textit{ 80 } & \textit{ 80 } & \textit{ 80 } \\
Hours Worked Per Week & 1.92 & 2.40 & 2.64 \\
& (2.68) & (3.04) & (3.42) \\
& \textit{ 78 } & \textit{ 78 } & \textit{ 78 } \\
Married or Cohabitating & 0.14 & 0.09 & 0.08 \\
& (0.11) & (0.10) & (0.12) \\
& \textit{ 80 } & \textit{ 80 } & \textit{ 80 } \\
Not Obese & -0.13 & -0.08 & -0.16 \\
& (0.11) & (0.12) & (0.13) \\
& \textit{ 80 } & \textit{ 80 } & \textit{ 80 } \\
Not Overweight & 0.07 & 0.03 & 0.00 \\
& (0.10) & (0.10) & (0.10) \\
& \textit{ 80 } & \textit{ 80 } & \textit{ 80 } \\
Locus of Control - positive & 0.05 & 0.07 & 0.11 \\
& (0.19) & (0.20) & (0.24) \\
& \textit{ 77 } & \textit{ 77 } & \textit{ 77 } \\
Depression Score - positive & & & \\
& () & () & () \\
& & & \\
Ever Voted for Municipal & 0.10 & 0.15 & 0.06 \\
& (0.12) & (0.11) & (0.12) \\
& \textit{ 72 } & \textit{ 72 } & \textit{ 72 } \\
Ever Voted for Regional & 0.14 & \textbf{ 0.20 } & 0.08 \\
& (0.12) & (0.11) & (0.12) \\
& \textit{ 72 } & \textit{ 72 } & \textit{ 72 } \\
\bottomrule
\end{tabular}
}
\vspace{1ex} \\
\footnotesize\raggedright{Note: This table shows the estimates of the coefficient for attending Reggio Approach preschools from multiple methods. We compare Reggio Approach people with people who attended other preschools.  Column title indicates the corresponding control set and and model. ``None40'' refers to the OLS estimate with no control variables. ``BIC40'' refers to the OLS estimate with controls selected by Bayesian Information Criterion (BIC) and additional controls for caregiver's religion. ``Full40'' refers to the OLS estimate with the full set of controls. ``DidPm40'' refers to the difference-in-difference estimate of (Reggio Muni - Parma Muni) - (Reggio None - Parma None). ``DidPv40'' refers to the difference-in-difference estimate of (Reggio Muni - Padova Muni) - (Reggio None - Padova None).Robust standard errors are reported in parentheses. Bold number shows that the estimate is statistically significant at the 10\% level. Number of observations used in estimation is reported in italic.}
\end{table}

\begin{table}[H] \caption{Estimation Results for Main Outcomes, Comparison to No Preschools, Adult 40s Cohort, Dropping Questionnable Interviewers} \label{ols-M-adult40-reg-nopres-dropint}
\scalebox{0.75}{\begin{tabular}{l c c c}
\toprule
 & None40 & BIC40 & Full40 \\
\midrule
IQ Score &      0.02 &      0.01 &      0.02 \\
& (     0.04 ) & (     0.04 ) & (     0.05 ) \\
& \textit{ 104 } & \textit{ 104 } & \textit{ 104 } \\
IQ Factor &      0.08 &      0.06 &      0.06 \\
& (     0.10 ) & (     0.09 ) & (     0.12 ) \\
& \textit{ 104 } & \textit{ 104 } & \textit{ 104 } \\
Graduate from High School &     -0.05 &      0.04 &     -0.01 \\
& (     0.06 ) & (     0.07 ) & (     0.07 ) \\
& \textit{ 104 } & \textit{ 104 } & \textit{ 104 } \\
High School Grade &     -0.55 &      1.28 &      1.11 \\
& (     1.77 ) & (     1.72 ) & (     2.13 ) \\
& \textit{ 91 } & \textit{ 91 } & \textit{ 91 } \\
High School Grade (Standardized) &     -0.22 &      1.51 &      0.55 \\
& (     2.40 ) & (     2.47 ) & (     2.93 ) \\
& \textit{ 91 } & \textit{ 91 } & \textit{ 91 } \\
Max Edu: University &      0.00 &      0.07 &      0.09 \\
& (     0.08 ) & (     0.09 ) & (     0.09 ) \\
& \textit{ 104 } & \textit{ 104 } & \textit{ 104 } \\
Employed & \textbf{      0.13 } & \textbf{      0.13 } & \textbf{      0.11 } \\
& (     0.05 ) & (     0.05 ) & (     0.05 ) \\
& \textit{ 103 } & \textit{ 103 } & \textit{ 103 } \\
Hours Worked Per Week & \textbf{      4.40 } & \textbf{      4.07 } & \textbf{      4.53 } \\
& (     2.67 ) & (     2.79 ) & (     2.99 ) \\
& \textit{ 100 } & \textit{ 100 } & \textit{ 100 } \\
Married or Cohabitating &     -0.01 &     -0.00 &      0.02 \\
& (     0.08 ) & (     0.09 ) & (     0.11 ) \\
& \textit{ 104 } & \textit{ 104 } & \textit{ 104 } \\
Obese &     -0.07 &      0.02 &      0.08 \\
& (     0.09 ) & (     0.10 ) & (     0.12 ) \\
& \textit{ 104 } & \textit{ 104 } & \textit{ 104 } \\
Overweight &      0.07 &     -0.01 &     -0.00 \\
& (     0.08 ) & (     0.09 ) & (     0.10 ) \\
& \textit{ 104 } & \textit{ 104 } & \textit{ 104 } \\
Locus of Control - positive &      0.21 & \textbf{      0.29 } & \textbf{      0.42 } \\
& (     0.16 ) & (     0.17 ) & (     0.19 ) \\
& \textit{ 99 } & \textit{ 99 } & \textit{ 99 } \\
Depression Score - positive &      1.67 & \textbf{      2.19 } &      2.14 \\
& (     1.17 ) & (     1.25 ) & (     1.57 ) \\
& \textit{ 102 } & \textit{ 102 } & \textit{ 102 } \\
Ever Voted for Municipal & \textbf{      0.18 } &      0.14 & \textbf{      0.16 } \\
& (     0.10 ) & (     0.10 ) & (     0.09 ) \\
& \textit{ 89 } & \textit{ 89 } & \textit{ 89 } \\
Ever Voted for Regional & \textbf{      0.18 } & \textbf{      0.16 } & \textbf{      0.16 } \\
& (     0.10 ) & (     0.10 ) & (     0.10 ) \\
& \textit{ 89 } & \textit{ 89 } & \textit{ 89 } \\
\bottomrule
\end{tabular}
}
\vspace{1ex} \\
\footnotesize\raggedright{Note: This table shows the estimates of the coefficient for attending Reggio Approach preschools from multiple methods. We compare Reggio Approach people with people who attended no preschool. Column title indicates the corresponding control set and and model. ``None40'' refers to the OLS estimate with no control variables. ``BIC40'' refers to the OLS estimate with controls selected by Bayesian Information Criterion (BIC) and additional controls for caregiver's religion. ``Full40'' refers to the OLS estimate with the full set of controls. ``DidPm40'' refers to the difference-in-difference estimate of (Reggio Muni - Parma Muni) - (Reggio None - Parma None). ``DidPv40'' refers to the difference-in-difference estimate of (Reggio Muni - Padova Muni) - (Reggio None - Padova None). Robust standard errors are reported in parentheses. Bold number shows that the estimate is statistically significant at the 10\% level. Number of observations used in estimation is reported in italic.}
\end{table}





\subsection{Estimation Results for Reggio Approach Preschools, Extended Outcomes} 
\subsubsection{Child Cohort}
\begin{table}[H] \caption{Estimation Results for Cognitive and Noncognitive Outcomes, Comparison to Non-RA Preschools, Child Cohort} \label{ols-CN-child-reg-reli}
\scalebox{0.8}{\begin{tabular}{l c c c c c c c c c}
\toprule
 & None & BIC & Full & PSM & AIPW & DidPm & PSMPm & DidPv & PSMPv \\
\midrule
IQ Factor & -0.06 & -0.10 & -0.10 & -0.11 & -0.09 & 0.04 & \textbf{-0.32} & 0.08 & -0.15 \\
& (0.10) & (0.10) & (0.10) & (0.10) & (0.10) & (0.13) & (0.09) & (0.15) & (0.11) \\
& \textit{ 408 } & \textit{ 408 } & \textit{ 408 } & \textit{ 408 } & \textit{ 408 } & \textit{ 756 } & \textit{ 544 } & \textit{ 787 } & \textit{ 590 } \\
IQ Score & -0.02 & -0.02 & -0.03 & -0.03 & -0.02 & 0.01 & \textbf{-0.07} & 0.00 & -0.04 \\
& (0.02) & (0.02) & (0.02) & (0.02) & (0.02) & (0.03) & (0.02) & (0.03) & (0.03) \\
& \textit{ 408 } & \textit{ 408 } & \textit{ 408 } & \textit{ 408 } & \textit{ 408 } & \textit{ 756 } & \textit{ 544 } & \textit{ 787 } & \textit{ 590 } \\
SDQ Composite - Child & \textbf{ 0.74 } & \textbf{ 0.82 } & \textbf{ 1.25 } & 0.62 & \textbf{0.79} & 0.16 & 0.27 & \textbf{ 1.27 } & \textbf{0.89} \\
& (0.47) & (0.46) & (0.45) & (0.50) & (0.50) & (0.66) & (0.47) & (0.75) & (0.49) \\
& \textit{ 407 } & \textit{ 407 } & \textit{ 407 } & \textit{ 407 } & \textit{ 407 } & \textit{ 755 } & \textit{ 544 } & \textit{ 786 } & \textit{ 590 } \\
SDQ Pro-social - Child & 0.24 & \textbf{ 0.39 } & 0.19 & \textbf{0.35} & \textbf{0.40} & 0.08 & 0.11 & 0.37 & 0.32 \\
& (0.18) & (0.18) & (0.18) & (0.19) & (0.18) & (0.27) & (0.17) & (0.26) & (0.19) \\
& \textit{ 407 } & \textit{ 407 } & \textit{ 407 } & \textit{ 407 } & \textit{ 407 } & \textit{ 755 } & \textit{ 544 } & \textit{ 786 } & \textit{ 590 } \\
SDQ Peer problems - Child & 0.00 & 0.00 & 0.11 & 0.03 & 0.02 & -0.13 & \textbf{0.25} & 0.10 & 0.14 \\
& (0.13) & (0.14) & (0.14) & (0.15) & (0.16) & (0.21) & (0.13) & (0.23) & (0.15) \\
& \textit{ 407 } & \textit{ 407 } & \textit{ 407 } & \textit{ 407 } & \textit{ 407 } & \textit{ 755 } & \textit{ 544 } & \textit{ 786 } & \textit{ 590 } \\
SDQ Hyper - Child & 0.12 & 0.06 & \textbf{ 0.31 } & -0.01 & 0.08 & 0.09 & -0.21 & -0.07 & 0.27 \\
& (0.23) & (0.23) & (0.21) & (0.24) & (0.24) & (0.32) & (0.23) & (0.33) & (0.23) \\
& \textit{ 407 } & \textit{ 407 } & \textit{ 407 } & \textit{ 407 } & \textit{ 407 } & \textit{ 755 } & \textit{ 544 } & \textit{ 786 } & \textit{ 590 } \\
SDQ Emotional - Child & \textbf{ 0.40 } & \textbf{ 0.52 } & \textbf{ 0.50 } & \textbf{0.45} & \textbf{0.46} & 0.17 & 0.21 & \textbf{ 0.88 } & 0.27 \\
& (0.17) & (0.17) & (0.18) & (0.19) & (0.17) & (0.24) & (0.17) & (0.27) & (0.16) \\
& \textit{ 407 } & \textit{ 407 } & \textit{ 407 } & \textit{ 407 } & \textit{ 407 } & \textit{ 755 } & \textit{ 544 } & \textit{ 786 } & \textit{ 590 } \\
SDQ Conduct - Child & \textbf{ 0.22 } & \textbf{ 0.24 } & \textbf{ 0.33 } & 0.16 & \textbf{0.23} & 0.02 & 0.02 & \textbf{ 0.35 } & 0.21 \\
& (0.14) & (0.15) & (0.15) & (0.16) & (0.13) & (0.22) & (0.13) & (0.23) & (0.16) \\
& \textit{ 407 } & \textit{ 407 } & \textit{ 407 } & \textit{ 407 } & \textit{ 407 } & \textit{ 755 } & \textit{ 544 } & \textit{ 786 } & \textit{ 590 } \\
\bottomrule
\end{tabular}
}
\vspace{1ex} \\
\footnotesize\raggedright{Note: This table shows the estimates of the coefficient for attending Reggio Approach preschools from multiple methods. We compare Reggio Approach people with people who attended religious preschools in Reggio. Column title indicates the corresponding control set and and model. ``None'' refers to the OLS estimate with no control variables. ``BIC'' refers to the OLS estimate with controls selected by Bayesian Information Criterion (BIC) and additional controls for caregiver's religion. ``Full'' refers to the OLS estimate with the full set of controls. ``PSM" refers to propensity score matching estimation. ``AIPW" refers to augmented inverse propensity weighting estimation. ``DidPmIt'' refers to the difference-in-difference estimate of (Reggio Muni - Parma Muni) - (Reggio Other - Parma Other). ``DidPv'' refers to the difference-in-difference estimate of (Reggio Muni - Padova Muni) - (Reggio Other - Padova Other). Bold number shows that the estimate is statistically significant at the 10\% level. Number of observations used in estimation is reported in italic.}
\end{table}


\begin{table}[H] \caption{Estimation Results for Social Outcomes, Comparison to Non-RA Preschools, Child Cohort} \label{ols-S-child-reg-reli}
\scalebox{0.8}{\begin{tabular}{l c c c c c c c}
\toprule
 & None & BIC & Full & PSM & AIPW & DidPm & DidPv \\
\midrule
Musical Instrument at Home &     -0.01 &     -0.04 &     -0.03 &     -0.04 &     -0.04 &     -0.02 &      0.01 \\
& (     0.05 ) & (     0.05 ) & (     0.05 ) & (     0.05 ) & (     0.05 ) & (     0.09 ) & (     0.07 ) \\
& \textit{ 408 } & \textit{ 408 } & \textit{ 408 } & \textit{ 408 } & \textit{ 408 } & \textit{ 756 } & \textit{ 787 } \\
Tell Worry at Home &      0.00 &      0.01 &      0.00 &     -0.01 &      0.02 &     -0.00 &      0.07 \\
& (     0.05 ) & (     0.05 ) & (     0.05 ) & (     0.05 ) & (     0.05 ) & (     0.09 ) & (     0.07 ) \\
& \textit{ 408 } & \textit{ 408 } & \textit{ 408 } & \textit{ 408 } & \textit{ 408 } & \textit{ 756 } & \textit{ 787 } \\
Tell Worry to Teacher &      0.06 &      0.03 &      0.04 &      0.04 &      0.04 & \textbf{      0.14 } &      0.05 \\
& (     0.04 ) & (     0.04 ) & (     0.05 ) & (     0.05 ) & (     0.05 ) & (     0.08 ) & (     0.07 ) \\
& \textit{ 408 } & \textit{ 408 } & \textit{ 408 } & \textit{ 408 } & \textit{ 408 } & \textit{ 756 } & \textit{ 787 } \\
Tell Worry to Friends &      0.01 &     -0.00 &      0.02 &     -0.02 &      0.01 & \textbf{      0.14 } &     -0.06 \\
& (     0.04 ) & (     0.04 ) & (     0.04 ) & (     0.04 ) & (     0.04 ) & (     0.06 ) & (     0.06 ) \\
& \textit{ 408 } & \textit{ 408 } & \textit{ 408 } & \textit{ 408 } & \textit{ 408 } & \textit{ 756 } & \textit{ 787 } \\
Keep Worry to Myself & \textbf{     -0.06 } & \textbf{     -0.07 } & \textbf{     -0.06 } &     -0.04 &     -0.07 &     -0.03 & \textbf{     -0.09 } \\
& (     0.04 ) & (     0.04 ) & (     0.04 ) & (     0.04 ) & (     0.03 ) & (     0.06 ) & (     0.05 ) \\
& \textit{ 408 } & \textit{ 408 } & \textit{ 408 } & \textit{ 408 } & \textit{ 408 } & \textit{ 756 } & \textit{ 787 } \\
\bottomrule
\end{tabular}
}
\vspace{1ex} \\
\footnotesize\raggedright{Note: This table shows the estimates of the coefficient for attending Reggio Approach preschools from multiple methods. We compare Reggio Approach people with people who attended religious preschools in Reggio. Column title indicates the corresponding control set and and model. ``None'' refers to the OLS estimate with no control variables. ``BIC'' refers to the OLS estimate with controls selected by Bayesian Information Criterion (BIC) and additional controls for caregiver's religion. ``Full'' refers to the OLS estimate with the full set of controls. ``PSM" refers to propensity score matching estimation. ``AIPW" refers to augmented inverse propensity weighting estimation. ``DidPmIt'' refers to the difference-in-difference estimate of (Reggio Muni - Parma Muni) - (Reggio Other - Parma Other). ``DidPv'' refers to the difference-in-difference estimate of (Reggio Muni - Padova Muni) - (Reggio Other - Padova Other). Bold number shows that the estimate is statistically significant at the 10\% level. Number of observations used in estimation is reported in italic.}
\end{table}


\begin{table}[H] \caption{Estimation Results for Health Outcomes, Comparison to Non-RA Preschools, Child Cohort} \label{ols-H-child-reg-reli}
\scalebox{0.8}{\begin{tabular}{l c c c c c c c}
\toprule
 & None & BIC & Full & PSM & AIPW & DidPm & DidPv \\
\midrule
Obese &      0.00 &      0.04 &      0.03 &      0.04 &      0.05 &      0.03 &     -0.05 \\
& (     0.05 ) & (     0.05 ) & (     0.05 ) & (     0.05 ) & (     0.05 ) & (     0.08 ) & (     0.07 ) \\
& \textit{ 408 } & \textit{ 408 } & \textit{ 408 } & \textit{ 408 } & \textit{ 408 } & \textit{ 756 } & \textit{ 787 } \\
Overweight &      0.04 &      0.03 &      0.05 &      0.03 &      0.03 &     -0.06 & \textbf{      0.09 } \\
& (     0.03 ) & (     0.03 ) & (     0.04 ) & (     0.03 ) & (     0.03 ) & (     0.07 ) & (     0.05 ) \\
& \textit{ 408 } & \textit{ 408 } & \textit{ 408 } & \textit{ 408 } & \textit{ 408 } & \textit{ 756 } & \textit{ 787 } \\
Health is Good &     -0.04 &     -0.02 &     -0.01 &     -0.02 &     -0.02 &      0.09 &     -0.04 \\
& (     0.05 ) & (     0.05 ) & (     0.05 ) & (     0.05 ) & (     0.05 ) & (     0.09 ) & (     0.07 ) \\
& \textit{ 407 } & \textit{ 407 } & \textit{ 407 } & \textit{ 407 } & \textit{ 407 } & \textit{ 755 } & \textit{ 785 } \\
Number of Sick Days &      0.03 &     -0.04 &     -0.05 &     -0.08 &     -0.01 &      0.04 &      0.02 \\
& (     0.08 ) & (     0.08 ) & (     0.08 ) & (     0.09 ) & (     0.08 ) & (     0.12 ) & (     0.12 ) \\
& \textit{ 405 } & \textit{ 405 } & \textit{ 405 } & \textit{ 405 } & \textit{ 405 } & \textit{ 752 } & \textit{ 783 } \\
\bottomrule
\end{tabular}
}
\vspace{1ex} \\
\footnotesize\raggedright{Note: This table shows the estimates of the coefficient for attending Reggio Approach preschools from multiple methods. We compare Reggio Approach people with people who attended religious preschools in Reggio. Column title indicates the corresponding control set and and model. ``None'' refers to the OLS estimate with no control variables. ``BIC'' refers to the OLS estimate with controls selected by Bayesian Information Criterion (BIC) and additional controls for caregiver's religion. ``Full'' refers to the OLS estimate with the full set of controls. ``PSM" refers to propensity score matching estimation. ``AIPW" refers to augmented inverse propensity weighting estimation. ``DidPmIt'' refers to the difference-in-difference estimate of (Reggio Muni - Parma Muni) - (Reggio Other - Parma Other). ``DidPv'' refers to the difference-in-difference estimate of (Reggio Muni - Padova Muni) - (Reggio Other - Padova Other). Bold number shows that the estimate is statistically significant at the 10\% level. Number of observations used in estimation is reported in italic.}
\end{table}


\begin{table}[H] \caption{Estimation Results for Behavioral Outcomes, Comparison to Non-RA Preschools, Child Cohort} \label{ols-B-child-reg-reli}
\scalebox{0.8}{\begin{tabular}{l c c c c c c c c c}
\toprule
 & None & BIC & Full & PSM & AIPW & DidPm & PSMPm & DidPv & PSMPv \\
\midrule
Not Excited to Learn & -0.00 & 0.00 & -0.00 & -0.00 & -0.00 & 0.01 & -0.03 & -0.03 & -0.01 \\
& (0.02) & (0.02) & (0.02) & (0.02) & (0.02) & (0.03) & (0.02) & (0.04) & (0.03) \\
& \textit{ 408 } & \textit{ 408 } & \textit{ 408 } & \textit{ 408 } & \textit{ 408 } & \textit{ 756 } & \textit{ 544 } & \textit{ 787 } & \textit{ 590 } \\
Problems Sitting Still & 0.02 & 0.03 & 0.00 & 0.04 & 0.02 & -0.04 & 0.02 & -0.04 & 0.04 \\
& (0.03) & (0.03) & (0.03) & (0.04) & (0.04) & (0.05) & (0.04) & (0.05) & (0.04) \\
& \textit{ 408 } & \textit{ 408 } & \textit{ 408 } & \textit{ 408 } & \textit{ 408 } & \textit{ 756 } & \textit{ 544 } & \textit{ 787 } & \textit{ 590 } \\
How Much Child Likes School & 0.01 & -0.00 & 0.03 & 0.01 & -0.01 & 0.07 & -0.05 & 0.09 & \textbf{0.15} \\
& (0.06) & (0.06) & (0.06) & (0.07) & (0.05) & (0.08) & (0.05) & (0.09) & (0.06) \\
& \textit{ 406 } & \textit{ 406 } & \textit{ 406 } & \textit{ 406 } & \textit{ 406 } & \textit{ 752 } & \textit{ 542 } & \textit{ 785 } & \textit{ 590 } \\
Happy in General & -0.03 & 0.07 & 0.08 & 0.12 & 0.10 & 0.11 & 0.24 & 0.29 & -0.18 \\
& (0.17) & (0.17) & (0.18) & (0.17) & (0.17) & (0.24) & (0.17) & (0.25) & (0.15) \\
& \textit{ 408 } & \textit{ 408 } & \textit{ 408 } & \textit{ 408 } & \textit{ 408 } & \textit{ 756 } & \textit{ 544 } & \textit{ 787 } & \textit{ 590 } \\
\bottomrule
\end{tabular}
}
\vspace{1ex} \\
\footnotesize\raggedright{Note: This table shows the estimates of the coefficient for attending Reggio Approach preschools from multiple methods. We compare Reggio Approach people with people who attended religious preschools in Reggio. Column title indicates the corresponding control set and and model. ``None'' refers to the OLS estimate with no control variables. ``BIC'' refers to the OLS estimate with controls selected by Bayesian Information Criterion (BIC) and additional controls for caregiver's religion. ``Full'' refers to the OLS estimate with the full set of controls. ``PSM" refers to propensity score matching estimation. ``AIPW" refers to augmented inverse propensity weighting estimation. ``DidPmIt'' refers to the difference-in-difference estimate of (Reggio Muni - Parma Muni) - (Reggio Other - Parma Other). ``DidPv'' refers to the difference-in-difference estimate of (Reggio Muni - Padova Muni) - (Reggio Other - Padova Other). Bold number shows that the estimate is statistically significant at the 10\% level. Number of observations used in estimation is reported in italic.}
\end{table}



\subsubsection{Adolescent Cohort}
\begin{table}[H] \caption{Estimation Results for Cognitive and Noncognitive Outcomes, Comparison to Non-RA Preschools, Adolescent Cohort} \label{ols-CN-adol-reg-reli}
\scalebox{0.8}{\begin{tabular}{l c c c c c c c}
\toprule
 & None & Bic & Full & PSM & AIPW & DidPm & DidPv \\
\midrule
IQ Factor &     -0.12 & \textbf{     -0.18 } &     -0.03 &     -0.18 &     -0.12 &     -0.02 &     -0.22 \\
& (     0.10 ) & (     0.11 ) & (     0.11 ) & (     0.15 ) & (     0.11 ) & (     0.17 ) & (     0.17 ) \\
& \textit{ 285 } & \textit{ 285 } & \textit{ 285 } & \textit{ 285 } & \textit{ 285 } & \textit{ 524 } & \textit{ 559 } \\
IQ Score &     -0.03 &     -0.04 &      0.00 &     -0.04 &     -0.03 &     -0.02 &     -0.05 \\
& (     0.03 ) & (     0.03 ) & (     0.03 ) & (     0.04 ) & (     0.04 ) & (     0.05 ) & (     0.05 ) \\
& \textit{ 285 } & \textit{ 285 } & \textit{ 285 } & \textit{ 285 } & \textit{ 285 } & \textit{ 524 } & \textit{ 559 } \\
SDQ Composite - Child &      0.01 &     -0.08 &      0.37 &     -0.07 &      0.17 &      0.40 &     -0.56 \\
& (     0.59 ) & (     0.70 ) & (     0.62 ) & (     0.67 ) & (     0.79 ) & (     1.08 ) & (     0.80 ) \\
& \textit{ 285 } & \textit{ 285 } & \textit{ 285 } & \textit{ 285 } & \textit{ 285 } & \textit{ 524 } & \textit{ 554 } \\
SDQ Pro-social - Child &      0.16 &      0.05 &     -0.13 &      0.27 &     -0.12 &      0.19 &     -0.01 \\
& (     0.22 ) & (     0.26 ) & (     0.23 ) & (     0.31 ) & (     0.30 ) & (     0.37 ) & (     0.32 ) \\
& \textit{ 285 } & \textit{ 285 } & \textit{ 285 } & \textit{ 285 } & \textit{ 285 } & \textit{ 524 } & \textit{ 555 } \\
SDQ Peer problems - Child &     -0.12 &     -0.21 &     -0.05 &     -0.13 &     -0.17 & \textbf{     -0.57 } &     -0.33 \\
& (     0.19 ) & (     0.22 ) & (     0.20 ) & (     0.21 ) & (     0.23 ) & (     0.29 ) & (     0.26 ) \\
& \textit{ 285 } & \textit{ 285 } & \textit{ 285 } & \textit{ 285 } & \textit{ 285 } & \textit{ 524 } & \textit{ 556 } \\
SDQ Hyper - Child &      0.14 &      0.04 &      0.17 &     -0.02 &      0.19 &      0.53 &     -0.16 \\
& (     0.23 ) & (     0.25 ) & (     0.23 ) & (     0.24 ) & (     0.24 ) & (     0.41 ) & (     0.32 ) \\
& \textit{ 285 } & \textit{ 285 } & \textit{ 285 } & \textit{ 285 } & \textit{ 285 } & \textit{ 524 } & \textit{ 555 } \\
SDQ Emotional - Child &     -0.02 &     -0.07 &      0.04 &      0.03 &     -0.05 &     -0.03 &     -0.12 \\
& (     0.24 ) & (     0.27 ) & (     0.26 ) & (     0.26 ) & (     0.24 ) & (     0.45 ) & (     0.33 ) \\
& \textit{ 285 } & \textit{ 285 } & \textit{ 285 } & \textit{ 285 } & \textit{ 285 } & \textit{ 524 } & \textit{ 555 } \\
SDQ Conduct - Child &      0.02 &      0.16 &      0.22 &      0.05 &      0.20 &      0.47 &      0.03 \\
& (     0.17 ) & (     0.20 ) & (     0.18 ) & (     0.21 ) & (     0.22 ) & (     0.33 ) & (     0.25 ) \\
& \textit{ 285 } & \textit{ 285 } & \textit{ 285 } & \textit{ 285 } & \textit{ 285 } & \textit{ 524 } & \textit{ 554 } \\
SDQ Composite &      0.90 &      0.93 &      0.72 &      1.04 &      0.44 &      1.25 &      0.72 \\
& (     0.63 ) & (     0.72 ) & (     0.72 ) & (     0.75 ) & (     0.70 ) & (     1.07 ) & (     0.95 ) \\
& \textit{ 283 } & \textit{ 283 } & \textit{ 283 } & \textit{ 283 } & \textit{ 283 } & \textit{ 520 } & \textit{ 555 } \\
SDQ Pro-social &      0.10 &     -0.09 &     -0.06 &      0.02 &     -0.17 &     -0.31 &     -0.39 \\
& (     0.21 ) & (     0.23 ) & (     0.23 ) & (     0.25 ) & (     0.26 ) & (     0.36 ) & (     0.31 ) \\
& \textit{ 283 } & \textit{ 283 } & \textit{ 283 } & \textit{ 283 } & \textit{ 283 } & \textit{ 520 } & \textit{ 555 } \\
SDQ Peer problems &     -0.09 &     -0.15 &     -0.15 &     -0.24 &     -0.23 &     -0.29 &      0.06 \\
& (     0.18 ) & (     0.21 ) & (     0.19 ) & (     0.20 ) & (     0.22 ) & (     0.28 ) & (     0.29 ) \\
& \textit{ 283 } & \textit{ 283 } & \textit{ 283 } & \textit{ 283 } & \textit{ 283 } & \textit{ 520 } & \textit{ 555 } \\
SDQ Hyper & \textbf{      0.38 } &      0.40 &      0.30 & \textbf{     0.56} &      0.27 & \textbf{      0.65 } &      0.25 \\
& (     0.25 ) & (     0.28 ) & (     0.28 ) & (     0.29 ) & (     0.29 ) & (     0.44 ) & (     0.36 ) \\
& \textit{ 283 } & \textit{ 283 } & \textit{ 283 } & \textit{ 283 } & \textit{ 283 } & \textit{ 520 } & \textit{ 555 } \\
SDQ Emotional &      0.27 &      0.17 &      0.18 &      0.29 &     -0.02 &      0.23 &     -0.07 \\
& (     0.27 ) & (     0.29 ) & (     0.29 ) & (     0.32 ) & (     0.24 ) & (     0.47 ) & (     0.38 ) \\
& \textit{ 283 } & \textit{ 283 } & \textit{ 283 } & \textit{ 283 } & \textit{ 283 } & \textit{ 520 } & \textit{ 555 } \\
SDQ Conduct & \textbf{      0.35 } & \textbf{      0.51 } & \textbf{      0.38 } & \textbf{     0.43} & \textbf{     0.41} & \textbf{      0.67 } & \textbf{      0.48 } \\
& (     0.19 ) & (     0.21 ) & (     0.22 ) & (     0.25 ) & (     0.21 ) & (     0.35 ) & (     0.28 ) \\
& \textit{ 283 } & \textit{ 283 } & \textit{ 283 } & \textit{ 283 } & \textit{ 283 } & \textit{ 520 } & \textit{ 555 } \\
Depression Score - positive & \textbf{      1.46 } & \textbf{      1.91 } & \textbf{      1.81 } & \textbf{     1.64} & \textbf{     1.20} & \textbf{      2.84 } & \textbf{      1.97 } \\
& (     0.78 ) & (     0.88 ) & (     0.91 ) & (     0.90 ) & (     0.73 ) & (     1.14 ) & (     1.14 ) \\
& \textit{ 278 } & \textit{ 278 } & \textit{ 278 } & \textit{ 278 } & \textit{ 278 } & \textit{ 506 } & \textit{ 548 } \\
\bottomrule
\end{tabular}
}
\vspace{1ex} \\
\footnotesize\raggedright{Note: This table shows the estimates of the coefficient for attending Reggio Approach preschools from multiple methods. We compare Reggio Approach people with people who attended religious preschools in Reggio. Column title indicates the corresponding control set and and model. ``None'' refers to the OLS estimate with no control variables. ``BIC'' refers to the OLS estimate with controls selected by Bayesian Information Criterion (BIC) and additional controls for caregiver's religion. ``Full'' refers to the OLS estimate with the full set of controls. ``PSM" refers to propensity score matching estimation. ``AIPW" refers to augmented inverse propensity weighting estimation. ``DidPmIt'' refers to the difference-in-difference estimate of (Reggio Muni - Parma Muni) - (Reggio Other - Parma Other). ``DidPv'' refers to the difference-in-difference estimate of (Reggio Muni - Padova Muni) - (Reggio Other - Padova Other). Bold number shows that the estimate is statistically significant at the 10\% level. Number of observations used in estimation is reported in italic.}
\end{table}


\begin{table}[H] \caption{Estimation Results for Social Outcomes, Comparison to Non-RA Preschools,  Adolescent Cohort} \label{ols-S-adol-reg-reli}
\scalebox{0.8}{\begin{tabular}{l c c c c c c c}
\toprule
 & None & Bic & Full & PSM & AIPW & DidPm & DidPv \\
\midrule
Num. of Friends &     -0.76 &     -0.46 &     -0.35 &     -0.38 &      0.56 &     -1.65 &      0.15 \\
& (     1.25 ) & (     1.02 ) & (     1.16 ) & (     1.05 ) & (     1.07 ) & (     2.52 ) & (     2.14 ) \\
& \textit{ 277 } & \textit{ 277 } & \textit{ 277 } & \textit{ 277 } & \textit{ 277 } & \textit{ 500 } & \textit{ 497 } \\
Doesn't Talk About Activities &      0.09 &      0.11 &      0.02 &      0.09 &      0.09 &      0.14 &      0.04 \\
& (     0.08 ) & (     0.09 ) & (     0.09 ) & (     0.09 ) & (     0.09 ) & (     0.14 ) & (     0.11 ) \\
& \textit{ 284 } & \textit{ 284 } & \textit{ 284 } & \textit{ 284 } & \textit{ 284 } & \textit{ 523 } & \textit{ 554 } \\
Doesn't Talk About School &      0.06 &      0.07 &      0.02 &      0.07 &      0.03 & \textbf{      0.20 } &      0.06 \\
& (     0.07 ) & (     0.08 ) & (     0.08 ) & (     0.08 ) & (     0.08 ) & (     0.13 ) & (     0.11 ) \\
& \textit{ 284 } & \textit{ 284 } & \textit{ 284 } & \textit{ 284 } & \textit{ 284 } & \textit{ 523 } & \textit{ 554 } \\
\bottomrule
\end{tabular}
}
\vspace{1ex} \\
\footnotesize\raggedright{Note: This table shows the estimates of the coefficient for attending Reggio Approach preschools from multiple methods. We compare Reggio Approach people with people who attended religious preschools in Reggio. Column title indicates the corresponding control set and and model. ``None'' refers to the OLS estimate with no control variables. ``BIC'' refers to the OLS estimate with controls selected by Bayesian Information Criterion (BIC) and additional controls for caregiver's religion. ``Full'' refers to the OLS estimate with the full set of controls. ``PSM" refers to propensity score matching estimation. ``AIPW" refers to augmented inverse propensity weighting estimation. ``DidPmIt'' refers to the difference-in-difference estimate of (Reggio Muni - Parma Muni) - (Reggio Other - Parma Other). ``DidPv'' refers to the difference-in-difference estimate of (Reggio Muni - Padova Muni) - (Reggio Other - Padova Other). Bold number shows that the estimate is statistically significant at the 10\% level. Number of observations used in estimation is reported in italic.}
\end{table}


\begin{table}[H] \caption{Estimation Results for Health Outcomes, Comparison to Non-RA Preschools,  Adolescent Cohort} \label{ols-H-adol-reg-reli}
\scalebox{0.8}{\begin{tabular}{l c c c c c c c c c}
\toprule
 & None & Bic & Full & PSM & AIPW & DidPm & PSMPm & DidPv & PSMPv \\
\midrule
Not Obese & \textbf{ -0.08 } & \textbf{ -0.11 } & \textbf{ -0.09 } & \textbf{-0.07} & -0.08 & 0.02 & \textbf{-0.07} & -0.09 & 0.07 \\
& (0.04) & (0.05) & (0.04) & (0.04) & (0.04) & (0.06) & (0.04) & (0.07) & (0.05) \\
& \textit{ 285 } & \textit{ 285 } & \textit{ 285 } & \textit{ 285 } & \textit{ 285 } & \textit{ 524 } & \textit{ 396 } & \textit{ 559 } & \textit{ 431 } \\
Not Overweight & 0.01 & -0.02 & -0.00 & -0.03 & -0.02 & \textbf{ 0.08 } & 0.01 & -0.03 & -0.03 \\
& (0.02) & (0.03) & (0.02) & (0.03) & (0.03) & (0.04) & (0.03) & (0.03) & (0.02) \\
& \textit{ 285 } & \textit{ 285 } & \textit{ 285 } & \textit{ 285 } & \textit{ 285 } & \textit{ 524 } & \textit{ 396 } & \textit{ 559 } & \textit{ 431 } \\
Health is Good & 0.06 & 0.07 & 0.09 & 0.05 & 0.06 & 0.10 & \textbf{0.16} & \textbf{ 0.13 } & 0.04 \\
& (0.06) & (0.06) & (0.06) & (0.07) & (0.06) & (0.09) & (0.06) & (0.09) & (0.07) \\
& \textit{ 284 } & \textit{ 284 } & \textit{ 284 } & \textit{ 284 } & \textit{ 284 } & \textit{ 523 } & \textit{ 396 } & \textit{ 558 } & \textit{ 431 } \\
Number of Sick Days & 0.02 & -0.02 & 0.00 & -0.01 & -0.02 & -0.18 & -0.01 & 0.15 & -0.04 \\
& (0.10) & (0.11) & (0.10) & (0.11) & (0.11) & (0.14) & (0.09) & (0.14) & (0.10) \\
& \textit{ 285 } & \textit{ 285 } & \textit{ 285 } & \textit{ 285 } & \textit{ 285 } & \textit{ 521 } & \textit{ 393 } & \textit{ 546 } & \textit{ 418 } \\
Ever Suspended from School & 0.02 & 0.01 & 0.03 & 0.03 & -0.00 & 0.04 & 0.06 & 0.00 & 0.05 \\
& (0.03) & (0.03) & (0.03) & (0.04) & (0.03) & (0.04) & (0.04) & (0.04) & (0.05) \\
& \textit{ 285 } & \textit{ 285 } & \textit{ 285 } & \textit{ 285 } & \textit{ 285 } & \textit{ 524 } & \textit{ 396 } & \textit{ 559 } & \textit{ 431 } \\
\bottomrule
\end{tabular}
}
\vspace{1ex} \\
\footnotesize\raggedright{Note: This table shows the estimates of the coefficient for attending Reggio Approach preschools from multiple methods. We compare Reggio Approach people with people who attended religious preschools in Reggio. Column title indicates the corresponding control set and and model. ``None'' refers to the OLS estimate with no control variables. ``BIC'' refers to the OLS estimate with controls selected by Bayesian Information Criterion (BIC) and additional controls for caregiver's religion. ``Full'' refers to the OLS estimate with the full set of controls. ``PSM" refers to propensity score matching estimation. ``AIPW" refers to augmented inverse propensity weighting estimation. ``DidPmIt'' refers to the difference-in-difference estimate of (Reggio Muni - Parma Muni) - (Reggio Other - Parma Other). ``DidPv'' refers to the difference-in-difference estimate of (Reggio Muni - Padova Muni) - (Reggio Other - Padova Other). Bold number shows that the estimate is statistically significant at the 10\% level. Number of observations used in estimation is reported in italic.}
\end{table}


\begin{table}[H] \caption{Estimation Results for Behavioral Outcomes, Comparison to Non-RA Preschools,  Adolescent Cohort} \label{ols-B-adol-reg-reli}
\scalebox{0.8}{\begin{tabular}{l c c c c c c c}
\toprule
 & None & Bic & Full & PSM & AIPW & DidPm & DidPv \\
\midrule
Not Excited to Learn &     -0.01 &     -0.01 &     -0.01 &     -0.00 &     -0.01 &     -0.04 &      0.02 \\
& (     0.02 ) & (     0.02 ) & (     0.02 ) & (     0.02 ) & (     0.02 ) & (     0.04 ) & (     0.03 ) \\
& \textit{ 285 } & \textit{ 285 } & \textit{ 285 } & \textit{ 285 } & \textit{ 285 } & \textit{ 524 } & \textit{ 559 } \\
Problems Sitting Still &      0.00 &      0.03 &      0.01 &      0.02 &      0.02 &     -0.07 &      0.04 \\
& (     0.03 ) & (     0.04 ) & (     0.03 ) & (     0.03 ) & (     0.03 ) & (     0.07 ) & (     0.05 ) \\
& \textit{ 285 } & \textit{ 285 } & \textit{ 285 } & \textit{ 285 } & \textit{ 285 } & \textit{ 524 } & \textit{ 559 } \\
Go To School &      0.03 &      0.01 &      0.03 &     -0.01 &      0.01 &     -0.00 &      0.04 \\
& (     0.02 ) & (     0.02 ) & (     0.03 ) & (     0.02 ) & (     0.02 ) & (     0.03 ) & (     0.03 ) \\
& \textit{ 285 } & \textit{ 285 } & \textit{ 285 } & \textit{ 285 } & \textit{ 285 } & \textit{ 524 } & \textit{ 559 } \\
How Much Child Likes School &     -0.11 &     -0.09 &     -0.17 &     -0.08 &     -0.01 &      0.02 &     -0.10 \\
& (     0.11 ) & (     0.13 ) & (     0.12 ) & (     0.13 ) & (     0.12 ) & (     0.18 ) & (     0.16 ) \\
& \textit{ 272 } & \textit{ 272 } & \textit{ 272 } & \textit{ 272 } & \textit{ 272 } & \textit{ 502 } & \textit{ 541 } \\
Bothered by Migrants & \textbf{      0.25 } & \textbf{      0.20 } & \textbf{      0.22 } &      0.12 & \textbf{     0.20} & \textbf{      0.41 } &      0.14 \\
& (     0.11 ) & (     0.11 ) & (     0.11 ) & (     0.12 ) & (     0.10 ) & (     0.20 ) & (     0.15 ) \\
& \textit{ 282 } & \textit{ 282 } & \textit{ 282 } & \textit{ 282 } & \textit{ 282 } & \textit{ 512 } & \textit{ 546 } \\
Trust Score &      0.03 &     -0.07 &      0.04 &     -0.11 &     -0.10 &      0.30 &     -0.11 \\
& (     0.18 ) & (     0.20 ) & (     0.19 ) & (     0.20 ) & (     0.19 ) & (     0.30 ) & (     0.26 ) \\
& \textit{ 283 } & \textit{ 283 } & \textit{ 283 } & \textit{ 283 } & \textit{ 283 } & \textit{ 520 } & \textit{ 550 } \\
Days of Sport (Weekly) & \textbf{     -0.43 } & \textbf{     -0.48 } &     -0.33 &     -0.47 &     -0.44 & \textbf{     -0.82 } & \textbf{     -0.53 } \\
& (     0.23 ) & (     0.26 ) & (     0.26 ) & (     0.31 ) & (     0.23 ) & (     0.38 ) & (     0.34 ) \\
& \textit{ 279 } & \textit{ 279 } & \textit{ 279 } & \textit{ 279 } & \textit{ 279 } & \textit{ 510 } & \textit{ 534 } \\
\bottomrule
\end{tabular}
}
\vspace{1ex} \\
\footnotesize\raggedright{Note: This table shows the estimates of the coefficient for attending Reggio Approach preschools from multiple methods. We compare Reggio Approach people with people who attended religious preschools in Reggio. Column title indicates the corresponding control set and and model. ``None'' refers to the OLS estimate with no control variables. ``BIC'' refers to the OLS estimate with controls selected by Bayesian Information Criterion (BIC) and additional controls for caregiver's religion. ``Full'' refers to the OLS estimate with the full set of controls. ``PSM" refers to propensity score matching estimation. ``AIPW" refers to augmented inverse propensity weighting estimation. ``DidPmIt'' refers to the difference-in-difference estimate of (Reggio Muni - Parma Muni) - (Reggio Other - Parma Other). ``DidPv'' refers to the difference-in-difference estimate of (Reggio Muni - Padova Muni) - (Reggio Other - Padova Other). Bold number shows that the estimate is statistically significant at the 10\% level. Number of observations used in estimation is reported in italic.}
\end{table}




\subsubsection{Adult 30s Cohort}
\begin{table}[H] \caption{Estimation Results for Cognitive and Education Outcomes, Comparison to Non-RA Preschools, Adult 30s Cohort} \label{ols-CN-adult30-reg-other}
\scalebox{0.8}{\begin{tabular}{l c c c c c c c}
\toprule
 & None & BIC & Full & PSM & AIPW & DidPm & DidPv \\
\midrule
IQ Factor &      0.01 &     -0.02 &      0.04 &     -0.12 &      0.03 &     -0.19 &      0.13 \\
& (     0.16 ) & (     0.15 ) & (     0.15 ) & (     0.23 ) & (     0.16 ) & (     0.20 ) & (     0.23 ) \\
& \textit{ 168 } & \textit{ 168 } & \textit{ 168 } & \textit{ 168 } & \textit{ 168 } & \textit{ 296 } & \textit{ 340 } \\
High School Grade &      1.05 &      0.55 &      0.66 &      1.32 &      0.57 &      4.75 &      0.82 \\
& (     1.53 ) & (     1.50 ) & (     1.55 ) & (     1.61 ) & (     1.50 ) & (     5.12 ) & (     3.70 ) \\
& \textit{ 129 } & \textit{ 129 } & \textit{ 129 } & \textit{ 129 } & \textit{ 129 } & \textit{ 244 } & \textit{ 264 } \\
Graduate from High School &     -0.05 &     -0.04 &     -0.06 &     -0.03 &     -0.06 & \textbf{      0.18 } & \textbf{     -0.12 } \\
& (     0.05 ) & (     0.05 ) & (     0.05 ) & (     0.05 ) & (     0.05 ) & (     0.11 ) & (     0.07 ) \\
& \textit{ 168 } & \textit{ 168 } & \textit{ 168 } & \textit{ 168 } & \textit{ 168 } & \textit{ 296 } & \textit{ 340 } \\
Max Edu: University &      0.02 &      0.01 &      0.00 &     -0.04 &     -0.02 &      0.14 &      0.19 \\
& (     0.06 ) & (     0.07 ) & (     0.07 ) & (     0.09 ) & (     0.08 ) & (     0.12 ) & (     0.14 ) \\
& \textit{ 168 } & \textit{ 168 } & \textit{ 168 } & \textit{ 168 } & \textit{ 168 } & \textit{ 296 } & \textit{ 340 } \\
\bottomrule
\end{tabular}
}
\vspace{1ex} \\
\footnotesize\raggedright{Note: This table shows the estimates of the coefficient for attending Reggio Approach preschools from multiple methods. We compare Reggio Approach people with people who attended religious preschools in Reggio. Column title indicates the corresponding control set and and model. ``None'' refers to the OLS estimate with no control variables. ``BIC'' refers to the OLS estimate with controls selected by Bayesian Information Criterion (BIC) and additional controls for caregiver's religion. ``Full'' refers to the OLS estimate with the full set of controls. ``PSM" refers to propensity score matching estimation. ``AIPW" refers to augmented inverse propensity weighting estimation. ``DidPmIt'' refers to the difference-in-difference estimate of (Reggio Muni - Parma Muni) - (Reggio Other - Parma Other). ``DidPv'' refers to the difference-in-difference estimate of (Reggio Muni - Padova Muni) - (Reggio Other - Padova Other). Bold number shows that the estimate is statistically significant at the 10\% level. Number of observations used in estimation is reported in italic.}
\end{table}

\begin{table}[H] \caption{Estimation Results for Cognitive and Education Outcomes, Comparison to No Preschool, Adult 30s Cohort} \label{ols-CN-adult30-reg-none}
\scalebox{0.8}{\begin{tabular}{l c c c c c c c c c}
\toprule
 & None & BIC & Full & PSM & AIPW & DidPm & PSMPm & DidPv & PSMPv \\
\midrule
IQ Factor & 0.14 & 0.03 & -0.05 & 0.15 & 0.05 & -0.24 & \textbf{-0.57} & -0.11 & \textbf{-0.28} \\
& (0.16) & (0.15) & (0.16) & (0.19) & (0.15) & (0.22) & (0.18) & (0.27) & (0.13) \\
& \textit{ 167 } & \textit{ 167 } & \textit{ 167 } & \textit{ 167 } & \textit{ 167 } & \textit{ 252 } & \textit{ 153 } & \textit{ 233 } & \textit{ 157 } \\
High School Grade & \textbf{ 4.54 } & \textbf{ 4.98 } & \textbf{ 4.62 } & \textbf{5.57} & \textbf{5.90} & 0.35 & \textbf{12.70} & 3.16 & \textbf{3.68} \\
& (2.01) & (2.13) & (2.26) & (1.98) & (1.75) & (4.46) & (2.56) & (4.19) & (2.19) \\
& \textit{ 123 } & \textit{ 123 } & \textit{ 123 } & \textit{ 123 } & \textit{ 123 } & \textit{ 194 } & \textit{ 118 } & \textit{ 176 } & \textit{ 118 } \\
Graduate from High School & -0.03 & 0.02 & 0.03 & 0.03 & 0.03 & 0.12 & 0.00 & -0.05 & -0.01 \\
& (0.05) & (0.05) & (0.05) & (0.07) & (0.05) & (0.09) & (0.09) & (0.09) & (0.05) \\
& \textit{ 167 } & \textit{ 167 } & \textit{ 167 } & \textit{ 167 } & \textit{ 167 } & \textit{ 252 } & \textit{ 153 } & \textit{ 233 } & \textit{ 157 } \\
Max Edu: University & -0.07 & -0.03 & -0.04 & -0.02 & -0.01 & 0.01 & \textbf{-0.16} & -0.15 & 0.03 \\
& (0.07) & (0.07) & (0.07) & (0.08) & (0.08) & (0.12) & (0.08) & (0.15) & (0.07) \\
& \textit{ 167 } & \textit{ 167 } & \textit{ 167 } & \textit{ 167 } & \textit{ 167 } & \textit{ 252 } & \textit{ 153 } & \textit{ 233 } & \textit{ 157 } \\
\bottomrule
\end{tabular}
}
\vspace{1ex} \\
\footnotesize\raggedright{Note: This table shows the estimates of the coefficient for attending Reggio Approach preschools from multiple methods. We compare Reggio Approach people with people who attended religious preschools in Reggio. Column title indicates the corresponding control set and and model. ``None'' refers to the OLS estimate with no control variables. ``BIC'' refers to the OLS estimate with controls selected by Bayesian Information Criterion (BIC) and additional controls for caregiver's religion. ``Full'' refers to the OLS estimate with the full set of controls. ``PSM" refers to propensity score matching estimation. ``AIPW" refers to augmented inverse propensity weighting estimation. ``DidPmIt'' refers to the difference-in-difference estimate of (Reggio Muni - Parma Muni) - (Reggio None - Parma None). ``DidPv'' refers to the difference-in-difference estimate of (Reggio Muni - Padova Muni) - (Reggio None - Padova None). Bold number shows that the estimate is statistically significant at the 10\% level. Number of observations used in estimation is reported in italic.}
\end{table}


\begin{table}[H] \caption{Estimation Results for Employment Outcomes, Comparison to Non-RA Preschools, Adult 30s Cohort} \label{ols-W-adult30-reg-other}
\scalebox{0.8}{\begin{tabular}{l c c c c c c c}
\toprule
 & None & BIC & Full & PSM & AIPW & DidPm & DidPv \\
\midrule
Employed &     -0.03 &     -0.03 &     -0.02 &     -0.02 &     -0.02 &      0.01 &     -0.05 \\
& (     0.03 ) & (     0.04 ) & (     0.04 ) & (     0.04 ) & (     0.04 ) & (     0.10 ) & (     0.09 ) \\
& \textit{ 168 } & \textit{ 168 } & \textit{ 168 } & \textit{ 168 } & \textit{ 168 } & \textit{ 296 } & \textit{ 340 } \\
Self-Employed &     -0.02 &     -0.05 &     -0.06 &     -0.04 &     -0.06 &     -0.11 &      0.07 \\
& (     0.05 ) & (     0.05 ) & (     0.06 ) & (     0.06 ) & (     0.06 ) & (     0.09 ) & (     0.06 ) \\
& \textit{ 164 } & \textit{ 164 } & \textit{ 164 } & \textit{ 164 } & \textit{ 164 } & \textit{ 292 } & \textit{ 334 } \\
Hours Worked Per Week &     -0.02 &      0.14 &      0.63 &     -0.32 &      0.54 &      0.06 &      0.30 \\
& (     1.91 ) & (     2.04 ) & (     2.11 ) & (     3.58 ) & (     2.85 ) & (     4.46 ) & (     3.95 ) \\
& \textit{ 138 } & \textit{ 138 } & \textit{ 138 } & \textit{ 138 } & \textit{ 138 } & \textit{ 263 } & \textit{ 306 } \\
Income: 5,000 Euros of Less &     -0.01 &     -0.01 &     -0.03 &     -0.01 &     -0.00 &     -0.01 &     -0.06 \\
& (     0.04 ) & (     0.04 ) & (     0.04 ) & (     0.08 ) & (     0.04 ) & (     0.04 ) & (     0.07 ) \\
& \textit{ 168 } & \textit{ 168 } & \textit{ 168 } & \textit{ 168 } & \textit{ 168 } & \textit{ 296 } & \textit{ 340 } \\
Income: 5,001-10,000 Euros &      0.01 &      0.02 &      0.02 &      0.02 &      0.01 &      0.01 &      0.02 \\
& (     0.01 ) & (     0.02 ) & (     0.02 ) & (     0.02 ) & (     0.01 ) & (     0.01 ) & (     0.01 ) \\
& \textit{ 168 } & \textit{ 168 } & \textit{ 168 } & \textit{ 168 } & \textit{ 168 } & \textit{ 296 } & \textit{ 340 } \\
Income: 10,001-25,000 Euros &     -0.03 &      0.01 &      0.01 &      0.01 &      0.03 &     -0.08 &     -0.01 \\
& (     0.07 ) & (     0.07 ) & (     0.08 ) & (     0.07 ) & (     0.08 ) & (     0.14 ) & (     0.14 ) \\
& \textit{ 168 } & \textit{ 168 } & \textit{ 168 } & \textit{ 168 } & \textit{ 168 } & \textit{ 296 } & \textit{ 340 } \\
Income: 25,001-50,000 Euros &     -0.07 & \textbf{     -0.12 } &     -0.11 &     -0.12 &     -0.15 &     -0.11 &     -0.10 \\
& (     0.08 ) & (     0.08 ) & (     0.08 ) & (     0.11 ) & (     0.08 ) & (     0.15 ) & (     0.14 ) \\
& \textit{ 168 } & \textit{ 168 } & \textit{ 168 } & \textit{ 168 } & \textit{ 168 } & \textit{ 296 } & \textit{ 340 } \\
Income: 50,001-100,000 Euros & \textbf{      0.10 } & \textbf{      0.10 } & \textbf{      0.11 } & \textbf{     0.10} & \textbf{     0.10} & \textbf{      0.18 } & \textbf{      0.14 } \\
& (     0.04 ) & (     0.04 ) & (     0.04 ) & (     0.04 ) & (     0.04 ) & (     0.05 ) & (     0.07 ) \\
& \textit{ 168 } & \textit{ 168 } & \textit{ 168 } & \textit{ 168 } & \textit{ 168 } & \textit{ 296 } & \textit{ 340 } \\
\bottomrule
\end{tabular}
}
\vspace{1ex} \\
\footnotesize\raggedright{Note: This table shows the estimates of the coefficient for attending Reggio Approach preschools from multiple methods. We compare Reggio Approach people with people who attended religious preschools in Reggio. Column title indicates the corresponding control set and and model. ``None'' refers to the OLS estimate with no control variables. ``BIC'' refers to the OLS estimate with controls selected by Bayesian Information Criterion (BIC) and additional controls for caregiver's religion. ``Full'' refers to the OLS estimate with the full set of controls. ``PSM" refers to propensity score matching estimation. ``AIPW" refers to augmented inverse propensity weighting estimation. ``DidPmIt'' refers to the difference-in-difference estimate of (Reggio Muni - Parma Muni) - (Reggio Other - Parma Other). ``DidPv'' refers to the difference-in-difference estimate of (Reggio Muni - Padova Muni) - (Reggio Other - Padova Other). Bold number shows that the estimate is statistically significant at the 10\% level. Number of observations used in estimation is reported in italic.}
\end{table}

\begin{table}[H] \caption{Estimation Results for Employment Outcomes, Comparison to No Preschool, Adult 30s Cohort} \label{ols-W-adult30-reg-none}
\scalebox{0.8}{\begin{tabular}{l c c c c c c c c c}
\toprule
 & None & BIC & Full & PSM & AIPW & DidPm & PSMPm & DidPv & PSMPv \\
\midrule
Employed & 0.04 & 0.02 & 0.04 & 0.05 & 0.01 & \textbf{ 0.14 } & -0.02 & 0.03 & 0.08 \\
& (0.05) & (0.05) & (0.05) & (0.05) & (0.04) & (0.09) & (0.03) & (0.10) & (0.08) \\
& \textit{ 167 } & \textit{ 167 } & \textit{ 167 } & \textit{ 167 } & \textit{ 167 } & \textit{ 252 } & \textit{ 153 } & \textit{ 233 } & \textit{ 157 } \\
Self-Employed & -0.07 & \textbf{ -0.10 } & -0.08 & -0.10 & -0.10 & -0.02 & -0.05 & -0.06 & 0.04 \\
& (0.06) & (0.06) & (0.06) & (0.07) & (0.06) & (0.09) & (0.09) & (0.07) & (0.04) \\
& \textit{ 159 } & \textit{ 159 } & \textit{ 159 } & \textit{ 159 } & \textit{ 159 } & \textit{ 243 } & \textit{ 149 } & \textit{ 224 } & \textit{ 154 } \\
Hours Worked Per Week & \textbf{ 6.84 } & \textbf{ 4.30 } & \textbf{ 5.16 } & 2.80 & \textbf{3.57} & \textbf{ 9.35 } & 1.75 & 5.25 & 2.77 \\
& (2.73) & (2.76) & (2.80) & (2.94) & (2.21) & (4.39) & (3.52) & (4.97) & (3.14) \\
& \textit{ 140 } & \textit{ 140 } & \textit{ 140 } & \textit{ 140 } & \textit{ 140 } & \textit{ 223 } & \textit{ 134 } & \textit{ 206 } & \textit{ 138 } \\
Income: 5,000 Euros of Less & \textbf{ 0.07 } & \textbf{ 0.08 } & \textbf{ 0.07 } & \textbf{0.07} & \textbf{0.08} & \textbf{ 0.08 } & \textbf{0.06} & 0.02 & \textbf{0.07} \\
& (0.02) & (0.03) & (0.03) & (0.03) & (0.02) & (0.03) & (0.02) & (0.07) & (0.03) \\
& \textit{ 167 } & \textit{ 167 } & \textit{ 167 } & \textit{ 167 } & \textit{ 167 } & \textit{ 252 } & \textit{ 153 } & \textit{ 233 } & \textit{ 157 } \\
Income: 5,001-10,000 Euros & -0.03 & -0.02 & -0.02 & -0.01 & -0.02 & 0.04 & -0.02 & -0.01 & 0.01 \\
& (0.03) & (0.03) & (0.02) & (0.03) & (0.02) & (0.04) & (0.02) & (0.03) & (0.01) \\
& \textit{ 167 } & \textit{ 167 } & \textit{ 167 } & \textit{ 167 } & \textit{ 167 } & \textit{ 252 } & \textit{ 153 } & \textit{ 233 } & \textit{ 157 } \\
Income: 10,001-25,000 Euros & \textbf{ -0.21 } & \textbf{ -0.21 } & \textbf{ -0.21 } & \textbf{-0.16} & -0.20 & \textbf{ -0.39 } & -0.12 & -0.05 & \textbf{-0.26} \\
& (0.08) & (0.08) & (0.08) & (0.10) & (0.10) & (0.14) & (0.10) & (0.16) & (0.10) \\
& \textit{ 167 } & \textit{ 167 } & \textit{ 167 } & \textit{ 167 } & \textit{ 167 } & \textit{ 252 } & \textit{ 153 } & \textit{ 233 } & \textit{ 157 } \\
Income: 25,001-50,000 Euros & 0.08 & 0.06 & 0.09 & 0.01 & 0.06 & 0.13 & -0.01 & 0.03 & 0.06 \\
& (0.08) & (0.09) & (0.09) & (0.10) & (0.09) & (0.14) & (0.11) & (0.16) & (0.10) \\
& \textit{ 167 } & \textit{ 167 } & \textit{ 167 } & \textit{ 167 } & \textit{ 167 } & \textit{ 252 } & \textit{ 153 } & \textit{ 233 } & \textit{ 157 } \\
Income: 50,001-100,000 Euros & \textbf{ 0.08 } & \textbf{ 0.08 } & \textbf{ 0.07 } & \textbf{0.09} & \textbf{0.08} & \textbf{ 0.14 } & \textbf{0.08} & 0.02 & \textbf{0.12} \\
& (0.04) & (0.04) & (0.04) & (0.04) & (0.04) & (0.06) & (0.03) & (0.07) & (0.03) \\
& \textit{ 167 } & \textit{ 167 } & \textit{ 167 } & \textit{ 167 } & \textit{ 167 } & \textit{ 252 } & \textit{ 153 } & \textit{ 233 } & \textit{ 157 } \\
\bottomrule
\end{tabular}
}
\vspace{1ex} \\
\footnotesize\raggedright{Note: This table shows the estimates of the coefficient for attending Reggio Approach preschools from multiple methods. We compare Reggio Approach people with people who attended religious preschools in Reggio. Column title indicates the corresponding control set and and model. ``None'' refers to the OLS estimate with no control variables. ``BIC'' refers to the OLS estimate with controls selected by Bayesian Information Criterion (BIC) and additional controls for caregiver's religion. ``Full'' refers to the OLS estimate with the full set of controls. ``PSM" refers to propensity score matching estimation. ``AIPW" refers to augmented inverse propensity weighting estimation. ``DidPmIt'' refers to the difference-in-difference estimate of (Reggio Muni - Parma Muni) - (Reggio None - Parma None). ``DidPv'' refers to the difference-in-difference estimate of (Reggio Muni - Padova Muni) - (Reggio None - Padova None). Bold number shows that the estimate is statistically significant at the 10\% level. Number of observations used in estimation is reported in italic.}
\end{table}


\begin{table}[H] \caption{Estimation Results for Living Environment Outcomes, Comparison to Non-RA Preschools, Adult 30s Cohort} \label{ols-W-adult30-reg-other}
\scalebox{0.8}{\begin{tabular}{l c c c c c c c c c}
\toprule
 & None & BIC & Full & PSM & AIPW & DidPm & PSMPm & DidPv & PSMPv \\
\midrule
Married or Cohabitating & 0.08 & 0.06 & 0.05 & -0.02 & 0.02 & 0.15 & -0.06 & 0.20 & \textbf{-0.14} \\
& (0.07) & (0.07) & (0.08) & (0.09) & (0.07) & (0.11) & (0.07) & (0.14) & (0.06) \\
& \textit{ 168 } & \textit{ 168 } & \textit{ 168 } & \textit{ 168 } & \textit{ 168 } & \textit{ 296 } & \textit{ 238 } & \textit{ 340 } & \textit{ 282 } \\
Divorced & -0.03 & -0.03 & -0.03 & -0.02 & -0.02 & \textbf{ -0.05 } & -0.01 & -0.02 & -0.02 \\
& (0.02) & (0.02) & (0.02) & (0.01) & (0.02) & (0.03) & (0.01) & (0.03) & (0.02) \\
& \textit{ 168 } & \textit{ 168 } & \textit{ 168 } & \textit{ 168 } & \textit{ 168 } & \textit{ 293 } & \textit{ 235 } & \textit{ 337 } & \textit{ 279 } \\
Num. of Children in House & 0.00 & 0.01 & -0.01 & -0.05 & 0.01 & 0.03 & -0.11 & \textbf{ 0.20 } & \textbf{-0.17} \\
& (0.05) & (0.05) & (0.05) & (0.07) & (0.05) & (0.12) & (0.09) & (0.13) & (0.08) \\
& \textit{ 168 } & \textit{ 168 } & \textit{ 168 } & \textit{ 168 } & \textit{ 168 } & \textit{ 296 } & \textit{ 238 } & \textit{ 340 } & \textit{ 282 } \\
Own House & 0.05 & 0.05 & 0.04 & 0.02 & 0.01 & 0.09 & \textbf{0.16} & \textbf{ 0.24 } & -0.00 \\
& (0.08) & (0.08) & (0.08) & (0.07) & (0.07) & (0.12) & (0.07) & (0.14) & (0.06) \\
& \textit{ 168 } & \textit{ 168 } & \textit{ 168 } & \textit{ 168 } & \textit{ 168 } & \textit{ 296 } & \textit{ 238 } & \textit{ 340 } & \textit{ 282 } \\
Live With Parents & -0.02 & -0.01 & 0.04 & -0.01 & 0.01 & -0.12 & -0.04 & -0.08 & \textbf{-0.16} \\
& (0.06) & (0.06) & (0.06) & (0.06) & (0.06) & (0.11) & (0.06) & (0.12) & (0.06) \\
& \textit{ 168 } & \textit{ 168 } & \textit{ 168 } & \textit{ 168 } & \textit{ 168 } & \textit{ 296 } & \textit{ 238 } & \textit{ 340 } & \textit{ 282 } \\
\bottomrule
\end{tabular}
}
\vspace{1ex} \\
\footnotesize\raggedright{Note: This table shows the estimates of the coefficient for attending Reggio Approach preschools from multiple methods. We compare Reggio Approach people with people who attended religious preschools in Reggio. Column title indicates the corresponding control set and and model. ``None'' refers to the OLS estimate with no control variables. ``BIC'' refers to the OLS estimate with controls selected by Bayesian Information Criterion (BIC) and additional controls for caregiver's religion. ``Full'' refers to the OLS estimate with the full set of controls. ``PSM" refers to propensity score matching estimation. ``AIPW" refers to augmented inverse propensity weighting estimation. ``DidPmIt'' refers to the difference-in-difference estimate of (Reggio Muni - Parma Muni) - (Reggio Other - Parma Other). ``DidPv'' refers to the difference-in-difference estimate of (Reggio Muni - Padova Muni) - (Reggio Other - Padova Other). Bold number shows that the estimate is statistically significant at the 10\% level. Number of observations used in estimation is reported in italic.}
\end{table}

\begin{table}[H] \caption{Estimation Results for Living Environment Outcomes, Comparison to No Preschool, Adult 30s Cohort} \label{ols-W-adult30-reg-none}
\scalebox{0.8}{\begin{tabular}{l c c c c c c c c c}
\toprule
 & None & BIC & Full & PSM & AIPW & DidPm & PSMPm & DidPv & PSMPv \\
\midrule
Married or Cohabitating & -0.01 & -0.08 & -0.10 & -0.05 & -0.07 & -0.09 & 0.04 & -0.01 & -0.05 \\
& (0.08) & (0.08) & (0.08) & (0.09) & (0.09) & (0.13) & (0.11) & (0.16) & (0.10) \\
& \textit{ 167 } & \textit{ 167 } & \textit{ 167 } & \textit{ 167 } & \textit{ 167 } & \textit{ 252 } & \textit{ 153 } & \textit{ 233 } & \textit{ 157 } \\
Divorced & 0 & 0 & 0 & 0 & \textbf{0.00} & 0.00 & -0.01 & 0 & 0 \\
& () & () & () & () & (0.00) & (0.03) & (0.01) & () & () \\
& 167 & 167 & 167 & 167 & \textit{ 167 } & \textit{ 251 } & \textit{ 152 } & 233 & 157 \\
Num. of Children in House & -0.01 & -0.02 & -0.05 & 0.00 & -0.02 & -0.17 & 0.08 & -0.03 & -0.07 \\
& (0.05) & (0.05) & (0.05) & (0.07) & (0.05) & (0.13) & (0.05) & (0.13) & (0.09) \\
& \textit{ 167 } & \textit{ 167 } & \textit{ 167 } & \textit{ 167 } & \textit{ 167 } & \textit{ 252 } & \textit{ 153 } & \textit{ 233 } & \textit{ 157 } \\
Own House & 0.06 & 0.09 & \textbf{ 0.13 } & 0.03 & 0.06 & 0.09 & 0.10 & 0.17 & -0.07 \\
& (0.08) & (0.08) & (0.09) & (0.09) & (0.09) & (0.14) & (0.12) & (0.16) & (0.08) \\
& \textit{ 167 } & \textit{ 167 } & \textit{ 167 } & \textit{ 167 } & \textit{ 167 } & \textit{ 252 } & \textit{ 153 } & \textit{ 233 } & \textit{ 157 } \\
Live With Parents & -0.00 & -0.03 & -0.01 & -0.05 & -0.05 & -0.13 & 0.07 & -0.12 & \textbf{-0.24} \\
& (0.06) & (0.06) & (0.06) & (0.08) & (0.07) & (0.11) & (0.05) & (0.14) & (0.10) \\
& \textit{ 167 } & \textit{ 167 } & \textit{ 167 } & \textit{ 167 } & \textit{ 167 } & \textit{ 252 } & \textit{ 153 } & \textit{ 233 } & \textit{ 157 } \\
\bottomrule
\end{tabular}
}
\vspace{1ex} \\
\footnotesize\raggedright{Note: This table shows the estimates of the coefficient for attending Reggio Approach preschools from multiple methods. We compare Reggio Approach people with people who attended religious preschools in Reggio. Column title indicates the corresponding control set and and model. ``None'' refers to the OLS estimate with no control variables. ``BIC'' refers to the OLS estimate with controls selected by Bayesian Information Criterion (BIC) and additional controls for caregiver's religion. ``Full'' refers to the OLS estimate with the full set of controls. ``PSM" refers to propensity score matching estimation. ``AIPW" refers to augmented inverse propensity weighting estimation. ``DidPmIt'' refers to the difference-in-difference estimate of (Reggio Muni - Parma Muni) - (Reggio None - Parma None). ``DidPv'' refers to the difference-in-difference estimate of (Reggio Muni - Padova Muni) - (Reggio None - Padova None). Bold number shows that the estimate is statistically significant at the 10\% level. Number of observations used in estimation is reported in italic.}
\end{table}


\begin{table}[H] \caption{Estimation Results for Health and Risk Outcomes, Comparison to Non-RA Preschools, Adult 30s Cohort} \label{ols-H-adult30-reg-other}
\scalebox{0.8}{\begin{tabular}{l c c c c c c c}
\toprule
 & None & BIC & Full & PSM & AIPW & DidPm & DidPv \\
\midrule
Tried Marijuana & \textbf{     -0.10 } & \textbf{     -0.10 } & \textbf{     -0.10 } &     -0.07 &     -0.09 &     -0.08 &     -0.15 \\
& (     0.07 ) & (     0.07 ) & (     0.06 ) & (     0.09 ) & (     0.08 ) & (     0.10 ) & (     0.12 ) \\
& \textit{ 168 } & \textit{ 168 } & \textit{ 168 } & \textit{ 168 } & \textit{ 168 } & \textit{ 296 } & \textit{ 340 } \\
Num. of Cigarettes Per Day &      0.24 &      1.09 &      1.46 &      0.00 &      0.43 &      0.56 &      2.44 \\
& (     1.24 ) & (     1.28 ) & (     1.35 ) & (     0.00 ) & (     1.18 ) & (     3.07 ) & (     4.19 ) \\
& \textit{ 63 } & \textit{ 63 } & \textit{ 63 } & \textit{ 0 } & \textit{ 63 } & \textit{ 109 } & \textit{ 105 } \\
BMI &      0.16 &     -0.13 &     -0.02 &      0.11 &     -0.17 & \textbf{     -1.63 } & \textbf{      1.02 } \\
& (     0.39 ) & (     0.36 ) & (     0.38 ) & (     0.33 ) & (     0.39 ) & (     0.69 ) & (     0.66 ) \\
& \textit{ 124 } & \textit{ 124 } & \textit{ 124 } & \textit{ 124 } & \textit{ 124 } & \textit{ 232 } & \textit{ 272 } \\
Obese &     -0.01 &     -0.00 &     -0.03 &      0.02 &     -0.03 &      0.14 &      0.03 \\
& (     0.07 ) & (     0.06 ) & (     0.06 ) & (     0.08 ) & (     0.06 ) & (     0.10 ) & (     0.11 ) \\
& \textit{ 168 } & \textit{ 168 } & \textit{ 168 } & \textit{ 168 } & \textit{ 168 } & \textit{ 296 } & \textit{ 340 } \\
Overweight &      0.06 &      0.01 &      0.02 &     -0.05 &      0.01 &     -0.17 &      0.04 \\
& (     0.07 ) & (     0.06 ) & (     0.06 ) & (     0.08 ) & (     0.07 ) & (     0.12 ) & (     0.11 ) \\
& \textit{ 168 } & \textit{ 168 } & \textit{ 168 } & \textit{ 168 } & \textit{ 168 } & \textit{ 296 } & \textit{ 340 } \\
Good Health &     -0.08 &     -0.08 &     -0.08 &     -0.06 &     -0.12 &     -0.10 &     -0.07 \\
& (     0.10 ) & (     0.09 ) & (     0.09 ) & (     0.07 ) & (     0.11 ) & (     0.18 ) & (     0.19 ) \\
& \textit{ 168 } & \textit{ 168 } & \textit{ 168 } & \textit{ 168 } & \textit{ 168 } & \textit{ 296 } & \textit{ 334 } \\
No Problematic Health Condition &      0.00 &     -0.01 &     -0.05 &     -0.02 &     -0.01 &      0.05 &     -0.00 \\
& (     0.08 ) & (     0.08 ) & (     0.08 ) & (     0.11 ) & (     0.10 ) & (     0.15 ) & (     0.15 ) \\
& \textit{ 158 } & \textit{ 158 } & \textit{ 158 } & \textit{ 158 } & \textit{ 158 } & \textit{ 281 } & \textit{ 325 } \\
Num. of Days Sick Past Month & \textbf{      0.12 } & \textbf{      0.15 } & \textbf{      0.20 } & \textbf{     0.19} & \textbf{     0.16} & \textbf{      0.24 } & \textbf{      0.23 } \\
& (     0.08 ) & (     0.08 ) & (     0.08 ) & (     0.08 ) & (     0.07 ) & (     0.09 ) & (     0.12 ) \\
& \textit{ 162 } & \textit{ 162 } & \textit{ 162 } & \textit{ 162 } & \textit{ 162 } & \textit{ 282 } & \textit{ 328 } \\
Ever Suspended from School &     -0.01 &     -0.02 &     -0.03 &     -0.00 &     -0.01 &      0.03 &     -0.11 \\
& (     0.04 ) & (     0.04 ) & (     0.04 ) & (     0.04 ) & (     0.04 ) & (     0.07 ) & (     0.08 ) \\
& \textit{ 168 } & \textit{ 168 } & \textit{ 168 } & \textit{ 168 } & \textit{ 168 } & \textit{ 296 } & \textit{ 340 } \\
Age At First Drink &      0.52 &      0.08 &      0.74 &      0.19 &     -0.35 & \textbf{     -3.22 } &     -1.80 \\
& (     1.34 ) & (     1.16 ) & (     1.10 ) & (     1.58 ) & (     1.29 ) & (     1.83 ) & (     1.91 ) \\
& \textit{ 163 } & \textit{ 163 } & \textit{ 163 } & \textit{ 163 } & \textit{ 163 } & \textit{ 287 } & \textit{ 331 } \\
\bottomrule
\end{tabular}
}
\vspace{1ex} \\
\footnotesize\raggedright{Note: This table shows the estimates of the coefficient for attending Reggio Approach preschools from multiple methods. We compare Reggio Approach people with people who attended religious preschools in Reggio. Column title indicates the corresponding control set and and model. ``None'' refers to the OLS estimate with no control variables. ``BIC'' refers to the OLS estimate with controls selected by Bayesian Information Criterion (BIC) and additional controls for caregiver's religion. ``Full'' refers to the OLS estimate with the full set of controls. ``PSM" refers to propensity score matching estimation. ``AIPW" refers to augmented inverse propensity weighting estimation. ``DidPmIt'' refers to the difference-in-difference estimate of (Reggio Muni - Parma Muni) - (Reggio Other - Parma Other). ``DidPv'' refers to the difference-in-difference estimate of (Reggio Muni - Padova Muni) - (Reggio Other - Padova Other). Bold number shows that the estimate is statistically significant at the 10\% level. Number of observations used in estimation is reported in italic.}
\end{table}

\begin{table}[H] \caption{Estimation Results for Health and Risk Outcomes, Comparison to No Preschool, Adult 30s Cohort} \label{ols-H-adult30-reg-none}
\scalebox{0.8}{\begin{tabular}{l c c c c c c c c c}
\toprule
 & None & BIC & Full & PSM & AIPW & DidPm & PSMPm & DidPv & PSMPv \\
\midrule
Tried Marijuana & 0.05 & 0.05 & 0.03 & 0.04 & \textbf{0.06} & -0.05 & 0.06 & -0.11 & 0.09 \\
& (0.05) & (0.05) & (0.05) & (0.06) & (0.05) & (0.08) & (0.08) & (0.12) & (0.06) \\
& \textit{ 167 } & \textit{ 167 } & \textit{ 167 } & \textit{ 167 } & \textit{ 167 } & \textit{ 252 } & \textit{ 153 } & \textit{ 233 } & \textit{ 157 } \\
Num. of Cigarettes Per Day & 0.85 & 0.86 & 1.13 & 1.04 & 0.93 & -0.59 & 1.71 & 1.65 & \textbf{6.59} \\
& (1.27) & (1.28) & (1.78) & (1.26) & (1.62) & (2.64) & (4.31) & (3.87) & (1.18) \\
& \textit{ 60 } & \textit{ 60 } & \textit{ 60 } & \textit{ 60 } & \textit{ 60 } & \textit{ 95 } & \textit{ 58 } & \textit{ 79 } & \textit{ 56 } \\
BMI & \textbf{ 1.06 } & \textbf{ 0.59 } & \textbf{ 0.69 } & 0.51 & \textbf{0.62} & -0.18 & -0.60 & \textbf{ 1.68 } & -0.22 \\
& (0.47) & (0.41) & (0.42) & (0.53) & (0.46) & (0.63) & (0.46) & (0.78) & (0.38) \\
& \textit{ 124 } & \textit{ 124 } & \textit{ 124 } & \textit{ 124 } & \textit{ 124 } & \textit{ 195 } & \textit{ 121 } & \textit{ 171 } & \textit{ 113 } \\
Not Obese & -0.00 & -0.06 & -0.06 & -0.09 & -0.07 & 0.03 & \textbf{-0.24} & \textbf{ -0.25 } & 0.10 \\
& (0.07) & (0.06) & (0.06) & (0.07) & (0.06) & (0.11) & (0.05) & (0.14) & (0.09) \\
& \textit{ 167 } & \textit{ 167 } & \textit{ 167 } & \textit{ 167 } & \textit{ 167 } & \textit{ 252 } & \textit{ 153 } & \textit{ 233 } & \textit{ 157 } \\
Not Overweight & -0.07 & 0.01 & -0.02 & 0.03 & 0.01 & -0.00 & 0.15 & 0.00 & -0.07 \\
& (0.07) & (0.06) & (0.06) & (0.08) & (0.06) & (0.12) & (0.10) & (0.12) & (0.06) \\
& \textit{ 167 } & \textit{ 167 } & \textit{ 167 } & \textit{ 167 } & \textit{ 167 } & \textit{ 252 } & \textit{ 153 } & \textit{ 233 } & \textit{ 157 } \\
Good Health & \textbf{ 0.21 } & \textbf{ 0.16 } & \textbf{ 0.15 } & \textbf{0.15} & \textbf{0.16} & \textbf{ 0.28 } & \textbf{0.24} & 0.20 & \textbf{0.28} \\
& (0.08) & (0.07) & (0.08) & (0.08) & (0.07) & (0.15) & (0.12) & (0.20) & (0.12) \\
& \textit{ 167 } & \textit{ 167 } & \textit{ 167 } & \textit{ 167 } & \textit{ 167 } & \textit{ 252 } & \textit{ 153 } & \textit{ 230 } & \textit{ 156 } \\
No Problematic Health Condition & \textbf{ -0.24 } & \textbf{ -0.23 } & \textbf{ -0.21 } & \textbf{-0.19} & -0.24 & -0.15 & -0.02 & -0.22 & -0.08 \\
& (0.07) & (0.08) & (0.08) & (0.10) & (0.08) & (0.13) & (0.10) & (0.17) & (0.11) \\
& \textit{ 156 } & \textit{ 156 } & \textit{ 156 } & \textit{ 156 } & \textit{ 156 } & \textit{ 241 } & \textit{ 144 } & \textit{ 218 } & \textit{ 146 } \\
Num. of Days Sick Past Month & \textbf{ 0.18 } & \textbf{ 0.21 } & \textbf{ 0.18 } & \textbf{0.19} & \textbf{0.21} & \textbf{ 0.17 } & \textbf{0.29} & 0.16 & \textbf{0.32} \\
& (0.09) & (0.09) & (0.09) & (0.11) & (0.09) & (0.09) & (0.06) & (0.11) & (0.07) \\
& \textit{ 157 } & \textit{ 157 } & \textit{ 157 } & \textit{ 157 } & \textit{ 157 } & \textit{ 235 } & \textit{ 147 } & \textit{ 218 } & \textit{ 149 } \\
Ever Suspended from School & \textbf{ -0.10 } & \textbf{ -0.12 } & \textbf{ -0.15 } & \textbf{-0.17} & -0.12 & \textbf{ -0.11 } & -0.04 & -0.15 & -0.04 \\
& (0.05) & (0.05) & (0.05) & (0.07) & (0.06) & (0.07) & (0.06) & (0.11) & (0.07) \\
& \textit{ 167 } & \textit{ 167 } & \textit{ 167 } & \textit{ 167 } & \textit{ 167 } & \textit{ 252 } & \textit{ 153 } & \textit{ 233 } & \textit{ 157 } \\
Age At First Drink & 1.95 & 0.44 & -0.20 & 0.50 & 0.23 & 1.40 & \textbf{-3.23} & -2.64 & \textbf{-2.24} \\
& (1.38) & (1.28) & (1.24) & (1.37) & (1.39) & (1.94) & (1.79) & (2.30) & (1.09) \\
& \textit{ 159 } & \textit{ 159 } & \textit{ 159 } & \textit{ 159 } & \textit{ 159 } & \textit{ 238 } & \textit{ 147 } & \textit{ 224 } & \textit{ 154 } \\
\bottomrule
\end{tabular}
}
\vspace{1ex} \\
\footnotesize\raggedright{Note: This table shows the estimates of the coefficient for attending Reggio Approach preschools from multiple methods. We compare Reggio Approach people with people who attended religious preschools in Reggio. Column title indicates the corresponding control set and and model. ``None'' refers to the OLS estimate with no control variables. ``BIC'' refers to the OLS estimate with controls selected by Bayesian Information Criterion (BIC) and additional controls for caregiver's religion. ``Full'' refers to the OLS estimate with the full set of controls. ``PSM" refers to propensity score matching estimation. ``AIPW" refers to augmented inverse propensity weighting estimation. ``DidPmIt'' refers to the difference-in-difference estimate of (Reggio Muni - Parma Muni) - (Reggio None - Parma None). ``DidPv'' refers to the difference-in-difference estimate of (Reggio Muni - Padova Muni) - (Reggio None - Padova None). Bold number shows that the estimate is statistically significant at the 10\% level. Number of observations used in estimation is reported in italic.}
\end{table}


\begin{table}[H] \caption{Estimation Results for Noncognitive Outcomes, Comparison to Non-RA Preschools, Adult 30s Cohort} \label{ols-N-adult30-reg-other}
\scalebox{0.8}{\begin{tabular}{l c c c c c c c c c}
\toprule
 & None & BIC & Full & PSM & AIPW & DidPm & PSMPm & DidPv & PSMPv \\
\midrule
Locus of Control - positive & 0.11 & 0.08 & 0.06 & 0.07 & 0.09 & \textbf{ 0.35 } & \textbf{0.22} & 0.27 & \textbf{-0.28} \\
& (0.12) & (0.11) & (0.12) & (0.12) & (0.12) & (0.22) & (0.11) & (0.22) & (0.10) \\
& \textit{ 163 } & \textit{ 163 } & \textit{ 163 } & \textit{ 163 } & \textit{ 163 } & \textit{ 283 } & \textit{ 227 } & \textit{ 328 } & \textit{ 272 } \\
Depression Score - positive & 0.16 & -0.03 & 0.04 & -0.29 & -0.06 & 1.12 & \textbf{-2.53} & 0.60 & \textbf{-3.21} \\
& (1.03) & (0.69) & (0.68) & (0.86) & (0.76) & (1.34) & (0.86) & (1.67) & (0.66) \\
& \textit{ 166 } & \textit{ 166 } & \textit{ 166 } & \textit{ 166 } & \textit{ 166 } & \textit{ 294 } & \textit{ 237 } & \textit{ 335 } & \textit{ 278 } \\
Stress & -0.13 & -0.12 & \textbf{ -0.15 } & \textbf{-0.24} & -0.12 & -0.17 & 0.10 & \textbf{ -0.42 } & -0.04 \\
& (0.12) & (0.10) & (0.09) & (0.10) & (0.10) & (0.18) & (0.10) & (0.19) & (0.10) \\
& \textit{ 168 } & \textit{ 168 } & \textit{ 168 } & \textit{ 168 } & \textit{ 168 } & \textit{ 296 } & \textit{ 238 } & \textit{ 340 } & \textit{ 282 } \\
Work is Source of Stress & -0.06 & -0.06 & -0.03 & -0.14 & -0.03 & 0.11 & 0.08 & 0.19 & 0.08 \\
& (0.11) & (0.13) & (0.12) & (0.20) & (0.14) & (0.18) & (0.09) & (0.22) & (0.09) \\
& \textit{ 86 } & \textit{ 86 } & \textit{ 86 } & \textit{ 86 } & \textit{ 86 } & \textit{ 164 } & \textit{ 140 } & \textit{ 185 } & \textit{ 161 } \\
Satisfied with Income & 0.03 & 0.02 & 0.04 & 0.03 & 0.02 & \textbf{ 0.40 } & \textbf{0.50} & -0.07 & 0.12 \\
& (0.13) & (0.12) & (0.13) & (0.12) & (0.13) & (0.20) & (0.15) & (0.27) & (0.13) \\
& \textit{ 167 } & \textit{ 167 } & \textit{ 167 } & \textit{ 167 } & \textit{ 167 } & \textit{ 295 } & \textit{ 238 } & \textit{ 335 } & \textit{ 278 } \\
Satisfied with Work & 0.14 & 0.11 & 0.11 & 0.04 & 0.14 & \textbf{ 0.39 } & \textbf{0.35} & 0.26 & 0.02 \\
& (0.13) & (0.12) & (0.14) & (0.12) & (0.14) & (0.25) & (0.15) & (0.25) & (0.13) \\
& \textit{ 167 } & \textit{ 167 } & \textit{ 167 } & \textit{ 167 } & \textit{ 167 } & \textit{ 295 } & \textit{ 238 } & \textit{ 333 } & \textit{ 276 } \\
Satisfied with Health & \textbf{ -0.18 } & \textbf{ -0.21 } & \textbf{ -0.22 } & \textbf{-0.19} & -0.20 & -0.13 & -0.14 & \textbf{ -0.47 } & -0.04 \\
& (0.11) & (0.10) & (0.11) & (0.10) & (0.11) & (0.15) & (0.10) & (0.16) & (0.10) \\
& \textit{ 167 } & \textit{ 167 } & \textit{ 167 } & \textit{ 167 } & \textit{ 167 } & \textit{ 295 } & \textit{ 238 } & \textit{ 337 } & \textit{ 280 } \\
Satisfied with Family & 0.06 & 0.02 & -0.01 & -0.03 & -0.06 & \textbf{ 0.48 } & -0.08 & \textbf{ 0.60 } & \textbf{-0.23} \\
& (0.15) & (0.15) & (0.15) & (0.13) & (0.13) & (0.21) & (0.13) & (0.30) & (0.13) \\
& \textit{ 166 } & \textit{ 166 } & \textit{ 166 } & \textit{ 166 } & \textit{ 166 } & \textit{ 292 } & \textit{ 235 } & \textit{ 335 } & \textit{ 278 } \\
Optimistic Look in Life & \textbf{ 0.14 } & \textbf{ 0.13 } & \textbf{ 0.13 } & 0.07 & \textbf{0.17} & -0.02 & -0.06 & \textbf{ -0.22 } & -0.03 \\
& (0.08) & (0.08) & (0.09) & (0.10) & (0.07) & (0.13) & (0.07) & (0.11) & (0.09) \\
& \textit{ 155 } & \textit{ 155 } & \textit{ 155 } & \textit{ 155 } & \textit{ 155 } & \textit{ 271 } & \textit{ 217 } & \textit{ 304 } & \textit{ 250 } \\
Positive Reciprocity & -0.05 & -0.02 & 0.04 & -0.04 & -0.00 & -0.15 & \textbf{-0.16} & -0.13 & -0.08 \\
& (0.10) & (0.09) & (0.09) & (0.09) & (0.09) & (0.14) & (0.08) & (0.21) & (0.10) \\
& \textit{ 167 } & \textit{ 167 } & \textit{ 167 } & \textit{ 167 } & \textit{ 167 } & \textit{ 295 } & \textit{ 238 } & \textit{ 337 } & \textit{ 280 } \\
Negative Reciprocity & -0.08 & -0.04 & -0.04 & 0.03 & -0.12 & 0.00 & \textbf{0.56} & 0.41 & \textbf{0.66} \\
& (0.15) & (0.14) & (0.15) & (0.17) & (0.13) & (0.24) & (0.16) & (0.33) & (0.14) \\
& \textit{ 167 } & \textit{ 167 } & \textit{ 167 } & \textit{ 167 } & \textit{ 167 } & \textit{ 295 } & \textit{ 238 } & \textit{ 337 } & \textit{ 280 } \\
\bottomrule
\end{tabular}
}
\vspace{1ex} \\
\footnotesize\raggedright{Note: This table shows the estimates of the coefficient for attending Reggio Approach preschools from multiple methods. We compare Reggio Approach people with people who attended religious preschools in Reggio. Column title indicates the corresponding control set and and model. ``None'' refers to the OLS estimate with no control variables. ``BIC'' refers to the OLS estimate with controls selected by Bayesian Information Criterion (BIC) and additional controls for caregiver's religion. ``Full'' refers to the OLS estimate with the full set of controls. ``PSM" refers to propensity score matching estimation. ``AIPW" refers to augmented inverse propensity weighting estimation. ``DidPmIt'' refers to the difference-in-difference estimate of (Reggio Muni - Parma Muni) - (Reggio Other - Parma Other). ``DidPv'' refers to the difference-in-difference estimate of (Reggio Muni - Padova Muni) - (Reggio Other - Padova Other). Bold number shows that the estimate is statistically significant at the 10\% level. Number of observations used in estimation is reported in italic.}
\end{table}

\begin{table}[H] \caption{Estimation Results for Noncognitive Outcomes, Comparison to No Preschool, Adult 30s Cohort} \label{ols-N-adult30-reg-none}
\scalebox{0.8}{\begin{tabular}{l c c c c c c}
\toprule
 & None & BIC & Full & AIPW & DidPm & DidPv \\
\midrule
Locus of Control - positive &      0.07 &     -0.05 &     -0.08 &     -0.04 &     -0.33 &      0.04 \\
& (     0.14 ) & (     0.13 ) & (     0.14 ) & (     0.13 ) & (     0.31 ) & (     0.27 ) \\
& \textit{ 163 } & \textit{ 163 } & \textit{ 163 } & \textit{ 163 } & \textit{ 219 } & \textit{ 221 } \\
Depression Score - positive &      1.26 &     -0.09 &     -0.20 &     -0.01 &      0.76 &     -0.59 \\
& (     0.97 ) & (     0.85 ) & (     0.91 ) & (     0.85 ) & (     1.82 ) & (     1.94 ) \\
& \textit{ 165 } & \textit{ 165 } & \textit{ 165 } & \textit{ 165 } & \textit{ 228 } & \textit{ 230 } \\
Stress &      0.09 &      0.04 &      0.05 &      0.04 &     -0.10 &     -0.18 \\
& (     0.12 ) & (     0.11 ) & (     0.12 ) & (     0.11 ) & (     0.25 ) & (     0.21 ) \\
& \textit{ 167 } & \textit{ 167 } & \textit{ 167 } & \textit{ 167 } & \textit{ 230 } & \textit{ 233 } \\
Work is Source of Stress &      0.05 &      0.00 &      0.02 &      0.01 & \textbf{      0.38 } &     -0.01 \\
& (     0.10 ) & (     0.10 ) & (     0.10 ) & (     0.10 ) & (     0.21 ) & (     0.24 ) \\
& \textit{ 97 } & \textit{ 97 } & \textit{ 97 } & \textit{ 97 } & \textit{ 133 } & \textit{ 133 } \\
Satisfied with Income & \textbf{      0.29 } & \textbf{      0.31 } & \textbf{      0.30 } & \textbf{     0.31} & \textbf{      0.71 } &      0.24 \\
& (     0.15 ) & (     0.15 ) & (     0.15 ) & (     0.15 ) & (     0.28 ) & (     0.29 ) \\
& \textit{ 167 } & \textit{ 167 } & \textit{ 167 } & \textit{ 167 } & \textit{ 230 } & \textit{ 232 } \\
Satisfied with Work &      0.08 &      0.07 &      0.10 &      0.07 &      0.22 &     -0.04 \\
& (     0.15 ) & (     0.16 ) & (     0.16 ) & (     0.14 ) & (     0.32 ) & (     0.30 ) \\
& \textit{ 167 } & \textit{ 167 } & \textit{ 167 } & \textit{ 167 } & \textit{ 230 } & \textit{ 231 } \\
Satisfied with Health &     -0.04 &     -0.06 &     -0.08 &     -0.02 &     -0.13 & \textbf{     -0.47 } \\
& (     0.11 ) & (     0.12 ) & (     0.12 ) & (     0.12 ) & (     0.16 ) & (     0.18 ) \\
& \textit{ 167 } & \textit{ 167 } & \textit{ 167 } & \textit{ 167 } & \textit{ 230 } & \textit{ 232 } \\
Satisfied with Family &     -0.04 &     -0.12 &     -0.09 &     -0.12 &     -0.13 &      0.04 \\
& (     0.14 ) & (     0.14 ) & (     0.15 ) & (     0.14 ) & (     0.25 ) & (     0.32 ) \\
& \textit{ 166 } & \textit{ 166 } & \textit{ 166 } & \textit{ 166 } & \textit{ 227 } & \textit{ 231 } \\
Optimistic Look in Life & \textbf{     -0.19 } & \textbf{     -0.18 } & \textbf{     -0.15 } &     -0.20 & \textbf{     -0.30 } & \textbf{     -0.49 } \\
& (     0.08 ) & (     0.08 ) & (     0.09 ) & (     0.08 ) & (     0.17 ) & (     0.13 ) \\
& \textit{ 154 } & \textit{ 154 } & \textit{ 154 } & \textit{ 154 } & \textit{ 210 } & \textit{ 205 } \\
Positive Reciprocity &      0.02 &     -0.05 &     -0.04 &     -0.06 &      0.01 &     -0.28 \\
& (     0.12 ) & (     0.11 ) & (     0.13 ) & (     0.11 ) & (     0.15 ) & (     0.25 ) \\
& \textit{ 167 } & \textit{ 167 } & \textit{ 167 } & \textit{ 167 } & \textit{ 230 } & \textit{ 231 } \\
Negative Reciprocity & \textbf{      0.41 } & \textbf{      0.42 } & \textbf{      0.45 } & \textbf{     0.43} &      0.41 & \textbf{      0.55 } \\
& (     0.17 ) & (     0.17 ) & (     0.17 ) & (     0.17 ) & (     0.31 ) & (     0.34 ) \\
& \textit{ 167 } & \textit{ 167 } & \textit{ 167 } & \textit{ 167 } & \textit{ 230 } & \textit{ 231 } \\
\bottomrule
\end{tabular}
}
\vspace{1ex} \\
\footnotesize\raggedright{Note: This table shows the estimates of the coefficient for attending Reggio Approach preschools from multiple methods. We compare Reggio Approach people with people who attended religious preschools in Reggio. Column title indicates the corresponding control set and and model. ``None'' refers to the OLS estimate with no control variables. ``BIC'' refers to the OLS estimate with controls selected by Bayesian Information Criterion (BIC) and additional controls for caregiver's religion. ``Full'' refers to the OLS estimate with the full set of controls. ``PSM" refers to propensity score matching estimation. ``AIPW" refers to augmented inverse propensity weighting estimation. ``DidPmIt'' refers to the difference-in-difference estimate of (Reggio Muni - Parma Muni) - (Reggio None - Parma None). ``DidPv'' refers to the difference-in-difference estimate of (Reggio Muni - Padova Muni) - (Reggio None - Padova None). Bold number shows that the estimate is statistically significant at the 10\% level. Number of observations used in estimation is reported in italic.}
\end{table}


\begin{table}[H] \caption{Estimation Results for Social Outcomes, Comparison to Non-RA Preschools, Adult 30s Cohort} \label{ols-S-adult30-reg-other}
\scalebox{0.8}{\begin{tabular}{l c c c c c c c}
\toprule
 & None & BIC & Full & PSM & AIPW & DidPm & DidPv \\
\midrule
Bothered by Migrants & \textbf{      0.15 } & \textbf{      0.16 } & \textbf{      0.14 } &      0.12 & \textbf{     0.18} &      0.08 &      0.20 \\
& (     0.08 ) & (     0.08 ) & (     0.09 ) & (     0.08 ) & (     0.08 ) & (     0.20 ) & (     0.24 ) \\
& \textit{ 165 } & \textit{ 165 } & \textit{ 165 } & \textit{ 165 } & \textit{ 165 } & \textit{ 293 } & \textit{ 334 } \\
Num. of Friends &      0.73 &      0.61 &      0.86 &      1.34 &      0.54 &      1.87 &      1.26 \\
& (     0.97 ) & (     1.20 ) & (     1.38 ) & (     1.74 ) & (     1.07 ) & (     2.06 ) & (     1.90 ) \\
& \textit{ 150 } & \textit{ 150 } & \textit{ 150 } & \textit{ 150 } & \textit{ 150 } & \textit{ 277 } & \textit{ 311 } \\
Has Migrant Friends & \textbf{      0.13 } & \textbf{      0.14 } & \textbf{      0.12 } &      0.07 & \textbf{     0.12} &      0.17 & \textbf{      0.27 } \\
& (     0.07 ) & (     0.07 ) & (     0.08 ) & (     0.08 ) & (     0.08 ) & (     0.13 ) & (     0.14 ) \\
& \textit{ 168 } & \textit{ 168 } & \textit{ 168 } & \textit{ 168 } & \textit{ 168 } & \textit{ 296 } & \textit{ 340 } \\
Volunteers & \textbf{      0.11 } & \textbf{      0.11 } & \textbf{      0.11 } & \textbf{     0.10} & \textbf{     0.11} &     -0.09 &     -0.03 \\
& (     0.03 ) & (     0.03 ) & (     0.03 ) & (     0.03 ) & (     0.03 ) & (     0.12 ) & (     0.11 ) \\
& \textit{ 168 } & \textit{ 168 } & \textit{ 168 } & \textit{ 168 } & \textit{ 168 } & \textit{ 296 } & \textit{ 340 } \\
Ever Voted for Municipal &     -0.07 &     -0.03 &     -0.02 &      0.05 &     -0.01 &      0.06 & \textbf{      0.24 } \\
& (     0.08 ) & (     0.06 ) & (     0.07 ) & (     0.05 ) & (     0.06 ) & (     0.09 ) & (     0.11 ) \\
& \textit{ 166 } & \textit{ 166 } & \textit{ 166 } & \textit{ 166 } & \textit{ 166 } & \textit{ 292 } & \textit{ 328 } \\
Ever Voted for Regional &     -0.11 &     -0.07 &     -0.07 &     -0.01 &     -0.07 &      0.01 & \textbf{      0.29 } \\
& (     0.08 ) & (     0.06 ) & (     0.07 ) & (     0.06 ) & (     0.07 ) & (     0.09 ) & (     0.11 ) \\
& \textit{ 166 } & \textit{ 166 } & \textit{ 166 } & \textit{ 166 } & \textit{ 166 } & \textit{ 292 } & \textit{ 328 } \\
\bottomrule
\end{tabular}
}
\vspace{1ex} \\
\footnotesize\raggedright{Note: This table shows the estimates of the coefficient for attending Reggio Approach preschools from multiple methods. We compare Reggio Approach people with people who attended religious preschools in Reggio. Column title indicates the corresponding control set and and model. ``None'' refers to the OLS estimate with no control variables. ``BIC'' refers to the OLS estimate with controls selected by Bayesian Information Criterion (BIC) and additional controls for caregiver's religion. ``Full'' refers to the OLS estimate with the full set of controls. ``PSM" refers to propensity score matching estimation. ``AIPW" refers to augmented inverse propensity weighting estimation. ``DidPmIt'' refers to the difference-in-difference estimate of (Reggio Muni - Parma Muni) - (Reggio Other - Parma Other). ``DidPv'' refers to the difference-in-difference estimate of (Reggio Muni - Padova Muni) - (Reggio Other - Padova Other). Bold number shows that the estimate is statistically significant at the 10\% level. Number of observations used in estimation is reported in italic.}
\end{table}

\begin{table}[H] \caption{Estimation Results for Social Outcomes, Comparison to No Preschool, Adult 30s Cohort} \label{ols-S-adult30-reg-none}
\scalebox{0.8}{\begin{tabular}{l c c c c c c c c c}
\toprule
 & None & BIC & Full & PSM & AIPW & DidPm & PSMPm & DidPv & PSMPv \\
\midrule
Bothered by Migrants & -0.05 & -0.04 & -0.05 & -0.00 & -0.01 & 0.11 & \textbf{0.16} & -0.10 & \textbf{0.32} \\
& (0.09) & (0.09) & (0.10) & (0.09) & (0.08) & (0.18) & (0.08) & (0.25) & (0.09) \\
& \textit{ 167 } & \textit{ 167 } & \textit{ 167 } & \textit{ 167 } & \textit{ 167 } & \textit{ 252 } & \textit{ 153 } & \textit{ 231 } & \textit{ 156 } \\
Num. of Friends & 0.02 & 0.24 & 0.20 & 0.02 & 0.03 & 1.59 & \textbf{-2.62} & \textbf{ 4.74 } & -1.19 \\
& (1.44) & (1.55) & (1.80) & (2.15) & (1.70) & (2.14) & (1.56) & (2.67) & (1.91) \\
& \textit{ 148 } & \textit{ 148 } & \textit{ 148 } & \textit{ 148 } & \textit{ 148 } & \textit{ 233 } & \textit{ 139 } & \textit{ 213 } & \textit{ 143 } \\
Has Migrant Friends & 0.10 & 0.11 & 0.07 & \textbf{0.15} & \textbf{0.13} & 0.15 & \textbf{0.19} & 0.04 & \textbf{0.24} \\
& (0.07) & (0.08) & (0.08) & (0.09) & (0.08) & (0.13) & (0.09) & (0.15) & (0.10) \\
& \textit{ 167 } & \textit{ 167 } & \textit{ 167 } & \textit{ 167 } & \textit{ 167 } & \textit{ 252 } & \textit{ 153 } & \textit{ 233 } & \textit{ 157 } \\
Volunteers & -0.08 & \textbf{ -0.08 } & \textbf{ -0.08 } & -0.03 & -0.06 & \textbf{ -0.18 } & \textbf{-0.20} & \textbf{ -0.33 } & 0.05 \\
& (0.06) & (0.05) & (0.05) & (0.07) & (0.06) & (0.12) & (0.12) & (0.13) & (0.04) \\
& \textit{ 167 } & \textit{ 167 } & \textit{ 167 } & \textit{ 167 } & \textit{ 167 } & \textit{ 252 } & \textit{ 153 } & \textit{ 233 } & \textit{ 157 } \\
Ever Voted for Municipal & 0.10 & 0.03 & 0.04 & -0.07 & 0.02 & -0.05 & \textbf{0.22} & -0.01 & \textbf{0.27} \\
& (0.08) & (0.06) & (0.06) & (0.09) & (0.08) & (0.09) & (0.09) & (0.13) & (0.09) \\
& \textit{ 164 } & \textit{ 164 } & \textit{ 164 } & \textit{ 164 } & \textit{ 164 } & \textit{ 248 } & \textit{ 152 } & \textit{ 223 } & \textit{ 152 } \\
Ever Voted for Regional & 0.05 & -0.02 & -0.01 & -0.09 & -0.02 & -0.05 & \textbf{0.23} & 0.06 & \textbf{0.22} \\
& (0.08) & (0.07) & (0.07) & (0.09) & (0.06) & (0.09) & (0.09) & (0.13) & (0.09) \\
& \textit{ 164 } & \textit{ 164 } & \textit{ 164 } & \textit{ 164 } & \textit{ 164 } & \textit{ 248 } & \textit{ 152 } & \textit{ 223 } & \textit{ 152 } \\
\bottomrule
\end{tabular}
}
\vspace{1ex} \\
\footnotesize\raggedright{Note: This table shows the estimates of the coefficient for attending Reggio Approach preschools from multiple methods. We compare Reggio Approach people with people who attended religious preschools in Reggio. Column title indicates the corresponding control set and and model. ``None'' refers to the OLS estimate with no control variables. ``BIC'' refers to the OLS estimate with controls selected by Bayesian Information Criterion (BIC) and additional controls for caregiver's religion. ``Full'' refers to the OLS estimate with the full set of controls. ``PSM" refers to propensity score matching estimation. ``AIPW" refers to augmented inverse propensity weighting estimation. ``DidPmIt'' refers to the difference-in-difference estimate of (Reggio Muni - Parma Muni) - (Reggio None - Parma None). ``DidPv'' refers to the difference-in-difference estimate of (Reggio Muni - Padova Muni) - (Reggio None - Padova None). Bold number shows that the estimate is statistically significant at the 10\% level. Number of observations used in estimation is reported in italic.}
\end{table}













\subsubsection{Adult 40s Cohort}
\begin{table}[H] \caption{Estimation Results for Cognitive and Education Outcomes, Comparison to Non-RA Preschools, Adult 40s Cohort} \label{ols-CN-adult40-reg-other}
\scalebox{0.8}{\begin{tabular}{l c c c c c c c}
\toprule
 & None & BIC & Full & PSM & AIPW & PSMPm & PSMPv \\
\midrule
IQ Factor & -0.15 & -0.12 & -0.14 & -0.11 & -0.18 & \textbf{-0.30} & \textbf{-0.25} \\
& (0.12) & (0.11) & (0.11) & (0.12) & (0.11) & (0.12) & (0.14) \\
& \textit{ 159 } & \textit{ 159 } & \textit{ 159 } & \textit{ 159 } & \textit{ 159 } & \textit{ 197 } & \textit{ 239 } \\
High School Grade & -0.66 & -0.09 & 0.36 & -0.84 & 0.53 & \textbf{3.74} & \textbf{5.91} \\
& (1.56) & (1.65) & (1.71) & (1.64) & (2.11) & (1.81) & (1.67) \\
& \textit{ 117 } & \textit{ 117 } & \textit{ 117 } & \textit{ 117 } & \textit{ 117 } & \textit{ 161 } & \textit{ 188 } \\
Graduate from High School & \textbf{ 0.13 } & \textbf{ 0.10 } & \textbf{ 0.12 } & 0.09 & 0.02 & 0.05 & 0.05 \\
& (0.07) & (0.07) & (0.07) & (0.07) & (0.06) & (0.05) & (0.04) \\
& \textit{ 159 } & \textit{ 159 } & \textit{ 159 } & \textit{ 159 } & \textit{ 159 } & \textit{ 197 } & \textit{ 239 } \\
Max Edu: University & 0.07 & 0.05 & 0.03 & 0.01 & 0.04 & \textbf{-0.15} & \textbf{-0.12} \\
& (0.06) & (0.05) & (0.05) & (0.07) & (0.06) & (0.07) & (0.06) \\
& \textit{ 159 } & \textit{ 159 } & \textit{ 159 } & \textit{ 159 } & \textit{ 159 } & \textit{ 197 } & \textit{ 239 } \\
\bottomrule
\end{tabular}
}
\vspace{1ex} \\
\footnotesize\raggedright{Note: This table shows the estimates of the coefficient for attending Reggio Approach preschools from multiple methods. We compare Reggio Approach people with people who attended religious preschools in Reggio. Column title indicates the corresponding control set and and model. ``None'' refers to the OLS estimate with no control variables. ``BIC'' refers to the OLS estimate with controls selected by Bayesian Information Criterion (BIC) and additional controls for caregiver's religion. ``Full'' refers to the OLS estimate with the full set of controls. ``PSM" refers to propensity score matching estimation. ``AIPW" refers to augmented inverse propensity weighting estimation. ``DidPmIt'' refers to the difference-in-difference estimate of (Reggio Muni - Parma Muni) - (Reggio Other - Parma Other). ``DidPv'' refers to the difference-in-difference estimate of (Reggio Muni - Padova Muni) - (Reggio Other - Padova Other). Bold number shows that the estimate is statistically significant at the 10\% level. Number of observations used in estimation is reported in italic.}
\end{table}

\begin{table}[H] \caption{Estimation Results for Cognitive and Education Outcomes, Comparison to No Preschool, Adult 40s Cohort} \label{ols-CN-adult40-reg-none}
\scalebox{0.8}{\begin{tabular}{l c c c c c c c}
\toprule
 & None & BIC & Full & PSM & AIPW & DidPm & DidPv \\
\midrule
IQ Factor &      0.01 &      0.01 &      0.04 &      0.07 &      0.00 &      0.18 &      0.09 \\
& (     0.13 ) & (     0.14 ) & (     0.16 ) & (     0.16 ) & (     0.16 ) & (     0.16 ) & (     0.15 ) \\
& \textit{ 170 } & \textit{ 170 } & \textit{ 170 } & \textit{ 170 } & \textit{ 170 } & \textit{ 382 } & \textit{ 375 } \\
High School Grade &      0.59 &      1.65 &      1.77 &      1.40 &      1.51 & \textbf{     -4.02 } &      2.46 \\
& (     1.51 ) & (     1.59 ) & (     1.86 ) & (     1.75 ) & (     1.55 ) & (     2.50 ) & (     2.33 ) \\
& \textit{ 135 } & \textit{ 135 } & \textit{ 135 } & \textit{ 135 } & \textit{ 135 } & \textit{ 311 } & \textit{ 297 } \\
Graduate from High School &     -0.07 &     -0.01 &     -0.06 &     -0.00 &     -0.01 &      0.03 &     -0.09 \\
& (     0.05 ) & (     0.05 ) & (     0.06 ) & (     0.05 ) & (     0.06 ) & (     0.07 ) & (     0.07 ) \\
& \textit{ 170 } & \textit{ 170 } & \textit{ 170 } & \textit{ 170 } & \textit{ 170 } & \textit{ 382 } & \textit{ 375 } \\
Max Edu: University &      0.01 &      0.06 & \textbf{      0.11 } &      0.06 &      0.06 &     -0.11 &     -0.06 \\
& (     0.06 ) & (     0.06 ) & (     0.06 ) & (     0.07 ) & (     0.07 ) & (     0.08 ) & (     0.09 ) \\
& \textit{ 170 } & \textit{ 170 } & \textit{ 170 } & \textit{ 170 } & \textit{ 170 } & \textit{ 382 } & \textit{ 375 } \\
\bottomrule
\end{tabular}
}
\vspace{1ex} \\
\footnotesize\raggedright{Note: This table shows the estimates of the coefficient for attending Reggio Approach preschools from multiple methods. We compare Reggio Approach people with people who attended religious preschools in Reggio. Column title indicates the corresponding control set and and model. ``None'' refers to the OLS estimate with no control variables. ``BIC'' refers to the OLS estimate with controls selected by Bayesian Information Criterion (BIC) and additional controls for caregiver's religion. ``Full'' refers to the OLS estimate with the full set of controls. ``PSM" refers to propensity score matching estimation. ``AIPW" refers to augmented inverse propensity weighting estimation. ``DidPmIt'' refers to the difference-in-difference estimate of (Reggio Muni - Parma Muni) - (Reggio None - Parma None). ``DidPv'' refers to the difference-in-difference estimate of (Reggio Muni - Padova Muni) - (Reggio None - Padova None). Bold number shows that the estimate is statistically significant at the 10\% level. Number of observations used in estimation is reported in italic.}
\end{table}


\begin{table}[H] \caption{Estimation Results for Employment Outcomes, Comparison to Non-RA Preschools, Adult 40s Cohort} \label{ols-W-adult40-reg-other}
\scalebox{0.8}{\begin{tabular}{l c c c c c}
\toprule
 & None & BIC & Full & PSM & AIPW \\
\midrule
Employed &      0.01 &      0.01 &      0.01 &      0.04 &      0.03 \\
& (     0.03 ) & (     0.04 ) & (     0.04 ) & (     0.05 ) & (     0.04 ) \\
& \textit{ 159 } & \textit{ 159 } & \textit{ 159 } & \textit{ 159 } & \textit{ 159 } \\
Self-Employed & \textbf{     -0.10 } & \textbf{     -0.11 } & \textbf{     -0.12 } &     -0.08 &     -0.07 \\
& (     0.06 ) & (     0.06 ) & (     0.06 ) & (     0.06 ) & (     0.06 ) \\
& \textit{ 156 } & \textit{ 156 } & \textit{ 156 } & \textit{ 156 } & \textit{ 156 } \\
Hours Worked Per Week &     -0.90 &     -1.02 &     -1.28 &     -1.09 &      0.02 \\
& (     1.93 ) & (     2.09 ) & (     2.20 ) & (     2.25 ) & (     2.35 ) \\
& \textit{ 144 } & \textit{ 144 } & \textit{ 144 } & \textit{ 144 } & \textit{ 144 } \\
Income: 5,000 Euros of Less &     -0.01 &     -0.02 &     -0.02 &     -0.01 &     -0.02 \\
& (     0.01 ) & (     0.02 ) & (     0.02 ) & (     0.02 ) & (     0.02 ) \\
& \textit{ 159 } & \textit{ 159 } & \textit{ 159 } & \textit{ 159 } & \textit{ 159 } \\
Income: 5,001-10,000 Euros &     -0.01 &     -0.02 &     -0.02 &     -0.01 &     -0.02 \\
& (     0.01 ) & (     0.02 ) & (     0.02 ) & (     0.01 ) & (     0.02 ) \\
& \textit{ 159 } & \textit{ 159 } & \textit{ 159 } & \textit{ 159 } & \textit{ 159 } \\
Income: 10,001-25,000 Euros &     -0.04 &     -0.03 &     -0.02 &     -0.05 &     -0.04 \\
& (     0.07 ) & (     0.07 ) & (     0.08 ) & (     0.07 ) & (     0.07 ) \\
& \textit{ 159 } & \textit{ 159 } & \textit{ 159 } & \textit{ 159 } & \textit{ 159 } \\
Income: 25,001-50,000 Euros &      0.11 &      0.11 &      0.12 &      0.11 & \textbf{     0.18} \\
& (     0.08 ) & (     0.08 ) & (     0.09 ) & (     0.09 ) & (     0.09 ) \\
& \textit{ 159 } & \textit{ 159 } & \textit{ 159 } & \textit{ 159 } & \textit{ 159 } \\
Income: 50,001-100,000 Euros &     -0.01 &     -0.02 &     -0.03 &     -0.01 &     -0.06 \\
& (     0.05 ) & (     0.05 ) & (     0.05 ) & (     0.06 ) & (     0.05 ) \\
& \textit{ 159 } & \textit{ 159 } & \textit{ 159 } & \textit{ 159 } & \textit{ 159 } \\
Income: 100,001-250,000 Euros &     -0.02 &     -0.02 &     -0.04 &     -0.04 &     -0.03 \\
& (     0.03 ) & (     0.04 ) & (     0.04 ) & (     0.04 ) & (     0.04 ) \\
& \textit{ 159 } & \textit{ 159 } & \textit{ 159 } & \textit{ 159 } & \textit{ 159 } \\
\bottomrule
\end{tabular}
}
\vspace{1ex} \\
\footnotesize\raggedright{Note: This table shows the estimates of the coefficient for attending Reggio Approach preschools from multiple methods. We compare Reggio Approach people with people who attended religious preschools in Reggio. Column title indicates the corresponding control set and and model. ``None'' refers to the OLS estimate with no control variables. ``BIC'' refers to the OLS estimate with controls selected by Bayesian Information Criterion (BIC) and additional controls for caregiver's religion. ``Full'' refers to the OLS estimate with the full set of controls. ``PSM" refers to propensity score matching estimation. ``AIPW" refers to augmented inverse propensity weighting estimation. ``DidPmIt'' refers to the difference-in-difference estimate of (Reggio Muni - Parma Muni) - (Reggio Other - Parma Other). ``DidPv'' refers to the difference-in-difference estimate of (Reggio Muni - Padova Muni) - (Reggio Other - Padova Other). Bold number shows that the estimate is statistically significant at the 10\% level. Number of observations used in estimation is reported in italic.}
\end{table}

\begin{table}[H] \caption{Estimation Results for Employment Outcomes, Comparison to No Preschool, Adult 40s Cohort} \label{ols-W-adult40-reg-none}
\scalebox{0.8}{\begin{tabular}{l c c c c c c c c c}
\toprule
 & None & BIC & Full & PSM & AIPW & DidPm & PSMPm & DidPv & PSMPv \\
\midrule
Employed & \textbf{ 0.06 } & \textbf{ 0.05 } & 0.05 & 0.06 & \textbf{0.05} & 0.03 & 0.00 & \textbf{ 0.09 } & 0.03 \\
& (0.04) & (0.04) & (0.03) & (0.04) & (0.03) & (0.05) & (0.03) & (0.06) & (0.03) \\
& \textit{ 169 } & \textit{ 169 } & \textit{ 169 } & \textit{ 169 } & \textit{ 169 } & \textit{ 356 } & \textit{ 205 } & \textit{ 374 } & \textit{ 165 } \\
Self-Employed & 0.00 & 0.00 & -0.00 & 0.01 & 0.00 & 0.03 & 0.02 & 0.03 & -0.01 \\
& (0.05) & (0.05) & (0.06) & (0.06) & (0.05) & (0.07) & (0.05) & (0.07) & (0.05) \\
& \textit{ 166 } & \textit{ 166 } & \textit{ 166 } & \textit{ 166 } & \textit{ 166 } & \textit{ 351 } & \textit{ 200 } & \textit{ 369 } & \textit{ 162 } \\
Hours Worked Per Week & \textbf{ 5.71 } & \textbf{ 6.51 } & \textbf{ 7.39 } & \textbf{7.43} & \textbf{6.46} & \textbf{ 4.67 } & 1.55 & \textbf{ 7.06 } & \textbf{4.22} \\
& (2.42) & (2.40) & (2.60) & (2.54) & (2.65) & (2.89) & (1.63) & (3.08) & (2.03) \\
& \textit{ 151 } & \textit{ 151 } & \textit{ 151 } & \textit{ 151 } & \textit{ 151 } & \textit{ 333 } & \textit{ 192 } & \textit{ 355 } & \textit{ 153 } \\
Income: 5,000 Euros of Less & -0.01 & -0.01 & -0.02 & -0.01 & -0.01 & -0.02 & -0.01 & -0.01 & 0 \\
& (0.01) & (0.01) & (0.02) & (0.01) & (0.01) & (0.02) & (0.01) & (0.01) & () \\
& \textit{ 170 } & \textit{ 170 } & \textit{ 170 } & \textit{ 170 } & \textit{ 170 } & \textit{ 357 } & \textit{ 205 } & \textit{ 375 } & 165 \\
Income: 5,001-10,000 Euros & 0 & 0 & 0 & 0 & \textbf{0.00} & 0.02 & -0.01 & \textbf{ -0.02 } & 0 \\
& () & () & () & () & (0.00) & (0.01) & (0.01) & (0.01) & () \\
& 170 & 170 & 170 & 170 & \textit{ 170 } & \textit{ 357 } & \textit{ 205 } & \textit{ 375 } & 165 \\
Income: 10,001-25,000 Euros & \textbf{ -0.12 } & -0.08 & \textbf{ -0.13 } & -0.09 & -0.08 & -0.04 & -0.11 & 0.06 & \textbf{-0.15} \\
& (0.07) & (0.07) & (0.08) & (0.08) & (0.07) & (0.10) & (0.07) & (0.10) & (0.08) \\
& \textit{ 170 } & \textit{ 170 } & \textit{ 170 } & \textit{ 170 } & \textit{ 170 } & \textit{ 357 } & \textit{ 205 } & \textit{ 375 } & \textit{ 165 } \\
Income: 25,001-50,000 Euros & 0.11 & 0.06 & 0.03 & 0.05 & 0.06 & -0.05 & 0.07 & 0.07 & 0.13 \\
& (0.08) & (0.07) & (0.08) & (0.09) & (0.07) & (0.11) & (0.08) & (0.11) & (0.09) \\
& \textit{ 170 } & \textit{ 170 } & \textit{ 170 } & \textit{ 170 } & \textit{ 170 } & \textit{ 357 } & \textit{ 205 } & \textit{ 375 } & \textit{ 165 } \\
Income: 50,001-100,000 Euros & 0.05 & \textbf{ 0.05 } & \textbf{ 0.09 } & \textbf{0.08} & 0.05 & \textbf{ 0.10 } & 0.05 & -0.05 & -0.00 \\
& (0.04) & (0.04) & (0.04) & (0.04) & (0.04) & (0.05) & (0.03) & (0.06) & (0.06) \\
& \textit{ 170 } & \textit{ 170 } & \textit{ 170 } & \textit{ 170 } & \textit{ 170 } & \textit{ 357 } & \textit{ 205 } & \textit{ 375 } & \textit{ 165 } \\
Income: 100,001-250,000 Euros & -0.03 & -0.02 & 0.03 & -0.03 & -0.02 & -0.01 & 0.01 & -0.05 & \textbf{0.02} \\
& (0.03) & (0.03) & (0.04) & (0.04) & (0.04) & (0.03) & (0.02) & (0.03) & (0.01) \\
& \textit{ 170 } & \textit{ 170 } & \textit{ 170 } & \textit{ 170 } & \textit{ 170 } & \textit{ 357 } & \textit{ 205 } & \textit{ 375 } & \textit{ 165 } \\
\bottomrule
\end{tabular}
}
\vspace{1ex} \\
\footnotesize\raggedright{Note: This table shows the estimates of the coefficient for attending Reggio Approach preschools from multiple methods. We compare Reggio Approach people with people who attended religious preschools in Reggio. Column title indicates the corresponding control set and and model. ``None'' refers to the OLS estimate with no control variables. ``BIC'' refers to the OLS estimate with controls selected by Bayesian Information Criterion (BIC) and additional controls for caregiver's religion. ``Full'' refers to the OLS estimate with the full set of controls. ``PSM" refers to propensity score matching estimation. ``AIPW" refers to augmented inverse propensity weighting estimation. ``DidPmIt'' refers to the difference-in-difference estimate of (Reggio Muni - Parma Muni) - (Reggio None - Parma None). ``DidPv'' refers to the difference-in-difference estimate of (Reggio Muni - Padova Muni) - (Reggio None - Padova None). Bold number shows that the estimate is statistically significant at the 10\% level. Number of observations used in estimation is reported in italic.}
\end{table}


\begin{table}[H] \caption{Estimation Results for Living Environment Outcomes, Comparison to Non-RA Preschools, Adult 40s Cohort} \label{ols-W-adult40-reg-other}
\scalebox{0.8}{\begin{tabular}{l c c c c c c c}
\toprule
 & None & BIC & Full & PSM & AIPW & PSMPm & PSMPv \\
\midrule
Married or Cohabitating & 0.03 & 0.02 & 0.02 & 0.01 & 0.01 & 0.10 & \textbf{0.11} \\
& (0.07) & (0.07) & (0.07) & (0.07) & (0.06) & (0.07) & (0.07) \\
& \textit{ 159 } & \textit{ 159 } & \textit{ 159 } & \textit{ 159 } & \textit{ 159 } & \textit{ 197 } & \textit{ 239 } \\
Divorced & -0.06 & -0.04 & -0.03 & -0.03 & -0.02 & -0.02 & -0.02 \\
& (0.05) & (0.05) & (0.05) & (0.04) & (0.05) & (0.04) & (0.04) \\
& \textit{ 159 } & \textit{ 159 } & \textit{ 159 } & \textit{ 159 } & \textit{ 159 } & \textit{ 197 } & \textit{ 239 } \\
Num. of Children in House & \textbf{ -0.21 } & \textbf{ -0.20 } & \textbf{ -0.21 } & \textbf{-0.20} & -0.22 & -0.06 & \textbf{-0.20} \\
& (0.11) & (0.11) & (0.10) & (0.10) & (0.09) & (0.11) & (0.10) \\
& \textit{ 159 } & \textit{ 159 } & \textit{ 159 } & \textit{ 159 } & \textit{ 159 } & \textit{ 197 } & \textit{ 239 } \\
Own House & 0.04 & -0.00 & -0.01 & -0.03 & -0.07 & -0.06 & \textbf{-0.11} \\
& (0.07) & (0.07) & (0.07) & (0.07) & (0.07) & (0.07) & (0.06) \\
& \textit{ 159 } & \textit{ 159 } & \textit{ 159 } & \textit{ 159 } & \textit{ 159 } & \textit{ 197 } & \textit{ 239 } \\
Live With Parents & -0.01 & 0.00 & -0.00 & -0.00 & -0.00 & -0.05 & \textbf{-0.15} \\
& (0.03) & (0.03) & (0.03) & (0.03) & (0.02) & (0.03) & (0.04) \\
& \textit{ 159 } & \textit{ 159 } & \textit{ 159 } & \textit{ 159 } & \textit{ 159 } & \textit{ 197 } & \textit{ 239 } \\
\bottomrule
\end{tabular}
}
\vspace{1ex} \\
\footnotesize\raggedright{Note: This table shows the estimates of the coefficient for attending Reggio Approach preschools from multiple methods. We compare Reggio Approach people with people who attended religious preschools in Reggio. Column title indicates the corresponding control set and and model. ``None'' refers to the OLS estimate with no control variables. ``BIC'' refers to the OLS estimate with controls selected by Bayesian Information Criterion (BIC) and additional controls for caregiver's religion. ``Full'' refers to the OLS estimate with the full set of controls. ``PSM" refers to propensity score matching estimation. ``AIPW" refers to augmented inverse propensity weighting estimation. ``DidPmIt'' refers to the difference-in-difference estimate of (Reggio Muni - Parma Muni) - (Reggio Other - Parma Other). ``DidPv'' refers to the difference-in-difference estimate of (Reggio Muni - Padova Muni) - (Reggio Other - Padova Other). Bold number shows that the estimate is statistically significant at the 10\% level. Number of observations used in estimation is reported in italic.}
\end{table}

\begin{table}[H] \caption{Estimation Results for Living Environment Outcomes, Comparison to No Preschool, Adult 40s Cohort} \label{ols-W-adult40-reg-none}
\scalebox{0.8}{\begin{tabular}{l c c c c c c c c c}
\toprule
 & None & BIC & Full & PSM & AIPW & DidPm & PSMPm & DidPv & PSMPv \\
\midrule
Married or Cohabitating & 0.02 & -0.01 & 0.05 & -0.00 & -0.01 & -0.06 & \textbf{0.19} & -0.14 & \textbf{0.23} \\
& (0.07) & (0.07) & (0.08) & (0.08) & (0.07) & (0.10) & (0.07) & (0.10) & (0.09) \\
& \textit{ 170 } & \textit{ 170 } & \textit{ 170 } & \textit{ 170 } & \textit{ 170 } & \textit{ 357 } & \textit{ 205 } & \textit{ 375 } & \textit{ 165 } \\
Divorced & -0.06 & -0.04 & -0.03 & -0.03 & -0.04 & 0.08 & \textbf{-0.13} & \textbf{ 0.14 } & \textbf{-0.23} \\
& (0.05) & (0.04) & (0.06) & (0.05) & (0.05) & (0.07) & (0.05) & (0.07) & (0.07) \\
& \textit{ 170 } & \textit{ 170 } & \textit{ 170 } & \textit{ 170 } & \textit{ 170 } & \textit{ 357 } & \textit{ 205 } & \textit{ 374 } & \textit{ 164 } \\
Num. of Children in House & 0.01 & -0.04 & -0.04 & -0.03 & -0.04 & -0.19 & 0.08 & \textbf{ -0.51 } & 0.10 \\
& (0.09) & (0.09) & (0.09) & (0.09) & (0.09) & (0.15) & (0.09) & (0.14) & (0.11) \\
& \textit{ 170 } & \textit{ 170 } & \textit{ 170 } & \textit{ 170 } & \textit{ 170 } & \textit{ 357 } & \textit{ 205 } & \textit{ 375 } & \textit{ 165 } \\
Own House & -0.03 & -0.00 & 0.01 & 0.00 & -0.00 & -0.03 & -0.04 & -0.11 & -0.02 \\
& (0.07) & (0.07) & (0.07) & (0.07) & (0.07) & (0.10) & (0.07) & (0.09) & (0.08) \\
& \textit{ 170 } & \textit{ 170 } & \textit{ 170 } & \textit{ 170 } & \textit{ 170 } & \textit{ 357 } & \textit{ 205 } & \textit{ 375 } & \textit{ 165 } \\
Live With Parents & -0.03 & -0.03 & \textbf{ -0.05 } & -0.04 & -0.03 & -0.06 & -0.03 & \textbf{ -0.11 } & -0.05 \\
& (0.03) & (0.03) & (0.03) & (0.03) & (0.03) & (0.05) & (0.02) & (0.05) & (0.04) \\
& \textit{ 170 } & \textit{ 170 } & \textit{ 170 } & \textit{ 170 } & \textit{ 170 } & \textit{ 357 } & \textit{ 205 } & \textit{ 375 } & \textit{ 165 } \\
\bottomrule
\end{tabular}
}
\vspace{1ex} \\
\footnotesize\raggedright{Note: This table shows the estimates of the coefficient for attending Reggio Approach preschools from multiple methods. We compare Reggio Approach people with people who attended religious preschools in Reggio. Column title indicates the corresponding control set and and model. ``None'' refers to the OLS estimate with no control variables. ``BIC'' refers to the OLS estimate with controls selected by Bayesian Information Criterion (BIC) and additional controls for caregiver's religion. ``Full'' refers to the OLS estimate with the full set of controls. ``PSM" refers to propensity score matching estimation. ``AIPW" refers to augmented inverse propensity weighting estimation. ``DidPmIt'' refers to the difference-in-difference estimate of (Reggio Muni - Parma Muni) - (Reggio None - Parma None). ``DidPv'' refers to the difference-in-difference estimate of (Reggio Muni - Padova Muni) - (Reggio None - Padova None). Bold number shows that the estimate is statistically significant at the 10\% level. Number of observations used in estimation is reported in italic.}
\end{table}


\begin{table}[H] \caption{Estimation Results for Health and Risk Outcomes, Comparison to Non-RA Preschools, Adult 40s Cohort} \label{ols-H-adult40-reg-other}
\scalebox{0.8}{\begin{tabular}{l c c c c c c c}
\toprule
 & None & BIC & Full & PSM & AIPW & PSMPm & PSMPv \\
\midrule
Tried Marijuana & \textbf{ 0.09 } & \textbf{ 0.11 } & \textbf{ 0.09 } & \textbf{0.11} & \textbf{0.08} & 0.06 & 0.06 \\
& (0.05) & (0.05) & (0.05) & (0.05) & (0.05) & (0.05) & (0.04) \\
& \textit{ 159 } & \textit{ 159 } & \textit{ 159 } & \textit{ 159 } & \textit{ 159 } & \textit{ 197 } & \textit{ 239 } \\
Num. of Cigarettes Per Day & 2.59 & 2.98 & 1.15 & \textbf{3.21} & \textbf{2.92} & \textbf{3.58} & \textbf{5.34} \\
& (1.82) & (2.13) & (2.32) & (1.74) & (1.84) & (1.79) & (1.87) \\
& \textit{ 50 } & \textit{ 50 } & \textit{ 50 } & \textit{ 50 } & \textit{ 50 } & \textit{ 66 } & \textit{ 56 } \\
BMI & -0.01 & -0.04 & -0.05 & -0.19 & 0.30 & -0.14 & 0.35 \\
& (0.52) & (0.56) & (0.55) & (0.57) & (0.59) & (0.60) & (0.57) \\
& \textit{ 119 } & \textit{ 119 } & \textit{ 119 } & \textit{ 119 } & \textit{ 119 } & \textit{ 148 } & \textit{ 184 } \\
Not Obese & -0.04 & 0.02 & 0.04 & 0.03 & 0.06 & -0.07 & -0.08 \\
& (0.07) & (0.07) & (0.07) & (0.08) & (0.07) & (0.07) & (0.07) \\
& \textit{ 159 } & \textit{ 159 } & \textit{ 159 } & \textit{ 159 } & \textit{ 159 } & \textit{ 197 } & \textit{ 239 } \\
Not Overweight & 0.05 & 0.03 & 0.03 & -0.01 & -0.04 & 0.01 & -0.03 \\
& (0.07) & (0.07) & (0.07) & (0.07) & (0.09) & (0.07) & (0.06) \\
& \textit{ 159 } & \textit{ 159 } & \textit{ 159 } & \textit{ 159 } & \textit{ 159 } & \textit{ 197 } & \textit{ 239 } \\
Good Health & \textbf{ -0.18 } & \textbf{ -0.17 } & \textbf{ -0.18 } & -0.15 & -0.11 & \textbf{0.33} & \textbf{0.22} \\
& (0.09) & (0.10) & (0.10) & (0.10) & (0.08) & (0.11) & (0.10) \\
& \textit{ 159 } & \textit{ 159 } & \textit{ 159 } & \textit{ 159 } & \textit{ 159 } & \textit{ 197 } & \textit{ 239 } \\
No Problematic Health Condition & 0.00 & -0.03 & -0.03 & 0.02 & 0.01 & \textbf{0.18} & 0.03 \\
& (0.08) & (0.09) & (0.09) & (0.10) & (0.10) & (0.07) & (0.08) \\
& \textit{ 139 } & \textit{ 139 } & \textit{ 139 } & \textit{ 139 } & \textit{ 139 } & \textit{ 176 } & \textit{ 223 } \\
Num. of Days Sick Past Month & 0.06 & \textbf{ 0.09 } & \textbf{ 0.10 } & \textbf{0.08} & \textbf{0.09} & -0.01 & -0.03 \\
& (0.05) & (0.05) & (0.05) & (0.05) & (0.04) & (0.05) & (0.04) \\
& \textit{ 149 } & \textit{ 149 } & \textit{ 149 } & \textit{ 149 } & \textit{ 149 } & \textit{ 190 } & \textit{ 231 } \\
Ever Suspended from School & -0.00 & -0.02 & -0.01 & -0.05 & -0.03 & 0.01 & 0.02 \\
& (0.04) & (0.04) & (0.04) & (0.05) & (0.04) & (0.04) & (0.04) \\
& \textit{ 159 } & \textit{ 159 } & \textit{ 159 } & \textit{ 159 } & \textit{ 159 } & \textit{ 197 } & \textit{ 239 } \\
Age At First Drink & -0.36 & 0.11 & -0.07 & -0.24 & -0.19 & \textbf{-2.83} & \textbf{-2.29} \\
& (1.38) & (1.32) & (1.27) & (1.40) & (1.47) & (1.24) & (1.26) \\
& \textit{ 154 } & \textit{ 154 } & \textit{ 154 } & \textit{ 154 } & \textit{ 154 } & \textit{ 192 } & \textit{ 235 } \\
\bottomrule
\end{tabular}
}
\vspace{1ex} \\
\footnotesize\raggedright{Note: This table shows the estimates of the coefficient for attending Reggio Approach preschools from multiple methods. We compare Reggio Approach people with people who attended religious preschools in Reggio. Column title indicates the corresponding control set and and model. ``None'' refers to the OLS estimate with no control variables. ``BIC'' refers to the OLS estimate with controls selected by Bayesian Information Criterion (BIC) and additional controls for caregiver's religion. ``Full'' refers to the OLS estimate with the full set of controls. ``PSM" refers to propensity score matching estimation. ``AIPW" refers to augmented inverse propensity weighting estimation. ``DidPmIt'' refers to the difference-in-difference estimate of (Reggio Muni - Parma Muni) - (Reggio Other - Parma Other). ``DidPv'' refers to the difference-in-difference estimate of (Reggio Muni - Padova Muni) - (Reggio Other - Padova Other). Bold number shows that the estimate is statistically significant at the 10\% level. Number of observations used in estimation is reported in italic.}
\end{table}

\begin{table}[H] \caption{Estimation Results for Health and Risk Outcomes, Comparison to No Preschool, Adult 40s Cohort} \label{ols-H-adult40-reg-none}
\scalebox{0.8}{\begin{tabular}{l c c c c c c c}
\toprule
 & None & BIC & Full & PSM & AIPW & DidPm & DidPv \\
\midrule
Tried Marijuana & \textbf{      0.08 } & \textbf{      0.08 } & \textbf{      0.10 } &      0.06 & \textbf{     0.09} &      0.01 &      0.03 \\
& (     0.05 ) & (     0.05 ) & (     0.05 ) & (     0.06 ) & (     0.05 ) & (     0.06 ) & (     0.06 ) \\
& \textit{ 170 } & \textit{ 170 } & \textit{ 170 } & \textit{ 170 } & \textit{ 170 } & \textit{ 382 } & \textit{ 375 } \\
Num. of Cigarettes Per Day &     -0.18 &     -0.54 &     -0.73 &     -0.43 &     -0.54 &     -0.03 &      1.69 \\
& (     1.71 ) & (     1.81 ) & (     1.82 ) & (     1.84 ) & (     2.07 ) & (     2.21 ) & (     2.37 ) \\
& \textit{ 59 } & \textit{ 59 } & \textit{ 59 } & \textit{ 59 } & \textit{ 59 } & \textit{ 133 } & \textit{ 108 } \\
BMI &     -0.22 &     -0.50 &     -0.30 &     -0.64 &     -0.50 & \textbf{     -1.11 } &     -0.03 \\
& (     0.55 ) & (     0.54 ) & (     0.51 ) & (     0.49 ) & (     0.59 ) & (     0.69 ) & (     0.66 ) \\
& \textit{ 114 } & \textit{ 114 } & \textit{ 114 } & \textit{ 114 } & \textit{ 114 } & \textit{ 289 } & \textit{ 274 } \\
Obese & \textbf{     -0.14 } &     -0.08 &     -0.01 &     -0.09 &     -0.07 & \textbf{     -0.26 } &     -0.03 \\
& (     0.07 ) & (     0.08 ) & (     0.08 ) & (     0.08 ) & (     0.09 ) & (     0.09 ) & (     0.10 ) \\
& \textit{ 170 } & \textit{ 170 } & \textit{ 170 } & \textit{ 170 } & \textit{ 170 } & \textit{ 382 } & \textit{ 375 } \\
Overweight &      0.03 &     -0.04 &     -0.07 &     -0.05 &     -0.06 &     -0.01 &      0.04 \\
& (     0.07 ) & (     0.07 ) & (     0.07 ) & (     0.07 ) & (     0.07 ) & (     0.09 ) & (     0.08 ) \\
& \textit{ 170 } & \textit{ 170 } & \textit{ 170 } & \textit{ 170 } & \textit{ 170 } & \textit{ 382 } & \textit{ 375 } \\
Good Health &      0.10 & \textbf{      0.15 } &      0.12 & \textbf{     0.18} & \textbf{     0.17} & \textbf{      0.32 } &     -0.05 \\
& (     0.09 ) & (     0.10 ) & (     0.11 ) & (     0.09 ) & (     0.10 ) & (     0.13 ) & (     0.13 ) \\
& \textit{ 169 } & \textit{ 169 } & \textit{ 169 } & \textit{ 169 } & \textit{ 169 } & \textit{ 381 } & \textit{ 374 } \\
No Problematic Health Condition &      0.01 &      0.03 &      0.07 &      0.07 &      0.01 &      0.10 &     -0.06 \\
& (     0.08 ) & (     0.09 ) & (     0.10 ) & (     0.10 ) & (     0.08 ) & (     0.11 ) & (     0.11 ) \\
& \textit{ 140 } & \textit{ 140 } & \textit{ 140 } & \textit{ 140 } & \textit{ 140 } & \textit{ 341 } & \textit{ 342 } \\
Num. of Days Sick Past Month &      0.02 &      0.00 &     -0.04 &      0.03 &      0.01 &     -0.05 &     -0.00 \\
& (     0.06 ) & (     0.06 ) & (     0.06 ) & (     0.06 ) & (     0.06 ) & (     0.10 ) & (     0.09 ) \\
& \textit{ 158 } & \textit{ 158 } & \textit{ 158 } & \textit{ 158 } & \textit{ 158 } & \textit{ 367 } & \textit{ 360 } \\
Ever Suspended from School &     -0.03 &     -0.04 &     -0.02 & \textbf{    -0.08} &     -0.05 &     -0.02 &     -0.05 \\
& (     0.04 ) & (     0.04 ) & (     0.06 ) & (     0.05 ) & (     0.04 ) & (     0.05 ) & (     0.05 ) \\
& \textit{ 170 } & \textit{ 170 } & \textit{ 170 } & \textit{ 170 } & \textit{ 170 } & \textit{ 382 } & \textit{ 375 } \\
Age At First Drink &      1.00 &     -0.18 &     -0.73 &      0.10 &     -0.25 &      0.55 &     -0.52 \\
& (     1.37 ) & (     1.39 ) & (     1.45 ) & (     1.42 ) & (     1.42 ) & (     1.63 ) & (     1.78 ) \\
& \textit{ 162 } & \textit{ 162 } & \textit{ 162 } & \textit{ 162 } & \textit{ 162 } & \textit{ 365 } & \textit{ 365 } \\
\bottomrule
\end{tabular}
}
\vspace{1ex} \\
\footnotesize\raggedright{Note: This table shows the estimates of the coefficient for attending Reggio Approach preschools from multiple methods. We compare Reggio Approach people with people who attended religious preschools in Reggio. Column title indicates the corresponding control set and and model. ``None'' refers to the OLS estimate with no control variables. ``BIC'' refers to the OLS estimate with controls selected by Bayesian Information Criterion (BIC) and additional controls for caregiver's religion. ``Full'' refers to the OLS estimate with the full set of controls. ``PSM" refers to propensity score matching estimation. ``AIPW" refers to augmented inverse propensity weighting estimation. ``DidPmIt'' refers to the difference-in-difference estimate of (Reggio Muni - Parma Muni) - (Reggio None - Parma None). ``DidPv'' refers to the difference-in-difference estimate of (Reggio Muni - Padova Muni) - (Reggio None - Padova None). Bold number shows that the estimate is statistically significant at the 10\% level. Number of observations used in estimation is reported in italic.}
\end{table}


\begin{table}[H] \caption{Estimation Results for Noncognitive Outcomes, Comparison to Non-RA Preschools, Adult 40s Cohort} \label{ols-N-adult40-reg-other}
\scalebox{0.8}{\begin{tabular}{l c c c c c}
\toprule
 & None & BIC & Full & PSM & AIPW \\
\midrule
Locus of Control - positive &      0.13 &      0.14 &      0.11 &      0.11 &      0.07 \\
& (     0.14 ) & (     0.14 ) & (     0.14 ) & (     0.17 ) & (     0.15 ) \\
& \textit{ 156 } & \textit{ 156 } & \textit{ 156 } & \textit{ 156 } & \textit{ 156 } \\
Depression Score - positive &      0.56 & \textbf{      1.43 } &      1.09 & \textbf{     1.44} & \textbf{     1.10} \\
& (     0.92 ) & (     0.85 ) & (     0.89 ) & (     0.85 ) & (     0.77 ) \\
& \textit{ 156 } & \textit{ 156 } & \textit{ 156 } & \textit{ 156 } & \textit{ 156 } \\
Stress &      0.05 &      0.09 &      0.09 &      0.17 &      0.12 \\
& (     0.11 ) & (     0.12 ) & (     0.12 ) & (     0.14 ) & (     0.13 ) \\
& \textit{ 159 } & \textit{ 159 } & \textit{ 159 } & \textit{ 159 } & \textit{ 159 } \\
Work is Source of Stress & \textbf{      0.30 } & \textbf{      0.23 } & \textbf{      0.19 } & \textbf{     0.34} & \textbf{     0.27} \\
& (     0.10 ) & (     0.11 ) & (     0.12 ) & (     0.09 ) & (     0.12 ) \\
& \textit{ 97 } & \textit{ 97 } & \textit{ 97 } & \textit{ 97 } & \textit{ 97 } \\
Satisfied with Income &      0.15 &      0.14 &      0.15 &      0.19 &      0.07 \\
& (     0.14 ) & (     0.13 ) & (     0.14 ) & (     0.14 ) & (     0.17 ) \\
& \textit{ 159 } & \textit{ 159 } & \textit{ 159 } & \textit{ 159 } & \textit{ 159 } \\
Satisfied with Work &      0.12 &      0.18 &      0.16 & \textbf{     0.32} &      0.15 \\
& (     0.12 ) & (     0.13 ) & (     0.13 ) & (     0.17 ) & (     0.15 ) \\
& \textit{ 159 } & \textit{ 159 } & \textit{ 159 } & \textit{ 159 } & \textit{ 159 } \\
Satisfied with Health & \textbf{     -0.16 } &     -0.11 &     -0.12 &     -0.10 &     -0.12 \\
& (     0.09 ) & (     0.09 ) & (     0.10 ) & (     0.10 ) & (     0.10 ) \\
& \textit{ 159 } & \textit{ 159 } & \textit{ 159 } & \textit{ 159 } & \textit{ 159 } \\
Satisfied with Family &      0.02 &      0.06 &      0.08 &      0.02 &      0.01 \\
& (     0.13 ) & (     0.13 ) & (     0.14 ) & (     0.13 ) & (     0.12 ) \\
& \textit{ 159 } & \textit{ 159 } & \textit{ 159 } & \textit{ 159 } & \textit{ 159 } \\
Optimistic Look in Life &     -0.03 &     -0.04 &     -0.06 &     -0.02 &     -0.04 \\
& (     0.08 ) & (     0.08 ) & (     0.08 ) & (     0.09 ) & (     0.07 ) \\
& \textit{ 153 } & \textit{ 153 } & \textit{ 153 } & \textit{ 153 } & \textit{ 153 } \\
Positive Reciprocity & \textbf{     -0.23 } & \textbf{     -0.15 } & \textbf{     -0.18 } &     -0.17 &     -0.16 \\
& (     0.10 ) & (     0.10 ) & (     0.10 ) & (     0.11 ) & (     0.08 ) \\
& \textit{ 159 } & \textit{ 159 } & \textit{ 159 } & \textit{ 159 } & \textit{ 159 } \\
Negative Reciprocity &      0.09 &      0.03 &      0.10 &      0.20 &      0.01 \\
& (     0.14 ) & (     0.15 ) & (     0.15 ) & (     0.16 ) & (     0.15 ) \\
& \textit{ 159 } & \textit{ 159 } & \textit{ 159 } & \textit{ 159 } & \textit{ 159 } \\
\bottomrule
\end{tabular}
}
\vspace{1ex} \\
\footnotesize\raggedright{Note: This table shows the estimates of the coefficient for attending Reggio Approach preschools from multiple methods. We compare Reggio Approach people with people who attended religious preschools in Reggio. Column title indicates the corresponding control set and and model. ``None'' refers to the OLS estimate with no control variables. ``BIC'' refers to the OLS estimate with controls selected by Bayesian Information Criterion (BIC) and additional controls for caregiver's religion. ``Full'' refers to the OLS estimate with the full set of controls. ``PSM" refers to propensity score matching estimation. ``AIPW" refers to augmented inverse propensity weighting estimation. ``DidPmIt'' refers to the difference-in-difference estimate of (Reggio Muni - Parma Muni) - (Reggio Other - Parma Other). ``DidPv'' refers to the difference-in-difference estimate of (Reggio Muni - Padova Muni) - (Reggio Other - Padova Other). Bold number shows that the estimate is statistically significant at the 10\% level. Number of observations used in estimation is reported in italic.}
\end{table}

\begin{table}[H] \caption{Estimation Results for Noncognitive Outcomes, Comparison to No Preschool, Adult 40s Cohort} \label{ols-N-adult40-reg-none}
\scalebox{0.8}{\begin{tabular}{l c c c c c c c}
\toprule
 & None & BIC & Full & PSM & AIPW & DidPm & DidPv \\
\midrule
Locus of Control - positive &      0.14 & \textbf{      0.20 } & \textbf{      0.28 } &      0.22 & \textbf{     0.27} &      0.12 & \textbf{      0.31 } \\
& (     0.13 ) & (     0.13 ) & (     0.14 ) & (     0.15 ) & (     0.15 ) & (     0.17 ) & (     0.17 ) \\
& \textit{ 165 } & \textit{ 165 } & \textit{ 165 } & \textit{ 165 } & \textit{ 165 } & \textit{ 364 } & \textit{ 357 } \\
Depression Score - positive & \textbf{      2.25 } & \textbf{      2.23 } & \textbf{      2.10 } & \textbf{     3.23} & \textbf{     2.47} &      0.20 & \textbf{      2.26 } \\
& (     0.92 ) & (     0.95 ) & (     1.07 ) & (     1.04 ) & (     1.07 ) & (     1.13 ) & (     1.18 ) \\
& \textit{ 168 } & \textit{ 168 } & \textit{ 168 } & \textit{ 168 } & \textit{ 168 } & \textit{ 380 } & \textit{ 371 } \\
Stress & \textbf{      0.19 } & \textbf{      0.20 } &      0.06 & \textbf{     0.31} & \textbf{     0.26} &      0.13 & \textbf{      0.46 } \\
& (     0.11 ) & (     0.12 ) & (     0.14 ) & (     0.13 ) & (     0.12 ) & (     0.16 ) & (     0.14 ) \\
& \textit{ 170 } & \textit{ 170 } & \textit{ 170 } & \textit{ 170 } & \textit{ 170 } & \textit{ 382 } & \textit{ 375 } \\
Work is Source of Stress & \textbf{      0.17 } & \textbf{      0.20 } & \textbf{      0.23 } & \textbf{     0.19} & \textbf{     0.19} & \textbf{      0.38 } &      0.11 \\
& (     0.09 ) & (     0.09 ) & (     0.10 ) & (     0.10 ) & (     0.09 ) & (     0.13 ) & (     0.14 ) \\
& \textit{ 108 } & \textit{ 108 } & \textit{ 108 } & \textit{ 108 } & \textit{ 108 } & \textit{ 239 } & \textit{ 218 } \\
Satisfied with Income & \textbf{      0.27 } & \textbf{      0.28 } &      0.15 & \textbf{     0.32} & \textbf{     0.32} &     -0.03 &      0.22 \\
& (     0.13 ) & (     0.14 ) & (     0.16 ) & (     0.16 ) & (     0.13 ) & (     0.17 ) & (     0.17 ) \\
& \textit{ 170 } & \textit{ 170 } & \textit{ 170 } & \textit{ 170 } & \textit{ 170 } & \textit{ 382 } & \textit{ 375 } \\
Satisfied with Work & \textbf{      0.31 } & \textbf{      0.29 } &      0.16 & \textbf{     0.27} & \textbf{     0.31} &      0.08 & \textbf{      0.45 } \\
& (     0.12 ) & (     0.12 ) & (     0.13 ) & (     0.13 ) & (     0.15 ) & (     0.16 ) & (     0.18 ) \\
& \textit{ 170 } & \textit{ 170 } & \textit{ 170 } & \textit{ 170 } & \textit{ 170 } & \textit{ 381 } & \textit{ 373 } \\
Satisfied with Health &      0.03 &      0.01 &     -0.03 &      0.01 &      0.02 &      0.16 &      0.03 \\
& (     0.08 ) & (     0.09 ) & (     0.09 ) & (     0.08 ) & (     0.09 ) & (     0.13 ) & (     0.11 ) \\
& \textit{ 170 } & \textit{ 170 } & \textit{ 170 } & \textit{ 170 } & \textit{ 170 } & \textit{ 382 } & \textit{ 375 } \\
Satisfied with Family & \textbf{      0.19 } &      0.18 &      0.18 &      0.25 & \textbf{     0.26} &     -0.12 &     -0.12 \\
& (     0.12 ) & (     0.13 ) & (     0.15 ) & (     0.15 ) & (     0.17 ) & (     0.15 ) & (     0.16 ) \\
& \textit{ 170 } & \textit{ 170 } & \textit{ 170 } & \textit{ 170 } & \textit{ 170 } & \textit{ 379 } & \textit{ 374 } \\
Optimistic Look in Life &     -0.10 &     -0.06 &      0.06 &     -0.07 &     -0.04 &     -0.10 & \textbf{     -0.19 } \\
& (     0.08 ) & (     0.08 ) & (     0.08 ) & (     0.09 ) & (     0.08 ) & (     0.10 ) & (     0.12 ) \\
& \textit{ 163 } & \textit{ 163 } & \textit{ 163 } & \textit{ 163 } & \textit{ 163 } & \textit{ 348 } & \textit{ 309 } \\
Positive Reciprocity &     -0.02 &     -0.01 &     -0.05 &     -0.08 &     -0.01 &      0.11 &      0.14 \\
& (     0.11 ) & (     0.11 ) & (     0.12 ) & (     0.11 ) & (     0.11 ) & (     0.15 ) & (     0.15 ) \\
& \textit{ 170 } & \textit{ 170 } & \textit{ 170 } & \textit{ 170 } & \textit{ 170 } & \textit{ 382 } & \textit{ 372 } \\
Negative Reciprocity &      0.02 &     -0.04 &     -0.05 &     -0.02 &      0.00 &      0.04 &     -0.18 \\
& (     0.15 ) & (     0.15 ) & (     0.17 ) & (     0.16 ) & (     0.15 ) & (     0.20 ) & (     0.20 ) \\
& \textit{ 170 } & \textit{ 170 } & \textit{ 170 } & \textit{ 170 } & \textit{ 170 } & \textit{ 382 } & \textit{ 372 } \\
\bottomrule
\end{tabular}
}
\vspace{1ex} \\
\footnotesize\raggedright{Note: This table shows the estimates of the coefficient for attending Reggio Approach preschools from multiple methods. We compare Reggio Approach people with people who attended religious preschools in Reggio. Column title indicates the corresponding control set and and model. ``None'' refers to the OLS estimate with no control variables. ``BIC'' refers to the OLS estimate with controls selected by Bayesian Information Criterion (BIC) and additional controls for caregiver's religion. ``Full'' refers to the OLS estimate with the full set of controls. ``PSM" refers to propensity score matching estimation. ``AIPW" refers to augmented inverse propensity weighting estimation. ``DidPmIt'' refers to the difference-in-difference estimate of (Reggio Muni - Parma Muni) - (Reggio None - Parma None). ``DidPv'' refers to the difference-in-difference estimate of (Reggio Muni - Padova Muni) - (Reggio None - Padova None). Bold number shows that the estimate is statistically significant at the 10\% level. Number of observations used in estimation is reported in italic.}
\end{table}


\begin{table}[H] \caption{Estimation Results for Social Outcomes, Comparison to Non-RA Preschools, Adult 40s Cohort} \label{ols-S-adult40-reg-other}
\scalebox{0.8}{\begin{tabular}{l c c c c c}
\toprule
 & None & BIC & Full & PSM & AIPW \\
\midrule
Bothered by Migrants &      0.03 &      0.05 &      0.04 &     -0.04 &      0.09 \\
& (     0.08 ) & (     0.09 ) & (     0.09 ) & (     0.13 ) & (     0.09 ) \\
& \textit{ 157 } & \textit{ 157 } & \textit{ 157 } & \textit{ 157 } & \textit{ 157 } \\
Num. of Friends & \textbf{      1.39 } &      0.83 &      1.09 &      0.53 &      0.66 \\
& (     0.95 ) & (     0.93 ) & (     1.02 ) & (     0.83 ) & (     0.91 ) \\
& \textit{ 136 } & \textit{ 136 } & \textit{ 136 } & \textit{ 136 } & \textit{ 136 } \\
Has Migrant Friends &      0.00 &     -0.04 &     -0.04 &     -0.09 &      0.00 \\
& (     0.08 ) & (     0.08 ) & (     0.08 ) & (     0.08 ) & (     0.08 ) \\
& \textit{ 159 } & \textit{ 159 } & \textit{ 159 } & \textit{ 159 } & \textit{ 159 } \\
Volunteers &      0.05 &      0.01 &      0.03 &      0.00 &      0.01 \\
& (     0.04 ) & (     0.04 ) & (     0.04 ) & (     0.06 ) & (     0.05 ) \\
& \textit{ 159 } & \textit{ 159 } & \textit{ 159 } & \textit{ 159 } & \textit{ 159 } \\
Ever Voted for Municipal &     -0.07 &      0.07 &      0.06 &      0.09 & \textbf{     0.10} \\
& (     0.08 ) & (     0.07 ) & (     0.07 ) & (     0.07 ) & (     0.07 ) \\
& \textit{ 151 } & \textit{ 151 } & \textit{ 151 } & \textit{ 151 } & \textit{ 151 } \\
Ever Voted for Regional &     -0.05 &      0.08 &      0.07 &      0.10 & \textbf{     0.10} \\
& (     0.08 ) & (     0.07 ) & (     0.07 ) & (     0.08 ) & (     0.06 ) \\
& \textit{ 151 } & \textit{ 151 } & \textit{ 151 } & \textit{ 151 } & \textit{ 151 } \\
\bottomrule
\end{tabular}
}
\vspace{1ex} \\
\footnotesize\raggedright{Note: This table shows the estimates of the coefficient for attending Reggio Approach preschools from multiple methods. We compare Reggio Approach people with people who attended religious preschools in Reggio. Column title indicates the corresponding control set and and model. ``None'' refers to the OLS estimate with no control variables. ``BIC'' refers to the OLS estimate with controls selected by Bayesian Information Criterion (BIC) and additional controls for caregiver's religion. ``Full'' refers to the OLS estimate with the full set of controls. ``PSM" refers to propensity score matching estimation. ``AIPW" refers to augmented inverse propensity weighting estimation. ``DidPmIt'' refers to the difference-in-difference estimate of (Reggio Muni - Parma Muni) - (Reggio Other - Parma Other). ``DidPv'' refers to the difference-in-difference estimate of (Reggio Muni - Padova Muni) - (Reggio Other - Padova Other). Bold number shows that the estimate is statistically significant at the 10\% level. Number of observations used in estimation is reported in italic.}
\end{table}

\begin{table}[H] \caption{Estimation Results for Social Outcomes, Comparison to No Preschool, Adult 40s Cohort} \label{ols-S-adult40-reg-none}
\scalebox{0.8}{\begin{tabular}{l c c c c c c c}
\toprule
 & None & BIC & Full & PSM & AIPW & DidPm & DidPv \\
\midrule
Bothered by Migrants &     -0.07 &     -0.02 &     -0.00 &      0.00 &     -0.01 &      0.12 &      0.18 \\
& (     0.09 ) & (     0.09 ) & (     0.11 ) & (     0.10 ) & (     0.10 ) & (     0.14 ) & (     0.14 ) \\
& \textit{ 170 } & \textit{ 170 } & \textit{ 170 } & \textit{ 170 } & \textit{ 170 } & \textit{ 382 } & \textit{ 371 } \\
Num. of Friends &     -0.68 &     -0.46 &      0.75 &     -0.33 &     -0.34 & \textbf{      2.74 } & \textbf{      2.71 } \\
& (     1.06 ) & (     0.94 ) & (     1.47 ) & (     1.06 ) & (     1.02 ) & (     1.36 ) & (     1.49 ) \\
& \textit{ 124 } & \textit{ 124 } & \textit{ 124 } & \textit{ 124 } & \textit{ 124 } & \textit{ 335 } & \textit{ 320 } \\
Has Migrant Friends & \textbf{     -0.13 } & \textbf{     -0.12 } & \textbf{     -0.12 } &     -0.08 &     -0.11 &      0.10 & \textbf{     -0.20 } \\
& (     0.07 ) & (     0.07 ) & (     0.08 ) & (     0.08 ) & (     0.07 ) & (     0.09 ) & (     0.10 ) \\
& \textit{ 170 } & \textit{ 170 } & \textit{ 170 } & \textit{ 170 } & \textit{ 170 } & \textit{ 382 } & \textit{ 375 } \\
Volunteers & \textbf{     -0.11 } & \textbf{     -0.08 } & \textbf{     -0.11 } &     -0.05 &     -0.07 &     -0.06 & \textbf{     -0.16 } \\
& (     0.06 ) & (     0.06 ) & (     0.07 ) & (     0.06 ) & (     0.04 ) & (     0.08 ) & (     0.07 ) \\
& \textit{ 170 } & \textit{ 170 } & \textit{ 170 } & \textit{ 170 } & \textit{ 170 } & \textit{ 382 } & \textit{ 375 } \\
Ever Voted for Municipal & \textbf{      0.19 } & \textbf{      0.12 } &      0.11 &      0.12 & \textbf{     0.14} &     -0.02 &     -0.09 \\
& (     0.08 ) & (     0.08 ) & (     0.08 ) & (     0.11 ) & (     0.08 ) & (     0.10 ) & (     0.10 ) \\
& \textit{ 153 } & \textit{ 153 } & \textit{ 153 } & \textit{ 153 } & \textit{ 153 } & \textit{ 365 } & \textit{ 340 } \\
Ever Voted for Regional & \textbf{      0.20 } & \textbf{      0.14 } & \textbf{      0.13 } &      0.14 & \textbf{     0.16} &      0.05 &     -0.10 \\
& (     0.08 ) & (     0.08 ) & (     0.08 ) & (     0.11 ) & (     0.08 ) & (     0.09 ) & (     0.10 ) \\
& \textit{ 153 } & \textit{ 153 } & \textit{ 153 } & \textit{ 153 } & \textit{ 153 } & \textit{ 365 } & \textit{ 340 } \\
\bottomrule
\end{tabular}
}
\vspace{1ex} \\
\footnotesize\raggedright{Note: This table shows the estimates of the coefficient for attending Reggio Approach preschools from multiple methods. We compare Reggio Approach people with people who attended religious preschools in Reggio. Column title indicates the corresponding control set and and model. ``None'' refers to the OLS estimate with no control variables. ``BIC'' refers to the OLS estimate with controls selected by Bayesian Information Criterion (BIC) and additional controls for caregiver's religion. ``Full'' refers to the OLS estimate with the full set of controls. ``PSM" refers to propensity score matching estimation. ``AIPW" refers to augmented inverse propensity weighting estimation. ``DidPmIt'' refers to the difference-in-difference estimate of (Reggio Muni - Parma Muni) - (Reggio None - Parma None). ``DidPv'' refers to the difference-in-difference estimate of (Reggio Muni - Padova Muni) - (Reggio None - Padova None). Bold number shows that the estimate is statistically significant at the 10\% level. Number of observations used in estimation is reported in italic.}
\end{table}



















% ==============================================================================%
\subsection{Multinomial Logit}\label{appendix:mlogit}
% ==============================================================================% 

\begin{table}[H] 
\centering
\caption{Multinomial Logit, Child and Adolescent Cohorts, Reggio Emilia} \label{mlogit-chi-ado-RE}
\begin{adjustbox}{width=\textwidth}
\begin{threeparttable}
{
\def\sym#1{\ifmmode^{#1}\else\(^{#1}\)\fi}
\begin{tabular}{l*{6}{c}}
\toprule
			& \multicolumn{3}{c}{Children} & \multicolumn{3}{c}{Adolescents} \\
                    &\multicolumn{1}{c}{None}&\multicolumn{1}{c}{Other}&\multicolumn{1}{c}{Municipal}&\multicolumn{1}{c}{None}&\multicolumn{1}{c}{Other}&\multicolumn{1}{c}{Municipal}\\
\midrule
Male                &       -0.01         &        0.00         &        0.00         &        0.02         &        0.01         &       -0.03         \\
                    &      (0.01)         &      (0.05)         &      (0.05)         &      (0.02)         &      (0.06)         &      (0.06)         \\
\addlinespace
Low birthweight     &       -0.18         &       -0.13         &        0.31         &       -0.01         &       -0.10         &        0.11         \\
                    &     (43.39)         &     (20.93)         &     (22.46)         &      (0.05)         &      (0.15)         &      (0.15)         \\
\addlinespace
Premature birth     &       -0.17         &        0.05         &        0.12         &        0.01         &        0.21         &       -0.22         \\
                    &     (36.10)         &     (17.41)         &     (18.69)         &      (0.05)         &      (0.14)         &      (0.14)         \\
\addlinespace
Mom born in province&        0.18         &       -0.12         &       -0.06         &        0.02         &       -0.17\sym{**} &        0.15\sym{*}  \\
                    &     (23.41)         &     (11.29)         &     (12.12)         &      (0.02)         &      (0.06)         &      (0.06)         \\
\addlinespace
Dad born in province&        0.17         &       -0.09         &       -0.09         &       -0.03         &        0.02         &        0.00         \\
                    &     (15.96)         &      (7.70)         &      (8.26)         &      (0.02)         &      (0.06)         &      (0.06)         \\
\addlinespace
Mom Max Edu: University&       -0.18         &        0.05         &        0.12         &       -0.89\sym{**} &        0.58\sym{***}&        0.31         \\
                    &     (26.20)         &     (12.64)         &     (13.56)         &      (0.30)         &      (0.15)         &      (0.19)         \\
\addlinespace
Dad Max Edu: University&       -0.32         &        0.15         &        0.17         &        0.04         &       -0.16\sym{*}  &        0.12         \\
                    &     (27.95)         &     (13.48)         &     (14.47)         &      (0.02)         &      (0.08)         &      (0.08)         \\
\addlinespace
Has 2 siblings      &        0.02         &        0.11         &       -0.13         &       -0.01         &        0.01         &       -0.00         \\
                    &      (0.01)         &      (0.07)         &      (0.07)         &      (0.03)         &      (0.07)         &      (0.07)         \\
\addlinespace
Has more than 2 siblings&        0.02         &       -0.12         &        0.10         &        0.02         &       -0.06         &        0.04         \\
                    &      (0.02)         &      (0.09)         &      (0.09)         &      (0.02)         &      (0.10)         &      (0.10)         \\
\addlinespace
Caregiver is Catholic&        0.33         &       -0.02         &       -0.30         &       -0.05         &        0.25         &       -0.20         \\
                    &     (43.83)         &     (21.14)         &     (22.69)         &      (0.04)         &      (0.30)         &      (0.30)         \\
\addlinespace
Caregiver is Muslim &        0.32         &       -0.10         &       -0.22         &       -0.89\sym{**} &        0.56         &        0.33         \\
                    &     (43.83)         &     (21.14)         &     (22.69)         &      (0.30)         &      (0.45)         &      (0.45)         \\
\addlinespace
Caregiver is religious&       -0.35         &        0.22         &        0.13         &        0.05         &        0.01         &       -0.06         \\
                    &     (43.83)         &     (21.14)         &     (22.69)         &      (0.04)         &      (0.31)         &      (0.31)         \\
\addlinespace
Mother: born outside of Italy&        0.36         &       -0.11         &       -0.25         &        0.26\sym{**} &        4.83\sym{***}&       -5.09\sym{***}\\
                    &     (28.34)         &     (13.67)         &     (14.67)         &      (0.10)         &      (0.39)         &      (0.39)         \\
\midrule
Observations        &         421         &         421         &         421         &         300         &         300         &         300         \\
\bottomrule
\end{tabular}
}


\begin{tablenotes}
\footnotesize\raggedright{Note: This table shows the results from a multinomial logit that uses baseline characteristics to predict enrollment in municipal preschool, other preschool, or no preschool. The columns titled ``None'' display the marginal effects and standard errors of attending no preschool. Similarly, the columns titled ``Other'' display the same estimates for attending a non-municipal preschool and those titled ``Municipal" display estimates for attending a municipal school. Standard errors are reported in parentheses. Stars show statistical significance as follows: * $p < 0.05$, ** $p < 0.01$, *** $p < 0.001$.}
\end{tablenotes}
\end{threeparttable}
\end{adjustbox}
\end{table}

\begin{table}[H] 
\centering
\caption{Multinomial Logit, Adult Cohorts, Reggio Emilia} \label{mlogit-adult-RE}
\begin{adjustbox}{width=\textwidth}
\begin{threeparttable}
{
\def\sym#1{\ifmmode^{#1}\else\(^{#1}\)\fi}
\begin{tabular}{l*{6}{c}}
\toprule
 &\multicolumn{3}{c}{Adults 30s}&\multicolumn{3}{c}{Adults 40s} \\
                    &\multicolumn{1}{c}{None}&\multicolumn{1}{c}{Other}&\multicolumn{1}{c}{Municipal}&\multicolumn{1}{c}{None}&\multicolumn{1}{c}{Other}&\multicolumn{1}{c}{Municipal}\\
\midrule
Male                &       -0.07         &       -0.04         &        0.11         &       -0.05         &        0.01         &        0.03         \\
                    &      (0.05)         &      (0.05)         &      (0.06)         &      (0.05)         &      (0.05)         &      (0.06)         \\
\addlinespace
Mom born in province&        0.05         &       -0.00         &       -0.05         &       -0.25\sym{***}&        0.04         &        0.21\sym{*}  \\
                    &      (0.07)         &      (0.08)         &      (0.09)         &      (0.05)         &      (0.08)         &      (0.09)         \\
\addlinespace
Dad born in province&       -0.01         &       -0.02         &        0.04         &       -0.08         &       -0.04         &        0.11         \\
                    &      (0.07)         &      (0.08)         &      (0.09)         &      (0.06)         &      (0.07)         &      (0.07)         \\
\addlinespace
Has 2 siblings      &        0.05         &       -0.01         &       -0.04         &        0.06         &       -0.08         &        0.02         \\
                    &      (0.05)         &      (0.07)         &      (0.08)         &      (0.06)         &      (0.07)         &      (0.07)         \\
\addlinespace
Has more than 2 siblings&        0.08         &       -0.06         &       -0.02         &        0.04         &       -0.07         &        0.03         \\
                    &      (0.07)         &      (0.09)         &      (0.10)         &      (0.06)         &      (0.08)         &      (0.08)         \\
\addlinespace
Caregiver was religious&        0.11\sym{*}  &        0.07         &       -0.18\sym{**} &        0.06         &        0.03         &       -0.09         \\
                    &      (0.05)         &      (0.05)         &      (0.06)         &      (0.05)         &      (0.06)         &      (0.06)         \\
\addlinespace
Mom Max Edu: Middle School&       -1.05         &       -2.21         &        3.27         &       -1.38         &        2.84         &       -1.46         \\
                    &    (560.60)         &    (350.42)         &    (615.62)         &     (36.11)         &    (103.47)         &     (67.36)         \\
\addlinespace
Mom Max Edu: High School&        0.85         &       -2.91         &        2.06         &       -1.11         &        2.77         &       -1.65         \\
                    &    (553.74)         &    (349.14)         &    (612.93)         &     (36.11)         &    (103.47)         &     (67.36)         \\
\addlinespace
Mom Max Edu: University&        0.95         &       -3.06         &        2.11         &       -1.07         &        2.75         &       -1.68         \\
                    &    (553.74)         &    (349.14)         &    (612.93)         &     (36.11)         &    (103.47)         &     (67.36)         \\
\midrule
Observations        &         280         &         280         &         280         &         285         &         285         &         285         \\
\bottomrule
\end{tabular}
}

\begin{tablenotes}
\footnotesize\raggedright{Note: This table shows the results from a multinomial logit that uses baseline characteristics to predict enrollment in municipal preschool, other preschool, or no preschool. The columns titled ``None'' display the marginal effects and standard errors of attending no preschool. Similarly, the columns titled ``Other'' display the same estimates for attending a non-municipal preschool and those titled ``Municipal" display estimates for attending a municipal school. Standard errors are reported in parentheses. Stars show statistical significance as follows: * $p < 0.05$, ** $p < 0.01$, *** $p < 0.001$.}
\end{tablenotes}
\end{threeparttable}
\end{adjustbox}
\end{table}

\begin{table}[H] 
\centering
\caption{Multinomial Logit, Child and Adolescent Cohorts, Parma} \label{mlogit-chi-ado-PR}
\begin{adjustbox}{width=\textwidth}
\begin{threeparttable}
{
\def\sym#1{\ifmmode^{#1}\else\(^{#1}\)\fi}
\begin{tabular}{l*{6}{c}}
\toprule
& \multicolumn{3}{c}{Children} & \multicolumn{3}{c}{Adolescents} \\
                    &\multicolumn{1}{c}{None}&\multicolumn{1}{c}{Other}&\multicolumn{1}{c}{Municipal}&\multicolumn{1}{c}{None}&\multicolumn{1}{c}{Other}&\multicolumn{1}{c}{Municipal}\\
\midrule
Male                &        0.02         &       -0.03         &        0.01         &        0.01         &        0.15\sym{**} &       -0.16\sym{***}\\
                    &      (0.02)         &      (0.04)         &      (0.04)         &      (0.02)         &      (0.05)         &      (0.05)         \\
\addlinespace
Low birthweight     &       -0.22         &        2.00         &       -1.78         &        0.21         &       -0.24         &        0.03         \\
                    &     (60.54)         &    (207.66)         &    (207.44)         &     (15.49)         &     (12.97)         &      (2.52)         \\
\addlinespace
Premature birth     &       -0.28         &        0.22         &        0.06         &       -0.19         &        0.33         &       -0.14         \\
                    &     (54.48)         &     (45.90)         &      (8.58)         &     (15.49)         &     (12.97)         &      (2.52)         \\
\addlinespace
Mom born in province&        0.02         &       -0.07         &        0.05         &       -0.02         &       -0.02         &        0.04         \\
                    &      (0.02)         &      (0.05)         &      (0.05)         &      (0.02)         &      (0.05)         &      (0.05)         \\
\addlinespace
Dad born in province&       -0.04         &        0.00         &        0.04         &       -0.37         &        0.43         &       -0.06         \\
                    &      (0.03)         &      (0.05)         &      (0.04)         &     (21.70)         &     (18.18)         &      (3.53)         \\
\addlinespace
Mom Max Edu: University&        0.04         &       -0.06         &        0.02         &       -0.00         &       -0.05         &        0.05         \\
                    &      (0.02)         &      (0.05)         &      (0.04)         &      (0.02)         &      (0.06)         &      (0.05)         \\
\addlinespace
Dad Max Edu: University&       -0.01         &        0.05         &       -0.03         &        0.01         &        0.06         &       -0.08         \\
                    &      (0.02)         &      (0.05)         &      (0.04)         &      (0.02)         &      (0.06)         &      (0.06)         \\
\addlinespace
Has 2 siblings      &       -0.01         &       -0.03         &        0.05         &        0.02         &        0.14         &       -0.17\sym{*}  \\
                    &      (0.02)         &      (0.05)         &      (0.04)         &      (0.02)         &      (0.08)         &      (0.08)         \\
\addlinespace
Has more than 2 siblings&       -0.34         &        0.13         &        0.20         &       -0.17         &        0.24         &       -0.07         \\
                    &     (31.11)         &     (26.21)         &      (4.90)         &     (46.49)         &     (38.94)         &      (7.56)         \\
\addlinespace
Caregiver is Catholic&       -0.02         &        0.03         &       -0.01         &        0.14         &       -2.06         &        1.92         \\
                    &      (0.03)         &      (0.09)         &      (0.08)         &     (92.65)         &    (578.27)         &    (583.95)         \\
\addlinespace
Caregiver is Muslim &        0.08\sym{**} &       -0.15         &        0.07         &                 &                 &                 \\
                    &      (0.03)         &      (0.10)         &      (0.10)         &                  &                  &                  \\
\addlinespace
Caregiver is religious&        0.01         &       -0.10         &        0.09         &       -0.17         &        2.11         &       -1.94         \\
                    &      (0.03)         &      (0.11)         &      (0.10)         &     (92.65)         &    (578.27)         &    (583.95)         \\
\addlinespace
Mom born outside of Italy&       -0.01         &        0.01         &       -0.00         &       -0.21         &        2.07         &       -1.86         \\
                    &      (0.03)         &      (0.08)         &      (0.08)         &    (105.49)         &    (508.01)         &    (509.92)         \\
\midrule
Observations        &         349         &         349         &         349         &         254         &         254         &         254         \\
\bottomrule
\end{tabular}
}

\begin{tablenotes}
\footnotesize\raggedright{Note: This table shows the results from a multinomial logit that uses baseline characteristics to predict enrollment in municipal preschool, other preschool, or no preschool. The columns titled ``None'' display the marginal effects and standard errors of attending no preschool. Similarly, the columns titled ``Other'' display the same estimates for attending a non-municipal preschool and those titled ``Municipal" display estimates for attending a municipal school. Standard errors are reported in parentheses. Stars show statistical significance as follows: * $p < 0.05$, ** $p < 0.01$, *** $p < 0.001$.}
\end{tablenotes}
\end{threeparttable}
\end{adjustbox}
\end{table}

\begin{table}[H] 
\centering
\caption{Multinomial Logit, Adult Cohorts, Parma} \label{mlogit-adult-PR}
\begin{threeparttable}
{
\def\sym#1{\ifmmode^{#1}\else\(^{#1}\)\fi}
\begin{tabular}{l*{3}{c}}
\toprule
                    &\multicolumn{1}{c}{None}&\multicolumn{1}{c}{Other}&\multicolumn{1}{c}{Municipal}\\
\midrule
Male                &        0.01         &       -0.07         &        0.05         \\
                    &      (0.05)         &      (0.06)         &      (0.06)         \\
\addlinespace
Mom up to high school &        0.04         &        0.17         &       -0.22         \\
                    &      (0.09)         &      (0.13)         &      (0.12)         \\
\addlinespace
Mom at least uni.  &       -0.06         &        0.15         &       -0.09         \\
                    &      (0.09)         &      (0.13)         &      (0.11)         \\
\addlinespace
Caregiver was religious&        0.07         &       -0.14\sym{*}  &        0.07         \\
                    &      (0.06)         &      (0.07)         &      (0.07)         \\
\addlinespace
Mom born in province&        0.06         &        0.06         &       -0.11         \\
                    &      (0.05)         &      (0.07)         &      (0.06)         \\
\addlinespace
Dad born in province&        0.05         &        0.03         &       -0.07         \\
                    &      (0.06)         &      (0.08)         &      (0.07)         \\
\addlinespace
Has 2 siblings      &        0.12\sym{*}  &       -0.10         &       -0.02         \\
                    &      (0.06)         &      (0.07)         &      (0.07)         \\
\addlinespace
Has more than 2 siblings&        0.20\sym{***}&       -0.25\sym{**} &        0.05         \\
                    &      (0.06)         &      (0.08)         &      (0.08)         \\
\midrule
Observations        &         251         &         251         &         251         \\
\bottomrule
\end{tabular}
}

\begin{tablenotes}
\footnotesize\raggedright{Note: This table shows the results from a multinomial logit that uses baseline characteristics to predict enrollment in municipal preschool, other preschool, or no preschool. The columns titled ``None'' display the marginal effects and standard errors of attending no preschool. Similarly, the columns titled ``Other'' display the same estimates for attending a non-municipal preschool and those titled ``Municipal" display estimates for attending a municipal school. Standard errors are reported in parentheses. Stars show statistical significance as follows: * $p < 0.05$, ** $p < 0.01$, *** $p < 0.001$.}
\end{tablenotes}
\end{threeparttable}
\end{table}

\begin{table}[H] 
\centering
\caption{Multinomial Logit, Child and Adolescent Cohorts, Padova} \label{mlogit-chi-ado-PD}
\begin{adjustbox}{width=\textwidth}
\begin{threeparttable}
\input{../../output/mlogit_Padova_chi-ado_ready.tex}
\begin{tablenotes}
\footnotesize\raggedright{Note: This table shows the results from a multinomial logit that uses baseline characteristics to predict enrollment in municipal preschool, other preschool, or no preschool. The columns titled ``None'' display the marginal effects and standard errors of attending no preschool. Similarly, the columns titled ``Other'' display the same estimates for attending a non-municipal preschool and those titled ``Municipal" display estimates for attending a municipal school. Standard errors are reported in parentheses. Stars show statistical significance as follows: * $p < 0.05$, ** $p < 0.01$, *** $p < 0.001$.}
\end{tablenotes}
\end{threeparttable}
\end{adjustbox}
\end{table}


\begin{table}[H] 
\centering
\caption{Multinomial Logit, Adult Cohorts, Padova} \label{mlogit-adult-PD}
\begin{threeparttable}
{
\def\sym#1{\ifmmode^{#1}\else\(^{#1}\)\fi}
\begin{tabular}{l*{3}{c}}
\toprule
& \multicolumn{3}{c}{Adults 30s} \\
                    &\multicolumn{1}{c}{None}&\multicolumn{1}{c}{Other}&\multicolumn{1}{c}{Municipal}\\
\midrule
Male                &        0.12\sym{*}  &       -0.12\sym{*}  &       -0.01         \\
                    &      (0.05)         &      (0.06)         &      (0.04)         \\
\addlinespace
Mom born in province&        0.00         &        0.02         &       -0.02         \\
                    &      (0.05)         &      (0.06)         &      (0.04)         \\
\addlinespace
Dad born in province&        0.08         &       -0.09         &        0.01         \\
                    &      (0.06)         &      (0.07)         &      (0.05)         \\
\addlinespace
Has 2 siblings      &        0.14\sym{**} &       -0.16\sym{*}  &        0.02         \\
                    &      (0.05)         &      (0.06)         &      (0.04)         \\
\addlinespace
Has more than 2 siblings&        0.11         &       -0.10         &       -0.01         \\
                    &      (0.06)         &      (0.08)         &      (0.06)         \\
\addlinespace
Caregiver was religious&        0.06         &       -0.03         &       -0.03         \\
                    &      (0.06)         &      (0.07)         &      (0.04)         \\
\addlinespace
Mom Max Edu: Middle School&        1.63         &       -1.28         &       -0.36         \\
                    &     (66.89)         &     (57.64)         &      (9.25)         \\
\addlinespace
Mom Max Edu: High School&        1.70         &       -1.37         &       -0.33         \\
                    &     (66.89)         &     (57.64)         &      (9.25)         \\
\addlinespace
Mom Max Edu: University&        1.64         &       -1.30         &       -0.34         \\
                    &     (66.89)         &     (57.64)         &      (9.25)         \\
\midrule
Observations        &         251         &         251         &         251         \\
\bottomrule
\end{tabular}
}

\begin{tablenotes}
\footnotesize\raggedright{Note: This table shows the results from a multinomial logit that uses baseline characteristics to predict enrollment in municipal preschool, other preschool, or no preschool. The columns titled ``None'' display the marginal effects and standard errors of attending no preschool. Similarly, the columns titled ``Other'' display the same estimates for attending a non-municipal preschool and those titled ``Municipal" display estimates for attending a municipal school. Standard errors are reported in parentheses. Stars show statistical significance as follows: * $p < 0.05$, ** $p < 0.01$, *** $p < 0.001$.}
\end{tablenotes}
\end{threeparttable}
\end{table}
