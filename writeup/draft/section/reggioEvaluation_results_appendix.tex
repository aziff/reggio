\subsection{Estimation Results for Reggio Approach Preschools, Comparison with Other School Types} \label{app:comparison-reli-stat}


\begin{table}[H] \caption{Estimation Results for Main Outcomes, Comparison to Religious Preschools, Child Cohort} \label{ols-M-child-reg-reli}
\scalebox{0.59}{\begin{tabular}{l c c c c c c c c c c c}
\toprule
& \multicolumn{5}{c}{Within Reggio} & \multicolumn{3}{c}{With Parma} & \multicolumn{3}{c}{With Padova} \\\cmidrule(lr){2-6} \cmidrule(lr){7-9} \cmidrule(lr){10-12}
 & None & BIC & Full & PSMR & KMR & DidPm & PSMPm & KMPm & DidPv & PSMPv & KMPv \\
\midrule
IQ Factor & -0.32 & -0.32 & -0.32 & -0.29 & -0.35 & -0.03 & -0.44 & -0.50 & -0.11 & -0.44 & -0.40 \\
\quad \textit{Unadjusted P-Value} & (0.00)*** & (0.00)*** & (0.00)*** & (0.01)*** & (0.00)*** & (0.83) & (0.00)*** & (0.00)*** & (0.52) & (0.00)*** & (0.00)*** \\
\quad \textit{Stepdown P-Value} & (0.07)** & (0.07)** & (0.03)*** & (0.09)** & (0.02)*** & (0.99) & (0.00)*** & (0.01)*** & (0.89) & (0.00)*** & (0.01)*** \\
SDQ Composite - Child & 1.19 & 1.28 & 1.87 & 0.83 & 1.18 & 0.62 & 1.13 & 1.26 & 1.99 & 0.48 & 0.78 \\
\quad \textit{Unadjusted P-Value} & (0.04)*** & (0.03)*** & (0.00)*** & (0.19) & (0.06)** & (0.43) & (0.11)* & (0.10)** & (0.03)*** & (0.42) & (0.19) \\
\quad \textit{Stepdown P-Value} & (0.34) & (0.20) & (0.01)*** & (0.77) & (0.45) & (0.99) & (0.52) & (0.46) & (0.18) & (0.86) & (0.69) \\
Not Obese & -0.10 & -0.11 & -0.15 & -0.14 & -0.11 & -0.01 & -0.15 & -0.15 & -0.02 & -0.10 & -0.07 \\
\quad \textit{Unadjusted P-Value} & (0.07)** & (0.06)** & (0.02)*** & (0.01)*** & (0.09)** & (0.84) & (0.01)*** & (0.02)*** & (0.86) & (0.09)** & (0.23) \\
\quad \textit{Stepdown P-Value} & (0.49) & (0.43) & (0.10)** & (0.11) & (0.55) & (0.99) & (0.10) & (0.18) & (0.94) & (0.42) & (0.69) \\
Not Overweight & -0.02 & -0.03 & -0.05 & -0.02 & -0.02 & -0.02 & 0.10 & 0.00 & -0.07 & -0.01 & -0.01 \\
\quad \textit{Unadjusted P-Value} & (0.58) & (0.56) & (0.36) & (0.62) & (0.62) & (0.76) & (0.19) & (0.94) & (0.26) & (0.77) & (0.76) \\
\quad \textit{Stepdown P-Value} & (0.97) & (0.98) & (0.79) & (0.99) & (0.99) & (0.99) & (0.60) & (0.99) & (0.84) & (0.95) & (0.98) \\
Health is Good & -0.01 & -0.01 & 0.01 & -0.04 & -0.02 & 0.07 & 0.06 & -0.03 & -0.03 & -0.04 & -0.06 \\
\quad \textit{Unadjusted P-Value} & (0.82) & (0.83) & (0.90) & (0.57) & (0.74) & (0.43) & (0.41) & (0.70) & (0.76) & (0.54) & (0.31) \\
\quad \textit{Stepdown P-Value} & (0.99) & (0.98) & (0.98) & (0.99) & (0.99) & (0.99) & (0.71) & (0.99) & (0.94) & (0.90) & (0.76) \\
Not Excited to Learn & 0.00 & 0.01 & 0.00 & 0.00 & 0.00 & -0.00 & -0.05 & -0.04 & -0.04 & 0.01 & 0.01 \\
\quad \textit{Unadjusted P-Value} & (0.91) & (0.73) & (0.89) & (0.93) & (0.87) & (0.95) & (0.23) & (0.27) & (0.30) & (0.77) & (0.81) \\
\quad \textit{Stepdown P-Value} & (0.99) & (0.98) & (0.98) & (0.99) & (0.99) & (0.99) & (0.63) & (0.81) & (0.82) & (0.95) & (0.98) \\
Problems Sitting Still & -0.04 & -0.02 & -0.10 & -0.01 & -0.00 & -0.08 & 0.01 & -0.00 & -0.13 & 0.06 & 0.00 \\
\quad \textit{Unadjusted P-Value} & (0.38) & (0.62) & (0.05)*** & (0.80) & (0.95) & (0.16) & (0.84) & (0.97) & (0.07)** & (0.20) & (0.96) \\
\quad \textit{Stepdown P-Value} & (0.91) & (0.98) & (0.23) & (0.99) & (0.99) & (0.77) & (0.82) & (0.99) & (0.36) & (0.63) & (0.98) \\
How Much Child Likes School & 0.12 & 0.11 & 0.16 & 0.15 & 0.11 & 0.24 & -0.15 & -0.13 & 0.31 & 0.20 & 0.26 \\
\quad \textit{Unadjusted P-Value} & (0.12)* & (0.14)* & (0.07)** & (0.10)* & (0.20) & (0.01)*** & (0.01)*** & (0.08)** & (0.02)*** & (0.02)*** & (0.00)*** \\
\quad \textit{Stepdown P-Value} & (0.51) & (0.61) & (0.27) & (0.59) & (0.78) & (0.11) & (0.13) & (0.43) & (0.09)** & (0.14) & (0.03)*** \\
Num. of Friends & -0.22 & -0.20 & -0.01 & -0.07 & -0.15 & -0.19 & -1.07 & -0.77 & 0.17 & -1.60 & -1.56 \\
\quad \textit{Unadjusted P-Value} & (0.44) & (0.48) & (0.97) & (0.80) & (0.63) & (0.74) & (0.10)* & (0.35) & (0.85) & (0.00)*** & (0.00)*** \\
\quad \textit{Stepdown P-Value} & (0.93) & (0.98) & (0.98) & (0.99) & (0.99) & (0.99) & (0.52) & (0.88) & (0.94) & (0.00)*** & (0.03)*** \\
Candy Game: Willing to Share Candies & 0.00 & 0.00 & 0.03 & -0.03 & 0.01 & 0.01 & -0.04 & -0.00 & 0.03 & -0.07 & -0.05 \\
\quad \textit{Unadjusted P-Value} & (0.96) & (0.98) & (0.46) & (0.47) & (0.86) & (0.77) & (0.36) & (1.00) & (0.64) & (0.06)** & (0.09)** \\
\quad \textit{Stepdown P-Value} & (0.99) & (0.98) & (0.86) & (0.98) & (0.99) & (0.99) & (0.71) & (0.99) & (0.93) & (0.33) & (0.45) \\
\bottomrule
\end{tabular}
}
\vspace{1ex} \\
\footnotesize\raggedright{\underline{Note1 :} This table shows the estimates of the coefficient for attending Reggio Approach preschools from multiple methods. We compare Reggio Approach individuals with those who attended religious preschools. Column title indicates the corresponding control set and and model. \textbf{None} = OLS estimate with no control variables. \textbf{BIC} = OLS estimate with controls selected by Bayesian Information Criterion (BIC) and additional controls for male indicator, migrant indicator, and ITC attendance indicator. \textbf{Full} = OLS estimate with the full set of controls. \textbf{PSMR} =  propensity score matching between Reggio Approach people and people in Reggio who attended other types of preschool. \textbf{KMR} = Epanechinikov kernel matching between Reggio Approach people and people in Reggio who attended religious preschools. \textbf{DidPm} = difference-in-difference estimate of (Reggio Muni - Parma Muni) - (Reggio Reli - Parma Reli). \textbf{PSMPm} = propensity score matching between Reggio Approach people and people who attended Parma religious preschools. \textbf{KMPm} = Epanechinikov kernel matching between Reggio Approach people and people who attended Parma religious preschools. \textbf{DidPv} = difference-in-difference estimate of (Reggio Muni - Padova Muni) - (Reggio Reli - Padova Reli). \textbf{PSMPv} = propensity score matching between Reggio Approach people and people who attended Padova religious preschools. \textbf{KMPv} = Epanechinikov kernel matching between Reggio Approach people and people who attended Padova religious preschools.} 

\footnotesize\raggedright{\underline{Note2 :} Both unadjusted p-value and stepdown p-value are reported. ***, **, and * indicate significance of the coefficients at the 15\%, 10\%, and 5\% levels respectively.}

\end{table}



\begin{table}[H] \caption{Estimation Results for Main Outcomes, Comparison to State Preschools, Child Cohort} \label{ols-M-child-reg-stat}
\scalebox{0.59}{\begin{tabular}{l c c c c c c c c c c c}
\toprule
 & None & BIC & Full & PSMR & KMR & DidPm & PSMPm & KMPm & DidPv & PSMPv & KMPv \\
\midrule
IQ Factor & 0.24 & 0.00 & 0.04 & 0.16 & 0.02 & -0.03 & -0.58 & -0.57 & -0.24 & -0.18 & -0.14 \\
\quad \textit{Unadjusted P-Value} & (0.22) & (0.99) & (0.84) & (0.42) & (0.92) & (0.83) & (0.00)*** & (0.00)*** & (0.40) & (0.25) & (0.54) \\
\quad \textit{Stepdown P-Value} & (0.68) & (0.99) & (0.94) & (0.98) & (0.99) & (0.99) & (0.04)*** & (0.05)*** & (0.71) & (0.82) & (0.97) \\
SDQ Composite - Child & 2.21 & 1.54 & 2.37 & 1.66 & 0.89 & 0.62 & -0.50 & -1.10 & 2.33 & 0.59 & 0.16 \\
\quad \textit{Unadjusted P-Value} & (0.03)*** & (0.11)* & (0.01)*** & (0.09)** & (0.51) & (0.43) & (0.63) & (0.32) & (0.07)** & (0.47) & (0.87) \\
\quad \textit{Stepdown P-Value} & (0.07)** & (0.52) & (0.02)*** & (0.61) & (0.99) & (0.99) & (0.99) & (0.86) & (0.30) & (0.94) & (0.98) \\
Not Obese & 0.09 & -0.02 & 0.01 & -0.06 & -0.01 & -0.01 & -0.27 & -0.28 & -0.08 & 0.00 & -0.02 \\
\quad \textit{Unadjusted P-Value} & (0.27) & (0.84) & (0.89) & (0.54) & (0.94) & (0.84) & (0.00)*** & (0.00)*** & (0.52) & (0.98) & (0.88) \\
\quad \textit{Stepdown P-Value} & (0.81) & (0.99) & (0.96) & (0.99) & (0.99) & (0.99) & (0.02)*** & (0.05)*** & (0.71) & (0.99) & (0.98) \\
Not Overweight & -0.04 & -0.02 & -0.02 & 0.03 & 0.01 & -0.02 & 0.01 & -0.03 & 0.01 & -0.09 & -0.08 \\
\quad \textit{Unadjusted P-Value} & (0.42) & (0.78) & (0.76) & (0.78) & (0.87) & (0.76) & (0.84) & (0.78) & (0.87) & (0.13)* & (0.05)*** \\
\quad \textit{Stepdown P-Value} & (0.94) & (0.92) & (0.92) & (0.99) & (0.99) & (0.99) & (0.99) & (0.99) & (0.94) & (0.60) & (0.36) \\
Health is Good & -0.02 & 0.04 & 0.02 & -0.02 & -0.01 & 0.07 & 0.05 & -0.02 & 0.11 & -0.12 & -0.10 \\
\quad \textit{Unadjusted P-Value} & (0.77) & (0.64) & (0.84) & (0.85) & (0.90) & (0.43) & (0.67) & (0.89) & (0.33) & (0.09)** & (0.27) \\
\quad \textit{Stepdown P-Value} & (0.94) & (0.92) & (0.94) & (0.99) & (0.99) & (0.99) & (0.99) & (0.99) & (0.71) & (0.50) & (0.87) \\
Not Excited to Learn & -0.05 & -0.04 & -0.03 & 0.01 & 0.00 & -0.00 & -0.00 & 0.00 & -0.04 & -0.01 & 0.00 \\
\quad \textit{Unadjusted P-Value} & (0.27) & (0.33) & (0.34) & (0.83) & (0.96) & (0.95) & (0.86) & (0.95) & (0.47) & (0.78) & (0.97) \\
\quad \textit{Stepdown P-Value} & (0.68) & (0.92) & (0.59) & (0.99) & (0.99) & (0.99) & (0.99) & (0.99) & (0.71) & (0.98) & (0.98) \\
Problems Sitting Still & 0.04 & 0.06 & 0.07 & 0.04 & 0.05 & -0.08 & 0.07 & 0.09 & -0.10 & 0.11 & 0.11 \\
\quad \textit{Unadjusted P-Value} & (0.42) & (0.30) & (0.29) & (0.46) & (0.46) & (0.16) & (0.04)*** & (0.25) & (0.18) & (0.01)*** & (0.02)*** \\
\quad \textit{Stepdown P-Value} & (0.94) & (0.92) & (0.58) & (0.98) & (0.99) & (0.77) & (0.25) & (0.77) & (0.65) & (0.09)** & (0.26) \\
How Much Child Likes School & 0.19 & 0.09 & 0.12 & 0.09 & -0.00 & 0.24 & -0.00 & -0.08 & 0.23 & 0.06 & 0.06 \\
\quad \textit{Unadjusted P-Value} & (0.10)* & (0.39) & (0.33) & (0.40) & (0.99) & (0.01)*** & (0.99) & (0.57) & (0.17) & (0.59) & (0.67) \\
\quad \textit{Stepdown P-Value} & (0.40) & (0.92) & (0.58) & (0.98) & (0.99) & (0.11) & (0.99) & (0.98) & (0.52) & (0.98) & (0.98) \\
Num. of Friends & -0.28 & -0.77 & -0.64 & -0.98 & -0.90 & -0.19 & 0.35 & 0.18 & -1.42 & -0.17 & -0.42 \\
\quad \textit{Unadjusted P-Value} & (0.47) & (0.07)** & (0.13)* & (0.14)* & (0.07)** & (0.74) & (0.49) & (0.73) & (0.15) & (0.67) & (0.44) \\
\quad \textit{Stepdown P-Value} & (0.94) & (0.39) & (0.28) & (0.74) & (0.46) & (0.99) & (0.98) & (0.99) & (0.52) & (0.98) & (0.96) \\
Candy Game: Willing to Share Candies & 0.04 & 0.01 & 0.05 & -0.01 & -0.02 & 0.01 & -0.09 & -0.08 & 0.01 & -0.09 & -0.08 \\
\quad \textit{Unadjusted P-Value} & (0.48) & (0.93) & (0.38) & (0.90) & (0.77) & (0.77) & (0.00)*** & (0.09)** & (0.85) & (0.01)*** & (0.10)** \\
\quad \textit{Stepdown P-Value} & (0.94) & (0.99) & (0.59) & (0.99) & (0.99) & (0.99) & (0.06)** & (0.40) & (0.94) & (0.09)** & (0.53) \\
\bottomrule
\end{tabular}
}
\vspace{1ex} \\
\footnotesize\raggedright{\underline{Note1 :} This table shows the estimates of the coefficient for attending Reggio Approach preschools from multiple methods. We compare Reggio Approach individuals with those who attended state preschools. Column title indicates the corresponding control set and and model. \textbf{None} = OLS estimate with no control variables. \textbf{BIC} = OLS estimate with controls selected by Bayesian Information Criterion (BIC) and additional controls for male indicator, migrant indicator, and ITC attendance indicator. \textbf{Full} = OLS estimate with the full set of controls. \textbf{PSMR} =  propensity score matching between Reggio Approach people and people in Reggio who attended state preschool. \textbf{KMR} = Epanechinikov kernel matching between Reggio Approach people and people in Reggio who attended state preschools. \textbf{DidPm} = difference-in-difference estimate of (Reggio Muni - Parma Muni) - (Reggio Reli - Parma Reli). \textbf{PSMPm} = propensity score matching between Reggio Approach people and people who attended Parma state preschools. \textbf{KMPm} = Epanechinikov kernel matching between Reggio Approach people and people who attended Parma state preschools. \textbf{DidPv} = difference-in-difference estimate of (Reggio Muni - Padova Muni) - (Reggio Reli - Padova Reli). \textbf{PSMPv} = propensity score matching between Reggio Approach people and people who attended Padova state preschools. \textbf{KMPv} = Epanechinikov kernel matching between Reggio Approach people and people who attended Padova state preschools.} 

\footnotesize\raggedright{\underline{Note2 :} Both unadjusted p-value and stepdown p-value are reported. ***, **, and * indicate significance of the coefficients at the 15\%, 10\%, and 5\% levels respectively.}

\end{table}


\begin{table}[H] \caption{Estimation Results for Main Outcomes, Comparison to Religious Preschools, Adolescent Cohort} \label{ols-M-adol-reg-reli}
\scalebox{0.59}{\begin{tabular}{l c c c c c c c c c c c}
\toprule
 & None & BIC & Full & PSMR & KMR & DidPm & PSMPm & KMPm & DidPv & PSMPv & KMPv \\
\midrule
IQ Factor & -0.13 & -0.15 & -0.07 & -0.16 & -0.19 & -0.13 & 0.04 & 0.04 & -0.13 & 0.14 & 0.13 \\
\quad \textit{Unadjusted P-Value} & (0.21) & (0.15)* & (0.49) & (0.10)** & (0.14)* & (0.42) & (0.78) & (0.80) & (0.51) & (0.32) & (0.58) \\
\quad \textit{Stepdown P-Value} & (0.95) & (0.88) & (0.76) & (0.63) & (0.79) & (0.99) & (0.99) & (0.98) & (0.97) & (0.95) & (0.99) \\
SDQ Composite - Child & 0.55 & 0.82 & 1.25 & 0.36 & 0.81 & -0.75 & 2.95 & 3.16 & -0.12 & 1.05 & 0.53 \\
\quad \textit{Unadjusted P-Value} & (0.43) & (0.30) & (0.10)** & (0.73) & (0.38) & (0.49) & (0.04)*** & (0.02)*** & (0.90) & (0.20) & (0.61) \\
\quad \textit{Stepdown P-Value} & (0.98) & (0.91) & (0.30) & (0.99) & (0.95) & (0.99) & (0.41) & (0.21) & (0.97) & (0.86) & (0.99) \\
SDQ Composite & 1.44 & 1.74 & 1.54 & 1.96 & 1.69 & 1.12 & 1.01 & 0.59 & 0.85 & 2.16 & 2.38 \\
\quad \textit{Unadjusted P-Value} & (0.04)*** & (0.03)*** & (0.06)** & (0.06)** & (0.06)** & (0.28) & (0.35) & (0.63) & (0.43) & (0.06)** & (0.05)** \\
\quad \textit{Stepdown P-Value} & (0.37) & (0.27) & (0.20) & (0.53) & (0.54) & (0.97) & (0.87) & (0.98) & (0.97) & (0.62) & (0.51) \\
Depression Score - positive & 1.89 & 2.97 & 2.34 & 2.84 & 3.27 & 1.76 & 1.96 & 2.36 & 1.61 & 1.84 & 2.41 \\
\quad \textit{Unadjusted P-Value} & (0.03)*** & (0.00)*** & (0.02)*** & (0.01)*** & (0.00)*** & (0.15)* & (0.04)*** & (0.08)** & (0.23) & (0.11)* & (0.09)** \\
\quad \textit{Stepdown P-Value} & (0.27) & (0.02)*** & (0.08)** & (0.14) & (0.05)** & (0.86) & (0.41) & (0.55) & (0.94) & (0.76) & (0.67) \\
Locus of Control - positive & 0.08 & 0.16 & 0.09 & 0.23 & 0.14 & 0.00 & 0.25 & 0.11 & 0.12 & 0.31 & 0.38 \\
\quad \textit{Unadjusted P-Value} & (0.41) & (0.12)* & (0.38) & (0.01)*** & (0.25) & (1.00) & (0.03)*** & (0.55) & (0.42) & (0.11)* & (0.03)*** \\
\quad \textit{Stepdown P-Value} & (0.98) & (0.78) & (0.76) & (0.15) & (0.89) & (0.99) & (0.37) & (0.98) & (0.97) & (0.76) & (0.34) \\
Not Obese & -0.11 & -0.12 & -0.10 & -0.08 & -0.06 & 0.00 & -0.14 & -0.16 & -0.08 & 0.07 & 0.08 \\
\quad \textit{Unadjusted P-Value} & (0.01)*** & (0.02)*** & (0.03)*** & (0.08)** & (0.20) & (0.96) & (0.00)*** & (0.01)*** & (0.28) & (0.36) & (0.41) \\
\quad \textit{Stepdown P-Value} & (0.14) & (0.20) & (0.14) & (0.59) & (0.88) & (0.99) & (0.02)*** & (0.12) & (0.97) & (0.95) & (0.97) \\
Not Overweight & 0.00 & -0.02 & -0.02 & -0.04 & 0.01 & 0.08 & -0.04 & -0.02 & -0.04 & -0.04 & -0.03 \\
\quad \textit{Unadjusted P-Value} & (1.00) & (0.54) & (0.54) & (0.38) & (0.85) & (0.04)*** & (0.08)** & (0.46) & (0.18) & (0.20) & (0.47) \\
\quad \textit{Stepdown P-Value} & (0.98) & (0.99) & (0.76) & (0.98) & (0.99) & (0.51) & (0.48) & (0.98) & (0.94) & (0.86) & (0.99) \\
Health is Good & 0.04 & 0.05 & 0.06 & -0.03 & 0.00 & 0.20 & -0.02 & -0.04 & 0.10 & 0.06 & 0.01 \\
\quad \textit{Unadjusted P-Value} & (0.53) & (0.42) & (0.37) & (0.68) & (0.95) & (0.05)** & (0.77) & (0.73) & (0.31) & (0.59) & (0.91) \\
\quad \textit{Stepdown P-Value} & (0.98) & (0.99) & (0.76) & (0.99) & (0.99) & (0.51) & (0.99) & (0.98) & (0.97) & (0.97) & (0.99) \\
Go To School & 0.03 & 0.01 & 0.02 & -0.02 & -0.00 & 0.04 & 0.01 & 0.00 & 0.05 & -0.04 & -0.02 \\
\quad \textit{Unadjusted P-Value} & (0.30) & (0.79) & (0.38) & (0.59) & (0.91) & (0.42) & (0.90) & (0.93) & (0.17) & (0.12)* & (0.30) \\
\quad \textit{Stepdown P-Value} & (0.95) & (0.99) & (0.76) & (0.99) & (0.99) & (0.98) & (0.99) & (0.98) & (0.75) & (0.76) & (0.97) \\
How Much Child Likes School & -0.09 & -0.03 & -0.12 & -0.01 & -0.09 & -0.15 & 0.39 & 0.37 & -0.07 & 0.05 & -0.11 \\
\quad \textit{Unadjusted P-Value} & (0.45) & (0.82) & (0.36) & (0.92) & (0.53) & (0.41) & (0.09)** & (0.12)* & (0.70) & (0.77) & (0.60) \\
\quad \textit{Stepdown P-Value} & (0.98) & (0.99) & (0.76) & (0.99) & (0.98) & (0.99) & (0.48) & (0.67) & (0.97) & (0.97) & (0.99) \\
Days of Sport (Weekly) & -0.25 & -0.33 & -0.07 & -0.41 & -0.42 & -0.16 & -1.06 & -1.16 & -0.42 & -0.38 & -0.53 \\
\quad \textit{Unadjusted P-Value} & (0.32) & (0.25) & (0.81) & (0.33) & (0.20) & (0.69) & (0.05)*** & (0.02)*** & (0.30) & (0.41) & (0.20) \\
\quad \textit{Stepdown P-Value} & (0.96) & (0.89) & (0.88) & (0.98) & (0.88) & (0.99) & (0.41) & (0.20) & (0.97) & (0.95) & (0.88) \\
Num. of Friends & -0.06 & 0.02 & 0.29 & 0.08 & 1.07 & -0.71 & -1.92 & -1.95 & -2.17 & 0.05 & 0.91 \\
\quad \textit{Unadjusted P-Value} & (0.96) & (0.99) & (0.82) & (0.95) & (0.57) & (0.76) & (0.38) & (0.38) & (0.40) & (0.98) & (0.71) \\
\quad \textit{Stepdown P-Value} & (0.98) & (0.99) & (0.92) & (0.99) & (0.99) & (0.99) & (0.87) & (0.97) & (0.97) & (0.97) & (0.99) \\
Volunteers & -0.03 & 0.03 & 0.03 & -0.02 & -0.00 & -0.01 & 0.25 & 0.22 & -0.03 & 0.06 & 0.03 \\
\quad \textit{Unadjusted P-Value} & (0.66) & (0.73) & (0.72) & (0.77) & (0.95) & (0.92) & (0.00)*** & (0.04)*** & (0.72) & (0.52) & (0.75) \\
\quad \textit{Stepdown P-Value} & (0.98) & (0.99) & (0.83) & (0.99) & (0.99) & (0.99) & (0.04)*** & (0.32) & (0.97) & (0.97) & (0.99) \\
Trust Score & -0.00 & 0.10 & 0.01 & 0.23 & 0.23 & 0.12 & -0.50 & -0.27 & -0.19 & 0.13 & 0.16 \\
\quad \textit{Unadjusted P-Value} & (0.98) & (0.65) & (0.97) & (0.44) & (0.36) & (0.70) & (0.29) & (0.46) & (0.51) & (0.61) & (0.60) \\
\quad \textit{Stepdown P-Value} & (0.98) & (0.99) & (0.99) & (0.98) & (0.95) & (0.99) & (0.80) & (0.98) & (0.97) & (0.97) & (0.99) \\
\bottomrule
\end{tabular}
}
\vspace{1ex} \\
\footnotesize\raggedright{\underline{Note1 :} This table shows the estimates of the coefficient for attending Reggio Approach preschools from multiple methods. We compare Reggio Approach individuals with those who attended religious preschools. Column title indicates the corresponding control set and and model. \textbf{None} = OLS estimate with no control variables. \textbf{BIC} = OLS estimate with controls selected by Bayesian Information Criterion (BIC) and additional controls for male indicator, migrant indicator, and ITC attendance indicator. \textbf{Full} = OLS estimate with the full set of controls. \textbf{PSMR} =  propensity score matching between Reggio Approach people and people in Reggio who attended other types of preschool. \textbf{KMR} = Epanechinikov kernel matching between Reggio Approach people and people in Reggio who attended religious preschools. \textbf{DidPm} = difference-in-difference estimate of (Reggio Muni - Parma Muni) - (Reggio Reli - Parma Reli). \textbf{PSMPm} = propensity score matching between Reggio Approach people and people who attended Parma religious preschools. \textbf{KMPm} = Epanechinikov kernel matching between Reggio Approach people and people who attended Parma religious preschools. \textbf{DidPv} = difference-in-difference estimate of (Reggio Muni - Padova Muni) - (Reggio Reli - Padova Reli). \textbf{PSMPv} = propensity score matching between Reggio Approach people and people who attended Padova religious preschools. \textbf{KMPv} = Epanechinikov kernel matching between Reggio Approach people and people who attended Padova religious preschools.} 

\footnotesize\raggedright{\underline{Note2 :} Both unadjusted p-value and stepdown p-value are reported. ***, **, and * indicate significance of the coefficients at the 15\%, 10\%, and 5\% levels respectively.}

\end{table}

\begin{table}[H] \caption{Estimation Results for Main Outcomes, Comparison to State Preschools, Adolescent Cohort} \label{ols-M-adol-reg-stat}
\scalebox{0.59}{\begin{tabular}{l c c c c c c c c c c c}
\toprule
& \multicolumn{5}{c}{Within Reggio} & \multicolumn{3}{c}{With Parma} & \multicolumn{3}{c}{With Padova} \\\cmidrule(lr){2-6} \cmidrule(lr){7-9} \cmidrule(lr){10-12}
 & None & BIC & Full & PSMR & KMR & DidPm & KMDidPm & KMPm & DidPv & KMDidPv & KMPv \\
\midrule
IQ Factor & -0.07 & -0.16 & 0.02 & -0.03 & -0.19 & -0.33 & -0.19 & -0.01 & -0.58 & -0.33 & 0.82 \\
\quad \textit{Unadjusted P-Value} & (0.71) & (0.44) & (0.93) & (0.88) & (0.69) & (0.11)* & (0.30) & (0.96) & (0.07)** & (0.19) & (0.04)*** \\
\quad \textit{Stepdown P-Value} & (0.96) & (0.99) & (0.99) & (0.96) & (0.99) & (0.61) & (0.99) & (0.97) & (0.37) & (0.90) & (0.46) \\
SDQ Composite - Child & -2.09 & -2.18 & -1.61 & -1.34 & -1.44 & -0.91 & -2.37 & -0.89 & -3.67 & -2.18 & 0.33 \\
\quad \textit{Unadjusted P-Value} & (0.00)*** & (0.01)*** & (0.02)*** & (0.09)** & (0.32) & (0.37) & (0.06)** & (0.23) & (0.00)*** & (0.06)** & (0.81) \\
\quad \textit{Stepdown P-Value} & (0.22) & (0.30) & (0.58) & (0.40) & (0.91) & (0.83) & (0.38) & (0.77) & (0.04)*** & (0.49) & (0.98) \\
SDQ Composite & -1.02 & -1.14 & -1.32 & -0.23 & -0.46 & 1.09 & -0.43 & -1.68 & -0.60 & -1.14 & -0.21 \\
\quad \textit{Unadjusted P-Value} & (0.31) & (0.27) & (0.24) & (0.90) & (0.85) & (0.41) & (0.86) & (0.08)** & (0.65) & (0.53) & (0.87) \\
\quad \textit{Stepdown P-Value} & (0.94) & (0.97) & (0.88) & (0.96) & (0.99) & (0.83) & (0.99) & (0.41) & (0.98) & (0.92) & (0.99) \\
Depression Score - positive & 0.09 & 0.70 & 0.36 & -1.05 & -1.04 & 2.40 & -0.28 & -1.11 & 2.85 & -1.40 & -0.74 \\
\quad \textit{Unadjusted P-Value} & (0.94) & (0.61) & (0.80) & (0.38) & (0.75) & (0.12)* & (0.88) & (0.30) & (0.10)** & (0.48) & (0.65) \\
\quad \textit{Stepdown P-Value} & (0.96) & (0.99) & (0.99) & (0.81) & (0.99) & (0.61) & (0.99) & (0.77) & (0.68) & (0.92) & (0.98) \\
Locus of Control - positive & -0.14 & -0.04 & -0.07 & -0.58 & -0.50 & -0.58 & -0.78 & 0.69 & -0.04 & -0.53 & 0.14 \\
\quad \textit{Unadjusted P-Value} & (0.33) & (0.79) & (0.70) & (0.03)*** & (0.17) & (0.01)*** & (0.00)*** & (0.00)*** & (0.83) & (0.07)** & (0.55) \\
\quad \textit{Stepdown P-Value} & (0.94) & (0.99) & (0.99) & (0.20) & (0.87) & (0.08)** & (0.26) & (0.00)*** & (0.98) & (0.67) & (0.98) \\
Not Obese & 0.03 & -0.05 & -0.01 & -0.09 & -0.10 & 0.18 & 0.02 & -0.11 & -0.11 & -0.09 & 0.21 \\
\quad \textit{Unadjusted P-Value} & (0.73) & (0.59) & (0.88) & (0.05)*** & (0.66) & (0.07)** & (0.78) & (0.04)*** & (0.40) & (0.26) & (0.16) \\
\quad \textit{Stepdown P-Value} & (0.96) & (0.99) & (0.99) & (0.29) & (0.91) & (0.47) & (0.99) & (0.28) & (0.95) & (0.90) & (0.89) \\
Not Overweight & 0.04 & -0.01 & 0.02 & -0.02 & -0.02 & 0.11 & 0.06 & 0.03 & -0.03 & -0.03 & -0.02 \\
\quad \textit{Unadjusted P-Value} & (0.44) & (0.83) & (0.74) & (0.39) & (0.86) & (0.03)*** & (0.10)* & (0.45) & (0.59) & (0.20) & (0.67) \\
\quad \textit{Stepdown P-Value} & (0.94) & (0.99) & (0.99) & (0.81) & (0.99) & (0.47) & (0.77) & (0.84) & (0.97) & (0.90) & (0.98) \\
Health is Good & 0.05 & 0.06 & 0.10 & 0.35 & 0.29 & -0.03 & 0.36 & 0.27 & 0.24 & 0.33 & -0.08 \\
\quad \textit{Unadjusted P-Value} & (0.60) & (0.55) & (0.37) & (0.15) & (0.25) & (0.83) & (0.07)** & (0.00)*** & (0.07)** & (0.08)** & (0.45) \\
\quad \textit{Stepdown P-Value} & (0.96) & (0.99) & (0.94) & (0.57) & (0.90) & (0.96) & (0.66) & (0.04)*** & (0.47) & (0.74) & (0.98) \\
Go To School & 0.05 & 0.02 & 0.04 & -0.01 & -0.02 & 0.07 & -0.01 & -0.03 & 0.02 & -0.00 & -0.02 \\
\quad \textit{Unadjusted P-Value} & (0.36) & (0.65) & (0.52) & (0.66) & (0.90) & (0.15) & (0.82) & (0.04)*** & (0.71) & (0.95) & (0.64) \\
\quad \textit{Stepdown P-Value} & (0.90) & (0.99) & (0.94) & (0.90) & (0.99) & (0.61) & (0.99) & (0.30) & (0.98) & (0.95) & (0.98) \\
How Much Child Likes School & -0.24 & -0.24 & -0.41 & 0.61 & 0.64 & 0.00 & 0.58 & -0.19 & -0.21 & 0.63 & -0.27 \\
\quad \textit{Unadjusted P-Value} & (0.21) & (0.27) & (0.04)*** & (0.26) & (0.18) & (1.00) & (0.36) & (0.26) & (0.43) & (0.29) & (0.31) \\
\quad \textit{Stepdown P-Value} & (0.92) & (0.97) & (0.53) & (0.72) & (0.87) & (0.97) & (0.93) & (0.77) & (0.97) & (0.90) & (0.98) \\
Days of Sport (Weekly) & -0.87 & -1.07 & -0.95 & -1.54 & -1.74 & -1.26 & -1.63 & 0.05 & -0.84 & -1.72 & -0.15 \\
\quad \textit{Unadjusted P-Value} & (0.02)*** & (0.01)*** & (0.03)*** & (0.00)*** & (0.06)** & (0.01)*** & (0.01)*** & (0.86) & (0.14)* & (0.01)*** & (0.81) \\
\quad \textit{Stepdown P-Value} & (0.33) & (0.23) & (0.33) & (0.04)*** & (0.56) & (0.15) & (0.16) & (0.97) & (0.71) & (0.18) & (0.99) \\
Num. of Friends & -3.56 & -5.14 & -4.30 & -8.72 & -9.23 & -6.27 & -11.65 & 3.44 & -2.35 & -9.69 & -3.96 \\
\quad \textit{Unadjusted P-Value} & (0.17) & (0.04)*** & (0.08)** & (0.07)** & (0.17) & (0.07)** & (0.00)*** & (0.04)*** & (0.49) & (0.03)*** & (0.28) \\
\quad \textit{Stepdown P-Value} & (0.49) & (0.22) & (0.40) & (0.35) & (0.87) & (0.30) & (0.17) & (0.28) & (0.97) & (0.42) & (0.98) \\
Volunteers & 0.09 & 0.05 & 0.12 & 0.25 & 0.24 & 0.01 & 0.22 & 0.18 & 0.01 & 0.23 & -0.01 \\
\quad \textit{Unadjusted P-Value} & (0.38) & (0.59) & (0.22) & (0.00)*** & (0.33) & (0.91) & (0.21) & (0.03)*** & (0.93) & (0.21) & (0.90) \\
\quad \textit{Stepdown P-Value} & (0.94) & (0.99) & (0.87) & (0.04)*** & (0.91) & (0.97) & (0.93) & (0.24) & (0.98) & (0.90) & (0.99) \\
Trust Score & 0.22 & -0.08 & 0.07 & -0.28 & -0.23 & 0.72 & 0.10 & -0.89 & -0.09 & -0.24 & -0.11 \\
\quad \textit{Unadjusted P-Value} & (0.47) & (0.81) & (0.82) & (0.62) & (0.77) & (0.10)** & (0.84) & (0.00)*** & (0.82) & (0.57) & (0.81) \\
\quad \textit{Stepdown P-Value} & (0.94) & (0.99) & (0.99) & (0.90) & (0.99) & (0.47) & (0.99) & (0.04)*** & (0.98) & (0.92) & (0.99) \\
\bottomrule
\end{tabular}
}
\vspace{1ex} \\
\footnotesize\raggedright{\underline{Note1 :} This table shows the estimates of the coefficient for attending Reggio Approach preschools from multiple methods. We compare Reggio Approach individuals with those who attended state preschools. Column title indicates the corresponding control set and and model. \textbf{None} = OLS estimate with no control variables. \textbf{BIC} = OLS estimate with controls selected by Bayesian Information Criterion (BIC) and additional controls for male indicator, migrant indicator, and ITC attendance indicator. \textbf{Full} = OLS estimate with the full set of controls. \textbf{PSMR} =  propensity score matching between Reggio Approach people and people in Reggio who attended state preschool. \textbf{KMR} = Epanechinikov kernel matching between Reggio Approach people and people in Reggio who attended state preschools. \textbf{DidPm} = difference-in-difference estimate of (Reggio Muni - Parma Muni) - (Reggio Reli - Parma Reli). \textbf{PSMPm} = propensity score matching between Reggio Approach people and people who attended Parma state preschools. \textbf{KMPm} = Epanechinikov kernel matching between Reggio Approach people and people who attended Parma state preschools. \textbf{DidPv} = difference-in-difference estimate of (Reggio Muni - Padova Muni) - (Reggio Reli - Padova Reli). \textbf{PSMPv} = propensity score matching between Reggio Approach people and people who attended Padova state preschools. \textbf{KMPv} = Epanechinikov kernel matching between Reggio Approach people and people who attended Padova state preschools.} 

\footnotesize\raggedright{\underline{Note2 :} Both unadjusted p-value and stepdown p-value are reported. ***, **, and * indicate significance of the coefficients at the 15\%, 10\%, and 5\% levels respectively.}

\end{table}


\begin{table}[H] \caption{Estimation Results for Main Outcomes, Comparison to Religious Preschools, Adult-30 Cohort} \label{ols-M-adult30-reg-reli}
\scalebox{0.59}{\begin{tabular}{l c c c c c c c c c c c}
\toprule
& \multicolumn{5}{c}{Within Reggio} & \multicolumn{3}{c}{With Parma} & \multicolumn{3}{c}{With Padova} \\\cmidrule(lr){2-6} \cmidrule(lr){7-9} \cmidrule(lr){10-12}
 & None & BIC & Full & PSMR & KMR & DidPm & KMDidPm & KMPm & DidPv & KMDidPv & KMPv \\
\midrule
IQ Factor & -0.42 & -0.41 & -0.36 & -0.46 & -0.45 & -0.85 & & -0.66 & -0.43 & & -0.70 \\
\quad \textit{Unadjusted P-Value} & (0.01)*** & (0.02)*** & (0.05)** & (0.01)*** & (0.02)*** & (0.00)*** & & (0.00)*** & (0.10)* & & (0.00)*** \\
\quad \textit{Stepdown P-Value} & (0.33) & (0.32) & (0.21) & (0.15) & (0.23) & (0.00)*** & & (0.00)*** & (0.52) & & (0.00)*** \\
Graduate from High School & -0.02 & -0.00 & -0.02 & -0.06 & 0.05 & 0.14 & & -0.10 & -0.05 & & 0.00 \\
\quad \textit{Unadjusted P-Value} & (0.77) & (0.98) & (0.76) & (0.61) & (0.54) & (0.13)* & & (0.06)** & (0.56) & & (0.97) \\
\quad \textit{Stepdown P-Value} & (0.99) & (0.99) & (0.94) & (0.98) & (0.99) & (0.85) & & (0.45) & (0.99) & & (0.97) \\
High School Grade & 3.11 & 2.44 & 2.37 & 2.91 & 2.00 & 4.77 & & 7.53 & -0.21 & & 7.06 \\
\quad \textit{Unadjusted P-Value} & (0.06)** & (0.15) & (0.20) & (0.04)*** & (0.30) & (0.27) & & (0.00)*** & (0.96) & & (0.00)*** \\
\quad \textit{Stepdown P-Value} & (0.63) & (0.88) & (0.48) & (0.34) & (0.97) & (0.85) & & (0.06)** & (0.99) & & (0.00)*** \\
High School Grade (Standardized) & 4.38 & 3.29 & 4.05 & 4.15 & 3.06 & 5.95 & & 2.20 & 2.20 & & 3.37 \\
\quad \textit{Unadjusted P-Value} & (0.05)** & (0.17) & (0.07)** & (0.01)*** & (0.26) & (0.08)** & & (0.20) & (0.64) & & (0.07)** \\
\quad \textit{Stepdown P-Value} & (0.49) & (0.88) & (0.30) & (0.09)** & (0.97) & (0.71) & & (0.82) & (0.99) & & (0.39) \\
Max Edu: University & 0.09 & 0.08 & 0.04 & 0.07 & 0.06 & 0.18 & & -0.27 & 0.26 & & -0.27 \\
\quad \textit{Unadjusted P-Value} & (0.18) & (0.24) & (0.66) & (0.32) & (0.44) & (0.22) & & (0.01)*** & (0.07)** & & (0.00)*** \\
\quad \textit{Stepdown P-Value} & (0.81) & (0.96) & (0.87) & (0.92) & (0.98) & (0.85) & & (0.13) & (0.58) & & (0.00)*** \\
Employed & -0.06 & -0.06 & -0.04 & -0.06 & -0.07 & 0.04 & & -0.03 & -0.07 & & 0.04 \\
\quad \textit{Unadjusted P-Value} & (0.01)*** & (0.02)*** & (0.11)* & (0.01)*** & (0.01)*** & (0.68) & & (0.53) & (0.43) & & (0.32) \\
\quad \textit{Stepdown P-Value} & (0.69) & (0.88) & (0.71) & (0.15) & (0.16) & (0.99) & & (0.89) & (0.99) & & (0.94) \\
Hours Worked Per Week & -2.09 & -2.16 & -2.38 & -2.25 & -2.66 & -1.11 & & 2.96 & -0.87 & & 0.40 \\
\quad \textit{Unadjusted P-Value} & (0.20) & (0.21) & (0.17) & (0.21) & (0.14)* & (0.80) & & (0.44) & (0.85) & & (0.86) \\
\quad \textit{Stepdown P-Value} & (0.92) & (0.96) & (0.63) & (0.86) & (0.84) & (0.99) & & (0.89) & (0.99) & & (0.97) \\
Married or Cohabitating & 0.01 & 0.04 & 0.06 & -0.13 & -0.04 & -0.01 & & -0.13 & 0.07 & & -0.13 \\
\quad \textit{Unadjusted P-Value} & (0.89) & (0.71) & (0.56) & (0.33) & (0.74) & (0.96) & & (0.22) & (0.64) & & (0.06)** \\
\quad \textit{Stepdown P-Value} & (0.99) & (0.99) & (0.77) & (0.92) & (0.99) & (0.99) & & (0.82) & (0.99) & & (0.39) \\
Not Obese & -0.12 & -0.10 & -0.06 & -0.11 & -0.14 & -0.24 & & -0.06 & -0.22 & & -0.11 \\
\quad \textit{Unadjusted P-Value} & (0.12)* & (0.18) & (0.47) & (0.16) & (0.13)* & (0.07)** & & (0.48) & (0.11)* & & (0.06)** \\
\quad \textit{Stepdown P-Value} & (0.70) & (0.88) & (0.71) & (0.79) & (0.84) & (0.54) & & (0.89) & (0.58) & & (0.39) \\
Not Overweight & -0.01 & 0.01 & -0.02 & -0.02 & -0.02 & 0.18 & & -0.07 & -0.08 & & 0.05 \\
\quad \textit{Unadjusted P-Value} & (0.90) & (0.95) & (0.78) & (0.81) & (0.87) & (0.13)* & & (0.34) & (0.52) & & (0.41) \\
\quad \textit{Stepdown P-Value} & (0.99) & (0.99) & (0.96) & (0.99) & (0.99) & (0.85) & & (0.89) & (0.99) & & (0.96) \\
Locus of Control - positive & -0.11 & -0.04 & -0.12 & 0.08 & 0.05 & 0.37 & & -0.14 & 0.01 & & -0.33 \\
\quad \textit{Unadjusted P-Value} & (0.50) & (0.78) & (0.46) & (0.64) & (0.79) & (0.16) & & (0.41) & (0.98) & & (0.00)*** \\
\quad \textit{Stepdown P-Value} & (0.96) & (0.99) & (0.70) & (0.98) & (0.99) & (0.84) & & (0.89) & (0.99) & & (0.02)*** \\
Depression Score - positive & -1.83 & -1.01 & -1.04 & -1.57 & -0.43 & -1.35 & & -1.03 & -1.99 & & -2.57 \\
\quad \textit{Unadjusted P-Value} & (0.09)** & (0.23) & (0.26) & (0.30) & (0.75) & (0.38) & & (0.32) & (0.30) & & (0.00)*** \\
\quad \textit{Stepdown P-Value} & (0.69) & (0.96) & (0.56) & (0.92) & (0.99) & (0.97) & & (0.88) & (0.98) & & (0.00)*** \\
Volunteers & 0.11 & 0.10 & 0.13 & 0.11 & 0.10 & 0.03 & & -0.21 & -0.01 & & -0.12 \\
\quad \textit{Unadjusted P-Value} & (0.00)*** & (0.00)*** & (0.00)*** & (0.00)*** & (0.00)*** & (0.78) & & (0.02)*** & (0.96) & & (0.02)*** \\
\quad \textit{Stepdown P-Value} & (0.49) & (0.65) & (0.11) & (0.00)*** & (0.05)*** & (0.99) & & (0.21) & (0.99) & & (0.20) \\
Ever Voted for Municipal & -0.16 & -0.03 & -0.04 & 0.03 & 0.04 & -0.04 & & 0.18 & 0.14 & & -0.03 \\
\quad \textit{Unadjusted P-Value} & (0.09)** & (0.66) & (0.67) & (0.65) & (0.76) & (0.75) & & (0.09)** & (0.29) & & (0.64) \\
\quad \textit{Stepdown P-Value} & (0.69) & (0.99) & (0.82) & (0.98) & (0.99) & (0.99) & & (0.56) & (0.99) & & (0.97) \\
Ever Voted for Regional & -0.16 & -0.04 & -0.06 & 0.00 & 0.01 & -0.04 & & 0.25 & 0.24 & & -0.05 \\
\quad \textit{Unadjusted P-Value} & (0.09)** & (0.59) & (0.51) & (0.98) & (0.91) & (0.74) & & (0.02)*** & (0.06)** & & (0.53) \\
\quad \textit{Stepdown P-Value} & (0.69) & (0.99) & (0.75) & (0.99) & (0.99) & (0.99) & & (0.21) & (0.82) & & (0.96) \\
Num. of Friends & -0.82 & -0.90 & -0.03 & 0.03 & -1.18 & 6.06 & & -6.92 & 0.06 & & -0.87 \\
\quad \textit{Unadjusted P-Value} & (0.47) & (0.56) & (0.99) & (0.99) & (0.39) & (0.02)*** & & (0.01)*** & (0.98) & & (0.39) \\
\quad \textit{Stepdown P-Value} & (0.96) & (0.98) & (0.96) & (0.99) & (0.98) & (0.20) & & (0.16) & (0.99) & & (0.96) \\
Trust Score & 0.43 & 0.31 & 0.51 & 0.34 & 0.39 & 0.38 & & 0.58 & 0.40 & & 0.84 \\
\quad \textit{Unadjusted P-Value} & (0.12)* & (0.29) & (0.12)* & (0.15)* & (0.26) & (0.46) & & (0.05)*** & (0.35) & & (0.00)*** \\
\quad \textit{Stepdown P-Value} & (0.69) & (0.96) & (0.27) & (0.79) & (0.97) & (0.98) & & (0.40) & (0.99) & & (0.00)*** \\
\bottomrule
\end{tabular}
}
\vspace{1ex} \\
\footnotesize\raggedright{\underline{Note1 :} This table shows the estimates of the coefficient for attending Reggio Approach preschools from multiple methods. We compare Reggio Approach individuals with those who attended religious preschools. Column title indicates the corresponding control set and and model. \textbf{None} = OLS estimate with no control variables. \textbf{BIC} = OLS estimate with controls selected by Bayesian Information Criterion (BIC) and additional controls for male indicator, migrant indicator, and ITC attendance indicator. \textbf{Full} = OLS estimate with the full set of controls. \textbf{PSMR} =  propensity score matching between Reggio Approach people and people in Reggio who attended other types of preschool. \textbf{KMR} = Epanechinikov kernel matching between Reggio Approach people and people in Reggio who attended religious preschools. \textbf{DidPm} = difference-in-difference estimate of (Reggio Muni - Parma Muni) - (Reggio Reli - Parma Reli). \textbf{PSMPm} = propensity score matching between Reggio Approach people and people who attended Parma religious preschools. \textbf{KMPm} = Epanechinikov kernel matching between Reggio Approach people and people who attended Parma religious preschools. \textbf{DidPv} = difference-in-difference estimate of (Reggio Muni - Padova Muni) - (Reggio Reli - Padova Reli). \textbf{PSMPv} = propensity score matching between Reggio Approach people and people who attended Padova religious preschools. \textbf{KMPv} = Epanechinikov kernel matching between Reggio Approach people and people who attended Padova religious preschools.} 

\footnotesize\raggedright{\underline{Note2 :} Both unadjusted p-value and stepdown p-value are reported. ***, **, and * indicate significance of the coefficients at the 15\%, 10\%, and 5\% levels respectively.}

\end{table}

\begin{table}[H] \caption{Estimation Results for Main Outcomes, Comparison to State Preschools, Age-30 Cohort} \label{ols-M-adult30-reg-stat}
\scalebox{0.59}{\begin{tabular}{l c c c c c c c c c c c}
\toprule
& \multicolumn{5}{c}{Within Reggio} & \multicolumn{3}{c}{With Parma} & \multicolumn{3}{c}{With Padova} \\\cmidrule(lr){2-6} \cmidrule(lr){7-9} \cmidrule(lr){10-12}
 & None & BIC & Full & PSMR & KMR & DidPm & KMDidPm & KMPm & DidPv & KMDidPv & KMPv \\
\midrule
IQ Factor & 0.64 & 0.45 & 0.44 & 0.29 & 0.43 & 0.15 & & -0.49 & 0.33 & & -0.71 \\
\quad \textit{Unadjusted P-Value} & (0.01)*** & (0.02)*** & (0.02)*** & (0.11)* & (0.12)* & (0.57) & & (0.00)*** & (0.39) & & (0.02)*** \\
\quad \textit{Stepdown P-Value} & (0.12) & (0.40) & (0.17) & (0.67) & (0.82) & (0.99) & & (0.02)*** & (0.99) & & (0.30) \\
Graduate from High School & -0.14 & -0.10 & -0.11 & -0.12 & -0.14 & 0.00 & & -0.02 & -0.15 & & -0.08 \\
\quad \textit{Unadjusted P-Value} & (0.00)*** & (0.01)*** & (0.03)*** & (0.00)*** & (0.00)*** & (1.00) & & (0.74) & (0.06)** & & (0.51) \\
\quad \textit{Stepdown P-Value} & (0.51) & (0.86) & (0.33) & (0.01)*** & (0.01)*** & (0.99) & & (0.99) & (0.94) & & (0.93) \\
High School Grade & -2.59 & -2.08 & -1.64 & 1.64 & 1.98 & -0.22 & & 4.09 & -0.59 & & -4.63 \\
\quad \textit{Unadjusted P-Value} & (0.28) & (0.36) & (0.51) & (0.39) & (0.48) & (0.96) & & (0.31) & (0.92) & & (0.43) \\
\quad \textit{Stepdown P-Value} & (0.85) & (0.95) & (0.71) & (0.97) & (0.98) & (0.99) & & (0.93) & (0.99) & & (0.93) \\
High School Grade (Standardized) & -0.23 & -0.04 & -1.14 & 4.54 & 4.15 & 1.49 & & -0.56 & -1.09 & & -2.74 \\
\quad \textit{Unadjusted P-Value} & (0.93) & (0.99) & (0.68) & (0.02)*** & (0.17) & (0.69) & & (0.86) & (0.85) & & (0.56) \\
\quad \textit{Stepdown P-Value} & (0.98) & (0.99) & (0.85) & (0.24) & (0.87) & (0.99) & & (0.99) & (0.99) & & (0.93) \\
Max Edu: University & -0.10 & -0.07 & -0.02 & -0.00 & -0.09 & -0.10 & & -0.15 & 0.14 & & -0.21 \\
\quad \textit{Unadjusted P-Value} & (0.36) & (0.49) & (0.84) & (0.98) & (0.46) & (0.50) & & (0.10)* & (0.51) & & (0.31) \\
\quad \textit{Stepdown P-Value} & (0.89) & (0.97) & (0.95) & (0.99) & (0.98) & (0.99) & & (0.68) & (0.99) & & (0.91) \\
Employed & 0.02 & 0.01 & -0.02 & 0.01 & 0.05 & 0.17 & & -0.05 & 0.11 & & -0.06 \\
\quad \textit{Unadjusted P-Value} & (0.71) & (0.93) & (0.73) & (0.94) & (0.53) & (0.07)** & & (0.14)* & (0.31) & & (0.01)*** \\
\quad \textit{Stepdown P-Value} & (0.96) & (0.99) & (0.87) & (0.99) & (0.99) & (0.64) & & (0.76) & (0.98) & & (0.22) \\
Hours Worked Per Week & 5.65 & 6.06 & 5.20 & 11.40 & 9.38 & 12.80 & & -3.59 & 11.06 & & -3.08 \\
\quad \textit{Unadjusted P-Value} & (0.21) & (0.15)* & (0.21) & (0.01)*** & (0.08)** & (0.03)*** & & (0.26) & (0.05)** & & (0.24) \\
\quad \textit{Stepdown P-Value} & (0.76) & (0.85) & (0.43) & (0.18) & (0.66) & (0.27) & & (0.91) & (0.59) & & (0.89) \\
Married or Cohabitating & 0.16 & 0.07 & 0.02 & 0.02 & 0.07 & 0.18 & & -0.08 & 0.25 & & -0.17 \\
\quad \textit{Unadjusted P-Value} & (0.08)** & (0.43) & (0.82) & (0.86) & (0.54) & (0.20) & & (0.38) & (0.24) & & (0.38) \\
\quad \textit{Stepdown P-Value} & (0.71) & (0.97) & (0.93) & (0.99) & (0.99) & (0.90) & & (0.94) & (0.94) & & (0.93) \\
Not Obese & 0.21 & 0.11 & 0.10 & 0.02 & 0.06 & 0.12 & & -0.21 & 0.09 & & -0.25 \\
\quad \textit{Unadjusted P-Value} & (0.06)** & (0.13)* & (0.16) & (0.73) & (0.64) & (0.29) & & (0.00)*** & (0.63) & & (0.01)*** \\
\quad \textit{Stepdown P-Value} & (0.41) & (0.86) & (0.44) & (0.99) & (0.99) & (0.95) & & (0.02)*** & (0.99) & & (0.22) \\
Not Overweight & -0.12 & -0.02 & 0.01 & -0.05 & -0.12 & 0.03 & & 0.04 & 0.08 & & -0.16 \\
\quad \textit{Unadjusted P-Value} & (0.16) & (0.81) & (0.88) & (0.64) & (0.21) & (0.79) & & (0.65) & (0.54) & & (0.19) \\
\quad \textit{Stepdown P-Value} & (0.85) & (0.99) & (0.97) & (0.99) & (0.90) & (0.99) & & (0.99) & (0.99) & & (0.86) \\
Locus of Control - positive & 0.40 & 0.22 & 0.22 & 0.28 & 0.26 & 0.32 & & 0.24 & -0.20 & & 0.13 \\
\quad \textit{Unadjusted P-Value} & (0.01)*** & (0.12)* & (0.16) & (0.06)** & (0.14)* & (0.21) & & (0.23) & (0.53) & & (0.65) \\
\quad \textit{Stepdown P-Value} & (0.26) & (0.86) & (0.43) & (0.51) & (0.82) & (0.90) & & (0.90) & (0.99) & & (0.93) \\
Depression Score - positive & 3.04 & 1.11 & 1.34 & -0.39 & 0.29 & 3.00 & & -2.77 & -0.58 & & -2.04 \\
\quad \textit{Unadjusted P-Value} & (0.06)** & (0.25) & (0.19) & (0.64) & (0.88) & (0.09)** & & (0.02)*** & (0.81) & & (0.39) \\
\quad \textit{Stepdown P-Value} & (0.33) & (0.94) & (0.45) & (0.99) & (0.99) & (0.67) & & (0.27) & (0.99) & & (0.93) \\
Volunteers & 0.11 & 0.10 & 0.09 & 0.10 & 0.11 & -0.11 & & -0.04 & -0.06 & & -0.22 \\
\quad \textit{Unadjusted P-Value} & (0.00)*** & (0.00)*** & (0.04)*** & (0.00)*** & (0.00)*** & (0.26) & & (0.53) & (0.70) & & (0.17) \\
\quad \textit{Stepdown P-Value} & (0.65) & (0.86) & (0.40) & (0.02)*** & (0.02)*** & (0.95) & & (0.97) & (0.99) & & (0.83) \\
Ever Voted for Municipal & 0.07 & -0.01 & 0.04 & -0.08 & -0.05 & -0.09 & & 0.22 & 0.44 & & -0.24 \\
\quad \textit{Unadjusted P-Value} & (0.54) & (0.93) & (0.64) & (0.37) & (0.70) & (0.48) & & (0.01)*** & (0.01)*** & & (0.21) \\
\quad \textit{Stepdown P-Value} & (0.96) & (0.99) & (0.80) & (0.97) & (0.99) & (0.99) & & (0.21) & (0.20) & & (0.89) \\
Ever Voted for Regional & -0.02 & -0.10 & -0.04 & -0.15 & -0.14 & -0.10 & & 0.24 & 0.48 & & -0.24 \\
\quad \textit{Unadjusted P-Value} & (0.88) & (0.33) & (0.72) & (0.16) & (0.30) & (0.46) & & (0.01)*** & (0.01)*** & & (0.21) \\
\quad \textit{Stepdown P-Value} & (0.98) & (0.94) & (0.87) & (0.73) & (0.94) & (0.99) & & (0.14) & (0.19) & & (0.89) \\
Num. of Friends & 2.84 & 2.85 & 2.05 & 1.75 & 1.93 & 4.37 & & -1.47 & 2.05 & & 3.43 \\
\quad \textit{Unadjusted P-Value} & (0.00)*** & (0.01)*** & (0.07)** & (0.10)** & (0.07)** & (0.02)*** & & (0.38) & (0.34) & & (0.12)* \\
\quad \textit{Stepdown P-Value} & (0.62) & (0.77) & (0.49) & (0.67) & (0.63) & (0.55) & & (0.94) & (0.99) & & (0.76) \\
Trust Score & 0.28 & 0.28 & 0.13 & 0.14 & 0.18 & 0.58 & & -0.03 & 1.30 & & -0.90 \\
\quad \textit{Unadjusted P-Value} & (0.26) & (0.27) & (0.61) & (0.63) & (0.52) & (0.21) & & (0.94) & (0.01)*** & & (0.16) \\
\quad \textit{Stepdown P-Value} & (0.89) & (0.95) & (0.83) & (0.99) & (0.99) & (0.90) & & (0.99) & (0.22) & & (0.83) \\
\bottomrule
\end{tabular}
}
\vspace{1ex} \\
\footnotesize\raggedright{\underline{Note1 :} This table shows the estimates of the coefficient for attending Reggio Approach preschools from multiple methods. We compare Reggio Approach individuals with those who attended state preschools. Column title indicates the corresponding control set and and model. \textbf{None} = OLS estimate with no control variables. \textbf{BIC} = OLS estimate with controls selected by Bayesian Information Criterion (BIC) and additional controls for male indicator, migrant indicator, and ITC attendance indicator. \textbf{Full} = OLS estimate with the full set of controls. \textbf{PSMR} =  propensity score matching between Reggio Approach people and people in Reggio who attended state preschool. \textbf{KMR} = Epanechinikov kernel matching between Reggio Approach people and people in Reggio who attended state preschools. \textbf{DidPm} = difference-in-difference estimate of (Reggio Muni - Parma Muni) - (Reggio Reli - Parma Reli). \textbf{PSMPm} = propensity score matching between Reggio Approach people and people who attended Parma state preschools. \textbf{KMPm} = Epanechinikov kernel matching between Reggio Approach people and people who attended Parma state preschools. \textbf{DidPv} = difference-in-difference estimate of (Reggio Muni - Padova Muni) - (Reggio Reli - Padova Reli). \textbf{PSMPv} = propensity score matching between Reggio Approach people and people who attended Padova state preschools. \textbf{KMPv} = Epanechinikov kernel matching between Reggio Approach people and people who attended Padova state preschools.} 

\footnotesize\raggedright{\underline{Note2 :} Both unadjusted p-value and stepdown p-value are reported. ***, **, and * indicate significance of the coefficients at the 15\%, 10\%, and 5\% levels respectively.}

\end{table}


\begin{table}[H] \caption{Estimation Results for Main Outcomes, Comparison to Religious Preschools, Adult-40 Cohort} \label{ols-M-adult40-reg-reli}
\scalebox{0.59}{\begin{tabular}{l c c c c c c c}
\toprule
& \multicolumn{5}{c}{Within Reggio} & With Parma & With Padova \\\cmidrule(lr){2-6} \cmidrule(lr){7-7} \cmidrule(lr){8-8}
 & None & BIC & Full & PSMR & KMR & KMPm & KMPv \\
\midrule
IQ Factor & -0.27 & -0.23 & -0.23 & -0.28 & -0.30 & -0.41 & -0.35 \\
\quad \textit{Unadjusted P-Value} & (0.04)*** & (0.06)** & (0.08)** & (0.03)*** & (0.04)*** & (0.00)*** & (0.00)*** \\
\quad \textit{Stepdown P-Value} & (0.64) & (0.90) & (0.31) & (0.38) & (0.47) & (0.01)*** & (0.04)*** \\
Graduate from High School & 0.10 & 0.08 & 0.10 & 0.16 & 0.13 & -0.05 & 0.07 \\
\quad \textit{Unadjusted P-Value} & (0.20) & (0.30) & (0.20) & (0.12)* & (0.17) & (0.37) & (0.29) \\
\quad \textit{Stepdown P-Value} & (0.94) & (0.98) & (0.36) & (0.81) & (0.95) & (0.96) & (0.92) \\
High School Grade & 0.46 & 1.26 & 1.46 & 3.02 & 3.23 & 2.43 & 8.11 \\
\quad \textit{Unadjusted P-Value} & (0.80) & (0.50) & (0.46) & (0.25) & (0.13)* & (0.29) & (0.00)*** \\
\quad \textit{Stepdown P-Value} & (0.99) & (0.99) & (0.70) & (0.95) & (0.89) & (0.96) & (0.01)*** \\
High School Grade (Standardized) & -0.14 & 1.19 & 1.53 & 3.02 & 3.75 & -2.61 & 4.05 \\
\quad \textit{Unadjusted P-Value} & (0.96) & (0.66) & (0.59) & (0.43) & (0.23) & (0.19) & (0.09)** \\
\quad \textit{Stepdown P-Value} & (0.99) & (0.99) & (0.75) & (0.98) & (0.96) & (0.92) & (0.65) \\
Max Edu: University & 0.07 & 0.06 & 0.04 & 0.04 & 0.01 & -0.10 & -0.14 \\
\quad \textit{Unadjusted P-Value} & (0.26) & (0.36) & (0.52) & (0.51) & (0.93) & (0.24) & (0.07)** \\
\quad \textit{Stepdown P-Value} & (0.98) & (0.98) & (0.75) & (0.98) & (0.99) & (0.96) & (0.61) \\
Employed & -0.01 & -0.01 & -0.00 & -0.01 & 0.03 & -0.01 & 0.09 \\
\quad \textit{Unadjusted P-Value} & (0.74) & (0.82) & (0.94) & (0.80) & (0.32) & (0.83) & (0.07)** \\
\quad \textit{Stepdown P-Value} & (0.99) & (0.99) & (0.96) & (0.99) & (0.98) & (0.99) & (0.61) \\
Hours Worked Per Week & -2.01 & -2.58 & -2.54 & -2.28 & -0.64 & 0.27 & 3.61 \\
\quad \textit{Unadjusted P-Value} & (0.30) & (0.28) & (0.28) & (0.47) & (0.76) & (0.90) & (0.21) \\
\quad \textit{Stepdown P-Value} & (0.99) & (0.97) & (0.49) & (0.98) & (0.99) & (0.99) & (0.89) \\
Married or Cohabitating & 0.02 & 0.03 & 0.03 & -0.01 & -0.05 & 0.06 & 0.11 \\
\quad \textit{Unadjusted P-Value} & (0.78) & (0.72) & (0.67) & (0.88) & (0.61) & (0.47) & (0.16) \\
\quad \textit{Stepdown P-Value} & (0.99) & (0.99) & (0.86) & (0.99) & (0.99) & (0.96) & (0.86) \\
Not Obese & -0.11 & -0.07 & -0.04 & -0.10 & -0.09 & -0.10 & -0.18 \\
\quad \textit{Unadjusted P-Value} & (0.14)* & (0.33) & (0.58) & (0.35) & (0.32) & (0.19) & (0.02)*** \\
\quad \textit{Stepdown P-Value} & (0.94) & (0.98) & (0.78) & (0.98) & (0.98) & (0.92) & (0.24) \\
Not Overweight & 0.08 & 0.07 & 0.05 & 0.12 & 0.07 & 0.13 & 0.08 \\
\quad \textit{Unadjusted P-Value} & (0.35) & (0.39) & (0.56) & (0.27) & (0.52) & (0.13)* & (0.29) \\
\quad \textit{Stepdown P-Value} & (0.99) & (0.98) & (0.75) & (0.96) & (0.99) & (0.84) & (0.92) \\
Locus of Control - positive & -0.01 & 0.02 & -0.05 & -0.05 & -0.07 & 0.18 & 0.05 \\
\quad \textit{Unadjusted P-Value} & (0.95) & (0.92) & (0.75) & (0.72) & (0.67) & (0.27) & (0.75) \\
\quad \textit{Stepdown P-Value} & (0.99) & (0.99) & (0.91) & (0.99) & (0.99) & (0.96) & (0.98) \\
Depression Score - positive & 0.22 & 1.12 & 0.86 & 1.65 & 0.78 & -1.02 & 0.03 \\
\quad \textit{Unadjusted P-Value} & (0.81) & (0.20) & (0.38) & (0.10)** & (0.47) & (0.31) & (0.97) \\
\quad \textit{Stepdown P-Value} & (0.99) & (0.98) & (0.65) & (0.77) & (0.99) & (0.96) & (0.98) \\
Volunteers & 0.04 & 0.00 & 0.02 & -0.07 & -0.07 & -0.14 & -0.08 \\
\quad \textit{Unadjusted P-Value} & (0.42) & (0.96) & (0.69) & (0.21) & (0.26) & (0.07)** & (0.23) \\
\quad \textit{Stepdown P-Value} & (0.99) & (0.99) & (0.88) & (0.94) & (0.98) & (0.61) & (0.89) \\
Ever Voted for Municipal & -0.02 & 0.12 & 0.10 & 0.14 & 0.10 & 0.04 & 0.05 \\
\quad \textit{Unadjusted P-Value} & (0.80) & (0.13)* & (0.22) & (0.10)* & (0.37) & (0.65) & (0.60) \\
\quad \textit{Stepdown P-Value} & (0.99) & (0.90) & (0.41) & (0.77) & (0.98) & (0.97) & (0.98) \\
Ever Voted for Regional & 0.01 & 0.15 & 0.12 & 0.17 & 0.13 & 0.20 & 0.05 \\
\quad \textit{Unadjusted P-Value} & (0.90) & (0.05)** & (0.14)* & (0.04)*** & (0.22) & (0.04)*** & (0.56) \\
\quad \textit{Stepdown P-Value} & (0.99) & (0.62) & (0.30) & (0.47) & (0.96) & (0.45) & (0.98) \\
Num. of Friends & 0.62 & 0.13 & 0.14 & 0.08 & 0.40 & 0.13 & -0.20 \\
\quad \textit{Unadjusted P-Value} & (0.53) & (0.90) & (0.88) & (0.93) & (0.69) & (0.91) & (0.87) \\
\quad \textit{Stepdown P-Value} & (0.99) & (0.99) & (0.96) & (0.99) & (0.99) & (0.99) & (0.98) \\
Trust Score & 0.07 & -0.06 & 0.12 & -0.24 & -0.19 & -0.22 & 0.16 \\
\quad \textit{Unadjusted P-Value} & (0.81) & (0.83) & (0.67) & (0.37) & (0.56) & (0.43) & (0.50) \\
\quad \textit{Stepdown P-Value} & (0.99) & (0.99) & (0.85) & (0.98) & (0.99) & (0.96) & (0.98) \\
\bottomrule
\end{tabular}
}
\vspace{1ex} \\
\footnotesize\raggedright{\underline{Note1 :} This table shows the estimates of the coefficient for attending Reggio Approach preschools from multiple methods. We compare Reggio Approach individuals with those who attended religious preschools. Column title indicates the corresponding control set and and model. \textbf{None} = OLS estimate with no control variables. \textbf{BIC} = OLS estimate with controls selected by Bayesian Information Criterion (BIC) and additional controls for male indicator, migrant indicator, and ITC attendance indicator. \textbf{Full} = OLS estimate with the full set of controls. \textbf{PSMR} =  propensity score matching between Reggio Approach people and people in Reggio who attended other types of preschool. \textbf{KMR} = Epanechinikov kernel matching between Reggio Approach people and people in Reggio who attended religious preschools. \textbf{DidPm} = difference-in-difference estimate of (Reggio Muni - Parma Other) - (Reggio Reli - Parma Reli). \textbf{PSMPm} = propensity score matching between Reggio Approach people and people who attended Parma religious preschools. \textbf{KMPm} = Epanechinikov kernel matching between Reggio Approach people and people who attended Parma religious preschools. \textbf{DidPv} = difference-in-difference estimate of (Reggio Muni - Padova Other) - (Reggio Reli - Padova Reli). \textbf{PSMPv} = propensity score matching between Reggio Approach people and people who attended Padova religious preschools. \textbf{KMPv} = Epanechinikov kernel matching between Reggio Approach people and people who attended Padova religious preschools.} 

\footnotesize\raggedright{\underline{Note2 :} Both unadjusted p-value and stepdown p-value are reported. ***, **, and * indicate significance of the coefficients at the 15\%, 10\%, and 5\% levels respectively.}

\end{table}





\subsection{Estimation Results for Reggio Approach Preschools, Comparison with Municipal-Affiliated} \label{app:comparison-reli-stat}

\begin{table}[H] \caption{Estimation Results for Main Outcomes, Comparison to Municipal-Affiliated Preschools, Child Cohort} \label{ols-M-child-reg-affi}
\scalebox{0.7}{\begin{tabular}{l c c c}
\toprule
 & OLSR & OLSPm & OLSPv \\
\midrule
IQ Factor & -0.19 & -0.12 & 0.16 \\
\quad \textit{Unadjusted P-Value} & (0.56) & (0.41) & (0.57) \\
\quad \textit{Stepdown P-Value} & (0.93) & (0.99) & (0.73) \\
SDQ Composite - Child & 3.31 & 0.42 & 1.69 \\
\quad \textit{Unadjusted P-Value} & (0.28) & (0.56) & (0.29) \\
\quad \textit{Stepdown P-Value} & (0.55) & (0.99) & (0.43) \\
Not Obese & -0.12 & -0.13 & -0.18 \\
\quad \textit{Unadjusted P-Value} & (0.43) & (0.03)*** & (0.06)** \\
\quad \textit{Stepdown P-Value} & (0.93) & (0.42) & (0.43) \\
Not Overweight & -0.03 & 0.04 & -0.14 \\
\quad \textit{Unadjusted P-Value} & (0.87) & (0.57) & (0.00)*** \\
\quad \textit{Stepdown P-Value} & (0.99) & (0.99) & (0.43) \\
Health is Good & 0.00 & -0.02 & -0.19 \\
\quad \textit{Unadjusted P-Value} & (0.99) & (0.84) & (0.04)*** \\
\quad \textit{Stepdown P-Value} & (0.99) & (0.99) & (0.43) \\
Not Excited to Learn & 0.02 & 0.01 & -0.07 \\
\quad \textit{Unadjusted P-Value} & (0.37) & (0.54) & (0.29) \\
\quad \textit{Stepdown P-Value} & (0.99) & (0.99) & (0.43) \\
Problems Sitting Still & 0.15 & 0.04 & -0.20 \\
\quad \textit{Unadjusted P-Value} & (0.00)*** & (0.44) & (0.07)** \\
\quad \textit{Stepdown P-Value} & (0.93) & (0.99) & (0.15) \\
How Much Child Likes School & 0.03 & -0.05 & 0.60 \\
\quad \textit{Unadjusted P-Value} & (0.85) & (0.55) & (0.00)*** \\
\quad \textit{Stepdown P-Value} & (0.99) & (0.99) & (0.01)*** \\
Num. of Friends & -0.34 & 0.26 & -1.83 \\
\quad \textit{Unadjusted P-Value} & (0.70) & (0.54) & (0.06)** \\
\quad \textit{Stepdown P-Value} & (0.99) & (0.99) & (0.03)*** \\
Candy Game: Willing to Share Candies & 0.05 & -0.01 & 0.01 \\
\quad \textit{Unadjusted P-Value} & (0.73) & (0.89) & (0.87) \\
\quad \textit{Stepdown P-Value} & (0.93) & (0.99) & (0.85) \\
\bottomrule
\end{tabular}
}
\vspace{1ex} \\
\footnotesize\raggedright{\underline{Note1 :} This table shows the estimates of the coefficient for attending Reggio Approach preschools from multiple methods. We compare Reggio Approach individuals with those who attended municipal-affiliated preschools. Column title indicates the corresponding control set and and model. \textbf{OLSR} = OLS comparison between Reggio people who attended Reggio Approach preschool and people in Reggio who attended municipal-affiliated preschool. \textbf{OLSPm} = OLS comparison between Reggio people who attended Reggio Approach preschool and people in Parma who attended municipal-affiliated preschool. \textbf{OLSPv} = OLS comparison between Reggio people who attended Reggio Approach preschool and people in Padova who attended municipal-affiliated preschool.} 

\footnotesize\raggedright{\underline{Note2 :} Both unadjusted p-value and stepdown p-value are reported. ***, **, and * indicate significance of the coefficients at the 15\%, 10\%, and 5\% levels respectively.}

\end{table}


% ============================================================================================== %
% EXTENDED OUTCOMES

\subsection{Estimation Results for Reggio Approach Preschools, Extended Outcomes}  \label{appsec:extended-outcome}
\subsubsection{Child Cohort}
\begin{table}[H] \caption{Estimation Results for Cognitive and Noncognitive Outcomes, Comparison to Non-RA Preschools, Child Cohort} \label{combined_child_CN_Other}
\scalebox{0.59}{\begin{tabular}{l c c c c c c c}
\toprule
 & None & BIC & Full & PSM & DidPm & DidPv \\
\midrule
IQ Factor & -0.05 & -0.09 & -0.09 & -0.06 & 0.05 & 0.09 \\
\quad \textit{Unadjusted P-Value} & (0.62) & (0.38) & (0.38) & (0.54) & (0.68) & (0.55) \\
\quad \textit{Stepdown P-Value} & (0.93) & (0.82) & (0.72) & (0.92) & (0.99) & (0.93) \\
IQ Score & -0.02 & -0.02 & -0.02 & -0.02 & 0.01 & 0.01 \\
\quad \textit{Unadjusted P-Value} & (0.46) & (0.37) & (0.35) & (0.43) & (0.75) & (0.83) \\
\quad \textit{Stepdown P-Value} & (0.91) & (0.82) & (0.72) & (0.90) & (0.99) & (0.97) \\
SDQ Composite - Child & 0.76 & 0.83 & \textbf{ 1.28 } & 0.68 & 0.18 & 1.28 \\
\quad \textit{Unadjusted P-Value} & (0.11) & (0.07) & (0.00) & (0.18) & (0.79) & (0.09) \\
\quad \textit{Stepdown P-Value} & (0.53) & (0.37) & (0.03) & (0.67) & (0.99) & (0.47) \\
SDQ Pro-social - Child & 0.25 & 0.40 & 0.19 & 0.30 & 0.09 & 0.38 \\
\quad \textit{Unadjusted P-Value} & (0.15) & (0.03) & (0.29) & (0.14) & (0.75) & (0.14) \\
\quad \textit{Stepdown P-Value} & (0.59) & (0.16) & (0.72) & (0.64) & (0.99) & (0.53) \\
SDQ Peer problems - Child & 0.01 & 0.01 & 0.13 & 0.05 & -0.12 & 0.11 \\
\quad \textit{Unadjusted P-Value} & (0.93) & (0.95) & (0.36) & (0.73) & (0.57) & (0.63) \\
\quad \textit{Stepdown P-Value} & (0.93) & (0.96) & (0.72) & (0.92) & (0.99) & (0.93) \\
SDQ Hyper - Child & 0.12 & 0.05 & 0.31 & -0.04 & 0.09 & -0.08 \\
\quad \textit{Unadjusted P-Value} & (0.60) & (0.82) & (0.14) & (0.89) & (0.78) & (0.82) \\
\quad \textit{Stepdown P-Value} & (0.93) & (0.95) & (0.57) & (0.92) & (0.99) & (0.97) \\
SDQ Emotional - Child & \textbf{ 0.40 } & \textbf{ 0.51 } & \textbf{ 0.49 } & \textbf{ 0.44 } & 0.17 & \textbf{ 0.88 } \\
\quad \textit{Unadjusted P-Value} & (0.02) & (0.00) & (0.01) & (0.02) & (0.49) & (0.00) \\
\quad \textit{Stepdown P-Value} & (0.13) & (0.00) & (0.03) & (0.15) & (0.99) & (0.00) \\
SDQ Conduct - Child & 0.24 & 0.26 & \textbf{ 0.35 } & 0.22 & 0.04 & 0.37 \\
\quad \textit{Unadjusted P-Value} & (0.10) & (0.08) & (0.02) & (0.18) & (0.86) & (0.11) \\
\quad \textit{Stepdown P-Value} & (0.53) & (0.37) & (0.14) & (0.67) & (0.99) & (0.49) \\
\bottomrule
\end{tabular}
}
\vspace{1ex} \\
\footnotesize\raggedright{\underline{Note1 :} This table shows the estimates of the coefficient for attending Reggio Approach preschools from multiple methods. We compare Reggio Approach individuals with those who attended other preschools. Column title indicates the corresponding control set and and model. \textbf{None} = OLS estimate with no control variables. \textbf{BIC} = OLS estimate with controls selected by Bayesian Information Criterion (BIC) and additional controls for male indicator, migrant indicator, and ITC attendance indicator. \textbf{Full} = OLS estimate with the full set of controls. \textbf{PSMR} =  propensity score matching between Reggio Approach people and people in Reggio who attended other types of preschool. \textbf{KMR} = Epanechinikov kernel matching between Reggio Approach people and people in Reggio who attended other types of preschool. \textbf{DidPm} = difference-in-difference estimate of (Reggio Muni - Parma Muni) - (Reggio Other - Parma Other). \textbf{PSMPm} = propensity score matching between Reggio Approach people and people who attended Parma preschools. \textbf{KMPm} = Epanechinikov kernel matching between Reggio Approach people and people who attended Parma preschools. \textbf{DidPv} = difference-in-difference estimate of (Reggio Muni - Padova Muni) - (Reggio Other - Padova Other). \textbf{PSMPv} = propensity score matching between Reggio Approach people and people who attended Padova preschools. \textbf{KMPv} = Epanechinikov kernel matching between Reggio Approach people and people who attended Padova preschools.} 

\footnotesize\raggedright{\underline{Note2 :} Both unadjusted p-value and stepdown p-value are reported. ***, **, and * indicate significance of the coefficients at the 15\%, 10\%, and 5\% levels respectively.}
\end{table}


\begin{table}[H] \caption{Estimation Results for Social Outcomes, Comparison to Non-RA Preschools, Child Cohort} \label{ols-S-child-reg-reli}
\scalebox{0.59}{\begin{tabular}{l c c c c c c c}
\toprule
 & None & BIC & Full & PSM & DidPm & DidPv \\
\midrule
Musical Instrument at Home & -0.01 & -0.04 & -0.03 & -0.04 & 0.02 & 0.02 \\
\quad \textit{Unadjusted P-Value} & (0.88) & (0.40) & (0.57) & (0.49) & (0.82) & (0.74) \\
\quad \textit{Stepdown P-Value} & (0.96) & (0.93) & (0.96) & (0.92) & (0.98) & (0.93) \\
Tell Worry at Home & 0.00 & -0.01 & 0.00 & -0.03 & -0.05 & 0.07 \\
\quad \textit{Unadjusted P-Value} & (0.98) & (0.88) & (0.97) & (0.62) & (0.50) & (0.34) \\
\quad \textit{Stepdown P-Value} & (0.98) & (0.99) & (0.98) & (0.92) & (0.98) & (0.87) \\
Tell Worry to Teacher & 0.06 & 0.03 & 0.04 & 0.05 & 0.02 & 0.06 \\
\quad \textit{Unadjusted P-Value} & (0.20) & (0.56) & (0.41) & (0.33) & (0.72) & (0.39) \\
\quad \textit{Stepdown P-Value} & (0.74) & (0.97) & (0.93) & (0.92) & (0.98) & (0.87) \\
Tell Worry to Friends & 0.01 & -0.00 & 0.02 & 0.02 & 0.07 & -0.05 \\
\quad \textit{Unadjusted P-Value} & (0.89) & (0.97) & (0.57) & (0.60) & (0.22) & (0.38) \\
\quad \textit{Stepdown P-Value} & (0.96) & (0.99) & (0.96) & (0.92) & (0.82) & (0.87) \\
Keep Worry to Myself & -0.06 & -0.05 & -0.06 & -0.04 & 0.01 & -0.09 \\
\quad \textit{Unadjusted P-Value} & (0.13) & (0.17) & (0.13) & (0.30) & (0.80) & (0.09) \\
\quad \textit{Stepdown P-Value} & (0.62) & (0.74) & (0.59) & (0.92) & (0.98) & (0.45) \\
Num. of Friends & -0.20 & -0.29 & -0.15 & -0.27 & -0.47 & -0.08 \\
\quad \textit{Unadjusted P-Value} & (0.41) & (0.24) & (0.57) & (0.29) & (0.31) & (0.90) \\
\quad \textit{Stepdown P-Value} & (0.96) & (0.80) & (0.96) & (0.92) & (0.87) & (0.93) \\
Candy Game: Willing to Share Candies & 0.00 & -0.00 & 0.00 & -0.03 & 0.02 & 0.05 \\
\quad \textit{Unadjusted P-Value} & (0.93) & (0.99) & (0.89) & (0.32) & (0.65) & (0.23) \\
\quad \textit{Stepdown P-Value} & (0.98) & (0.99) & (0.98) & (0.92) & (0.98) & (0.76) \\
\bottomrule
\end{tabular}
}
\vspace{1ex} \\
\footnotesize\raggedright{\underline{Note1 :} This table shows the estimates of the coefficient for attending Reggio Approach preschools from multiple methods. We compare Reggio Approach individuals with those who attended other preschools. Column title indicates the corresponding control set and and model. \textbf{None} = OLS estimate with no control variables. \textbf{BIC} = OLS estimate with controls selected by Bayesian Information Criterion (BIC) and additional controls for male indicator, migrant indicator, and ITC attendance indicator. \textbf{Full} = OLS estimate with the full set of controls. \textbf{PSMR} =  propensity score matching between Reggio Approach people and people in Reggio who attended other types of preschool. \textbf{KMR} = Epanechinikov kernel matching between Reggio Approach people and people in Reggio who attended other types of preschool. \textbf{DidPm} = difference-in-difference estimate of (Reggio Muni - Parma Muni) - (Reggio Other - Parma Other). \textbf{PSMPm} = propensity score matching between Reggio Approach people and people who attended Parma preschools. \textbf{KMPm} = Epanechinikov kernel matching between Reggio Approach people and people who attended Parma preschools. \textbf{DidPv} = difference-in-difference estimate of (Reggio Muni - Padova Muni) - (Reggio Other - Padova Other). \textbf{PSMPv} = propensity score matching between Reggio Approach people and people who attended Padova preschools. \textbf{KMPv} = Epanechinikov kernel matching between Reggio Approach people and people who attended Padova preschools.} 

\footnotesize\raggedright{\underline{Note2 :} Both unadjusted p-value and stepdown p-value are reported. ***, **, and * indicate significance of the coefficients at the 15\%, 10\%, and 5\% levels respectively.}
\end{table}


\begin{table}[H] \caption{Estimation Results for Health Outcomes, Comparison to Non-RA Preschools, Child Cohort} \label{ols-H-child-reg-reli}
\scalebox{0.59}{\begin{tabular}{l c c c c c c c c c c c}
\toprule
& \multicolumn{5}{c}{Within Reggio} & \multicolumn{3}{c}{With Parma} & \multicolumn{3}{c}{With Padova} \\\cmidrule(lr){2-6} \cmidrule(lr){7-9} \cmidrule(lr){10-12}
 & None & BIC & Full & PSMR & KMR & DidPm & PSMPm & KMPm & DidPv & PSMPv & KMPv \\
\midrule
Not Obese & -0.04 & -0.07 & -0.08 & -0.08 & -0.06 & -0.01 & -0.13 & -0.16 & 0.02 & -0.06 & -0.06 \\
\quad \textit{Unadjusted P-Value} & (0.47) & (0.16) & (0.14)* & (0.16) & (0.28) & (0.84) & (0.01)*** & (0.00)*** & (0.83) & (0.29) & (0.23) \\
\quad \textit{Stepdown P-Value} & (0.91) & (0.51) & (0.39) & (0.45) & (0.70) & (0.98) & (0.02)*** & (0.00)*** & (0.98) & (0.69) & (0.57) \\
Not Overweight & -0.02 & -0.01 & -0.02 & 0.00 & -0.01 & -0.02 & 0.05 & 0.02 & -0.04 & -0.04 & -0.04 \\
\quad \textit{Unadjusted P-Value} & (0.54) & (0.87) & (0.64) & (0.99) & (0.79) & (0.76) & (0.18) & (0.53) & (0.44) & (0.26) & (0.24) \\
\quad \textit{Stepdown P-Value} & (0.91) & (0.98) & (0.82) & (0.99) & (0.94) & (0.98) & (0.36) & (0.61) & (0.87) & (0.69) & (0.57) \\
Health is Good & -0.02 & -0.00 & 0.01 & -0.02 & -0.03 & 0.07 & 0.07 & 0.04 & -0.01 & -0.03 & -0.09 \\
\quad \textit{Unadjusted P-Value} & (0.78) & (0.99) & (0.87) & (0.70) & (0.64) & (0.43) & (0.16) & (0.39) & (0.93) & (0.55) & (0.06)** \\
\quad \textit{Stepdown P-Value} & (0.91) & (0.98) & (0.87) & (0.92) & (0.94) & (0.88) & (0.36) & (0.61) & (0.98) & (0.82) & (0.22) \\
Number of Sick Days & -0.03 & -0.10 & -0.09 & -0.07 & -0.05 & 0.03 & 0.13 & 0.14 & 0.02 & -0.03 & 0.11 \\
\quad \textit{Unadjusted P-Value} & (0.73) & (0.33) & (0.34) & (0.46) & (0.60) & (0.80) & (0.13)* & (0.09)** & (0.90) & (0.73) & (0.24) \\
\quad \textit{Stepdown P-Value} & (0.91) & (0.69) & (0.72) & (0.84) & (0.94) & (0.98) & (0.36) & (0.24) & (0.98) & (0.82) & (0.57) \\
\bottomrule
\end{tabular}
}
\vspace{1ex} \\
\footnotesize\raggedright{\underline{Note1 :} This table shows the estimates of the coefficient for attending Reggio Approach preschools from multiple methods. We compare Reggio Approach individuals with those who attended other preschools. Column title indicates the corresponding control set and and model. \textbf{None} = OLS estimate with no control variables. \textbf{BIC} = OLS estimate with controls selected by Bayesian Information Criterion (BIC) and additional controls for male indicator, migrant indicator, and ITC attendance indicator. \textbf{Full} = OLS estimate with the full set of controls. \textbf{PSMR} =  propensity score matching between Reggio Approach people and people in Reggio who attended other types of preschool. \textbf{KMR} = Epanechinikov kernel matching between Reggio Approach people and people in Reggio who attended other types of preschool. \textbf{DidPm} = difference-in-difference estimate of (Reggio Muni - Parma Muni) - (Reggio Other - Parma Other). \textbf{PSMPm} = propensity score matching between Reggio Approach people and people who attended Parma preschools. \textbf{KMPm} = Epanechinikov kernel matching between Reggio Approach people and people who attended Parma preschools. \textbf{DidPv} = difference-in-difference estimate of (Reggio Muni - Padova Muni) - (Reggio Other - Padova Other). \textbf{PSMPv} = propensity score matching between Reggio Approach people and people who attended Padova preschools. \textbf{KMPv} = Epanechinikov kernel matching between Reggio Approach people and people who attended Padova preschools.} 

\footnotesize\raggedright{\underline{Note2 :} Both unadjusted p-value and stepdown p-value are reported. ***, **, and * indicate significance of the coefficients at the 15\%, 10\%, and 5\% levels respectively.}
\end{table}


\begin{table}[H] \caption{Estimation Results for Behavioral Outcomes, Comparison to Non-RA Preschools, Child Cohort} \label{ols-B-child-reg-reli}
\scalebox{0.59}{\begin{tabular}{l c c c c c c c c c c c}
\toprule
& \multicolumn{5}{c}{Within Reggio} & \multicolumn{3}{c}{With Parma} & \multicolumn{3}{c}{With Padova} \\\cmidrule(lr){2-6} \cmidrule(lr){7-9} \cmidrule(lr){10-12}
 & None & BIC & Full & PSMR & KMR & DidPm & KMDidPm & KMPm & DidPv & KMDidPv & KMPv \\
\midrule
Not Excited to Learn & -0.01 & -0.00 & -0.01 & 0.00 & -0.00 & -0.00 & -0.01 & -0.02 & -0.04 & -0.03 & -0.02 \\
\quad \textit{Unadjusted P-Value} & (0.60) & (0.84) & (0.69) & (0.92) & (0.99) & (0.95) & (0.75) & (0.28) & (0.31) & (0.59) & (0.41) \\
\quad \textit{Stepdown P-Value} & (0.92) & (0.97) & (0.86) & (0.92) & (0.98) & (0.99) & (0.71) & (0.65) & (0.42) & (0.81) & (0.78) \\
Problems Sitting Still & -0.00 & 0.01 & -0.03 & 0.02 & 0.02 & -0.08 & -0.06 & -0.01 & -0.08 & -0.03 & -0.00 \\
\quad \textit{Unadjusted P-Value} & (0.90) & (0.78) & (0.51) & (0.71) & (0.63) & (0.16) & (0.74) & (0.85) & (0.20) & (0.64) & (0.90) \\
\quad \textit{Stepdown P-Value} & (0.98) & (0.97) & (0.85) & (0.92) & (0.95) & (0.42) & (0.69) & (0.83) & (0.39) & (0.81) & (0.98) \\
How Much Child Likes School & 0.14 & 0.11 & 0.15 & 0.10 & 0.11 & 0.24 & 0.17 & -0.04 & 0.29 & 0.25 & 0.33 \\
\quad \textit{Unadjusted P-Value} & (0.05)*** & (0.11)* & (0.04)*** & (0.19) & (0.15)* & (0.01)*** & (0.22) & (0.45) & (0.01)*** & (0.05)** & (0.00)*** \\
\quad \textit{Stepdown P-Value} & (0.18) & (0.39) & (0.16) & (0.54) & (0.45) & (0.03)*** & (0.30) & (0.70) & (0.04)*** & (0.11) & (0.00)*** \\
Happy in General & -0.02 & 0.06 & 0.08 & 0.13 & -0.03 & 0.00 & 0.22 & 0.37 & 0.27 & 0.21 & -0.02 \\
\quad \textit{Unadjusted P-Value} & (0.91) & (0.76) & (0.70) & (0.52) & (0.89) & (0.99) & (0.43) & (0.03)*** & (0.37) & (0.48) & (0.90) \\
\quad \textit{Stepdown P-Value} & (0.98) & (0.97) & (0.86) & (0.85) & (0.98) & (0.99) & (0.69) & (0.10)** & (0.45) & (0.81) & (0.98) \\
\bottomrule
\end{tabular}
}
\vspace{1ex} \\
\footnotesize\raggedright{\underline{Note1 :} This table shows the estimates of the coefficient for attending Reggio Approach preschools from multiple methods. We compare Reggio Approach individuals with those who attended other preschools. Column title indicates the corresponding control set and and model. \textbf{None} = OLS estimate with no control variables. \textbf{BIC} = OLS estimate with controls selected by Bayesian Information Criterion (BIC) and additional controls for male indicator, migrant indicator, and ITC attendance indicator. \textbf{Full} = OLS estimate with the full set of controls. \textbf{PSMR} =  propensity score matching between Reggio Approach people and people in Reggio who attended other types of preschool. \textbf{KMR} = Epanechinikov kernel matching between Reggio Approach people and people in Reggio who attended other types of preschool. \textbf{DidPm} = difference-in-difference estimate of (Reggio Muni - Parma Muni) - (Reggio Other - Parma Other). \textbf{PSMPm} = propensity score matching between Reggio Approach people and people who attended Parma preschools. \textbf{KMPm} = Epanechinikov kernel matching between Reggio Approach people and people who attended Parma preschools. \textbf{DidPv} = difference-in-difference estimate of (Reggio Muni - Padova Muni) - (Reggio Other - Padova Other). \textbf{PSMPv} = propensity score matching between Reggio Approach people and people who attended Padova preschools. \textbf{KMPv} = Epanechinikov kernel matching between Reggio Approach people and people who attended Padova preschools.} 

\footnotesize\raggedright{\underline{Note2 :} Both unadjusted p-value and stepdown p-value are reported. ***, **, and * indicate significance of the coefficients at the 15\%, 10\%, and 5\% levels respectively.}
\end{table}



\subsubsection{Adolescent Cohort}
\begin{table}[H] \caption{Estimation Results for Cognitive and Noncognitive Outcomes, Comparison to Non-RA Preschools, Adolescent Cohort} \label{ols-CN-adol-reg-reli}
\scalebox{0.6}{\begin{tabular}{l c c c c c c c}
\toprule
 & None & BIC & Full & PSM & DidPm & DidPv \\
\midrule
IQ Factor & -0.12 & -0.15 & -0.03 & -0.06 & -0.16 & -0.25 \\
\quad \textit{Unadjusted P-Value} & (0.22) & (0.15) & (0.78) & (0.53) & (0.21) & (0.15) \\
\quad \textit{Stepdown P-Value} & (0.94) & (0.87) & (0.99) & (0.99) & (0.92) & (0.91) \\
IQ Score & -0.03 & -0.03 & 0.00 & -0.01 & -0.06 & -0.06 \\
\quad \textit{Unadjusted P-Value} & (0.30) & (0.33) & (0.96) & (0.80) & (0.13) & (0.20) \\
\quad \textit{Stepdown P-Value} & (0.96) & (0.98) & (0.99) & (0.99) & (0.77) & (0.95) \\
SDQ Composite - Child & 0.01 & 0.18 & 0.37 & -0.56 & -0.41 & -0.52 \\
\quad \textit{Unadjusted P-Value} & (0.98) & (0.80) & (0.55) & (0.49) & (0.64) & (0.51) \\
\quad \textit{Stepdown P-Value} & (0.99) & (0.99) & (0.99) & (0.99) & (0.99) & (0.99) \\
SDQ Pro-social - Child & 0.16 & 0.02 & -0.13 & 0.08 & 0.06 & 0.00 \\
\quad \textit{Unadjusted P-Value} & (0.48) & (0.94) & (0.58) & (0.78) & (0.84) & (0.99) \\
\quad \textit{Stepdown P-Value} & (0.99) & (0.99) & (0.99) & (0.99) & (0.99) & (0.99) \\
SDQ Peer problems - Child & -0.12 & -0.19 & -0.05 & -0.37 & -0.84 & -0.35 \\
\quad \textit{Unadjusted P-Value} & (0.51) & (0.38) & (0.81) & (0.09) & (0.00) & (0.18) \\
\quad \textit{Stepdown P-Value} & (0.99) & (0.98) & (0.99) & (0.69) & (0.02) & (0.94) \\
SDQ Hyper - Child & 0.14 & 0.13 & 0.17 & -0.07 & -0.03 & -0.16 \\
\quad \textit{Unadjusted P-Value} & (0.53) & (0.61) & (0.46) & (0.78) & (0.94) & (0.61) \\
\quad \textit{Stepdown P-Value} & (0.99) & (0.99) & (0.99) & (0.99) & (0.99) & (0.99) \\
SDQ Emotional - Child & -0.02 & 0.04 & 0.04 & -0.22 & 0.05 & -0.08 \\
\quad \textit{Unadjusted P-Value} & (0.93) & (0.89) & (0.89) & (0.56) & (0.88) & (0.81) \\
\quad \textit{Stepdown P-Value} & (0.99) & (0.99) & (0.99) & (0.99) & (0.99) & (0.99) \\
SDQ Conduct - Child & 0.02 & 0.20 & 0.22 & 0.11 & 0.41 & 0.05 \\
\quad \textit{Unadjusted P-Value} & (0.91) & (0.30) & (0.24) & (0.64) & (0.11) & (0.83) \\
\quad \textit{Stepdown P-Value} & (0.99) & (0.95) & (0.94) & (0.99) & (0.75) & (0.99) \\
SDQ Composite & 0.90 & 1.03 & 0.72 & 1.02 & 0.90 & 0.71 \\
\quad \textit{Unadjusted P-Value} & (0.15) & (0.14) & (0.32) & (0.22) & (0.32) & (0.45) \\
\quad \textit{Stepdown P-Value} & (0.84) & (0.87) & (0.97) & (0.96) & (0.94) & (0.96) \\
SDQ Pro-social & 0.10 & -0.09 & -0.06 & 0.06 & -0.07 & -0.35 \\
\quad \textit{Unadjusted P-Value} & (0.65) & (0.70) & (0.81) & (0.81) & (0.83) & (0.26) \\
\quad \textit{Stepdown P-Value} & (0.99) & (0.99) & (0.99) & (0.99) & (0.99) & (0.96) \\
SDQ Peer problems & -0.09 & -0.17 & -0.15 & -0.05 & -0.50 & -0.04 \\
\quad \textit{Unadjusted P-Value} & (0.60) & (0.38) & (0.44) & (0.85) & (0.04) & (0.88) \\
\quad \textit{Stepdown P-Value} & (0.99) & (0.98) & (0.99) & (0.99) & (0.43) & (0.99) \\
SDQ Hyper & 0.38 & 0.41 & 0.30 & 0.39 & 0.39 & 0.28 \\
\quad \textit{Unadjusted P-Value} & (0.13) & (0.14) & (0.28) & (0.21) & (0.31) & (0.45) \\
\quad \textit{Stepdown P-Value} & (0.83) & (0.87) & (0.97) & (0.96) & (0.94) & (0.99) \\
SDQ Emotional & 0.27 & 0.24 & 0.18 & 0.23 & 0.40 & -0.03 \\
\quad \textit{Unadjusted P-Value} & (0.32) & (0.40) & (0.53) & (0.51) & (0.29) & (0.93) \\
\quad \textit{Stepdown P-Value} & (0.96) & (0.98) & (0.99) & (0.99) & (0.94) & (0.99) \\
SDQ Conduct & 0.35 & 0.55 & 0.38 & 0.44 & 0.60 & 0.51 \\
\quad \textit{Unadjusted P-Value} & (0.07) & (0.01) & (0.09) & (0.12) & (0.04) & (0.07) \\
\quad \textit{Stepdown P-Value} & (0.60) & (0.10) & (0.48) & (0.78) & (0.35) & (0.65) \\
Depression Score - positive & 1.46 & 2.39 & 1.81 & 2.24 & 2.21 & 2.19 \\
\quad \textit{Unadjusted P-Value} & (0.06) & (0.01) & (0.05) & (0.03) & (0.03) & (0.05) \\
\quad \textit{Stepdown P-Value} & (0.60) & (0.09) & (0.33) & (0.35) & (0.37) & (0.52) \\
\bottomrule
\end{tabular}
}
\vspace{1ex} \\
\footnotesize\raggedright{\underline{Note1 :} This table shows the estimates of the coefficient for attending Reggio Approach preschools from multiple methods. We compare Reggio Approach individuals with those who attended other preschools. Column title indicates the corresponding control set and and model. \textbf{None} = OLS estimate with no control variables. \textbf{BIC} = OLS estimate with controls selected by Bayesian Information Criterion (BIC) and additional controls for male indicator, migrant indicator, and ITC attendance indicator. \textbf{Full} = OLS estimate with the full set of controls. \textbf{PSMR} =  propensity score matching between Reggio Approach people and people in Reggio who attended nother types of preschool. \textbf{KMR} = Epanechinikov kernel matching between Reggio Approach people and people in Reggio who attended nother types of preschool. \textbf{DidPm} = difference-in-difference estimate of (Reggio Muni - Parma Muni) - (Reggio Other - Parma Other). \textbf{PSMPm} = propensity score matching between Reggio Approach people and people who attended Parma preschools. \textbf{KMPm} = Epanechinikov kernel matching between Reggio Approach people and people who attended Parma preschools. \textbf{DidPv} = difference-in-difference estimate of (Reggio Muni - Padova Muni) - (Reggio Other - Padova Other). \textbf{PSMPv} = propensity score matching between Reggio Approach people and people who attended Padova preschools. \textbf{KMPv} = Epanechinikov kernel matching between Reggio Approach people and people who attended Padova preschools.} 

\footnotesize\raggedright{\underline{Note2 :} Both unadjusted p-value and stepdown p-value are reported. ***, **, and * indicate significance of the coefficients at the 15\%, 10\%, and 5\% levels respectively.}
\end{table}


\begin{table}[H] \caption{Estimation Results for Social Outcomes, Comparison to Non-RA Preschools,  Adolescent Cohort} \label{ols-S-adol-reg-reli}
\scalebox{0.6}{\begin{tabular}{l c c c c c c c c c c c}
\toprule
& \multicolumn{5}{c}{Within Reggio} & \multicolumn{3}{c}{With Parma} & \multicolumn{3}{c}{With Padova} \\\cmidrule(lr){2-6} \cmidrule(lr){7-9} \cmidrule(lr){10-12}
 & None & BIC & Full & PSMR & KMR & DidPm & KMDidPm & KMPm & DidPv & KMDidPv & KMPv \\
\midrule
Num. of Friends & -0.76 & -0.57 & -0.35 & -0.69 & 0.18 & -2.81 & -2.24 & 0.55 & -2.53 & -0.28 & -1.16 \\
\quad \textit{Unadjusted P-Value} & (0.54) & (0.59) & (0.76) & (0.56) & (0.92) & (0.14)* & (0.14)* & (0.61) & (0.27) & (0.86) & (0.40) \\
\quad \textit{Stepdown P-Value} & (0.85) & (0.90) & (0.96) & (0.70) & (0.99) & (0.30) & (0.45) & (0.63) & (0.53) & (0.94) & (0.78) \\
Doesn't Talk About Activities & 0.09 & 0.09 & 0.02 & 0.12 & 0.08 & 0.22 & 0.25 & -0.27 & 0.11 & 0.13 & 0.02 \\
\quad \textit{Unadjusted P-Value} & (0.24) & (0.33) & (0.82) & (0.25) & (0.38) & (0.09)** & (0.11)* & (0.00)*** & (0.35) & (0.33) & (0.77) \\
\quad \textit{Stepdown P-Value} & (0.66) & (0.78) & (0.96) & (0.66) & (0.88) & (0.21) & (0.26) & (0.01)*** & (0.53) & (0.78) & (0.92) \\
Doesn't Talk About School & 0.06 & 0.06 & 0.02 & 0.13 & 0.02 & 0.12 & 0.07 & -0.17 & 0.15 & 0.12 & -0.01 \\
\quad \textit{Unadjusted P-Value} & (0.45) & (0.50) & (0.83) & (0.22) & (0.81) & (0.32) & (0.66) & (0.02)*** & (0.17) & (0.39) & (0.89) \\
\quad \textit{Stepdown P-Value} & (0.85) & (0.90) & (0.96) & (0.66) & (0.99) & (0.53) & (0.86) & (0.04)*** & (0.50) & (0.78) & (0.92) \\
Volunteers & -0.02 & 0.01 & 0.04 & -0.05 & -0.01 & -0.02 & -0.04 & 0.20 & -0.04 & -0.03 & 0.08 \\
\quad \textit{Unadjusted P-Value} & (0.71) & (0.92) & (0.50) & (0.52) & (0.84) & (0.79) & (0.80) & (0.00)*** & (0.68) & (0.74) & (0.15)* \\
\quad \textit{Stepdown P-Value} & (0.85) & (0.94) & (0.89) & (0.70) & (0.99) & (0.80) & (0.86) & (0.00)*** & (0.66) & (0.94) & (0.44) \\
\bottomrule
\end{tabular}
}
\vspace{1ex} \\
\footnotesize\raggedright{\underline{Note1 :} This table shows the estimates of the coefficient for attending Reggio Approach preschools from multiple methods. We compare Reggio Approach individuals with those who attended other preschools. Column title indicates the corresponding control set and and model. \textbf{None} = OLS estimate with no control variables. \textbf{BIC} = OLS estimate with controls selected by Bayesian Information Criterion (BIC) and additional controls for male indicator, migrant indicator, and ITC attendance indicator. \textbf{Full} = OLS estimate with the full set of controls. \textbf{PSMR} =  propensity score matching between Reggio Approach people and people in Reggio who attended nother types of preschool. \textbf{KMR} = Epanechinikov kernel matching between Reggio Approach people and people in Reggio who attended nother types of preschool. \textbf{DidPm} = difference-in-difference estimate of (Reggio Muni - Parma Muni) - (Reggio Other - Parma Other). \textbf{PSMPm} = propensity score matching between Reggio Approach people and people who attended Parma preschools. \textbf{KMPm} = Epanechinikov kernel matching between Reggio Approach people and people who attended Parma preschools. \textbf{DidPv} = difference-in-difference estimate of (Reggio Muni - Padova Muni) - (Reggio Other - Padova Other). \textbf{PSMPv} = propensity score matching between Reggio Approach people and people who attended Padova preschools. \textbf{KMPv} = Epanechinikov kernel matching between Reggio Approach people and people who attended Padova preschools.} 

\footnotesize\raggedright{\underline{Note2 :} Both unadjusted p-value and stepdown p-value are reported. ***, **, and * indicate significance of the coefficients at the 15\%, 10\%, and 5\% levels respectively.}
\end{table}


\begin{table}[H] \caption{Estimation Results for Health Outcomes, Comparison to Non-RA Preschools,  Adolescent Cohort} \label{ols-H-adol-reg-reli}
\scalebox{0.6}{\begin{tabular}{l c c c c c c c c c c c}
\toprule
 & None & BIC & Full & PSMR & KMR & DidPm & PSMPm & KMPm & DidPv & PSMPv & KMPv \\
\midrule
Not Obese & -0.08 & -0.11 & -0.09 & -0.07 & -0.07 & 0.03 & -0.07 & -0.07 & -0.09 & 0.07 & 0.07 \\
\quad \textit{Unadjusted P-Value} & (0.04)*** & (0.03)*** & (0.03)*** & (0.10)** & (0.15)* & (0.65) & (0.09)** & (0.07)** & (0.23) & (0.14)* & (0.22) \\
\quad \textit{Stepdown P-Value} & (0.24) & (0.15) & (0.15) & (0.41) & (0.55) & (0.61) & (0.31) & (0.23) & (0.56) & (0.48) & (0.60) \\
Not Overweight & 0.01 & -0.02 & -0.00 & -0.03 & 0.01 & 0.09 & 0.01 & 0.03 & -0.03 & -0.03 & -0.03 \\
\quad \textit{Unadjusted P-Value} & (0.75) & (0.58) & (0.98) & (0.42) & (0.84) & (0.03)*** & (0.80) & (0.17) & (0.31) & (0.13)* & (0.19) \\
\quad \textit{Stepdown P-Value} & (0.93) & (0.89) & (0.98) & (0.77) & (0.99) & (0.07)** & (0.94) & (0.47) & (0.56) & (0.48) & (0.60) \\
Health is Good & 0.06 & 0.07 & 0.09 & 0.05 & 0.02 & 0.11 & 0.16 & 0.17 & 0.16 & 0.04 & 0.04 \\
\quad \textit{Unadjusted P-Value} & (0.32) & (0.28) & (0.15) & (0.50) & (0.82) & (0.22) & (0.01)*** & (0.00)*** & (0.07)** & (0.60) & (0.50) \\
\quad \textit{Stepdown P-Value} & (0.76) & (0.73) & (0.43) & (0.77) & (0.99) & (0.53) & (0.05)** & (0.02)*** & (0.25) & (0.80) & (0.80) \\
Number of Sick Days & 0.02 & -0.02 & 0.00 & -0.01 & -0.03 & -0.20 & -0.01 & -0.11 & 0.16 & -0.04 & 0.04 \\
\quad \textit{Unadjusted P-Value} & (0.86) & (0.88) & (0.97) & (0.89) & (0.83) & (0.17) & (0.91) & (0.22) & (0.27) & (0.70) & (0.73) \\
\quad \textit{Stepdown P-Value} & (0.93) & (0.89) & (0.98) & (0.90) & (0.99) & (0.51) & (0.94) & (0.47) & (0.56) & (0.80) & (0.80) \\
Ever Suspended from School & 0.02 & 0.01 & 0.03 & 0.03 & 0.02 & 0.04 & 0.06 & 0.02 & -0.00 & 0.05 & 0.05 \\
\quad \textit{Unadjusted P-Value} & (0.46) & (0.68) & (0.46) & (0.32) & (0.66) & (0.34) & (0.12)* & (0.44) & (0.92) & (0.36) & (0.04)*** \\
\quad \textit{Stepdown P-Value} & (0.82) & (0.89) & (0.73) & (0.77) & (0.98) & (0.56) & (0.31) & (0.47) & (0.92) & (0.76) & (0.17) \\
\bottomrule
\end{tabular}
}
\vspace{1ex} \\
\footnotesize\raggedright{\underline{Note1 :} This table shows the estimates of the coefficient for attending Reggio Approach preschools from multiple methods. We compare Reggio Approach individuals with those who attended other preschools. Column title indicates the corresponding control set and and model. \textbf{None} = OLS estimate with no control variables. \textbf{BIC} = OLS estimate with controls selected by Bayesian Information Criterion (BIC) and additional controls for male indicator, migrant indicator, and ITC attendance indicator. \textbf{Full} = OLS estimate with the full set of controls. \textbf{PSMR} =  propensity score matching between Reggio Approach people and people in Reggio who attended nother types of preschool. \textbf{KMR} = Epanechinikov kernel matching between Reggio Approach people and people in Reggio who attended nother types of preschool. \textbf{DidPm} = difference-in-difference estimate of (Reggio Muni - Parma Muni) - (Reggio Other - Parma Other). \textbf{PSMPm} = propensity score matching between Reggio Approach people and people who attended Parma preschools. \textbf{KMPm} = Epanechinikov kernel matching between Reggio Approach people and people who attended Parma preschools. \textbf{DidPv} = difference-in-difference estimate of (Reggio Muni - Padova Muni) - (Reggio Other - Padova Other). \textbf{PSMPv} = propensity score matching between Reggio Approach people and people who attended Padova preschools. \textbf{KMPv} = Epanechinikov kernel matching between Reggio Approach people and people who attended Padova preschools.} 

\footnotesize\raggedright{\underline{Note2 :} Both unadjusted p-value and stepdown p-value are reported. ***, **, and * indicate significance of the coefficients at the 15\%, 10\%, and 5\% levels respectively.}
\end{table}


\begin{table}[H] \caption{Estimation Results for Behavioral Outcomes, Comparison to Non-RA Preschools,  Adolescent Cohort} \label{ols-B-adol-reg-reli}
\scalebox{0.6}{\begin{tabular}{l c c c c c c c c c c c}
\toprule
 & None & BIC & Full & PSMR & KMR & DidPm & PSMPm & KMPm & DidPv & PSMPv & KMPv \\
\midrule
Not Excited to Learn & -0.01 & -0.00 & -0.01 & -0.00 & 0.01 & -0.05 & 0.01 & 0.01 & 0.02 & -0.03 & -0.01 \\
\quad \textit{Unadjusted P-Value} & (0.75) & (0.88) & (0.55) & (0.96) & (0.74) & (0.07) & (0.79) & (0.55) & (0.59) & (0.46) & (0.73) \\
\quad \textit{Stepdown P-Value} & (0.97) & (0.99) & (0.84) & (0.99) & (0.98) & (0.33) & (0.98) & (0.81) & (0.89) & (0.93) & (0.99) \\
Problems Sitting Still & 0.00 & 0.03 & 0.01 & 0.05 & 0.01 & 0.00 & -0.00 & -0.06 & 0.05 & -0.02 & -0.04 \\
\quad \textit{Unadjusted P-Value} & (0.97) & (0.47) & (0.81) & (0.27) & (0.70) & (0.99) & (0.93) & (0.06) & (0.33) & (0.76) & (0.30) \\
\quad \textit{Stepdown P-Value} & (0.97) & (0.91) & (0.89) & (0.82) & (0.98) & (0.98) & (0.99) & (0.24) & (0.75) & (0.93) & (0.84) \\
Go To School & 0.03 & 0.01 & 0.03 & -0.01 & -0.00 & 0.03 & -0.00 & 0.03 & 0.04 & -0.02 & -0.00 \\
\quad \textit{Unadjusted P-Value} & (0.22) & (0.78) & (0.22) & (0.76) & (0.96) & (0.35) & (0.92) & (0.14) & (0.20) & (0.60) & (0.90) \\
\quad \textit{Stepdown P-Value} & (0.67) & (0.99) & (0.63) & (0.99) & (0.98) & (0.63) & (0.99) & (0.40) & (0.54) & (0.93) & (0.99) \\
How Much Child Likes School & -0.11 & -0.05 & -0.17 & -0.04 & -0.08 & -0.04 & 0.08 & 0.01 & -0.10 & 0.11 & -0.11 \\
\quad \textit{Unadjusted P-Value} & (0.33) & (0.67) & (0.17) & (0.74) & (0.55) & (0.82) & (0.46) & (0.89) & (0.56) & (0.44) & (0.36) \\
\quad \textit{Stepdown P-Value} & (0.79) & (0.99) & (0.63) & (0.99) & (0.98) & (0.96) & (0.91) & (0.90) & (0.89) & (0.93) & (0.84) \\
Bothered by Migrants & \textbf{ 0.25 } & 0.27 & 0.22 & 0.22 & 0.25 & \textbf{ 0.51 } & -0.16 & -0.09 & 0.20 & 0.18 & 0.15 \\
\quad \textit{Unadjusted P-Value} & (0.02) & (0.02) & (0.06) & (0.09) & (0.06) & (0.00) & (0.11) & (0.43) & (0.23) & (0.15) & (0.16) \\
\quad \textit{Stepdown P-Value} & (0.10) & (0.16) & (0.26) & (0.46) & (0.27) & (0.01) & (0.43) & (0.81) & (0.67) & (0.68) & (0.61) \\
Trust Score & 0.03 & 0.06 & 0.04 & 0.09 & 0.13 & 0.45 & \textbf{ -0.48 } & -0.38 & -0.09 & -0.15 & -0.06 \\
\quad \textit{Unadjusted P-Value} & (0.85) & (0.76) & (0.83) & (0.71) & (0.57) & (0.08) & (0.01) & (0.03) & (0.72) & (0.49) & (0.74) \\
\quad \textit{Stepdown P-Value} & (0.97) & (0.99) & (0.89) & (0.99) & (0.98) & (0.33) & (0.05) & (0.20) & (0.89) & (0.93) & (0.99) \\
Days of Sport (Weekly) & -0.43 & -0.56 & -0.33 & -0.32 & -0.66 & -0.62 & -0.37 & -0.42 & -0.57 & -0.39 & \textbf{ -0.56 } \\
\quad \textit{Unadjusted P-Value} & (0.06) & (0.04) & (0.20) & (0.33) & (0.03) & (0.06) & (0.09) & (0.04) & (0.13) & (0.22) & (0.02) \\
\quad \textit{Stepdown P-Value} & (0.28) & (0.16) & (0.63) & (0.85) & (0.20) & (0.31) & (0.41) & (0.22) & (0.54) & (0.76) & (0.12) \\
\bottomrule
\end{tabular}
}
\vspace{1ex} \\
\footnotesize\raggedright{\underline{Note1 :} This table shows the estimates of the coefficient for attending Reggio Approach preschools from multiple methods. We compare Reggio Approach individuals with those who attended other preschools. Column title indicates the corresponding control set and and model. \textbf{None} = OLS estimate with no control variables. \textbf{BIC} = OLS estimate with controls selected by Bayesian Information Criterion (BIC) and additional controls for male indicator, migrant indicator, and ITC attendance indicator. \textbf{Full} = OLS estimate with the full set of controls. \textbf{PSMR} =  propensity score matching between Reggio Approach people and people in Reggio who attended nother types of preschool. \textbf{KMR} = Epanechinikov kernel matching between Reggio Approach people and people in Reggio who attended nother types of preschool. \textbf{DidPm} = difference-in-difference estimate of (Reggio Muni - Parma Muni) - (Reggio Other - Parma Other). \textbf{PSMPm} = propensity score matching between Reggio Approach people and people who attended Parma preschools. \textbf{KMPm} = Epanechinikov kernel matching between Reggio Approach people and people who attended Parma preschools. \textbf{DidPv} = difference-in-difference estimate of (Reggio Muni - Padova Muni) - (Reggio Other - Padova Other). \textbf{PSMPv} = propensity score matching between Reggio Approach people and people who attended Padova preschools. \textbf{KMPv} = Epanechinikov kernel matching between Reggio Approach people and people who attended Padova preschools.} 

\footnotesize\raggedright{\underline{Note2 :} Both unadjusted p-value and stepdown p-value are reported. ***, **, and * indicate significance of the coefficients at the 15\%, 10\%, and 5\% levels respectively.}
\end{table}




\subsubsection{Age-30 Cohort}
\begin{table}[H] \caption{Estimation Results for Cognitive and Education Outcomes, Comparison to Non-RA Preschools, Age-30 Cohort} \label{ols-CN-adult30-reg-other}
\scalebox{0.6}{\begin{tabular}{l c c c c c c c}
\toprule
 & None & BIC & Full & PSM & DidPm & DidPv \\
\midrule
IQ Factor & 0.01 & -0.01 & 0.04 & -0.12 & -0.29 & 0.13 \\
\quad \textit{Unadjusted P-Value} & (0.95) & (0.92) & (0.77) & (0.58) & (0.11) & (0.56) \\
\quad \textit{Stepdown P-Value} & (0.99) & (0.99) & (0.96) & (0.91) & (0.53) & (0.79) \\
High School Grade & 1.05 & 0.56 & 0.66 & 1.40 & 1.70 & 0.94 \\
\quad \textit{Unadjusted P-Value} & (0.49) & (0.71) & (0.67) & (0.40) & (0.63) & (0.80) \\
\quad \textit{Stepdown P-Value} & (0.98) & (0.92) & (0.90) & (0.91) & (0.92) & (0.79) \\
Graduate from High School & -0.05 & -0.04 & -0.06 & -0.04 & 0.04 & -0.12 \\
\quad \textit{Unadjusted P-Value} & (0.31) & (0.38) & (0.23) & (0.44) & (0.57) & (0.09) \\
\quad \textit{Stepdown P-Value} & (0.90) & (0.92) & (0.72) & (0.91) & (0.92) & (0.55) \\
Max Edu: University & 0.02 & 0.01 & 0.00 & -0.03 & 0.08 & 0.19 \\
\quad \textit{Unadjusted P-Value} & (0.76) & (0.89) & (1.00) & (0.71) & (0.48) & (0.16) \\
\quad \textit{Stepdown P-Value} & (0.99) & (0.92) & (0.96) & (0.91) & (0.92) & (0.52) \\
\bottomrule
\end{tabular}
}
\vspace{1ex} \\
\footnotesize\raggedright{\underline{Note1 :} This table shows the estimates of the coefficient for attending Reggio Approach preschools from multiple methods. We compare Reggio Approach individuals with those who attended other preschools. Column title indicates the corresponding control set and and model. \textbf{None} = OLS estimate with no control variables. \textbf{BIC} = OLS estimate with controls selected by Bayesian Information Criterion (BIC) and additional controls for male indicator, migrant indicator, and ITC attendance indicator. \textbf{Full} = OLS estimate with the full set of controls. \textbf{PSMR} =  propensity score matching between Reggio Approach people and people in Reggio who attended nother types of preschool. \textbf{KMR} = Epanechinikov kernel matching between Reggio Approach people and people in Reggio who attended nother types of preschool. \textbf{DidPm} = difference-in-difference estimate of (Reggio Muni - Parma Muni) - (Reggio Other - Parma Other). \textbf{PSMPm} = propensity score matching between Reggio Approach people and people who attended Parma preschools. \textbf{KMPm} = Epanechinikov kernel matching between Reggio Approach people and people who attended Parma preschools. \textbf{DidPv} = difference-in-difference estimate of (Reggio Muni - Padova Muni) - (Reggio Other - Padova Other). \textbf{PSMPv} = propensity score matching between Reggio Approach people and people who attended Padova preschools. \textbf{KMPv} = Epanechinikov kernel matching between Reggio Approach people and people who attended Padova preschools.} 

\footnotesize\raggedright{\underline{Note2 :} Both unadjusted p-value and stepdown p-value are reported. ***, **, and * indicate significance of the coefficients at the 15\%, 10\%, and 5\% levels respectively.}
\end{table}

\begin{table}[H] \caption{Estimation Results for Cognitive and Education Outcomes, Comparison to No Preschool, Age-30 Cohort} \label{ols-CN-adult30-reg-none}
\scalebox{0.6}{\begin{tabular}{l c c c c c c c}
\toprule
 & None & BIC & Full & PSM & DidPm & DidPv \\
\midrule
IQ Factor & 0.14 & 0.03 & -0.05 & 0.15 & -0.24 & -0.11 \\
\quad \textit{Unadjusted P-Value} & (0.39) & (0.82) & (0.74) & (0.43) & (0.26) & (0.69) \\
\quad \textit{Stepdown P-Value} & (0.71) & (0.98) & (0.91) & (0.81) & (0.75) & (0.94) \\
High School Grade & \textbf{ 4.54 } & \textbf{ 4.98 } & \textbf{ 4.62 } & \textbf{ 5.57 } & 0.35 & 3.16 \\
\quad \textit{Unadjusted P-Value} & (0.03) & (0.02) & (0.04) & (0.00) & (0.94) & (0.45) \\
\quad \textit{Stepdown P-Value} & (0.07) & (0.04) & (0.44) & (0.03) & (0.92) & (0.86) \\
Graduate from High School & -0.03 & 0.02 & 0.03 & 0.03 & 0.12 & -0.05 \\
\quad \textit{Unadjusted P-Value} & (0.55) & (0.62) & (0.57) & (0.66) & (0.19) & (0.58) \\
\quad \textit{Stepdown P-Value} & (0.71) & (0.98) & (0.82) & (0.89) & (0.67) & (0.94) \\
Max Edu: University & -0.07 & -0.03 & -0.04 & -0.02 & 0.01 & -0.15 \\
\quad \textit{Unadjusted P-Value} & (0.32) & (0.72) & (0.57) & (0.80) & (0.93) & (0.30) \\
\quad \textit{Stepdown P-Value} & (0.71) & (0.98) & (0.80) & (0.89) & (0.82) & (0.82) \\
\bottomrule
\end{tabular}
}
\vspace{1ex} \\
\footnotesize\raggedright{\underline{Note1 :} This table shows the estimates of the coefficient for attending Reggio Approach preschools from multiple methods. We compare Reggio Approach individuals with those who did not attend any preschools. Column title indicates the corresponding control set and and model. \textbf{None} = OLS estimate with no control variables. \textbf{BIC} = OLS estimate with controls selected by Bayesian Information Criterion (BIC) and additional controls for male indicator, migrant indicator, and ITC attendance indicator. \textbf{Full} = OLS estimate with the full set of controls. \textbf{PSMR} =  propensity score matching between Reggio Approach people and people in Reggio who did not attend any preschool. \textbf{KMR} = Epanechinikov kernel matching between Reggio Approach people and people in Reggio who attended other types of preschool. \textbf{DidPm} = difference-in-difference estimate of (Reggio Muni - Parma Muni) - (Reggio None - Parma None). \textbf{PSMPm} = propensity score matching between Reggio Approach people and people in Parma who did not attend any preschool. \textbf{KMPm} = Epanechinikov kernel matching between Reggio Approach people and people in Parma who did not attend any preschool. \textbf{DidPv} = difference-in-difference estimate of (Reggio Muni - Padova Muni) - (Reggio None - Padova None). \textbf{PSMPv} = propensity score matching between Reggio Approach people and people in Padova who did not attend any preschool. \textbf{KMPv} = Epanechinikov kernel matching between Reggio Approach people and people in Padova who did not attend any preschool.} 

\footnotesize\raggedright{\underline{Note2 :} Both unadjusted p-value and stepdown p-value are reported. ***, **, and * indicate significance of the coefficients at the 15\%, 10\%, and 5\% levels respectively.}
\end{table}


\begin{table}[H] \caption{Estimation Results for Employment Outcomes, Comparison to Non-RA Preschools, Age-30 Cohort} \label{ols-W-adult30-reg-other}
\scalebox{0.6}{\begin{tabular}{l c c c c c c c c c c c}
\toprule
 & None & BIC & Full & PSMR & KMR & DidPm & PSMPm & KMPm & DidPv & PSMPv & KMPv \\
\midrule
Employed & -0.03 & -0.03 & -0.02 & -0.02 & -0.02 & 0.11 & 0.02 & 0.01 & -0.03 & 0.04 & 0.05 \\
\quad \textit{Unadjusted P-Value} & (0.39) & (0.43) & (0.64) & (0.56) & (0.59) & (0.17) & (0.62) & (0.78) & (0.75) & (0.35) & (0.19) \\
\quad \textit{Stepdown P-Value} & (0.95) & (0.91) & (0.92) & (0.97) & (0.97) & (0.49) & (0.70) & (0.75) & (0.99) & (0.85) & (0.69) \\
Self-Employed & -0.02 & -0.05 & -0.06 & -0.05 & -0.04 & -0.12 & 0.05 & 0.08 & 0.05 & -0.04 & -0.04 \\
\quad \textit{Unadjusted P-Value} & (0.71) & (0.32) & (0.31) & (0.42) & (0.47) & (0.07)** & (0.09)** & (0.03)*** & (0.42) & (0.25) & (0.31) \\
\quad \textit{Stepdown P-Value} & (0.98) & (0.84) & (0.78) & (0.93) & (0.96) & (0.47) & (0.36) & (0.18) & (0.99) & (0.82) & (0.85) \\
Hours Worked Per Week & -0.02 & 0.19 & 0.63 & 0.64 & 0.80 & 3.26 & 2.59 & 1.82 & 2.21 & 1.77 & 0.54 \\
\quad \textit{Unadjusted P-Value} & (0.99) & (0.93) & (0.77) & (0.85) & (0.70) & (0.44) & (0.31) & (0.47) & (0.64) & (0.56) & (0.78) \\
\quad \textit{Stepdown P-Value} & (0.99) & (0.99) & (0.96) & (0.99) & (0.97) & (0.80) & (0.63) & (0.75) & (0.99) & (0.91) & (0.92) \\
Income: 5,000 Euros of Less & -0.01 & -0.01 & -0.03 & -0.01 & -0.02 & -0.03 & 0.07 & 0.07 & -0.09 & 0.06 & 0.05 \\
\quad \textit{Unadjusted P-Value} & (0.76) & (0.85) & (0.49) & (0.94) & (0.71) & (0.48) & (0.05)** & (0.01)*** & (0.27) & (0.16) & (0.08)** \\
\quad \textit{Stepdown P-Value} & (0.98) & (0.91) & (0.86) & (0.99) & (0.97) & (0.80) & (0.27) & (0.11) & (0.73) & (0.71) & (0.45) \\
Income: 5,001-10,000 Euros & 0.01 & 0.02 & 0.02 & 0.02 & 0.01 & 0.02 & 0.02 & 0.01 & 0.02 & 0.03 & 0.01 \\
\quad \textit{Unadjusted P-Value} & (0.32) & (0.31) & (0.31) & (0.32) & (0.32) & (0.32) & (0.52) & (0.32) & (0.25) & (0.40) & (0.56) \\
\quad \textit{Stepdown P-Value} & (0.95) & (0.71) & (0.73) & (0.85) & (0.84) & (0.80) & (0.70) & (0.60) & (0.88) & (0.85) & (0.88) \\
Income: 10,001-25,000 Euros & -0.03 & 0.00 & 0.01 & 0.02 & 0.03 & -0.14 & -0.29 & -0.29 & 0.02 & -0.10 & -0.11 \\
\quad \textit{Unadjusted P-Value} & (0.69) & (0.95) & (0.86) & (0.83) & (0.70) & (0.27) & (0.00)*** & (0.00)*** & (0.89) & (0.10)* & (0.07)** \\
\quad \textit{Stepdown P-Value} & (0.98) & (0.99) & (0.99) & (0.99) & (0.97) & (0.80) & (0.01)*** & (0.00)*** & (0.99) & (0.58) & (0.45) \\
Income: 25,001-50,000 Euros & -0.07 & -0.12 & -0.11 & -0.13 & -0.13 & -0.05 & 0.14 & 0.14 & -0.12 & 0.03 & 0.07 \\
\quad \textit{Unadjusted P-Value} & (0.41) & (0.16) & (0.17) & (0.23) & (0.16) & (0.72) & (0.04)*** & (0.05)** & (0.40) & (0.69) & (0.33) \\
\quad \textit{Stepdown P-Value} & (0.95) & (0.71) & (0.68) & (0.75) & (0.61) & (0.80) & (0.27) & (0.19) & (0.93) & (0.91) & (0.85) \\
Income: 50,001-100,000 Euros & 0.10 & 0.10 & 0.11 & 0.10 & 0.10 & 0.19 & 0.06 & 0.08 & 0.16 & -0.01 & -0.00 \\
\quad \textit{Unadjusted P-Value} & (0.00)*** & (0.01)*** & (0.01)*** & (0.01)*** & (0.01)*** & (0.00)*** & (0.09)** & (0.03)*** & (0.03)*** & (0.90) & (0.93) \\
\quad \textit{Stepdown P-Value} & (0.19) & (0.18) & (0.59) & (0.05)*** & (0.09)** & (0.03)*** & (0.36) & (0.18) & (0.42) & (0.91) & (0.94) \\
\bottomrule
\end{tabular}
}
\vspace{1ex} \\
\footnotesize\raggedright{\underline{Note1 :} This table shows the estimates of the coefficient for attending Reggio Approach preschools from multiple methods. We compare Reggio Approach individuals with those who attended other preschools. Column title indicates the corresponding control set and and model. \textbf{None} = OLS estimate with no control variables. \textbf{BIC} = OLS estimate with controls selected by Bayesian Information Criterion (BIC) and additional controls for male indicator, migrant indicator, and ITC attendance indicator. \textbf{Full} = OLS estimate with the full set of controls. \textbf{PSMR} =  propensity score matching between Reggio Approach people and people in Reggio who attended nother types of preschool. \textbf{KMR} = Epanechinikov kernel matching between Reggio Approach people and people in Reggio who attended nother types of preschool. \textbf{DidPm} = difference-in-difference estimate of (Reggio Muni - Parma Muni) - (Reggio Other - Parma Other). \textbf{PSMPm} = propensity score matching between Reggio Approach people and people who attended Parma preschools. \textbf{KMPm} = Epanechinikov kernel matching between Reggio Approach people and people who attended Parma preschools. \textbf{DidPv} = difference-in-difference estimate of (Reggio Muni - Padova Muni) - (Reggio Other - Padova Other). \textbf{PSMPv} = propensity score matching between Reggio Approach people and people who attended Padova preschools. \textbf{KMPv} = Epanechinikov kernel matching between Reggio Approach people and people who attended Padova preschools.} 

\footnotesize\raggedright{\underline{Note2 :} Both unadjusted p-value and stepdown p-value are reported. ***, **, and * indicate significance of the coefficients at the 15\%, 10\%, and 5\% levels respectively.}
\end{table}

\begin{table}[H] \caption{Estimation Results for Employment Outcomes, Comparison to No Preschool,Age-30 Cohort} \label{ols-W-adult30-reg-none}
\scalebox{0.6}{\begin{tabular}{l c c c c c c c c c c c}
\toprule
& \multicolumn{5}{c}{Within Reggio} & \multicolumn{3}{c}{With Parma} & \multicolumn{3}{c}{With Padova} \\\cmidrule(lr){2-6} \cmidrule(lr){7-9} \cmidrule(lr){10-12}
 & None & BIC & Full & PSMR & KMR & DidPm & KMDidPm & KMPm & DidPv & KMDidPv & KMPv \\
\midrule
Employed & 0.04 & 0.02 & 0.04 & 0.05 & 0.01 & 0.13 & 0.15 & 0.01 & 0.01 & 0.01 & 0.05 \\
\quad \textit{Unadjusted P-Value} & (0.38) & (0.69) & (0.35) & (0.39) & (0.90) & (0.13)* & (0.01)*** & (0.83) & (0.95) & (0.92) & (0.31) \\
\quad \textit{Stepdown P-Value} & (0.67) & (0.80) & (0.89) & (0.80) & (0.93) & (0.48) & (0.12) & (0.99) & (0.99) & (0.98) & (0.68) \\
Self-Employed & -0.07 & -0.10 & -0.08 & -0.10 & -0.12 & -0.05 & -0.04 & -0.01 & -0.09 & -0.05 & 0.04 \\
\quad \textit{Unadjusted P-Value} & (0.25) & (0.11)* & (0.21) & (0.19) & (0.09)** & (0.62) & (0.82) & (0.89) & (0.21) & (0.60) & (0.37) \\
\quad \textit{Stepdown P-Value} & (0.67) & (0.43) & (0.51) & (0.65) & (0.39) & (0.63) & (0.92) & (0.99) & (0.96) & (0.98) & (0.68) \\
Hours Worked Per Week & 6.84 & 4.30 & 5.16 & 2.80 & 3.63 & 8.95 & 10.43 & 1.65 & 4.79 & 3.82 & 3.31 \\
\quad \textit{Unadjusted P-Value} & (0.01)*** & (0.12)* & (0.07)** & (0.34) & (0.33) & (0.07)** & (0.00)*** & (0.63) & (0.37) & (0.30) & (0.23) \\
\quad \textit{Stepdown P-Value} & (0.04)*** & (0.43) & (0.34) & (0.80) & (0.78) & (0.28) & (0.04)*** & (0.97) & (0.96) & (0.98) & (0.64) \\
Income: 5,000 Euros of Less & 0.07 & 0.08 & 0.07 & 0.07 & 0.07 & 0.05 & 0.07 & 0.07 & -0.01 & 0.02 & 0.07 \\
\quad \textit{Unadjusted P-Value} & (0.00)*** & (0.01)*** & (0.01)*** & (0.01)*** & (0.00)*** & (0.06)** & (0.00)*** & (0.00)*** & (0.91) & (0.77) & (0.00)*** \\
\quad \textit{Stepdown P-Value} & (0.23) & (0.10) & (0.27) & (0.05)*** & (0.03)*** & (0.63) & (0.02)*** & (0.04)*** & (0.99) & (0.98) & (0.06)** \\
Income: 5,001-10,000 Euros & -0.03 & -0.02 & -0.02 & -0.01 & -0.02 & 0.04 & 0.01 & -0.05 & -0.01 & -0.02 & 0.01 \\
\quad \textit{Unadjusted P-Value} & (0.32) & (0.49) & (0.32) & (0.63) & (0.52) & (0.36) & (0.81) & (0.29) & (0.67) & (0.38) & (0.32) \\
\quad \textit{Stepdown P-Value} & (0.67) & (0.80) & (0.79) & (0.86) & (0.83) & (0.63) & (0.92) & (0.84) & (0.96) & (0.96) & (0.68) \\
Income: 10,001-25,000 Euros & -0.21 & -0.21 & -0.21 & -0.16 & -0.21 & -0.42 & -0.22 & -0.09 & -0.08 & -0.11 & -0.25 \\
\quad \textit{Unadjusted P-Value} & (0.01)*** & (0.01)*** & (0.02)*** & (0.10)** & (0.02)*** & (0.00)*** & (0.14)* & (0.41) & (0.61) & (0.55) & (0.00)*** \\
\quad \textit{Stepdown P-Value} & (0.04)*** & (0.05)*** & (0.18) & (0.46) & (0.10)** & (0.02)*** & (0.45) & (0.94) & (0.96) & (0.98) & (0.06)** \\
Income: 25,001-50,000 Euros & 0.08 & 0.06 & 0.09 & 0.01 & 0.07 & 0.20 & 0.11 & -0.03 & 0.08 & 0.07 & 0.05 \\
\quad \textit{Unadjusted P-Value} & (0.33) & (0.45) & (0.31) & (0.89) & (0.47) & (0.16) & (0.49) & (0.77) & (0.63) & (0.73) & (0.56) \\
\quad \textit{Stepdown P-Value} & (0.67) & (0.80) & (0.68) & (0.86) & (0.83) & (0.52) & (0.85) & (0.99) & (0.96) & (0.98) & (0.68) \\
\bottomrule
\end{tabular}
}
\vspace{1ex} \\
\footnotesize\raggedright{\underline{Note1 :} This table shows the estimates of the coefficient for attending Reggio Approach preschools from multiple methods. We compare Reggio Approach individuals with those who did not attend any preschools. Column title indicates the corresponding control set and and model. \textbf{None} = OLS estimate with no control variables. \textbf{BIC} = OLS estimate with controls selected by Bayesian Information Criterion (BIC) and additional controls for male indicator, migrant indicator, and ITC attendance indicator. \textbf{Full} = OLS estimate with the full set of controls. \textbf{PSMR} =  propensity score matching between Reggio Approach people and people in Reggio who did not attend any preschool. \textbf{KMR} = Epanechinikov kernel matching between Reggio Approach people and people in Reggio who attended other types of preschool. \textbf{DidPm} = difference-in-difference estimate of (Reggio Muni - Parma Muni) - (Reggio None - Parma None). \textbf{PSMPm} = propensity score matching between Reggio Approach people and people in Parma who did not attend any preschool. \textbf{KMPm} = Epanechinikov kernel matching between Reggio Approach people and people in Parma who did not attend any preschool. \textbf{DidPv} = difference-in-difference estimate of (Reggio Muni - Padova Muni) - (Reggio None - Padova None). \textbf{PSMPv} = propensity score matching between Reggio Approach people and people in Padova who did not attend any preschool. \textbf{KMPv} = Epanechinikov kernel matching between Reggio Approach people and people in Padova who did not attend any preschool.} 

\footnotesize\raggedright{\underline{Note2 :} Both unadjusted p-value and stepdown p-value are reported. ***, **, and * indicate significance of the coefficients at the 15\%, 10\%, and 5\% levels respectively.}
\end{table}


\begin{table}[H] \caption{Estimation Results for Living Environment Outcomes, Comparison to Non-RA Preschools, Age-30 Cohort} \label{ols-L-adult30-reg-other}
\scalebox{0.6}{\begin{tabular}{l c c c c c c c c c c c}
\toprule
& \multicolumn{5}{c}{Within Reggio} & \multicolumn{3}{c}{With Parma} & \multicolumn{3}{c}{With Padova} \\\cmidrule(lr){2-6} \cmidrule(lr){7-9} \cmidrule(lr){10-12}
 & None & BIC & Full & PSMR & KMR & DidPm & KMDidPm & KMPm & DidPv & KMDidPv & KMPv \\
\midrule
Married or Cohabitating & 0.08 & 0.06 & 0.05 & -0.02 & -0.03 & 0.10 & 0.08 & -0.01 & 0.16 & 0.14 & -0.10 \\
\quad \textit{Unadjusted P-Value} & (0.29) & (0.46) & (0.52) & (0.85) & (0.73) & (0.40) & (0.60) & (0.91) & (0.26) & (0.37) & (0.12)* \\
\quad \textit{Stepdown P-Value} & (0.75) & (0.92) & (0.92) & (0.99) & (0.99) & (0.79) & (0.97) & (0.99) & (0.61) & (0.80) & (0.23) \\
Divorced & -0.03 & -0.03 & -0.03 & -0.02 & -0.01 & -0.05 & -0.03 & -0.01 & -0.02 & 0.00 & -0.02 \\
\quad \textit{Unadjusted P-Value} & (0.15) & (0.16) & (0.16) & (0.16) & (0.60) & (0.08)** & (0.19) & (0.51) & (0.53) & (0.87) & (0.10)* \\
\quad \textit{Stepdown P-Value} & (0.20) & (0.48) & (0.80) & (0.53) & (0.99) & (0.20) & (0.46) & (0.81) & (0.72) & (0.80) & (0.23) \\
Num. of Children in House & 0.00 & 0.01 & -0.01 & -0.05 & -0.04 & 0.01 & -0.06 & -0.07 & 0.19 & 0.13 & -0.18 \\
\quad \textit{Unadjusted P-Value} & (0.93) & (0.85) & (0.84) & (0.47) & (0.54) & (0.96) & (0.49) & (0.22) & (0.13)* & (0.41) & (0.01)*** \\
\quad \textit{Stepdown P-Value} & (0.96) & (0.96) & (0.99) & (0.92) & (0.99) & (0.94) & (0.97) & (0.51) & (0.61) & (0.80) & (0.07)** \\
Own House & 0.05 & 0.05 & 0.04 & 0.02 & 0.01 & 0.08 & -0.05 & 0.10 & 0.28 & 0.16 & -0.05 \\
\quad \textit{Unadjusted P-Value} & (0.53) & (0.47) & (0.58) & (0.83) & (0.95) & (0.50) & (0.66) & (0.17) & (0.06)** & (0.32) & (0.41) \\
\quad \textit{Stepdown P-Value} & (0.92) & (0.92) & (0.96) & (0.99) & (0.99) & (0.79) & (0.97) & (0.49) & (0.20) & (0.80) & (0.42) \\
Live With Parents & -0.02 & -0.01 & 0.04 & -0.01 & 0.02 & -0.12 & -0.05 & -0.01 & -0.09 & -0.06 & -0.14 \\
\quad \textit{Unadjusted P-Value} & (0.79) & (0.83) & (0.56) & (0.92) & (0.76) & (0.24) & (0.60) & (0.92) & (0.47) & (0.56) & (0.01)*** \\
\quad \textit{Stepdown P-Value} & (0.96) & (0.96) & (0.94) & (0.99) & (0.99) & (0.65) & (0.97) & (0.99) & (0.72) & (0.80) & (0.07)** \\
\bottomrule
\end{tabular}
}
\vspace{1ex} \\
\footnotesize\raggedright{\underline{Note1 :} This table shows the estimates of the coefficient for attending Reggio Approach preschools from multiple methods. We compare Reggio Approach individuals with those who attended other preschools. Column title indicates the corresponding control set and and model. \textbf{None} = OLS estimate with no control variables. \textbf{BIC} = OLS estimate with controls selected by Bayesian Information Criterion (BIC) and additional controls for male indicator, migrant indicator, and ITC attendance indicator. \textbf{Full} = OLS estimate with the full set of controls. \textbf{PSMR} =  propensity score matching between Reggio Approach people and people in Reggio who attended nother types of preschool. \textbf{KMR} = Epanechinikov kernel matching between Reggio Approach people and people in Reggio who attended nother types of preschool. \textbf{DidPm} = difference-in-difference estimate of (Reggio Muni - Parma Muni) - (Reggio Other - Parma Other). \textbf{PSMPm} = propensity score matching between Reggio Approach people and people who attended Parma preschools. \textbf{KMPm} = Epanechinikov kernel matching between Reggio Approach people and people who attended Parma preschools. \textbf{DidPv} = difference-in-difference estimate of (Reggio Muni - Padova Muni) - (Reggio Other - Padova Other). \textbf{PSMPv} = propensity score matching between Reggio Approach people and people who attended Padova preschools. \textbf{KMPv} = Epanechinikov kernel matching between Reggio Approach people and people who attended Padova preschools.} 

\footnotesize\raggedright{\underline{Note2 :} Both unadjusted p-value and stepdown p-value are reported. ***, **, and * indicate significance of the coefficients at the 15\%, 10\%, and 5\% levels respectively.}
\end{table}

\begin{table}[H] \caption{Estimation Results for Living Environment Outcomes, Comparison to No Preschool, Age-30 Cohort} \label{ols-L-adult30-reg-none}
\scalebox{0.6}{\begin{tabular}{l c c c c c c c c c c c}
\toprule
& \multicolumn{5}{c}{Within Reggio} & \multicolumn{3}{c}{With Parma} & \multicolumn{3}{c}{With Padova} \\\cmidrule(lr){2-6} \cmidrule(lr){7-9} \cmidrule(lr){10-12}
 & None & BIC & Full & PSMR & KMR & DidPm & KMDidPm & KMPm & DidPv & KMDidPv & KMPv \\
\midrule
Married or Cohabitating & -0.01 & -0.08 & -0.10 & -0.05 & -0.03 & -0.12 & & 0.05 & -0.03 & & -0.01 \\
\quad \textit{Unadjusted P-Value} & (0.85) & (0.33) & (0.25) & (0.56) & (0.73) & (0.37) & & (0.66) & (0.86) & & (0.90) \\
\quad \textit{Stepdown P-Value} & (0.97) & (0.68) & (0.63) & (0.94) & (0.98) & (0.74) & & (0.93) & (0.98) & & (0.91) \\
Divorced & 0 & 0 & 0 & 0 & 0 & 0.01 & & -0.01 & 0 & & 0 \\
\quad \textit{Unadjusted P-Value} & & & & & & (0.74) & & (0.66) & & & \\
\quad \textit{Stepdown P-Value} & 0 & 0 & .4142999947071075 & 0 & 0 & (0.74) & & (0.93) & 0 & & 0 \\
Num. of Children in House & -0.01 & -0.02 & -0.05 & 0.00 & -0.02 & -0.17 & & 0.03 & -0.01 & & -0.04 \\
\quad \textit{Unadjusted P-Value} & (0.79) & (0.71) & (0.39) & (0.96) & (0.73) & (0.21) & & (0.54) & (0.92) & & (0.56) \\
\quad \textit{Stepdown P-Value} & (0.92) & (0.86) & (0.82) & (0.96) & (0.98) & (0.45) & & (0.92) & (0.98) & & (0.79) \\
Own House & 0.06 & 0.09 & 0.13 & 0.03 & 0.02 & 0.11 & & 0.00 & 0.21 & & -0.08 \\
\quad \textit{Unadjusted P-Value} & (0.48) & (0.25) & (0.14)* & (0.76) & (0.86) & (0.42) & & (0.96) & (0.19) & & (0.36) \\
\quad \textit{Stepdown P-Value} & (0.92) & (0.66) & (0.51) & (0.96) & (0.98) & (0.74) & & (0.96) & (0.47) & & (0.73) \\
Live With Parents & -0.00 & -0.03 & -0.01 & -0.05 & -0.05 & -0.15 & & 0.10 & -0.08 & & -0.25 \\
\quad \textit{Unadjusted P-Value} & (0.97) & (0.62) & (0.87) & (0.51) & (0.46) & (0.16) & & (0.25) & (0.55) & & (0.00)*** \\
\quad \textit{Stepdown P-Value} & (0.97) & (0.86) & (0.97) & (0.94) & (0.92) & (0.59) & & (0.73) & (0.85) & & (0.01)*** \\
\bottomrule
\end{tabular}
}
\vspace{1ex} \\
\footnotesize\raggedright{\underline{Note1 :} This table shows the estimates of the coefficient for attending Reggio Approach preschools from multiple methods. We compare Reggio Approach individuals with those who did not attend any preschools. Column title indicates the corresponding control set and and model. \textbf{None} = OLS estimate with no control variables. \textbf{BIC} = OLS estimate with controls selected by Bayesian Information Criterion (BIC) and additional controls for male indicator, migrant indicator, and ITC attendance indicator. \textbf{Full} = OLS estimate with the full set of controls. \textbf{PSMR} =  propensity score matching between Reggio Approach people and people in Reggio who did not attend any preschool. \textbf{KMR} = Epanechinikov kernel matching between Reggio Approach people and people in Reggio who attended other types of preschool. \textbf{DidPm} = difference-in-difference estimate of (Reggio Muni - Parma Muni) - (Reggio None - Parma None). \textbf{PSMPm} = propensity score matching between Reggio Approach people and people in Parma who did not attend any preschool. \textbf{KMPm} = Epanechinikov kernel matching between Reggio Approach people and people in Parma who did not attend any preschool. \textbf{DidPv} = difference-in-difference estimate of (Reggio Muni - Padova Muni) - (Reggio None - Padova None). \textbf{PSMPv} = propensity score matching between Reggio Approach people and people in Padova who did not attend any preschool. \textbf{KMPv} = Epanechinikov kernel matching between Reggio Approach people and people in Padova who did not attend any preschool.} 

\footnotesize\raggedright{\underline{Note2 :} Both unadjusted p-value and stepdown p-value are reported. ***, **, and * indicate significance of the coefficients at the 15\%, 10\%, and 5\% levels respectively.}
\end{table}


\begin{table}[H] \caption{Estimation Results for Health and Risk Outcomes, Comparison to Non-RA Preschools, Age-30 Cohort} \label{ols-H-adult30-reg-other}
\scalebox{0.6}{\begin{tabular}{l c c c c c c c}
\toprule
 & None & BIC & Full & PSM & DidPm & DidPv \\
\midrule
Tried Marijuana & -0.10 & -0.10 & -0.10 & -0.06 & -0.12 & -0.15 \\
\quad \textit{Unadjusted P-Value} & (0.12) & (0.13) & (0.09) & (0.50) & (0.22) & (0.22) \\
\quad \textit{Stepdown P-Value} & (0.63) & (0.68) & (0.63) & (0.99) & (0.86) & (0.79) \\
Num. of Cigarettes Per Day & 0.24 & 1.32 & 1.46 & 0.79 & 0.18 & 2.48 \\
\quad \textit{Unadjusted P-Value} & (0.85) & (0.30) & (0.28) & (0.56) & (0.94) & (0.55) \\
\quad \textit{Stepdown P-Value} & (0.99) & (0.96) & (0.78) & (0.99) & (0.99) & (0.94) \\
BMI & 0.16 & -0.07 & -0.02 & 0.01 & -1.55 & 1.02 \\
\quad \textit{Unadjusted P-Value} & (0.68) & (0.86) & (0.95) & (0.98) & (0.01) & (0.12) \\
\quad \textit{Stepdown P-Value} & (0.99) & (0.99) & (0.99) & (0.99) & (0.06) & (0.79) \\
Not Obese & 0.01 & 0.00 & 0.03 & -0.03 & 0.01 & -0.03 \\
\quad \textit{Unadjusted P-Value} & (0.87) & (0.95) & (0.61) & (0.76) & (0.94) & (0.81) \\
\quad \textit{Stepdown P-Value} & (0.99) & (0.99) & (0.95) & (0.99) & (0.99) & (0.98) \\
Not Overweight & -0.06 & -0.01 & -0.02 & 0.04 & 0.07 & -0.04 \\
\quad \textit{Unadjusted P-Value} & (0.40) & (0.89) & (0.81) & (0.58) & (0.46) & (0.70) \\
\quad \textit{Stepdown P-Value} & (0.96) & (0.99) & (0.99) & (0.99) & (0.98) & (0.98) \\
Good Health & -0.08 & -0.07 & -0.08 & -0.08 & -0.12 & -0.07 \\
\quad \textit{Unadjusted P-Value} & (0.39) & (0.40) & (0.39) & (0.29) & (0.41) & (0.71) \\
\quad \textit{Stepdown P-Value} & (0.96) & (0.99) & (0.79) & (0.96) & (0.98) & (0.98) \\
No Problematic Health Condition & 0.00 & -0.02 & -0.05 & -0.03 & 0.08 & 0.00 \\
\quad \textit{Unadjusted P-Value} & (0.97) & (0.84) & (0.54) & (0.77) & (0.55) & (1.00) \\
\quad \textit{Stepdown P-Value} & (0.99) & (0.99) & (0.92) & (0.99) & (0.98) & (0.98) \\
Num. of Days Sick Past Month & 0.12 & 0.15 & 0.20 & 0.21 & 0.20 & 0.23 \\
\quad \textit{Unadjusted P-Value} & (0.14) & (0.05) & (0.01) & (0.00) & (0.04) & (0.05) \\
\quad \textit{Stepdown P-Value} & (0.86) & (0.68) & (0.55) & (0.03) & (0.71) & (0.79) \\
Ever Suspended from School & -0.01 & -0.02 & -0.03 & -0.01 & 0.02 & -0.11 \\
\quad \textit{Unadjusted P-Value} & (0.72) & (0.63) & (0.54) & (0.79) & (0.75) & (0.19) \\
\quad \textit{Stepdown P-Value} & (0.99) & (0.99) & (0.86) & (0.99) & (0.98) & (0.57) \\
Age At First Drink & 0.52 & 0.19 & 0.74 & -0.01 & -0.56 & -1.76 \\
\quad \textit{Unadjusted P-Value} & (0.70) & (0.87) & (0.50) & (1.00) & (0.75) & (0.36) \\
\quad \textit{Stepdown P-Value} & (0.96) & (0.99) & (0.88) & (0.99) & (0.98) & (0.93) \\
\bottomrule
\end{tabular}
}
\vspace{1ex} \\
\footnotesize\raggedright{\underline{Note1 :} This table shows the estimates of the coefficient for attending Reggio Approach preschools from multiple methods. We compare Reggio Approach individuals with those who attended other preschools. Column title indicates the corresponding control set and and model. \textbf{None} = OLS estimate with no control variables. \textbf{BIC} = OLS estimate with controls selected by Bayesian Information Criterion (BIC) and additional controls for male indicator, migrant indicator, and ITC attendance indicator. \textbf{Full} = OLS estimate with the full set of controls. \textbf{PSMR} =  propensity score matching between Reggio Approach people and people in Reggio who attended nother types of preschool. \textbf{KMR} = Epanechinikov kernel matching between Reggio Approach people and people in Reggio who attended nother types of preschool. \textbf{DidPm} = difference-in-difference estimate of (Reggio Muni - Parma Muni) - (Reggio Other - Parma Other). \textbf{PSMPm} = propensity score matching between Reggio Approach people and people who attended Parma preschools. \textbf{KMPm} = Epanechinikov kernel matching between Reggio Approach people and people who attended Parma preschools. \textbf{DidPv} = difference-in-difference estimate of (Reggio Muni - Padova Muni) - (Reggio Other - Padova Other). \textbf{PSMPv} = propensity score matching between Reggio Approach people and people who attended Padova preschools. \textbf{KMPv} = Epanechinikov kernel matching between Reggio Approach people and people who attended Padova preschools.} 

\footnotesize\raggedright{\underline{Note2 :} Both unadjusted p-value and stepdown p-value are reported. ***, **, and * indicate significance of the coefficients at the 15\%, 10\%, and 5\% levels respectively.}
\end{table}

\begin{table}[H] \caption{Estimation Results for Health and Risk Outcomes, Comparison to No Preschool, Age-30 Cohort} \label{ols-H-adult30-reg-none}
\scalebox{0.6}{\begin{tabular}{l c c c c c c c}
\toprule
 & None & BIC & Full & PSM & DidPm & DidPv \\
\midrule
Tried Marijuana & 0.05 & 0.05 & 0.03 & 0.04 & -0.05 & -0.11 \\
\quad \textit{Unadjusted P-Value} & (0.36) & (0.30) & (0.63) & (0.53) & (0.52) & (0.32) \\
\quad \textit{Stepdown P-Value} & (0.74) & (0.80) & (0.87) & (0.90) & (0.98) & (0.82) \\
Num. of Cigarettes Per Day & 0.85 & 0.86 & 1.13 & 1.04 & -0.59 & 1.65 \\
\quad \textit{Unadjusted P-Value} & (0.51) & (0.51) & (0.53) & (0.41) & (0.82) & (0.67) \\
\quad \textit{Stepdown P-Value} & (0.75) & (0.90) & (0.82) & (0.89) & (0.98) & (0.87) \\
BMI & 1.06 & 0.59 & 0.69 & 0.51 & -0.18 & 1.68 \\
\quad \textit{Unadjusted P-Value} & (0.03) & (0.15) & (0.10) & (0.33) & (0.77) & (0.03) \\
\quad \textit{Stepdown P-Value} & (0.14) & (0.53) & (0.37) & (0.88) & (0.98) & (0.17) \\
Not Obese & -0.00 & -0.06 & -0.06 & -0.09 & 0.03 & -0.25 \\
\quad \textit{Unadjusted P-Value} & (0.99) & (0.30) & (0.32) & (0.18) & (0.81) & (0.07) \\
\quad \textit{Stepdown P-Value} & (0.99) & (0.80) & (0.66) & (0.70) & (0.98) & (0.45) \\
Not Overweight & -0.07 & 0.01 & -0.02 & 0.03 & -0.00 & 0.00 \\
\quad \textit{Unadjusted P-Value} & (0.29) & (0.87) & (0.78) & (0.66) & (0.98) & (0.98) \\
\quad \textit{Stepdown P-Value} & (0.74) & (0.92) & (0.94) & (0.90) & (0.98) & (0.96) \\
Good Health & 0.21 & 0.16 & 0.15 & 0.15 & 0.28 & 0.20 \\
\quad \textit{Unadjusted P-Value} & (0.01) & (0.03) & (0.05) & (0.05) & (0.06) & (0.32) \\
\quad \textit{Stepdown P-Value} & (0.12) & (0.37) & (0.39) & (0.34) & (0.52) & (0.82) \\
No Problematic Health Condition & -0.24 & -0.23 & -0.21 & -0.19 & -0.15 & -0.22 \\
\quad \textit{Unadjusted P-Value} & (0.00) & (0.00) & (0.01) & (0.06) & (0.24) & (0.19) \\
\quad \textit{Stepdown P-Value} & (0.02) & (0.05) & (0.27) & (0.38) & (0.87) & (0.73) \\
Num. of Days Sick Past Month & 0.18 & 0.21 & 0.18 & 0.19 & 0.17 & 0.16 \\
\quad \textit{Unadjusted P-Value} & (0.05) & (0.02) & (0.03) & (0.08) & (0.06) & (0.17) \\
\quad \textit{Stepdown P-Value} & (0.39) & (0.33) & (0.46) & (0.41) & (0.86) & (0.82) \\
Ever Suspended from School & -0.10 & -0.12 & -0.15 & -0.17 & -0.11 & -0.15 \\
\quad \textit{Unadjusted P-Value} & (0.05) & (0.03) & (0.01) & (0.02) & (0.14) & (0.15) \\
\quad \textit{Stepdown P-Value} & (0.16) & (0.16) & (0.19) & (0.21) & (0.69) & (0.55) \\
Age At First Drink & 1.95 & 0.44 & -0.20 & 0.50 & 1.40 & -2.64 \\
\quad \textit{Unadjusted P-Value} & (0.16) & (0.73) & (0.87) & (0.71) & (0.47) & (0.25) \\
\quad \textit{Stepdown P-Value} & (0.51) & (0.92) & (0.97) & (0.90) & (0.98) & (0.82) \\
\bottomrule
\end{tabular}
}
\vspace{1ex} \\
\footnotesize\raggedright{\underline{Note1 :} This table shows the estimates of the coefficient for attending Reggio Approach preschools from multiple methods. We compare Reggio Approach individuals with those who did not attend any preschools. Column title indicates the corresponding control set and and model. \textbf{None} = OLS estimate with no control variables. \textbf{BIC} = OLS estimate with controls selected by Bayesian Information Criterion (BIC) and additional controls for male indicator, migrant indicator, and ITC attendance indicator. \textbf{Full} = OLS estimate with the full set of controls. \textbf{PSMR} =  propensity score matching between Reggio Approach people and people in Reggio who did not attend any preschool. \textbf{KMR} = Epanechinikov kernel matching between Reggio Approach people and people in Reggio who attended other types of preschool. \textbf{DidPm} = difference-in-difference estimate of (Reggio Muni - Parma Muni) - (Reggio None - Parma None). \textbf{PSMPm} = propensity score matching between Reggio Approach people and people in Parma who did not attend any preschool. \textbf{KMPm} = Epanechinikov kernel matching between Reggio Approach people and people in Parma who did not attend any preschool. \textbf{DidPv} = difference-in-difference estimate of (Reggio Muni - Padova Muni) - (Reggio None - Padova None). \textbf{PSMPv} = propensity score matching between Reggio Approach people and people in Padova who did not attend any preschool. \textbf{KMPv} = Epanechinikov kernel matching between Reggio Approach people and people in Padova who did not attend any preschool.} 

\footnotesize\raggedright{\underline{Note2 :} Both unadjusted p-value and stepdown p-value are reported. ***, **, and * indicate significance of the coefficients at the 15\%, 10\%, and 5\% levels respectively.}
\end{table}


\begin{table}[H] \caption{Estimation Results for Noncognitive Outcomes, Comparison to Non-RA Preschools, Age-30 Cohort} \label{ols-N-adult30-reg-other}
\scalebox{0.6}{\begin{tabular}{l c c c c c c c c c c c}
\toprule
& \multicolumn{5}{c}{Within Reggio} & \multicolumn{3}{c}{With Parma} & \multicolumn{3}{c}{With Padova} \\\cmidrule(lr){2-6} \cmidrule(lr){7-9} \cmidrule(lr){10-12}
 & None & BIC & Full & PSMR & KMR & DidPm & PSMPm & KMPm & DidPv & PSMPv & KMPv \\
\midrule
Locus of Control - positive & 0.11 & 0.08 & 0.06 & 0.07 & 0.09 & 0.31 & 0.22 & 0.22 & 0.16 & -0.28 & -0.22 \\
\quad \textit{Unadjusted P-Value} & (0.40) & (0.49) & (0.59) & (0.60) & (0.52) & (0.16) & (0.03)*** & (0.08)** & (0.52) & (0.00)*** & (0.04)*** \\
\quad \textit{Stepdown P-Value} & (0.98) & (0.99) & (0.88) & (0.99) & (0.99) & (0.73) & (0.20) & (0.41) & (0.94) & (0.04)*** & (0.24) \\
Depression Score - positive & 0.16 & -0.03 & 0.04 & -0.29 & -0.32 & 1.24 & -2.53 & -1.71 & -0.21 & -3.21 & -2.32 \\
\quad \textit{Unadjusted P-Value} & (0.87) & (0.97) & (0.96) & (0.74) & (0.79) & (0.39) & (0.00)*** & (0.05)*** & (0.91) & (0.00)*** & (0.00)*** \\
\quad \textit{Stepdown P-Value} & (0.99) & (0.99) & (0.99) & (0.99) & (0.99) & (0.94) & (0.03)*** & (0.33) & (0.95) & (0.00)*** & (0.02)*** \\
Stress & -0.13 & -0.12 & -0.15 & -0.24 & -0.10 & -0.18 & 0.10 & 0.16 & -0.46 & -0.04 & 0.05 \\
\quad \textit{Unadjusted P-Value} & (0.28) & (0.20) & (0.10)* & (0.02)*** & (0.44) & (0.36) & (0.32) & (0.12)* & (0.02)*** & (0.67) & (0.57) \\
\quad \textit{Stepdown P-Value} & (0.90) & (0.84) & (0.69) & (0.24) & (0.99) & (0.94) & (0.79) & (0.54) & (0.19) & (0.98) & (0.90) \\
Work is Source of Stress & -0.06 & -0.06 & -0.03 & -0.14 & -0.16 & 0.11 & 0.08 & 0.11 & 0.16 & 0.08 & 0.04 \\
\quad \textit{Unadjusted P-Value} & (0.57) & (0.65) & (0.81) & (0.48) & (0.43) & (0.55) & (0.34) & (0.21) & (0.49) & (0.37) & (0.69) \\
\quad \textit{Stepdown P-Value} & (0.99) & (0.99) & (0.97) & (0.99) & (0.99) & (0.95) & (0.79) & (0.57) & (0.94) & (0.92) & (0.90) \\
Satisfied with Income & 0.03 & 0.02 & 0.04 & 0.03 & 0.03 & 0.55 & 0.50 & 0.50 & -0.06 & 0.12 & 0.18 \\
\quad \textit{Unadjusted P-Value} & (0.80) & (0.88) & (0.77) & (0.80) & (0.81) & (0.01)*** & (0.00)*** & (0.00)*** & (0.82) & (0.33) & (0.09)** \\
\quad \textit{Stepdown P-Value} & (0.99) & (0.99) & (0.96) & (0.99) & (0.99) & (0.05)*** & (0.00)*** & (0.00)*** & (0.95) & (0.91) & (0.50) \\
Satisfied with Work & 0.14 & 0.11 & 0.11 & 0.04 & 0.07 & 0.45 & 0.35 & 0.39 & 0.23 & 0.02 & 0.08 \\
\quad \textit{Unadjusted P-Value} & (0.29) & (0.36) & (0.44) & (0.73) & (0.65) & (0.09)** & (0.02)*** & (0.01)*** & (0.38) & (0.86) & (0.49) \\
\quad \textit{Stepdown P-Value} & (0.91) & (0.98) & (0.81) & (0.99) & (0.99) & (0.45) & (0.15) & (0.05)** & (0.94) & (0.98) & (0.90) \\
Satisfied with Health & -0.18 & -0.21 & -0.22 & -0.19 & -0.19 & -0.12 & -0.14 & -0.08 & -0.56 & -0.04 & 0.06 \\
\quad \textit{Unadjusted P-Value} & (0.11)* & (0.05)*** & (0.04)*** & (0.05)** & (0.12)* & (0.46) & (0.17) & (0.41) & (0.00)*** & (0.72) & (0.53) \\
\quad \textit{Stepdown P-Value} & (0.75) & (0.53) & (0.55) & (0.46) & (0.72) & (0.95) & (0.59) & (0.79) & (0.07)** & (0.98) & (0.90) \\
Satisfied with Family & 0.06 & 0.02 & -0.01 & -0.03 & -0.08 & 0.53 & -0.08 & -0.07 & 0.60 & -0.23 & -0.27 \\
\quad \textit{Unadjusted P-Value} & (0.67) & (0.87) & (0.97) & (0.79) & (0.62) & (0.02)*** & (0.53) & (0.56) & (0.05)** & (0.08)** & (0.03)*** \\
\quad \textit{Stepdown P-Value} & (0.99) & (0.99) & (0.99) & (0.99) & (0.99) & (0.08)** & (0.79) & (0.79) & (0.19) & (0.45) & (0.17) \\
Optimistic Look in Life & 0.14 & 0.13 & 0.13 & 0.07 & 0.14 & 0.05 & -0.06 & -0.06 & -0.12 & -0.03 & -0.09 \\
\quad \textit{Unadjusted P-Value} & (0.11)* & (0.11)* & (0.12)* & (0.47) & (0.11)* & (0.69) & (0.46) & (0.43) & (0.33) & (0.75) & (0.23) \\
\quad \textit{Stepdown P-Value} & (0.71) & (0.63) & (0.63) & (0.46) & (0.72) & (0.95) & (0.79) & (0.79) & (0.94) & (0.98) & (0.79) \\
Positive Reciprocity & -0.05 & -0.02 & 0.04 & -0.04 & -0.06 & -0.15 & -0.16 & -0.12 & -0.18 & -0.08 & -0.09 \\
\quad \textit{Unadjusted P-Value} & (0.61) & (0.85) & (0.63) & (0.66) & (0.61) & (0.28) & (0.04)*** & (0.17) & (0.39) & (0.42) & (0.35) \\
\quad \textit{Stepdown P-Value} & (0.99) & (0.99) & (0.93) & (0.99) & (0.99) & (0.94) & (0.21) & (0.57) & (0.94) & (0.94) & (0.84) \\
Negative Reciprocity & -0.08 & -0.04 & -0.04 & 0.03 & -0.06 & -0.04 & 0.56 & 0.54 & 0.37 & 0.66 & 0.55 \\
\quad \textit{Unadjusted P-Value} & (0.61) & (0.78) & (0.81) & (0.88) & (0.71) & (0.86) & (0.00)*** & (0.00)*** & (0.26) & (0.00)*** & (0.00)*** \\
\quad \textit{Stepdown P-Value} & (0.99) & (0.99) & (0.98) & (0.99) & (0.99) & (0.95) & (0.00)*** & (0.00)*** & (0.84) & (0.00)*** & (0.00)*** \\
\bottomrule
\end{tabular}
}
\vspace{1ex} \\
\footnotesize\raggedright{\underline{Note1 :} This table shows the estimates of the coefficient for attending Reggio Approach preschools from multiple methods. We compare Reggio Approach individuals with those who attended other preschools. Column title indicates the corresponding control set and and model. \textbf{None} = OLS estimate with no control variables. \textbf{BIC} = OLS estimate with controls selected by Bayesian Information Criterion (BIC) and additional controls for male indicator, migrant indicator, and ITC attendance indicator. \textbf{Full} = OLS estimate with the full set of controls. \textbf{PSMR} =  propensity score matching between Reggio Approach people and people in Reggio who attended nother types of preschool. \textbf{KMR} = Epanechinikov kernel matching between Reggio Approach people and people in Reggio who attended nother types of preschool. \textbf{DidPm} = difference-in-difference estimate of (Reggio Muni - Parma Muni) - (Reggio Other - Parma Other). \textbf{PSMPm} = propensity score matching between Reggio Approach people and people who attended Parma preschools. \textbf{KMPm} = Epanechinikov kernel matching between Reggio Approach people and people who attended Parma preschools. \textbf{DidPv} = difference-in-difference estimate of (Reggio Muni - Padova Muni) - (Reggio Other - Padova Other). \textbf{PSMPv} = propensity score matching between Reggio Approach people and people who attended Padova preschools. \textbf{KMPv} = Epanechinikov kernel matching between Reggio Approach people and people who attended Padova preschools.} 

\footnotesize\raggedright{\underline{Note2 :} Both unadjusted p-value and stepdown p-value are reported. ***, **, and * indicate significance of the coefficients at the 15\%, 10\%, and 5\% levels respectively.}
\end{table}

\begin{table}[H] \caption{Estimation Results for Noncognitive Outcomes, Comparison to No Preschool, Age-30 Cohort} \label{ols-N-adult30-reg-none}
\scalebox{0.6}{\begin{tabular}{l c c c c c c c}
\toprule
 & None & BIC & Full & PSM & DidPm & DidPv \\
\midrule
Locus of Control - positive & 0.07 & -0.05 & -0.08 & -0.11 & -0.15 & 0.04 \\
\quad \textit{Unadjusted P-Value} & (0.59) & (0.71) & (0.56) & (0.34) & (0.56) & (0.88) \\
\quad \textit{Stepdown P-Value} & (0.99) & (0.99) & (0.77) & (0.95) & (0.98) & (0.99) \\
Depression Score - positive & 1.26 & -0.04 & -0.20 & 0.37 & 0.74 & -0.53 \\
\quad \textit{Unadjusted P-Value} & (0.20) & (0.97) & (0.83) & (0.70) & (0.64) & (0.78) \\
\quad \textit{Stepdown P-Value} & (0.80) & (0.99) & (0.95) & (0.99) & (0.98) & (0.94) \\
Stress & 0.09 & 0.04 & 0.05 & 0.01 & -0.00 & -0.19 \\
\quad \textit{Unadjusted P-Value} & (0.44) & (0.71) & (0.67) & (0.96) & (0.99) & (0.37) \\
\quad \textit{Stepdown P-Value} & (0.96) & (0.99) & (0.84) & (0.99) & (0.99) & (0.94) \\
Work is Source of Stress & 0.05 & -0.01 & 0.02 & -0.02 & 0.30 & -0.01 \\
\quad \textit{Unadjusted P-Value} & (0.66) & (0.94) & (0.84) & (0.88) & (0.09) & (0.96) \\
\quad \textit{Stepdown P-Value} & (0.99) & (0.99) & (0.97) & (0.99) & (0.54) & (0.99) \\
Satisfied with Income & 0.29 & 0.30 & 0.30 & 0.14 & 0.60 & 0.25 \\
\quad \textit{Unadjusted P-Value} & (0.05) & (0.04) & (0.05) & (0.43) & (0.01) & (0.39) \\
\quad \textit{Stepdown P-Value} & (0.25) & (0.26) & (0.22) & (0.98) & (0.09) & (0.94) \\
Satisfied with Work & 0.08 & 0.06 & 0.10 & 0.05 & 0.30 & -0.03 \\
\quad \textit{Unadjusted P-Value} & (0.56) & (0.69) & (0.54) & (0.78) & (0.30) & (0.91) \\
\quad \textit{Stepdown P-Value} & (0.99) & (0.99) & (0.76) & (0.99) & (0.89) & (0.99) \\
Satisfied with Health & -0.04 & -0.07 & -0.08 & 0.02 & 0.03 & -0.47 \\
\quad \textit{Unadjusted P-Value} & (0.75) & (0.57) & (0.52) & (0.87) & (0.86) & (0.01) \\
\quad \textit{Stepdown P-Value} & (0.99) & (0.99) & (0.77) & (0.99) & (0.99) & (0.25) \\
Satisfied with Family & -0.04 & -0.11 & -0.09 & -0.07 & -0.04 & 0.05 \\
\quad \textit{Unadjusted P-Value} & (0.77) & (0.43) & (0.54) & (0.65) & (0.86) & (0.87) \\
\quad \textit{Stepdown P-Value} & (0.99) & (0.99) & (0.78) & (0.99) & (0.99) & (0.94) \\
Optimistic Look in Life & -0.19 & -0.18 & -0.15 & -0.19 & -0.38 & -0.50 \\
\quad \textit{Unadjusted P-Value} & (0.02) & (0.03) & (0.10) & (0.03) & (0.01) & (0.00) \\
\quad \textit{Stepdown P-Value} & (0.18) & (0.24) & (0.31) & (0.24) & (0.08) & (0.03) \\
Positive Reciprocity & 0.02 & -0.05 & -0.04 & -0.10 & -0.04 & -0.29 \\
\quad \textit{Unadjusted P-Value} & (0.85) & (0.64) & (0.74) & (0.30) & (0.80) & (0.25) \\
\quad \textit{Stepdown P-Value} & (0.99) & (0.99) & (0.88) & (0.95) & (0.99) & (0.85) \\
Negative Reciprocity & 0.41 & 0.43 & 0.45 & 0.48 & 0.72 & 0.57 \\
\quad \textit{Unadjusted P-Value} & (0.02) & (0.01) & (0.01) & (0.01) & (0.01) & (0.11) \\
\quad \textit{Stepdown P-Value} & (0.12) & (0.08) & (0.14) & (0.11) & (0.08) & (0.50) \\
\bottomrule
\end{tabular}
}
\vspace{1ex} \\
\footnotesize\raggedright{\underline{Note1 :} This table shows the estimates of the coefficient for attending Reggio Approach preschools from multiple methods. We compare Reggio Approach individuals with those who did not attend any preschools. Column title indicates the corresponding control set and and model. \textbf{None} = OLS estimate with no control variables. \textbf{BIC} = OLS estimate with controls selected by Bayesian Information Criterion (BIC) and additional controls for male indicator, migrant indicator, and ITC attendance indicator. \textbf{Full} = OLS estimate with the full set of controls. \textbf{PSMR} =  propensity score matching between Reggio Approach people and people in Reggio who did not attend any preschool. \textbf{KMR} = Epanechinikov kernel matching between Reggio Approach people and people in Reggio who attended other types of preschool. \textbf{DidPm} = difference-in-difference estimate of (Reggio Muni - Parma Muni) - (Reggio None - Parma None). \textbf{PSMPm} = propensity score matching between Reggio Approach people and people in Parma who did not attend any preschool. \textbf{KMPm} = Epanechinikov kernel matching between Reggio Approach people and people in Parma who did not attend any preschool. \textbf{DidPv} = difference-in-difference estimate of (Reggio Muni - Padova Muni) - (Reggio None - Padova None). \textbf{PSMPv} = propensity score matching between Reggio Approach people and people in Padova who did not attend any preschool. \textbf{KMPv} = Epanechinikov kernel matching between Reggio Approach people and people in Padova who did not attend any preschool.} 

\footnotesize\raggedright{\underline{Note2 :} Both unadjusted p-value and stepdown p-value are reported. ***, **, and * indicate significance of the coefficients at the 15\%, 10\%, and 5\% levels respectively.}
\end{table}


\begin{table}[H] \caption{Estimation Results for Social Outcomes, Comparison to Non-RA Preschools, Age-30 Cohort} \label{ols-S-adult30-reg-other}
\scalebox{0.6}{\begin{tabular}{l c c c c c c c c c c c}
\toprule
 & None & BIC & Full & PSMR & KMR & DidPm & PSMPm & KMPm & DidPv & PSMPv & KMPv \\
\midrule
Bothered by Migrants & 0.15 & 0.16 & 0.14 & 0.13 & 0.19 & 0.11 & 0.04 & 0.13 & 0.12 & 0.26 & 0.33 \\
\quad \textit{Unadjusted P-Value} & (0.08)** & (0.06)** & (0.10)* & (0.09)** & (0.05)*** & (0.56) & (0.72) & (0.20) & (0.60) & (0.00)*** & (0.00)*** \\
\quad \textit{Stepdown P-Value} & (0.29) & (0.27) & (0.74) & (0.39) & (0.18) & (0.93) & (0.90) & (0.34) & (0.81) & (0.01)*** & (0.00)*** \\
Num. of Friends & 0.73 & 0.62 & 0.86 & 1.25 & 0.41 & 4.67 & -1.96 & -2.74 & 1.83 & 1.31 & -0.53 \\
\quad \textit{Unadjusted P-Value} & (0.45) & (0.60) & (0.53) & (0.52) & (0.72) & (0.01)*** & (0.23) & (0.06)** & (0.33) & (0.57) & (0.58) \\
\quad \textit{Stepdown P-Value} & (0.63) & (0.85) & (0.92) & (0.88) & (0.97) & (0.09)** & (0.60) & (0.21) & (0.73) & (0.62) & (0.92) \\
Has Migrant Friends & 0.13 & 0.14 & 0.12 & 0.07 & 0.11 & 0.17 & 0.02 & 0.02 & 0.25 & 0.07 & 0.09 \\
\quad \textit{Unadjusted P-Value} & (0.09)** & (0.06)** & (0.13)* & (0.32) & (0.18) & (0.16) & (0.69) & (0.69) & (0.08)** & (0.27) & (0.13)* \\
\quad \textit{Stepdown P-Value} & (0.29) & (0.27) & (0.80) & (0.79) & (0.53) & (0.51) & (0.90) & (0.68) & (0.27) & (0.52) & (0.47) \\
Volunteers & 0.11 & 0.10 & 0.11 & 0.10 & 0.11 & -0.06 & -0.19 & -0.14 & -0.01 & -0.13 & -0.12 \\
\quad \textit{Unadjusted P-Value} & (0.00)*** & (0.00)*** & (0.00)*** & (0.00)*** & (0.00)*** & (0.50) & (0.00)*** & (0.01)*** & (0.94) & (0.00)*** & (0.01)*** \\
\quad \textit{Stepdown P-Value} & (0.05)*** & (0.05)** & (0.68) & (0.00)*** & (0.00)*** & (0.93) & (0.00)*** & (0.05)** & (0.93) & (0.02)*** & (0.05)*** \\
Ever Voted for Municipal & -0.07 & -0.03 & -0.02 & 0.04 & 0.02 & -0.05 & 0.09 & 0.12 & 0.19 & -0.10 & -0.04 \\
\quad \textit{Unadjusted P-Value} & (0.36) & (0.66) & (0.77) & (0.51) & (0.82) & (0.61) & (0.12)* & (0.07)** & (0.11)* & (0.12)* & (0.55) \\
\quad \textit{Stepdown P-Value} & (0.63) & (0.85) & (0.99) & (0.88) & (0.97) & (0.93) & (0.42) & (0.21) & (0.49) & (0.41) & (0.92) \\
Ever Voted for Regional & -0.11 & -0.08 & -0.07 & -0.02 & -0.04 & -0.05 & 0.13 & 0.15 & 0.26 & -0.09 & -0.04 \\
\quad \textit{Unadjusted P-Value} & (0.18) & (0.23) & (0.29) & (0.71) & (0.66) & (0.64) & (0.06)** & (0.02)*** & (0.02)*** & (0.25) & (0.55) \\
\quad \textit{Stepdown P-Value} & (0.44) & (0.62) & (0.87) & (0.88) & (0.97) & (0.93) & (0.25) & (0.13) & (0.27) & (0.52) & (0.92) \\
\bottomrule
\end{tabular}
}
\vspace{1ex} \\
\footnotesize\raggedright{\underline{Note1 :} This table shows the estimates of the coefficient for attending Reggio Approach preschools from multiple methods. We compare Reggio Approach individuals with those who attended other preschools. Column title indicates the corresponding control set and and model. \textbf{None} = OLS estimate with no control variables. \textbf{BIC} = OLS estimate with controls selected by Bayesian Information Criterion (BIC) and additional controls for male indicator, migrant indicator, and ITC attendance indicator. \textbf{Full} = OLS estimate with the full set of controls. \textbf{PSMR} =  propensity score matching between Reggio Approach people and people in Reggio who attended nother types of preschool. \textbf{KMR} = Epanechinikov kernel matching between Reggio Approach people and people in Reggio who attended nother types of preschool. \textbf{DidPm} = difference-in-difference estimate of (Reggio Muni - Parma Muni) - (Reggio Other - Parma Other). \textbf{PSMPm} = propensity score matching between Reggio Approach people and people who attended Parma preschools. \textbf{KMPm} = Epanechinikov kernel matching between Reggio Approach people and people who attended Parma preschools. \textbf{DidPv} = difference-in-difference estimate of (Reggio Muni - Padova Muni) - (Reggio Other - Padova Other). \textbf{PSMPv} = propensity score matching between Reggio Approach people and people who attended Padova preschools. \textbf{KMPv} = Epanechinikov kernel matching between Reggio Approach people and people who attended Padova preschools.} 

\footnotesize\raggedright{\underline{Note2 :} Both unadjusted p-value and stepdown p-value are reported. ***, **, and * indicate significance of the coefficients at the 15\%, 10\%, and 5\% levels respectively.}
\end{table}

\begin{table}[H] \caption{Estimation Results for Social Outcomes, Comparison to No Preschool, Age-30 Cohort} \label{ols-S-adult30-reg-none}
\scalebox{0.6}{\begin{tabular}{l c c c c c c c c c c c}
\toprule
& \multicolumn{5}{c}{Within Reggio} & \multicolumn{3}{c}{With Parma} & \multicolumn{3}{c}{With Padova} \\\cmidrule(lr){2-6} \cmidrule(lr){7-9} \cmidrule(lr){10-12}
 & None & BIC & Full & PSMR & KMR & DidPm & KMDidPm & KMPm & DidPv & KMDidPv & KMPv \\
\midrule
Bothered by Migrants & -0.05 & -0.04 & -0.05 & -0.00 & 0.00 & 0.04 & & 0.10 & -0.08 & & 0.43 \\
\quad \textit{Unadjusted P-Value} & (0.55) & (0.65) & (0.62) & (0.97) & (0.99) & (0.85) & & (0.52) & (0.74) & & (0.00)*** \\
\quad \textit{Stepdown P-Value} & (0.89) & (0.98) & (0.82) & (0.94) & (0.98) & (0.89) & & (0.49) & (0.96) & & (0.00)*** \\
Num. of Friends & 0.02 & 0.24 & 0.20 & 0.02 & -0.41 & 2.16 & & -2.69 & 4.48 & & -1.20 \\
\quad \textit{Unadjusted P-Value} & (0.99) & (0.88) & (0.91) & (0.99) & (0.81) & (0.31) & & (0.14)* & (0.08)** & & (0.50) \\
\quad \textit{Stepdown P-Value} & (0.98) & (0.98) & (0.93) & (0.99) & (0.98) & (0.85) & & (0.27) & (0.41) & & (0.70) \\
Has Migrant Friends & 0.10 & 0.11 & 0.07 & 0.15 & 0.18 & 0.09 & & 0.18 & 0.03 & & 0.30 \\
\quad \textit{Unadjusted P-Value} & (0.19) & (0.17) & (0.38) & (0.08)** & (0.03)*** & (0.49) & & (0.08)** & (0.83) & & (0.00)*** \\
\quad \textit{Stepdown P-Value} & (0.57) & (0.56) & (0.64) & (0.36) & (0.18) & (0.89) & & (0.23) & (0.97) & & (0.01)*** \\
Volunteers & -0.08 & -0.08 & -0.08 & -0.03 & -0.03 & -0.12 & & -0.18 & -0.32 & & 0.04 \\
\quad \textit{Unadjusted P-Value} & (0.17) & (0.12)* & (0.13)* & (0.61) & (0.63) & (0.34) & & (0.06)** & (0.01)*** & & (0.45) \\
\quad \textit{Stepdown P-Value} & (0.55) & (0.56) & (0.36) & (0.94) & (0.98) & (0.85) & & (0.22) & (0.01)*** & & (0.70) \\
Ever Voted for Municipal & 0.10 & 0.03 & 0.04 & -0.07 & 0.02 & -0.08 & & 0.31 & -0.07 & & 0.34 \\
\quad \textit{Unadjusted P-Value} & (0.20) & (0.61) & (0.53) & (0.43) & (0.83) & (0.42) & & (0.00)*** & (0.59) & & (0.00)*** \\
\quad \textit{Stepdown P-Value} & (0.57) & (0.98) & (0.75) & (0.88) & (0.98) & (0.89) & & (0.00)*** & (0.96) & & (0.00)*** \\
Ever Voted for Regional & 0.05 & -0.02 & -0.01 & -0.09 & -0.01 & -0.06 & & 0.31 & 0.03 & & 0.27 \\
\quad \textit{Unadjusted P-Value} & (0.55) & (0.75) & (0.92) & (0.33) & (0.92) & (0.54) & & (0.00)*** & (0.84) & & (0.00)*** \\
\quad \textit{Stepdown P-Value} & (0.89) & (0.98) & (0.94) & (0.83) & (0.98) & (0.89) & & (0.00)*** & (0.97) & & (0.01)*** \\
\bottomrule
\end{tabular}
}
\vspace{1ex} \\
\footnotesize\raggedright{\underline{Note1 :} This table shows the estimates of the coefficient for attending Reggio Approach preschools from multiple methods. We compare Reggio Approach individuals with those who did not attend any preschools. Column title indicates the corresponding control set and and model. \textbf{None} = OLS estimate with no control variables. \textbf{BIC} = OLS estimate with controls selected by Bayesian Information Criterion (BIC) and additional controls for male indicator, migrant indicator, and ITC attendance indicator. \textbf{Full} = OLS estimate with the full set of controls. \textbf{PSMR} =  propensity score matching between Reggio Approach people and people in Reggio who did not attend any preschool. \textbf{KMR} = Epanechinikov kernel matching between Reggio Approach people and people in Reggio who attended other types of preschool. \textbf{DidPm} = difference-in-difference estimate of (Reggio Muni - Parma Muni) - (Reggio None - Parma None). \textbf{PSMPm} = propensity score matching between Reggio Approach people and people in Parma who did not attend any preschool. \textbf{KMPm} = Epanechinikov kernel matching between Reggio Approach people and people in Parma who did not attend any preschool. \textbf{DidPv} = difference-in-difference estimate of (Reggio Muni - Padova Muni) - (Reggio None - Padova None). \textbf{PSMPv} = propensity score matching between Reggio Approach people and people in Padova who did not attend any preschool. \textbf{KMPv} = Epanechinikov kernel matching between Reggio Approach people and people in Padova who did not attend any preschool.} 

\footnotesize\raggedright{\underline{Note2 :} Both unadjusted p-value and stepdown p-value are reported. ***, **, and * indicate significance of the coefficients at the 15\%, 10\%, and 5\% levels respectively.}
\end{table}













\subsubsection{Age-40 Cohort}
\begin{table}[H] \caption{Estimation Results for Cognitive and Education Outcomes, Comparison to Non-RA Preschools, Age-40 Cohort} \label{ols-CN-adult40-reg-other}
\scalebox{0.6}{\begin{tabular}{l c c c c c c c c c}
\toprule
& \multicolumn{5}{c}{Within Reggio} & \multicolumn{2}{c}{With Parma} & \multicolumn{2}{c}{With Padova} \\\cmidrule(lr){2-6} \cmidrule(lr){7-8} \cmidrule(lr){9-10}
 & None & BIC & Full & PSMR & KMR & PSMPm & KMPm & PSMPv & KMPv \\
\midrule
IQ Factor & -0.15 & -0.12 & -0.14 & -0.11 & -0.19 & -0.30 & -0.32 & -0.25 & -0.09 \\
\quad \textit{Unadjusted P-Value} & (0.22) & (0.29) & (0.22) & (0.34) & (0.16) & (0.01)*** & (0.00)*** & (0.07)** & (0.44) \\
\quad \textit{Stepdown P-Value} & (0.51) & (0.71) & (0.91) & (0.74) & (0.49) & (0.04)*** & (0.01)*** & (0.25) & (0.78) \\
High School Grade & -0.66 & -0.09 & 0.36 & -0.84 & -0.57 & 3.74 & 4.32 & 5.91 & 6.54 \\
\quad \textit{Unadjusted P-Value} & (0.67) & (0.96) & (0.83) & (0.61) & (0.74) & (0.04)*** & (0.04)*** & (0.00)*** & (0.00)*** \\
\quad \textit{Stepdown P-Value} & (0.68) & (0.95) & (0.98) & (0.89) & (0.94) & (0.12) & (0.15) & (0.00)*** & (0.00)*** \\
Graduate from High School & 0.13 & 0.10 & 0.12 & 0.09 & 0.08 & 0.05 & 0.03 & 0.05 & 0.01 \\
\quad \textit{Unadjusted P-Value} & (0.05)*** & (0.14)* & (0.09)** & (0.20) & (0.32) & (0.31) & (0.61) & (0.26) & (0.82) \\
\quad \textit{Stepdown P-Value} & (0.18) & (0.51) & (0.81) & (0.59) & (0.69) & (0.30) & (0.59) & (0.26) & (0.80) \\
Max Edu: University & 0.07 & 0.05 & 0.03 & 0.01 & -0.01 & -0.15 & -0.12 & -0.12 & -0.16 \\
\quad \textit{Unadjusted P-Value} & (0.20) & (0.34) & (0.62) & (0.92) & (0.88) & (0.03)*** & (0.07)** & (0.07)** & (0.02)*** \\
\quad \textit{Stepdown P-Value} & (0.51) & (0.71) & (0.98) & (0.93) & (0.94) & (0.12) & (0.18) & (0.25) & (0.08)** \\
\bottomrule
\end{tabular}
}
\vspace{1ex} \\
\footnotesize\raggedright{\underline{Note1 :} This table shows the estimates of the coefficient for attending Reggio Approach preschools from multiple methods. We compare Reggio Approach individuals with those who attended other preschools. Column title indicates the corresponding control set and and model. \textbf{None} = OLS estimate with no control variables. \textbf{BIC} = OLS estimate with controls selected by Bayesian Information Criterion (BIC) and additional controls for male indicator, migrant indicator, and ITC attendance indicator. \textbf{Full} = OLS estimate with the full set of controls. \textbf{PSMR} =  propensity score matching between Reggio Approach people and people in Reggio who attended nother types of preschool. \textbf{KMR} = Epanechinikov kernel matching between Reggio Approach people and people in Reggio who attended nother types of preschool. \textbf{PSMPm} = propensity score matching between Reggio Approach people and people who attended Parma preschools. \textbf{KMPm} = Epanechinikov kernel matching between Reggio Approach people and people who attended Parma preschools. \textbf{PSMPv} = propensity score matching between Reggio Approach people and people who attended Padova preschools. \textbf{KMPv} = Epanechinikov kernel matching between Reggio Approach people and people who attended Padova preschools. \textit{Difference-indifference is not available for this cohort due to non-existence of municipal preschools in Parma and Padova.}} 

\footnotesize\raggedright{\underline{Note2 :} Both unadjusted p-value and stepdown p-value are reported. ***, **, and * indicate significance of the coefficients at the 15\%, 10\%, and 5\% levels respectively.}
\end{table}

\begin{table}[H] \caption{Estimation Results for Cognitive and Education Outcomes, Comparison to No Preschool, Age-40 Cohort} \label{ols-CN-adult40-reg-none}
\scalebox{0.6}{\begin{tabular}{l c c c c c c c c c c c}
\toprule
 & None & BIC & Full & PSMR & KMR & DidPm & PSMPm & KMPm & DidPv & PSMPv & KMPv \\
\midrule
IQ Factor & 0.01 & 0.02 & 0.04 & 0.13 & 0.04 & 0.51 & -0.44 & -0.40 & 0.19 & -0.46 & -0.34 \\
\quad \textit{Unadjusted P-Value} & (0.97) & (0.86) & (0.80) & (0.36) & (0.80) & (0.06)** & (0.00)*** & (0.00)*** & (0.45) & (0.00)*** & (0.00)*** \\
\quad \textit{Stepdown P-Value} & (0.98) & (0.92) & (0.89) & (0.82) & (0.94) & (0.25) & (0.02)*** & (0.01)*** & (0.85) & (0.00)*** & (0.03)*** \\
High School Grade & 0.59 & 1.13 & 1.77 & 1.53 & 1.28 & -3.50 & 9.07 & 8.62 & -1.17 & 4.35 & 4.49 \\
\quad \textit{Unadjusted P-Value} & (0.70) & (0.47) & (0.35) & (0.34) & (0.45) & (0.40) & (0.00)*** & (0.00)*** & (0.75) & (0.03)*** & (0.06)** \\
\quad \textit{Stepdown P-Value} & (0.98) & (0.92) & (0.57) & (0.82) & (0.94) & (0.83) & (0.00)*** & (0.01)*** & (0.91) & (0.11) & (0.17) \\
Graduate from High School & -0.07 & -0.04 & -0.06 & -0.07 & -0.04 & 0.04 & 0.01 & -0.03 & -0.14 & 0.07 & 0.09 \\
\quad \textit{Unadjusted P-Value} & (0.17) & (0.47) & (0.33) & (0.25) & (0.45) & (0.74) & (0.79) & (0.64) & (0.21) & (0.25) & (0.28) \\
\quad \textit{Stepdown P-Value} & (0.64) & (0.92) & (0.55) & (0.73) & (0.94) & (0.88) & (0.78) & (0.86) & (0.76) & (0.47) & (0.53) \\
Max Edu: University & 0.01 & 0.05 & 0.11 & 0.03 & 0.04 & -0.08 & 0.04 & 0.03 & -0.13 & 0.00 & 0.03 \\
\quad \textit{Unadjusted P-Value} & (0.82) & (0.39) & (0.07)** & (0.64) & (0.48) & (0.53) & (0.49) & (0.62) & (0.34) & (0.98) & (0.75) \\
\quad \textit{Stepdown P-Value} & (0.98) & (0.92) & (0.24) & (0.88) & (0.94) & (0.84) & (0.70) & (0.86) & (0.85) & (0.98) & (0.86) \\
\bottomrule
\end{tabular}
}
\vspace{1ex} \\
\footnotesize\raggedright{\underline{Note1 :} This table shows the estimates of the coefficient for attending Reggio Approach preschools from multiple methods. We compare Reggio Approach individuals with those who did not attend any preschools. Column title indicates the corresponding control set and and model. \textbf{None} = OLS estimate with no control variables. \textbf{BIC} = OLS estimate with controls selected by Bayesian Information Criterion (BIC) and additional controls for male indicator, migrant indicator, and ITC attendance indicator. \textbf{Full} = OLS estimate with the full set of controls. \textbf{PSMR} =  propensity score matching between Reggio Approach people and people in Reggio who did not attend any preschool. \textbf{KMR} = Epanechinikov kernel matching between Reggio Approach people and people in Reggio who attended other types of preschool. \textbf{DidPm} = difference-in-difference estimate of (Reggio Muni - Parma Other) - (Reggio None - Parma None). \textbf{PSMPm} = propensity score matching between Reggio Approach people and people in Parma who did not attend any preschool. \textbf{KMPm} = Epanechinikov kernel matching between Reggio Approach people and people in Parma who did not attend any preschool. \textbf{DidPv} = difference-in-difference estimate of (Reggio Muni - Padova Other) - (Reggio None - Padova None). \textbf{PSMPv} = propensity score matching between Reggio Approach people and people in Padova who did not attend any preschool. \textbf{KMPv} = Epanechinikov kernel matching between Reggio Approach people and people in Padova who did not attend any preschool.} 

\footnotesize\raggedright{\underline{Note2 :} Both unadjusted p-value and stepdown p-value are reported. ***, **, and * indicate significance of the coefficients at the 15\%, 10\%, and 5\% levels respectively.}
\end{table}


\begin{table}[H] \caption{Estimation Results for Employment Outcomes, Comparison to Non-RA Preschools, Age-40 Cohort} \label{ols-W-adult40-reg-other}
\scalebox{0.6}{\begin{tabular}{l c c c c c}
\toprule
 & None & BIC & Full & PSM \\
\midrule
Employed & 0.01 & 0.01 & 0.01 & 0.03 \\
\quad \textit{Unadjusted P-Value} & (0.75) & (0.79) & (0.73) & (0.46) \\
\quad \textit{Stepdown P-Value} & (0.96) & (0.96) & (0.97) & (0.91) \\
Self-Employed & -0.10 & -0.11 & -0.12 & -0.10 \\
\quad \textit{Unadjusted P-Value} & (0.09) & (0.07) & (0.05) & (0.11) \\
\quad \textit{Stepdown P-Value} & (0.48) & (0.49) & (0.59) & (0.55) \\
Hours Worked Per Week & -0.90 & -1.17 & -1.28 & -1.71 \\
\quad \textit{Unadjusted P-Value} & (0.64) & (0.58) & (0.56) & (0.38) \\
\quad \textit{Stepdown P-Value} & (0.96) & (0.96) & (0.89) & (0.91) \\
Income: 5,000 Euros of Less & -0.01 & -0.02 & -0.02 & -0.01 \\
\quad \textit{Unadjusted P-Value} & (0.32) & (0.30) & (0.31) & (0.31) \\
\quad \textit{Stepdown P-Value} & (0.78) & (0.75) & (0.72) & (0.89) \\
Income: 5,001-10,000 Euros & -0.01 & -0.02 & -0.02 & -0.03 \\
\quad \textit{Unadjusted P-Value} & (0.32) & (0.31) & (0.28) & (0.32) \\
\quad \textit{Stepdown P-Value} & (0.78) & (0.69) & (0.67) & (0.89) \\
Income: 10,001-25,000 Euros & -0.04 & -0.02 & -0.02 & -0.06 \\
\quad \textit{Unadjusted P-Value} & (0.55) & (0.77) & (0.80) & (0.44) \\
\quad \textit{Stepdown P-Value} & (0.96) & (0.96) & (0.99) & (0.91) \\
Income: 25,001-50,000 Euros & 0.11 & 0.10 & 0.12 & 0.11 \\
\quad \textit{Unadjusted P-Value} & (0.17) & (0.23) & (0.17) & (0.22) \\
\quad \textit{Stepdown P-Value} & (0.69) & (0.75) & (0.69) & (0.80) \\
Income: 50,001-100,000 Euros & -0.01 & -0.02 & -0.03 & 0.01 \\
\quad \textit{Unadjusted P-Value} & (0.79) & (0.72) & (0.61) & (0.91) \\
\quad \textit{Stepdown P-Value} & (0.96) & (0.96) & (0.96) & (0.91) \\
\bottomrule
\end{tabular}
}
\vspace{1ex} \\
\footnotesize\raggedright{\underline{Note1 :} This table shows the estimates of the coefficient for attending Reggio Approach preschools from multiple methods. We compare Reggio Approach individuals with those who attended other preschools. Column title indicates the corresponding control set and and model. \textbf{None} = OLS estimate with no control variables. \textbf{BIC} = OLS estimate with controls selected by Bayesian Information Criterion (BIC) and additional controls for male indicator, migrant indicator, and ITC attendance indicator. \textbf{Full} = OLS estimate with the full set of controls. \textbf{PSMR} =  propensity score matching between Reggio Approach people and people in Reggio who attended nother types of preschool. \textbf{KMR} = Epanechinikov kernel matching between Reggio Approach people and people in Reggio who attended nother types of preschool. \textbf{PSMPm} = propensity score matching between Reggio Approach people and people who attended Parma preschools. \textbf{KMPm} = Epanechinikov kernel matching between Reggio Approach people and people who attended Parma preschools. \textbf{PSMPv} = propensity score matching between Reggio Approach people and people who attended Padova preschools. \textbf{KMPv} = Epanechinikov kernel matching between Reggio Approach people and people who attended Padova preschools. \textit{Difference-indifference is not available for this cohort due to non-existence of municipal preschools in Parma and Padova.}} 

\footnotesize\raggedright{\underline{Note2 :} Both unadjusted p-value and stepdown p-value are reported. ***, **, and * indicate significance of the coefficients at the 15\%, 10\%, and 5\% levels respectively.}
\end{table}

\begin{table}[H] \caption{Estimation Results for Employment Outcomes, Comparison to No Preschool, Age-40 Cohort} \label{ols-W-adult40-reg-none}
\scalebox{0.6}{\begin{tabular}{l c c c c c c c}
\toprule
 & None & BIC & Full & PSM & DidPm & DidPv \\
\midrule
Employed & 0.06 & 0.05 & 0.05 & 0.06 & 0.03 & 0.09 \\
\quad \textit{Unadjusted P-Value} & (0.14) & (0.14) & (0.16) & (0.18) & (0.49) & (0.10) \\
\quad \textit{Stepdown P-Value} & (0.51) & (0.61) & (0.47) & (0.59) & (0.95) & (0.57) \\
Self-Employed & 0.00 & 0.00 & -0.00 & 0.01 & 0.03 & 0.03 \\
\quad \textit{Unadjusted P-Value} & (0.98) & (0.99) & (0.93) & (0.84) & (0.71) & (0.71) \\
\quad \textit{Stepdown P-Value} & (0.98) & (0.98) & (0.95) & (0.85) & (0.99) & (0.87) \\
Hours Worked Per Week & 5.71 & 6.51 & 7.39 & 7.43 & 4.67 & 7.06 \\
\quad \textit{Unadjusted P-Value} & (0.02) & (0.01) & (0.01) & (0.00) & (0.11) & (0.02) \\
\quad \textit{Stepdown P-Value} & (0.12) & (0.04) & (0.02) & (0.04) & (0.57) & (0.21) \\
Income: 5,000 Euros of Less & -0.01 & -0.01 & -0.02 & -0.01 & -0.02 & -0.01 \\
\quad \textit{Unadjusted P-Value} & (0.32) & (0.31) & (0.29) & (0.32) & (0.36) & (0.31) \\
\quad \textit{Stepdown P-Value} & (0.53) & (0.72) & (0.34) & (0.72) & (0.87) & (0.74) \\
Income: 5,001-10,000 Euros & 0 & 0 & 0 & 0 & 0.02 & -0.02 \\
\quad \textit{Unadjusted P-Value} & () & () & () & () & (0.16) & (0.09) \\
\quad \textit{Stepdown P-Value} & (0) & (0) & (.0040000001899898) & (0) & (0.87) & (0.79) \\
Income: 10,001-25,000 Euros & -0.12 & -0.08 & -0.13 & -0.09 & -0.04 & 0.06 \\
\quad \textit{Unadjusted P-Value} & (0.10) & (0.23) & (0.10) & (0.25) & (0.69) & (0.55) \\
\quad \textit{Stepdown P-Value} & (0.43) & (0.72) & (0.25) & (0.69) & (0.99) & (0.87) \\
Income: 25,001-50,000 Euros & 0.11 & 0.06 & 0.03 & 0.05 & -0.05 & 0.07 \\
\quad \textit{Unadjusted P-Value} & (0.16) & (0.40) & (0.74) & (0.57) & (0.67) & (0.50) \\
\quad \textit{Stepdown P-Value} & (0.53) & (0.78) & (0.85) & (0.85) & (0.99) & (0.87) \\
Income: 50,001-100,000 Euros & 0.05 & 0.05 & 0.09 & 0.08 & 0.10 & -0.05 \\
\quad \textit{Unadjusted P-Value} & (0.17) & (0.14) & (0.02) & (0.04) & (0.03) & (0.39) \\
\quad \textit{Stepdown P-Value} & (0.53) & (0.62) & (0.09) & (0.22) & (0.21) & (0.87) \\
\bottomrule
\end{tabular}
}
\vspace{1ex} \\
\footnotesize\raggedright{\underline{Note1 :} This table shows the estimates of the coefficient for attending Reggio Approach preschools from multiple methods. We compare Reggio Approach individuals with those who did not attend any preschools. Column title indicates the corresponding control set and and model. \textbf{None} = OLS estimate with no control variables. \textbf{BIC} = OLS estimate with controls selected by Bayesian Information Criterion (BIC) and additional controls for male indicator, migrant indicator, and ITC attendance indicator. \textbf{Full} = OLS estimate with the full set of controls. \textbf{PSMR} =  propensity score matching between Reggio Approach people and people in Reggio who did not attend any preschool. \textbf{KMR} = Epanechinikov kernel matching between Reggio Approach people and people in Reggio who attended other types of preschool. \textbf{DidPm} = difference-in-difference estimate of (Reggio Muni - Parma Other) - (Reggio None - Parma None). \textbf{PSMPm} = propensity score matching between Reggio Approach people and people in Parma who did not attend any preschool. \textbf{KMPm} = Epanechinikov kernel matching between Reggio Approach people and people in Parma who did not attend any preschool. \textbf{DidPv} = difference-in-difference estimate of (Reggio Muni - Padova Other) - (Reggio None - Padova None). \textbf{PSMPv} = propensity score matching between Reggio Approach people and people in Padova who did not attend any preschool. \textbf{KMPv} = Epanechinikov kernel matching between Reggio Approach people and people in Padova who did not attend any preschool.} 

\footnotesize\raggedright{\underline{Note2 :} Both unadjusted p-value and stepdown p-value are reported. ***, **, and * indicate significance of the coefficients at the 15\%, 10\%, and 5\% levels respectively.}
\end{table}


\begin{table}[H] \caption{Estimation Results for Living Environment Outcomes, Comparison to Non-RA Preschools, Age-40 Cohort} \label{ols-L-adult40-reg-other}
\scalebox{0.6}{\begin{tabular}{l c c c c c}
\toprule
 & None & BIC & Full & PSM \\
\midrule
Married or Cohabitating & 0.03 & 0.02 & 0.02 & 0.01 \\
\quad \textit{Unadjusted P-Value} & (0.69) & (0.81) & (0.80) & (0.84) \\
\quad \textit{Stepdown P-Value} & (0.92) & (0.99) & (0.97) & (0.98) \\
Divorced & -0.06 & -0.04 & -0.03 & -0.03 \\
\quad \textit{Unadjusted P-Value} & (0.19) & (0.44) & (0.55) & (0.44) \\
\quad \textit{Stepdown P-Value} & (0.53) & (0.89) & (0.93) & (0.90) \\
Num. of Children in House & -0.21 & -0.20 & -0.21 & -0.20 \\
\quad \textit{Unadjusted P-Value} & (0.05) & (0.07) & (0.05) & (0.05) \\
\quad \textit{Stepdown P-Value} & (0.20) & (0.22) & (0.78) & (0.27) \\
Own House & 0.04 & -0.00 & -0.01 & -0.03 \\
\quad \textit{Unadjusted P-Value} & (0.58) & (0.95) & (0.89) & (0.72) \\
\quad \textit{Stepdown P-Value} & (0.92) & (0.99) & (0.98) & (0.98) \\
Live With Parents & -0.01 & 0.00 & -0.00 & -0.00 \\
\quad \textit{Unadjusted P-Value} & (0.79) & (0.95) & (0.92) & (0.92) \\
\quad \textit{Stepdown P-Value} & (0.92) & (0.99) & (0.98) & (0.98) \\
\bottomrule
\end{tabular}
}
\vspace{1ex} \\
\footnotesize\raggedright{\underline{Note1 :} This table shows the estimates of the coefficient for attending Reggio Approach preschools from multiple methods. We compare Reggio Approach individuals with those who attended other preschools. Column title indicates the corresponding control set and and model. \textbf{None} = OLS estimate with no control variables. \textbf{BIC} = OLS estimate with controls selected by Bayesian Information Criterion (BIC) and additional controls for male indicator, migrant indicator, and ITC attendance indicator. \textbf{Full} = OLS estimate with the full set of controls. \textbf{PSMR} =  propensity score matching between Reggio Approach people and people in Reggio who attended nother types of preschool. \textbf{KMR} = Epanechinikov kernel matching between Reggio Approach people and people in Reggio who attended nother types of preschool. \textbf{PSMPm} = propensity score matching between Reggio Approach people and people who attended Parma preschools. \textbf{KMPm} = Epanechinikov kernel matching between Reggio Approach people and people who attended Parma preschools. \textbf{PSMPv} = propensity score matching between Reggio Approach people and people who attended Padova preschools. \textbf{KMPv} = Epanechinikov kernel matching between Reggio Approach people and people who attended Padova preschools. \textit{Difference-indifference is not available for this cohort due to non-existence of municipal preschools in Parma and Padova.}} 

\footnotesize\raggedright{\underline{Note2 :} Both unadjusted p-value and stepdown p-value are reported. ***, **, and * indicate significance of the coefficients at the 15\%, 10\%, and 5\% levels respectively.}
\end{table}

\begin{table}[H] \caption{Estimation Results for Living Environment Outcomes, Comparison to No Preschool, Age-40 Cohort} \label{ols-L-adult40-reg-none}
\scalebox{0.6}{\begin{tabular}{l c c c c c c c}
\toprule
 & None & BIC & Full & PSM & DidPm & DidPv \\
\midrule
Married or Cohabitating & 0.02 & -0.01 & 0.05 & -0.00 & -0.06 & -0.14 \\
\quad \textit{Unadjusted P-Value} & (0.80) & (0.88) & (0.57) & (0.96) & (0.56) & (0.16) \\
\quad \textit{Stepdown P-Value} & (0.96) & (0.98) & (0.77) & (0.98) & (0.82) & (0.31) \\
Divorced & -0.06 & -0.04 & -0.03 & -0.03 & 0.08 & 0.14 \\
\quad \textit{Unadjusted P-Value} & (0.20) & (0.32) & (0.59) & (0.51) & (0.21) & (0.05) \\
\quad \textit{Stepdown P-Value} & (0.69) & (0.86) & (0.79) & (0.94) & (0.59) & (0.12) \\
Num. of Children in House & 0.01 & -0.04 & -0.04 & -0.03 & -0.19 & -0.51 \\
\quad \textit{Unadjusted P-Value} & (0.94) & (0.63) & (0.66) & (0.73) & (0.19) & (0.00) \\
\quad \textit{Stepdown P-Value} & (0.96) & (0.94) & (0.81) & (0.98) & (0.59) & (0.00) \\
Own House & -0.03 & -0.00 & 0.01 & 0.00 & -0.03 & -0.11 \\
\quad \textit{Unadjusted P-Value} & (0.68) & (0.96) & (0.88) & (0.97) & (0.78) & (0.19) \\
\quad \textit{Stepdown P-Value} & (0.96) & (0.98) & (0.92) & (0.98) & (0.83) & (0.31) \\
Live With Parents & -0.03 & -0.03 & -0.05 & -0.04 & -0.06 & -0.11 \\
\quad \textit{Unadjusted P-Value} & (0.34) & (0.38) & (0.11) & (0.27) & (0.20) & (0.03) \\
\quad \textit{Stepdown P-Value} & (0.78) & (0.86) & (0.30) & (0.75) & (0.59) & (0.13) \\
\bottomrule
\end{tabular}
}
\vspace{1ex} \\
\footnotesize\raggedright{\underline{Note1 :} This table shows the estimates of the coefficient for attending Reggio Approach preschools from multiple methods. We compare Reggio Approach individuals with those who did not attend any preschools. Column title indicates the corresponding control set and and model. \textbf{None} = OLS estimate with no control variables. \textbf{BIC} = OLS estimate with controls selected by Bayesian Information Criterion (BIC) and additional controls for male indicator, migrant indicator, and ITC attendance indicator. \textbf{Full} = OLS estimate with the full set of controls. \textbf{PSMR} =  propensity score matching between Reggio Approach people and people in Reggio who did not attend any preschool. \textbf{KMR} = Epanechinikov kernel matching between Reggio Approach people and people in Reggio who attended other types of preschool. \textbf{DidPm} = difference-in-difference estimate of (Reggio Muni - Parma Other) - (Reggio None - Parma None). \textbf{PSMPm} = propensity score matching between Reggio Approach people and people in Parma who did not attend any preschool. \textbf{KMPm} = Epanechinikov kernel matching between Reggio Approach people and people in Parma who did not attend any preschool. \textbf{DidPv} = difference-in-difference estimate of (Reggio Muni - Padova Other) - (Reggio None - Padova None). \textbf{PSMPv} = propensity score matching between Reggio Approach people and people in Padova who did not attend any preschool. \textbf{KMPv} = Epanechinikov kernel matching between Reggio Approach people and people in Padova who did not attend any preschool.} 

\footnotesize\raggedright{\underline{Note2 :} Both unadjusted p-value and stepdown p-value are reported. ***, **, and * indicate significance of the coefficients at the 15\%, 10\%, and 5\% levels respectively.}
\end{table}


\begin{table}[H] \caption{Estimation Results for Health and Risk Outcomes, Comparison to Non-RA Preschools, Age-40 Cohort} \label{ols-H-adult40-reg-other}
\scalebox{0.6}{\begin{tabular}{l c c c c c c c c c}
\toprule
 & None & BIC & Full & PSMR & KMR & PSMPm & KMPm & PSMPv & KMPv \\
\midrule
Tried Marijuana & 0.09 & 0.11 & 0.09 & 0.11 & 0.10 & 0.06 & 0.06 & 0.06 & 0.08 \\
\quad \textit{Unadjusted P-Value} & (0.06)** & (0.04)*** & (0.10)* & (0.04)*** & (0.07)** & (0.23) & (0.27) & (0.19) & (0.10)** \\
\quad \textit{Stepdown P-Value} & (0.49) & (0.31) & (0.56) & (0.32) & (0.51) & (0.80) & (0.83) & (0.73) & (0.59) \\
Num. of Cigarettes Per Day & 2.59 & 2.98 & 1.15 & 3.21 & 3.05 & 3.58 & 1.88 & 5.34 & 4.98 \\
\quad \textit{Unadjusted P-Value} & (0.16) & (0.17) & (0.62) & (0.06)** & (0.21) & (0.04)*** & (0.35) & (0.00)*** & (0.02)*** \\
\quad \textit{Stepdown P-Value} & (0.75) & (0.64) & (0.85) & (0.43) & (0.77) & (0.29) & (0.86) & (0.05)*** & (0.16) \\
BMI & -0.01 & -0.04 & -0.05 & -0.19 & -0.29 & -0.14 & -0.18 & 0.35 & 0.45 \\
\quad \textit{Unadjusted P-Value} & (0.99) & (0.95) & (0.93) & (0.74) & (0.63) & (0.81) & (0.74) & (0.54) & (0.49) \\
\quad \textit{Stepdown P-Value} & (0.99) & (0.99) & (0.96) & (0.88) & (0.99) & (0.99) & (0.97) & (0.97) & (0.97) \\
Not Obese & -0.04 & 0.02 & 0.04 & 0.03 & 0.03 & -0.07 & -0.10 & -0.08 & -0.00 \\
\quad \textit{Unadjusted P-Value} & (0.59) & (0.80) & (0.56) & (0.76) & (0.76) & (0.32) & (0.16) & (0.28) & (1.00) \\
\quad \textit{Stepdown P-Value} & (0.99) & (0.99) & (0.84) & (0.99) & (0.99) & (0.85) & (0.69) & (0.86) & (0.98) \\
Not Overweight & 0.05 & 0.03 & 0.03 & -0.01 & -0.00 & 0.01 & 0.06 & -0.03 & -0.03 \\
\quad \textit{Unadjusted P-Value} & (0.48) & (0.63) & (0.67) & (0.92) & (0.99) & (0.84) & (0.41) & (0.56) & (0.68) \\
\quad \textit{Stepdown P-Value} & (0.98) & (0.99) & (0.88) & (0.99) & (0.99) & (0.99) & (0.86) & (0.97) & (0.98) \\
Good Health & -0.18 & -0.17 & -0.18 & -0.15 & -0.17 & 0.33 & 0.34 & 0.22 & 0.15 \\
\quad \textit{Unadjusted P-Value} & (0.05)*** & (0.08)** & (0.07)** & (0.13)* & (0.10)** & (0.00)*** & (0.00)*** & (0.02)*** & (0.15) \\
\quad \textit{Stepdown P-Value} & (0.39) & (0.48) & (0.49) & (0.62) & (0.60) & (0.02)*** & (0.02)*** & (0.20) & (0.66) \\
No Problematic Health Condition & 0.00 & -0.03 & -0.03 & 0.02 & 0.01 & 0.18 & 0.17 & 0.03 & -0.05 \\
\quad \textit{Unadjusted P-Value} & (0.96) & (0.70) & (0.71) & (0.85) & (0.91) & (0.01)*** & (0.05)*** & (0.71) & (0.52) \\
\quad \textit{Stepdown P-Value} & (0.99) & (0.99) & (0.90) & (0.99) & (0.99) & (0.14) & (0.36) & (0.97) & (0.97) \\
Num. of Days Sick Past Month & 0.06 & 0.09 & 0.10 & 0.08 & 0.08 & -0.01 & -0.00 & -0.03 & 0.02 \\
\quad \textit{Unadjusted P-Value} & (0.27) & (0.09)** & (0.06)** & (0.07)** & (0.15) & (0.78) & (0.96) & (0.55) & (0.72) \\
\quad \textit{Stepdown P-Value} & (0.86) & (0.48) & (0.51) & (0.43) & (0.71) & (0.99) & (0.97) & (0.97) & (0.98) \\
Ever Suspended from School & -0.00 & -0.02 & -0.01 & -0.05 & -0.06 & 0.01 & 0.01 & 0.02 & 0.00 \\
\quad \textit{Unadjusted P-Value} & (0.95) & (0.49) & (0.77) & (0.28) & (0.19) & (0.76) & (0.68) & (0.69) & (0.93) \\
\quad \textit{Stepdown P-Value} & (0.99) & (0.99) & (0.93) & (0.88) & (0.77) & (0.85) & (0.97) & (0.97) & (0.98) \\
Age At First Drink & -0.36 & 0.11 & -0.07 & -0.24 & -0.51 & -2.83 & -3.17 & -2.29 & -2.00 \\
\quad \textit{Unadjusted P-Value} & (0.79) & (0.94) & (0.95) & (0.86) & (0.75) & (0.02)*** & (0.00)*** & (0.07)** & (0.10)* \\
\quad \textit{Stepdown P-Value} & (0.99) & (0.99) & (0.98) & (0.99) & (0.99) & (0.18) & (0.05)*** & (0.39) & (0.59) \\
\bottomrule
\end{tabular}
}
\vspace{1ex} \\
\footnotesize\raggedright{\underline{Note1 :} This table shows the estimates of the coefficient for attending Reggio Approach preschools from multiple methods. We compare Reggio Approach individuals with those who attended other preschools. Column title indicates the corresponding control set and and model. \textbf{None} = OLS estimate with no control variables. \textbf{BIC} = OLS estimate with controls selected by Bayesian Information Criterion (BIC) and additional controls for male indicator, migrant indicator, and ITC attendance indicator. \textbf{Full} = OLS estimate with the full set of controls. \textbf{PSMR} =  propensity score matching between Reggio Approach people and people in Reggio who attended nother types of preschool. \textbf{KMR} = Epanechinikov kernel matching between Reggio Approach people and people in Reggio who attended nother types of preschool. \textbf{PSMPm} = propensity score matching between Reggio Approach people and people who attended Parma preschools. \textbf{KMPm} = Epanechinikov kernel matching between Reggio Approach people and people who attended Parma preschools. \textbf{PSMPv} = propensity score matching between Reggio Approach people and people who attended Padova preschools. \textbf{KMPv} = Epanechinikov kernel matching between Reggio Approach people and people who attended Padova preschools. \textit{Difference-indifference is not available for this cohort due to non-existence of municipal preschools in Parma and Padova.}} 

\footnotesize\raggedright{\underline{Note2 :} Both unadjusted p-value and stepdown p-value are reported. ***, **, and * indicate significance of the coefficients at the 15\%, 10\%, and 5\% levels respectively.}
\end{table}

\begin{table}[H] \caption{Estimation Results for Health and Risk Outcomes, Comparison to No Preschool, Age-40 Cohort} \label{ols-H-adult40-reg-none}
\scalebox{0.6}{\begin{tabular}{l c c c c c c c}
\toprule
 & None & BIC & Full & PSM & DidPm & DidPv \\
\midrule
Tried Marijuana & 0.08 & 0.09 & 0.10 & 0.09 & 0.01 & 0.04 \\
\quad \textit{Unadjusted P-Value} & (0.12) & (0.07) & (0.08) & (0.14) & (0.91) & (0.55) \\
\quad \textit{Stepdown P-Value} & (0.67) & (0.48) & (0.27) & (0.73) & (0.99) & (0.99) \\
Num. of Cigarettes Per Day & -0.18 & -0.48 & -0.73 & 0.69 & 0.78 & 1.67 \\
\quad \textit{Unadjusted P-Value} & (0.92) & (0.79) & (0.69) & (0.69) & (0.73) & (0.48) \\
\quad \textit{Stepdown P-Value} & (0.98) & (0.99) & (0.86) & (0.91) & (0.99) & (0.99) \\
BMI & -0.22 & -0.37 & -0.30 & -0.30 & -1.15 & 0.01 \\
\quad \textit{Unadjusted P-Value} & (0.69) & (0.49) & (0.55) & (0.59) & (0.12) & (0.99) \\
\quad \textit{Stepdown P-Value} & (0.98) & (0.98) & (0.86) & (0.91) & (0.57) & (0.99) \\
Not Obese & 0.14 & 0.11 & 0.01 & 0.12 & 0.23 & 0.03 \\
\quad \textit{Unadjusted P-Value} & (0.06) & (0.14) & (0.91) & (0.11) & (0.01) & (0.73) \\
\quad \textit{Stepdown P-Value} & (0.50) & (0.69) & (0.95) & (0.69) & (0.12) & (0.99) \\
Not Overweight & -0.03 & 0.03 & 0.07 & 0.05 & 0.02 & -0.04 \\
\quad \textit{Unadjusted P-Value} & (0.66) & (0.68) & (0.33) & (0.43) & (0.86) & (0.63) \\
\quad \textit{Stepdown P-Value} & (0.98) & (0.99) & (0.65) & (0.88) & (0.99) & (0.99) \\
Good Health & 0.10 & 0.13 & 0.12 & 0.16 & 0.30 & -0.05 \\
\quad \textit{Unadjusted P-Value} & (0.24) & (0.17) & (0.25) & (0.11) & (0.03) & (0.69) \\
\quad \textit{Stepdown P-Value} & (0.90) & (0.69) & (0.58) & (0.69) & (0.20) & (0.99) \\
No Problematic Health Condition & 0.01 & 0.01 & 0.07 & 0.10 & 0.06 & -0.06 \\
\quad \textit{Unadjusted P-Value} & (0.88) & (0.86) & (0.47) & (0.25) & (0.62) & (0.57) \\
\quad \textit{Stepdown P-Value} & (0.98) & (0.99) & (0.78) & (0.87) & (0.99) & (0.99) \\
Num. of Days Sick Past Month & 0.02 & 0.00 & -0.04 & 0.01 & -0.06 & 0.00 \\
\quad \textit{Unadjusted P-Value} & (0.70) & (0.98) & (0.44) & (0.90) & (0.61) & (1.00) \\
\quad \textit{Stepdown P-Value} & (0.98) & (0.99) & (0.79) & (0.91) & (0.99) & (0.99) \\
Ever Suspended from School & -0.03 & -0.03 & -0.02 & -0.04 & 0.01 & -0.05 \\
\quad \textit{Unadjusted P-Value} & (0.43) & (0.41) & (0.70) & (0.29) & (0.90) & (0.34) \\
\quad \textit{Stepdown P-Value} & (0.97) & (0.97) & (0.86) & (0.87) & (0.99) & (0.97) \\
Age At First Drink & 1.00 & 0.42 & -0.73 & 1.53 & 0.66 & -0.53 \\
\quad \textit{Unadjusted P-Value} & (0.47) & (0.76) & (0.62) & (0.30) & (0.69) & (0.77) \\
\quad \textit{Stepdown P-Value} & (0.97) & (0.99) & (0.86) & (0.87) & (0.99) & (0.99) \\
\bottomrule
\end{tabular}
}
\vspace{1ex} \\
\footnotesize\raggedright{\underline{Note1 :} This table shows the estimates of the coefficient for attending Reggio Approach preschools from multiple methods. We compare Reggio Approach individuals with those who did not attend any preschools. Column title indicates the corresponding control set and and model. \textbf{None} = OLS estimate with no control variables. \textbf{BIC} = OLS estimate with controls selected by Bayesian Information Criterion (BIC) and additional controls for male indicator, migrant indicator, and ITC attendance indicator. \textbf{Full} = OLS estimate with the full set of controls. \textbf{PSMR} =  propensity score matching between Reggio Approach people and people in Reggio who did not attend any preschool. \textbf{KMR} = Epanechinikov kernel matching between Reggio Approach people and people in Reggio who attended other types of preschool. \textbf{DidPm} = difference-in-difference estimate of (Reggio Muni - Parma Other) - (Reggio None - Parma None). \textbf{PSMPm} = propensity score matching between Reggio Approach people and people in Parma who did not attend any preschool. \textbf{KMPm} = Epanechinikov kernel matching between Reggio Approach people and people in Parma who did not attend any preschool. \textbf{DidPv} = difference-in-difference estimate of (Reggio Muni - Padova Other) - (Reggio None - Padova None). \textbf{PSMPv} = propensity score matching between Reggio Approach people and people in Padova who did not attend any preschool. \textbf{KMPv} = Epanechinikov kernel matching between Reggio Approach people and people in Padova who did not attend any preschool.} 

\footnotesize\raggedright{\underline{Note2 :} Both unadjusted p-value and stepdown p-value are reported. ***, **, and * indicate significance of the coefficients at the 15\%, 10\%, and 5\% levels respectively.}
\end{table}


\begin{table}[H] \caption{Estimation Results for Noncognitive Outcomes, Comparison to Non-RA Preschools, Age-40 Cohort} \label{ols-N-adult40-reg-other}
\scalebox{0.6}{\begin{tabular}{l c c c c c c c c c}
\toprule
& \multicolumn{5}{c}{Within Reggio} & With Parma & With Padova \\\cmidrule(lr){2-6} \cmidrule(lr){7} \cmidrule(lr){8}
 & None & BIC & Full & PSMR & KMR & KMPm & KMPv \\
\midrule
Locus of Control - positive & 0.13 & 0.14 & 0.11 & 0.19 & 0.11 & 0.23 & 0.17 \\
\quad \textit{Unadjusted P-Value} & (0.36) & (0.31) & (0.44) & (0.27) & (0.48) & (0.09)** & (0.18) \\
\quad \textit{Stepdown P-Value} & (0.93) & (0.87) & (0.88) & (0.72) & (0.94) & (0.39) & (0.69) \\
Depression Score - positive & 0.56 & 1.37 & 1.09 & 1.28 & 0.98 & -0.72 & 0.91 \\
\quad \textit{Unadjusted P-Value} & (0.55) & (0.11)* & (0.22) & (0.16) & (0.36) & (0.40) & (0.27) \\
\quad \textit{Stepdown P-Value} & (0.96) & (0.69) & (0.77) & (0.69) & (0.94) & (0.88) & (0.83) \\
Stress & 0.05 & 0.09 & 0.09 & 0.17 & 0.10 & 0.03 & 0.10 \\
\quad \textit{Unadjusted P-Value} & (0.62) & (0.44) & (0.43) & (0.18) & (0.40) & (0.79) & (0.33) \\
\quad \textit{Stepdown P-Value} & (0.96) & (0.90) & (0.85) & (0.69) & (0.94) & (0.97) & (0.83) \\
Work is Source of Stress & 0.30 & 0.23 & 0.19 & 0.34 & 0.32 & 0.37 & 0.17 \\
\quad \textit{Unadjusted P-Value} & (0.00)*** & (0.04)*** & (0.12)* & (0.00)*** & (0.02)*** & (0.00)*** & (0.06)** \\
\quad \textit{Stepdown P-Value} & (0.00)*** & (0.22) & (0.55) & (0.00)*** & (0.22) & (0.00)*** & (0.43) \\
Satisfied with Income & 0.15 & 0.14 & 0.15 & 0.24 & 0.17 & 0.30 & 0.19 \\
\quad \textit{Unadjusted P-Value} & (0.28) & (0.30) & (0.28) & (0.11)* & (0.30) & (0.01)*** & (0.12)* \\
\quad \textit{Stepdown P-Value} & (0.92) & (0.87) & (0.79) & (0.65) & (0.92) & (0.08)** & (0.62) \\
Satisfied with Work & 0.12 & 0.18 & 0.16 & 0.31 & 0.27 & 0.34 & 0.53 \\
\quad \textit{Unadjusted P-Value} & (0.31) & (0.17) & (0.22) & (0.04)*** & (0.05)** & (0.00)*** & (0.00)*** \\
\quad \textit{Stepdown P-Value} & (0.92) & (0.69) & (0.71) & (0.34) & (0.39) & (0.03)*** & (0.00)*** \\
Satisfied with Health & -0.16 & -0.12 & -0.12 & -0.08 & -0.13 & 0.21 & 0.10 \\
\quad \textit{Unadjusted P-Value} & (0.07)** & (0.21) & (0.22) & (0.40) & (0.18) & (0.06)** & (0.31) \\
\quad \textit{Stepdown P-Value} & (0.45) & (0.79) & (0.74) & (0.80) & (0.76) & (0.35) & (0.83) \\
Satisfied with Family & 0.02 & 0.05 & 0.08 & 0.09 & -0.02 & -0.06 & 0.10 \\
\quad \textit{Unadjusted P-Value} & (0.86) & (0.73) & (0.56) & (0.52) & (0.88) & (0.61) & (0.39) \\
\quad \textit{Stepdown P-Value} & (0.96) & (0.96) & (0.97) & (0.80) & (0.94) & (0.96) & (0.83) \\
Optimistic Look in Life & -0.03 & -0.04 & -0.06 & -0.08 & -0.03 & 0.19 & 0.05 \\
\quad \textit{Unadjusted P-Value} & (0.70) & (0.64) & (0.46) & (0.33) & (0.74) & (0.01)*** & (0.54) \\
\quad \textit{Stepdown P-Value} & (0.96) & (0.96) & (0.89) & (0.80) & (0.94) & (0.10)** & (0.83) \\
Positive Reciprocity & -0.23 & -0.15 & -0.18 & -0.16 & -0.20 & -0.03 & 0.21 \\
\quad \textit{Unadjusted P-Value} & (0.03)*** & (0.14)* & (0.08)** & (0.12)* & (0.07)** & (0.78) & (0.11)* \\
\quad \textit{Stepdown P-Value} & (0.28) & (0.76) & (0.66) & (0.66) & (0.47) & (0.97) & (0.62) \\
Negative Reciprocity & 0.09 & 0.03 & 0.10 & 0.12 & 0.12 & 0.41 & 0.25 \\
\quad \textit{Unadjusted P-Value} & (0.54) & (0.84) & (0.52) & (0.42) & (0.48) & (0.01)*** & (0.10)* \\
\quad \textit{Stepdown P-Value} & (0.96) & (0.96) & (0.94) & (0.80) & (0.94) & (0.06)** & (0.62) \\
\bottomrule
\end{tabular}
}
\vspace{1ex} \\
\footnotesize\raggedright{\underline{Note1 :} This table shows the estimates of the coefficient for attending Reggio Approach preschools from multiple methods. We compare Reggio Approach individuals with those who attended other preschools. Column title indicates the corresponding control set and and model. \textbf{None} = OLS estimate with no control variables. \textbf{BIC} = OLS estimate with controls selected by Bayesian Information Criterion (BIC) and additional controls for male indicator, migrant indicator, and ITC attendance indicator. \textbf{Full} = OLS estimate with the full set of controls. \textbf{PSMR} =  propensity score matching between Reggio Approach people and people in Reggio who attended nother types of preschool. \textbf{KMR} = Epanechinikov kernel matching between Reggio Approach people and people in Reggio who attended nother types of preschool. \textbf{PSMPm} = propensity score matching between Reggio Approach people and people who attended Parma preschools. \textbf{KMPm} = Epanechinikov kernel matching between Reggio Approach people and people who attended Parma preschools. \textbf{PSMPv} = propensity score matching between Reggio Approach people and people who attended Padova preschools. \textbf{KMPv} = Epanechinikov kernel matching between Reggio Approach people and people who attended Padova preschools. \textit{Difference-indifference is not available for this cohort due to non-existence of municipal preschools in Parma and Padova.}} 

\footnotesize\raggedright{\underline{Note2 :} Both unadjusted p-value and stepdown p-value are reported. ***, **, and * indicate significance of the coefficients at the 15\%, 10\%, and 5\% levels respectively.}
\end{table}

\begin{table}[H] \caption{Estimation Results for Noncognitive Outcomes, Comparison to No Preschool, Age-40 Cohort} \label{ols-N-adult40-reg-none}
\scalebox{0.6}{\begin{tabular}{l c c c c c c c c c c c}
\toprule
& \multicolumn{5}{c}{Within Reggio} & \multicolumn{3}{c}{With Parma} & \multicolumn{3}{c}{With Padova} \\\cmidrule(lr){2-6} \cmidrule(lr){7-9} \cmidrule(lr){10-12}
 & None & BIC & Full & PSMR & KMR & DidPm & KMDidPm & KMPm & DidPv & KMDidPv & KMPv \\
\midrule
Locus of Control - positive & 0.14 & 0.23 & 0.28 & 0.26 & 0.21 & 0.31 & 0.19 & 0.12 & 0.20 & 0.31 & 0.04 \\
\quad \textit{Unadjusted P-Value} & (0.29) & (0.07)** & (0.05)** & (0.05)** & (0.13)* & (0.28) & (0.34) & (0.39) & (0.47) & (0.12)* & (0.81) \\
\quad \textit{Stepdown P-Value} & (0.76) & (0.46) & (0.25) & (0.37) & (0.63) & (0.95) & (0.98) & (0.80) & (0.96) & (0.68) & (0.96) \\
Depression Score - positive & 2.25 & 2.24 & 2.10 & 2.90 & 2.16 & -1.72 & 0.12 & 0.93 & 2.20 & 2.03 & 0.35 \\
\quad \textit{Unadjusted P-Value} & (0.02)*** & (0.02)*** & (0.05)** & (0.00)*** & (0.03)*** & (0.37) & (0.92) & (0.26) & (0.25) & (0.14)* & (0.73) \\
\quad \textit{Stepdown P-Value} & (0.14) & (0.13) & (0.18) & (0.03)*** & (0.25) & (0.95) & (0.99) & (0.77) & (0.88) & (0.68) & (0.96) \\
Stress & 0.19 & 0.20 & 0.06 & 0.23 & 0.16 & -0.01 & 0.06 & 0.21 & 0.58 & 0.38 & -0.07 \\
\quad \textit{Unadjusted P-Value} & (0.10)** & (0.07)** & (0.67) & (0.05)*** & (0.18) & (0.96) & (0.76) & (0.09)** & (0.01)*** & (0.00)*** & (0.59) \\
\quad \textit{Stepdown P-Value} & (0.47) & (0.46) & (0.88) & (0.37) & (0.63) & (0.99) & (0.99) & (0.39) & (0.12) & (0.15) & (0.96) \\
Work is Source of Stress & 0.17 & 0.18 & 0.23 & 0.18 & 0.14 & 0.25 & 0.44 & 0.07 & -0.11 & 0.09 & 0.31 \\
\quad \textit{Unadjusted P-Value} & (0.07)** & (0.04)*** & (0.02)*** & (0.04)*** & (0.15)* & (0.25) & (0.03)*** & (0.45) & (0.61) & (0.60) & (0.01)*** \\
\quad \textit{Stepdown P-Value} & (0.41) & (0.33) & (0.08)** & (0.37) & (0.63) & (0.93) & (0.13) & (0.80) & (0.98) & (0.91) & (0.07)** \\
Satisfied with Income & 0.27 & 0.27 & 0.15 & 0.22 & 0.25 & -0.03 & 0.02 & 0.50 & 0.16 & 0.22 & 0.29 \\
\quad \textit{Unadjusted P-Value} & (0.05)*** & (0.06)** & (0.33) & (0.14)* & (0.09)** & (0.90) & (0.94) & (0.00)*** & (0.52) & (0.32) & (0.04)*** \\
\quad \textit{Stepdown P-Value} & (0.31) & (0.33) & (0.69) & (0.56) & (0.49) & (0.99) & (0.99) & (0.00)*** & (0.98) & (0.86) & (0.24) \\
Satisfied with Work & 0.31 & 0.29 & 0.16 & 0.21 & 0.27 & 0.04 & 0.14 & 0.42 & 0.30 & 0.38 & 0.48 \\
\quad \textit{Unadjusted P-Value} & (0.01)*** & (0.02)*** & (0.20) & (0.11)* & (0.04)*** & (0.87) & (0.47) & (0.00)*** & (0.27) & (0.05)** & (0.00)*** \\
\quad \textit{Stepdown P-Value} & (0.11) & (0.16) & (0.60) & (0.52) & (0.32) & (0.99) & (0.99) & (0.00)*** & (0.92) & (0.48) & (0.02)*** \\
Satisfied with Health & 0.03 & 0.02 & -0.03 & 0.03 & 0.00 & 0.33 & 0.13 & 0.09 & 0.15 & 0.01 & 0.19 \\
\quad \textit{Unadjusted P-Value} & (0.75) & (0.82) & (0.72) & (0.71) & (0.96) & (0.07)** & (0.44) & (0.35) & (0.35) & (0.97) & (0.04)*** \\
\quad \textit{Stepdown P-Value} & (0.98) & (0.99) & (0.88) & (0.96) & (0.98) & (0.72) & (0.99) & (0.80) & (0.95) & (0.97) & (0.24) \\
Satisfied with Family & 0.19 & 0.15 & 0.18 & 0.15 & 0.18 & 0.14 & -0.15 & 0.26 & 0.13 & -0.03 & 0.26 \\
\quad \textit{Unadjusted P-Value} & (0.11)* & (0.22) & (0.24) & (0.26) & (0.17) & (0.65) & (0.44) & (0.03)*** & (0.65) & (0.79) & (0.07)** \\
\quad \textit{Stepdown P-Value} & (0.47) & (0.68) & (0.59) & (0.74) & (0.63) & (0.99) & (0.99) & (0.26) & (0.98) & (0.97) & (0.33) \\
Optimistic Look in Life & -0.10 & -0.06 & 0.06 & -0.11 & -0.07 & -0.09 & -0.09 & 0.21 & -0.25 & -0.15 & 0.05 \\
\quad \textit{Unadjusted P-Value} & (0.18) & (0.47) & (0.47) & (0.24) & (0.41) & (0.58) & (0.55) & (0.01)*** & (0.16) & (0.25) & (0.63) \\
\quad \textit{Stepdown P-Value} & (0.61) & (0.90) & (0.79) & (0.74) & (0.86) & (0.99) & (0.99) & (0.06)** & (0.80) & (0.87) & (0.96) \\
Positive Reciprocity & -0.02 & 0.01 & -0.05 & -0.05 & -0.04 & 0.22 & -0.03 & -0.02 & 0.10 & 0.12 & -0.13 \\
\quad \textit{Unadjusted P-Value} & (0.89) & (0.96) & (0.69) & (0.68) & (0.75) & (0.51) & (0.90) & (0.85) & (0.77) & (0.52) & (0.34) \\
\quad \textit{Stepdown P-Value} & (0.98) & (0.99) & (0.88) & (0.96) & (0.98) & (0.95) & (0.99) & (0.83) & (0.98) & (0.91) & (0.91) \\
Negative Reciprocity & 0.02 & -0.05 & -0.05 & -0.05 & -0.05 & -0.41 & -0.08 & 0.35 & -0.45 & -0.23 & 0.48 \\
\quad \textit{Unadjusted P-Value} & (0.92) & (0.74) & (0.78) & (0.76) & (0.75) & (0.22) & (0.76) & (0.03)*** & (0.17) & (0.29) & (0.00)*** \\
\quad \textit{Stepdown P-Value} & (0.98) & (0.99) & (0.88) & (0.96) & (0.98) & (0.89) & (0.99) & (0.26) & (0.80) & (0.87) & (0.03)*** \\
\bottomrule
\end{tabular}
}
\vspace{1ex} \\
\footnotesize\raggedright{\underline{Note1 :} This table shows the estimates of the coefficient for attending Reggio Approach preschools from multiple methods. We compare Reggio Approach individuals with those who did not attend any preschools. Column title indicates the corresponding control set and and model. \textbf{None} = OLS estimate with no control variables. \textbf{BIC} = OLS estimate with controls selected by Bayesian Information Criterion (BIC) and additional controls for male indicator, migrant indicator, and ITC attendance indicator. \textbf{Full} = OLS estimate with the full set of controls. \textbf{PSMR} =  propensity score matching between Reggio Approach people and people in Reggio who did not attend any preschool. \textbf{KMR} = Epanechinikov kernel matching between Reggio Approach people and people in Reggio who attended other types of preschool. \textbf{DidPm} = difference-in-difference estimate of (Reggio Muni - Parma Other) - (Reggio None - Parma None). \textbf{PSMPm} = propensity score matching between Reggio Approach people and people in Parma who did not attend any preschool. \textbf{KMPm} = Epanechinikov kernel matching between Reggio Approach people and people in Parma who did not attend any preschool. \textbf{DidPv} = difference-in-difference estimate of (Reggio Muni - Padova Other) - (Reggio None - Padova None). \textbf{PSMPv} = propensity score matching between Reggio Approach people and people in Padova who did not attend any preschool. \textbf{KMPv} = Epanechinikov kernel matching between Reggio Approach people and people in Padova who did not attend any preschool.} 

\footnotesize\raggedright{\underline{Note2 :} Both unadjusted p-value and stepdown p-value are reported. ***, **, and * indicate significance of the coefficients at the 15\%, 10\%, and 5\% levels respectively.}
\end{table}


\begin{table}[H] \caption{Estimation Results for Social Outcomes, Comparison to Non-RA Preschools, Age-40 Cohort} \label{ols-S-adult40-reg-other}
\scalebox{0.6}{\begin{tabular}{l c c c c c c c c c}
\toprule
& \multicolumn{5}{c}{Within Reggio} & \multicolumn{2}{c}{With Parma} & \multicolumn{2}{c}{With Padova} \\\cmidrule(lr){2-6} \cmidrule(lr){7-8} \cmidrule(lr){9-10}
 & None & BIC & Full & PSMR & KMR & PSMPm & KMPm & PSMPv & KMPv \\
\midrule
Bothered by Migrants & 0.03 & 0.06 & 0.04 & -0.00 & 0.05 & -0.07 & -0.03 & 0.19 & 0.26 \\
\quad \textit{Unadjusted P-Value} & (0.71) & (0.55) & (0.65) & (0.99) & (0.58) & (0.52) & (0.79) & (0.05)*** & (0.01)*** \\
\quad \textit{Stepdown P-Value} & (0.91) & (0.94) & (0.95) & (0.99) & (0.98) & (0.74) & (0.97) & (0.26) & (0.07)** \\
Num. of Friends & 1.39 & 0.95 & 1.09 & 0.88 & 1.42 & 0.47 & 0.25 & 0.39 & 0.16 \\
\quad \textit{Unadjusted P-Value} & (0.15)* & (0.34) & (0.29) & (0.44) & (0.16) & (0.61) & (0.80) & (0.67) & (0.88) \\
\quad \textit{Stepdown P-Value} & (0.65) & (0.80) & (0.67) & (0.85) & (0.67) & (0.74) & (0.97) & (0.91) & (0.90) \\
Has Migrant Friends & 0.00 & -0.04 & -0.04 & -0.09 & -0.05 & 0.08 & 0.07 & 0.09 & 0.13 \\
\quad \textit{Unadjusted P-Value} & (1.00) & (0.61) & (0.57) & (0.29) & (0.61) & (0.32) & (0.34) & (0.19) & (0.08)** \\
\quad \textit{Stepdown P-Value} & (0.91) & (0.94) & (0.92) & (0.75) & (0.98) & (0.65) & (0.67) & (0.51) & (0.30) \\
Volunteers & 0.05 & 0.01 & 0.03 & -0.01 & -0.00 & -0.07 & -0.09 & -0.01 & -0.03 \\
\quad \textit{Unadjusted P-Value} & (0.23) & (0.75) & (0.54) & (0.93) & (0.96) & (0.21) & (0.12)* & (0.88) & (0.50) \\
\quad \textit{Stepdown P-Value} & (0.77) & (0.94) & (0.90) & (0.85) & (0.98) & (0.60) & (0.39) & (0.91) & (0.78) \\
Ever Voted for Municipal & -0.07 & 0.07 & 0.06 & 0.09 & 0.04 & 0.14 & 0.13 & 0.07 & 0.07 \\
\quad \textit{Unadjusted P-Value} & (0.38) & (0.28) & (0.38) & (0.15)* & (0.69) & (0.05)*** & (0.08)** & (0.30) & (0.36) \\
\quad \textit{Stepdown P-Value} & (0.88) & (0.80) & (0.81) & (0.60) & (0.98) & (0.21) & (0.37) & (0.65) & (0.78) \\
Ever Voted for Regional & -0.05 & 0.08 & 0.07 & 0.10 & 0.04 & 0.27 & 0.25 & 0.14 & 0.19 \\
\quad \textit{Unadjusted P-Value} & (0.53) & (0.23) & (0.34) & (0.17) & (0.67) & (0.00)*** & (0.00)*** & (0.05)** & (0.01)*** \\
\quad \textit{Stepdown P-Value} & (0.91) & (0.76) & (0.77) & (0.60) & (0.98) & (0.00)*** & (0.00)*** & (0.26) & (0.06)** \\
\bottomrule
\end{tabular}
}
\vspace{1ex} \\
\footnotesize\raggedright{\underline{Note1 :} This table shows the estimates of the coefficient for attending Reggio Approach preschools from multiple methods. We compare Reggio Approach individuals with those who attended other preschools. Column title indicates the corresponding control set and and model. \textbf{None} = OLS estimate with no control variables. \textbf{BIC} = OLS estimate with controls selected by Bayesian Information Criterion (BIC) and additional controls for male indicator, migrant indicator, and ITC attendance indicator. \textbf{Full} = OLS estimate with the full set of controls. \textbf{PSMR} =  propensity score matching between Reggio Approach people and people in Reggio who attended nother types of preschool. \textbf{KMR} = Epanechinikov kernel matching between Reggio Approach people and people in Reggio who attended nother types of preschool. \textbf{PSMPm} = propensity score matching between Reggio Approach people and people who attended Parma preschools. \textbf{KMPm} = Epanechinikov kernel matching between Reggio Approach people and people who attended Parma preschools. \textbf{PSMPv} = propensity score matching between Reggio Approach people and people who attended Padova preschools. \textbf{KMPv} = Epanechinikov kernel matching between Reggio Approach people and people who attended Padova preschools. \textit{Difference-indifference is not available for this cohort due to non-existence of municipal preschools in Parma and Padova.}} 

\footnotesize\raggedright{\underline{Note2 :} Both unadjusted p-value and stepdown p-value are reported. ***, **, and * indicate significance of the coefficients at the 15\%, 10\%, and 5\% levels respectively.}
\end{table}

\begin{table}[H] \caption{Estimation Results for Social Outcomes, Comparison to No Preschool, Age-40 Cohort} \label{ols-S-adult40-reg-none}
\scalebox{0.6}{\begin{tabular}{l c c c c c c c c c c c}
\toprule
& \multicolumn{5}{c}{Within Reggio} & \multicolumn{3}{c}{With Parma} & \multicolumn{3}{c}{With Padova} \\\cmidrule(lr){2-6} \cmidrule(lr){7-9} \cmidrule(lr){10-12}
 & None & BIC & Full & PSMR & KMR & DidPm & PSMPm & KMPm & DidPv & PSMPv & KMPv \\
\midrule
Bothered by Migrants & -0.07 & -0.01 & -0.00 & 0.04 & -0.01 & -0.19 & -0.25 & -0.15 & 0.13 & 0.05 & 0.13 \\
\quad \textit{Unadjusted P-Value} & (0.45) & (0.89) & (0.98) & (0.67) & (0.93) & (0.42) & (0.01)*** & (0.10)* & (0.56) & (0.71) & (0.32) \\
\quad \textit{Stepdown P-Value} & (0.65) & (0.98) & (0.99) & (0.88) & (0.92) & (0.90) & (0.06)** & (0.16) & (0.97) & (0.71) & (0.75) \\
Num. of Friends & -0.68 & -0.07 & 0.75 & -0.13 & -0.48 & 2.17 & -2.25 & -4.77 & 0.35 & -1.92 & -0.84 \\
\quad \textit{Unadjusted P-Value} & (0.52) & (0.95) & (0.61) & (0.92) & (0.67) & (0.42) & (0.16) & (0.00)*** & (0.90) & (0.09)** & (0.61) \\
\quad \textit{Stepdown P-Value} & (0.65) & (0.98) & (0.71) & (0.92) & (0.88) & (0.90) & (0.34) & (0.00)*** & (0.97) & (0.32) & (0.80) \\
Has Migrant Friends & -0.13 & -0.11 & -0.12 & -0.09 & -0.11 & 0.03 & -0.11 & -0.10 & -0.46 & 0.11 & 0.10 \\
\quad \textit{Unadjusted P-Value} & (0.05)*** & (0.10)** & (0.13)* & (0.23) & (0.13)* & (0.87) & (0.14)* & (0.17) & (0.00)*** & (0.23) & (0.29) \\
\quad \textit{Stepdown P-Value} & (0.20) & (0.35) & (0.31) & (0.59) & (0.40) & (0.90) & (0.34) & (0.18) & (0.02)*** & (0.53) & (0.75) \\
Volunteers & -0.11 & -0.08 & -0.11 & -0.07 & -0.07 & 0.11 & -0.09 & -0.14 & -0.06 & 0.06 & 0.02 \\
\quad \textit{Unadjusted P-Value} & (0.05)*** & (0.16) & (0.13)* & (0.30) & (0.29) & (0.37) & (0.21) & (0.03)*** & (0.56) & (0.26) & (0.71) \\
\quad \textit{Stepdown P-Value} & (0.20) & (0.43) & (0.23) & (0.63) & (0.63) & (0.90) & (0.34) & (0.06)** & (0.97) & (0.53) & (0.80) \\
Ever Voted for Municipal & 0.19 & 0.15 & 0.11 & 0.17 & 0.19 & 0.08 & 0.35 & 0.32 & -0.06 & 0.36 & 0.41 \\
\quad \textit{Unadjusted P-Value} & (0.02)*** & (0.05)** & (0.18) & (0.08)** & (0.03)*** & (0.61) & (0.00)*** & (0.00)*** & (0.68) & (0.00)*** & (0.00)*** \\
\quad \textit{Stepdown P-Value} & (0.07)** & (0.20) & (0.47) & (0.35) & (0.12) & (0.90) & (0.00)*** & (0.00)*** & (0.97) & (0.00)*** & (0.00)*** \\
Ever Voted for Regional & 0.20 & 0.16 & 0.13 & 0.18 & 0.20 & 0.15 & 0.43 & 0.41 & -0.09 & 0.39 & 0.41 \\
\quad \textit{Unadjusted P-Value} & (0.01)*** & (0.04)*** & (0.14)* & (0.07)** & (0.02)*** & (0.32) & (0.00)*** & (0.00)*** & (0.54) & (0.00)*** & (0.00)*** \\
\quad \textit{Stepdown P-Value} & (0.06)** & (0.17) & (0.41) & (0.33) & (0.12) & (0.89) & (0.00)*** & (0.00)*** & (0.97) & (0.00)*** & (0.00)*** \\
\bottomrule
\end{tabular}
}
\vspace{1ex} \\
\footnotesize\raggedright{\underline{Note1 :} This table shows the estimates of the coefficient for attending Reggio Approach preschools from multiple methods. We compare Reggio Approach individuals with those who did not attend any preschools. Column title indicates the corresponding control set and and model. \textbf{None} = OLS estimate with no control variables. \textbf{BIC} = OLS estimate with controls selected by Bayesian Information Criterion (BIC) and additional controls for male indicator, migrant indicator, and ITC attendance indicator. \textbf{Full} = OLS estimate with the full set of controls. \textbf{PSMR} =  propensity score matching between Reggio Approach people and people in Reggio who did not attend any preschool. \textbf{KMR} = Epanechinikov kernel matching between Reggio Approach people and people in Reggio who attended other types of preschool. \textbf{DidPm} = difference-in-difference estimate of (Reggio Muni - Parma Other) - (Reggio None - Parma None). \textbf{PSMPm} = propensity score matching between Reggio Approach people and people in Parma who did not attend any preschool. \textbf{KMPm} = Epanechinikov kernel matching between Reggio Approach people and people in Parma who did not attend any preschool. \textbf{DidPv} = difference-in-difference estimate of (Reggio Muni - Padova Other) - (Reggio None - Padova None). \textbf{PSMPv} = propensity score matching between Reggio Approach people and people in Padova who did not attend any preschool. \textbf{KMPv} = Epanechinikov kernel matching between Reggio Approach people and people in Padova who did not attend any preschool.} 

\footnotesize\raggedright{\underline{Note2 :} Both unadjusted p-value and stepdown p-value are reported. ***, **, and * indicate significance of the coefficients at the 15\%, 10\%, and 5\% levels respectively.}
\end{table}
















\subsection{Estimation Results for Preschools in Parma}

% ========================================================================= %
% CHILD COHORT



\begin{table}[H] \caption{Estimation Results for Main Outcomes, Preschool vs. No Preschool, Adult 30s Cohort in Parma} \label{ols-M-adult30-reg-pres-parma}
\scalebox{0.7}{\begin{tabular}{l c c c c c}
\toprule
 & None & BIC & Full & PSM & AIPW \\
\midrule
IQ Score & \textbf{      0.06 } & \textbf{      0.07 } & \textbf{      0.07 } &      0.02 & \textbf{     0.10} \\
& (     0.04 ) & (     0.04 ) & (     0.03 ) & (     0.03 ) & (     0.04 ) \\
& \textit{ 187 } & \textit{ 187 } & \textit{ 187 } & \textit{ 187 } & \textit{ 187 } \\
IQ Factor & \textbf{      0.16 } & \textbf{      0.20 } & \textbf{      0.18 } &      0.08 & \textbf{     0.24} \\
& (     0.11 ) & (     0.11 ) & (     0.10 ) & (     0.09 ) & (     0.05 ) \\
& \textit{ 187 } & \textit{ 187 } & \textit{ 187 } & \textit{ 187 } & \textit{ 187 } \\
Graduate from High School &      0.04 &     -0.02 &     -0.04 &     -0.05 &     -0.15 \\
& (     0.06 ) & (     0.06 ) & (     0.06 ) & (     0.04 ) & (     0.15 ) \\
& \textit{ 187 } & \textit{ 187 } & \textit{ 187 } & \textit{ 187 } & \textit{ 187 } \\
High School Grade & \textbf{      7.56 } & \textbf{      7.92 } & \textbf{      7.40 } &      4.08 & \textbf{     4.60} \\
& (     3.33 ) & (     2.75 ) & (     2.84 ) & (     3.05 ) & (     2.95 ) \\
& \textit{ 164 } & \textit{ 164 } & \textit{ 164 } & \textit{ 164 } & \textit{ 164 } \\
High School Grade (Standardized) & \textbf{      4.75 } & \textbf{      5.22 } & \textbf{      4.98 } &      2.53 &      2.06 \\
& (     2.55 ) & (     2.14 ) & (     2.22 ) & (     1.66 ) & (     2.39 ) \\
& \textit{ 159 } & \textit{ 159 } & \textit{ 159 } & \textit{ 159 } & \textit{ 159 } \\
Max Edu: University &      0.05 &     -0.04 &     -0.07 &     -0.01 &     -0.48 \\
& (     0.08 ) & (     0.08 ) & (     0.08 ) & (     0.08 ) & (     0.13 ) \\
& \textit{ 187 } & \textit{ 187 } & \textit{ 187 } & \textit{ 187 } & \textit{ 187 } \\
Employed &     -0.03 &     -0.02 &     -0.03 &     -0.04 &      0.04 \\
& (     0.05 ) & (     0.05 ) & (     0.04 ) & (     0.05 ) & (     0.13 ) \\
& \textit{ 187 } & \textit{ 187 } & \textit{ 187 } & \textit{ 187 } & \textit{ 187 } \\
Hours Worked Per Week &     -2.10 &     -1.43 &     -1.91 &     -2.12 &     -5.91 \\
& (     2.33 ) & (     2.31 ) & (     2.26 ) & (     2.44 ) & (     6.85 ) \\
& \textit{ 184 } & \textit{ 184 } & \textit{ 184 } & \textit{ 184 } & \textit{ 184 } \\
Married or Cohabitating &      0.04 &      0.06 &      0.08 &      0.04 &     -0.25 \\
& (     0.08 ) & (     0.09 ) & (     0.09 ) & (     0.11 ) & (     0.18 ) \\
& \textit{ 187 } & \textit{ 187 } & \textit{ 187 } & \textit{ 187 } & \textit{ 187 } \\
Obese & \textbf{      0.08 } & \textbf{      0.12 } & \textbf{      0.12 } & \textbf{     0.11} &      0.01 \\
& (     0.05 ) & (     0.06 ) & (     0.06 ) & (     0.05 ) & (     0.15 ) \\
& \textit{ 187 } & \textit{ 187 } & \textit{ 187 } & \textit{ 187 } & \textit{ 187 } \\
Overweight &     -0.10 &     -0.08 &     -0.08 &     -0.01 &     -0.10 \\
& (     0.08 ) & (     0.08 ) & (     0.08 ) & (     0.08 ) & (     0.13 ) \\
& \textit{ 187 } & \textit{ 187 } & \textit{ 187 } & \textit{ 187 } & \textit{ 187 } \\
Locus of Control - positive & \textbf{      0.36 } & \textbf{      0.34 } & \textbf{      0.30 } & \textbf{     0.33} &     -0.16 \\
& (     0.16 ) & (     0.15 ) & (     0.15 ) & (     0.14 ) & (     0.57 ) \\
& \textit{ 172 } & \textit{ 172 } & \textit{ 172 } & \textit{ 172 } & \textit{ 172 } \\
Depression Score - positive &      0.51 &     -0.13 &     -0.31 &      0.74 &     -3.72 \\
& (     0.98 ) & (     1.03 ) & (     1.02 ) & (     1.10 ) & (     1.62 ) \\
& \textit{ 187 } & \textit{ 187 } & \textit{ 187 } & \textit{ 187 } & \textit{ 187 } \\
Ever Voted for Municipal & \textbf{      0.15 } & \textbf{      0.10 } & \textbf{      0.08 } &      0.06 &     -0.21 \\
& (     0.06 ) & (     0.05 ) & (     0.05 ) & (     0.04 ) & (     0.04 ) \\
& \textit{ 185 } & \textit{ 185 } & \textit{ 185 } & \textit{ 185 } & \textit{ 185 } \\
Ever Voted for Regional & \textbf{      0.13 } & \textbf{      0.08 } &      0.06 &      0.03 &     -0.20 \\
& (     0.06 ) & (     0.05 ) & (     0.05 ) & (     0.04 ) & (     0.05 ) \\
& \textit{ 185 } & \textit{ 185 } & \textit{ 185 } & \textit{ 185 } & \textit{ 185 } \\
\bottomrule
\end{tabular}
}
\vspace{1ex} \\
\footnotesize\raggedright{Note: This table shows the estimates of the coefficient for attending preschools in Parma from multiple methods. We compare people in Parma who attended preschools with people in Parma who attended no preschools. Column title indicates the corresponding control set and and model. ``None'' refers to the OLS estimate with no control variables. ``BIC'' refers to the OLS estimate with controls selected by Bayesian Information Criterion (BIC) and additional controls for caregiver's religion. ``Full'' refers to the OLS estimate with the full set of controls. ``PSM'' refers to propensity score matching estimation. ``AIPW'' refers to the augmented inverse propensity weighting estimation. Robust standard errors are reported in parentheses. Bold number shows that the estimate is statistically significant at the 10\% level. Number of observations used in estimation is reported in italic.}

\end{table}




\begin{table}[H] \caption{Estimation Results for Main Outcomes, Preschool vs. No Preschool, Adult 40s Cohort in Parma} \label{ols-M-adult40-reg-pres-parma}
\scalebox{0.7}{\begin{tabular}{l c c c c c}
\toprule
 & None & BIC & Full & PSM & AIPW \\
\midrule
IQ Score &     -0.03 &     -0.03 &     -0.02 &     -0.02 &      0.00 \\
& (     0.03 ) & (     0.03 ) & (     0.03 ) & (     0.03 ) & (     0.00 ) \\
& \textit{ 222 } & \textit{ 222 } & \textit{ 222 } & \textit{ 222 } & \textit{ 222 } \\
IQ Factor &     -0.10 &     -0.10 &     -0.07 &     -0.05 &      0.00 \\
& (     0.08 ) & (     0.08 ) & (     0.08 ) & (     0.09 ) & (     0.00 ) \\
& \textit{ 222 } & \textit{ 222 } & \textit{ 222 } & \textit{ 222 } & \textit{ 222 } \\
Graduate from High School &      0.02 &      0.02 &     -0.03 &     -0.02 &      0.00 \\
& (     0.05 ) & (     0.05 ) & (     0.05 ) & (     0.05 ) & (     0.00 ) \\
& \textit{ 222 } & \textit{ 222 } & \textit{ 222 } & \textit{ 222 } & \textit{ 222 } \\
High School Grade & \textbf{      7.99 } & \textbf{      7.99 } & \textbf{      4.49 } &      3.32 &      0.00 \\
& (     1.91 ) & (     1.91 ) & (     1.85 ) & (     3.61 ) & (     0.00 ) \\
& \textit{ 186 } & \textit{ 186 } & \textit{ 186 } & \textit{ 186 } & \textit{ 186 } \\
High School Grade (Standardized) & \textbf{      4.48 } & \textbf{      4.48 } & \textbf{      2.70 } & \textbf{     3.68} &      0.00 \\
& (     1.31 ) & (     1.31 ) & (     1.26 ) & (     1.28 ) & (     0.00 ) \\
& \textit{ 180 } & \textit{ 180 } & \textit{ 180 } & \textit{ 180 } & \textit{ 180 } \\
Max Edu: University & \textbf{      0.24 } & \textbf{      0.24 } & \textbf{      0.19 } & \textbf{     0.19} &      0.00 \\
& (     0.06 ) & (     0.06 ) & (     0.06 ) & (     0.06 ) & (     0.00 ) \\
& \textit{ 222 } & \textit{ 222 } & \textit{ 222 } & \textit{ 222 } & \textit{ 222 } \\
Employed &      0.01 &      0.01 &      0.02 &      0.03 &      0.00 \\
& (     0.03 ) & (     0.03 ) & (     0.03 ) & (     0.03 ) & (     0.00 ) \\
& \textit{ 222 } & \textit{ 222 } & \textit{ 222 } & \textit{ 222 } & \textit{ 222 } \\
Hours Worked Per Week &      1.92 &      1.92 &      1.57 &      1.39 &      0.00 \\
& (     1.43 ) & (     1.43 ) & (     1.60 ) & (     1.61 ) & (     0.00 ) \\
& \textit{ 215 } & \textit{ 215 } & \textit{ 215 } & \textit{ 215 } & \textit{ 215 } \\
Married or Cohabitating & \textbf{      0.12 } & \textbf{      0.12 } & \textbf{      0.13 } &      0.11 &      0.00 \\
& (     0.07 ) & (     0.07 ) & (     0.07 ) & (     0.07 ) & (     0.00 ) \\
& \textit{ 222 } & \textit{ 222 } & \textit{ 222 } & \textit{ 222 } & \textit{ 222 } \\
Obese & \textbf{      0.14 } & \textbf{      0.14 } & \textbf{      0.17 } & \textbf{     0.14} &      0.00 \\
& (     0.05 ) & (     0.05 ) & (     0.05 ) & (     0.05 ) & (     0.00 ) \\
& \textit{ 222 } & \textit{ 222 } & \textit{ 222 } & \textit{ 222 } & \textit{ 222 } \\
Overweight &     -0.03 &     -0.03 &     -0.01 &     -0.01 &      0.00 \\
& (     0.06 ) & (     0.06 ) & (     0.06 ) & (     0.07 ) & (     0.00 ) \\
& \textit{ 222 } & \textit{ 222 } & \textit{ 222 } & \textit{ 222 } & \textit{ 222 } \\
Locus of Control - positive &      0.10 &      0.10 &      0.06 &      0.01 &      0.00 \\
& (     0.12 ) & (     0.12 ) & (     0.12 ) & (     0.12 ) & (     0.00 ) \\
& \textit{ 209 } & \textit{ 209 } & \textit{ 209 } & \textit{ 209 } & \textit{ 209 } \\
Depression Score - positive & \textbf{      2.20 } & \textbf{      2.20 } & \textbf{      1.93 } & \textbf{     1.90} &      0.00 \\
& (     0.70 ) & (     0.70 ) & (     0.68 ) & (     0.70 ) & (     0.00 ) \\
& \textit{ 222 } & \textit{ 222 } & \textit{ 222 } & \textit{ 222 } & \textit{ 222 } \\
Ever Voted for Municipal & \textbf{      0.30 } & \textbf{      0.30 } & \textbf{      0.18 } & \textbf{     0.20} &      0.00 \\
& (     0.06 ) & (     0.06 ) & (     0.06 ) & (     0.07 ) & (     0.00 ) \\
& \textit{ 222 } & \textit{ 222 } & \textit{ 222 } & \textit{ 222 } & \textit{ 222 } \\
Ever Voted for Regional & \textbf{      0.23 } & \textbf{      0.23 } & \textbf{      0.14 } & \textbf{     0.13} &      0.00 \\
& (     0.05 ) & (     0.05 ) & (     0.05 ) & (     0.07 ) & (     0.00 ) \\
& \textit{ 222 } & \textit{ 222 } & \textit{ 222 } & \textit{ 222 } & \textit{ 222 } \\
\bottomrule
\end{tabular}
}
\vspace{1ex} \\
\footnotesize\raggedright{Note: This table shows the estimates of the coefficient for attending preschools in Parma from multiple methods. We compare people in Parma who attended preschools with people in Parma who attended no preschools. Column title indicates the corresponding control set and and model. ``None'' refers to the OLS estimate with no control variables. ``BIC'' refers to the OLS estimate with controls selected by Bayesian Information Criterion (BIC) and additional controls for caregiver's religion. ``Full'' refers to the OLS estimate with the full set of controls. ``PSM'' refers to propensity score matching estimation. ``AIPW'' refers to the augmented inverse propensity weighting estimation. Robust standard errors are reported in parentheses. Bold number shows that the estimate is statistically significant at the 10\% level. Number of observations used in estimation is reported in italic.}

\end{table}




\subsection{Estimation Results for Preschools in Padova} \label{subsection:padova-estimation}

% ========================================================================= %
% CHILD COHORT



\begin{table}[H] \caption{Estimation Results for Main Outcomes, Preschool vs. No Preschool, Adult 30s Cohort in Padova} \label{ols-M-adult30-reg-pres-padova}
\scalebox{0.7}{\begin{tabular}{l c c c c c}
\toprule
 & None & BIC & Full & PSM & AIPW \\
\midrule
IQ Score & \textbf{      0.10 } & \textbf{      0.10 } & \textbf{      0.10 } & \textbf{     0.10} &      0.00 \\
& (     0.04 ) & (     0.04 ) & (     0.03 ) & (     0.03 ) & (     0.00 ) \\
& \textit{ 219 } & \textit{ 219 } & \textit{ 219 } & \textit{ 219 } & \textit{ 219 } \\
IQ Factor & \textbf{      0.25 } & \textbf{      0.28 } & \textbf{      0.25 } & \textbf{     0.27} &      0.00 \\
& (     0.12 ) & (     0.11 ) & (     0.10 ) & (     0.09 ) & (     0.00 ) \\
& \textit{ 219 } & \textit{ 219 } & \textit{ 219 } & \textit{ 219 } & \textit{ 219 } \\
Graduate from High School &      0.01 &      0.00 &     -0.01 &      0.05 &      0.00 \\
& (     0.05 ) & (     0.05 ) & (     0.05 ) & (     0.06 ) & (     0.00 ) \\
& \textit{ 219 } & \textit{ 219 } & \textit{ 219 } & \textit{ 219 } & \textit{ 219 } \\
High School Grade &      0.33 &      0.34 &      1.15 &      2.24 &      0.00 \\
& (     2.02 ) & (     2.07 ) & (     1.98 ) & (     1.54 ) & (     0.00 ) \\
& \textit{ 171 } & \textit{ 171 } & \textit{ 171 } & \textit{ 171 } & \textit{ 171 } \\
High School Grade (Standardized) &     -0.49 &     -0.52 &      0.27 &      0.23 &      0.00 \\
& (     1.97 ) & (     2.11 ) & (     2.07 ) & (     1.97 ) & (     0.00 ) \\
& \textit{ 169 } & \textit{ 169 } & \textit{ 169 } & \textit{ 169 } & \textit{ 169 } \\
Max Edu: University & \textbf{      0.25 } & \textbf{      0.24 } & \textbf{      0.24 } & \textbf{     0.24} &      0.00 \\
& (     0.07 ) & (     0.07 ) & (     0.08 ) & (     0.07 ) & (     0.00 ) \\
& \textit{ 219 } & \textit{ 219 } & \textit{ 219 } & \textit{ 219 } & \textit{ 219 } \\
Employed &     -0.03 &     -0.01 &      0.01 &      0.10 &      0.00 \\
& (     0.05 ) & (     0.05 ) & (     0.05 ) & (     0.11 ) & (     0.00 ) \\
& \textit{ 219 } & \textit{ 219 } & \textit{ 219 } & \textit{ 219 } & \textit{ 219 } \\
Hours Worked Per Week &     -0.16 &      1.33 &      1.74 &      4.28 &      0.00 \\
& (     2.23 ) & (     2.34 ) & (     2.45 ) & (     4.36 ) & (     0.00 ) \\
& \textit{ 215 } & \textit{ 215 } & \textit{ 215 } & \textit{ 215 } & \textit{ 215 } \\
Married or Cohabitating &      0.05 &      0.08 &      0.10 &      0.15 &      0.00 \\
& (     0.08 ) & (     0.08 ) & (     0.08 ) & (     0.13 ) & (     0.00 ) \\
& \textit{ 219 } & \textit{ 219 } & \textit{ 219 } & \textit{ 219 } & \textit{ 219 } \\
Obese & \textbf{     -0.18 } & \textbf{     -0.18 } & \textbf{     -0.17 } & \textbf{    -0.21} &      0.00 \\
& (     0.08 ) & (     0.08 ) & (     0.08 ) & (     0.11 ) & (     0.00 ) \\
& \textit{ 219 } & \textit{ 219 } & \textit{ 219 } & \textit{ 219 } & \textit{ 219 } \\
Overweight &     -0.02 &      0.05 &      0.01 &      0.07 &      0.00 \\
& (     0.07 ) & (     0.07 ) & (     0.07 ) & (     0.11 ) & (     0.00 ) \\
& \textit{ 219 } & \textit{ 219 } & \textit{ 219 } & \textit{ 219 } & \textit{ 219 } \\
Locus of Control - positive &      0.08 &      0.14 &      0.20 &      0.18 &      0.00 \\
& (     0.14 ) & (     0.13 ) & (     0.14 ) & (     0.19 ) & (     0.00 ) \\
& \textit{ 206 } & \textit{ 206 } & \textit{ 206 } & \textit{ 206 } & \textit{ 206 } \\
Depression Score - positive &      1.27 & \textbf{      1.97 } & \textbf{      2.11 } & \textbf{     3.21} &      0.00 \\
& (     0.92 ) & (     0.90 ) & (     0.94 ) & (     1.39 ) & (     0.00 ) \\
& \textit{ 216 } & \textit{ 216 } & \textit{ 216 } & \textit{ 216 } & \textit{ 216 } \\
Ever Voted for Municipal & \textbf{      0.32 } & \textbf{      0.30 } & \textbf{      0.31 } & \textbf{     0.33} &      0.00 \\
& (     0.07 ) & (     0.07 ) & (     0.07 ) & (     0.07 ) & (     0.00 ) \\
& \textit{ 205 } & \textit{ 205 } & \textit{ 205 } & \textit{ 205 } & \textit{ 205 } \\
Ever Voted for Regional & \textbf{      0.29 } & \textbf{      0.27 } & \textbf{      0.27 } & \textbf{     0.32} &      0.00 \\
& (     0.07 ) & (     0.07 ) & (     0.07 ) & (     0.07 ) & (     0.00 ) \\
& \textit{ 205 } & \textit{ 205 } & \textit{ 205 } & \textit{ 205 } & \textit{ 205 } \\
\bottomrule
\end{tabular}
}
\vspace{1ex} \\
\footnotesize\raggedright{Note: This table shows the estimates of the coefficient for attending preschools in Padova from multiple methods. We compare people in Padova who attended preschools with people in Padova who attended no preschools. Column title indicates the corresponding control set and and model. ``None'' refers to the OLS estimate with no control variables. ``BIC'' refers to the OLS estimate with controls selected by Bayesian Information Criterion (BIC) and additional controls for caregiver's religion. ``Full'' refers to the OLS estimate with the full set of controls. ``PSM'' refers to propensity score matching estimation. ``AIPW'' refers to the augmented inverse propensity weighting estimation. Robust standard errors are reported in parentheses. Bold number shows that the estimate is statistically significant at the 10\% level. Number of observations used in estimation is reported in italic.}

\end{table}




\begin{table}[H] \caption{Estimation Results for Main Outcomes, Preschool vs. No Preschool, Adult 40s Cohort in Padova} \label{ols-M-adult40-reg-pres-padova}
\scalebox{0.7}{\begin{tabular}{l c c c c c}
\toprule
 & None & BIC & Full & PSM & AIPW \\
\midrule
IQ Score &     -0.01 &     -0.01 &      0.01 &      0.00 &      0.00 \\
& (     0.03 ) & (     0.03 ) & (     0.02 ) & (     0.02 ) & (     0.00 ) \\
& \textit{ 224 } & \textit{ 224 } & \textit{ 224 } & \textit{ 224 } & \textit{ 224 } \\
IQ Factor & \textbf{     -0.14 } & \textbf{     -0.14 } &     -0.06 &     -0.09 &      0.00 \\
& (     0.07 ) & (     0.07 ) & (     0.07 ) & (     0.06 ) & (     0.00 ) \\
& \textit{ 224 } & \textit{ 224 } & \textit{ 224 } & \textit{ 224 } & \textit{ 224 } \\
Graduate from High School &      0.04 &      0.04 & \textbf{      0.09 } &      0.03 &      0.00 \\
& (     0.06 ) & (     0.06 ) & (     0.05 ) & (     0.07 ) & (     0.00 ) \\
& \textit{ 224 } & \textit{ 224 } & \textit{ 224 } & \textit{ 224 } & \textit{ 224 } \\
High School Grade &     -1.61 &     -1.61 &     -0.93 &     -1.72 &      0.00 \\
& (     1.72 ) & (     1.72 ) & (     1.82 ) & (     1.99 ) & (     0.00 ) \\
& \textit{ 178 } & \textit{ 178 } & \textit{ 178 } & \textit{ 178 } & \textit{ 178 } \\
High School Grade (Standardized) &     -2.03 &     -2.03 &     -1.56 &     -2.15 &      0.00 \\
& (     1.81 ) & (     1.81 ) & (     1.99 ) & (     2.18 ) & (     0.00 ) \\
& \textit{ 178 } & \textit{ 178 } & \textit{ 178 } & \textit{ 178 } & \textit{ 178 } \\
Max Edu: University &      0.06 &      0.06 & \textbf{      0.12 } & \textbf{     0.13} &      0.00 \\
& (     0.07 ) & (     0.07 ) & (     0.06 ) & (     0.06 ) & (     0.00 ) \\
& \textit{ 224 } & \textit{ 224 } & \textit{ 224 } & \textit{ 224 } & \textit{ 224 } \\
Employed &     -0.03 &     -0.03 &     -0.03 &     -0.04 &      0.00 \\
& (     0.04 ) & (     0.04 ) & (     0.04 ) & (     0.04 ) & (     0.00 ) \\
& \textit{ 224 } & \textit{ 224 } & \textit{ 224 } & \textit{ 224 } & \textit{ 224 } \\
Hours Worked Per Week &     -2.19 &     -2.19 &     -0.51 &     -1.32 &      0.00 \\
& (     1.90 ) & (     1.90 ) & (     1.93 ) & (     1.89 ) & (     0.00 ) \\
& \textit{ 223 } & \textit{ 223 } & \textit{ 223 } & \textit{ 223 } & \textit{ 223 } \\
Married or Cohabitating & \textbf{      0.13 } & \textbf{      0.13 } & \textbf{      0.15 } &      0.11 &      0.00 \\
& (     0.07 ) & (     0.07 ) & (     0.07 ) & (     0.08 ) & (     0.00 ) \\
& \textit{ 224 } & \textit{ 224 } & \textit{ 224 } & \textit{ 224 } & \textit{ 224 } \\
Obese &     -0.05 &     -0.05 &     -0.07 &     -0.05 &      0.00 \\
& (     0.06 ) & (     0.06 ) & (     0.06 ) & (     0.07 ) & (     0.00 ) \\
& \textit{ 224 } & \textit{ 224 } & \textit{ 224 } & \textit{ 224 } & \textit{ 224 } \\
Overweight &     -0.08 &     -0.08 &     -0.07 &     -0.04 &      0.00 \\
& (     0.06 ) & (     0.06 ) & (     0.05 ) & (     0.07 ) & (     0.00 ) \\
& \textit{ 224 } & \textit{ 224 } & \textit{ 224 } & \textit{ 224 } & \textit{ 224 } \\
Locus of Control - positive & \textbf{     -0.23 } & \textbf{     -0.23 } &     -0.16 &     -0.16 &      0.00 \\
& (     0.11 ) & (     0.11 ) & (     0.12 ) & (     0.12 ) & (     0.00 ) \\
& \textit{ 211 } & \textit{ 211 } & \textit{ 211 } & \textit{ 211 } & \textit{ 211 } \\
Depression Score - positive &     -0.21 &     -0.21 &      0.16 &     -0.11 &      0.00 \\
& (     0.72 ) & (     0.72 ) & (     0.75 ) & (     0.78 ) & (     0.00 ) \\
& \textit{ 222 } & \textit{ 222 } & \textit{ 222 } & \textit{ 222 } & \textit{ 222 } \\
Ever Voted for Municipal & \textbf{      0.30 } & \textbf{      0.30 } & \textbf{      0.25 } & \textbf{     0.28} &      0.00 \\
& (     0.06 ) & (     0.06 ) & (     0.06 ) & (     0.08 ) & (     0.00 ) \\
& \textit{ 203 } & \textit{ 203 } & \textit{ 203 } & \textit{ 203 } & \textit{ 203 } \\
Ever Voted for Regional & \textbf{      0.29 } & \textbf{      0.29 } & \textbf{      0.26 } & \textbf{     0.24} &      0.00 \\
& (     0.06 ) & (     0.06 ) & (     0.06 ) & (     0.07 ) & (     0.00 ) \\
& \textit{ 203 } & \textit{ 203 } & \textit{ 203 } & \textit{ 203 } & \textit{ 203 } \\
\bottomrule
\end{tabular}
}
\vspace{1ex} \\
\footnotesize\raggedright{Note: This table shows the estimates of the coefficient for attending preschools in Padova from multiple methods. We compare people in Padova who attended preschools with people in Padova who attended no preschools. Column title indicates the corresponding control set and and model. ``None'' refers to the OLS estimate with no control variables. ``BIC'' refers to the OLS estimate with controls selected by Bayesian Information Criterion (BIC) and additional controls for caregiver's religion. ``Full'' refers to the OLS estimate with the full set of controls. ``PSM'' refers to propensity score matching estimation. ``AIPW'' refers to the augmented inverse propensity weighting estimation. Robust standard errors are reported in parentheses. Bold number shows that the estimate is statistically significant at the 10\% level. Number of observations used in estimation is reported in italic.}

\end{table}








