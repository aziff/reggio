We present the estimates of the methods described above for a handful of key outcomes.\footnote{We choose outcomes that are economically significant,  outcomes that have limited missing values, and outcomes with sufficient variation across individuals. Results on the full set of outcomes are reported in Appendix~\ref{appsec:extended-outcome}.}\footnote{A brief description of the outcomes is as follows: We rescale non-cognitive outcomes, including SDQ (Strengths and Difficulties Questionnaire) score, Locus of Control, and Depression score, so that the higher value has a more socially positive meaning; SDQ Composite - Child is reported by mother, and SDQ Composite is self-reported; IQ Score is measured using Raven's Progressive Matrices; How Much Child Likes School is a single question with three answers, where 1 means ``A little'', 2 means ``So so'', and 3 means ``A lot''; High School Grade has the maximum scoring of 100; since the mean and variance is not always the same, we standardize the high school grade for each city, cohort, and high school type based on our data to have mean zero and unit variance; All the other measures reported in the estimation results are binary indicators.} In addition to unadjusted p-values, we report step-down p-values for each set of estimates to account for the potential problem that arises from arbitrarily selecting ``significant'' results from a set of possible outcomes. We first present the results from the analysis of infant-toddler care. The results are not consistently statistically significant with some negative effects appearing for the older cohorts. We then present the results from our analysis of the preschool data. Although these results are stronger than those from the infant-toddler care, very few outcomes show statistically significant treatment effects that are robust across different estimation procedures. The strongest results are from the comparison of Reggio Approach preschool against no preschool for the age-40 cohort.

\subsection{Infant-Toddler Care}

Tables~\ref{ols-M-child-reg-nopres-asilo} to~\ref{ols-M-adult40-reg-nopres-asilo} show estimates of the treatment effect of Reggio Approach infant-toddler care compared to no treatment. The results that are robustly significant across different methods are as follows. In the child cohort, Reggio Approach infant-toddler centers had significantly positive effect on IQ, obesity, and number of friends relative to no infant-toddler care in Reggio Emilia. However, the effect on IQ does not persist when compared to people who had municipal infant-toddler care in Parma or Padova. The effect on number of friends persists in a comparison to Parma children. In the adolescent cohort, Reggio Approach infant-toddler care had significantly positive effect on number of friends, which persists even compared to Parma adolescents, but did not have a clear effect relative to no infant-toddler care on all other outcomes. In the age-30 cohort, Reggio Approach infant-toddler care had a significantly negative effect on IQ, high school grade, university graduation, volunteer behavior, number of friends, and trust score. However, Reggio Approach infant-toddler centers had a significantly positive effect on employment status, hours worked per week, obesity,  marriage, obesity, and voting behaviors. In the age-40 cohort, Reggio Approach also had a significantly negative effect on IQ, volunteer behavior, and number of friends. A positive effect was found for employment and hours worked.

To summarize, we have a mixed positive and negative effects of Reggio Approach infant-toddler centers that are generally different for younger and older cohorts. Reggio Approach infant-toddler centers generally have positive effect on IQ and number of friends for younger cohorts. However, Reggio Approach infant-toddler center have a negative effect on IQ and number of friends for older cohorts, whereas they generally have an increasing effect on employment and hours worked for those cohorts. Accounting for multiple hypothesis testing weakens the inference further.

\begin{table}[H] \caption{Estimation Results for Main Outcomes, Comparison to No Infant-Toddler Care, Child Cohort} \label{ols-M-child-reg-nopres-asilo}
\scalebox{0.6}{\begin{tabular}{l c c c c c c c}
\toprule
 & None & BIC & Full & PSM & KM & DidPm & DidPv \\
\midrule
IQ Factor & 0.55 & 0.47 & 0.35 & 0.25 & 0.31 & 0.34 & -0.57 \\
\quad \textit{Unadjusted P-Value} & (0.00)*** & (0.01)*** & (0.06)** & (0.12)* & (0.18) & (0.20) & (0.10)** \\
\quad \textit{Stepdown P-Value} & (0.01)*** & (0.05)*** & (0.29) & (0.56) & (0.76) & (0.79) & (0.50) \\
SDQ Composite - Child & 0.55 & 1.04 & 0.95 & 1.02 & 1.24 & 1.19 & 0.26 \\
\quad \textit{Unadjusted P-Value} & (0.50) & (0.20) & (0.28) & (0.23) & (0.17) & (0.35) & (0.86) \\
\quad \textit{Stepdown P-Value} & (0.84) & (0.82) & (0.72) & (0.76) & (0.76) & (0.89) & (0.98) \\
Not Obese & 0.29 & 0.22 & 0.10 & 0.14 & 0.17 & 0.15 & -0.13 \\
\quad \textit{Unadjusted P-Value} & (0.00)*** & (0.02)*** & (0.33) & (0.08)** & (0.12)* & (0.27) & (0.44) \\
\quad \textit{Stepdown P-Value} & (0.00)*** & (0.11) & (0.72) & (0.47) & (0.64) & (0.82) & (0.91) \\
Not Overweight & -0.06 & -0.02 & 0.02 & -0.02 & -0.05 & 0.10 & -0.13 \\
\quad \textit{Unadjusted P-Value} & (0.30) & (0.69) & (0.80) & (0.73) & (0.38) & (0.29) & (0.09)** \\
\quad \textit{Stepdown P-Value} & (0.78) & (0.92) & (0.93) & (0.87) & (0.83) & (0.89) & (0.71) \\
Health is Good & -0.11 & -0.06 & -0.08 & -0.08 & -0.13 & -0.01 & -0.02 \\
\quad \textit{Unadjusted P-Value} & (0.20) & (0.45) & (0.48) & (0.40) & (0.18) & (0.96) & (0.92) \\
\quad \textit{Stepdown P-Value} & (0.71) & (0.88) & (0.72) & (0.83) & (0.76) & (0.94) & (0.98) \\
Not Excited to Learn & 0.00 & -0.01 & 0.01 & -0.02 & 0.01 & -0.07 & 0.11 \\
\quad \textit{Unadjusted P-Value} & (0.95) & (0.71) & (0.73) & (0.65) & (0.79) & (0.11)* & (0.09)** \\
\quad \textit{Stepdown P-Value} & (0.96) & (0.92) & (0.93) & (0.87) & (0.93) & (0.79) & (0.50) \\
Problems Sitting Still & -0.10 & -0.07 & -0.12 & -0.04 & -0.04 & -0.19 & -0.27 \\
\quad \textit{Unadjusted P-Value} & (0.15) & (0.30) & (0.12)* & (0.48) & (0.61) & (0.08)** & (0.05)*** \\
\quad \textit{Stepdown P-Value} & (0.54) & (0.84) & (0.33) & (0.86) & (0.92) & (0.56) & (0.18) \\
How Much Child Likes School & 0.11 & 0.08 & 0.01 & 0.09 & -0.02 & 0.15 & 0.20 \\
\quad \textit{Unadjusted P-Value} & (0.25) & (0.38) & (0.95) & (0.35) & (0.86) & (0.31) & (0.38) \\
\quad \textit{Stepdown P-Value} & (0.73) & (0.88) & (0.96) & (0.83) & (0.93) & (0.89) & (0.85) \\
Num. of Friends & 0.86 & 0.79 & 0.70 & 0.88 & 0.87 & 1.40 & 0.38 \\
\quad \textit{Unadjusted P-Value} & (0.01)*** & (0.02)*** & (0.12)* & (0.02)*** & (0.01)*** & (0.04)*** & (0.78) \\
\quad \textit{Stepdown P-Value} & (0.12) & (0.23) & (0.43) & (0.16) & (0.07)** & (0.35) & (0.97) \\
Candy Game: Willing to Share Candies & 0.01 & 0.06 & 0.00 & 0.09 & 0.08 & 0.09 & -0.05 \\
\quad \textit{Unadjusted P-Value} & (0.84) & (0.31) & (0.96) & (0.29) & (0.30) & (0.27) & (0.62) \\
\quad \textit{Stepdown P-Value} & (0.96) & (0.88) & (0.98) & (0.78) & (0.79) & (0.89) & (0.97) \\
\bottomrule
\end{tabular}
}
\vspace{1ex} \\
\footnotesize\raggedright{\underline{Note 1:} This table shows the estimates of the coefficient for attending Reggio Approach infant-toddler centers from multiple methods. We compare people who attended both municipal infant-toddler centers and preschools (1,1) with people who only attended municipal preschools and no infant-toddler center (0,1). As in Section Column title indicates the corresponding control set and and model.  \textbf{None} = within-Reggio Emilia OLS estimate with no control variables. \textbf{BIC} = within-Reggio Emilia OLS estimate with controls selected by Bayesian Information Criterion (BIC) and additional controls for male indicator and ITC attendance indicator. \textbf{Full} = within-Reggio Emilia OLS estimate with the full set of controls. \textbf{PSM} =  within-Reggio Emilia propensity score matching estimation. \textbf{KM} =  within-Reggio Emilia Epanechnikov kernel matching estimation. \textbf{DidPm} = difference-in-difference estimates of (Reggio (1,1) - Reggio (0,1)) - (Parma (1,1) - Parma (0,1)). \textbf{PSMPm} =  propensity score matching estimation between Reggio (1,1) and Parma (0,1). \textbf{KMPm} =  Epanechnikov kernel matching estimation between Reggio (1,1) and Parma (0,1).   \textbf{DidPv} = difference-in-difference estimates of (Reggio (1,1) - Reggio (0,1)) - (Padova (1,1) - Padova (0,1)). \textbf{PSMPv} =  propensity score matching estimation between Reggio (1,1) and Padova (0,1). \textbf{KMPv} =  Epanechnikov kernel matching estimation between Reggio (1,1) and Padova (0,1).}

\footnotesize\raggedright{\underline{Note 2:} Both unadjusted p-value and stepdown p-value are reported. ***, **, and * indicate significance of the coefficients at the 15\%, 10\%, and 5\% levels respectively. Empty cells show that the estimation cannot be carried out for that outcome.}
\end{table}

\begin{table}[H] \caption{Estimation Results for Main Outcomes, Comparison to No Infant-Toddler Care, Adolescent Cohort} \label{ols-M-adol-reg-nopres-asilo}
\scalebox{0.62}{\begin{tabular}{l c c c c c c c}
\toprule
 & None & BIC & Full & PSM & KM & DidPm & DidPv \\
\midrule
IQ Factor & 0.20 & 0.22 & 0.12 & 0.27 & 0.27 & 0.27 & -0.07 \\
\quad \textit{Unadjusted P-Value} & (0.30) & (0.22) & (0.52) & (0.13)* & (0.21) & (0.30) & (0.81) \\
\quad \textit{Stepdown P-Value} & (0.98) & (0.90) & (0.94) & (0.80) & (0.91) & (0.98) & (0.99) \\
SDQ Composite - Child & 1.35 & 0.89 & 1.71 & 0.82 & 1.04 & 0.52 & 0.41 \\
\quad \textit{Unadjusted P-Value} & (0.11)* & (0.32) & (0.07)** & (0.44) & (0.26) & (0.72) & (0.74) \\
\quad \textit{Stepdown P-Value} & (0.72) & (0.91) & (0.32) & (0.96) & (0.94) & (0.99) & (0.99) \\
SDQ Composite & -0.69 & -1.15 & -0.66 & 0.09 & -0.71 & -1.20 & -1.42 \\
\quad \textit{Unadjusted P-Value} & (0.48) & (0.24) & (0.53) & (0.93) & (0.51) & (0.42) & (0.34) \\
\quad \textit{Stepdown P-Value} & (0.99) & (0.91) & (0.94) & (0.99) & (0.94) & (0.99) & (0.98) \\
Depression Score - positive & -0.90 & -1.46 & -1.67 & -1.52 & -1.62 & -1.52 & -1.59 \\
\quad \textit{Unadjusted P-Value} & (0.46) & (0.23) & (0.17) & (0.18) & (0.23) & (0.37) & (0.40) \\
\quad \textit{Stepdown P-Value} & (0.99) & (0.91) & (0.82) & (0.89) & (0.93) & (0.99) & (0.99) \\
Locus of Control - positive & -0.13 & -0.17 & -0.16 & -0.19 & -0.23 & 0.02 & -0.59 \\
\quad \textit{Unadjusted P-Value} & (0.33) & (0.21) & (0.22) & (0.24) & (0.12)* & (0.91) & (0.00)*** \\
\quad \textit{Stepdown P-Value} & (0.99) & (0.90) & (0.86) & (0.89) & (0.82) & (0.99) & (0.08)** \\
Not Obese & 0.05 & 0.10 & 0.08 & 0.11 & 0.09 & -0.03 & 0.02 \\
\quad \textit{Unadjusted P-Value} & (0.55) & (0.16) & (0.27) & (0.20) & (0.29) & (0.80) & (0.87) \\
\quad \textit{Stepdown P-Value} & (0.99) & (0.86) & (0.86) & (0.89) & (0.94) & (0.99) & (0.99) \\
Not Overweight & 0.07 & 0.07 & 0.07 & 0.06 & 0.05 & 0.04 & 0.06 \\
\quad \textit{Unadjusted P-Value} & (0.14)* & (0.11)* & (0.14)* & (0.10)* & (0.34) & (0.63) & (0.22) \\
\quad \textit{Stepdown P-Value} & (0.43) & (0.44) & (0.25) & (0.76) & (0.94) & (0.99) & (0.88) \\
Health is Good & 0.03 & -0.01 & -0.00 & -0.03 & -0.03 & 0.02 & 0.14 \\
\quad \textit{Unadjusted P-Value} & (0.72) & (0.94) & (0.97) & (0.71) & (0.74) & (0.88) & (0.34) \\
\quad \textit{Stepdown P-Value} & (0.99) & (0.97) & (0.97) & (0.99) & (0.94) & (0.99) & (0.98) \\
Go To School & 0.03 & 0.02 & -0.01 & 0.01 & 0.02 & 0.07 & 0.03 \\
\quad \textit{Unadjusted P-Value} & (0.44) & (0.63) & (0.73) & (0.84) & (0.61) & (0.15)* & (0.62) \\
\quad \textit{Stepdown P-Value} & (0.99) & (0.97) & (0.94) & (0.99) & (0.94) & (0.85) & (0.99) \\
How Much Child Likes School & -0.06 & -0.10 & -0.13 & 0.01 & -0.17 & -0.37 & -0.30 \\
\quad \textit{Unadjusted P-Value} & (0.75) & (0.62) & (0.56) & (0.95) & (0.41) & (0.16) & (0.29) \\
\quad \textit{Stepdown P-Value} & (0.99) & (0.97) & (0.94) & (0.99) & (0.94) & (0.92) & (0.98) \\
Days of Sport (Weekly) & 0.29 & 0.26 & 0.45 & 0.18 & 0.21 & 0.19 & -1.00 \\
\quad \textit{Unadjusted P-Value} & (0.46) & (0.50) & (0.27) & (0.62) & (0.65) & (0.72) & (0.09)** \\
\quad \textit{Stepdown P-Value} & (0.99) & (0.97) & (0.86) & (0.99) & (0.94) & (0.99) & (0.71) \\
Num. of Friends & 4.70 & 4.77 & 4.43 & 4.57 & 4.37 & 8.07 & 0.82 \\
\quad \textit{Unadjusted P-Value} & (0.00)*** & (0.00)*** & (0.01)*** & (0.00)*** & (0.01)*** & (0.01)*** & (0.84) \\
\quad \textit{Stepdown P-Value} & (0.03)*** & (0.05)*** & (0.14) & (0.04)*** & (0.09)** & (0.09)** & (0.99) \\
Volunteers & 0.09 & 0.07 & 0.06 & 0.10 & 0.11 & 0.10 & -0.09 \\
\quad \textit{Unadjusted P-Value} & (0.35) & (0.39) & (0.53) & (0.26) & (0.26) & (0.47) & (0.51) \\
\quad \textit{Stepdown P-Value} & (0.99) & (0.97) & (0.94) & (0.89) & (0.94) & (0.99) & (0.99) \\
Trust Score & 0.00 & 0.08 & 0.09 & 0.42 & 0.31 & 0.42 & -0.48 \\
\quad \textit{Unadjusted P-Value} & (0.99) & (0.77) & (0.78) & (0.18) & (0.30) & (0.32) & (0.29) \\
\quad \textit{Stepdown P-Value} & (0.99) & (0.97) & (0.94) & (0.89) & (0.94) & (0.98) & (0.97) \\
\bottomrule
\end{tabular}
}
\vspace{1ex} \\
\footnotesize\raggedright{\underline{Note 1:} This table shows the estimates of the coefficient for attending Reggio Approach infant-toddler centers from multiple methods. We compare people who attended both municipal infant-toddler centers and preschools (1,1) with people who only attended municipal preschools and no infant-toddler center (0,1). As in Section Column title indicates the corresponding control set and and model.  \textbf{None} = within-Reggio Emilia OLS estimate with no control variables. \textbf{BIC} = within-Reggio Emilia OLS estimate with controls selected by Bayesian Information Criterion (BIC) and additional controls for male indicator and ITC attendance indicator. \textbf{Full} = within-Reggio Emilia OLS estimate with the full set of controls. \textbf{PSM} =  within-Reggio Emilia propensity score matching estimation. \textbf{KM} =  within-Reggio Emilia Epanechnikov kernel matching estimation. \textbf{DidPm} = difference-in-difference estimates of (Reggio (1,1) - Reggio (0,1)) - (Parma (1,1) - Parma (0,1)). \textbf{PSMPm} =  propensity score matching estimation between Reggio (1,1) and Parma (0,1). \textbf{KMPm} =  Epanechnikov kernel matching estimation between Reggio (1,1) and Parma (0,1). \textbf{DidPv} = difference-in-difference estimates of (Reggio (1,1) - Reggio (0,1)) - (Padova (1,1) - Padova (0,1)). \textbf{PSMPv} =  propensity score matching estimation between Reggio (1,1) and Padova (0,1). \textbf{KMPv} =  Epanechnikov kernel matching estimation between Reggio (1,1) and Padova (0,1).}

\footnotesize\raggedright{\underline{Note 2:} Both unadjusted p-value and stepdown p-value are reported. ***, **, and * indicate significance of the coefficients at the 15\%, 10\%, and 5\% levels respectively. Empty cells show that the estimation cannot be carried out for that outcome.}
\end{table}

\begin{table}[H] \caption{Estimation Results for Main Outcomes, Comparison to No Infant-Toddler Care, Age-30 Cohort} \label{ols-M-adult30-reg-nopres-asilo}
\scalebox{0.60}{\begin{tabular}{l c c c c c c c c c c c}
\toprule
& \multicolumn{5}{c}{Within Reggio} & \multicolumn{3}{c}{With Parma} & \multicolumn{3}{c}{With Padova} \\\cmidrule(lr){2-6} \cmidrule(lr){7-9} \cmidrule(lr){10-12}
 & None & BIC & Full & PSM & KM & DidPm & PSMPm & KMPm & DidPv & PSMPv & KMPv \\
\midrule
IQ Factor & -0.20 & -0.25 & -0.29 & -0.16 & -0.28 & 0.24 & -0.73 & -0.85 & -0.27 & -0.49 & -0.50 \\
\quad \textit{Unadjusted P-Value} & (0.13)* & (0.07)** & (0.04)*** & (0.24) & (0.04)*** & (0.29) & (0.00)*** & (0.00)*** & (0.45) & (0.00)*** & (0.08)** \\
\quad \textit{Stepdown P-Value} & (0.63) & (0.56) & (0.24) & (0.71) & (0.31) & (0.96) & (0.00)*** & (0.00)*** & (0.99) & (0.03)*** & (0.63) \\
Graduate from High School & 0.01 & -0.03 & -0.06 & -0.04 & -0.01 & -0.16 & 0.05 & 0.01 & -0.01 & -0.11 & -0.12 \\
\quad \textit{Unadjusted P-Value} & (0.92) & (0.55) & (0.29) & (0.45) & (0.84) & (0.10)* & (0.37) & (0.95) & (0.93) & (0.02)*** & (0.29) \\
\quad \textit{Stepdown P-Value} & (0.96) & (0.90) & (0.73) & (0.84) & (0.98) & (0.80) & (0.85) & (0.99) & (0.99) & (0.25) & (0.88) \\
High School Grade & -3.06 & -3.48 & -3.18 & -3.03 & -3.94 & 6.34 & 5.34 & 3.08 & -1.64 & 3.21 & 2.40 \\
\quad \textit{Unadjusted P-Value} & (0.05)** & (0.04)*** & (0.07)** & (0.06)** & (0.02)*** & (0.31) & (0.03)*** & (0.43) & (0.77) & (0.29) & (0.57) \\
\quad \textit{Stepdown P-Value} & (0.45) & (0.37) & (0.22) & (0.40) & (0.22) & (0.84) & (0.23) & (0.97) & (0.99) & (0.80) & (0.97) \\
High School Grade (Standardized) & -2.88 & -3.67 & -3.00 & -3.38 & -3.99 & 3.41 & -0.25 & -2.26 & 0.30 & 0.69 & 0.43 \\
\quad \textit{Unadjusted P-Value} & (0.15) & (0.07)** & (0.15)* & (0.09)** & (0.06)** & (0.44) & (0.93) & (0.46) & (0.96) & (0.83) & (0.93) \\
\quad \textit{Stepdown P-Value} & (0.58) & (0.52) & (0.43) & (0.54) & (0.35) & (0.96) & (0.90) & (0.97) & (0.99) & (0.94) & (0.98) \\
Max Edu: University & -0.14 & -0.16 & -0.19 & -0.15 & -0.17 & -0.25 & -0.16 & -0.22 & -0.64 & -0.09 & -0.05 \\
\quad \textit{Unadjusted P-Value} & (0.01)*** & (0.00)*** & (0.00)*** & (0.02)*** & (0.00)*** & (0.05)*** & (0.20) & (0.08)** & (0.00)*** & (0.60) & (0.77) \\
\quad \textit{Stepdown P-Value} & (0.24) & (0.14) & (0.08)** & (0.22) & (0.02)*** & (0.53) & (0.71) & (0.55) & (0.01)*** & (0.92) & (0.98) \\
Employed & 0.06 & 0.06 & 0.07 & 0.06 & 0.06 & -0.04 & 0.14 & 0.09 & -0.04 & 0.37 & 0.40 \\
\quad \textit{Unadjusted P-Value} & (0.01)*** & (0.01)*** & (0.02)*** & (0.01)*** & (0.03)*** & (0.64) & (0.00)*** & (0.35) & (0.61) & (0.02)*** & (0.00)*** \\
\quad \textit{Stepdown P-Value} & (0.50) & (0.56) & (0.27) & (0.10)** & (0.27) & (0.99) & (0.00)*** & (0.96) & (0.99) & (0.24) & (0.10)** \\
Hours Worked Per Week & 4.75 & 5.54 & 5.70 & 5.41 & 5.52 & 1.79 & 10.07 & 9.26 & 2.62 & 19.02 & 20.87 \\
\quad \textit{Unadjusted P-Value} & (0.00)*** & (0.00)*** & (0.00)*** & (0.00)*** & (0.00)*** & (0.67) & (0.00)*** & (0.04)*** & (0.64) & (0.00)*** & (0.00)*** \\
\quad \textit{Stepdown P-Value} & (0.21) & (0.08)** & (0.07)** & (0.02)*** & (0.02)*** & (0.99) & (0.00)*** & (0.38) & (0.99) & (0.07)** & (0.04)*** \\
Married or Cohabitating & 0.14 & 0.13 & 0.14 & 0.11 & 0.13 & 0.05 & 0.13 & 0.20 & 0.12 & 0.24 & 0.31 \\
\quad \textit{Unadjusted P-Value} & (0.10)* & (0.13)* & (0.12)* & (0.22) & (0.13)* & (0.74) & (0.26) & (0.15) & (0.54) & (0.20) & (0.13)* \\
\quad \textit{Stepdown P-Value} & (0.52) & (0.56) & (0.36) & (0.71) & (0.56) & (0.99) & (0.78) & (0.70) & (0.99) & (0.73) & (0.74) \\
Not Obese & 0.15 & 0.12 & 0.09 & 0.14 & 0.09 & 0.19 & 0.06 & 0.01 & -0.10 & 0.26 & 0.26 \\
\quad \textit{Unadjusted P-Value} & (0.01)*** & (0.05)*** & (0.12)* & (0.03)*** & (0.17) & (0.18) & (0.57) & (0.94) & (0.46) & (0.10)** & (0.13)* \\
\quad \textit{Stepdown P-Value} & (0.28) & (0.49) & (0.49) & (0.27) & (0.56) & (0.82) & (0.85) & (0.99) & (0.99) & (0.60) & (0.74) \\
Not Overweight & 0.07 & 0.04 & 0.02 & 0.04 & 0.04 & -0.03 & 0.21 & 0.23 & -0.09 & -0.02 & -0.07 \\
\quad \textit{Unadjusted P-Value} & (0.31) & (0.58) & (0.75) & (0.50) & (0.62) & (0.83) & (0.01)*** & (0.08)** & (0.51) & (0.82) & (0.62) \\
\quad \textit{Stepdown P-Value} & (0.73) & (0.90) & (0.95) & (0.84) & (0.94) & (0.99) & (0.10)** & (0.55) & (0.99) & (0.94) & (0.97) \\
Locus of Control - positive & -0.03 & -0.02 & 0.01 & 0.02 & -0.02 & 0.12 & 0.29 & -0.02 & 0.09 & -0.17 & -0.34 \\
\quad \textit{Unadjusted P-Value} & (0.77) & (0.88) & (0.90) & (0.84) & (0.84) & (0.66) & (0.14)* & (0.93) & (0.81) & (0.35) & (0.23) \\
\quad \textit{Stepdown P-Value} & (0.96) & (0.90) & (0.98) & (0.84) & (0.98) & (0.99) & (0.66) & (0.99) & (0.99) & (0.83) & (0.86) \\
Depression Score - positive & -1.26 & -1.31 & -1.70 & -1.28 & -1.44 & -5.21 & -1.13 & -0.59 & 2.21 & -2.79 & -2.13 \\
\quad \textit{Unadjusted P-Value} & (0.16) & (0.14)* & (0.06)** & (0.16) & (0.13)* & (0.00)*** & (0.40) & (0.73) & (0.39) & (0.18) & (0.45) \\
\quad \textit{Stepdown P-Value} & (0.63) & (0.56) & (0.18) & (0.67) & (0.56) & (0.03)*** & (0.85) & (0.99) & (0.99) & (0.73) & (0.97) \\
Volunteers & -0.09 & -0.08 & -0.06 & -0.10 & -0.08 & 0.12 & -0.37 & -0.42 & 0.12 & -0.28 & -0.31 \\
\quad \textit{Unadjusted P-Value} & (0.02)*** & (0.03)*** & (0.13)* & (0.00)*** & (0.04)*** & (0.29) & (0.00)*** & (0.00)*** & (0.42) & (0.10)* & (0.13)* \\
\quad \textit{Stepdown P-Value} & (0.50) & (0.56) & (0.58) & (0.06)** & (0.31) & (0.94) & (0.00)*** & (0.02)*** & (0.99) & (0.60) & (0.74) \\
Ever Voted for Municipal & 0.23 & 0.18 & 0.11 & 0.22 & 0.16 & 0.13 & 0.43 & 0.48 & -0.16 & 0.63 & 0.69 \\
\quad \textit{Unadjusted P-Value} & (0.01)*** & (0.02)*** & (0.17) & (0.01)*** & (0.06)** & (0.26) & (0.00)*** & (0.00)*** & (0.47) & (0.00)*** & (0.00)*** \\
\quad \textit{Stepdown P-Value} & (0.11) & (0.17) & (0.40) & (0.10)** & (0.35) & (0.94) & (0.00)*** & (0.00)*** & (0.99) & (0.00)*** & (0.03)*** \\
Ever Voted for Regional & 0.23 & 0.20 & 0.15 & 0.23 & 0.19 & 0.17 & 0.46 & 0.55 & -0.29 & 0.70 & 0.74 \\
\quad \textit{Unadjusted P-Value} & (0.01)*** & (0.01)*** & (0.06)** & (0.01)*** & (0.03)*** & (0.12)* & (0.00)*** & (0.00)*** & (0.14)* & (0.00)*** & (0.00)*** \\
\quad \textit{Stepdown P-Value} & (0.11) & (0.14) & (0.16) & (0.09)** & (0.27) & (0.84) & (0.00)*** & (0.00)*** & (0.86) & (0.00)*** & (0.00)*** \\
Num. of Friends & -3.62 & -3.70 & -4.22 & -4.15 & -3.63 & -7.73 & -3.36 & -2.36 & -2.35 & -1.42 & -0.58 \\
\quad \textit{Unadjusted P-Value} & (0.00)*** & (0.00)*** & (0.00)*** & (0.00)*** & (0.00)*** & (0.08)** & (0.00)*** & (0.13)* & (0.43) & (0.17) & (0.87) \\
\quad \textit{Stepdown P-Value} & (0.08)** & (0.07)** & (0.04)*** & (0.00)*** & (0.01)*** & (0.13) & (0.07)** & (0.70) & (0.99) & (0.73) & (0.98) \\
Trust Score & -1.15 & -1.16 & -1.14 & -1.24 & -1.14 & -1.64 & -1.38 & -1.10 & -0.76 & -0.28 & -0.31 \\
\quad \textit{Unadjusted P-Value} & (0.00)*** & (0.00)*** & (0.00)*** & (0.00)*** & (0.00)*** & (0.00)*** & (0.00)*** & (0.03)*** & (0.23) & (0.38) & (0.53) \\
\quad \textit{Stepdown P-Value} & (0.00)*** & (0.00)*** & (0.04)*** & (0.00)*** & (0.00)*** & (0.01)*** & (0.00)*** & (0.29) & (0.94) & (0.83) & (0.97) \\
\bottomrule
\end{tabular}
}
\vspace{1ex} \\
\footnotesize\raggedright{\underline{Note 1:} This table shows the estimates of the coefficient for attending Reggio Approach infant-toddler centers from multiple methods. We compare people who attended both municipal infant-toddler centers and preschools (1,1) with people who only attended municipal preschools and no infant-toddler center (0,1). As in Section Column title indicates the corresponding control set and and model.  \textbf{None} = within-Reggio Emilia OLS estimate with no control variables. \textbf{BIC} = within-Reggio Emilia OLS estimate with controls selected by Bayesian Information Criterion (BIC) and additional controls for male indicator and ITC attendance indicator. \textbf{Full} = within-Reggio Emilia OLS estimate with the full set of controls. \textbf{PSM} =  within-Reggio Emilia propensity score matching estimation. \textbf{KM} =  within-Reggio Emilia Epanechnikov kernel matching estimation. \textbf{DidPm} = difference-in-difference estimates of (Reggio (1,1) - Reggio (0,1)) - (Parma (1,1) - Parma (0,1)). \textbf{PSMPm} =  propensity score matching estimation between Reggio (1,1) and Parma (0,1). \textbf{KMPm} =  Epanechnikov kernel matching estimation between Reggio (1,1) and Parma (0,1). \textbf{DidPv} = difference-in-difference estimates of (Reggio (1,1) - Reggio (0,1)) - (Padova (1,1) - Padova (0,1)). \textbf{PSMPv} =  propensity score matching estimation between Reggio (1,1) and Padova (0,1). \textbf{KMPv} =  Epanechnikov kernel matching estimation between Reggio (1,1) and Padova (0,1).}

\footnotesize\raggedright{\underline{Note 2:} Both unadjusted p-value and stepdown p-value are reported. ***, **, and * indicate significance of the coefficients at the 15\%, 10\%, and 5\% levels respectively. Empty cells show that the estimation cannot be carried out for that outcome.}
\end{table}

\begin{table}[H] \caption{Estimation Results for Main Outcomes, Comparison to No Infant-Toddler Care, Age-40 Cohort} \label{ols-M-adult40-reg-nopres-asilo}
\scalebox{0.63}{\begin{tabular}{l c c}
\toprule
&  \\
  \\
\midrule
IQ Factor \\
\quad \textit{Unadjusted P-Value} \\
\quad \textit{Stepdown P-Value} \\
Graduate from High School \\
\quad \textit{Unadjusted P-Value} \\
\quad \textit{Stepdown P-Value} \\
High School Grade \\
\quad \textit{Unadjusted P-Value} \\
\quad \textit{Stepdown P-Value} \\
High School Grade (Standardized) \\
\quad \textit{Unadjusted P-Value} \\
\quad \textit{Stepdown P-Value} \\
Max Edu: University \\
\quad \textit{Unadjusted P-Value} \\
\quad \textit{Stepdown P-Value} \\
Employed \\
\quad \textit{Unadjusted P-Value} \\
\quad \textit{Stepdown P-Value} \\
Hours Worked Per Week \\
\quad \textit{Unadjusted P-Value} \\
\quad \textit{Stepdown P-Value} \\
Married or Cohabitating \\
\quad \textit{Unadjusted P-Value} \\
\quad \textit{Stepdown P-Value} \\
Not Obese \\
\quad \textit{Unadjusted P-Value} \\
\quad \textit{Stepdown P-Value} \\
Not Overweight \\
\quad \textit{Unadjusted P-Value} \\
\quad \textit{Stepdown P-Value} \\
Locus of Control - positive \\
\quad \textit{Unadjusted P-Value} \\
\quad \textit{Stepdown P-Value} \\
Depression Score - positive \\
\quad \textit{Unadjusted P-Value} \\
\quad \textit{Stepdown P-Value} \\
Volunteers \\
\quad \textit{Unadjusted P-Value} \\
\quad \textit{Stepdown P-Value} \\
Ever Voted for Municipal \\
\quad \textit{Unadjusted P-Value} \\
\quad \textit{Stepdown P-Value} \\
Ever Voted for Regional \\
\quad \textit{Unadjusted P-Value} \\
\quad \textit{Stepdown P-Value} \\
Num. of Friends \\
\quad \textit{Unadjusted P-Value} \\
\quad \textit{Stepdown P-Value} \\
Trust Score \\
\quad \textit{Unadjusted P-Value} \\
\quad \textit{Stepdown P-Value} \\
\bottomrule
\end{tabular}
}
\vspace{1ex} \\
\footnotesize\raggedright{\underline{Note 1:} This table shows the estimates of the coefficient for attending Reggio Approach infant-toddler centers from multiple methods. We compare people who attended both Reggio Approach infant-toddler centers and preschools with people who only attended Reggio Approach preschools. Column title indicates the corresponding control set and and model.  \textbf{None} = OLS estimate with no control variables. \textbf{BIC} = OLS estimate with controls selected by Bayesian Information Criterion (BIC) and additional controls for male indicator and ITC attendance indicator. \textbf{Full} = OLS estimate with the full set of controls. \textbf{PSM} =  propensity score matching estimation.}

\footnotesize\raggedright{\underline{Note 2:} Both unadjusted p-value and stepdown p-value are reported. ***, **, and * indicate significance of the coefficients at the 15\%, 10\%, and 5\% levels respectively. Empty cells show that the estimation cannot be carried out for that outcome.}
\end{table}

\subsection{Preschool}

\subsubsection{Results for the Child Cohort}

We next discuss the results that are robust across methods from the analysis of preschool.\footnote{Appendix~\ref{sec:results} includes more estimates including comparisons to specific school types and additional outcomes.} In the child cohort (Table \ref{ols-M-child-reg-pres}), the Reggio Approach increased the SDQ (Strengths and Difficulties Questionnaire) scores when compared to children who attended other preschools within Reggio Emilia\footnote{The SDQ is a widely-used scale inquiring about emotional symptoms, conduct problems, hyperactivity/inattention, peer relationships problems, and pro-social behavior \citep{Goodman_1997_JCPP}. For ease of interpretation, we have converted the SDQ score such that higher values correspond to more positive outcomes.}. This result becomes more positive after controlling for more background characteristics. However, significantly positive effects for SDQ score are not found when compared with people in Parma and Padova. When we consider the sub-scales of the SDQ as outcomes, the results are positive and significant for the emotional symptoms, positive conduct, and pro-social tests while not significant on the hyperactivity and peer problems tests (see Table~\ref{combined_child_CN_Other}). The Reggio Approach significantly decreased IQ when compared to comparison children groups in all three cities, and significantly increased how child likes school when compared to comparison groups in Reggio Emilia and Padova. The other main outcomes do not show significant effects.

When we compare the Reggio Approach individuals in the child cohort to those who attended religious schools (Table~\ref{ols-M-child-reg-reli}), the Reggio Approach individuals had lower IQ scores and were more obese both within Reggio Emilia and in comparison to the other cities. Compared with the state schools (Table~\ref{ols-M-child-reg-stat}), Reggio Approach children had higher IQ scores except in comparison to Parma. The SDQ score was positive when compared with Padova, but not as positive for within Reggio Emilia as was seen when comparing to all non-Reggio Approach schools. Adjusting for multiple hypothesis testing weakens the inference further.

\begin{table}[H] \caption{Estimation Results for Main Outcomes, Comparison to Non-RA Preschools, Child Cohort} \label{ols-M-child-reg-pres}
\scalebox{0.59}{\begin{tabular}{l c c c c c c c c c c c}
\toprule
 & None & BIC & Full & PSMR & KMR & DidPm & PSMPm & KMPm & DidPv & PSMPv & KMPv \\
\midrule
IQ Factor & -0.13 & -0.20 & -0.19 & -0.21 & -0.15 & -0.03 & -0.34 & -0.39 & -0.14 & -0.21 & -0.25 \\
\quad \textit{Unadjusted P-Value} & (0.22) & (0.06)** & (0.06)** & (0.05)** & (0.20) & (0.83) & (0.00)*** & (0.00)*** & (0.43) & (0.04)*** & (0.03)*** \\
\quad \textit{Stepdown P-Value} & (0.85) & (0.42) & (0.33) & (0.37) & (0.81) & (0.99) & (0.03)*** & (0.00)*** & (0.94) & (0.29) & (0.19) \\
SDQ Composite - Child & 1.59 & 1.47 & 2.14 & 1.39 & 1.13 & 0.62 & 0.30 & 0.24 & 1.91 & 0.75 & 0.71 \\
\quad \textit{Unadjusted P-Value} & (0.00)*** & (0.01)*** & (0.00)*** & (0.01)*** & (0.06)** & (0.43) & (0.52) & (0.60) & (0.03)*** & (0.17) & (0.16) \\
\quad \textit{Stepdown P-Value} & (0.04)*** & (0.07)** & (0.00)*** & (0.15) & (0.45) & (0.99) & (0.93) & (0.96) & (0.22) & (0.67) & (0.58) \\
Not Obese & -0.04 & -0.07 & -0.08 & -0.08 & -0.06 & -0.01 & -0.13 & -0.16 & 0.02 & -0.06 & -0.06 \\
\quad \textit{Unadjusted P-Value} & (0.47) & (0.16) & (0.14)* & (0.16) & (0.28) & (0.84) & (0.01)*** & (0.00)*** & (0.83) & (0.29) & (0.23) \\
\quad \textit{Stepdown P-Value} & (0.98) & (0.63) & (0.59) & (0.71) & (0.84) & (0.99) & (0.08)** & (0.01)*** & (0.96) & (0.69) & (0.65) \\
Not Overweight & -0.02 & -0.01 & -0.02 & 0.00 & -0.01 & -0.02 & 0.05 & 0.02 & -0.04 & -0.04 & -0.04 \\
\quad \textit{Unadjusted P-Value} & (0.54) & (0.87) & (0.64) & (0.99) & (0.79) & (0.76) & (0.18) & (0.53) & (0.44) & (0.26) & (0.24) \\
\quad \textit{Stepdown P-Value} & (0.98) & (0.63) & (0.92) & (0.99) & (0.99) & (0.99) & (0.75) & (0.96) & (0.94) & (0.69) & (0.65) \\
Health is Good & -0.02 & -0.00 & 0.01 & -0.02 & -0.03 & 0.07 & 0.07 & 0.04 & -0.01 & -0.03 & -0.09 \\
\quad \textit{Unadjusted P-Value} & (0.78) & (0.99) & (0.87) & (0.70) & (0.64) & (0.43) & (0.16) & (0.39) & (0.93) & (0.55) & (0.06)** \\
\quad \textit{Stepdown P-Value} & (0.98) & (0.99) & (0.92) & (0.99) & (0.99) & (0.99) & (0.75) & (0.96) & (0.96) & (0.69) & (0.35) \\
Not Excited to Learn & -0.01 & -0.00 & -0.01 & 0.00 & -0.00 & -0.00 & -0.02 & -0.02 & -0.04 & -0.03 & -0.02 \\
\quad \textit{Unadjusted P-Value} & (0.60) & (0.84) & (0.69) & (0.92) & (0.99) & (0.95) & (0.41) & (0.28) & (0.31) & (0.22) & (0.41) \\
\quad \textit{Stepdown P-Value} & (0.98) & (0.63) & (0.92) & (0.99) & (0.99) & (0.99) & (0.93) & (0.92) & (0.83) & (0.69) & (0.65) \\
Problems Sitting Still & -0.00 & 0.01 & -0.03 & 0.02 & 0.02 & -0.08 & -0.01 & -0.01 & -0.08 & 0.05 & -0.00 \\
\quad \textit{Unadjusted P-Value} & (0.90) & (0.78) & (0.51) & (0.71) & (0.63) & (0.16) & (0.71) & (0.85) & (0.20) & (0.24) & (0.90) \\
\quad \textit{Stepdown P-Value} & (0.98) & (0.63) & (0.92) & (0.99) & (0.99) & (0.77) & (0.93) & (0.96) & (0.76) & (0.69) & (0.92) \\
How Much Child Likes School & 0.14 & 0.11 & 0.15 & 0.10 & 0.11 & 0.24 & -0.03 & -0.04 & 0.29 & 0.28 & 0.33 \\
\quad \textit{Unadjusted P-Value} & (0.05)*** & (0.11)* & (0.04)*** & (0.19) & (0.15)* & (0.01)*** & (0.58) & (0.45) & (0.01)*** & (0.00)*** & (0.00)*** \\
\quad \textit{Stepdown P-Value} & (0.29) & (0.59) & (0.25) & (0.71) & (0.76) & (0.11) & (0.93) & (0.96) & (0.08)** & (0.00)*** & (0.00)*** \\
Num. of Friends & -0.30 & -0.42 & -0.35 & -0.36 & -0.38 & -0.19 & -0.36 & -0.34 & -0.22 & -1.54 & -1.57 \\
\quad \textit{Unadjusted P-Value} & (0.23) & (0.09)** & (0.18) & (0.15) & (0.15)* & (0.74) & (0.26) & (0.27) & (0.79) & (0.00)*** & (0.00)*** \\
\quad \textit{Stepdown P-Value} & (0.85) & (0.52) & (0.64) & (0.71) & (0.76) & (0.99) & (0.83) & (0.92) & (0.96) & (0.00)*** & (0.00)*** \\
Candy Game: Willing to Share Candies & 0.01 & 0.00 & 0.03 & -0.03 & 0.01 & 0.01 & -0.03 & -0.01 & 0.02 & -0.06 & -0.04 \\
\quad \textit{Unadjusted P-Value} & (0.70) & (0.90) & (0.39) & (0.44) & (0.89) & (0.77) & (0.41) & (0.63) & (0.65) & (0.11)* & (0.14)* \\
\quad \textit{Stepdown P-Value} & (0.98) & (0.99) & (0.89) & (0.96) & (0.99) & (0.99) & (0.93) & (0.96) & (0.95) & (0.59) & (0.58) \\
\bottomrule
\end{tabular}
}
\vspace{1ex} \\
\footnotesize\raggedright{\underline{Note 1:} This table shows the estimates of the coefficient for attending Reggio Approach preschools from multiple methods. We compare Reggio Approach individuals with those who attended other preschools. Column title indicates the corresponding control set and and model. \textbf{None} = OLS estimate with no control variables. \textbf{BIC} = OLS estimate with controls selected by Bayesian Information Criterion (BIC) and additional controls for male indicator, migrant indicator, and ITC attendance indicator. \textbf{Full} = OLS estimate with the full set of controls. \textbf{PSMR} =  propensity score matching between Reggio Approach people and people in Reggio who attended other types of preschool. \textbf{KMR} = Epanechinikov kernel matching between Reggio Approach people and people in Reggio who attended other types of preschool. \textbf{DidPm} = difference-in-difference estimate of (Reggio Muni - Parma Muni) - (Reggio Other - Parma Other). \textbf{PSMPm} = propensity score matching between Reggio Approach people and people who attended Parma preschools. \textbf{KMPm} = Epanechinikov kernel matching between Reggio Approach people and people who attended Parma preschools. \textbf{DidPv} = difference-in-difference estimate of (Reggio Muni - Padova Muni) - (Reggio Other - Padova Other). \textbf{PSMPv} = propensity score matching between Reggio Approach people and people who attended Padova preschools. \textbf{KMPv} = Epanechinikov kernel matching between Reggio Approach people and people who attended Padova preschools.}

\footnotesize\raggedright{\underline{Note 2:} Both unadjusted p-value and stepdown p-value are reported. ***, **, and * indicate significance of the coefficients at the 15\%, 10\%, and 5\% levels respectively.}

\end{table}


\subsubsection{Results for the Adolescent Cohort}
In the adolescent cohort (Table~\ref{ols-M-adol-reg-pres}), adolescents who attended the Reggio Approach were significantly less likely to be depressed according to analyses done within Reggio Emilia and DiD estimates with Parma and Padova. However, the matching estimations with Parma and Padova adolescents did not show significant effect on depression. The Reggio Approach individuals were more likely to be obese than individuals who attended other types of preschool, and the estimate on obesity is consistent across most of the methods. Some methods show that Reggio Approach individuals were less likely to be involved in sport activities, which is consistent with the increase in obesity. Other outcomes did not have consistently significant results, except for being more bothered by migrants than others in Reggio Emilia (Table~\ref{ols-S-adol-reg-reli}).

In comparison to adolescents who attended religious schools (Table~\ref{ols-M-adol-reg-reli}) the IQ scores are lower for the Reggio Approach adolescents. This is consistent with the results for the child cohort. The SDQ score, capturing social-emotional skills, is higher both when considering the summary score and the individual sub-scales. Similar to the main specification, the adolescents had lower depression scores and higher obesity rates. There are fewer significant outcomes when comparing the Reggio Approach adolescents with those who attended state schools (Table~\ref{ols-M-adol-reg-stat}). Additionally, those that are statistically significant are negative: SDQ scores were lower and adolescents reported less exercise and fewer friends. Adjusting for multiple hypothesis testing weakens the inference further.



\begin{table}[H] \caption{Estimation Results for Main Outcomes, Comparison to Non-RA Preschools, Adolescent Cohort} \label{ols-M-adol-reg-pres}
\scalebox{0.6}{\begin{tabular}{l c c c c c c c}
\toprule
 & None & BIC & Full & PSM & DidPm & DidPv \\
\midrule
IQ Factor & -0.12 & \textbf{ -0.15 } & -0.03 & -0.06 & -0.16 & \textbf{ -0.25 } \\
\quad \textit{Unadjusted P-Value} & (0.22) & (0.15) & (0.78) & (0.53) & (0.21) & (0.15) \\
\quad \textit{Stepdown P-Value} & (0.88) & (0.84) & (0.98) & (0.99) & (0.87) & (0.84) \\
SDQ Composite - Child & 0.01 & 0.18 & 0.37 & -0.56 & -0.41 & -0.52 \\
\quad \textit{Unadjusted P-Value} & (0.98) & (0.80) & (0.55) & (0.49) & (0.64) & (0.51) \\
\quad \textit{Stepdown P-Value} & (0.99) & (0.99) & (0.97) & (0.99) & (0.98) & (0.96) \\
SDQ Composite & 0.90 & \textbf{ 1.03 } & 0.72 & 1.02 & 0.90 & 0.71 \\
\quad \textit{Unadjusted P-Value} & (0.15) & (0.14) & (0.32) & (0.22) & (0.32) & (0.45) \\
\quad \textit{Stepdown P-Value} & (0.82) & (0.82) & (0.90) & (0.94) & (0.88) & (0.96) \\
Depression Score - positive & \textbf{ 1.46 } & \textbf{ 2.39 } & \textbf{ 1.81 } & \textbf{ 2.24 } & \textbf{ 2.21 } & \textbf{ 2.19 } \\
\quad \textit{Unadjusted P-Value} & (0.06) & (0.01) & (0.05) & (0.03) & (0.03) & (0.05) \\
\quad \textit{Stepdown P-Value} & (0.58) & (0.09) & (0.31) & (0.36) & (0.36) & (0.49) \\
Locus of Control - positive & 0.03 & 0.10 & 0.04 & 0.07 & \textbf{ -0.25 } & 0.13 \\
\quad \textit{Unadjusted P-Value} & (0.68) & (0.27) & (0.63) & (0.52) & (0.07) & (0.29) \\
\quad \textit{Stepdown P-Value} & (0.98) & (0.94) & (0.97) & (0.99) & (0.55) & (0.93) \\
Not Obese & \textbf{ -0.08 } & \textbf{ -0.11 } & \textbf{ -0.09 } & \textbf{ -0.07 } & 0.02 & -0.09 \\
\quad \textit{Unadjusted P-Value} & (0.04) & (0.03) & (0.03) & (0.10) & (0.73) & (0.21) \\
\quad \textit{Stepdown P-Value} & (0.53) & (0.26) & (0.33) & (0.73) & (0.98) & (0.89) \\
Not Overweight & 0.01 & -0.02 & -0.00 & -0.03 & \textbf{ 0.08 } & -0.03 \\
\quad \textit{Unadjusted P-Value} & (0.75) & (0.58) & (0.98) & (0.42) & (0.03) & (0.35) \\
\quad \textit{Stepdown P-Value} & (0.99) & (0.99) & (0.98) & (0.98) & (0.21) & (0.94) \\
Health is Good & 0.06 & 0.07 & 0.09 & 0.05 & 0.10 & \textbf{ 0.13 } \\
\quad \textit{Unadjusted P-Value} & (0.32) & (0.28) & (0.15) & (0.50) & (0.23) & (0.13) \\
\quad \textit{Stepdown P-Value} & (0.94) & (0.94) & (0.80) & (0.99) & (0.87) & (0.81) \\
Go To School & 0.03 & 0.01 & 0.03 & -0.01 & 0.03 & 0.04 \\
\quad \textit{Unadjusted P-Value} & (0.22) & (0.78) & (0.22) & (0.76) & (0.33) & (0.22) \\
\quad \textit{Stepdown P-Value} & (0.88) & (0.99) & (0.84) & (0.99) & (0.88) & (0.89) \\
How Much Child Likes School & -0.11 & -0.05 & -0.17 & -0.04 & -0.08 & -0.09 \\
\quad \textit{Unadjusted P-Value} & (0.33) & (0.67) & (0.17) & (0.74) & (0.64) & (0.57) \\
\quad \textit{Stepdown P-Value} & (0.94) & (0.99) & (0.84) & (0.99) & (0.98) & (0.96) \\
Days of Sport (Weekly) & \textbf{ -0.43 } & \textbf{ -0.56 } & -0.33 & -0.32 & \textbf{ -0.67 } & \textbf{ -0.58 } \\
\quad \textit{Unadjusted P-Value} & (0.06) & (0.04) & (0.20) & (0.33) & (0.04) & (0.09) \\
\quad \textit{Stepdown P-Value} & (0.58) & (0.37) & (0.84) & (0.98) & (0.38) & (0.69) \\
Num. of Friends & -0.76 & -0.57 & -0.35 & -0.69 & \textbf{ -2.71 } & -0.69 \\
\quad \textit{Unadjusted P-Value} & (0.54) & (0.59) & (0.76) & (0.56) & (0.15) & (0.74) \\
\quad \textit{Stepdown P-Value} & (0.98) & (0.99) & (0.98) & (0.99) & (0.81) & (0.96) \\
Volunteers & -0.02 & 0.01 & 0.04 & -0.05 & -0.03 & -0.06 \\
\quad \textit{Unadjusted P-Value} & (0.71) & (0.92) & (0.50) & (0.52) & (0.70) & (0.51) \\
\quad \textit{Stepdown P-Value} & (0.99) & (0.99) & (0.97) & (0.99) & (0.98) & (0.96) \\
Trust Score & 0.03 & 0.06 & 0.04 & 0.09 & 0.36 & -0.06 \\
\quad \textit{Unadjusted P-Value} & (0.85) & (0.76) & (0.83) & (0.71) & (0.17) & (0.83) \\
\quad \textit{Stepdown P-Value} & (0.99) & (0.99) & (0.98) & (0.99) & (0.82) & (0.96) \\
\bottomrule
\end{tabular}
}
\vspace{1ex} \\
\footnotesize\raggedright{\underline{Note 1:} This table shows the estimates of the coefficient for attending Reggio Approach preschools from multiple methods. We compare Reggio Approach individuals with those who attended other preschools. Column title indicates the corresponding control set and and model. \textbf{None} = OLS estimate with no control variables. \textbf{BIC} = OLS estimate with controls selected by Bayesian Information Criterion (BIC) and additional controls for male indicator, migrant indicator, and ITC attendance indicator. \textbf{Full} = OLS estimate with the full set of controls. \textbf{PSMR} =  propensity score matching between Reggio Approach people and people in Reggio who attended nother types of preschool. \textbf{KMR} = Epanechinikov kernel matching between Reggio Approach people and people in Reggio who attended nother types of preschool. \textbf{DidPm} = difference-in-difference estimate of (Reggio Muni - Parma Muni) - (Reggio Other - Parma Other). \textbf{PSMPm} = propensity score matching between Reggio Approach people and people who attended Parma preschools. \textbf{KMPm} = Epanechinikov kernel matching between Reggio Approach people and people who attended Parma preschools. \textbf{DidPv} = difference-in-difference estimate of (Reggio Muni - Padova Muni) - (Reggio Other - Padova Other). \textbf{PSMPv} = propensity score matching between Reggio Approach people and people who attended Padova preschools. \textbf{KMPv} = Epanechinikov kernel matching between Reggio Approach people and people who attended Padova preschools.}

\footnotesize\raggedright{\underline{Note 2:} Both unadjusted p-value and stepdown p-value are reported. ***, **, and * indicate significance of the coefficients at the 15\%, 10\%, and 5\% levels respectively.}
\end{table}



\subsubsection{Results for Adult Cohorts}
In the adult cohorts, the results differ depending on the comparison group. The comparison with no preschool, shown in Tables~\ref{ols-M-adult30-reg-nopres} and~\ref{ols-M-adult40-reg-nopres}, shows many more statistically significant estimates within Reggio Emilia. In the comparison with the other preschools, shown in Tables~\ref{ols-M-adult30-reg-pres} and~\ref{ols-M-adult40-reg-pres}, the only outcomes that show any statistical significance within Reggio Emilia across different methods are volunteering behavior in the age-30 cohort, and high school graduation in the age-40 cohort. The OLS estimates show that the Reggio Approach individuals in the age-40 cohort are more likely to graduate from high school than others within Reggio Emilia.

There are more statistically significant outcomes when matching Reggio Approach individuals with people in Parma or Padova who attended preschools. Relative to people who attended preschools in Parma, the Reggio Approach for both adult-30 and adult-40 cohorts show a significantly positive effect on high school grade, locus of control, voting behavior, and a significantly negative effect on IQ, university graduation, obesity, volunteering behavior, and number of friends. Relative to people who attended preschools Padova for the adult-30 cohort, the Reggio Approach shows a significantly positive effect on high school grade and trust score, and a significantly negative effect on IQ and university graduation, depression score, volunteering behavior. Relative to people who attended preschools in Padova for the adult-40 cohort, the Reggio Approach show a significantly positive effect on high school grade, employment, hours worked, marriage, and  and a significantly negative effect on IQ.

In the age-30 cohort, Reggio Approach individuals had worse health along certain outcomes compared with others in Reggio Emilia who did not attend any preschool (Table~\ref{ols-H-adult30-reg-none}). This is seen in reporting more cigarettes per day and more sick days in the past months. Compared with those attended other preschools in Reggio Emilia, Reggio Approach adults were less satisfied with their health and more optimistic (Table~\ref{ols-H-adult30-reg-other}). These two estimates flip directions when comparing against those in Reggio Emilia who did not attend any preschool.

In comparison to those who attended religious schools (Tables~\ref{ols-M-adult30-reg-reli} and~\ref{ols-M-adult40-reg-reli}), age-30 and age-40 adults had lower IQ scores. This is similarly seen in the child and adolescent cohorts when comparing to individuals from religious schools. Individuals in the age-30 cohort also had lower employment levels than those who attended religious schools within Reggio Emilia. Similar to the child and adolescent cohorts, the results flip directions in comparison to state schools (Table~\ref{ols-M-adult30-reg-stat}). More results are positive in the comparison to state schools than the comparison to religious schools. Some examples include lower obesity and more positive locus of control.

In the comparison with no preschool, Reggio Approach individuals were significantly more likely to work more hours than other groups in both the age-30 and age-40 cohorts. For age-30 cohort, the Reggio Approach show a positive effect on high school grade and voting behaviors relative to people in all three cities who did not attend preschool and a positive effect on locus of control relative to Parma no preschool group. Negative effects are found for IQ relative to no preschool group in Parma and Padova, on obesity, volunteering behavior, and number of friends relative to no preschool group in Parma. For age-40 cohort, the Reggio Approach show additional positive effect on voting behavior relative to no preschool groups in all three cities, on obesity and depression score relative to no preschool group in Reggio Emilia, and on high school grade and marriage relative to no preschool group in Parma and Padova (Table~\ref{ols-M-adult40-reg-nopres}).

Moreover, the age-40 cohort was more stressed from work in comparison to both no preschool and other preschools, but also reported being more satisfied with work and their income than those in Parma and Padova (Tables~\ref{ols-W-adult40-reg-other} and~\ref{ols-W-adult40-reg-none}).

Comparisons with the age-50 cohort that preceded the Reggio Approach give additional insight (Table~\ref{ols-M-adult50-reg-nopres}). When simply comparing the age-30 and age-40 Reggio Approach people with age-50 Reggio people without accounting for cohort effect, the significant positive effects are shown on high school grades, voting behavior, and not being overweight. However, without eliminating cohort effect, the significantly positive effects are only shown in high school grades, hours worked per week, and trust score. When cohort effects are eliminated, there is significantly negative effect on voting behaviors.

To summarize, our estimation results on adult cohorts show mixed significance and positive effect depending on the comparison group. However, some of the effects that consistently appear with different comparison group are (i) the negative effect on IQ and (ii) positive effect on voting behavior. \textbf{[JJH: Group -- it also enrolled the disabled. Do we have a measure of disability? Does this include mental disability?][Team: Yes, mental and physical disabled were prioritized. Reggio questionnaires unfortunately do not contain a measure of disability or mental disability. Although SDQ scores show some extent of possible mental problem, it is not a good measure for ``disability".]} A possible explanation for the negative effect on IQ score is that the Reggio Approach does not explicitly teach predetermined skills, which may be important for cognitive assessments\citep{Cagliari-etal-eds_2016_BOOK_Loris-Malaguzzi}. A possible explanation for the positive effect of the Reggio Approach on voting behavior is that the Reggio Approach values children's democratic participation in the lives of their communities \citep{Lazzari_2012_Euro-J-Edu}.

% ========================================================================= %
% CHILD COHORT


\begin{table}[H] \caption{Estimation Results for Main Outcomes, Comparison to Non-RA Preschools, Age-30 Cohort} \label{ols-M-adult30-reg-pres}
\scalebox{0.6}{\begin{tabular}{l c c c c c c c}
\toprule
 & None & BIC & Full & PSM & DidPm & DidPv \\
\midrule
IQ Factor & 0.01 & -0.01 & 0.04 & -0.12 & -0.29 & 0.13 \\
\quad \textit{Unadjusted P-Value} & (0.95) & (0.92) & (0.77) & (0.58) & (0.11) & (0.56) \\
\quad \textit{Stepdown P-Value} & (0.99) & (0.99) & (0.96) & (0.89) & (0.84) & (0.99) \\
Graduate from High School & -0.05 & -0.04 & -0.06 & -0.04 & 0.04 & -0.12 \\
\quad \textit{Unadjusted P-Value} & (0.31) & (0.38) & (0.23) & (0.44) & (0.57) & (0.09) \\
\quad \textit{Stepdown P-Value} & (0.99) & (0.99) & (0.54) & (0.89) & (0.99) & (0.92) \\
High School Grade & 1.05 & 0.56 & 0.66 & 1.40 & 1.70 & 0.94 \\
\quad \textit{Unadjusted P-Value} & (0.49) & (0.71) & (0.67) & (0.40) & (0.63) & (0.80) \\
\quad \textit{Stepdown P-Value} & (0.99) & (0.99) & (0.89) & (0.89) & (0.99) & (0.99) \\
High School Grade (Standardized) & 2.78 & 1.96 & 2.01 & 2.73 & 3.81 & 2.61 \\
\quad \textit{Unadjusted P-Value} & (0.15) & (0.33) & (0.28) & (0.13) & (0.18) & (0.54) \\
\quad \textit{Stepdown P-Value} & (0.87) & (0.99) & (0.62) & (0.89) & (0.92) & (0.99) \\
Max Edu: University & 0.02 & 0.01 & 0.00 & -0.03 & 0.08 & 0.19 \\
\quad \textit{Unadjusted P-Value} & (0.76) & (0.89) & (1.00) & (0.71) & (0.48) & (0.16) \\
\quad \textit{Stepdown P-Value} & (0.99) & (0.99) & (0.99) & (0.89) & (0.99) & (0.89) \\
Employed & -0.03 & -0.03 & -0.02 & -0.02 & 0.12 & -0.05 \\
\quad \textit{Unadjusted P-Value} & (0.39) & (0.43) & (0.64) & (0.56) & (0.11) & (0.60) \\
\quad \textit{Stepdown P-Value} & (0.99) & (0.99) & (0.84) & (0.89) & (0.62) & (0.99) \\
Hours Worked Per Week & -0.02 & 0.19 & 0.63 & 0.64 & 3.80 & 0.29 \\
\quad \textit{Unadjusted P-Value} & (0.99) & (0.93) & (0.77) & (0.85) & (0.29) & (0.94) \\
\quad \textit{Stepdown P-Value} & (0.99) & (0.99) & (0.96) & (0.99) & (0.95) & (0.99) \\
Married or Cohabitating & 0.08 & 0.06 & 0.05 & -0.02 & 0.15 & 0.20 \\
\quad \textit{Unadjusted P-Value} & (0.29) & (0.46) & (0.52) & (0.85) & (0.18) & (0.16) \\
\quad \textit{Stepdown P-Value} & (0.98) & (0.99) & (0.78) & (0.99) & (0.92) & (0.90) \\
Not Obese & 0.01 & 0.00 & 0.03 & -0.03 & 0.01 & -0.03 \\
\quad \textit{Unadjusted P-Value} & (0.87) & (0.95) & (0.61) & (0.76) & (0.94) & (0.81) \\
\quad \textit{Stepdown P-Value} & (0.99) & (0.99) & (0.83) & (0.99) & (0.99) & (0.99) \\
Not Overweight & -0.06 & -0.01 & -0.02 & 0.04 & 0.07 & -0.04 \\
\quad \textit{Unadjusted P-Value} & (0.40) & (0.89) & (0.81) & (0.58) & (0.46) & (0.70) \\
\quad \textit{Stepdown P-Value} & (0.99) & (0.99) & (0.98) & (0.89) & (0.99) & (0.99) \\
Locus of Control - positive & 0.11 & 0.08 & 0.06 & 0.07 & 0.35 & 0.27 \\
\quad \textit{Unadjusted P-Value} & (0.40) & (0.49) & (0.59) & (0.60) & (0.11) & (0.22) \\
\quad \textit{Stepdown P-Value} & (0.99) & (0.99) & (0.83) & (0.89) & (0.70) & (0.97) \\
Depression Score - positive & 0.16 & -0.03 & 0.04 & -0.29 & 1.12 & 0.60 \\
\quad \textit{Unadjusted P-Value} & (0.87) & (0.97) & (0.96) & (0.74) & (0.40) & (0.72) \\
\quad \textit{Stepdown P-Value} & (0.99) & (0.99) & (0.99) & (0.99) & (0.98) & (0.99) \\
Volunteers & 0.11 & 0.10 & 0.11 & 0.10 & -0.05 & -0.03 \\
\quad \textit{Unadjusted P-Value} & (0.00) & (0.00) & (0.00) & (0.00) & (0.63) & (0.82) \\
\quad \textit{Stepdown P-Value} & (0.15) & (0.23) & (0.28) & (0.01) & (0.99) & (0.99) \\
Ever Voted for Municipal & -0.07 & -0.03 & -0.02 & 0.04 & -0.05 & 0.24 \\
\quad \textit{Unadjusted P-Value} & (0.36) & (0.66) & (0.77) & (0.51) & (0.57) & (0.03) \\
\quad \textit{Stepdown P-Value} & (0.99) & (0.99) & (0.95) & (0.89) & (0.99) & (0.68) \\
Ever Voted for Regional & -0.11 & -0.08 & -0.07 & -0.02 & -0.06 & 0.29 \\
\quad \textit{Unadjusted P-Value} & (0.18) & (0.23) & (0.29) & (0.71) & (0.49) & (0.01) \\
\quad \textit{Stepdown P-Value} & (0.90) & (0.99) & (0.62) & (0.89) & (0.99) & (0.39) \\
Num. of Friends & 0.73 & 0.62 & 0.86 & 1.25 & 3.37 & 1.26 \\
\quad \textit{Unadjusted P-Value} & (0.45) & (0.60) & (0.53) & (0.52) & (0.04) & (0.51) \\
\quad \textit{Stepdown P-Value} & (0.99) & (0.99) & (0.75) & (0.89) & (0.66) & (0.99) \\
Trust Score & 0.38 & 0.28 & 0.34 & 0.20 & 0.58 & 0.36 \\
\quad \textit{Unadjusted P-Value} & (0.07) & (0.20) & (0.13) & (0.59) & (0.15) & (0.30) \\
\quad \textit{Stepdown P-Value} & (0.69) & (0.98) & (0.41) & (0.89) & (0.84) & (0.98) \\
\bottomrule
\end{tabular}
}
\vspace{1ex} \\
\footnotesize\raggedright{\underline{Note 1:} This table shows the estimates of the coefficient for attending Reggio Approach preschools from multiple methods. We compare Reggio Approach individuals with those who attended other preschools. Column title indicates the corresponding control set and and model. \textbf{None} = OLS estimate with no control variables. \textbf{BIC} = OLS estimate with controls selected by Bayesian Information Criterion (BIC) and additional controls for male indicator, migrant indicator, and ITC attendance indicator. \textbf{Full} = OLS estimate with the full set of controls. \textbf{PSMR} =  propensity score matching between Reggio Approach people and people in Reggio who attended nother types of preschool. \textbf{KMR} = Epanechinikov kernel matching between Reggio Approach people and people in Reggio who attended nother types of preschool. \textbf{DidPm} = difference-in-difference estimate of (Reggio Muni - Parma Muni) - (Reggio Other - Parma Other). \textbf{PSMPm} = propensity score matching between Reggio Approach people and people who attended Parma preschools. \textbf{KMPm} = Epanechinikov kernel matching between Reggio Approach people and people who attended Parma preschools. \textbf{DidPv} = difference-in-difference estimate of (Reggio Muni - Padova Muni) - (Reggio Other - Padova Other). \textbf{PSMPv} = propensity score matching between Reggio Approach people and people who attended Padova preschools. \textbf{KMPv} = Epanechinikov kernel matching between Reggio Approach people and people who attended Padova preschools.}

\footnotesize\raggedright{\underline{Note 2:} Both unadjusted p-value and stepdown p-value are reported. ***, **, and * indicate significance of the coefficients at the 15\%, 10\%, and 5\% levels respectively.}
\end{table}

\begin{table}[H] \caption{Estimation Results for Main Outcomes, Comparison to No Preschools, Age-30 Cohort} \label{ols-M-adult30-reg-nopres}
\scalebox{0.6}{\begin{tabular}{l c c c c c c c c c c c}
\toprule
& \multicolumn{5}{c}{Within Reggio} & \multicolumn{3}{c}{With Parma} & \multicolumn{3}{c}{With Padova} \\\cmidrule(lr){2-6} \cmidrule(lr){7-9} \cmidrule(lr){10-12}
 & None & BIC & Full & PSMR & KMR & DidPm & KMDidPm & KMPm & DidPv & KMDidPv & KMPv \\
\midrule
IQ Factor & 0.14 & 0.03 & -0.05 & 0.15 & 0.10 & -0.41 & & -0.42 & -0.21 & & -0.25 \\
\quad \textit{Unadjusted P-Value} & (0.39) & (0.82) & (0.74) & (0.43) & (0.58) & (0.10)* & & (0.01)*** & (0.46) & & (0.11)* \\
\quad \textit{Stepdown P-Value} & (0.99) & (0.99) & (0.91) & (0.99) & (0.33) & (0.82) & & (0.16) & (0.99) & & (0.75) \\
Graduate from High School & -0.03 & 0.02 & 0.03 & 0.03 & 0.04 & 0.08 & & -0.01 & -0.05 & & -0.03 \\
\quad \textit{Unadjusted P-Value} & (0.55) & (0.62) & (0.57) & (0.66) & (0.44) & (0.37) & & (0.88) & (0.58) & & (0.61) \\
\quad \textit{Stepdown P-Value} & (0.99) & (0.99) & (0.82) & (0.99) & (0.33) & (0.99) & & (0.98) & (0.99) & & (0.98) \\
High School Grade & 4.54 & 4.98 & 4.62 & 5.57 & 5.60 & 2.20 & & 15.02 & 3.17 & & 6.43 \\
\quad \textit{Unadjusted P-Value} & (0.03)*** & (0.02)*** & (0.04)*** & (0.00)*** & (0.03)*** & (0.64) & & (0.01)*** & (0.45) & & (0.00)*** \\
\quad \textit{Stepdown P-Value} & (0.14) & (0.14) & (0.08)** & (0.05)** & (0.33) & (0.99) & & (0.12) & (0.99) & & (0.01)*** \\
High School Grade (Standardized) & 6.39 & 6.88 & 6.54 & 6.91 & 7.46 & 6.07 & & 7.76 & 6.23 & & 2.85 \\
\quad \textit{Unadjusted P-Value} & (0.01)*** & (0.00)*** & (0.01)*** & (0.00)*** & (0.01)*** & (0.13)* & & (0.07)** & (0.20) & & (0.18) \\
\quad \textit{Stepdown P-Value} & (0.03)*** & (0.04)*** & (0.04)*** & (0.04)*** & (0.16) & (0.84) & & (0.44) & (0.91) & & (0.84) \\
Max Edu: University & -0.07 & -0.03 & -0.04 & -0.02 & -0.02 & -0.02 & & -0.23 & -0.15 & & 0.01 \\
\quad \textit{Unadjusted P-Value} & (0.32) & (0.72) & (0.57) & (0.80) & (0.83) & (0.86) & & (0.03)*** & (0.30) & & (0.89) \\
\quad \textit{Stepdown P-Value} & (0.98) & (0.99) & (0.81) & (0.99) & (0.33) & (0.99) & & (0.26) & (0.99) & & (0.99) \\
Employed & 0.04 & 0.02 & 0.04 & 0.05 & 0.01 & 0.13 & & 0.01 & 0.01 & & 0.05 \\
\quad \textit{Unadjusted P-Value} & (0.38) & (0.69) & (0.35) & (0.39) & (0.90) & (0.13)* & & (0.83) & (0.95) & & (0.31) \\
\quad \textit{Stepdown P-Value} & (0.99) & (0.99) & (0.59) & (0.99) & (0.33) & (0.86) & & (0.98) & (0.99) & & (0.95) \\
Hours Worked Per Week & 6.84 & 4.30 & 5.16 & 2.80 & 3.63 & 8.95 & & 1.65 & 4.79 & & 3.31 \\
\quad \textit{Unadjusted P-Value} & (0.01)*** & (0.12)* & (0.07)** & (0.34) & (0.33) & (0.07)** & & (0.63) & (0.37) & & (0.23) \\
\quad \textit{Stepdown P-Value} & (0.10)** & (0.76) & (0.17) & (0.99) & (0.33) & (0.65) & & (0.96) & (0.99) & & (0.92) \\
Married or Cohabitating & -0.01 & -0.08 & -0.10 & -0.05 & -0.03 & -0.12 & & 0.05 & -0.03 & & -0.01 \\
\quad \textit{Unadjusted P-Value} & (0.85) & (0.33) & (0.25) & (0.56) & (0.73) & (0.37) & & (0.66) & (0.86) & & (0.90) \\
\quad \textit{Stepdown P-Value} & (0.99) & (0.99) & (0.51) & (0.99) & (0.33) & (0.99) & & (0.98) & (0.99) & & (0.99) \\
Not Obese & -0.00 & -0.06 & -0.06 & -0.09 & -0.06 & -0.04 & & -0.23 & -0.28 & & 0.13 \\
\quad \textit{Unadjusted P-Value} & (0.99) & (0.30) & (0.32) & (0.18) & (0.45) & (0.70) & & (0.00)*** & (0.05)** & & (0.13)* \\
\quad \textit{Stepdown P-Value} & (0.99) & (0.99) & (0.58) & (0.95) & (0.33) & (0.99) & & (0.08)** & (0.42) & & (0.78) \\
Not Overweight & -0.07 & 0.01 & -0.02 & 0.03 & 0.02 & 0.00 & & 0.14 & 0.01 & & -0.04 \\
\quad \textit{Unadjusted P-Value} & (0.29) & (0.87) & (0.78) & (0.66) & (0.74) & (0.99) & & (0.18) & (0.93) & & (0.60) \\
\quad \textit{Stepdown P-Value} & (0.98) & (0.99) & (0.94) & (0.99) & (0.33) & (0.99) & & (0.72) & (0.99) & & (0.98) \\
Locus of Control - positive & 0.07 & -0.05 & -0.08 & -0.11 & -0.01 & -0.08 & & 0.69 & 0.02 & & -0.04 \\
\quad \textit{Unadjusted P-Value} & (0.59) & (0.71) & (0.56) & (0.34) & (0.96) & (0.76) & & (0.00)*** & (0.94) & & (0.81) \\
\quad \textit{Stepdown P-Value} & (0.99) & (0.99) & (0.77) & (0.99) & (0.98) & (0.99) & & (0.04)*** & (0.99) & & (0.98) \\
Depression Score - positive & 1.26 & -0.04 & -0.20 & 0.37 & 0.29 & -0.14 & & -0.42 & -1.10 & & 0.28 \\
\quad \textit{Unadjusted P-Value} & (0.20) & (0.97) & (0.83) & (0.70) & (0.79) & (0.93) & & (0.74) & (0.58) & & (0.79) \\
\quad \textit{Stepdown P-Value} & (0.95) & (0.99) & (0.97) & (0.99) & (0.33) & (0.99) & & (0.98) & (0.99) & & (0.98) \\
Volunteers & -0.08 & -0.08 & -0.08 & -0.03 & -0.03 & -0.12 & & -0.18 & -0.32 & & 0.04 \\
\quad \textit{Unadjusted P-Value} & (0.17) & (0.12)* & (0.13)* & (0.61) & (0.63) & (0.34) & & (0.06)** & (0.01)*** & & (0.45) \\
\quad \textit{Stepdown P-Value} & (0.88) & (0.85) & (0.33) & (0.99) & (0.33) & (0.98) & & (0.40) & (0.06)** & & (0.98) \\
Ever Voted for Municipal & 0.10 & 0.03 & 0.04 & -0.07 & 0.02 & -0.08 & & 0.31 & -0.07 & & 0.34 \\
\quad \textit{Unadjusted P-Value} & (0.20) & (0.61) & (0.53) & (0.43) & (0.83) & (0.42) & & (0.00)*** & (0.59) & & (0.00)*** \\
\quad \textit{Stepdown P-Value} & (0.96) & (0.99) & (0.78) & (0.99) & (0.33) & (0.99) & & (0.02)*** & (0.99) & & (0.00)*** \\
Ever Voted for Regional & 0.05 & -0.02 & -0.01 & -0.09 & -0.01 & -0.06 & & 0.31 & 0.03 & & 0.27 \\
\quad \textit{Unadjusted P-Value} & (0.55) & (0.75) & (0.92) & (0.33) & (0.92) & (0.54) & & (0.00)*** & (0.84) & & (0.00)*** \\
\quad \textit{Stepdown P-Value} & (0.99) & (0.99) & (0.97) & (0.95) & (0.98) & (0.99) & & (0.02)*** & (0.99) & & (0.01)*** \\
Num. of Friends & 0.02 & 0.24 & 0.20 & 0.02 & -0.41 & 2.16 & & -2.69 & 4.48 & & -1.20 \\
\quad \textit{Unadjusted P-Value} & (0.99) & (0.88) & (0.91) & (0.99) & (0.81) & (0.31) & & (0.14)* & (0.08)** & & (0.50) \\
\quad \textit{Stepdown P-Value} & (0.99) & (0.99) & (0.97) & (0.99) & (0.33) & (0.98) & & (0.66) & (0.80) & & (0.98) \\
Trust Score & -0.06 & 0.01 & 0.21 & -0.00 & 0.07 & 0.26 & & 0.26 & 0.15 & & 0.79 \\
\quad \textit{Unadjusted P-Value} & (0.79) & (0.96) & (0.36) & (0.99) & (0.76) & (0.56) & & (0.46) & (0.69) & & (0.00)*** \\
\quad \textit{Stepdown P-Value} & (0.99) & (0.99) & (0.58) & (0.99) & (0.33) & (0.99) & & (0.96) & (0.99) & & (0.00)*** \\
\bottomrule
\end{tabular}
}
\vspace{1ex} \\
\footnotesize\raggedright{\underline{Note 1:} This table shows the estimates of the coefficient for attending Reggio Approach preschools from multiple methods. We compare Reggio Approach individuals with those who did not attend any preschools. Column title indicates the corresponding control set and and model. \textbf{None} = OLS estimate with no control variables. \textbf{BIC} = OLS estimate with controls selected by Bayesian Information Criterion (BIC) and additional controls for male indicator, migrant indicator, and ITC attendance indicator. \textbf{Full} = OLS estimate with the full set of controls. \textbf{PSMR} =  propensity score matching between Reggio Approach people and people in Reggio who did not attend any preschool. \textbf{KMR} = Epanechinikov kernel matching between Reggio Approach people and people in Reggio who attended other types of preschool. \textbf{DidPm} = difference-in-difference estimate of (Reggio Muni - Parma Muni) - (Reggio None - Parma None). \textbf{PSMPm} = propensity score matching between Reggio Approach people and people in Parma who did not attend any preschool. \textbf{KMPm} = Epanechinikov kernel matching between Reggio Approach people and people in Parma who did not attend any preschool. \textbf{DidPv} = difference-in-difference estimate of (Reggio Muni - Padova Muni) - (Reggio None - Padova None). \textbf{PSMPv} = propensity score matching between Reggio Approach people and people in Padova who did not attend any preschool. \textbf{KMPv} = Epanechinikov kernel matching between Reggio Approach people and people in Padova who did not attend any preschool.}

\footnotesize\raggedright{\underline{Note 2:} Both unadjusted p-value and stepdown p-value are reported. ***, **, and * indicate significance of the coefficients at the 15\%, 10\%, and 5\% levels respectively.}
\end{table}


\begin{table}[H] \caption{Estimation Results for Main Outcomes, Comparison to Non-RA Preschools, Age-40 Cohort} \label{ols-M-adult40-reg-pres}
\scalebox{0.6}{\begin{tabular}{l c c c c c}
\toprule
 & None & BIC & Full & PSM \\
\midrule
IQ Factor & -0.15 & -0.12 & -0.14 & -0.11 \\
\quad \textit{Unadjusted P-Value} & (0.22) & (0.29) & (0.22) & (0.34) \\
\quad \textit{Stepdown P-Value} & (0.97) & (0.99) & (0.64) & (0.99) \\
Graduate from High School & \textbf{ 0.13 } & \textbf{ 0.10 } & \textbf{ 0.12 } & 0.09 \\
\quad \textit{Unadjusted P-Value} & (0.05) & (0.14) & (0.09) & (0.20) \\
\quad \textit{Stepdown P-Value} & (0.46) & (0.89) & (0.45) & (0.96) \\
High School Grade & -0.66 & -0.09 & 0.36 & -0.84 \\
\quad \textit{Unadjusted P-Value} & (0.67) & (0.96) & (0.83) & (0.61) \\
\quad \textit{Stepdown P-Value} & (0.99) & (0.99) & (0.99) & (0.99) \\
High School Grade (Standardized) & -1.13 & -0.17 & 0.36 & 0.74 \\
\quad \textit{Unadjusted P-Value} & (0.58) & (0.94) & (0.87) & (0.76) \\
\quad \textit{Stepdown P-Value} & (0.99) & (0.99) & (0.99) & (0.99) \\
Max Edu: University & 0.07 & 0.05 & 0.03 & 0.01 \\
\quad \textit{Unadjusted P-Value} & (0.20) & (0.34) & (0.62) & (0.92) \\
\quad \textit{Stepdown P-Value} & (0.97) & (0.99) & (0.94) & (0.99) \\
Employed & 0.01 & 0.01 & 0.01 & 0.03 \\
\quad \textit{Unadjusted P-Value} & (0.75) & (0.79) & (0.73) & (0.46) \\
\quad \textit{Stepdown P-Value} & (0.99) & (0.99) & (0.97) & (0.99) \\
Hours Worked Per Week & -0.90 & -1.17 & -1.28 & -1.71 \\
\quad \textit{Unadjusted P-Value} & (0.64) & (0.58) & (0.56) & (0.38) \\
\quad \textit{Stepdown P-Value} & (0.99) & (0.99) & (0.85) & (0.99) \\
Married or Cohabitating & 0.03 & 0.02 & 0.02 & 0.01 \\
\quad \textit{Unadjusted P-Value} & (0.69) & (0.81) & (0.80) & (0.84) \\
\quad \textit{Stepdown P-Value} & (0.99) & (0.99) & (0.98) & (0.99) \\
Not Obese & -0.04 & 0.02 & 0.04 & 0.03 \\
\quad \textit{Unadjusted P-Value} & (0.59) & (0.80) & (0.56) & (0.76) \\
\quad \textit{Stepdown P-Value} & (0.99) & (0.99) & (0.91) & (0.99) \\
Not Overweight & 0.05 & 0.03 & 0.03 & -0.01 \\
\quad \textit{Unadjusted P-Value} & (0.48) & (0.63) & (0.67) & (0.92) \\
\quad \textit{Stepdown P-Value} & (0.99) & (0.99) & (0.96) & (0.99) \\
Locus of Control - positive & 0.13 & 0.14 & 0.11 & 0.19 \\
\quad \textit{Unadjusted P-Value} & (0.36) & (0.31) & (0.44) & (0.27) \\
\quad \textit{Stepdown P-Value} & (0.99) & (0.98) & (0.80) & (0.98) \\
Depression Score - positive & 0.56 & \textbf{ 1.37 } & 1.09 & 1.28 \\
\quad \textit{Unadjusted P-Value} & (0.55) & (0.11) & (0.22) & (0.16) \\
\quad \textit{Stepdown P-Value} & (0.99) & (0.89) & (0.60) & (0.92) \\
Volunteers & 0.05 & 0.01 & 0.03 & -0.01 \\
\quad \textit{Unadjusted P-Value} & (0.23) & (0.75) & (0.54) & (0.93) \\
\quad \textit{Stepdown P-Value} & (0.97) & (0.99) & (0.92) & (0.99) \\
Ever Voted for Municipal & -0.07 & 0.07 & 0.06 & \textbf{ 0.09 } \\
\quad \textit{Unadjusted P-Value} & (0.38) & (0.28) & (0.38) & (0.15) \\
\quad \textit{Stepdown P-Value} & (0.99) & (0.98) & (0.75) & (0.92) \\
Ever Voted for Regional & -0.05 & 0.08 & 0.07 & 0.10 \\
\quad \textit{Unadjusted P-Value} & (0.53) & (0.23) & (0.34) & (0.17) \\
\quad \textit{Stepdown P-Value} & (0.99) & (0.98) & (0.69) & (0.95) \\
Num. of Friends & \textbf{ 1.39 } & 0.95 & 1.09 & 0.88 \\
\quad \textit{Unadjusted P-Value} & (0.15) & (0.34) & (0.29) & (0.44) \\
\quad \textit{Stepdown P-Value} & (0.94) & (0.99) & (0.61) & (0.99) \\
Trust Score & 0.02 & -0.11 & 0.05 & -0.27 \\
\quad \textit{Unadjusted P-Value} & (0.95) & (0.64) & (0.84) & (0.33) \\
\quad \textit{Stepdown P-Value} & (0.99) & (0.99) & (0.99) & (0.99) \\
\bottomrule
\end{tabular}
}
\vspace{1ex} \\
\footnotesize\raggedright{\underline{Note 1:} This table shows the estimates of the coefficient for attending Reggio Approach preschools from multiple methods. We compare Reggio Approach individuals with those who attended other preschools. Column title indicates the corresponding control set and and model. \textbf{None} = OLS estimate with no control variables. \textbf{BIC} = OLS estimate with controls selected by Bayesian Information Criterion (BIC) and additional controls for male indicator, migrant indicator, and ITC attendance indicator. \textbf{Full} = OLS estimate with the full set of controls. \textbf{PSMR} =  propensity score matching between Reggio Approach people and people in Reggio who attended nother types of preschool. \textbf{KMR} = Epanechinikov kernel matching between Reggio Approach people and people in Reggio who attended nother types of preschool. \textbf{PSMPm} = propensity score matching between Reggio Approach people and people who attended Parma preschools. \textbf{KMPm} = Epanechinikov kernel matching between Reggio Approach people and people who attended Parma preschools. \textbf{PSMPv} = propensity score matching between Reggio Approach people and people who attended Padova preschools. \textbf{KMPv} = Epanechinikov kernel matching between Reggio Approach people and people who attended Padova preschools. \textit{Difference-indifference is not available for this cohort due to non-existence of municipal preschools in Parma and Padova.}}

\footnotesize\raggedright{\underline{Note 2:} Both unadjusted p-value and stepdown p-value are reported. ***, **, and * indicate significance of the coefficients at the 15\%, 10\%, and 5\% levels respectively.}
\end{table}

\begin{table}[H] \caption{Estimation Results for Main Outcomes, Comparison to No Preschools, Age-40 Cohort} \label{ols-M-adult40-reg-nopres}
\scalebox{0.6}{\begin{tabular}{l c c c c c c c c c c c}
\toprule
& \multicolumn{5}{c}{Within Reggio} & \multicolumn{3}{c}{With Parma} & \multicolumn{3}{c}{With Padova} \\\cmidrule(lr){2-6} \cmidrule(lr){7-9} \cmidrule(lr){10-12}
 & None & BIC & Full & PSMR & KMR & DidPm & KMDidPm & KMPm & DidPv & KMDidPv & KMPv \\
\midrule
IQ Factor & 0.01 & 0.02 & 0.04 & 0.13 & 0.04 & 0.51 & 0.15 & -0.40 & 0.19 & 0.09 & -0.34 \\
\quad \textit{Unadjusted P-Value} & (0.97) & (0.86) & (0.80) & (0.36) & (0.80) & (0.06)** & (0.42) & (0.00)*** & (0.45) & (0.67) & (0.00)*** \\
\quad \textit{Stepdown P-Value} & (0.99) & (0.99) & (0.88) & (0.95) & (0.99) & (0.51) & (0.99) & (0.02)*** & (0.99) & (0.99) & (0.06)** \\
Graduate from High School & -0.07 & -0.04 & -0.06 & -0.07 & -0.04 & 0.04 & 0.04 & -0.03 & -0.14 & -0.07 & 0.09 \\
\quad \textit{Unadjusted P-Value} & (0.17) & (0.47) & (0.33) & (0.25) & (0.45) & (0.74) & (0.61) & (0.64) & (0.21) & (0.43) & (0.28) \\
\quad \textit{Stepdown P-Value} & (0.86) & (0.99) & (0.59) & (0.92) & (0.99) & (0.99) & (0.99) & (0.97) & (0.98) & (0.99) & (0.95) \\
High School Grade & 0.59 & 1.13 & 1.77 & 1.53 & 1.28 & -3.50 & -4.57 & 8.62 & -1.17 & 3.12 & 4.49 \\
\quad \textit{Unadjusted P-Value} & (0.70) & (0.47) & (0.35) & (0.34) & (0.45) & (0.40) & (0.15) & (0.00)*** & (0.75) & (0.20) & (0.06)** \\
\quad \textit{Stepdown P-Value} & (0.99) & (0.99) & (0.59) & (0.95) & (0.99) & (0.99) & (0.74) & (0.01)*** & (0.99) & (0.94) & (0.52) \\
High School Grade (Standardized) & 0.43 & 0.81 & 0.94 & 1.54 & 0.82 & -4.24 & -2.88 & 0.26 & -2.19 & 3.23 & 0.03 \\
\quad \textit{Unadjusted P-Value} & (0.82) & (0.68) & (0.69) & (0.43) & (0.70) & (0.26) & (0.38) & (0.88) & (0.57) & (0.33) & (0.99) \\
\quad \textit{Stepdown P-Value} & (0.99) & (0.99) & (0.82) & (0.95) & (0.99) & (0.99) & (0.99) & (0.97) & (0.99) & (0.98) & (0.99) \\
Max Edu: University & 0.01 & 0.05 & 0.11 & 0.03 & 0.04 & -0.08 & -0.13 & 0.03 & -0.13 & -0.06 & 0.03 \\
\quad \textit{Unadjusted P-Value} & (0.82) & (0.39) & (0.07)** & (0.64) & (0.48) & (0.53) & (0.13)* & (0.62) & (0.34) & (0.48) & (0.75) \\
\quad \textit{Stepdown P-Value} & (0.99) & (0.97) & (0.35) & (0.98) & (0.99) & (0.99) & (0.86) & (0.97) & (0.98) & (0.99) & (0.99) \\
Employed & 0.06 & 0.05 & 0.05 & 0.06 & 0.05 & -0.02 & 0.03 & 0.00 & 0.04 & 0.11 & 0.02 \\
\quad \textit{Unadjusted P-Value} & (0.14)* & (0.14)* & (0.16) & (0.18) & (0.18) & (0.81) & (0.56) & (0.98) & (0.67) & (0.05)** & (0.66) \\
\quad \textit{Stepdown P-Value} & (0.78) & (0.79) & (0.58) & (0.84) & (0.88) & (0.99) & (0.99) & (0.99) & (0.99) & (0.79) & (0.95) \\
Hours Worked Per Week & 5.71 & 6.51 & 7.39 & 7.43 & 7.20 & 1.43 & 6.44 & -0.11 & 4.09 & 8.95 & 5.02 \\
\quad \textit{Unadjusted P-Value} & (0.02)*** & (0.01)*** & (0.01)*** & (0.00)*** & (0.01)*** & (0.75) & (0.03)*** & (0.96) & (0.41) & (0.01)*** & (0.07)** \\
\quad \textit{Stepdown P-Value} & (0.21) & (0.12) & (0.04)*** & (0.08)** & (0.12) & (0.99) & (0.60) & (0.99) & (0.99) & (0.23) & (0.55) \\
Married or Cohabitating & 0.02 & -0.01 & 0.05 & -0.00 & -0.02 & -0.07 & -0.07 & 0.18 & -0.15 & -0.16 & 0.22 \\
\quad \textit{Unadjusted P-Value} & (0.80) & (0.88) & (0.57) & (0.96) & (0.75) & (0.66) & (0.52) & (0.02)*** & (0.34) & (0.17) & (0.03)*** \\
\quad \textit{Stepdown P-Value} & (0.99) & (0.99) & (0.78) & (0.99) & (0.99) & (0.99) & (0.99) & (0.20) & (0.98) & (0.83) & (0.33) \\
Not Obese & 0.14 & 0.11 & 0.01 & 0.12 & 0.10 & 0.33 & 0.21 & -0.19 & 0.16 & 0.01 & 0.01 \\
\quad \textit{Unadjusted P-Value} & (0.06)** & (0.14)* & (0.91) & (0.11)* & (0.21) & (0.03)*** & (0.04)*** & (0.00)*** & (0.33) & (0.96) & (0.91) \\
\quad \textit{Stepdown P-Value} & (0.50) & (0.79) & (0.96) & (0.74) & (0.91) & (0.29) & (0.53) & (0.04)*** & (0.98) & (0.99) & (0.99) \\
Not Overweight & -0.03 & 0.03 & 0.07 & 0.05 & 0.06 & 0.07 & 0.10 & 0.02 & -0.08 & 0.01 & 0.05 \\
\quad \textit{Unadjusted P-Value} & (0.66) & (0.68) & (0.33) & (0.43) & (0.44) & (0.65) & (0.38) & (0.76) & (0.56) & (0.90) & (0.61) \\
\quad \textit{Stepdown P-Value} & (0.99) & (0.99) & (0.59) & (0.95) & (0.99) & (0.99) & (0.99) & (0.97) & (0.99) & (0.99) & (0.95) \\
Locus of Control - positive & 0.14 & 0.23 & 0.28 & 0.26 & 0.21 & 0.31 & 0.19 & 0.12 & 0.20 & 0.31 & 0.04 \\
\quad \textit{Unadjusted P-Value} & (0.29) & (0.07)** & (0.05)** & (0.05)** & (0.13)* & (0.28) & (0.34) & (0.39) & (0.47) & (0.12)* & (0.81) \\
\quad \textit{Stepdown P-Value} & (0.97) & (0.63) & (0.30) & (0.57) & (0.79) & (0.99) & (0.99) & (0.97) & (0.99) & (0.79) & (0.99) \\
Depression Score - positive & 2.25 & 2.24 & 2.10 & 2.90 & 2.16 & -1.72 & 0.12 & 0.93 & 2.20 & 2.03 & 0.35 \\
\quad \textit{Unadjusted P-Value} & (0.02)*** & (0.02)*** & (0.05)** & (0.00)*** & (0.03)*** & (0.37) & (0.92) & (0.26) & (0.25) & (0.14)* & (0.73) \\
\quad \textit{Stepdown P-Value} & (0.20) & (0.19) & (0.21) & (0.08)** & (0.33) & (0.99) & (0.99) & (0.90) & (0.98) & (0.80) & (0.99) \\
Volunteers & -0.11 & -0.08 & -0.11 & -0.07 & -0.07 & 0.11 & -0.01 & -0.14 & -0.06 & -0.13 & 0.02 \\
\quad \textit{Unadjusted P-Value} & (0.05)*** & (0.16) & (0.13)* & (0.30) & (0.29) & (0.37) & (0.90) & (0.03)*** & (0.56) & (0.12)* & (0.71) \\
\quad \textit{Stepdown P-Value} & (0.43) & (0.79) & (0.35) & (0.95) & (0.95) & (0.99) & (0.99) & (0.27) & (0.99) & (0.81) & (0.95) \\
Ever Voted for Municipal & 0.19 & 0.15 & 0.11 & 0.17 & 0.19 & 0.08 & -0.02 & 0.32 & -0.06 & -0.05 & 0.41 \\
\quad \textit{Unadjusted P-Value} & (0.02)*** & (0.05)** & (0.18) & (0.08)** & (0.03)*** & (0.61) & (0.97) & (0.00)*** & (0.68) & (0.65) & (0.00)*** \\
\quad \textit{Stepdown P-Value} & (0.21) & (0.40) & (0.53) & (0.66) & (0.33) & (0.99) & (0.99) & (0.01)*** & (0.99) & (0.99) & (0.00)*** \\
Ever Voted for Regional & 0.20 & 0.16 & 0.13 & 0.18 & 0.20 & 0.15 & 0.03 & 0.41 & -0.09 & -0.06 & 0.41 \\
\quad \textit{Unadjusted P-Value} & (0.01)*** & (0.04)*** & (0.14)* & (0.07)** & (0.02)*** & (0.32) & (0.84) & (0.00)*** & (0.54) & (0.52) & (0.00)*** \\
\quad \textit{Stepdown P-Value} & (0.20) & (0.37) & (0.40) & (0.64) & (0.32) & (0.99) & (0.99) & (0.00)*** & (0.99) & (0.99) & (0.00)*** \\
Num. of Friends & -0.68 & -0.07 & 0.75 & -0.13 & -0.48 & 2.17 & 2.68 & -4.77 & 0.35 & 1.44 & -0.84 \\
\quad \textit{Unadjusted P-Value} & (0.52) & (0.95) & (0.61) & (0.92) & (0.67) & (0.42) & (0.16) & (0.00)*** & (0.90) & (0.33) & (0.61) \\
\quad \textit{Stepdown P-Value} & (0.99) & (0.99) & (0.78) & (0.99) & (0.99) & (0.99) & (0.91) & (0.02)*** & (0.99) & (0.98) & (0.95) \\
Trust Score & -0.03 & 0.01 & -0.29 & -0.03 & 0.00 & -0.00 & 0.23 & -0.46 & 0.52 & 0.10 & 0.15 \\
\quad \textit{Unadjusted P-Value} & (0.90) & (0.97) & (0.24) & (0.91) & (1.00) & (1.00) & (0.59) & (0.06)** & (0.31) & (0.79) & (0.59) \\
\quad \textit{Stepdown P-Value} & (0.99) & (0.99) & (0.59) & (0.99) & (0.99) & (0.99) & (0.99) & (0.37) & (0.98) & (0.99) & (0.95) \\
\bottomrule
\end{tabular}
}
\vspace{1ex} \\
\footnotesize\raggedright{\underline{Note 1:} This table shows the estimates of the coefficient for attending Reggio Approach preschools from multiple methods. We compare Reggio Approach individuals with those who did not attend any preschools. Column title indicates the corresponding control set and and model. \textbf{None} = OLS estimate with no control variables. \textbf{BIC} = OLS estimate with controls selected by Bayesian Information Criterion (BIC) and additional controls for male indicator, migrant indicator, and ITC attendance indicator. \textbf{Full} = OLS estimate with the full set of controls. \textbf{PSMR} =  propensity score matching between Reggio Approach people and people in Reggio who did not attend any preschool. \textbf{KMR} = Epanechinikov kernel matching between Reggio Approach people and people in Reggio who attended other types of preschool. \textbf{DidPm} = difference-in-difference estimate of (Reggio Muni - Parma Other) - (Reggio None - Parma None). \textbf{PSMPm} = propensity score matching between Reggio Approach people and people in Parma who did not attend any preschool. \textbf{KMPm} = Epanechinikov kernel matching between Reggio Approach people and people in Parma who did not attend any preschool. \textbf{DidPv} = difference-in-difference estimate of (Reggio Muni - Padova Other) - (Reggio None - Padova None). \textbf{PSMPv} = propensity score matching between Reggio Approach people and people in Padova who did not attend any preschool. \textbf{KMPv} = Epanechinikov kernel matching between Reggio Approach people and people in Padova who did not attend any preschool.}

\footnotesize\raggedright{\underline{Note 2:} Both unadjusted p-value and stepdown p-value are reported. ***, **, and * indicate significance of the coefficients at the 15\%, 10\%, and 5\% levels respectively.}
\end{table}


\begin{table}[H] \caption{Estimation Results for Main Outcomes, Comparison to No Preschools, Comparison to Age-50 Cohort} \label{ols-M-adult50-reg-nopres}
\scalebox{0.56}{\begin{tabular}{l c c c c c c}
\toprule
& \multicolumn{3}{c}{Within Age-30} & \multicolumn{3}{c}{With Age-40} \\\cmidrule(lr){2-4} \cmidrule(lr){5-7}
 & OLS30 & DiD30 & KMDiD30 & OLS40 & DiD40 & KMDiD40 \\
\midrule
IQ Factor & -0.85 & -0.67 & 0.14 & -0.61 & -0.49 & 0.00 \\
\quad \textit{Unadjusted P-Value} & (0.00)*** & (0.03)*** & (0.45) & (0.00)*** & (0.01)*** & (0.99) \\
\quad \textit{Stepdown P-Value} & (0.00)*** & (0.87) & (0.99) & (0.00)*** & (0.95) & (0.99) \\
Graduate from High School & 0.03 & -0.39 & -0.11 & 0.06 & -0.56 & -0.25 \\
\quad \textit{Unadjusted P-Value} & (0.55) & (0.02)*** & (0.82) & (0.20) & (0.00)*** & (0.01)*** \\
\quad \textit{Stepdown P-Value} & (0.86) & (0.87) & (0.99) & (0.56) & (0.80) & (0.25) \\
High School Grade & 2.91 & 9.29 & 1.86 & 2.80 & 6.26 & 0.05 \\
\quad \textit{Unadjusted P-Value} & (0.02)*** & (0.04)*** & (0.56) & (0.03)*** & (0.01)*** & (1.00) \\
\quad \textit{Stepdown P-Value} & (0.18) & (0.87) & (0.99) & (0.19) & (0.95) & (0.99) \\
High School Grade (Standardized) & 3.94 & 20.60 & 3.68 & 2.37 & 12.27 & -0.56 \\
\quad \textit{Unadjusted P-Value} & (0.02)*** & (0.01)*** & (0.42) & (0.16) & (0.00)*** & (0.89) \\
\quad \textit{Stepdown P-Value} & (0.18) & (0.65) & (0.99) & (0.56) & (0.95) & (0.99) \\
Max Edu: University & 0.03 & -0.09 & -0.06 & 0.07 & -0.07 & -0.00 \\
\quad \textit{Unadjusted P-Value} & (0.50) & (0.50) & (0.54) & (0.13)* & (0.39) & (0.94) \\
\quad \textit{Stepdown P-Value} & (0.86) & (0.91) & (0.99) & (0.50) & (0.95) & (0.99) \\
Employed & 0.05 & -0.01 & 0.09 & 0.06 & -0.05 & 0.12 \\
\quad \textit{Unadjusted P-Value} & (0.17) & (0.91) & (0.30) & (0.04)*** & (0.29) & (0.86) \\
\quad \textit{Stepdown P-Value} & (0.72) & (0.98) & (0.99) & (0.35) & (0.95) & (0.75) \\
Hours Worked Per Week & 1.83 & 15.08 & 8.30 & 3.80 & 16.06 & 11.31 \\
\quad \textit{Unadjusted P-Value} & (0.30) & (0.00)*** & (0.04)*** & (0.01)*** & (0.00)*** & (0.00)*** \\
\quad \textit{Stepdown P-Value} & (0.83) & (0.84) & (0.24) & (0.18) & (0.90) & (0.05)** \\
Married or Cohabitating & -0.27 & -0.49 & -0.11 & 0.09 & -0.47 & -0.21 \\
\quad \textit{Unadjusted P-Value} & (0.00)*** & (0.01)*** & (0.33) & (0.15)* & (0.00)*** & (0.05)** \\
\quad \textit{Stepdown P-Value} & (0.00)*** & (0.87) & (0.99) & (0.56) & (0.95) & (0.61) \\
Not Obese & -0.09 & -0.34 & -0.08 & -0.09 & 0.17 & 0.05 \\
\quad \textit{Unadjusted P-Value} & (0.10)* & (0.00)*** & (0.30) & (0.10)** & (0.06)** & (0.56) \\
\quad \textit{Stepdown P-Value} & (0.50) & (0.87) & (0.99) & (0.47) & (0.95) & (0.99) \\
Not Overweight & 0.15 & -0.23 & -0.02 & 0.17 & -0.20 & 0.15 \\
\quad \textit{Unadjusted P-Value} & (0.01)*** & (0.22) & (0.87) & (0.00)*** & (0.12)* & (0.22) \\
\quad \textit{Stepdown P-Value} & (0.02)*** & (0.87) & (0.99) & (0.02)*** & (0.95) & (0.94) \\
Locus of Control - positive & -0.10 & -0.70 & 0.11 & 0.07 & -0.09 & 0.49 \\
\quad \textit{Unadjusted P-Value} & (0.33) & (0.02)*** & (0.66) & (0.49) & (0.66) & (0.05)** \\
\quad \textit{Stepdown P-Value} & (0.86) & (0.87) & (0.99) & (0.56) & (0.95) & (0.37) \\
Depression Score - positive & -0.78 & -10.75 & -1.07 & 1.57 & -4.87 & 1.46 \\
\quad \textit{Unadjusted P-Value} & (0.21) & (0.00)*** & (0.26) & (0.02)*** & (0.00)*** & (0.31) \\
\quad \textit{Stepdown P-Value} & (0.80) & (0.38) & (0.99) & (0.18) & (0.95) & (0.95) \\
Volunteers & -0.29 & 0.13 & 0.09 & -0.27 & 0.32 & 0.07 \\
\quad \textit{Unadjusted P-Value} & (0.00)*** & (0.33) & (0.48) & (0.00)*** & (0.00)*** & (0.50) \\
\quad \textit{Stepdown P-Value} & (0.00)*** & (0.91) & (0.99) & (0.00)*** & (0.95) & (0.99) \\
Ever Voted for Municipal & 0.17 & -0.63 & -0.04 & 0.29 & -0.39 & 0.14 \\
\quad \textit{Unadjusted P-Value} & (0.00)*** & (0.00)*** & (0.79) & (0.00)*** & (0.00)*** & (0.24) \\
\quad \textit{Stepdown P-Value} & (0.00)*** & (0.69) & (0.99) & (0.00)*** & (0.95) & (0.94) \\
Ever Voted for Regional & 0.19 & -0.58 & -0.10 & 0.32 & -0.39 & 0.13 \\
\quad \textit{Unadjusted P-Value} & (0.00)*** & (0.00)*** & (0.43) & (0.00)*** & (0.00)*** & (0.31) \\
\quad \textit{Stepdown P-Value} & (0.00)*** & (0.74) & (0.99) & (0.00)*** & (0.95) & (0.95) \\
Num. of Friends & -0.33 & 2.66 & -1.17 & -0.74 & 1.87 & -1.52 \\
\quad \textit{Unadjusted P-Value} & (0.57) & (0.04)*** & (0.57) & (0.35) & (0.24) & (0.25) \\
\quad \textit{Stepdown P-Value} & (0.86) & (0.87) & (0.99) & (0.56) & (0.95) & (0.88) \\
Trust Score & -0.19 & 1.73 & 0.29 & -0.46 & 3.82 & 0.66 \\
\quad \textit{Unadjusted P-Value} & (0.34) & (0.00)*** & (0.49) & (0.02)*** & (0.00)*** & (0.08)** \\
\quad \textit{Stepdown P-Value} & (0.86) & (0.84) & (0.99) & (0.19) & (0.19) & (0.75) \\
\bottomrule
\end{tabular}
}
\vspace{1ex} \\
\footnotesize\raggedright{\underline{Note 1:} This table shows the estimates of the coefficient for attending Reggio Approach preschools from multiple methods. We specify various ways to compare Reggio Approach individuals with age-50 people who did have access to Reggio Approach preschools. We compare Age-50 people, who preceded the Reggio Approach, with people in age-30 and age-40 cohorts who attended Reggio Approach preschools. Column title indicates the corresponding control set and and model. \textbf{OLS30} = OLS estimate that compares Reggio Age-30 people who attended Reggio Approach preschools with Reggio Age-50 people who did not attend any preschool.  \textbf{RDiD30} = difference-in-difference estimate of (Reggio Age-30 Muni - Reggio Age-50 Other) - (Reggio Age-30 None - Reggio Age-50 None). \textbf{OLS40} = OLS estimate that compares Reggio Age-40 people who attended Reggio Approach preschools with Reggio Age-50 people who did not attend any preschool.  \textbf{RDiD40} = difference-in-difference estimate of (Reggio Age-40 Muni - Reggio Age-50 Other) - (Reggio Age-40 None - Reggio Age-50 None). }

\footnotesize\raggedright{\underline{Note 2:} Both unadjusted p-value and stepdown p-value are reported. ***, **, and * indicate significance of the coefficients at the 15\%, 10\%, and 5\% levels respectively.}
\end{table}

