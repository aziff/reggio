The Reggio Approach is a birth to age-6 early childhood program implemented in Reggio Emilia, Italy starting in the early 1960s. It is based on a vision of the child as an individual with rights and potential. It has been a source of inspiration for hundreds of early childhood centers around the world.\footnote{The official \href{http://www.reggiochildren.it/network/?lang=en}{Reggio Children International Network} is present in 33 countries worldwide.} Reggio Approach schools have been awarded numerous prizes.\footnote{Examples include the Danish LEGO Prize (1992), the Kohl Foundation of Chicago award (1993), the Hans Christian Anderson Prize (1994), the Mediterranean Association of International Schools award (1994), the award from the French city of Blois (2001).} Despite its widespread recognition, the Reggio Approach has never been formally evaluated and there is no rigorous empirical evidence of its effects on children's life-cycle outcomes.

This paper presents an evaluation of the Reggio Approach using non-experimental comparison groups constructed from data on individuals from five different age cohorts (three cohorts of adults, one cohort of adolescents, and one cohort of children in their first year of elementary school) in three different cities: Reggio Emilia, Parma, and Padova. Although Parma and Padova are geographically close to Reggio Emilia and similar in economic and demographic characteristics, they have somewhat different preschool systems as described below. At issue is whether or not these differences are consequential. Children in each city are exposed to one of four different early childhood experiences: municipal, state, religious, or none. The Reggio Approach is delivered through the municipal early childhood schools of Reggio Emilia. Our evaluation strategy consists of comparing the outcomes of those who attended municipal institutions in Reggio Emilia (treatment group) to control groups who experienced other preschool types (including no preschool) either in Reggio Emilia or in Parma and Padova.

Our evaluation of the Reggio Approach faces several challenges. First, the non-experimental nature of the data raises concerns about bias from self-selection of individuals into different early childhood programs. We employ a number of econometric techniques in an attempt to control for potential selection problems. Second, other high-quality childcare programs are available in Northern Italy that enroll many youth. In the mid-20th century, Northern Italy witnessed a rise in local early childhood programs many of which were influenced by Loris Malaguzzi as well as other respected early childhood experts \citep{OECD_2001_Italy-Country-Note}. This rise in quality of childcare alternatives was accompanied by an increase in the preschool attendance rate of Italian children aged 3-6 years from 50\% in the 1960s to 96\% in the 1990s \citep{Hohnerlein_2015_Development-and-Diffusion}. The common influences across regions in our control group pose serious problems for any analysis based on comparison groups across cities in the region.  The evidence of common preschool practices currently in place in northern Italy is consistent with two interpretations: (i) that a common influence was at work across towns; or (ii) that the Reggio program was unique, but its essential elements diffused rapidly across towns and alternative schools within the same towns. Malaguzzi was active in promoting high-quality preschool throughout northern Italy.

In this paper, we compare individuals who attended Reggio Approach programs with those who attended other center-based programs within Reggio and in our comparison cities. These estimates capture the benefits of attending the Reggio Approach programs relative to other center-based programs. They are generally small and statistically insignificant. However, when we compare individuals who attended Reggio Approach schools with those who did not attend any center-based program, we find beneficial effects.

In contextualizing our findings, it is essential to understand the heterogeneity in early childhood approaches across school types, cities, and cohorts. Towards this end, Section~\ref{sec:ece-italy} presents key findings from an extensive review of the literature as well as results from a survey we conducted to quantify differences in administrative and pedagogical components among the different school types in the three cities. The survey allows us to track the evolution of differences in approaches in early childhood across cities and across school-types within cities. It is designed to elicit the availability of preschool and infant-toddler centers to cohorts of different ages in our sample. Results from our survey show that non-Reggio-Approach schools have historically shared many of the same features with Reggio Approach schools, and that the commonalities of these features increase over time (across cohorts). Given the overlaps in these features, it is reasonable to expect that comparisons of outcomes for Reggio Approach attendees with outcomes for those who attended alternative programs produce small, possibly negligible, treatment effects.

Results differ across age cohorts and with respect to the control group used. With the exception of some non-cognitive outcomes, we do not find any consistently statistically significant positive effects of the Reggio Approach on children and adolescents. Our most favorable comparisons are for the age-40 adult cohort when we compare Reggio Approach individuals with those from Reggio Emilia who did not attend preschool. Positive and statistically significant effects are estimated for employment, non-cognitive skills, and voting behavior. We do not reject the hypothesis that attending Reggio Approach preschools improved outcomes relative to not attending preschool.

However, when we compare outcomes for Reggio Approach attendees with those who attended alternative preschools within the city, few statistically significant effects are found. If any appear, they are found for the oldest cohorts. The lack of positive and statistically significant results remains when we make comparisons with those who attended municipal programs in other cities, especially Padova.\footnote{This is consistent with historical information about the lower availability of alternative preschools at this time and the unavailability of the municipal system in Padova before the age-30 cohorts.} We do not reject the hypothesis that attending Reggio Approach preschools did not improve outcomes relative to attending other regional preschools. We reach similar conclusions for infant-toddler centers, but the data are much more sparse. \textbf{[JJH: What if we compare attending any preschool attendance in Padova or Parma vs. no attendance? Effects same as in Reggio?][Team: This is shown in Appendix Tables \ref{ols-M-adult30-reg-pres-parma}, \ref{ols-M-adult40-reg-pres-parma}, ref{ols-M-adult30-reg-pres-padova}, \ref{ols-M-adult40-reg-pres-padova}. If we do within-Parma or within-Padova comparison between yes preschool vs. no preschool, some effects are similar to Reggio Approach: better high school grade, better voting behaviors. However, there are even better outcomes for yes preschool group in Parma or Padova: IQ factor, depression, etc..]}

The rest of the paper is organized in the following way. Section \ref{sec:ece-italy} describes the Reggio Approach. We discuss childcare programs in our three comparison group cities drawing from historical records and a survey we constructed and administered to officials across the different areas. Section \ref{sec:data} describes the research design, including the selection of cities, the survey data collection, and the questionnaires. Section \ref{sec:methodology} presents the methods used to estimate the Reggio treatment effects. Section \ref{sec:result} presents our estimates. Section \ref{sec:discussion} discusses the results in the context of historical information on different childcare programs.

