
As our study includes individuals who experienced the Reggio Approach as well as other types of early childhood programs, it is important to examine the similarities and differences of the programs in order to analyze the effects of the Reggio Approach. This section documents different types of early childhood systems in Northern Italy and the Reggio Approach, uniquely offered by the Municipality of Reggio Emilia for children ages 3 months to 6 years. Other early childhood education systems within Reggio Emilia, as well as in Parma and Padova, share certain features of program administration, practices for at-risk children and families, and pedagogical methods. Those who did not experience the Reggio Approach treatment participated in heterogenous early childhood experiences including home-based care, religious preschool for ages 3 to 6 years, state preschool for ages 3 to 6 years, municipal childcare and preschool education offered by Parma and Padova for ages 0 to 6 years. We thus sought to better understand the history and evolution of various early childhood options apart from what was available in the literature. We collected historical documents \citep{Padova-Admin-Data_1964-2011,Reggio-Admin-data_1966-2006,Reggio-Annual-Journals_1994-2011} and administered a historical survey to current and retired school administrators and educative coordinators in order to quantify pedagogical and administrative features of childcare experiences available during 1950-2010 \citep{CEHD_2016_Historical-Analysis}.\footnote{See Appendix~\ref{sec:survey} for the full survey.} 

We begin with a general history and evolution of early childhood policies and programming in Italy to clarify the availability and enrollment of center-based programs in our sample.

\subsection{Early Childhood Education in Italy}

Italy's early childhood policies reflect state Laws 444 and 1044, enacted in 1968 and 1971. These policies mark a key shift by formally legitimizing state involvement in the provision and programming of educational preschool (Law 444, for ages 3-6 years) and infant-toddler childcare (Law 1044, for ages 0-3 years). Prior to 1968, childcare was institutionally provided for children of working mothers and orphans within the religious, welfare, and social service spheres \citep{OECD_2001_Italy-Country-Note,Hohnerlein_2015_Development-and-Diffusion}. The Ministry of Education oversees preschools for ages 3-6 years; the Ministry of Health is responsible for regulating the infant-toddler childcare system \citep{Corsaro_1996_Early-Edu}.\footnote{Ongoing reform efforts led by the region of Emilia Romagna seek to unify a system of early childhood education to provide continuity of care in programs for ages 0-6 years \citep{CEHD_2016_Historical-Analysis}.} 

Accepted literature and administrative records distinguish state from non-state early childhood systems, including municipal, religious, and secular private programs \citep{Padova-Admin-Data_1964-2011,Reggio-Admin-data_1966-2006,Reggio-Annual-Journals_1994-2011,OECD_2001_Italy-Country-Note,Ribolzi_2013_Italy}. State preschools operate according to legislated policies and \textit{Orientamenti}, national guidelines defining program standards and general goals for early childhood education. As non-state programs, municipalities and religious programs are enabled to function autonomously, offering curricula and setting administrative regulations such as eligibility criteria for local preschools and infant-toddler programs. 

The Catholic Church is the oldest non-state early childhood system in Italy, providing religious training for disadvantaged children since the 19th century \citep{OECD_2001_Italy-Country-Note}. The majority of religious programs enroll children aged 3-6 years \citep{Malizia-Cicatelli_2011_BOOK_Catholic-School}. In the 1960's, educators and left-wing leaders within Emilia Romagna including Bruno Ciari and Loris Malaguzzi incited a `municipal school revolution' in Italy by organizing community programs for families with children aged 0-6 years. Influenced by Dewey's progressive model of early childhood education, these educators designed municipal systems of early childhood education as active-child learning alternatives to then-existing childcare models based on welfare, hygiene, and moral socialization. 

Pursuant to state, regional, and varying municipal policies, the availability, enrollment, and educational programming offered by state and local non-state preschool systems for the 5 cohorts in our evaluation varies over time and by city. We document the availability of systems for each cohort in Table~\ref{tab:itc-pre}, defining a system to include programs with 4 or more school sites.

\begin{table}[H]
\centering
\caption{Availability of Preschool Programs by City and School Type}\label{tab:itc-pre}
\begin{adjustbox}{width=\textwidth}
\begin{threeparttable}
	\begin{tabular}{l l c c c c c c c c c}
\toprule
\mc{1}{c}{Cohort} & \mc{1}{c}{Years} & \mc{3}{c}{Reggio Emilia} & \mc{3}{c}{Parma} & \mc{3}{c}{Padova} \\
& & Municipal & Catholic & State & Municipal & Catholic & State & Municipal & Catholic & State \\
\midrule
Adults 50s & 1957-1965 & & \checkmark & & & \checkmark & & & \checkmark & \\
Adults 40s & 1972-1976 & \checkmark & \checkmark & & & \checkmark & & & \checkmark & \\
Adults 30s & 1983-1987 & \checkmark & \checkmark & \checkmark & \checkmark & \checkmark & \checkmark & \checkmark & \checkmark & \checkmark \\
Adolescents & 1994-2000 & \checkmark & \checkmark & \checkmark & \checkmark & \checkmark & \checkmark & \checkmark & \checkmark & \checkmark \\
Children & 2009-2014 & \checkmark & \checkmark & \checkmark & \checkmark & \checkmark & \checkmark & \checkmark & \checkmark & \checkmark \\
\bottomrule
\end{tabular}

% Caption:
% Note: This table indicates the types of educational preschool systems (defined as programs with 4 or more sites) available to parents in each city during the years each cohort was eligible for a 3-6 year old program. 
\begin{tablenotes}
Note: This table indicates the majority of educational preschool systems available to parents in each city during the years each cohort was eligible for a 3-6 year old program. 
\end{tablenotes}
\end{threeparttable}
\end{adjustbox}
\end{table}

