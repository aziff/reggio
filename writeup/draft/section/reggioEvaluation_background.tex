
As our study includes individuals who attended the Reggio Approach program as well as other types of early childhood programs, it is important to examine the similarities and differences of the programs in order to analyze the effects of the Reggio Approach. This section documents different types of early childhood systems in Northern Italy and the Reggio Approach, uniquely offered by the Municipality of Reggio Emilia. Other early childhood education systems within Reggio Emilia, as well as in Parma and Padova, share certain features of program administration, practices for at-risk children and families, and pedagogical methods. Because those who did not experience the Reggio Approach did not receive a homogenous alternative early childhood education experience, we sought to better understand the history and evolution of various early childhood options apart from what was available in the literature. We collected historical documents \citep{Padova-Admin-Data_1964-2011,Reggio-Admin-data_1966-2006,Reggio-Annual-Journals_1994-2011} and administered a historical survey to current and retired school administrators and educative coordinators in order to quantify pedagogical and administrative features of childcare experiences available during 1950-2010 \citep{CEHD_2016_Historical-Analysis}.\footnote{See Appendix~\ref{sec:survey} for the full survey.} 

\subsection{Early Childhood Education in Italy}

Italy's current early childhood policies reflect key state Laws 444 and 1044, passed in 1968 and 1971, mandating the provision and programming of educational preschool and infant-toddler childcare by age.\footnote{Prior to 1968, childcare was institutionally provided within the welfare and religious sector for children of working mothers \citep{OECD_2001_Italy-Country-Note,Hohnerlein_2015_Development-and-Diffusion}.} The Ministry of Education oversees preschools for ages 3-6 years; the Ministry of Health is responsible for regulating the provision of infant-toddler childcare for children up to 3 years old \citep{Corsaro_1996_Early-Edu}.\footnote{Not until the early 2000's did the state formally recognize the role of infant-toddler services as educational in nature for the development of social, emotional, and cognitive skills.}\footnote{Ongoing ECE reform efforts led by the region of Emilia Romagna seek to unify the system of early childhood education to provide continuity of care from ages 0-6 years \citep{CEHD_2016_Historical-Analysis}.} Accepted literature and administrative records distinguish state and non-state early childhood systems, which include municipal, religious, and secular private programs \citep{Padova-Admin-Data_1964-2011,Reggio-Admin-data_1966-2006,Reggio-Annual-Journals_1994-2011,OECD_2001_Italy-Country-Note,Ribolzi_2013_Italy}. Section~\ref{sec:data} describes the selection of our sample into these different school types. 

The state preschool system expanded in the mid-1970's; prior to 1968, about 50\% of children aged 3-6 years in Italy were enrolled in childcare provided by non-state institutions and the Church was the largest provider \citep{Hohnerlein_2009_Paradox-Public-Preschools}. State preschools were funded only where local demand for preschool education was not already met by existing non-state systems \citep{Hohnerlein_2009_Paradox-Public-Preschools}. Today the state is currently the largest provider of preschool education for ages 3-6 years, however, it does not provide infant-toddler childcare. 

The Catholic Church is the oldest non-state early childhood system in Italy, offering both religious training and charitable social services for disadvantaged children since the 19th century \citep{OECD_2001_Italy-Country-Note}. The majority of religious programs enroll children ages 3-6 years. Religious institutions do not currently provide infant-toddler education, however, some religious sites offer programming for children beginning at 24 months \citep{Malizia-Cicatelli_2011_BOOK_Catholic-School}. 

Municipalities organize and fund community early childhood systems for families with children aged 0-6 years. In cities that established high quality municipal programs, fewer children attend state preschool \citep{OECD_2001_Italy-Country-Note}. Municipalities are also mandated to provide enough infant-toddler ``childcare slots'' to meet local demand, however, local provision varies considerably \citep{Saraceno_1984_Soc-Probs}.\footnote{Some northern municipalities provide infant-toddler programs for 30-45\% of children under 3 years, compared to to 2.3\% in the South \citep{Becchi-Ferrari_1990_Pub-Inf-Centres-Italy,Musatti-Picchio_2010_IJEC}.}

\noindent\textbf{[YKK: A brief explanation about the municipal-affiliated preschools should go here. Table 1 also needs municipal-affiliated column.]}

Thus, pursuant to state, regional, and municipal policies, the availability and enrollment in local state and non-state preschool systems for the 5 cohorts in our evaluation varies over time and by city. We summarize the availability of systems for each cohort in Table~\ref{tab:itc-pre}. 

\begin{table}[H]
\centering
\caption{Availability of Infant-toddler Centers and Preschools, by City and School Type}\label{tab:itc-pre}
\begin{adjustbox}{width=\textwidth}
\begin{threeparttable}
	\begin{tabular}{l l c c c c c c c c c}
\toprule
\mc{1}{c}{Cohort} & \mc{1}{c}{Years} & \mc{3}{c}{Reggio Emilia} & \mc{3}{c}{Parma} & \mc{3}{c}{Padova} \\
& & Municipal & Catholic & State & Municipal & Catholic & State & Municipal & Catholic & State \\
\midrule
Adults 50s & 1957-1965 & & \checkmark & & & \checkmark & & & \checkmark & \\
Adults 40s & 1972-1976 & \checkmark & \checkmark & & & \checkmark & & & \checkmark & \\
Adults 30s & 1983-1987 & \checkmark & \checkmark & \checkmark & \checkmark & \checkmark & \checkmark & \checkmark & \checkmark & \checkmark \\
Adolescents & 1994-2000 & \checkmark & \checkmark & \checkmark & \checkmark & \checkmark & \checkmark & \checkmark & \checkmark & \checkmark \\
Children & 2009-2014 & \checkmark & \checkmark & \checkmark & \checkmark & \checkmark & \checkmark & \checkmark & \checkmark & \checkmark \\
\bottomrule
\end{tabular}

% Caption:
% Note: This table indicates the types of educational preschool systems (defined as programs with 4 or more sites) available to parents in each city during the years each cohort was eligible for a 3-6 year old program. 
\begin{tablenotes}
Note: This table indicates the main types of educational preschool systems (defined as programs with 4 or more sites) available to parents in each city during the years each cohort was eligible for a 3-6 year old program. 
\end{tablenotes}
\end{threeparttable}
\end{adjustbox}
\end{table}

\subsection{The Reggio Approach}

The Reggio Approach is a form of progressive early childhood education designed by Loris Malaguzzi, an educator influenced by the educational practices and psychological theories of Dewey, Piaget, Erikson, Vygotsky, Bronfenbrenner, Kagan, and Gardner. Malaguzzi, along with Bruno Ciari in Bologna, was one of several left-wing educators within the region of Emilia Romagna who are credited with inciting a `municipal school revolution' in the 1960's. Under the guidance of Malaguzzi, Reggio Emilia opened its first preschool in 1963 for children aged 3-6 years. In 1965, Reggio Emilia mandated funding for infant-toddler centers for children aged 3 months to 3 years; in 1971, the first site opened. Reggio Emilia's municipal early childhood system thus preceded Italy's legislative reforms in 1968 and 1971 that established state-run preschools and mandated the local provision of infant-toddler centers \citep{Cagliari-etal-eds_2016_BOOK_Loris-Malaguzzi}. 

The Reggio Approach is notable for viewing curriculum as an ongoing, collaborative project without pre-determined learning goals or timelines. There is no institutionally-prescribed content knowledge that educators convey to children for ``school readiness.'' In contrast, teachers and children are viewed as researchers and co-creators of knowledge. For example, educators, children and families collaborate to define a question or topic. Learning is then pursued following a scientific process: theories are shared, tested, and revised through dialogue. 

In Reggio Approach preschools, the educative team is assigned specialized roles. Each incoming class of approximately twenty-five 3-year-olds is assigned two full-time co-teachers (teacher-child ratio of 2:25); at least 1 of the 2 teachers remains with this cohort of homogeneous-aged children for three consecutive years. This extended time provides continuity of care for children and enables strong teacher-family engagement. Each preschool site is further staffed by a full-time atelierista, an instructor with a background in visual arts. Teachers observe children's development, interact with children through questions and dialogue, and provide scaffolding to support learning. Children demonstrate their emerging knowledge through creative learning activities and art, with aid from the atelierista. Teachers document each child's development in a portfolio---a collection of work---which is shared and discussed with children and parents over the year \citep{Rinaldi_2006_ReggioEmilia_BOOK,Giudici-Nicolosi_2014_Reggio-Approach}. Auxiliary site staff, such as cooks and janitors, are considered members of the educative team and participate in trainings and professional development. A pedagogista---or educative coordinator---with a higher degree in psychology or education is assigned to support professional development on a biweekly basis for the educative staff of approximately 4-5 municipal preschools. 

The Reggio Approach school environment reflects a light-filled, open interior design, furnished with natural materials and a garden. Each site is equipped with an atelier, or dedicated studio laboratory used for creative instructional activities. In-house kitchens are surrounded by glass walls, to include children in the meal process, and is used daily for preparing meals. Reggio Approach preschools and infant-toddler centers are open five full-time days per week from September through June \citep{Giudici-Nicolosi_2014_Reggio-Approach}. Extended day options are available at a majority of Reggio municipal sites throughout the school year, as is educational programming throughout July. Children with disabilities and single parents have been prioritized in admission criteria from the early 1970's \citep{Edwards-etal-eds_1998_Hundred-Languages}. The engagement of families is embedded in Reggio practices, as is the invitation to all community members to participate in school management \citep{CEHD_2016_Historical-Analysis,Cagliari-etal-eds_2016_BOOK_Loris-Malaguzzi}. 

While eligible, Reggio Emilia did not receive state funding for its municipal early childhood system until the 1990s and 2000s. Ironically, the municipality contributed funds to local state schools each decade from the 1970s. In 1994, Reggio Approach staff provided training for religious preschool teachers in Reggio Emilia.  

\subsection{Comparison of Early Childhood Systems in Reggio Emilia, Parma, and Padova}

Early childhood education systems in Reggio Emilia, Parma, and Padova share certain features of program administration, practices for at-risk children and families, and pedagogical methods. We describe the alternative early childhood treatments below using published literature, results from our historical survey, and administrative data retrieved from municipal archives of Reggio Emilia and from Reggio Children. We first present the historical survey.

\subsubsection{Overview of Survey} \label{sec:survey-overview}

We conducted a Historical Survey in Reggio Emilia, Parma, and Padova to quantify pedagogical and administrative features of other available childcare experiences available from 1950-2010 \citep{CEHD_2016_Historical-Analysis}. We administered the survey to current and retired school administrators and educative coordinators from each system in each city. Survey responses were received from the following systems that allowed us to document early childhood programming in each municipality from 1960 to 2010: in Reggio Emilia, municipal and state; in Parma, municipal; and in Padova, municipal, state, and religious. Responses were also received from religious systems in Reggio Emilia and in Parma, however, they did not include historical data prior to 2000. 

\noindent\textbf{[YKK: I think since the tables show what questions were asked, it may be better to move the list of all questions to the appendix and just explain them briefly in the main paper, which I did. What do you think?]}

We inquire about administrative and pedagogical components and examine how the components differ in Reggio Approach program and other programs. The questionnaire for administrative components includes detailed questions on teacher training, educative coordinator, other staff members, work hours, enrollment, and funding. The questionnaire for pedagogical components inquire about children's learning experience in various dimensions. For a more detailed summary of questions, see Appendix \ref{sec:survey}.


\subsubsection{Results from Historical Survey}

The historical survey results show the similarities and differences in administrative and pedagogical components between the Reggio Approach and other programs. The general trend shows that the components endorsed by the Reggio Approach have been increasingly included in other systems, albeit in different degrees. We interpret the results from the historical survey as evidence for potential spillovers into alternative programs; it is widely accepted that Malaguzzi and other Reggio Emilia municipal leaders influenced ongoing state policies and legislation. 

In Table \ref{tab:programoperation}, we compare alternative early childhood treatments by considering surveyed reports of administrative operations of early childhood system from 1960 through 2010 against systematic features of Reggio Emilia's municipal program. We find that each of the municipalities, state, and religious systems invested in early childhood programming in different times and in different ways. We find innovation in the Reggio Approach and investment from the Reggio Emilia municipality into local state and religious schools. 

We further recognize the availability of early childhood experts to Parma and Padova in the presence of respected scholars from local universities. Table \ref{tab:programoperation} shows that municipal programs in Parma and Padova contain more program operation components endorsed by the Reggio Approach than other types of programs do, especially after the 1990s.

\begin{table}[H]
\caption{Comparison of Program Operations} \label{tab:programoperation}
\scriptsize
\centering
\begin{tabular}{L{4.5cm} C{0.7cm}  C{1.4cm}  C{1.4cm}  C{1.4cm}  C{1.4cm}  C{1.4cm}  C{1.4cm}}
\toprule
Program Operations	&		&	Reggio Municipal	&	Reggio State	&	Parma Municipal	&	Padova Municipal	&	Padova State	&	Padova Religious	\\	\midrule
\multirow{5}{4.5cm}{Schools receive funding from public sources}	&	1960	&	\checkmark	&		&		&		&		&		\\	
	&	1970	&	\checkmark	&	\checkmark	&		&	\checkmark	&	\checkmark	&		\\	
	&	1980	&	\checkmark	&	\checkmark	&		&	\checkmark	&	\checkmark	&		\\	
	&	1990	&	\checkmark	&	\checkmark	&	\checkmark	&	\checkmark	&	\checkmark	&	\checkmark	\\	
	&	2000	&	\checkmark	&	\checkmark	&	\checkmark	&	\checkmark	&	\checkmark	&	\checkmark	\\	\midrule
\multirow{5}{4.5cm}{Full-time Pedagogistas\footnote{A Pedagogista is a highly trained specialist in early childhood education. In some early childhood systems, this role is referred to as an Educative Coordinator; the training and responsibilities of Educative Coordinators vary across cities and ECE systems.} are hired by the system to oversee professional development for multiple program sites}	&	1960	&	\checkmark	&		&		&		&		&		\\	
	&	1970	&	\checkmark	&		&		&		&		&		\\	
	&	1980	&	\checkmark	&		&		&	\checkmark	&		&		\\	
	&	1990	&	\checkmark	&		&	\checkmark	&	\checkmark	&	\checkmark	&		\\	
	&	2000	&	\checkmark	&		&	\checkmark	&	\checkmark	&	\checkmark	&		\\	\midrule
\multirow{5}{4.5cm}{Professional development is provided by highly trained specialists to each program site every 1-2 weeks\footnote{In Padova's religious programs, professional development is provided by a mix of part-time and full-time employees.}}	&	1960	&	\checkmark	&		&		&		&		&		\\	
	&	1970	&	\checkmark	&		&		&		&		&	\checkmark	\\	
	&	1980	&	\checkmark	&		&		&		&		&	\checkmark	\\	
	&	1990	&	\checkmark	&		&	\checkmark	&		&	\checkmark	&	\checkmark	\\	
	&	2000	&	\checkmark	&		&	\checkmark	&		&	\checkmark	&	\checkmark	\\	\midrule
\multirow{5}{4.5cm}{Kitchen and janitorial staff join educators for professional development}	&	1960	&	\checkmark	&		&		&		&		&		\\	
	&	1970	&	\checkmark	&		&	\checkmark	&		&		&		\\	
	&	1980	&	\checkmark	&		&	\checkmark	&		&		&		\\	
	&	1990	&	\checkmark	&		&	\checkmark	&		&		&		\\	
	&	2000	&	\checkmark	&		&	\checkmark	&	\checkmark	&		&		\\	\midrule
\multirow{5}{4.5cm}{Classrooms are homogeneous in age} 	&	1960	&	\checkmark	&		&		&		&		&		\\	
	&	1970	&	\checkmark	&	\checkmark	&		&		&		&		\\	
	&	1980	&	\checkmark	&	\checkmark	&		&		&		&		\\	
	&	1990	&	\checkmark	&	\checkmark	&		&		&		&		\\	
	&	2000	&	\checkmark	&	\checkmark	&		&		&		&		\\	\midrule
\multirow{5}{4.5cm}{2 co-teachers are assigned to each incoming cohort of 3 year olds. At least 1 teacher stays with the cohort for the next two years to maintain continuity of care}	&	1960	&		&		&		&		&		&		\\	
	&	1970	&	\checkmark	&	\checkmark	&		&		&		&		\\	
	&	1980	&	\checkmark	&	\checkmark	&	\checkmark	&	\checkmark	&		&		\\	
	&	1990	&	\checkmark	&	\checkmark	&	\checkmark	&	\checkmark	&		&		\\	
	&	2000	&	\checkmark	&	\checkmark	&	\checkmark	&		&		&		\\	\midrule
\multirow{5}{4.5cm}{Full-time Atelierista\footnote{An Atelierista is an expert in the creative arts who designs creative learning activities and supports children's learning.} is staffed at each preschool site and collaborates with classroom teachers to design creative learning activities}
	&	1960	&	\checkmark	&		&		&		&		&		\\	
	&	1970	&	\checkmark	&		&		&		&		&		\\	
	&	1980	&	\checkmark	&		&		&	\checkmark	&		&		\\	
	&	1990	&	\checkmark	&		&		&	\checkmark	&		&		\\	
	&	2000	&	\checkmark	&		&		&	\checkmark	&		&		\\	\midrule
\multirow{5}{4.5cm}{Scheduled work hours are set aside weekly for teachers to document children's work}	&	1960	&	\checkmark	&		&		&		&		&		\\	
	&	1970	&	\checkmark	&		&		&	\checkmark	&		&		\\	
	&	1980	&	\checkmark	&		&	\checkmark	&	\checkmark	&		&	\checkmark	\\	
	&	1990	&	\checkmark	&		&	\checkmark	&	\checkmark	&		&	\checkmark	\\	
	&	2000	&	\checkmark	&		&	\checkmark	&	\checkmark	&		&	\checkmark	\\	\midrule
\multirow{5}{4.5cm}{Scheduled hours are set aside weekly for teachers to engage families}	&	1960	&	\checkmark	&		&		&		&		&		\\	
	&	1970	&	\checkmark	&		&	\checkmark	&		&		&	\checkmark	\\	
	&	1980	&	\checkmark	&		&	\checkmark	&		&		&	\checkmark	\\	
	&	1990	&	\checkmark	&		&	\checkmark	&	\checkmark	&		&	\checkmark	\\	
	&	2000	&	\checkmark	&		&	\checkmark	&	\checkmark	&		&	\checkmark	\\	\midrule
\multirow{5}{4.5cm}{Parental boards or advisory groups are encouraged as active participants in school culture}	&	1960	&	\checkmark	&		&		&		&		&		\\	
	&	1970	&	\checkmark	&	\checkmark	&	\checkmark	&	\checkmark	&		&	\checkmark	\\	
	&	1980	&	\checkmark	&	\checkmark	&	\checkmark	&	\checkmark	&		&	\checkmark	\\	
	&	1990	&	\checkmark	&	\checkmark	&	\checkmark	&	\checkmark	&		&	\checkmark	\\	
	&	2000	&	\checkmark	&	\checkmark	&	\checkmark	&	\checkmark	&		&	\checkmark	\\	\bottomrule

\end{tabular}
\end{table} 

In Table \ref{tab:administrative-atrisk}, we compare alternative early childhood treatments by presenting surveyed reports of administrative practices for at-risk children and families in each early childhood system from 1960 through 2010 relative to Reggio Emilia's municipal program.

The hours of center-based care is largely shared by the surveyed programs. All systems (except the state preschools in Padova) offered additional hours for working families. Similarly, by the 1990s, all surveyed systems received public funding. Municipal schools in the three cities shared priorities of enrolling economically disadvantaged children, children from a single-parent household, and children with disabilities.

\begin{table}[H] \caption{Comparison of Administrative Practices for At-Risk Children and Families}	\label{tab:administrative-atrisk}														
\scriptsize																	
\centering																	
\begin{tabular}{L{4.3cm} C{0.7cm}  C{1.4cm}  C{1.4cm}  C{1.4cm}  C{1.4cm}  C{1.4cm}  C{1.4cm}}															
\toprule																	
Administrative Practices for At-Risk Children and Families		&		&	Reggio Municipal	&	Reggio State	&	Parma Municipal	&	Padova Municipal	&	Padova State	&	Padova Religious	\\	\midrule
\multirow{5}{4.5cm}{Preschools are open 8 hours daily} 	&	1960	&	\checkmark	&		&	\checkmark	&		&		&		\\	
		&	1970	&	\checkmark	&	\checkmark	&	\checkmark	&	\checkmark	&	\checkmark	&	\checkmark	\\	
		&	1980	&	\checkmark	&	\checkmark	&	\checkmark	&	\checkmark	&	\checkmark	&	\checkmark	\\	
		&	1990	&	\checkmark	&	\checkmark	&	\checkmark	&	\checkmark	&	\checkmark	&	\checkmark	\\	
		&	2000	&	\checkmark	&	\checkmark	&	\checkmark	&	\checkmark	&	\checkmark	&	\checkmark	\\	\midrule
\multirow{5}{4.5cm}{Program sites offer extended hours for working families}	&	1960	&	\checkmark	&		&	\checkmark	&		&		&		\\	
		&	1970	&	\checkmark	&	\checkmark	&	\checkmark	&	\checkmark	&		&	\checkmark	\\	
		&	1980	&	\checkmark	&	\checkmark	&	\checkmark	&	\checkmark	&		&	\checkmark	\\	
		&	1990	&	\checkmark	&	\checkmark	&	\checkmark	&	\checkmark	&		&	\checkmark	\\	
		&	2000	&	\checkmark	&	\checkmark	&	\checkmark	&	\checkmark	&		&	\checkmark	\\	\midrule
\multirow{5}{4.5cm}{Priority of enrollment is given to economically disadvantaged families}	&	1960	&	\checkmark	&		&	\checkmark	&		&		&		\\	
		&	1970	&	\checkmark	&	\checkmark	&	\checkmark	&	\checkmark	&		&		\\	
		&	1980	&	\checkmark	&	\checkmark	&	\checkmark	&	\checkmark	&		&		\\	
		&	1990	&	\checkmark	&	\checkmark	&	\checkmark	&	\checkmark	&		&		\\	
		&	2000	&	\checkmark	&	\checkmark	&	\checkmark	&	\checkmark	&		&		\\	\midrule
\multirow{5}{4.5cm}{Priority of enrollment is given to children with disabilities}	&	1960	&	\checkmark	&		&		&		&		&		\\	
		&	1970	&	\checkmark	&	\checkmark	&		&	\checkmark	&	\checkmark	&		\\	
		&	1980	&	\checkmark	&	\checkmark	&		&	\checkmark	&	\checkmark	&		\\	
		&	1990	&	\checkmark	&	\checkmark	&	\checkmark	&	\checkmark	&	\checkmark	&	\checkmark	\\	
		&	2000	&	\checkmark	&	\checkmark	&	\checkmark	&	\checkmark	&	\checkmark	&	\checkmark	\\	\midrule
\multirow{5}{4.5cm}{Priority of enrollment is given to single-parent families}	&	1960	&	\checkmark	&		&		&		&		&		\\	
		&	1970	&	\checkmark	&		&		&	\checkmark	&		&		\\	
		&	1980	&	\checkmark	&		&	\checkmark	&	\checkmark	&		&		\\	
		&	1990	&	\checkmark	&		&	\checkmark	&	\checkmark	&	\checkmark	&		\\	
		&	2000	&	\checkmark	&		&	\checkmark	&	\checkmark	&	\checkmark	&		\\	\bottomrule
\end{tabular}																	
\end{table}											

In Table \ref{tab:educ-program}, we compare alternative early childhood treatments by considering surveyed reports of educational programming in each early childhood system from 1960 through 2010 against systematic features of Reggio Emilia's municipal program.  

Municipal systems in three cities after the 1990s all endorse research-based curriculum, have teachers document children's learning, and are influenced by academic theories of psychology and early childhood interventions. Despite these similarities, municipal systems in Parma and Padova implement daily activities that guide children in learning specific concepts and provide religious teaching, which are not the components in the Reggio Approach. The state systems in Reggio Emilia and Padova appears to share almost no educational programming components that the Reggio Approach schools have. 

\begin{table}[H]
\caption{Comparison of Educational Programming} \label{tab:educ-program}																
\scriptsize																	
\centering																	
\begin{tabular}{L{4.6cm} C{0.6cm}  C{1.4cm}  C{1.4cm}  C{1.4cm}  C{1.4cm}  C{1.4cm}  C{1.4cm}}															
\toprule																	
Pedagogical Components 	&		&	Reggio Municipal	&	Reggio State	&	Parma Municipal	&	Padova Municipal	&	Padova State	&	Padova Religious	\\	\midrule
\multirow{5}{4.5cm}{Curriculum emerges through research-based projects with unlimited timelines}	&	1960	&	\checkmark	&		&		&		&		&		\\	
		&	1970	&	\checkmark	&		&		&		&		&		\\	
		&	1980	&	\checkmark	&		&		&	\checkmark	&		&		\\	
		&	1990	&	\checkmark	&		&	\checkmark	&	\checkmark	&		&		\\	
		&	2000	&	\checkmark	&		&	\checkmark	&	\checkmark	&		&		\\	\midrule
\multirow{5}{4.5cm}{Visual arts help children learn}	&	1960	&	\checkmark	&		&	\checkmark	&		&		&	\checkmark	\\	
		&	1970	&	\checkmark	&		&	\checkmark	&		&		&	\checkmark	\\	
		&	1980	&	\checkmark	&		&	\checkmark	&		&		&	\checkmark	\\	
		&	1990	&	\checkmark	&		&	\checkmark	&		&		&	\checkmark	\\	
		&	2000	&	\checkmark	&		&	\checkmark	&		&		&	\checkmark	\\	\midrule
\multirow{5}{4.5cm}{Teachers document children's learning}	&	1960	&	\checkmark	&		&		&		&		&		\\	
		&	1970	&	\checkmark	&		&		&	\checkmark	&		&	\checkmark	\\	
		&	1980	&	\checkmark	&		&	\checkmark	&	\checkmark	&		&	\checkmark	\\	
		&	1990	&	\checkmark	&		&	\checkmark	&	\checkmark	&		&	\checkmark	\\	
		&	2000	&	\checkmark	&		&	\checkmark	&	\checkmark	&	\checkmark	&	\checkmark	\\	\midrule
\multirow{5}{4.5cm}{Educational practices promoted by Loris Malaguzzi for early childhood influenced the daily program}	&	1960	&	\checkmark	&		&		&		&		&		\\	
		&	1970	&	\checkmark	&		&		&		&		&		\\	
		&	1980	&	\checkmark	&		&		&		&		&		\\	
		&	1990	&	\checkmark	&		&		&		&		&		\\	
		&	2000	&	\checkmark	&		&		&		&		&	\checkmark	\\	\midrule
\multirow{5}{4.5cm}{Academic theories of psychology and early childhood education (e.g. Bloom, Bruner, Gardner, Piaget, Vygotsky) influenced educational approaches}	&	1960	&	\checkmark	&		&		&		&		&		\\	
		&	1970	&	\checkmark	&		&		&		&		&		\\	
		&	1980	&	\checkmark	&		&	\checkmark	&	\checkmark	&		&	\checkmark	\\	
		&	1990	&	\checkmark	&		&	\checkmark	&	\checkmark	&		&	\checkmark	\\	
		&	2000	&	\checkmark	&		&	\checkmark	&	\checkmark	&		&	\checkmark	\\	\midrule
\multirow{5}{4.5cm}{Early childhood methodologies endorsed by Agazzi, Froebl, and/or Montessori influenced the daily program}	&	1960	&		&		&	\checkmark	&		&		&		\\	
		&	1970	&		&	\checkmark	&	\checkmark	&		&		&	\checkmark	\\	
		&	1980	&		&	\checkmark	&		&		&		&	\checkmark	\\	
		&	1990	&		&	\checkmark	&		&		&		&	\checkmark	\\	
		&	2000	&		&	\checkmark	&		&		&		&	\checkmark	\\	\midrule
\multirow{5}{4.5cm}{Daily activities are implemented by following a program, to guide children in acquiring knowledge of specific concepts} 	&	1960	&		&		&	\checkmark	&		&		&		\\	
		&	1970	&		&	\checkmark	&	\checkmark	&	\checkmark	&		&	\checkmark	\\	
		&	1980	&		&	\checkmark	&	\checkmark	&	\checkmark	&	\checkmark	&	\checkmark	\\	
		&	1990	&		&	\checkmark	&	\checkmark	&	\checkmark	&	\checkmark	&	\checkmark	\\	
		&	2000	&		&	\checkmark	&	\checkmark	&	\checkmark	&	\checkmark	&	\checkmark	\\	\midrule
\multirow{5}{4.5cm}{The educational program is designed to promote morality, patriotism, and customs of family life}	&	1960	&		&		&	\checkmark	&		&		&	\checkmark	\\	
		&	1970	&		&	\checkmark	&	\checkmark	&		&	\checkmark	&	\checkmark	\\	
		&	1980	&		&	\checkmark	&		&		&	\checkmark	&	\checkmark	\\	
		&	1990	&		&	\checkmark	&		&		&	\checkmark	&	\checkmark	\\	
		&	2000	&		&	\checkmark	&		&		&	\checkmark	&	\checkmark	\\	\midrule
\multirow{5}{4.5cm}{Religious teaching is provided}	&	1960	&		&		&		&		&		&	\checkmark	\\	
		&	1970	&		&	\checkmark	&		&	\checkmark	&	\checkmark	&	\checkmark	\\	
		&	1980	&		&	\checkmark	&	\checkmark	&	\checkmark	&	\checkmark	&	\checkmark	\\	
		&	1990	&		&	\checkmark	&	\checkmark	&	\checkmark	&		&	\checkmark	\\	
		&	2000	&		&	\checkmark	&	\checkmark	&	\checkmark	&		&	\checkmark	\\	\bottomrule
\end{tabular}																	
\end{table}																	

In Table \ref{tab:environ-features}, we compare alternative early childhood treatments by considering surveyed reports of environmental components of early childhood system from 1960 through 2010 relative to systematic features of Reggio Emilia's municipal program. Municipal systems in three cities are shown to share similar environmental features, such as the presence of Atelier and open spaces.   

\begin{table}[H]
\caption{Comparison of Environmental Features} \label{tab:environ-features}			
												
\scriptsize																	
\centering																	
\begin{tabular}{L{4.5cm} C{0.7cm}  C{1.4cm}  C{1.4cm}  C{1.4cm}  C{1.4cm}  C{1.4cm}  C{1.4cm}}															
\toprule																	
	School Environment	&		&	Reggio Municipal	&	Reggio State	&	Parma Municipal	&	Padova Municipal	&	Padova State	&	Padova Religious	\\	\midrule
\multirow{5}{4.5cm}{Each preschool site includes an Atelier (or dedicated room) where children from different classrooms work individually or in small groups}	&	1960	&	\checkmark	&		&		&		&		&		\\	
		&	1970	&	\checkmark	&		&	\checkmark	&		&		&		\\	
		&	1980	&	\checkmark	&		&	\checkmark	&	\checkmark	&		&	\checkmark	\\	
		&	1990	&	\checkmark	&		&	\checkmark	&	\checkmark	&		&	\checkmark	\\	
		&	2000	&	\checkmark	&		&	\checkmark	&	\checkmark	&		&	\checkmark	\\	\midrule
\multirow{5}{4.5cm}{Open spaces, natural lighting and the use of natural furnishings are emphasized}	&	1960	&	\checkmark	&		&	\checkmark	&		&		&		\\	
		&	1970	&	\checkmark	&		&	\checkmark	&		&		&		\\	
		&	1980	&	\checkmark	&		&	\checkmark	&	\checkmark	&		&		\\	
		&	1990	&	\checkmark	&		&	\checkmark	&	\checkmark	&		&		\\	
		&	2000	&	\checkmark	&		&	\checkmark	&	\checkmark	&		&		\\	\bottomrule
\end{tabular}																	
\end{table}						
					
					To summarize, municipal systems in Reggio Emilia, Parma, and Padova share similar components in program operations, administrative practices for at-risk children, educational programming, and environmental features. \textbf{[YKK: I think more summary description is needed for difference between the Reggio Approach and Reggio state, Padova state, Padova religious, municipal-affiliated (in general), because they appear in the later sections. I also think it may be better to briefly explain why Reggio religious and Parma Religious and State are not included in the survey. I think we might need to (1) acknowledge that we are not able to get enough information about schools not surveyed and the reasons why and (2) briefly explain the differences between Reggio Approach and the systems that are not surveyed.]}			
