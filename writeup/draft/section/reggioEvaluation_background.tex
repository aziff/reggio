We first document the policies and different types of current early childhood systems in Italy. We then describe the Reggio Approach, which we consider to be the treatment in this evaluation. Because those who did not receive this treatment did not receive a homogenous alternative early childhood education experience, we sought to better understand the history and evolution of various early childhood options apart from what was available in the literature. 

We thus searched municipal archives to document child enrollment and teacher staffing of local schools; we successfully gathered historical materials from the municipal offices of Reggio Emilia and Padova, and from Reggio Children \citep{Padova-Admin-Data_1964-2011,Reggio-Admin-data_1966-2006,Reggio-Annual-Journals_1994-2011}. 

We further conducted a Historical Survey to quantify pedagogical and administrative features of other available childcare experiences available in Reggio Emilia, Parma, and Padova from 1950-2010. We administered the survey to current and retired school administrators and educative coordinators from each system in each city. Survey responses were received that allowed us to document the following programs in each decade from 1950: in Reggio Emilia, municipal and state; in Parma, municipal; and in Padova, municipal, state, and religious. Responses were also received from religious systems in Reggio Emilia and in Parma, however, they did not include historical data prior to 2000. 

We present this survey and its findings to offer a detailed description and comparison of early childhood experiences available to each of the cohorts in this evaluation. See Appendix~\ref{sec:survey} for the full survey.

\subsection{Early Childhood Education in Italy}

Italy's current early childhood policies reflect key state Laws 444 and 1044, passed in 1968 and 1971, mandating the provision and programming of educational childcare by age.\footnote{Prior to 1968, childcare was institutionally provided within the welfare and religious sector for children of working mothers \citep{OECD_2001_Italy-Country-Note,Hohnerlein_2015_Development-and-DiffusionEnrollment}.} The Ministry of Education oversees preschools for children ages 3-6 years; the Ministry of Health is responsible for regulating the provision of infant-toddler childcare for children up to 3 years old \citep{Corsaro_1996_Early-Edu}.\footnote{Current ECE reform efforts seek to unify the system of early childhood education to provide continuity of care from ages 0-6 years \citep{CEHD_2016_Historical-Analysis}.} Accepted literature and administrative records distinguish state and non-state early childhood systems, which include municipal, religious, and secular private programs \citep{Padova-Admin-Data_1964-2011,Reggio-Admin-data_1966-2006,Reggio-Annual-Journals_1994-2011,OECD_2001_Italy-Country-Note,Ribolzi_2013_Italy}. 

Prior to 1968, about 50\% of children aged 3-6 years in Italy were enrolled in childcare provided by non-state institutions \citep{Hohnerlein Paradox 2009}. By 1990, over 96\% of Italian children aged 3-6 years attended preschool, and the state was the main provider \citep{Hohnerlein_2015_Development-and-DiffusionEnrollment}.\footnote{In contrast, provision of infant-toddler childcare varies considerably across Italy; some municipalities in Northern Italy meet the childcare demand for 30-45\% of families with children under age 3 years, compared to to 2.3\% in the South \citep{Becchi-Ferrari_1990_Pub-Inf-Centres-Italy,Musatti-Picchio_2010_IJEC}.} As the state preschool system expanded, enrollment in religious programs declined from more than 50\% in the 1950s to less than 20\% by 2000. In cities that established high quality municipal programs, fewer children attend state preschool \citep{OECD_2001_Italy-Country-Note}. 

Thus, pursuant to state, regional, and municipal policies, the availability and enrollment in local state and non-state preschool systems for the 5 cohorts in our evaluation varies over time and by city. We summarize this in Table~\ref{tab:itc-pre}. 

\begin{table}[H]
\centering
\caption{Availability of Infant-toddler Centers and Preschools, by City and School Type}\label{tab:itc-pre}
\begin{adjustbox}{width=\textwidth}
\begin{threeparttable}
	\begin{tabular}{l l c c c c c c c c c}
\toprule
\mc{1}{c}{Cohort} & \mc{1}{c}{Years} & \mc{3}{c}{Reggio Emilia} & \mc{3}{c}{Parma} & \mc{3}{c}{Padova} \\
& & Municipal & Catholic & State & Municipal & Catholic & State & Municipal & Catholic & State \\
\midrule
Adults 50s & 1957-1965 & & \checkmark & & & \checkmark & & & \checkmark & \\
Adults 40s & 1972-1976 & \checkmark & \checkmark & & & \checkmark & & & \checkmark & \\
Adults 30s & 1983-1987 & \checkmark & \checkmark & \checkmark & \checkmark & \checkmark & \checkmark & \checkmark & \checkmark & \checkmark \\
Adolescents & 1994-2000 & \checkmark & \checkmark & \checkmark & \checkmark & \checkmark & \checkmark & \checkmark & \checkmark & \checkmark \\
Children & 2009-2014 & \checkmark & \checkmark & \checkmark & \checkmark & \checkmark & \checkmark & \checkmark & \checkmark & \checkmark \\
\bottomrule
\end{tabular}

% Caption:
% Note: This table indicates the types of educational preschool systems (defined as programs with 4 or more sites) available to parents in each city during the years each cohort was eligible for a 3-6 year old program. 
\begin{tablenotes}
Note: This table indicates the main types of educational preschool systems (defined as programs with 4 or more sites) available to parents in each city during the years each cohort was eligible for a 3-6 year old program. 
\end{tablenotes}
\end{threeparttable}
\end{adjustbox}
\end{table}

\subsection{State and Non-State Early Childhood Systems}

Infant-toddler childcare and preschool education are not uniformly regulated, funded, and managed by state and non-state systems. The state does not offer infant-toddler programs. It is currently, however, the largest provider of preschool education for ages 3-6 years. State preschools are only funded where local demand for preschool education is not already met by existing non-state systems \citep{Hohnerlein Paradox 2009}. State preschools charge only for meals and request voluntary contributions from families for extra programming such as field trips \citep{CEHD_2016_Historical-Analysis}. 

Non-state early childhood programs are largely enabled to function autonomously. Access to public funding for non-state schools varies, as does tuition to families. After 2000, state policies were revised to acknowledge non-state schools that complied with eight regulations; `equitable' programs enjoyed increased access to public funds. The distribution of state subsidies for equitable non-state schools is decided by the region \citep{Hohnerlein_2009_Paradox-Public-Preschools,Ribolzi_2013_Italy}. 

The Catholic Church is the oldest non-state early childhood system in Italy, offering both religious training and charitable social services for disadvantaged children since the 19th century \citep{OECD_2001_Italy-Country-Note}. The majority of religious schools enroll children ages 3-6 years; some religious sites offer programming for children beginning at 24 months \citep{Malizia-Cicatelli_2011_BOOK_Catholic-School}. Historically, religious preschools were considered ``schools for the rich'' with tuition purely the family's responsibility \citep{Ribolzi_2013_Italy}. Tuition for religious preschool now depends on family income. 

Municipalities organize and fund community early childhood systems for families with children aged 0-6 years. To comply with key state laws, municipalities must also provide enough infant-toddler ``childcare slots'' to meet local demand \citep{Saraceno_1984_Soc-Probs}. Some municipalities thus contract with private providers to offer a number of infant-toddler and preschool slots according to municipal regulations of eligibility and fees. In general, children of working mothers receive priority enrollment by municipal systems \citep{Saraceno_1984_Soc-Probs}.\footnote{Age of entry into infant-toddler programs varies across municipalities \citep{CEHD_2016_Historical-Analysis}. And while selection criteria are similar across municipalities, the weighting of distinct family characteristics varies \citep{Del-Boca-etal_2016_CESifo-ES}.} Some municipalities provide free preschool while others are offered on a sliding scale. Parental fees for municipal infant-toddler childcare are much higher, covering 21\% of total program costs on average \citep{Musatti-Picchio_2010_IJEC}. 

To summarize, early childhood education in Italy is publicly provided by the state, by the municipality, by religious institutions, and by secular private organizations. Section~\ref{sec:data} describes the selection of our sample into these different school types. 

\subsection{The Reggio Approach}

The Reggio Approach is a form of progressive early childhood education designed by Loris Malaguzzi, an educator influenced by the educational practices and psychological theories of Dewey, Piaget, Erikson, Vygotsky, Bronfenbrenner, Kagan, and Gardner. Malaguzzi, along with Bruno Ciari in Bologna, was one of several left-wing educators within the region of Emilia Romagna who are credited with inciting a `municipal school revolution' in the 1960's. Under the guidance of Malaguzzi, Reggio Emilia opened its first preschool in 1963 for children aged 3-6 years. In 1965, Reggio Emilia passed laws regulating funding for the first infant-toddler center for children aged 3 months to 3 years which later opened in 1971. The Reggio Emilia municipal early childhood system thus preceded Italy's legislative reforms in the 1960s and 1970s that established state-run preschools and mandated the local provision of infant-toddler centers \citep{Cagliari-etal-eds_2016_BOOK_Loris-Malaguzzi}. 

In Reggio Approach preschools, the educative team is assigned specialized roles. Each incoming class of approximately twenty-five 3-year-olds is assigned two full-time co-teachers; at least 1 of the 2 teachers remains with this cohort of homogeneous-aged children for three consecutive years. This extended time provides continuity of care for children and enables strong teacher-family engagement. Each school site is further staffed by a full-time atelierista, an instructor with a background in visual arts. Auxiliary site staff, such as cooks and janitors, are considered members of the educative team and participate in trainings and professional development. A pedagogista---or educative coordinator---with a higher degree in psychology or education is assigned to support professional development on a biweekly basis for the educative staff of approximately 4-5 municipal preschools. 

The Reggio Approach school environment reflects a light-filled, open interior design, furnished with natural materials and a garden. Each site is equipped with an atelier, or dedicated studio laboratory used for creative instructional activities. In-house kitchens are surrounded by glass walls, to include children in the meal process, and is used daily for preparing meals. The Reggio Approach is notable for viewing curriculum as an ongoing, collaborative project without pre-determined learning goals or timelines. There is no institutionally-prescribed content knowledge that educators convey to children for ``school readiness.'' In contrast, teachers and children are viewed as researchers and co-creators of knowledge. For example, educators, children and families collaborate to define a question or topic. Learning is then pursued following a scientific process: theories are shared, tested, and revised through dialogue. Teachers observe children's development, interact with children through questions and dialogue, and provide scaffolding to support learning. Children demonstrate their emerging knowledge through creative learning activities and art, with aid from the atelierista. Teachers document each child's development in a portfolio---a collection of work---which is shared and discussed with children and parents over the year \citep{Rinaldi_2006_ReggioEmilia_BOOK,Giudici-Nicolosi_2014_Reggio-Approach}. 

Reggio Approach preschools and infant-toddler centers are open five full-time days per week from September through June \citep{Giudici-Nicolosi_2014_Reggio-Approach}. Extended day options are available at a majority of Reggio municipal sites throughout the school year, as is educational programming throughout July. Children with disabilities and single parents have been prioritized in admission criteria from the early 1970's \citep{Edwards-etal-eds_1998_Hundred-Languages}. The engagement of families is embedded in Reggio practices, as is the invitation to all community members to participate in school management \citep{CEHD_2016_Historical-Analysis,Cagliari-etal-eds_2016_BOOK_Loris-Malaguzzi}. 

\subsection{Overview of Survey}
Below is a list of the administrative and pedagogical components that we inquire about. Components with a * next to them are present in the Reggio Approach. Components with a $^o$ are omitted. We omit these components because we received feedback from survey respondents that those questions were interpreted differently than originally drafted in the English version. These components were assembled based on published information of the Reggio Approach, and confirmed by expert scholars with firsthand knowledge of the Reggio Approach and early childhood programs in northern Italy.\footnote{See \citet{Edwards-etal-eds_1998_Hundred-Languages} and \citet{Corsaro_2008_Policy-Practice}.}

\begin{itemize}
 \item Administrative components
 \begin{itemize}
 	\item All teachers graduated from a teacher training institution, in accordance with national guidelines.$^o$
 	\item Full-time educative coordinators, with a university degree in psychology or education, were hired by the school system.*
 	\item Educative coordinators met biweekly with educative staff to provide mentoring and professional development.*
 	\item Kitchen staff participated in professional development and routine trainings with teachers.*
 	\item Janitorial staff participated in professional development and routine trainings with teachers.*
 	\item Teachers participated in professional development with teachers from other school systems (e.g. municipal and private Catholic).*
 	\item Schools were open daily for 8 hours.*
 	\item Schools offered extended hours for working families.*
 	\item Scheduled work hours are set aside weekly for teachers to engage families.*
 	\item Scheduled work hours are set aside weekly for teachers to document children's work.*
 	\item Scheduled work hours are set aside weekly for teachers to participate in professional development.*
 	\item Priority of enrollment is given to economically disadvantaged families.*
 	\item Priority of enrollment is given to single-parent families.*
 	\item Priority of enrollment is given to children with disabilities.*
 	\item Schools received funding from public sources.*
	\item Schools received equitable funding from public sources.$^o$
	\item Schools acquired ``paritaria'' status from the state.*$^o$
 \end{itemize}
 \item Pedagogical components
 \begin{itemize}
 	\item Daily activities were implemented by following a predefined program to guide children in acquiring knowledge of specific concepts.
 	\item Classrooms were homogenous in age.*
 	\item Two co-teachers were assigned to the same group of children. Continuity of care provided by keeping at least one teacher with the same group from year to year.*
 	\item A full-time, on-site teacher with specific training or experience in the fine arts helped educators design creative learning activities.*
 	\item Fine arts were used as a tool to help children learn.*
 	\item Children participate in religious teaching.
 	\item Teachers document children's learning in portfolios.*
 	\item The design of the school environment emphasizes open spaces, natural lighting, and the use of natural materials for furniture.*
 	\item The school environment included a dedicated room where children from different classrooms work individually or in small groups.*
 	\item An on-site kitchen was used daily to prepare meals.*$^o$
 	\item Project-based learning with unlimited timelines shapes the educational program.*
 	\item Academic theories of psychology and early childhood education influenced educational approaches.*
 	\item Early childhood practices endorsed by Agazzi, Froebl, and/or Montessori influenced the daily program.
 	\item Early childhood practices promoted by Loris Malaguzzi influenced the daily program.*
 	\item The educational program is designed to promote good morals of family life, and is based on love of family and the homeland.
 	\item Parental boards or advisory groups were encouraged and active participants in school culture.*
	\item Transitions between schools were supported by teacher visits to homes or scheduled visits for children to new schools.$^o$
 \end{itemize}
 \end{itemize}
 
The following information comes from a variety of sources: \citet{Reggio-Admin-data_1966-2006, Reggio-Annual-Journals_1994-2011, Padova-Admin-Data_1964-2011} and results from the historical survey \citep{CEHD_2016_Historical-Analysis}.
 
\subsection{Administrative Commonalities}

Results from our survey suggest the following administrative components were shared among municipal systems in each city, state preschools in Padova, and the religious schools of Padova: 

The hours of center-based care is largely shared by the surveyed programs. All systems (except the state preschools in Padova) offered additional hours for working families. Similarly, by the 1990s, all surveyed systems received public funding. Municipal schools in the three cities shared priorities of enrolling economically disadvantaged children, children from a single-parent household, and children with disabilities. 
 
%\subsubsection{Sources of Funding and Costs to Families }
%Until the early 2000s, tuition and fees to families enrolling children in religious preschools in all three cities were relatively more expensive than municipal and state programs. After the 2000s, religious schools acknowledged by the state for meeting quality components were eligible to receive state funding. Public funding across the three school systems is not equitably distributed. 

%Religious schools in Padova did not receive any form of public funding in the 1970s; families were responsible for 100\% of the costs. In the 1980s and 1990s, the municipality of Padova contributed 20\% and 40\% of program costs to religious schools. In the 2000s, families paid 60\% and the remaining 40\% was shared by the state and municipality of Padova. Municipal schools in Padova are free, families pay only for meals. For state schools, families in Padova also make a voluntary contribution, usually to accommodate expenses associated with field trips.\footnote{This information is further supported by an interview with Dr. Emilia Restiglian of University of Padua.}  
 
%Although eligible, the municipality of Reggio Emilia did not receive state funding for its preschool system until the 1990s and 2000s. Ironically, the municipality of Reggio Emilia contributed funds to its state schools each decade since the 1970s. Reggio Emilia also provided training for religious school teachers, beginning in 1994.  

\subsection{Differences Amongst Early Childhood Systems}

Early childhood education systems in Reggio Emilia, Parma, and Padova differ in certain aspects of program administration, environmental features, and pedagogical methods. We describe the alternative early childhood systems below using survey results, administrative municipal records, and published literature.

\subsubsection{Teacher-Child Ratios}

Although in Reggio Emilia, the teacher-child ratio for each classroom has been 2:25 since the 1960s, this number does not reflect the atelierista present at each school site, nor the pedagogista who supervises the educative staff of 4-5 schools. In Padova, the municipal preschool system began to consolidate in 1973, expanding from two to five sites by 1976. Teacher-child ratios for Padova's municipal preschools ranged from 1:12 to 1:24 in 1976. There were three state preschools in Padova by 1976; enrollment was relatively lower and teacher-child ratio approximately 1:15. In this same period, teacher-child ratios at religious schools ranged from 1:34-44.

\subsubsection{State Preschools}

State preschools operate according to legislated policies and \textit{Orientamenti}, or national guidelines defining program standards and general goals for early childhood education. Historically, policies were not consistently enforced throughout Italy, nor were Orientamenti considered binding. They are further revised periodically to reflect current educational practices and political ideology. The first Orientamenti for a system of free state preschools focused on education, development and care were published in 1969 \citep{Corsaro_1996_Early-Edu,Hohnerlein_2015_Development-and-DiffusionEnrollment}.

The cohorts in our sample thus had differential access to state preschools, and those that attended participated in varying early childhood curricula and administrative practices. While the modern state system was legislated in 1968, historical survey data and administrative records suggest that state preschools did not begin to appear until approximately 1973-1975 in both Reggio Emilia and Padova. The Age 40 cohort, however, had access to 3 or fewer state preschools in each city \citep{Reggio-Admin-data_1966-2006, Reggio-Annual-Journals_1994-2011, Padova-Admin-Data_1964-2011}. This sample is too small to distinguish in our evaluation.\footnote{Administrative from the Municipality of Parma is not available, nor is historical survey data about state preschools in Parma.} 

We thus focus our discussion on the state preschool program as it existed from the late 1970's through the present. By the late 1970's, state policies were reportedly influenced by municipal policies in the region of Emilia Romagna, including Reggio Emilia, Milan, and Pistoia \citep{OECD_2001_Italy-Country-Note}. For example, state preschools a) prioritized enrollment for children with disabilities; b) staffed each classroom with 2 fully trained teachers, and ; c) finally approved the hiring of male teachers \citep{Hohnerlein_2015_Development-and-DiffusionEnrollment}. In 1980, teacher child-ratio were 2:35; by 1995, teacher-child ratios were reduced to a maximum of 2:25 \citep{Hohnerlein_2009_Paradox-Public-Preschools}. 

In 1991, revised Orientamenti first emphasized social, affective and cognitive development; play, meals, and collaborative skills were defined as the key tasks of early childhood development \citep{Corsaro_1996_Early-Edu}. Further policy revisions in 1997 mandated university degrees and supervised experience for teachers, expanding traditional teacher training from Catholic institutions to secular higher education \citep{Ghedini_2001_Ital-Natl-Policy}. Religious teaching is offered in all state preschools; parents can opt out, however, an alternative educational experiences may not be offered. Teaching methodologies in state preschools include direct instruction as well as play-based learning \citep{CEHD_2016_Historical-Analysis}.

Compared to municipal preschools, teachers in state schools work fewer hours (33/week) than their municipal counterparts who work 36 hours/week. To coordinate legislated maximum working hours and the 8 hour school day, 1 state teacher arrives at 8am, and the other teacher stays until all parents arrive for pick up time. Children in state preschools thus spend more time in a classroom staffed by only 1 teacher than those in municipal schools. Teachers in state preschools further have less weekly time set aside for professional training, documentation, and engagement of parents.

\subsubsection{Parma Municipal Schools}

Detailed documentation of the Parma municipal early childhood system is scarce. Conversation with the authors of \citet{Edwards-etal-eds_1998_Hundred-Languages} suggests that the pedagogical approach of Parma's municipal system is similar to that of Reggio Emilia.\footnote{Kuperman, Interview with Carolyn Pope Edwards, 2016.} One relevant difference for this evaluation is that Parma's municipal system consolidated and expanded around 1975, about a decade after Reggio Emilia. 

As of 2001, Parma provided 16 infant-toddler centers throughout the municipality. Pedagogical coordinators perform both administrative and professional development roles. Assigned to a specific set of infant-toddler centers, these coordinators meet twice each month with all teachers collectively for shared reflection, on-site supervision, and to promote relationships with the families. The city director meets biweekly with all pedagogical coordinators for overall planning. University professors or administrators from other municipalities provide professional development in the form of continuing education \citep{Terzi-Cantarelli_2001_Parma}. Parma's infant-toddler centers are intentionally designed in the context of an apartment. \citet{Terzi-Cantarelli_2001_Parma} report mixed-age classes that include 18 total children from 13 months to 3 years in a single section, led by two teachers (9:1 child-teacher ratio). To accommodate parents, infant-toddler centers open at 7:30pm and offer three pick-up times: 2 p.m. (short-day), 3:30 p.m. (normal-day), or 5 p.m. (extended-day). Classrooms can be organized by single-age groups (e.g., 5-12 months, 12-24 months, and 24-36 months) or by mixed-age groups (e.g.,12-36 months) \citep{Majorano-etal_2009_CC-in-P}.

\subsubsection{Padova Municipal Schools}

Padova is located in the relatively more religious and politically diverse region of Veneto. Compared to Reggio Emilia and Parma, its municipal early childhood education system is smaller, offering 10 preschool centers. In contrast, Padova has a larger number of religious preschool centers. In 1989, the region of Veneto reported a total provision of childcare slots for 3.9\% of its infant-toddler population. In contrast, the region of Emilia Romagna reported a provision of infant-toddler childcare for 15.6\% of its population. The practice of professional development trainings for early childhood staff in Veneto first began in 1986 \citep{Becchi-Ferrari_1990_Pub-Inf-Centres-Italy}.

In Padova, the municipal preschool system began to consolidate in 1973, expanding from two to five sites by 1976. Teacher-child ratios for Padova's municipal preschools ranged from 1:12 to 1:24 in 1976 \citep{Padova-Admin-Data_1964-2011,CEHD_2016_Historical-Analysis}.

\subsubsection{Catholic Schools}

The Catholic Church offers religious early childhood education. The Catholic Church is greatly concerned with equity and parity of state funding for its non-state schools. Unlike municipal systems, the majority of costs for private religious preschool are subsidized by tuition paid by enrolled families. Private schools historically were considered options only for affluent families that could afford the tuition expense \citep{Hohnerlein_2009_Paradox-Public-Preschools}.\footnote{Prior to 2000, state funding for private schools reflected a 1947 constitutional clause that non-state schools could operate ``without financial burdens on the state.'' Today, depending on their income and the particular municipality's subsidy program, families enrolled in religious programs are eligible for subsidized tuition \citep{Hohnerlein_2009_Paradox-Public-Preschools}. In 2005, however, funding for state and private schools differed greatly. While equitable religious preschools could receive state funds, funding for religious primary schools (ages 6-11 years) was at 15.5\% the rate of public primary schools. Private secondary schools (ages 11-18 years) received no state funding \citep{Becchi-Ferrari_1990_Pub-Inf-Centres-Italy}.}

By 1976; teacher-child ratios at Padova's religious schools ranged from 1:34-44. Prior to revised state policies in 1997-2000, Catholic preschools were taught by nuns, teacher training was not publicly mandated, and teacher-child ratios were relatively low. Coincidentally, religious programs began to hire trained, secular teachers who implemented varied curricula. In the north, traditional instruction began to incorporate some active-child learning, and some sites began to provide entry to children at age 24 months.

Until the early 2000s, tuition and fees to families enrolling children in religious preschools in all three cities were relatively more expensive than municipal and state programs. Religious schools in Padova did not receive any form of public funding in the 1970s; families were responsible for 100\% of the costs. In the 1980s and 1990s, the municipality of Padova contributed 20\% and 40\% of program costs to religious schools. In the 2000s, families paid 60\% and the remaining 40\% was shared by the state and municipality of Padova. After the 2000s, religious schools acknowledged by the state for meeting quality components were eligible to receive state funding.\footnote{This information comes from a variety of sources: \citet{Reggio-Admin-data_1966-2006, Reggio-Annual-Journals_1994-2011, Padova-Admin-Data_1964-2011} and results from the survey \citep{CEHD_2016_Historical-Analysis}},





%\subsubsection{Infant-Toddler Programs} eligible age of entry for infant-toddler programs varies from 3 months in Reggio Emilia to 9 months in Padova
