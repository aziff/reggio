
As our study includes individuals who experienced the Reggio Approach as well as other types of early childhood programs, it is important to examine the similarities and differences of the programs in order to analyze the effects of the Reggio Approach. This section documents different types of early childhood systems in Northern Italy and the Reggio Approach, uniquely offered by the Municipality of Reggio Emilia for children ages 3 months to 6 years. Other early childhood education systems within Reggio Emilia, as well as in Parma and Padova, share certain features of program administration, practices for at-risk children and families, and pedagogical methods. 

Those who did not experience the Reggio Approach treatment participated in heterogenous early childhood experiences including home-based care, religious preschool for ages 3 to 6 years, state preschool for ages 3 to 6 years, municipal childcare and preschool education offered by Parma and Padova for ages 0 to 6 years. The availability of preschool programs by city is documented in Table \ref{tab:itc-pre}. We thus sought to better understand the history and evolution of various early childhood options apart from what was available in the literature. We collected historical documents \citep{Padova-Admin-Data_1964-2011,Reggio-Admin-data_1966-2006,Reggio-Annual-Journals_1994-2011} and administered a historical survey to current and retired school administrators and educative coordinators in order to quantify pedagogical and administrative features of childcare experiences available during 1950-2010 \citep{CEHD_2016_Historical-Analysis}.\footnote{See Appendix~\ref{sec:survey} for the full survey.} 

We first describe the Reggio Approach, uniquely offered by the Municipality of Reggio Emilia, which we consider to be the treatment in this evaluation. 

%We begin with a general history and evolution of early childhood policies and programming in Italy to clarify the availability and enrollment of center-based programs in our sample.

%\subsection{Early Childhood Education in Italy}

%Italy's early childhood policies reflect state Laws 444 and 1044, enacted in 1968 and 1971. These policies mark a key shift by formally legitimizing state involvement in the provision and programming of educational preschool (Law 444, for ages 3-6 years) and infant-toddler childcare (Law 1044, for ages 0-3 years). Prior to 1968, childcare was institutionally provided for children of working mothers and orphans within the religious, welfare, and social service spheres \citep{OECD_2001_Italy-Country-Note,Hohnerlein_2015_Development-and-Diffusion}. The Ministry of Education oversees preschools for ages 3-6 years; the Ministry of Health is responsible for regulating the infant-toddler childcare system \citep{Corsaro_1996_Early-Edu}.\footnote{Ongoing reform efforts led by the region of Emilia Romagna seek to unify a system of early childhood education to provide continuity of care in programs for ages 0-6 years \citep{CEHD_2016_Historical-Analysis}.} 

%Accepted literature and administrative records distinguish state from non-state early childhood systems, including municipal, religious, and secular private programs \citep{Padova-Admin-Data_1964-2011,Reggio-Admin-data_1966-2006,Reggio-Annual-Journals_1994-2011,OECD_2001_Italy-Country-Note,Ribolzi_2013_Italy}. State preschools operate according to legislated policies and \textit{Orientamenti}, national guidelines defining program standards and general goals for early childhood education. As non-state programs, municipalities and religious programs are enabled to function autonomously, offering curricula and setting administrative regulations such as eligibility criteria for local preschools and infant-toddler programs. 

%The Catholic Church is the oldest non-state early childhood system in Italy, providing religious training for disadvantaged children since the 19th century \citep{OECD_2001_Italy-Country-Note}. The majority of religious programs enroll children aged 3-6 years \citep{Malizia-Cicatelli_2011_BOOK_Catholic-School}. In the 1960's, educators and left-wing leaders within Emilia Romagna including Bruno Ciari and Loris Malaguzzi incited a `municipal school revolution' in Italy by organizing community programs for families with children aged 0-6 years. Influenced by Dewey's progressive model of early childhood education, these educators designed municipal systems of early childhood education as active-child learning alternatives to then-existing childcare models based on welfare, hygiene, and moral socialization. 

%Pursuant to state, regional, and varying municipal policies, the availability, enrollment, and educational programming offered by state and local non-state preschool systems for the 5 cohorts in our evaluation varies over time and by city. We document the availability of systems for each cohort in Table~\ref{tab:itc-pre}, defining a system to include programs with 4 or more school sites.

\begin{table}[H]
\centering
\caption{Availability of Preschool Programs by City and School Type}\label{tab:itc-pre}
\begin{adjustbox}{width=\textwidth}
\begin{threeparttable}
	\begin{tabular}{l l c c c c c c c c c}
\toprule
\mc{1}{c}{Cohort} & \mc{1}{c}{Years} & \mc{3}{c}{Reggio Emilia} & \mc{3}{c}{Parma} & \mc{3}{c}{Padova} \\
& & Municipal & Catholic & State & Municipal & Catholic & State & Municipal & Catholic & State \\
\midrule
Adults 50s & 1957-1965 & & \checkmark & & & \checkmark & & & \checkmark & \\
Adults 40s & 1972-1976 & \checkmark & \checkmark & & & \checkmark & & & \checkmark & \\
Adults 30s & 1983-1987 & \checkmark & \checkmark & \checkmark & \checkmark & \checkmark & \checkmark & \checkmark & \checkmark & \checkmark \\
Adolescents & 1994-2000 & \checkmark & \checkmark & \checkmark & \checkmark & \checkmark & \checkmark & \checkmark & \checkmark & \checkmark \\
Children & 2009-2014 & \checkmark & \checkmark & \checkmark & \checkmark & \checkmark & \checkmark & \checkmark & \checkmark & \checkmark \\
\bottomrule
\end{tabular}

% Caption:
% Note: This table indicates the types of educational preschool systems (defined as programs with 4 or more sites) available to parents in each city during the years each cohort was eligible for a 3-6 year old program. 
\begin{tablenotes}
Note: This table indicates the majority of educational preschool systems available to parents in each city during the years each cohort was eligible for a 3-6 year old program. 
\end{tablenotes}
\end{threeparttable}
\end{adjustbox}
\end{table}



%This section describes the early childhood treatments experienced by the cohorts in our study. We compare state, religious, and municipal systems, considering similarities and differences in administration and programming. To better understand the evolution of various early childhood options apart from what was available in the literature, we administered a historical survey in Reggio Emilia, Parma, and Padova to quantify pedagogical and administrative features of other available childcare experiences available from 1950-2010 \citep{CEHD_2016_Historical-Analysis}. We also sourced data from municipal archives of Reggio Emilia, Parma, and Padova to document child enrollment and teacher staffing of local preschools; we successfully gathered historical materials from the municipal offices of Reggio Emilia and Padova, and from Reggio Children \citep{Padova-Admin-Data_1964-2011,Reggio-Admin-data_1966-2006,Reggio-Annual-Journals_1994-2011}. Based on our new knowledge of similarities and differences across early childhood systems, we document our expectations for variance in outcomes between the Reggio Approach and alternative treatments.  

\subsection{The Reggio Approach}

Of the three municipal systems we compare, Reggio Emilia is particularly notable for its significant investment in innovative municipal programs and services. It is the earliest municipal system to evolve, and historically offers the largest number of municipal infant-toddler and preschool sites.\footnote{Similar to Parma and Padova, Reggio Emilia contracts with local private providers and cooperatives to offer a number of infant-toddler and preschool slots according to municipal regulations. These ``affiliated'' programs may not follow the Reggio Approach nor the municipal approaches in Parma and Padova. Accordingly, we do not include those who were enrolled in affiliated programs in our evaluation.} While eligible, Reggio Emilia did not receive state funding for its municipal early childhood system until the 1990s and 2000s. Ironically, the municipality contributed funds to its local state schools each decade from the 1970s. In 1994, Reggio Approach staff provided training for religious preschool teachers in Reggio Emilia. 
%SK: Should we add a cite per Daniela del B ``most  the principles of the Reggio Approach have become increasingly part of a large number of childcare experience in Italy and Europe (you can cite Bennet 2012 and Lazzari and Vandenbroek 2012) from our paper''? I think this perspective is supported by the historical survey. Also not sure whether JJH will want to include one of her citations...

The Reggio Approach is a form of progressive early childhood education influenced by Loris Malaguzzi, an educator promoting the educational practices and psychological theories of Dewey, Piaget, Erikson, Vygotsky, Bronfenbrenner, Kagan, and Gardner. In 1963, Reggio Emilia opened its first preschool for children aged 3-6 years. In 1965, municipal policies were enacted to provide funding for infant-toddler centers for children aged 3 months to 3 years; in 1971, the first site opened. Reggio Emilia's municipal early childhood system thus preceded Italy's key legislative reforms that established state-run preschools and mandated the local provision of infant-toddler centers \citep{Cagliari-etal-eds_2016_BOOK_Loris-Malaguzzi}. The Reggio Approach views curriculum as an ongoing, collaborative project without pre-determined learning goals or timelines. There is no institutionally-prescribed content knowledge that educators convey to children for ``school readiness.'' In contrast, teachers and children are viewed as researchers and co-creators of knowledge. For example, educators, children and families collaborate to define a question or topic. Learning is then pursued following a scientific process: theories are shared, tested, and revised through dialogue. 

In Reggio Approach preschools, the educative team is assigned specialized roles. Each incoming class of approximately twenty-five 3-year-olds is assigned two full-time co-teachers (teacher-child ratio of 2:25); at least 1 of the 2 teachers remains with this cohort of homogeneous-aged children for three consecutive years. This extended time provides continuity of care for children and enables strong teacher-family engagement. Each preschool site is further staffed by a full-time atelierista, an instructor with a background in visual arts. Teachers observe children's development, interact with children through questions and dialogue, and provide scaffolding to support learning. Children demonstrate their emerging knowledge through creative learning activities and art, with aid from the atelierista. Teachers document each child's development in a portfolio---a collection of work---which is shared and discussed with children and parents over the year \citep{Rinaldi_2006_ReggioEmilia_BOOK,Giudici-Nicolosi_2014_Reggio-Approach}. Auxiliary site staff, such as cooks and janitors, are considered members of the educative team and participate in trainings and professional development. A pedagogista---or educative coordinator---with a higher degree in psychology or education is assigned to support professional development on a biweekly basis for the educative staff of approximately 4-5 municipal preschools. 

The Reggio Approach school environment reflects a light-filled, open interior design, furnished with natural materials and a garden. Each site is equipped with an atelier, or dedicated studio laboratory used for creative instructional activities. In-house kitchens are surrounded by glass walls, to include children in the meal process, and is used daily for preparing meals. Reggio Approach preschools and infant-toddler centers are open five full-time days per week from September through June \citep{Giudici-Nicolosi_2014_Reggio-Approach}. Extended day options are available at a majority of Reggio municipal sites throughout the school year, as is educational programming throughout July. Children with disabilities and single parents have been prioritized in admission criteria from the early 1970's \citep{Edwards-etal-eds_1998_Hundred-Languages}. The engagement of families is embedded in Reggio practices, as is the invitation to all community members to participate in school management \citep{CEHD_2016_Historical-Analysis,Cagliari-etal-eds_2016_BOOK_Loris-Malaguzzi}. 



\subsection{State Preschool for Ages 3-6 Years}
In 1968, law 444 mandated the provision and funding of free, public preschools \textit{only} where local demand was not already met by existing non-state systems \citep{Hohnerlein_2009_Paradox-Public-Preschools}.\footnote{In state programs, parents pay only for meals, transportation, and extras such as field trips and extracurricular lessons.}\footnote{The state does not offer infant-toddler childcare but regulates and subsidizes local programs through regional governments.}  Historical records indicate that state preschools first appeared in Reggio Emilia and Padova between 1973-1975 \citep{Padova-Admin-Data_1964-2011,Reggio-Admin-data_1966-2006,Reggio-Annual-Journals_1994-2011}. While the state is currently the largest provider of preschool education in Italy, enrollment in state preschools in Reggio Emilia, Parma, and Padova have historically lagged behind municipal and religious programs.

State preschools are administrated with local primary and middle schools, operating as a single comprehensive institution for children aged 3 up to 17 years. Orientamenti, or national guidelines, provide the framework for state programs. Orientamenti are periodically revised to reflect current educational practices and political ideology.\footnote{For example, religious teaching was originally required by law in all state preschools. Subsequent revisions allowed parents to opt their child out of religious teaching, however, alternative educational experiences were not guaranteed \citep{CEHD_2016_Historical-Analysis}.} 

Reports suggest that Orientamenti were historically influenced by municipal programs in the region of Emilia Romagna, including Reggio Emilia, Milan, and Pistoia \citep{OECD_2001_Italy-Country-Note}. In contrast to municipal programs in Reggio Emilia, Parma and Padova, however, state preschools do not offer extended hours to working families; state teachers work shorter hours than their municipal counterparts and by law receive equivalent pay to teachers in primary schools. Improved Orientamenti, along with revised state policies mandating lower teacher-child ratios and higher qualifications for teacher education, are proposed as key quality indicators associated with diminishing disparities in state and non-state programs by the end of the 20th century \citep{Hohnerlein_2015_Development-and-Diffusion}. Below we list key historical revisions to Orientamenti, documenting mandated quality improvements in state preschools in the years that our cohorts were eligible to enroll.

%\begin{itemize}
% \item Revisions to Orientamenti and State Policies 
% \begin{itemize}
% 	\item In 1969, Orientamenti focused on education, development and care \citep{Corsaro_1996_Early-Edu,Hohnerlein_2015_Development-and-DiffusionEnrollment}.
%	\item In 1991, Orientamenti emphasized social, affective and cognitive development, defining play, mealtime behavior, and collaborative skills as the key tasks of early childhood \citep{Corsaro_1996_Early-Edu}. 
%	\item In 1997, revised teacher requirements included university degrees and supervised experience, expanding traditional teacher training from Catholic institutions to secular higher education \citep{Ghedini_2001_Ital-Natl-Policy}. 
%	\item In the early 2000's, infant-toddler services were first recognized as educational in nature, for the development of social, emotional, and cognitive skills.\footnote{We highlight the timing in comparison to the evolution of educational approaches offered by municipal infant-toddler programs in Reggio Emilia (by 1971), in Parma (by mid-late 1970s) and in Padova (by 1990) \citep{CEHD_2016_Historical-Analysis}.} 
% \end{itemize}
% \end{itemize}

The table below reflects the evolution of state mandates for minimum teacher-child ratios in the years our 5 cohorts were eligible for preschool. 

\begin{table}
\begin{center}
\caption{Evolution of Teacher-Child Ratios for children aged 3-6 years}
\label{tab:ratios}
\begin{tabular}{c c}
\toprule
Years & \multicolumn{1}{C{7em}}{Teacher-Child Ratio} \\
\midrule
1955-1956 & 1:36.9 \\
1969-1979 & 1:30.3 \\
1971-1972 & 1:27 \\
1980-1981 & 1:17.3 \\
1995-1996 & 1:13 \\
\bottomrule
\end{tabular}
\end{center}
\small
Source: \citet {Hohnerlein_2015_Development-and-Diffusion}
\end{table}

To summarize, the five cohorts in our evaluation had differential access to state preschools in each city, and those who enrolled in state programs experienced varying early childhood curricula and administrative practices. We hypothesize that the degree to which state preschool experiences vary in our sample is wider between age cohorts than between cities, however, we predict that state preschools in Reggio Emilia are more similar to the Reggio Approach than state preschools in Parma and Padova. 

\subsection{Religious Preschool}

The Catholic Church is the oldest non-state early childhood provider in Italy, offering both religious training and charitable social services for disadvantaged children since the 19th century \citep{OECD_2001_Italy-Country-Note}. While the Church does not offer infant-toddler programming nor a cohesive national system of preschool education, local religious programs began to assemble federations in the mid-1970s. Within federations, local religious sites are enabled to offer independent programs. 

The second half of the 20th century, however, marks a downward trend in enrollment and relative quality of religious early childhood programming throughout Italy.\footnote{Prior to 1968, more than 50\% of Italian children were enrolled in childcare provided mainly by the Church \citep{Hohnerlein_2009_Paradox-Public-Preschools}. Between 1981 and 1998, as the number of municipal and state preschools increased in Italy, enrollment in religious preschools dropped to 42.4\%.} Enrollment trends are reportedly associated with policies allowing non-state schools to operate ``without financial burdens on the state''; religious schools were thus perceived for affluent families that could afford the tuition \citep{Ribolzi_2013_Italy,Hohnerlein_2009_Paradox-Public-Preschools}. Enrollment trends in Reggio Emilia, Parma, and Padova reflect national levels. For the oldest four cohorts in our study, tuition for religious preschools in each of the 3 cities was comparatively more expensive than municipal and state programs. 

Equity in public subsidies was mandated in the late 1990s for non-state programs that met state guidelines. Coincidentally, religious programs began significant efforts to improve and quantify program quality \citep{Malizia-Cicatelli_2011_BOOK_Catholic-School}. For example, religious educators were replaced with secular teachers trained in higher institutions and teacher-child ratios began to reflect state standards. Parents of the youngest cohort, born in 2006, who enrolled in equitable religious programs were eligible for subsidized tuition on a sliding-scale basis, and their children experienced educational programming that reflected an influence by municipal systems in Emilia Romagna, including Reggio Emilia \citep{Hohnerlein_2009_Paradox-Public-Preschools,OECD_2001_Italy-Country-Note}. For example, while religious programs historically did not provide infant-toddler education, by the late 1990's, some sites in each of the three cities began to offer transitional programming for children aged 24 months \citep{Malizia-Cicatelli_2011_BOOK_Catholic-School,CEHD_2016_Historical-Analysis}. 

