This section describes the policies and different types of current early childhood systems in Italy, as well as the Reggio Approach. Because those who did not receive this treatment did not receive a homogenous alternative early childhood education experience, we sought to better understand the history and evolution of various early childhood options apart from what was available in the literature. Our documentation in this section is based on historical documents \citep{Padova-Admin-Data_1964-2011,Reggio-Admin-data_1966-2006,Reggio-Annual-Journals_1994-2011} and a historical survey we administered to current and retired school administrators and educative coordinators in order to quantify pedagogical and administrative features of childcare experiences available during 1950-2010 \citep{CEHD_2016_Historical-Analysis}.\footnote{See Appendix~\ref{sec:survey} for the full survey.}


\subsection{Early Childhood Education in Italy}

Italy's current early childhood policies reflect key state Laws 444 and 1044, passed in 1968 and 1971, mandating the provision and programming of educational preschool and infant-toddler childcare by age.\footnote{Prior to 1968, childcare was institutionally provided within the welfare and religious sector for children of working mothers \citep{OECD_2001_Italy-Country-Note,Hohnerlein_2015_Development-and-Diffusion}.} The Ministry of Education oversees preschools for ages 3-6 years; the Ministry of Health is responsible for regulating the provision of infant-toddler childcare for children up to 3 years old \citep{Corsaro_1996_Early-Edu}.\footnote{Not until the early 2000's did the state formally recognize the role of infant-toddler services as educational in nature for the development of social, emotional, and cognitive skills. Ongoing ECE reform efforts led by the region of Emilia Romagna seek to unify the system of early childhood education to provide continuity of care from ages 0-6 years \citep{CEHD_2016_Historical-Analysis}.} By 1990, over 96\% of Italian children aged 3-6 years attended preschool \citep{Hohnerlein_2015_Development-and-Diffusion}. In contrast, provision and attendance for infant-toddler services continues to vary considerably across Italy; some northern municipalities provide infant-toddler programs for 30-45\% of children under 3 years, compared to to 2.3\% in the South \citep{Becchi-Ferrari_1990_Pub-Inf-Centres-Italy,Musatti-Picchio_2010_IJEC}. 


Accepted literature and administrative records distinguish state and non-state early childhood systems, which include municipal, religious, and secular private programs \citep{Padova-Admin-Data_1964-2011,Reggio-Admin-data_1966-2006,Reggio-Annual-Journals_1994-2011,OECD_2001_Italy-Country-Note,Ribolzi_2013_Italy}. Prior to 1968, about 50\% of children aged 3-6 years in Italy were enrolled in childcare provided by non-state institutions \citep{Hohnerlein_2009_Paradox-Public-Preschools}. As the state preschool system expanded, enrollment in religious programs declined from more than 50\% in the 1950s to less than 20\% by 2000. In cities that established high quality municipal programs, fewer children attend state preschool \citep{OECD_2001_Italy-Country-Note}. 

Thus, pursuant to state, regional, and municipal policies, the availability and enrollment in local state and non-state preschool systems for the 5 cohorts in our evaluation varies over time and by city. We summarize this in Table~\ref{tab:itc-pre}. 

\begin{table}[H]
\centering
\caption{Availability of Infant-toddler Centers and Preschools, by City and School Type}\label{tab:itc-pre}
\begin{adjustbox}{width=\textwidth}
\begin{threeparttable}
	\begin{tabular}{l l c c c c c c c c c}
\toprule
\mc{1}{c}{Cohort} & \mc{1}{c}{Years} & \mc{3}{c}{Reggio Emilia} & \mc{3}{c}{Parma} & \mc{3}{c}{Padova} \\
& & Municipal & Catholic & State & Municipal & Catholic & State & Municipal & Catholic & State \\
\midrule
Adults 50s & 1957-1965 & & \checkmark & & & \checkmark & & & \checkmark & \\
Adults 40s & 1972-1976 & \checkmark & \checkmark & & & \checkmark & & & \checkmark & \\
Adults 30s & 1983-1987 & \checkmark & \checkmark & \checkmark & \checkmark & \checkmark & \checkmark & \checkmark & \checkmark & \checkmark \\
Adolescents & 1994-2000 & \checkmark & \checkmark & \checkmark & \checkmark & \checkmark & \checkmark & \checkmark & \checkmark & \checkmark \\
Children & 2009-2014 & \checkmark & \checkmark & \checkmark & \checkmark & \checkmark & \checkmark & \checkmark & \checkmark & \checkmark \\
\bottomrule
\end{tabular}

% Caption:
% Note: This table indicates the types of educational preschool systems (defined as programs with 4 or more sites) available to parents in each city during the years each cohort was eligible for a 3-6 year old program. 
\begin{tablenotes}
Note: This table indicates the main types of educational preschool systems (defined as programs with 4 or more sites) available to parents in each city during the years each cohort was eligible for a 3-6 year old program. 
\end{tablenotes}
\end{threeparttable}
\end{adjustbox}
\end{table}

\subsection{State and Non-State Early Childhood Systems}

Infant-toddler childcare and preschool education are not uniformly regulated, funded, and managed by state and non-state systems. While the state is currently the largest provider of preschool education for ages 3-6 years, it does not provide infant-toddler programs. Further, state preschools are funded only where local demand for preschool education is not already met by existing non-state systems \citep{Hohnerlein_2009_Paradox-Public-Preschools}. State preschools charge only for meals and request voluntary contributions from families for extra programming such as field trips \citep{CEHD_2016_Historical-Analysis}. 

Non-state early childhood systems are largely enabled to function autonomously. Access to public funding, however, for non-state preschool and infant-toddler programs varies, as does tuition to families. After 2000, state policies were revised to acknowledge non-state schools that complied with eight regulations. This policy redefined the concept of private early childhood systems, as `equitable' programs now claim higher quality and enjoy increased access to public funds \citep{Hohnerlein_2009_Paradox-Public-Preschools,Ribolzi_2013_Italy}. 

Municipalities organize and fund community early childhood systems for families with children aged 0-6 years. To comply with key state laws, municipalities must also provide enough infant-toddler ``childcare slots'' to meet local demand \citep{Saraceno_1984_Soc-Probs}. Some municipalities thus contract with private providers to offer a number of infant-toddler and preschool slots according to municipal regulations of eligibility and fees. In general, children of working mothers receive priority enrollment by municipal systems \citep{Saraceno_1984_Soc-Probs}.\footnote{Age of entry into infant-toddler programs varies across municipalities \citep{CEHD_2016_Historical-Analysis}. And while selection criteria are similar across municipalities, the weighting of distinct family characteristics varies \citep{Del-Boca-etal_2016_CESifo-ES}.} Some municipalities provide free preschool while others are offered on a sliding scale. Parental fees for municipal infant-toddler childcare are much higher, covering 21\% of total program costs on average \citep{Musatti-Picchio_2010_IJEC}. 

The Catholic Church is the oldest non-state early childhood system in Italy, offering both religious training and charitable social services for disadvantaged children since the 19th century \citep{OECD_2001_Italy-Country-Note}. The majority of religious programs enroll children ages 3-6 years. Religious institutions do not currently provide infant-toddler education, however, some religious sites offer programming for children beginning at 24 months \citep{Malizia-Cicatelli_2011_BOOK_Catholic-School}. Tuition for religious preschool currently depends on family income. Historically, religious education was considered ``schools for the rich'' as tuition was purely the family's responsibility \citep{Ribolzi_2013_Italy}. 

To summarize, early childhood education in Italy is publicly provided by the state, by the municipality, by religious institutions, and by secular private organizations. Section~\ref{sec:data} describes the selection of our sample into these different school types. 

\subsection{The Reggio Approach}

The Reggio Approach is a form of progressive early childhood education designed by Loris Malaguzzi, an educator influenced by the educational practices and psychological theories of Dewey, Piaget, Erikson, Vygotsky, Bronfenbrenner, Kagan, and Gardner. Malaguzzi, along with Bruno Ciari in Bologna, was one of several left-wing educators within the region of Emilia Romagna who are credited with inciting a `municipal school revolution' in the 1960's. Under the guidance of Malaguzzi, Reggio Emilia opened its first preschool in 1963 for children aged 3-6 years. In 1965, Reggio Emilia passed laws regulating funding for the first infant-toddler center for children aged 3 months to 3 years which later opened in 1971. The Reggio Emilia municipal early childhood system thus preceded Italy's legislative reforms in the 1960s and 1970s that established state-run preschools and mandated the local provision of infant-toddler centers \citep{Cagliari-etal-eds_2016_BOOK_Loris-Malaguzzi}. 

In Reggio Approach preschools, the educative team is assigned specialized roles. Each incoming class of approximately twenty-five 3-year-olds is assigned two full-time co-teachers; at least 1 of the 2 teachers remains with this cohort of homogeneous-aged children for three consecutive years. This extended time provides continuity of care for children and enables strong teacher-family engagement. Each school site is further staffed by a full-time atelierista, an instructor with a background in visual arts. Auxiliary site staff, such as cooks and janitors, are considered members of the educative team and participate in trainings and professional development. A pedagogista---or educative coordinator---with a higher degree in psychology or education is assigned to support professional development on a biweekly basis for the educative staff of approximately 4-5 municipal preschools. 

The Reggio Approach school environment reflects a light-filled, open interior design, furnished with natural materials and a garden. Each site is equipped with an atelier, or dedicated studio laboratory used for creative instructional activities. In-house kitchens are surrounded by glass walls, to include children in the meal process, and is used daily for preparing meals. The Reggio Approach is notable for viewing curriculum as an ongoing, collaborative project without pre-determined learning goals or timelines. There is no institutionally-prescribed content knowledge that educators convey to children for ``school readiness.'' In contrast, teachers and children are viewed as researchers and co-creators of knowledge. For example, educators, children and families collaborate to define a question or topic. Learning is then pursued following a scientific process: theories are shared, tested, and revised through dialogue. Teachers observe children's development, interact with children through questions and dialogue, and provide scaffolding to support learning. Children demonstrate their emerging knowledge through creative learning activities and art, with aid from the atelierista. Teachers document each child's development in a portfolio---a collection of work---which is shared and discussed with children and parents over the year \citep{Rinaldi_2006_ReggioEmilia_BOOK,Giudici-Nicolosi_2014_Reggio-Approach}. 

Reggio Approach preschools and infant-toddler centers are open five full-time days per week from September through June \citep{Giudici-Nicolosi_2014_Reggio-Approach}. Extended day options are available at a majority of Reggio municipal sites throughout the school year, as is educational programming throughout July. Children with disabilities and single parents have been prioritized in admission criteria from the early 1970's \citep{Edwards-etal-eds_1998_Hundred-Languages}. The engagement of families is embedded in Reggio practices, as is the invitation to all community members to participate in school management \citep{CEHD_2016_Historical-Analysis,Cagliari-etal-eds_2016_BOOK_Loris-Malaguzzi}. 

Although eligible, the municipality of Reggio Emilia did not receive state funding for its municipal system until the 1990s and 2000s. Ironically, the municipality of Reggio Emilia contributed funds to its state schools each decade since the 1970s. In 1994, the Municipality also provided training for religious preschool teachers in Reggio Emilia.  The teacher-child ratio for each classroom has been 2:25 since the 1960s; this ratio does not include the atelierista present at each school site, the kitchen and auxiliary staff, nor the pedagogista who supervises the educative staff of 4-5 schools. 

\subsection{Comparison of Early Childhood Systems in Reggio Emilia, Parma, and Padova}

Early childhood education systems in Reggio Emilia, Parma, and Padova share certain features of program administration, practices for at-risk children and families, and pedagogical methods. 

We describe the alternative early childhood treatments below using published literature, results from our historical survey, and administrative data retrieved from municipal archives of Reggio Emilia and Padova, and from Reggio Children \citet{Reggio-Admin-data_1966-2006, Reggio-Annual-Journals_1994-2011, Padova-Admin-Data_1964-2011,CEHD_2016_Historical-Analysis}. We first present the historical survey.

\subsubsection{Results from Historical Survey}

Below, we compare alternative early childhood treatments by considering surveyed reports of administrative operations of early childhood system from 1960 through 2010 against systematic features of Reggio Emilia's municipal program.  

\begin{table}[H]
\caption{Comparison of Program Operations}
\scriptsize
\centering
\begin{tabular}{L{4.5cm} C{0.8cm}  C{1.5cm}  C{1.5cm}  C{1.5cm}  C{1.5cm}  C{1.5cm}  C{1.5cm}}
\toprule
Program Operations	&		&	Reggio Municipal	&	Reggio State	&	Parma Municipal	&	Padova Municipal	&	Padova State	&	Padova Religious	\\	\midrule
\multirow{5}{4.5cm}{Schools receive funding from public sources}	&	1960	&	\checkmark	&		&		&		&		&		\\	
	&	1970	&	\checkmark	&	\checkmark	&		&	\checkmark	&	\checkmark	&		\\	
	&	1980	&	\checkmark	&	\checkmark	&		&	\checkmark	&	\checkmark	&		\\	
	&	1990	&	\checkmark	&	\checkmark	&	\checkmark	&	\checkmark	&	\checkmark	&	\checkmark	\\	
	&	2000	&	\checkmark	&	\checkmark	&	\checkmark	&	\checkmark	&	\checkmark	&	\checkmark	\\	\midrule
\multirow{5}{4.5cm}{Full-time Pedagogistas\footnote{A Pedagogista is a highly trained specialist in early childhood education. In some early childhood systems, this role is referred to as an Educative Coordinator; the training and responsibilities of Educative Coordinators vary across cities and ECE systems.} are hired by the system to oversee professional development for multiple program sites}	&	1960	&	\checkmark	&		&		&		&		&		\\	
	&	1970	&	\checkmark	&		&		&		&		&		\\	
	&	1980	&	\checkmark	&		&		&	\checkmark	&		&		\\	
	&	1990	&	\checkmark	&		&	\checkmark	&	\checkmark	&	\checkmark	&		\\	
	&	2000	&	\checkmark	&		&	\checkmark	&	\checkmark	&	\checkmark	&		\\	\midrule
\multirow{5}{4.5cm}{Professional development is provided by highly trained specialists to each program site every 1-2 weeks\footnote{In Padova's religious programs, professional development is provided by a mix of part-time and full-time employees.}}	&	1960	&	\checkmark	&		&		&		&		&		\\	
	&	1970	&	\checkmark	&		&		&		&		&	\checkmark	\\	
	&	1980	&	\checkmark	&		&		&		&		&	\checkmark	\\	
	&	1990	&	\checkmark	&		&	\checkmark	&		&	\checkmark	&	\checkmark	\\	
	&	2000	&	\checkmark	&		&	\checkmark	&		&	\checkmark	&	\checkmark	\\	\midrule
\multirow{5}{4.5cm}{Kitchen and janitorial staff join educators for professional development}	&	1960	&	\checkmark	&		&		&		&		&		\\	
	&	1970	&	\checkmark	&		&	\checkmark	&		&		&		\\	
	&	1980	&	\checkmark	&		&	\checkmark	&		&		&		\\	
	&	1990	&	\checkmark	&		&	\checkmark	&		&		&		\\	
	&	2000	&	\checkmark	&		&	\checkmark	&	\checkmark	&		&		\\	\midrule
\multirow{5}{4.5cm}{Classrooms are homogeneous in age} 	&	1960	&	\checkmark	&		&		&		&		&		\\	
	&	1970	&	\checkmark	&	\checkmark	&		&		&		&		\\	
	&	1980	&	\checkmark	&	\checkmark	&		&		&		&		\\	
	&	1990	&	\checkmark	&	\checkmark	&		&		&		&		\\	
	&	2000	&	\checkmark	&	\checkmark	&		&		&		&		\\	\midrule
\multirow{5}{4.5cm}{2 co-teachers are assigned to each incoming cohort of 3 year olds. At least 1 teacher stays with the cohort for the next two years to maintain continuity of care}	&	1960	&		&		&		&		&		&		\\	
	&	1970	&	\checkmark	&	\checkmark	&		&		&		&		\\	
	&	1980	&	\checkmark	&	\checkmark	&	\checkmark	&	\checkmark	&		&		\\	
	&	1990	&	\checkmark	&	\checkmark	&	\checkmark	&	\checkmark	&		&		\\	
	&	2000	&	\checkmark	&	\checkmark	&	\checkmark	&		&		&		\\	\midrule
\multirow{5}{4.5cm}{Full-time Atelierista\footnote{An Atelierista is an expert in the creative arts who designs creative learning activities and supports children's learning.} is staffed at each preschool site and collaborates with classroom teachers to design creative learning activities}
	&	1960	&	\checkmark	&		&		&		&		&		\\	
	&	1970	&	\checkmark	&		&		&		&		&		\\	
	&	1980	&	\checkmark	&		&		&	\checkmark	&		&		\\	
	&	1990	&	\checkmark	&		&		&	\checkmark	&		&		\\	
	&	2000	&	\checkmark	&		&		&	\checkmark	&		&		\\	\midrule
\multirow{5}{4.5cm}{Scheduled work hours are set aside weekly for teachers to document children's work}	&	1960	&	\checkmark	&		&		&		&		&		\\	
	&	1970	&	\checkmark	&		&		&	\checkmark	&		&		\\	
	&	1980	&	\checkmark	&		&	\checkmark	&	\checkmark	&		&	\checkmark	\\	
	&	1990	&	\checkmark	&		&	\checkmark	&	\checkmark	&		&	\checkmark	\\	
	&	2000	&	\checkmark	&		&	\checkmark	&	\checkmark	&		&	\checkmark	\\	\midrule
\multirow{5}{4.5cm}{Scheduled hours are set aside weekly for teachers to engage families}	&	1960	&	\checkmark	&		&		&		&		&		\\	
	&	1970	&	\checkmark	&		&	\checkmark	&		&		&	\checkmark	\\	
	&	1980	&	\checkmark	&		&	\checkmark	&		&		&	\checkmark	\\	
	&	1990	&	\checkmark	&		&	\checkmark	&	\checkmark	&		&	\checkmark	\\	
	&	2000	&	\checkmark	&		&	\checkmark	&	\checkmark	&		&	\checkmark	\\	\midrule
\multirow{5}{4.5cm}{Parental boards or advisory groups are encouraged as active participants in school culture}	&	1960	&	\checkmark	&		&		&		&		&		\\	
	&	1970	&	\checkmark	&	\checkmark	&	\checkmark	&	\checkmark	&		&	\checkmark	\\	
	&	1980	&	\checkmark	&	\checkmark	&	\checkmark	&	\checkmark	&		&	\checkmark	\\	
	&	1990	&	\checkmark	&	\checkmark	&	\checkmark	&	\checkmark	&		&	\checkmark	\\	
	&	2000	&	\checkmark	&	\checkmark	&	\checkmark	&	\checkmark	&		&	\checkmark	\\	\bottomrule

\end{tabular}
\end{table} 

The hours of center-based care is largely shared by the surveyed programs. All systems (except the state preschools in Padova) offered additional hours for working families. Similarly, by the 1990s, all surveyed systems received public funding. Municipal schools in the three cities shared priorities of enrolling economically disadvantaged children, children from a single-parent household, and children with disabilities.

\begin{table}[H] \caption{Comparison of Administrative Practices for At-Risk Children and Families}															
\scriptsize																	
\centering																	
\begin{tabular}{L{4.5cm} C{0.8cm}  C{1.5cm}  C{1.5cm}  C{1.5cm}  C{1.5cm}  C{1.5cm}  C{1.5cm}}															
\toprule																	
Administrative Practices for At-Risk Children and Families		&		&	Reggio Municipal	&	Reggio State	&	Parma Municipal	&	Padova Municipal	&	Padova State	&	Padova Religious	\\	\midrule
\multirow{5}{4.5cm}{Preschools are open 8 hours daily} 	&	1960	&	\checkmark	&		&	\checkmark	&		&		&		\\	
		&	1970	&	\checkmark	&	\checkmark	&	\checkmark	&	\checkmark	&	\checkmark	&	\checkmark	\\	
		&	1980	&	\checkmark	&	\checkmark	&	\checkmark	&	\checkmark	&	\checkmark	&	\checkmark	\\	
		&	1990	&	\checkmark	&	\checkmark	&	\checkmark	&	\checkmark	&	\checkmark	&	\checkmark	\\	
		&	2000	&	\checkmark	&	\checkmark	&	\checkmark	&	\checkmark	&	\checkmark	&	\checkmark	\\	\midrule
\multirow{5}{4.5cm}{Program sites offer extended hours for working families}	&	1960	&	\checkmark	&		&	\checkmark	&		&		&		\\	
		&	1970	&	\checkmark	&	\checkmark	&	\checkmark	&	\checkmark	&		&	\checkmark	\\	
		&	1980	&	\checkmark	&	\checkmark	&	\checkmark	&	\checkmark	&		&	\checkmark	\\	
		&	1990	&	\checkmark	&	\checkmark	&	\checkmark	&	\checkmark	&		&	\checkmark	\\	
		&	2000	&	\checkmark	&	\checkmark	&	\checkmark	&	\checkmark	&		&	\checkmark	\\	\midrule
\multirow{5}{4.5cm}{Priority of enrollment is given to economically disadvantaged families}	&	1960	&	\checkmark	&		&	\checkmark	&		&		&		\\	
		&	1970	&	\checkmark	&	\checkmark	&	\checkmark	&	\checkmark	&		&		\\	
		&	1980	&	\checkmark	&	\checkmark	&	\checkmark	&	\checkmark	&		&		\\	
		&	1990	&	\checkmark	&	\checkmark	&	\checkmark	&	\checkmark	&		&		\\	
		&	2000	&	\checkmark	&	\checkmark	&	\checkmark	&	\checkmark	&		&		\\	\midrule
\multirow{5}{4.5cm}{Priority of enrollment is given to children with disabilities}	&	1960	&	\checkmark	&		&		&		&		&		\\	
		&	1970	&	\checkmark	&	\checkmark	&		&	\checkmark	&	\checkmark	&		\\	
		&	1980	&	\checkmark	&	\checkmark	&		&	\checkmark	&	\checkmark	&		\\	
		&	1990	&	\checkmark	&	\checkmark	&	\checkmark	&	\checkmark	&	\checkmark	&	\checkmark	\\	
		&	2000	&	\checkmark	&	\checkmark	&	\checkmark	&	\checkmark	&	\checkmark	&	\checkmark	\\	\midrule
\multirow{5}{4.5cm}{Priority of enrollment is given to single-parent families}	&	1960	&	\checkmark	&		&		&		&		&		\\	
		&	1970	&	\checkmark	&		&		&	\checkmark	&		&		\\	
		&	1980	&	\checkmark	&		&	\checkmark	&	\checkmark	&		&		\\	
		&	1990	&	\checkmark	&		&	\checkmark	&	\checkmark	&	\checkmark	&		\\	
		&	2000	&	\checkmark	&		&	\checkmark	&	\checkmark	&	\checkmark	&		\\	\bottomrule
\end{tabular}																	
\end{table}											

Below, we compare alternative early childhood treatments by considering surveyed reports of educational programming in each early childhood system from 1960 through 2010 against systematic features of Reggio Emilia's municipal program.  

\begin{table}[H]
\caption{Comparison of Educational Programming}																	
\scriptsize																	
\centering																	
\begin{tabular}{L{4.5cm} C{0.8cm}  C{1.5cm}  C{1.5cm}  C{1.5cm}  C{1.5cm}  C{1.5cm}  C{1.5cm}}															
\toprule																	
Pedagogical Components 	&		&	Reggio Municipal	&	Reggio State	&	Parma Municipal	&	Padova Municipal	&	Padova State	&	Padova Religious	\\	\midrule
\multirow{5}{4.5cm}{Curriculum emerges through research-based projects with unlimited timelines}	&	1960	&	\checkmark	&		&		&		&		&		\\	
		&	1970	&	\checkmark	&		&		&		&		&		\\	
		&	1980	&	\checkmark	&		&		&	\checkmark	&		&		\\	
		&	1990	&	\checkmark	&		&	\checkmark	&	\checkmark	&		&		\\	
		&	2000	&	\checkmark	&		&	\checkmark	&	\checkmark	&		&		\\	\midrule
\multirow{5}{4.5cm}{Visual arts help children learn}	&	1960	&	\checkmark	&		&	\checkmark	&		&		&	\checkmark	\\	
		&	1970	&	\checkmark	&		&	\checkmark	&		&		&	\checkmark	\\	
		&	1980	&	\checkmark	&		&	\checkmark	&		&		&	\checkmark	\\	
		&	1990	&	\checkmark	&		&	\checkmark	&		&		&	\checkmark	\\	
		&	2000	&	\checkmark	&		&	\checkmark	&		&		&	\checkmark	\\	\midrule
\multirow{5}{4.5cm}{Teachers document children's learning}	&	1960	&	\checkmark	&		&		&		&		&		\\	
		&	1970	&	\checkmark	&		&		&	\checkmark	&		&	\checkmark	\\	
		&	1980	&	\checkmark	&		&	\checkmark	&	\checkmark	&		&	\checkmark	\\	
		&	1990	&	\checkmark	&		&	\checkmark	&	\checkmark	&		&	\checkmark	\\	
		&	2000	&	\checkmark	&		&	\checkmark	&	\checkmark	&	\checkmark	&	\checkmark	\\	\midrule
\multirow{5}{4.5cm}{Educational practices promoted by Loris Malaguzzi for early childhood influenced the daily program}	&	1960	&	\checkmark	&		&		&		&		&		\\	
		&	1970	&	\checkmark	&		&		&		&		&		\\	
		&	1980	&	\checkmark	&		&		&		&		&		\\	
		&	1990	&	\checkmark	&		&		&		&		&		\\	
		&	2000	&	\checkmark	&		&		&		&		&	\checkmark	\\	\midrule
\multirow{5}{4.5cm}{Academic theories of psychology and early childhood education (e.g. Bloom, Bowlby, Bronfenbrenner, Bruner, Gardner, Piaget, Vygotsky) influenced educational approaches}	&	1960	&	\checkmark	&		&		&		&		&		\\	
		&	1970	&	\checkmark	&		&		&		&		&		\\	
		&	1980	&	\checkmark	&		&	\checkmark	&	\checkmark	&		&	\checkmark	\\	
		&	1990	&	\checkmark	&		&	\checkmark	&	\checkmark	&		&	\checkmark	\\	
		&	2000	&	\checkmark	&		&	\checkmark	&	\checkmark	&		&	\checkmark	\\	\midrule
\multirow{5}{4.5cm}{Early childhood methodologies endorsed by Agazzi, Froebl, and/or Montessori influenced the daily program}	&	1960	&		&		&	\checkmark	&		&		&		\\	
		&	1970	&		&	\checkmark	&	\checkmark	&		&		&	\checkmark	\\	
		&	1980	&		&	\checkmark	&		&		&		&	\checkmark	\\	
		&	1990	&		&	\checkmark	&		&		&		&	\checkmark	\\	
		&	2000	&		&	\checkmark	&		&		&		&	\checkmark	\\	\midrule
\multirow{5}{4.5cm}{Daily activities are implemented by following a program, to guid children in acquiring knowledge of specific concepts} 	&	1960	&		&		&	\checkmark	&		&		&		\\	
		&	1970	&		&	\checkmark	&	\checkmark	&	\checkmark	&		&	\checkmark	\\	
		&	1980	&		&	\checkmark	&	\checkmark	&	\checkmark	&	\checkmark	&	\checkmark	\\	
		&	1990	&		&	\checkmark	&	\checkmark	&	\checkmark	&	\checkmark	&	\checkmark	\\	
		&	2000	&		&	\checkmark	&	\checkmark	&	\checkmark	&	\checkmark	&	\checkmark	\\	\midrule
\multirow{5}{4.5cm}{The educational program is designed to promote morality, patriotism, and customs of family life}	&	1960	&		&		&	\checkmark	&		&		&	\checkmark	\\	
		&	1970	&		&	\checkmark	&	\checkmark	&		&	\checkmark	&	\checkmark	\\	
		&	1980	&		&	\checkmark	&		&		&	\checkmark	&	\checkmark	\\	
		&	1990	&		&	\checkmark	&		&		&	\checkmark	&	\checkmark	\\	
		&	2000	&		&	\checkmark	&		&		&	\checkmark	&	\checkmark	\\	\midrule
\multirow{5}{4.5cm}{Religious teaching is provided}	&	1960	&		&		&		&		&		&	\checkmark	\\	
		&	1970	&		&	\checkmark	&		&	\checkmark	&	\checkmark	&	\checkmark	\\	
		&	1980	&		&	\checkmark	&	\checkmark	&	\checkmark	&	\checkmark	&	\checkmark	\\	
		&	1990	&		&	\checkmark	&	\checkmark	&	\checkmark	&		&	\checkmark	\\	
		&	2000	&		&	\checkmark	&	\checkmark	&	\checkmark	&		&	\checkmark	\\	\bottomrule
\end{tabular}																	
\end{table}																	

Below, we compare alternative early childhood treatments by considering surveyed reports of environmental components of early childhood system from 1960 through 2010 relative to systematic features of Reggio Emilia's municipal program.  

\begin{table}[H]
\caption{Comparison of Environmental Features}																	
\scriptsize																	
\centering																	
\begin{tabular}{L{4.5cm} C{0.8cm}  C{1.5cm}  C{1.5cm}  C{1.5cm}  C{1.5cm}  C{1.5cm}  C{1.5cm}}															
\toprule																	
	School Environment	&		&	Reggio Municipal	&	Reggio State	&	Parma Municipal	&	Padova Municipal	&	Padova State	&	Padova Religious	\\	\midrule
\multirow{5}{4.5cm}{Each preschool site includes an Atelier (or dedicated room) where children from different classrooms work individually or in small groups}	&	1960	&	\checkmark	&		&		&		&		&		\\	
		&	1970	&	\checkmark	&		&	\checkmark	&		&		&		\\	
		&	1980	&	\checkmark	&		&	\checkmark	&	\checkmark	&		&	\checkmark	\\	
		&	1990	&	\checkmark	&		&	\checkmark	&	\checkmark	&		&	\checkmark	\\	
		&	2000	&	\checkmark	&		&	\checkmark	&	\checkmark	&		&	\checkmark	\\	\midrule
\multirow{5}{4.5cm}{Open spaces, natural lighting and the use of natural furnishings are emphasized}	&	1960	&	\checkmark	&		&	\checkmark	&		&		&		\\	
		&	1970	&	\checkmark	&		&	\checkmark	&		&		&		\\	
		&	1980	&	\checkmark	&		&	\checkmark	&	\checkmark	&		&		\\	
		&	1990	&	\checkmark	&		&	\checkmark	&	\checkmark	&		&		\\	
		&	2000	&	\checkmark	&		&	\checkmark	&	\checkmark	&		&		\\	\bottomrule
\end{tabular}																	
\end{table}																	

%Sylvi will expand the following sections
\subsection{State Preschools}

State preschools operate according to legislated policies and \textit{Orientamenti}, or national guidelines defining program standards and general goals for early childhood education. Historically, policies were not consistently enforced throughout Italy, nor were Orientamenti considered binding. They are further revised periodically to reflect current educational practices and political ideology. The first Orientamenti for a system of free state preschools focused on education, development and care were published in 1969 \citep{Corsaro_1996_Early-Edu,Hohnerlein_2015_Development-and-Diffusion}.

The cohorts in our sample thus had differential access to state preschools, and those that attended participated in varying early childhood curricula and administrative practices. While the modern state system was legislated in 1968, historical survey data and administrative records suggest that state preschools did not begin to appear until approximately 1973-1975 in both Reggio Emilia and Padova. The Age 40 cohort, however, had access to 3 or fewer state preschools in each city \citep{Reggio-Admin-data_1966-2006, Reggio-Annual-Journals_1994-2011, Padova-Admin-Data_1964-2011}. This sample is too small to distinguish in our evaluation.\footnote{Administrative from the Municipality of Parma is not available, nor is historical survey data about state preschools in Parma.} 

We thus focus our discussion on the state preschool program as it existed from the late 1970's through the present. By the late 1970's, state policies were reportedly influenced by municipal policies in the region of Emilia Romagna, including Reggio Emilia, Milan, and Pistoia \citep{OECD_2001_Italy-Country-Note}. For example, state preschools a) prioritized enrollment for children with disabilities; b) staffed each classroom with 2 fully trained teachers, and ; c) finally approved the hiring of male teachers \citep{Hohnerlein_2015_Development-and-Diffusion}. In 1980, teacher child-ratio were 2:35; by 1995, teacher-child ratios were reduced to a maximum of 2:25 \citep{Hohnerlein_2009_Paradox-Public-Preschools}. 

In 1991, revised Orientamenti first emphasized social, affective and cognitive development; play, meals, and collaborative skills were defined as the key tasks of early childhood development \citep{Corsaro_1996_Early-Edu}. Further policy revisions in 1997 mandated university degrees and supervised experience for teachers, expanding traditional teacher training from Catholic institutions to secular higher education \citep{Ghedini_2001_Ital-Natl-Policy}. Religious teaching is offered in all state preschools; parents can opt out, however, an alternative educational experiences may not be offered. Teaching methodologies in state preschools include direct instruction as well as play-based learning \citep{CEHD_2016_Historical-Analysis}.

Teachers in state schools work fewer hours/week than their municipal counterparts, at 33 hours/week vs. 36 hours/week. To coordinate legislated maximum working hours and the 8 hour school day, 1 state teacher arrives at 8am, and the other teacher stays until all parents arrive for pick up at the end of the day. Children in state preschools thus spend more time in a classroom staffed by only 1 teacher than children in municipal schools. Teachers in state preschools further have less weekly time set aside for professional training, documentation, and engagement of parents.

\subsection{The Municipality of Parma}

Parma's municipal early childhood system consolidated and expanded around 1975, about a decade after Reggio Emilia. Detailed documentation of Parma's municipal preschools is limited; Conversation with experts familiar with the region suggest that the pedagogical approach of Parma's municipal system is similar to that of Reggio Emilia.\footnote{Kuperman, Interview with Carolyn Pope Edwards, 2016.} 

Indeed, in 2001,\citet{Terzi-Cantarelli_2001_Parma} offer descriptions of Parma's infant-toddler programming that appear similar to that of Reggio Emilia. For example, Parma offers 16 municipal infant-toddler centers. Educative coordinators perform both administrative and professional development roles similar to Pedagogistas in Reggio Emilia. Assigned to a specific set of infant-toddler centers, educative coordinators meet twice each month with all teachers collectively for shared reflection, on-site supervision, and to promote relationships with the families. The city director meets biweekly with all pedagogical coordinators for overall planning. University professors or administrators from other municipalities provide professional development in the form of continuing education \citep{Terzi-Cantarelli_2001_Parma}. Parma's infant-toddler centers are intentionally designed in the context of an apartment. Mixed-age classes include 18 total children from 13 months to 3 years in a single section, led by two teachers  for a 1:9 teacher-child ratio. Classrooms are also organized by single-age groups (e.g., 5-12 months, 12-24 months, and 24-36 months) or by mixed-age groups (e.g.,12-36 months) \citep{Majorano-etal_2009_CC-in-P}.To accommodate parents, extended hours are available at infant-toddler centers as are three pick-up times: 2 p.m. (short-day), 3:30 p.m. (normal-day), or 5 p.m. (extended-day). 

\subsection{The Municipality of Padova}

Padova is located in the relatively more religious and politically diverse region of Veneto. Compared to Reggio Emilia and Parma, its municipal early childhood education system is smaller, offering 10 preschool centers. In contrast, Padova has a larger number of religious preschool centers. In 1976, teacher-child ratios for Padova's municipal preschools ranged from 1:12 to 1:24 \citep{Padova-Admin-Data_1964-2011,CEHD_2016_Historical-Analysis}. In this same era, teacher-child ratios at Padova's religious schools ranged from 1:34-44.

Padova's municipal preschool system began to consolidate in 1973, expanding from two to five sites by 1976. Municipal preschools are free, and families pay only for meals. In 1989, the region of Veneto reported a total provision of childcare slots for 3.9\% of its infant-toddler population. In contrast, the region of Emilia Romagna reported a provision of infant-toddler childcare for 15.6\% of its population. The practice of professional development trainings for early childhood staff in Veneto first began in 1986 \citep{Becchi-Ferrari_1990_Pub-Inf-Centres-Italy}.

State preschools in Padova are also free, however, families make an additional voluntary contribution, for example, to accommodate expenses associated with field trips.\footnote{This information is further supported by an interview with Dr. Emilia Restiglian of University of Padua.} By 1976, there were three state preschools in Padova; enrollment was relatively lower and teacher-child ratio approximately 1:15.

\subsection{Catholic Schools}

Today, the Catholic Church is greatly concerned with equity and parity of state funding for its non-state schools \citep{Malizia-Cicatelli_2011_BOOK_Catholic-School}. Prior to 2000, state funding for private schools reflected a 1947 constitutional clause that non-state schools could operate ``without financial burdens on the state.''  \citep{Hohnerlein_2009_Paradox-Public-Preschools}. When equity of funding was publicly mandated in the late 1990s for programs meeting state guidelines, religious programs began significant efforts to quantify and demonstrate quality of programming \citep{Malizia-Cicatelli_2011_BOOK_Catholic-School}. 

For example, in 1976, teacher-child ratios at Padova's religious schools ranged from 1:34-44 \citep{Padova-Admin-Data_1964-2011}. Prior to 1997, preschool teachers at Catholic programs were religious staff, and teacher training was not publicly mandated. After 2000, the Church began to hire university-trained, secular teachers who were enabled to autonomously implement curricula and determine pedagogical methods. In the north, traditional instruction began to incorporate some active-child learning, and some sites began to provide entry to children at age 24 months \citep{Hohnerlein_2009_Paradox-Public-Preschools,CEHD_2016_Historical-Analysis}. 

Until the early 2000s, tuition and fees to families enrolling children in religious preschools in all three cities were relatively more expensive than municipal and state programs. Unlike municipal systems, the majority of program costs for religious preschool is subsidized by tuition paid by enrolled families; historically, religious private schools were considered options only for affluent families that could afford the tuition expense \citep{Hohnerlein_2009_Paradox-Public-Preschools}. Religious schools in Padova did not receive any form of public funding in the 1970s; families were responsible for 100\% of the costs. In the 1980s and 1990s, the municipality of Padova contributed 20\% and 40\% of program costs to religious schools. In the 2000s, families paid 60\% and the remaining 40\% was shared by the state and municipality of Padova.\footnote{This information comes from a variety of sources: \citet{Reggio-Admin-data_1966-2006, Reggio-Annual-Journals_1994-2011, Padova-Admin-Data_1964-2011} and results from the survey \citep{CEHD_2016_Historical-Analysis}},

Today, depending on their income and the particular municipality's subsidy program, families enrolled in religious programs are eligible for subsidized tuition \citep{Hohnerlein_2009_Paradox-Public-Preschools}. In 2005, however, funding for state and private schools differed greatly. While equitable religious preschools could receive state funds, funding for religious primary schools (ages 6-11 years) was at 15.5\% the rate of public primary schools. Private secondary schools (ages 11-18 years) received no state funding \citep{Becchi-Ferrari_1990_Pub-Inf-Centres-Italy}.


%\subsubsection{Infant-Toddler Programs} eligible age of entry for infant-toddler programs varies from 3 months in Reggio Emilia to 9 months in Padova
