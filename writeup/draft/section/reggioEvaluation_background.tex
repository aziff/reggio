We first describe the different types of early childhood education in Italy. We then detail the Reggio Approach, which we consider to be the treatment in this evaluation. Because those who did not receive this treatment did not receive uniform alternative early childhood education experiences, we also document the curricular and programmatic elements of these alternative experiences.

\subsection{Types of Early Childhood Education in Italy}

Italy's current system of early education is divided into two parts according to children's age: (i) infant-toddler care for children younger than 3 years old and (ii) preschool for children between the ages of 3 and 6.\footnote{Current early childhood education reform efforts seek to unify the system of early childhood education to provide continuity of care from 0-6 years.} 

The responsibilities for the funding and provision of preschool are as follows: the state passes laws, defines educational aims, and provides the majority of funding for preschools to regions. Each region may pass laws regarding the organization and distribution of funds for all municipal programs in that region. Municipalities subsidize and manage local schools. Over 96\% of children ages 3-6 years attend preschool which is provided by state, municipal, or private initiatives.  \citep{Becchi-Ferrari_1990_Pub-Inf-Centres-Italy}. For children under the age of 3 years, local provision of childcare in Italy varies considerably as regional and municipal governments are responsible for regulating and funding infant-toddler centers. Some cities in Northern Italy meet the childcare demand for 30-45\% of families with very young children, in contrast to 2.3\% in Southern Italy \citep{Musatti-Picchio_2010_IJEC}. 

Municipalities are enabled to set eligibility criteria for local preschools and infant-toddler centers. Selection criteria are similar across municipalities, however, the weighting of distinct family characteristics varies \citep{Del-Boca-etal_2016_CESifo-ES}. Fees to attend municipal preschools vary; about half of municipalities provide free preschool while others are offered on a sliding scale. State preschools are free to all families.\footnote{All school systems charge for meals and transportation.} In contrast, parental fees for infant-toddler care are much higher, covering 21\% of total program costs on average \citep{Musatti-Picchio_2010_IJEC}. 

The Catholic Church offers the majority of private religious early childhood programs. Until approximately 2000, private schools were considered ``schools for the rich'' given that families had to pay the full tuition themselves \citep{Ribolzi_2013_Italy}. After 2000, the public education system acknowledged non-state schools that complied with eight specific regulations by allowing regions to allocate funds to ``equitable'' private \textit{and} municipal schools. Tuition for religious preschools currently depends on family income, and the amount of public subsidies for non-state schools vary as decided by the region \citep{Hohnerlein_2009_Paradox-Public-Preschools,Ribolzi_2013_Italy}.

To summarize, early childhood education is publicly provided by the municipality or the state, and privately provided by religious institutions or secular ones. Section~\ref{sec:data} describes the selection of our sample into these different school types.

\subsection{The Reggio Approach}

The Reggio Approach is a form of municipal early childhood education designed by Loris Malaguzzi, an educator influenced by the educational practices and psychological theories of Dewey, Piaget, Erikson, Vygotsky, Bronfenbrenner, Kagan, and Gardner. Malaguzzi, along with Bruno Ciari in Bologna, was one of several left-wing educators within the region of Emilia Romagna who were influenced by these progressive models of education. Under the guidance of Malaguzzi, Reggio Emilia opened its first preschool in 1963 for children aged 3-6 years. In 1965, Reggio Emilia opened the first infant-toddler center for children aged 3 months to 3 years. The Reggio Emilia municipal early childhood system thus preceded Italy's legislative reforms in the 1960s and 1970s that established state-run preschools and mandated the local provision of infant-toddler centers \citep{Cagliari-etal-eds_2016_BOOK_Loris-Malaguzzi}. 

In the Reggio Approach, the educative team is assigned specialized roles. Each incoming class of 3-year-olds is assigned two full-time co-teachers who remain with this cohort of same-aged children for three consecutive years. This extended time provides continuity of care for children and enables strong teacher-family engagement. Each school site is further staffed by a full-time atelierista, an instructor with a background in visual arts. Auxiliary site staff, such as cooks and janitors, are considered members of the educative team and participate in trainings and professional development. A pedagogista---or educative coordinator---with a higher degree in psychology or education is assigned to support professional development on a biweekly basis for the educative staff of approximately 4-5 municipal preschools. 

The Reggio Approach is notable for viewing curriculum as an ongoing, collaborative project without pre-determined learning goals or timelines. There is no institutionally-prescribed content knowledge that educators convey to children for ``school readiness.'' In contrast, teachers and children are viewed as researchers and co-creators of knowledge. For example, educators, children and families collaborate to define a question or topic. Learning is then pursued following a scientific process: theories are shared, tested, and revised through dialogue. Teachers observe children's development, interact with children through questions and dialogue, and provide scaffolding to support learning. Children demonstrate their emerging knowledge through creative visual media with aid from the atelierista. Teachers document each child's development in a portfolio---a collection of work---which is shared and discussed with children and parents over the year \citep{Rinaldi_2006_ReggioEmilia_BOOK,Giudici-Nicolosi_2014_Reggio-Approach}. 

The municipal school environment reflects a light-filled and open interior design, furnished with natural materials and a garden. Each site is equipped with an atelier, or dedicated studio laboratory used for creative instructional activities, and in-house kitchen used for daily meal preparation. Preschools and infant-toddler centers are open five full-time days per week from September through June \citep{Giudici-Nicolosi_2014_Reggio-Approach}. Extended day options are available by request at several municipal preschool locations throughout the school year, as is educational programming in July. Children with disabilities and single parents have been prioritized in admission criteria for infant-toddler and preschool programs since before it was decreed by 1971 state regulations \citep{Edwards-etal-eds_1998_Hundred-Languages}.

%The Reggio Emilia municipal infant-toddler model varies somewhat in that caregivers are assigned to infants and toddlers in single-year increments. Infant-toddler teachers generally have lower levels of initial training and receive less pay relative to preschool educators. Although the Reggio Emilia infant-toddler model does not include a full-time atelierista, caregivers receive training and support for lesson plans from municipal atelieristas \citep{Cagliari-etal-eds_2016_BOOK_Loris-Malaguzzi,Giudici-Nicolosi_2014_Reggio-Approach}. 

\subsection{Other Early Childhood Education Experiences}

Municipal early childhood education programs in Reggio Emilia, Parma, and Padova differ in certain aspects of program administration, environmental features, and pedagogical methods. We discuss these differences below. \textbf{[AZ: This section will be expanded as we receive and translate responses from the survey.]}

\subsubsection{Parma Municipal Schools}

Detailed documentation of the Parma municipal early childhood system is scarce. Conversation with the authors of \citet{Edwards-etal-eds_1998_Hundred-Languages} suggests that municipal schools in Parma are parallel to those of Reggio Emilia.\footnote{Kuperman, Interview with Carolyn Pope Edwards, 2016.} 

As of 2001, 16 infant-toddler centers were offered throughout the municipality. The administration of these centers are managed by a director of services for children under 3 years of age. Pedagogical coordinators perform both administrative and professional development roles. Assigned to a specific set of infant-toddler centers, these coordinators meet twice each month with all teachers collectively for shared reflection, on-site supervision, and to promote relationships with the families. The city director meets biweekly with all pedagogical coordinators for overall planning. University professors or administrators from other municipalities provide professional development in the form of continuing education \citep{Terzi-Cantarelli_2001_Parma}.

In contrast to pre-fabricated preschool centers, Parma's infant-toddler centers are intentionally designed in the context of an apartment. \citet{Terzi-Cantarelli_2001_Parma} report mixed-age classes that include 18 total children from 13 months to 3 years in a single section, led by two teachers (9:1 child-teacher ratio). To accommodate parents, infant-toddler centers open at 7:30pm and offer three pick-up times: 2 p.m. (short-day), 3:30 p.m. (normal-day), or 5 p.m. (extended-day). Classrooms can be organized by single-age groups (e.g., 5-12 months, 12-24 months, and 24-36 months) or by mixed-age groups (e.g.,12-36 months) \citep{Majorano-etal_2009_CC-in-P}.

\subsubsection{Padova Municipal Schools}
Padova is located in the relatively more religious and politically diverse region of Veneto. Compared to Reggio Emilia and Parma, its municipal early childhood education system is smaller and it has a larger number of private religious programs. 

In 1989, the region of Veneto reported a total provision of childcare slots for 3.9\% of its infant-toddler population. In contrast, the region of Emilia Romagna reported a provision of infant-toddler childcare for 15.6\% of its population. The practice of professional development trainings for early childhood staff in Veneto first began in 1986 \citep{Becchi-Ferrari_1990_Pub-Inf-Centres-Italy}.

\subsubsection{Catholic Schools}
The Catholic Church offers the majority of private religious early childhood programs. Unlike municipal programs, tuition for private religious programs is at least partially the family's responsibility. In some cases, families can receive subsidies depending on their income and the particular municipality's subsidy program \citep{Hohnerlein_2009_Paradox-Public-Preschools}.  The Catholic Church in Italy is concerned with equity and parity of state funding for private schools.\footnote{In 2005, funding for state and private schools differed greatly. Authorized private schools received a state contribution for pre-primary schools (ages 2-6 years), while funding for private primary schools (ages 6-11 years) was at 15.5\% the rate of public primary schools. Private secondary schools (ages 11-18 years) received no state funding \citep{Becchi-Ferrari_1990_Pub-Inf-Centres-Italy}.} Prior to March 2000, state funding for private schools reflected a 1947 constitutional clause that non-state schools could operate ``without financial burdens on the state.'' Private schools were thus considered options only for affluent families that could afford the tuition expense \citep{Hohnerlein_2009_Paradox-Public-Preschools}.
