\subsection{Overview}
This section describes the Reggio Approach and why it is important to evaluate this program. Reggio Emilia is notable for its significant investment in innovative municipal programs and services. It is the earliest municipal system to evolve, and historically offers the largest number of municipal infant-toddler and preschool sites.\footnote{Similar to Parma and Padova, Reggio Emilia contracts with local private providers and cooperatives to offer a number of infant-toddler and preschool slots according to municipal regulations. These ``affiliated'' programs may not follow the Reggio Approach nor the municipal approaches in Parma and Padova. Accordingly, we do not include those who were enrolled in affiliated programs in our evaluation.} While eligible, Reggio Emilia did not receive state funding for its municipal early childhood system until the 1990s and 2000s. Ironically, the municipality contributed funds to its local state schools each decade from the 1970s. In 1994, Reggio Approach staff provided training for religious preschool teachers in Reggio Emilia. 
%SK: Should we add a cite per Daniela del B ``most  the principles of the Reggio Approach have become increasingly part of a large number of childcare experience in Italy and Europe (you can cite Bennet 2012 and Lazzari and Vandenbroek 2012) from our paper''? I think this perspective is supported by the historical survey. Also not sure whether JJH will want to include one of her citations...

The Reggio Approach is a form of progressive early childhood education influenced by Loris Malaguzzi, an educator promoting the educational practices and psychological theories of Dewey, Piaget, Erikson, Vygotsky, Bronfenbrenner, Kagan, and Gardner. In 1963, Reggio Emilia opened its first preschool for children aged 3-6 years. In 1965, municipal policies were enacted to provide funding for infant-toddler centers for children aged 3 months to 3 years; in 1971, the first site opened. Reggio Emilia's municipal early childhood system thus preceded Italy's key legislative reforms that established state-run preschools and mandated the local provision of infant-toddler centers \citep{Cagliari-etal-eds_2016_BOOK_Loris-Malaguzzi}. The Reggio Approach views curriculum as an ongoing, collaborative project without pre-determined learning goals or timelines. There is no institutionally-prescribed content knowledge that educators convey to children for ``school readiness.'' In contrast, teachers and children are viewed as researchers and co-creators of knowledge. For example, educators, children and families collaborate to define a question or topic. Learning is then pursued following a scientific process: theories are shared, tested, and revised through dialogue. 

In Reggio Approach preschools, the educative team is assigned specialized roles. Each incoming class of approximately twenty-five 3-year-olds is assigned two full-time co-teachers (teacher-child ratio of 2:25); at least 1 of the 2 teachers remains with this cohort of homogeneous-aged children for three consecutive years. This extended time provides continuity of care for children and enables strong teacher-family engagement. Each preschool site is further staffed by a full-time atelierista, an instructor with a background in visual arts. Teachers observe children's development, interact with children through questions and dialogue, and provide scaffolding to support learning. Children demonstrate their emerging knowledge through creative learning activities and art, with aid from the atelierista. Teachers document each child's development in a portfolio---a collection of work---which is shared and discussed with children and parents over the year \citep{Rinaldi_2006_ReggioEmilia_BOOK,Giudici-Nicolosi_2014_Reggio-Approach}. Auxiliary site staff, such as cooks and janitors, are considered members of the educative team and participate in trainings and professional development. A pedagogista---or educative coordinator---with a higher degree in psychology or education is assigned to support professional development on a biweekly basis for the educative staff of approximately 4-5 municipal preschools. 

The Reggio Approach school environment reflects a light-filled, open interior design, furnished with natural materials and a garden. Each site is equipped with an atelier, or dedicated studio laboratory used for creative instructional activities. In-house kitchens are surrounded by glass walls, to include children in the meal process, and is used daily for preparing meals. Reggio Approach preschools and infant-toddler centers are open five full-time days per week from September through June \citep{Giudici-Nicolosi_2014_Reggio-Approach}. Extended day options are available at a majority of Reggio municipal sites throughout the school year, as is educational programming throughout July. Children with disabilities and single parents have been prioritized in admission criteria from the early 1970's \citep{Edwards-etal-eds_1998_Hundred-Languages}. The engagement of families is embedded in Reggio practices, as is the invitation to all community members to participate in school management \citep{CEHD_2016_Historical-Analysis,Cagliari-etal-eds_2016_BOOK_Loris-Malaguzzi}. 

\subsection{Research Challenges}