
\textbf{[Sylvi: Please note that the following program descriptions have been revised from the previous draft to reflect the full set of survey results, with the exception of services for Immigrant populations, given that we don't include this sample in our analysis.]} 

\subsection{State Preschools}

The state regulates and provides preschool education for ages 3-6 years, however, it does not provide infant-toddler programs. State preschools are administered according to legislated policies. Educational practices are guided by Orientamenti which define program standards and general goals for early childhood education. Orientamenti are revised periodically to reflect political ideology and contemporary academic practices. Historically, however, early childhood policies were not consistently enforced throughout Italy, nor were Orientamenti considered binding. 

The cohorts in our sample thus had differential access to state preschools within and across cities; those who enrolled in state programs experienced varying early childhood curricula and administrative practices. The Age 40 cohort had access to 3 or fewer state preschools in each city \citep{Reggio-Admin-data_1966-2006, Reggio-Annual-Journals_1994-2011, Padova-Admin-Data_1964-2011}.\footnote{This sample is too small to distinguish in our evaluation.} In 1969, the first Orientamenti for free state preschools provided only vague guidelines for early childhood education, development and physical care \citep{Corsaro_1996_Early-Edu,Hohnerlein_2015_Development-and-Diffusion}. Children of the age 40 cohort enrolled in state preschools experienced: a) prioritized enrollment for children with disabilities; b) classrooms staffed with 2 fully trained teachers; and, c) male teachers \citep{Hohnerlein_2015_Development-and-Diffusion}. Children of the age 30 cohort who enrolled in state preschools experienced teacher child-ratios of 2:35, mandated in 1980. Both Adult cohorts in state preschools were taught by teachers trained in Catholic institutions, as opposed to secular and academic institutions.

The Adolescent and Child cohorts had access to several improvements in state preschools than did the Adult cohorts. In 1991, revised Orientamenti first emphasized social, affective and cognitive development; Play, collaboration, and mealtime skills were promoted as the key tasks of early childhood \citep{Corsaro_1996_Early-Edu}. In 1997, new mandates required university degrees and supervised experience for state teachers and equivalent pay to teachers in primary schools \citep{Ghedini_2001_Ital-Natl-Policy}. 

In contrast to the Reggio Approach, teaching practices reflect direct instruction as well as play-based learning \citep{CEHD_2016_Historical-Analysis}. Religious teaching is offered, and parents can opt out.\foonote{Alternative educational experiences, however, may not be offered during this time.} Teachers in state preschools work 33 hours per week than their municipal counterparts, who work 36 hours/week. To meet the maximum of 33 hours within an 8 hour school day, one state teacher arrives at 8am, while the second arrives midday and stays until all parents pick up at 4pm. Children in state preschools thus spend more time in a classroom staffed by only 1 teacher than children in municipal schools. Teachers in state preschools further have less weekly time set aside for professional training, documentation, and engagement of parents.

\subsection{Religious Schools}

\textbf{[JJH: I thought we were to use this. Sylvi: Correct, we will ensure the term remains consistent throughout.]}

The Catholic Church offers the majority of religious education and is the oldest of the three early childhood systems, providing for disadvantaged children since the 19th century \citep{OECD_2001_Italy-Country-Note}. All five cohorts in our evaluation had access to religious programming for ages 3-6 years. While infant-toddler religious education was not available for Adults and Adolescents cohorts, spillover from local municipal systems is evident in the availability of transitional religious programming for children aged 24 months for the Child cohorts in Reggio Emilia, Parma, and Padova \citep{Malizia-Cicatelli_2011_BOOK_Catholic-School}. 

Compared to Reggio Emilia and Parma, Padova offers the largest number of religious preschool programs. Compared to the municipal systems in each city, however, teacher-child ratios at Padova's religious schools are much higher. In the 1970s, historical records indicate one teacher for 34-44 preschool-aged children. By the late 1990s, equitable religious programming in each city would require hiring and staffing of teachers at state mandated minimums.

Historically, religious preschools were options only for families that could afford the expense \citep{Hohnerlein_2009_Paradox-Public-Preschools,Ribolzi_2013_Italy}. Prior to 2000, state funding for private schools reflected a 1947 constitutional clause that non-state schools could operate ``without financial burdens on the state.''  \citep{Hohnerlein_2009_Paradox-Public-Preschools}. Accordingly, tuition and fees for religious preschool programs in all three cities were relatively more expensive than municipal and state programs for the oldest four cohorts. Survey results and historical records indicate that religious schools in Padova did not receive any form of public funding in the 1970s; families of the Age 40 cohort who chose religious preschools were responsible for 100\% of the costs. In the 1980s, when the Age 30 cohort was eligible to attend, the municipality of Padova subsidized 20\% of program costs for local religious schools. In the 1990s, when Adolescents were eligible to attend, Padova contributed 40\% of program costs to local religious schools. In the 2000s, when the Child cohort was eligible, families paid 60\%; the remaining 40\% was shared by the state and by Padova \citep{Reggio-Admin-data_1966-2006, Reggio-Annual-Journals_1994-2011, Padova-Admin-Data_1964-2011,CEHD_2016_Historical-Analysis}. 

\subsection{The Municipality of Reggio Emilia}

Reggio Emilia's municipal system currently offers 19 preschool centers. Its infant-toddler system includes 9 full-day centers and 3 part-day centers; infants are eligible to attend from 3 months of age. The municipality further contracts with local private providers to offer a limited number of preschool and infant-toddler slots to resident families according to municipal regulations. While eligible, Reggio Emilia did not receive state funding for its municipal early childhood system until the 1990s and 2000s.\footnote{Ironically, the municipality contributed funds to its local state schools each decade from the 1970s. In 1994, Reggio Approach staff provided training for religious preschool teachers in Reggio Emilia.}
% Maybe for the discussion? Last two sentences above

Survey results indicate that from the 1990s to the present, contracted preschool and infant providers offer programming that vary quite a bit from each other, and vary somewhat from the Reggio Approach. Survey results further indicate the Reggio municipal system predicts its own programs vary a lot from those in Parma's municipal system, and vary a great deal from those in Padova's municipal system \citep{CEHD_2016_Historical-Analysis}. In our evaluation, we do not include former attendees of contracted programs within Reggio Emilia from any of the five cohorts in the evaluation of those who attended Reggio Approach schools. 

\subsection{The Municipality of Parma}

Parma's municipal early childhood system consolidated and expanded around 1975, about a decade after that of Reggio Emilia. Parma's municipal early childhood system is comparatively smaller than Reggio Emilia, currently offering 12 municipal preschools, 8 municipal infant-toddler centers, and 4 ``experimental'' centers for children aged 18 months through 6 years. Distinct from Reggio Emilia, the earliest age of entry into infant programs is later, ranging from 5 to 9 months. 

Like Reggio Emilia, Parma contracts with private and cooperative preschool and infant-toddler providers to meet mandated regulations for provisioning by local demand. Survey results indicate that the programming of contracted providers varies quite a bit from Parma's municipal early childhood approach.\footnote{We do not include former attendees of contracted programs within Parma in the sample of those who attended Parma municipal schools.} Parma predicts that its own municipal system varies a lot from that of Reggio Emilia. The municipality of Parma currently receives state funds for its municipal programs; survey results state funding was first provided in 1980s \citep{CEHD_2016_Historical-Analysis}.

Detailed documentation of Parma's municipal preschools is limited; Conversation with experts familiar with the region suggest that the pedagogical approach of Parma's municipal early childhood system is similar to that of Reggio Emilia.\footnote{Kuperman, Interview with Carolyn Pope Edwards, 2016.} In 2001,\citet{Terzi-Cantarelli_2001_Parma} offer descriptions of Parma's infant-toddler programming that do indeed appear very similar to the Reggio Approach. In contrast to Reggio Emilia, where children are eligible from age 3 months, Parma's infant-toddler programs enroll children beginning at age 5 months. Educative Coordinators perform both administrative and professional development roles similar to Pedagogistas in Reggio Emilia. Assigned to a specific set of infant-toddler centers, Educative Coordinators meet twice each month with all teachers collectively for shared reflection, on-site supervision, and to promote relationships with the families. The city director meets biweekly with Educational Coordinators for overall planning. University professors or administrators from other municipalities provide professional development in the form of continuing education \citep{Terzi-Cantarelli_2001_Parma}. Classrooms are organized by single-age groups (e.g., 5-12 months, 12-24 months, and 24-36 months) or by mixed-age groups (e.g.,12-36 months) \citep{Majorano-etal_2009_CC-in-P}. Mixed-age classes include 18 total children from 13 months to 3 years in a single section, led by two teachers for a 1:9 teacher-child ratio. To accommodate parents, extended hours are available at infant-toddler centers as are three pick-up times: 2 p.m. (short-day), 3:30 p.m. (normal-day), or 5 p.m. (extended-day).

\subsection{The Municipality of Padova}

Padova's municipal preschool system began to consolidate in 1973, expanding from two to five sites by 1976. Padova's municipal preschool system currently offers 10 preschool centers and 17 infant-toddler centers. Padova, distinct from Reggio Emilia and Parma, offers free municipal preschool, and families pay only for meals. Eligible age of entry to infant-toddler programs varies across sites, ranging from 3 to 13 months of age. Like Reggio Emilia and Parma, Padova charges families for infant-toddler programming. 

Reports and survey data suggest that provisioning and quality improvements in Padova's municipal early childhood system was the last of the three cities in our evaluation to evolve. Administrative records from 1976 indicate that teacher-child ratios in Padova's early municipal preschools ranged up to 1:24 \citep{Padova-Admin-Data_1964-2011,CEHD_2016_Historical-Analysis}. In 1989, the region of Veneto reported a total provision of childcare slots for 3.9\% of its infant-toddler population. In contrast, the region of Emilia Romagna reported a provision of infant-toddler childcare for 15.6\% of its population. Professional development for Padova's municipal early childhood staff first began in the mid-1980s, about 20 years after Reggio Emilia and a decade after Parma \citep{Becchi-Ferrari_1990_Pub-Inf-Centres-Italy,CEHD_2016_Historical-Analysis}. In Padova, educative coordinators were actually full-time teachers who were additionally tasked to serve this role on a rotating basis. Not until 2010 did Padova invest in providing pedagogical leaders for the educational staff of its municipal schools. Survey results indicate that Padova first received state funds for its municipal early childhood programs in the 1980s, and additionally received funds from the region of Veneto in the 1990s and 2000s. 

%In Padova, state preschools are also free, however, families are expected to make an additional contribution to accommodate expenses associated with field trips.\citep{CEHD_2016_Historical-Analysis}. In 1976, there were three state preschools in Padova; enrollment was relatively lower compared to religious and municipal programs. In the newly provided state preschools, teacher-child ratios were approximately 1:15.

