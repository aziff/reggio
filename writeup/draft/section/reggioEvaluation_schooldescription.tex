We present additional information on the early childhood systems. They are listed here in order of age of the program.

\subsection{Religious Early Childhood Programs}

The Catholic Church offers the majority of religious education and is the oldest of the three early childhood systems, providing for disadvantaged children since the 19th century \citep{OECD_2001_Italy-Country-Note}. All five cohorts in our evaluation had access to religious programming for children ages 3-6 years. The provision of religious infant-toddler childcare varies by cohort. Adolescents had access to transitional religious programs for children over 24 months of age. Child cohorts in Reggio Emilia, Parma, and Padova had some access to religious sites that offer programs from 12 months \citep{Malizia-Cicatelli_2011_BOOK_Catholic-School}.

Historically, religious preschools were options only for families that could afford the expense \citep{Hohnerlein_2009_Paradox-Public-Preschools,Ribolzi_2013_Italy}. Prior to 2000, state funding for private schools reflected a 1947 constitutional clause that non-state schools could operate ``without financial burdens on the state'' \citep{Hohnerlein_2009_Paradox-Public-Preschools}. Accordingly, tuition and fees for religious preschool programs in all three cities were relatively more expensive than municipal and state programs for the oldest four cohorts. Survey results and historical records indicate that religious schools in Padova did not receive any form of public funding in the 1970s; families of the age-40 cohort who chose religious preschools were responsible for 100\% of the costs. In the 1980s, when the age-30 cohort was eligible to attend, the municipality of Padova subsidized 20\% of program costs for local religious schools. In the 1990s, when the adolescent cohort was eligible to attend, Padova contributed 40\% of program costs to local religious schools. In the 2000s, when the child cohort was eligible to attend, families paid 60\% and the remaining 40\% was shared by the state and by Padova \citep{Reggio-Admin-data_1966-2006, Reggio-Annual-Journals_1994-2011, Padova-Admin-Data_1964-2011,CEHD_2016_Historical-Analysis}.

\subsection{The Municipality of Reggio Emilia}

Reggio Emilia's municipal system currently operates 19 preschool centers. There are 9 full-day infant-toddler centers and three part-day centers; infants are eligible to attend from 3 months of age.

Survey results indicate that Reggio Emilia's municipal system perceives that their programming varies a lot from the programming of Parma's municipal system and varies a great deal from that of Padova's municipal system \citep{CEHD_2016_Historical-Analysis}.

While eligible, Reggio Emilia did not receive state funding for its municipal early childhood system until the 1990s and 2000s. The municipality, however, contributed funds to state preschools in Reggio Emilia starting in the 1970s.

\textbf{[JJH: What are Reggio-affiliated programs? They are mentioned but never defined. Do they have the same curriculum as Reggio? Do they differ in outcomes?]}

\subsection{State Preschools}

The state regulates and provides preschool education for children ages 3-6 years, however, it does not provide infant-toddler programs. State preschools are administered according to legislated policies, most notably Law 444 enacted in 1968. Educational practices are guided by Orientamenti which define program standards and general goals for early childhood education. Orientamenti are revised periodically to reflect political ideology and contemporary academic practices. Historically, however, legislated policies were not consistently enforced throughout Italy, nor were Orientamenti considered binding.

The cohorts in our sample had differential access to state preschools within and across cities due to Law 444; those who enrolled in state programs experienced varying early childhood curricula and administrative practices. The age-40 cohort had access to less than 3 state preschools in each city \citep{Reggio-Admin-data_1966-2006,Reggio-Annual-Journals_1994-2011,Padova-Admin-Data_1964-2011}. In 1969, the first Orientamenti for free state preschools provided only vague guidelines for early childhood education, development and physical care \citep{Corsaro_1996_Early-Edu,Hohnerlein_2015_Development-and-Diffusion}. Children of the age-40 cohort enrolled in state preschools may have experienced: (i) prioritized enrollment for children with disabilities; (ii) classrooms staffed with 2 fully trained teachers; and, (iii) male teachers \citep{Hohnerlein_2015_Development-and-Diffusion}. Children of the age-30 cohort who enrolled in state preschools experienced teacher child-ratios of 2:35, mandated in 1980. Both adult cohorts in state preschools were taught by teachers trained in Catholic institutions, as opposed to secular academic universities.

The adolescent and child cohorts had access to several improvements in state preschools than did the adult cohorts. In 1991, revised Orientamenti first emphasized social, affective and cognitive development; play, collaboration, and mealtime skills were promoted as the key tasks of early childhood \citep{Corsaro_1996_Early-Edu}. In 1997, new mandates required university degrees and supervised experience for state teachers and equivalent pay to teachers in primary schools \citep{Ghedini_2001_Ital-Natl-Policy}.

In Padova, state preschools are free. However, families are expected to make an additional contribution to accommodate expenses associated with field trips \citep{CEHD_2016_Historical-Analysis}. In 1976, there were three state preschools in Padova; enrollment was relatively lower compared to religious and municipal programs. In the newly provided state preschools, teacher-child ratios are approximately 1:15.

\subsection{The Municipality of Parma}

Parma's municipal early childhood system consolidated and expanded around 1975, about a decade after that of Reggio Emilia. Parma's municipal early childhood system is comparatively smaller than Reggio Emilia, currently offering 12 municipal preschools, 8 municipal infant-toddler centers, and 4 ``experimental'' centers for children ages 18 months through 6 years. Distinct from Reggio Emilia, the earliest age of entry into infant programs is later, ranging from 5 to 9 months. Parma reports that its own municipal system varies a lot from that of Reggio Emilia.

The municipality of Parma currently receives state funds for its municipal programs; survey results state funding was first provided in 1980s \citep{CEHD_2016_Historical-Analysis}.

Detailed documentation of Parma's municipal preschools is limited; Conversation with experts familiar with the region suggest that the pedagogical approach of Parma's municipal early childhood system is similar to that of Reggio Emilia.\footnote{Kuperman, Interview with Carolyn Pope Edwards, 2016.}

%In 2001,\citet{Terzi-Cantarelli_2001_Parma} offer descriptions of Parma's infant-toddler programming that do indeed appear very similar to the Reggio Approach. In contrast to Reggio Emilia, where children are eligible from age 3 months, Parma's infant-toddler programs enroll children beginning at age 5 months. Educative Coordinators perform both administrative and professional development roles similar to Pedagogistas in Reggio Emilia. Assigned to a specific set of infant-toddler centers, Educative Coordinators meet twice each month with all teachers collectively for shared reflection, on-site supervision, and to promote relationships with the families. The city director meets biweekly with Educative Coordinators for overall planning. University professors or administrators from other municipalities provide professional development in the form of continuing education \citep{Terzi-Cantarelli_2001_Parma}. Classrooms are organized by single-age groups (e.g., 5-12 months, 12-24 months, and 24-36 months) or by mixed-age groups (e.g.,12-36 months) \citep{Majorano-etal_2009_CC-in-P}. Mixed-age classes include 18 total children from 13 months to 3 years in a single section, led by two teachers for a 1:9 teacher-child ratio. To accommodate parents, extended hours are available at infant-toddler centers as are three pick-up times: 2 p.m. (short-day), 3:30 p.m. (normal-day), or 5 p.m. (extended-day).

\subsection{The Municipality of Padova}

Padova's municipal preschool system began to consolidate in 1973, expanding from two to five sites by 1976. Padova's municipal preschool system currently offers 10 preschool centers and 17 infant-toddler centers. The eligible age of entry to infant-toddler childcare varies across municipal sites, ranging from 3 to 13 months of age.

Padova, distinct from Reggio Emilia and Parma, offers free municipal preschool, and families pay only for meals. Like Reggio Emilia and Parma, however, Padova charges families for infant-toddler services.

Reports and survey data suggest that investment in provision and quality improvements by Padova in its municipal early childhood system occurred in the 1980s, about 15 years after the Reggio Approach consolidated. For example, municipal archives dated 1976 indicate that teacher-child ratios in Padova's earliest municipal preschools were 1:24, implying that investment in staffing was very different in Padova than in the Reggio Approach \citep{Padova-Admin-Data_1964-2011,CEHD_2016_Historical-Analysis}. Professional development for Padova's municipal early childhood staff first began in the mid-1980s, about 20 years after Reggio Emilia and a decade after Parma \citep{Becchi-Ferrari_1990_Pub-Inf-Centres-Italy,CEHD_2016_Historical-Analysis}. In Padova, pedagogical coordinators were not highly trained nor full-time staff. Instead, full-time teachers were additionally tasked to serve this role on a rotating basis. In 2010, Padova first invested in expert pedagogical coordinators to supervise and train municipal teachers.

Survey results indicate that Padova first received state funds for its municipal early childhood programs in the 1980s, and additionally received regional funds from Veneto in the 1990s and 2000s.

