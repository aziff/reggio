%-----------------------------we--------------------------------------------------------------------------------------------%
% writeup
%-------------------------------------------------------------------------------------------------------------------------%
It is possible our estimates suffer from self-selection bias as parents' preschool choices for their children could depend on unobserved characteristics that also affect the outcomes of interest. We attempt to control for such potential selection by estimating an instrumental variables model using two-stage least squares. We estimate the following system of equations:
\begin{align}
D_i &= \alpha_0 + \bm{Z}_i \bm{\alpha} + \bm{X}_i \bm{\delta} + \nu_i \label{eq:first-stage}\\
Y_i &= \beta_0 + \beta_1 \hat{D_i} + \bm{X}_i \bm{\gamma} + \epsilon_i \label{eq:second-stage}
\end{align}
where %Equation~\eqref{eq:first-stage} corresponds to the first stage, Equation \eqref{eq:second-stage} corresponds to the second stage, 
$\bm{X}_i$ is the vector of BIC-selected baseline controls, $D_i$ is an indicator for attending a Reggio Approach preschool, $\hat{D_i}$ is the predicted value of $D_i$ obtained from the first-stage regression~\eqref{eq:first-stage}, and $\bm{Z}_i$ is a vector of instruments that influence the choice of preschool $D_i$ but are assumed to have no effect on the outcome $Y_i$. The vector of instruments, $\bm{Z}_i$, includes four variables measuring the distance between individuals' residence and the closest municipal, state, religious and private preschools; the squared terms for each of the four distance instruments; and a variable that approximates the score used by the Reggio Approach preschools to rank applicants and allocate available slots based on baseline background characteristics.\footnote{The score is approximated as a function of number of siblings, parents' employment status, parents' migrant status, whether parents were adoptive or custodial, whether both parents are present in household, and distance to grandparents' residence. The score was constructed using weights published by the \citet{Reggio-Emilia_2012_Criteria-admission-2013-14}.} Cost of attendance was explored as as a potential instrument. It is not included in the analysis because preschool fees are determined based on household income and our data suffers from missing income data.

The instrumental variable (IV) analysis is only presented for children and adolescents due to lack of available instruments for adults. Distance, which is constructed based on the residential address reported during time of interview, becomes increasingly unreliable as respondents age because individuals are more likely to leave their childhood homes and establish new residences as they enter adulthood. The score is constructed using guidelines published in 2012. This instrument also becomes increasingly unreliable with age of respondent because the weighting scheme used to allocate available slots to children is likely to have evolved over time as a result of changes in economic, social, demographic and cultural conditions.

Section~\ref{appendix:iv-first} presents results from the estimation of equation~\eqref{eq:first-stage}, the first stage regression. The results are generally insignificant and show that the instruments are weak predictors of enrollment in Reggio Approach preschools. Given the weak first stage, the IV estimation is unlikely to substantially correct for potential bias stemming from selection into Reggio Approach preschools. Estimates of the effect of the Reggio Approach (the second stage regression of equation~\eqref{eq:second-stage}) are presented in Section~\ref{appendix:iv-second} alongside analogous results estimated using propensity score matching and kernel matching approaches for comparison. We use a $z$-test to formally test for differences between the IV estimates and estimates from propensity score matching and kernel matching approaches. The propensity score matching (PSM) and kernel matching columns are marked with daggers, $\bm{\dagger}$, to denote rejection of the null of no difference between IV and comparison estimates at different significance levels.

%-------------------------------------------------------------------------------------------------------------------------%
\subsection{Estimation Results - First Stage}\label{appendix:iv-first}
%-------------------------------------------------------------------------------------------------------------------------%
\begin{table}[H]
\centering
\scalebox{.9}{
\begin{threeparttable}
	\caption{First Stage IV Estimates, Comparison to Non-RA preschools within Reggio Emilia}\label{iv-first-child}
	\begin{tabular}{l L{1cm} l L{1cm} l}
\toprule
 & & Child & & Adol \\
\midrule
Dist. Municipal & & -0.06 & & -0.10 \\
& & (0.07) & & (0.07) \\[5pt]
Dist. Municipal sq. & & 0.00 & & 0.00 \\
& & (0.01) & & (0.01) \\[5pt]
Dist. Private & & 0.03 & & 0.04 \\
& & (0.06) & & (0.04) \\[5pt]
Dist. Private sq. & & -0.01 & & 0.00 \\
& & (0.01) & & (0.01) \\[5pt]
Dist. Religious & & -0.07 & & -0.12 \\
& & (0.13) & & (0.10) \\[5pt]
Dist. Religious sq. & & 0.03 & & 0.06 ** \\
& & (0.04) & & (0.03) \\[5pt]
Dist. State & & 0.01 & & 0.06 \\
& & (0.06) & & (0.06) \\[5pt]
Dist. State sq. & & 0.00 & & -0.01 \\
& & (0.01) & & (0.01) \\[5pt]
Reggio Score & & 0.01 & & -0.01 \\
& & (0.00) & & (0.00) \\[5pt]
\midrule
\textit{F}-stat & & 6.90 & & 8.23 \\
R$^2$ & & 0.09 & & 0.27 \\
N & & 306 & & 285 \\
\bottomrule
\end{tabular}

\begin{tablenotes}
Note: This table presents first stage estimates for the IV approach corresponding to Eq.(\ref{eq:first-stage}). The columns labeled Child and Adol present estimates corresponding to the children and adolescent cohorts respectively. Robust standard errors are reported in parentheses below the point estimates. ***, **, and * indicate significance of the estimated coefficients at the 1\%, 5\%, and 10\% levels respectively. 
\end{tablenotes}
\end{threeparttable}
}
\end{table}

%TODO Add the F-statistic of excluded instruments, to show if there is any power in the first stage

%-------------------------------------------------------------------------------------------------------------------------%
\subsection{Estimation Results - Second Stage}\label{appendix:iv-second}
%-------------------------------------------------------------------------------------------------------------------------%
\subsubsection{Child Cohort}

\begin{table}[H]
\centering
\scalebox{.8}{
\begin{threeparttable}
	\caption{Second Stage IV Estimation Results for Main Outcomes, Comparison to Non-RA preschools within Reggio Emilia}\label{iv-M-child}
	\begin{tabular}{L{9cm} L{1cm} l L{1cm} l l l}
\toprule
 & &         & & \multicolumn{3}{c}{\textbf{\ul{Tests of Equality}}} \\[10pt]
 & & IV & & PSM & & Kernel \\
\midrule
Not Obese & & -0.30 & & -0.08  & & -0.06 \\
\quad \textit{Unadjusted P-Value} & & (0.11)  & & (0.16)  & & (0.28) \\
\quad \textit{Stepdown P-Value} & & (0.60)  & & (0.71)  & & (0.84) \\[3pt]
Not Overweight & & 0.04 & & 0.00  & & -0.01 \\
\quad \textit{Unadjusted P-Value} & & (0.84)  & & (0.99)  & & (0.79) \\
\quad \textit{Stepdown P-Value} & & (0.84)  & & (0.99)  & & (0.99) \\[3pt]
IQ Factor & & -0.32 & & -0.21  & & -0.15 \\
\quad \textit{Unadjusted P-Value} & & (0.42)  & & (0.05) * & & (0.20) \\
\quad \textit{Stepdown P-Value} & & (0.80)  & & (0.37)  & & (0.81) \\[3pt]
Candy Game: Willing to Share Candies & & -0.40 & & -0.03 $\bm{\dagger \dagger$} & & 0.01 $\bm{\dagger \dagger$} \\
\quad \textit{Unadjusted P-Value} & & (0.01) ** & & (0.44)  & & (0.89) \\
\quad \textit{Stepdown P-Value} & & (0.09) * & & (0.96)  & & (0.99) \\[3pt]
Num. of Friends & & 1.11 & & -0.36  & & -0.38 \\
\quad \textit{Unadjusted P-Value} & & (0.29)  & & (0.15)  & & (0.15) \\
\quad \textit{Stepdown P-Value} & & (0.73)  & & (0.71)  & & (0.76) \\[3pt]
Health is Good & & 0.47 & & -0.02 $\bm{\dagger$} & & -0.03 $\bm{\dagger$} \\
\quad \textit{Unadjusted P-Value} & & (0.10)  & & (0.70)  & & (0.64) \\
\quad \textit{Stepdown P-Value} & & (0.60)  & & (0.99)  & & (0.99) \\[3pt]
Not Excited to Learn & & 0.08 & & 0.00  & & -0.00 \\
\quad \textit{Unadjusted P-Value} & & (0.16)  & & (0.92)  & & (0.99) \\
\quad \textit{Stepdown P-Value} & & (0.61)  & & (0.99)  & & (0.99) \\[3pt]
Problems Sitting Still & & -0.10 & & 0.02  & & 0.02 \\
\quad \textit{Unadjusted P-Value} & & (0.66)  & & (0.71)  & & (0.63) \\
\quad \textit{Stepdown P-Value} & & (0.84)  & & (0.99)  & & (0.99) \\[3pt]
How Much Child Likes School & & -0.47 & & 0.10 $\bm{\dagger$} & & 0.11 $\bm{\dagger$} \\
\quad \textit{Unadjusted P-Value} & & (0.10)  & & (0.19)  & & (0.15) \\
\quad \textit{Stepdown P-Value} & & (0.60)  & & (0.71)  & & (0.76) \\[3pt]
SDQ Composite - Child & & 4.29 & & 1.39  & & 1.13 \\
\quad \textit{Unadjusted P-Value} & & (0.14)  & & (0.01) ** & & (0.06) * \\
\quad \textit{Stepdown P-Value} & & (0.60)  & & (0.15)  & & (0.45) \\[3pt]
\bottomrule
\end{tabular}

\begin{tablenotes}
Note: The column labeled IV presents second stage IV estimates corresponding to Equation~\eqref{eq:second-stage}. Below each estimated coefficient is an unadjusted p-value and a stepdown p-value. The columns labeled PSM and Kernel report analogous estimates generated using the corresponding methodologies. ***, **, and * indicate significance of the coefficients at the 1\%, 5\%, and 10\% levels respectively. $\dagger \dagger \dagger$, $\dagger \dagger$, and $\dagger$ denote significance at the 1\%, 5\%, and 10\% level respectively for the $z$-test of the null that no difference exists between the IV estimates and the estimates generated by the corresponding alternative methodologies.
\end{tablenotes}
\end{threeparttable}
}
\end{table}

\begin{table}[H]
\centering
\scalebox{.8}{
\begin{threeparttable}
	\caption{Second Stage IV Estimation Results for Cognitive Outcomes, Comparison to Non-RA preschools within Reggio Emilia}\label{iv-CN-child}
	\begin{tabular}{L{9cm} L{1cm} l L{1cm} l l l}
\toprule
 & &         & & \multicolumn{3}{c}{\textbf{\ul{Tests of Equality}}} \\[10pt]
 & & IV & & PSM & & Kernel \\
\midrule
IQ Factor & & -0.32 & & -0.21  & & -0.15 \\
\quad \textit{Unadjusted P-Value} & & (0.42)  & & (0.05) * & & (0.20) \\
\quad \textit{Stepdown P-Value} & & (0.88)  & & (0.24)  & & (0.63) \\[3pt]
IQ Score & & 0.01 & & -0.05  & & -0.04 \\
\quad \textit{Unadjusted P-Value} & & (0.96)  & & (0.05) * & & (0.17) \\
\quad \textit{Stepdown P-Value} & & (0.99)  & & (0.24)  & & (0.62) \\[3pt]
SDQ Conduct - Child & & 1.43 & & 0.39  & & 0.38 \\
\quad \textit{Unadjusted P-Value} & & (0.15)  & & (0.03) ** & & (0.04) ** \\
\quad \textit{Stepdown P-Value} & & (0.61)  & & (0.17)  & & (0.23) \\[3pt]
SDQ Emotional - Child & & 2.20 & & 0.68  & & 0.64 \\
\quad \textit{Unadjusted P-Value} & & (0.05) * & & (0.00) *** & & (0.00) *** \\
\quad \textit{Stepdown P-Value} & & (0.38)  & & (0.02) ** & & (0.04) ** \\[3pt]
SDQ Hyper - Child & & -0.24 & & 0.26  & & 0.13 \\
\quad \textit{Unadjusted P-Value} & & (0.87)  & & (0.36)  & & (0.65) \\
\quad \textit{Stepdown P-Value} & & (0.99)  & & (0.67)  & & (0.87) \\[3pt]
SDQ Peer problems - Child & & 0.90 & & 0.06  & & -0.02 \\
\quad \textit{Unadjusted P-Value} & & (0.27)  & & (0.73)  & & (0.91) \\
\quad \textit{Stepdown P-Value} & & (0.73)  & & (0.76)  & & (0.88) \\[3pt]
SDQ Pro-social - Child & & 0.07 & & 0.23  & & 0.15 \\
\quad \textit{Unadjusted P-Value} & & (0.95)  & & (0.30)  & & (0.48) \\
\quad \textit{Stepdown P-Value} & & (0.99)  & & (0.67)  & & (0.84) \\[3pt]
SDQ Composite - Child & & 4.29 & & 1.39  & & 1.13 \\
\quad \textit{Unadjusted P-Value} & & (0.14)  & & (0.01) ** & & (0.06) * \\
\quad \textit{Stepdown P-Value} & & (0.60)  & & (0.12)  & & (0.32) \\[3pt]
\bottomrule
\end{tabular}

\begin{tablenotes}
Note: The column labeled IV presents second stage IV estimates corresponding to Equation~\eqref{eq:second-stage}. Below each estimated coefficient is an unadjusted p-value and a stepdown p-value. The columns labeled PSM and Kernel report analogous estimates generated using the corresponding methodologies. ***, **, and * indicate significance of the coefficients at the 1\%, 5\%, and 10\% levels respectively. $\dagger \dagger \dagger$, $\dagger \dagger$, and $\dagger$ denote significance at the 1\%, 5\%, and 10\% level respectively for the $z$-test of the null that no difference exists between the IV estimates and the estimates generated by the corresponding alternative methodologies.
\end{tablenotes}
\end{threeparttable}
}
\end{table}

\begin{table}[H]
\centering
\scalebox{.8}{
\begin{threeparttable}
	\caption{Second Stage IV Estimation Results for Social Outcomes, Comparison to Non-RA preschools within Reggio Emilia}\label{iv-S-child}
	\begin{tabular}{L{9cm} L{1cm} l L{1cm} l l l}
\toprule
 & &         & & \multicolumn{3}{c}{\textbf{\ul{Tests of Equality}}} \\[10pt]
 & & IV & & PSM & & Kernel \\
\midrule
Candy Game: Willing to Share Candies & & -0.40 & & -0.03 $\bm{\dagger \dagger$} & & 0.01 $\bm{\dagger \dagger$} \\
\quad \textit{Unadjusted P-Value} & & (0.01) ** & & (0.44)  & & (0.89) \\
\quad \textit{Stepdown P-Value} & & (0.07) * & & (0.84)  & & (0.98) \\[3pt]
Num. of Friends & & 1.11 & & -0.36  & & -0.38 \\
\quad \textit{Unadjusted P-Value} & & (0.29)  & & (0.15)  & & (0.15) \\
\quad \textit{Stepdown P-Value} & & (0.87)  & & (0.68)  & & (0.71) \\[3pt]
Musical Instrument at Home & & 0.09 & & -0.05  & & -0.03 \\
\quad \textit{Unadjusted P-Value} & & (0.75)  & & (0.38)  & & (0.66) \\
\quad \textit{Stepdown P-Value} & & (0.98)  & & (0.84)  & & (0.98) \\[3pt]
Tell Worry to Friends & & -0.09 & & 0.03  & & 0.01 \\
\quad \textit{Unadjusted P-Value} & & (0.71)  & & (0.58)  & & (0.86) \\
\quad \textit{Stepdown P-Value} & & (0.98)  & & (0.84)  & & (0.98) \\[3pt]
Tell Worry at Home & & -0.24 & & -0.09  & & -0.07 \\
\quad \textit{Unadjusted P-Value} & & (0.38)  & & (0.15)  & & (0.22) \\
\quad \textit{Stepdown P-Value} & & (0.87)  & & (0.68)  & & (0.82) \\[3pt]
Keep Worry to Myself & & 0.03 & & -0.04  & & -0.02 \\
\quad \textit{Unadjusted P-Value} & & (0.90)  & & (0.43)  & & (0.68) \\
\quad \textit{Stepdown P-Value} & & (0.98)  & & (0.84)  & & (0.98) \\[3pt]
Tell Worry to Teacher & & 0.23 & & 0.07  & & 0.04 \\
\quad \textit{Unadjusted P-Value} & & (0.29)  & & (0.21)  & & (0.51) \\
\quad \textit{Stepdown P-Value} & & (0.87)  & & (0.69)  & & (0.97) \\[3pt]
\bottomrule
\end{tabular}

\begin{tablenotes}
Note: The column labeled IV presents second stage IV estimates corresponding to Equation~.\eqref{eq:second-stage}. Below each estimated coefficient is an unadjusted p-value and a stepdown p-value. The columns labeled PSM and Kernel report analogous estimates generated using the corresponding methodologies. ***, **, and * indicate significance of the coefficients at the 1\%, 5\%, and 10\% levels respectively. $\dagger \dagger \dagger$, $\dagger \dagger$, and $\dagger$ denote significance at the 1\%, 5\%, and 10\% level respectively for the $z$-test of the null that no difference exists between the IV estimates and the estimates generated by the corresponding alternative methodologies.
\end{tablenotes}
\end{threeparttable}
}
\end{table}

\begin{table}[H]
\centering
\scalebox{.8}{
\begin{threeparttable}
	\caption{Second Stage IV Estimation Results for Health Outcomes, Comparison to Non-RA preschools within Reggio Emilia}\label{iv-H-child}
	\begin{tabular}{L{9cm} L{1cm} l L{1cm} l l l}
\toprule
 & &         & & \multicolumn{3}{c}{\textbf{\ul{Tests of Equality}}} \\[10pt]
 & & IV & & PSM & & Kernel \\
\midrule
Not Obese & & -0.30 & & -0.08  & & -0.06 \\
\quad \textit{Unadjusted P-Value} & & (0.11)  & & (0.16)  & & (0.28) \\
\quad \textit{Stepdown P-Value} & & (0.35)  & & (0.45)  & & (0.70) \\[3pt]
Not Overweight & & 0.04 & & 0.00  & & -0.01 \\
\quad \textit{Unadjusted P-Value} & & (0.84)  & & (0.99)  & & (0.79) \\
\quad \textit{Stepdown P-Value} & & (0.83)  & & (0.99)  & & (0.94) \\[3pt]
Health is Good & & 0.47 & & -0.02 $\bm{\dagger$} & & -0.03 $\bm{\dagger$} \\
\quad \textit{Unadjusted P-Value} & & (0.10)  & & (0.70)  & & (0.64) \\
\quad \textit{Stepdown P-Value} & & (0.35)  & & (0.92)  & & (0.94) \\[3pt]
Number of Sick Days & & -0.30 & & -0.07  & & -0.05 \\
\quad \textit{Unadjusted P-Value} & & (0.56)  & & (0.46)  & & (0.60) \\
\quad \textit{Stepdown P-Value} & & (0.76)  & & (0.84)  & & (0.94) \\[3pt]
\bottomrule
\end{tabular}

\begin{tablenotes}
Note: The column labeled IV presents second stage IV estimates corresponding to Equation~\eqref{eq:second-stage}. Below each estimated coefficient is an unadjusted p-value and a stepdown p-value. The columns labeled PSM and Kernel report analogous estimates generated using the corresponding methodologies. ***, **, and * indicate significance of the coefficients at the 1\%, 5\%, and 10\% levels respectively. $\dagger \dagger \dagger$, $\dagger \dagger$, and $\dagger$ denote significance at the 1\%, 5\%, and 10\% level respectively for the $z$-test of the null that no difference exists between the IV estimates and the estimates generated by the corresponding alternative methodologies.
\end{tablenotes}
\end{threeparttable}
}
\end{table}

\newpage
\begin{table}[H]
\centering
\scalebox{.8}{
\begin{threeparttable}
	\caption{Second Stage IV Estimation Results for Behavioral Outcomes, Comparison to Non-RA preschools within Reggio Emilia}\label{iv-B-child}
	\begin{tabular}{L{9cm} L{1cm} l L{1cm} l l l}
\toprule
 & &         & & \multicolumn{3}{c}{\textbf{\ul{Tests of Equality}}} \\[10pt]
 & & IV & & PSM & & Kernel \\
\midrule
Not Excited to Learn & & 0.08 & & 0.00  & & -0.00 \\
\quad \textit{Unadjusted P-Value} & & (0.16)  & & (0.92)  & & (0.99) \\
\quad \textit{Stepdown P-Value} & & (0.38)  & & (0.92)  & & (0.98) \\[3pt]
Problems Sitting Still & & -0.10 & & 0.02  & & 0.02 \\
\quad \textit{Unadjusted P-Value} & & (0.66)  & & (0.71)  & & (0.63) \\
\quad \textit{Stepdown P-Value} & & (0.76)  & & (0.92)  & & (0.95) \\[3pt]
Happy in General & & 0.57 & & 0.13  & & -0.03 \\
\quad \textit{Unadjusted P-Value} & & (0.53)  & & (0.52)  & & (0.89) \\
\quad \textit{Stepdown P-Value} & & (0.76)  & & (0.85)  & & (0.98) \\[3pt]
How Much Child Likes School & & -0.47 & & 0.10 $\bm{\dagger$} & & 0.11 $\bm{\dagger$} \\
\quad \textit{Unadjusted P-Value} & & (0.10)  & & (0.19)  & & (0.15) \\
\quad \textit{Stepdown P-Value} & & (0.30)  & & (0.54)  & & (0.45) \\[3pt]
\bottomrule
\end{tabular}

\begin{tablenotes}
Note: The column labeled IV presents second stage IV estimates corresponding to Equation~\eqref{eq:second-stage}. Below each estimated coefficient is an unadjusted p-value and a stepdown p-value. The columns labeled PSM and Kernel report analogous estimates generated using the corresponding methodologies. ***, **, and * indicate significance of the coefficients at the 1\%, 5\%, and 10\% levels respectively. $\dagger \dagger \dagger$, $\dagger \dagger$, and $\dagger$ denote significance at the 1\%, 5\%, and 10\% level respectively for the $z$-test of the null that no difference exists between the IV estimates and the estimates generated by the corresponding alternative methodologies.
\end{tablenotes}
\end{threeparttable}
}
\end{table}
%-------------------------------------------------------------------------------------------------------
\subsubsection{Adolescent Cohort}

\begin{table}[H]
\centering
\scalebox{.8}{
\begin{threeparttable}
	\caption{Second Stage IV Estimation Results for Main Outcomes, Comparison to Non-RA preschools within Reggio Emilia}\label{iv-M-adol}
	\begin{tabular}{L{9cm} L{1cm} l L{1cm} l l l}
\toprule
 & &         & & \multicolumn{3}{c}{\textbf{\ul{Tests of Equality}}} \\[10pt]
 & & IV & & PSM & & Kernel \\
\midrule
Not Obese & & -0.01 & & -0.07  & & -0.07 \\
\quad \textit{Unadjusted P-Value} & & (0.96)  & & (0.10)  & & (0.15) \\
\quad \textit{Stepdown P-Value} & & (0.98)  & & (0.73)  & & (0.86) \\[3pt]
Not Overweight & & -0.13 & & -0.03  & & 0.01 \\
\quad \textit{Unadjusted P-Value} & & (0.30)  & & (0.42)  & & (0.84) \\
\quad \textit{Stepdown P-Value} & & (0.95)  & & (0.98)  & & (0.99) \\[3pt]
Num. of Friends & & 0.06 & & -0.69  & & 0.18 \\
\quad \textit{Unadjusted P-Value} & & (0.99)  & & (0.56)  & & (0.92) \\
\quad \textit{Stepdown P-Value} & & (0.98)  & & (0.99)  & & (0.99) \\[3pt]
IQ Factor & & -0.04 & & -0.06  & & -0.14 \\
\quad \textit{Unadjusted P-Value} & & (0.91)  & & (0.53)  & & (0.25) \\
\quad \textit{Stepdown P-Value} & & (0.98)  & & (0.99)  & & (0.96) \\[3pt]
Trust Score & & -1.17 & & 0.09  & & 0.13 \\
\quad \textit{Unadjusted P-Value} & & (0.16)  & & (0.71)  & & (0.57) \\
\quad \textit{Stepdown P-Value} & & (0.84)  & & (0.99)  & & (0.99) \\[3pt]
Health is Good & & 0.17 & & 0.05  & & 0.02 \\
\quad \textit{Unadjusted P-Value} & & (0.48)  & & (0.50)  & & (0.82) \\
\quad \textit{Stepdown P-Value} & & (0.98)  & & (0.99)  & & (0.99) \\[3pt]
Go To School & & -0.30 & & -0.01 $\bm{\dagger$} & & -0.00 $\bm{\dagger$} \\
\quad \textit{Unadjusted P-Value} & & (0.05) * & & (0.76)  & & (0.96) \\
\quad \textit{Stepdown P-Value} & & (0.54)  & & (0.99)  & & (0.99) \\[3pt]
How Much Child Likes School & & -0.95 & & -0.04  & & -0.08 \\
\quad \textit{Unadjusted P-Value} & & (0.10)  & & (0.74)  & & (0.55) \\
\quad \textit{Stepdown P-Value} & & (0.73)  & & (0.99)  & & (0.96) \\[3pt]
Depression Score - positive & & 2.03 & & 2.24  & & 2.70 \\
\quad \textit{Unadjusted P-Value} & & (0.48)  & & (0.03) ** & & (0.01) ** \\
\quad \textit{Stepdown P-Value} & & (0.98)  & & (0.36)  & & (0.14) \\[3pt]
Locus of Control - positive & & -0.01 & & 0.07  & & 0.07 \\
\quad \textit{Unadjusted P-Value} & & (0.99)  & & (0.52)  & & (0.55) \\
\quad \textit{Stepdown P-Value} & & (0.98)  & & (0.99)  & & (0.99) \\[3pt]
SDQ Composite & & 0.04 & & 1.02  & & 1.20 \\
\quad \textit{Unadjusted P-Value} & & (0.99)  & & (0.22)  & & (0.13) \\
\quad \textit{Stepdown P-Value} & & (0.98)  & & (0.94)  & & (0.86) \\[3pt]
SDQ Composite - Child & & -2.02 & & -0.56  & & 0.08 \\
\quad \textit{Unadjusted P-Value} & & (0.28)  & & (0.49)  & & (0.92) \\
\quad \textit{Stepdown P-Value} & & (0.95)  & & (0.99)  & & (0.99) \\[3pt]
Days of Sport (Weekly) & & -0.44 & & -0.32  & & -0.66 \\
\quad \textit{Unadjusted P-Value} & & (0.60)  & & (0.33)  & & (0.03) ** \\
\quad \textit{Stepdown P-Value} & & (0.98)  & & (0.98)  & & (0.32) \\[3pt]
Volunteers & & -0.17 & & -0.05  & & -0.01 \\
\quad \textit{Unadjusted P-Value} & & (0.46)  & & (0.52)  & & (0.84) \\
\quad \textit{Stepdown P-Value} & & (0.98)  & & (0.99)  & & (0.99) \\[3pt]
\bottomrule
\end{tabular}

\begin{tablenotes}
Note: The column labeled IV presents second stage IV estimates corresponding to Equation~\eqref{eq:second-stage}. Below each estimated coefficient is an unadjusted p-value and a stepdown p-value. The columns labeled PSM and Kernel report analogous estimates generated using the corresponding methodologies. ***, **, and * indicate significance of the coefficients at the 1\%, 5\%, and 10\% levels respectively. $\dagger \dagger \dagger$, $\dagger \dagger$, and $\dagger$ denote significance at the 1\%, 5\%, and 10\% level respectively for the $z$-test of the null that no difference exists between the IV estimates and the estimates generated by the corresponding alternative methodologies.
\end{tablenotes}
\end{threeparttable}
}
\end{table}

\begin{table}[H]
\centering
\scalebox{.8}{
\begin{threeparttable}
	\caption{Second Stage IV Estimation Results for Cognitive Outcomes, Comparison to Non-RA preschools within Reggio Emilia}\label{iv-CN-adol}
	\begin{tabular}{L{9cm} L{1cm} l L{1cm} l l l}
\toprule
 & &         & & \multicolumn{3}{c}{\textbf{\ul{Tests of Equality}}} \\[10pt]
 & & IV & & PSM & & Kernel \\
\midrule
IQ Factor & & -0.04 & & -0.06  & & -0.14 \\
\quad \textit{Unadjusted P-Value} & & (0.91)  & & (0.53)  & & (0.25) \\
\quad \textit{Stepdown P-Value} & & (0.99)  & & (0.99)  & & (0.96) \\[3pt]
IQ Score & & -0.01 & & -0.01  & & -0.03 \\
\quad \textit{Unadjusted P-Value} & & (0.89)  & & (0.80)  & & (0.40) \\
\quad \textit{Stepdown P-Value} & & (0.99)  & & (0.99)  & & (0.99) \\[3pt]
Depression Score - positive & & 2.03 & & 2.24  & & 2.70 \\
\quad \textit{Unadjusted P-Value} & & (0.48)  & & (0.03) ** & & (0.01) ** \\
\quad \textit{Stepdown P-Value} & & (0.99)  & & (0.35)  & & (0.11) \\[3pt]
SDQ Conduct & & -0.03 & & 0.44  & & 0.63 \\
\quad \textit{Unadjusted P-Value} & & (0.97)  & & (0.12)  & & (0.01) ** \\
\quad \textit{Stepdown P-Value} & & (0.99)  & & (0.78)  & & (0.13) \\[3pt]
SDQ Emotional & & 0.55 & & 0.23  & & 0.24 \\
\quad \textit{Unadjusted P-Value} & & (0.57)  & & (0.51)  & & (0.49) \\
\quad \textit{Stepdown P-Value} & & (0.99)  & & (0.99)  & & (0.99) \\[3pt]
SDQ Hyper & & -0.44 & & 0.39  & & 0.57 \\
\quad \textit{Unadjusted P-Value} & & (0.66)  & & (0.21)  & & (0.08) * \\
\quad \textit{Stepdown P-Value} & & (0.99)  & & (0.96)  & & (0.67) \\[3pt]
SDQ Peer problems & & -0.04 & & -0.05  & & -0.24 \\
\quad \textit{Unadjusted P-Value} & & (0.95)  & & (0.85)  & & (0.28) \\
\quad \textit{Stepdown P-Value} & & (0.99)  & & (0.99)  & & (0.96) \\[3pt]
SDQ Pro-social & & 0.06 & & 0.06  & & -0.18 \\
\quad \textit{Unadjusted P-Value} & & (0.95)  & & (0.81)  & & (0.50) \\
\quad \textit{Stepdown P-Value} & & (0.99)  & & (0.99)  & & (0.99) \\[3pt]
SDQ Composite & & 0.04 & & 1.02  & & 1.20 \\
\quad \textit{Unadjusted P-Value} & & (0.99)  & & (0.22)  & & (0.13) \\
\quad \textit{Stepdown P-Value} & & (0.99)  & & (0.96)  & & (0.81) \\[3pt]
SDQ Conduct - Child & & 0.43 & & 0.11  & & 0.18 \\
\quad \textit{Unadjusted P-Value} & & (0.49)  & & (0.64)  & & (0.42) \\
\quad \textit{Stepdown P-Value} & & (0.99)  & & (0.99)  & & (0.99) \\[3pt]
SDQ Emotional - Child & & -0.03 & & -0.22  & & -0.06 \\
\quad \textit{Unadjusted P-Value} & & (0.97)  & & (0.56)  & & (0.84) \\
\quad \textit{Stepdown P-Value} & & (0.99)  & & (0.99)  & & (0.99) \\[3pt]
SDQ Hyper - Child & & -1.15 & & -0.07  & & 0.18 \\
\quad \textit{Unadjusted P-Value} & & (0.13)  & & (0.78)  & & (0.53) \\
\quad \textit{Stepdown P-Value} & & (0.82)  & & (0.99)  & & (0.99) \\[3pt]
SDQ Peer problems - Child & & -1.26 & & -0.37  & & -0.22 \\
\quad \textit{Unadjusted P-Value} & & (0.07) * & & (0.09) * & & (0.37) \\
\quad \textit{Stepdown P-Value} & & (0.66)  & & (0.69)  & & (0.99) \\[3pt]
SDQ Pro-social - Child & & 1.02 & & 0.08  & & -0.11 \\
\quad \textit{Unadjusted P-Value} & & (0.22)  & & (0.78)  & & (0.70) \\
\quad \textit{Stepdown P-Value} & & (0.96)  & & (0.99)  & & (0.99) \\[3pt]
SDQ Composite - Child & & -2.02 & & -0.56  & & 0.08 \\
\quad \textit{Unadjusted P-Value} & & (0.28)  & & (0.49)  & & (0.92) \\
\quad \textit{Stepdown P-Value} & & (0.99)  & & (0.99)  & & (0.99) \\[3pt]
\bottomrule
\end{tabular}

\begin{tablenotes}
Note: The column labeled IV presents second stage IV estimates corresponding to Equation~\eqref{eq:second-stage}. Below each estimated coefficient is an unadjusted p-value and a stepdown p-value. The columns labeled PSM and Kernel report analogous estimates generated using the corresponding methodologies. ***, **, and * indicate significance of the coefficients at the 1\%, 5\%, and 10\% levels respectively. $\dagger \dagger \dagger$, $\dagger \dagger$, and $\dagger$ denote significance at the 1\%, 5\%, and 10\% level respectively for the $z$-test of the null that no difference exists between the IV estimates and the estimates generated by the corresponding alternative methodologies.
\end{tablenotes}
\end{threeparttable}
}
\end{table}

\begin{table}[H]
\centering
\scalebox{.8}{
\begin{threeparttable}
	\caption{Second Stage IV Estimation Results for Social Outcomes, Comparison to Non-RA preschools within Reggio Emilia}\label{iv-S-adol}
	\begin{tabular}{L{9cm} L{1cm} l L{1cm} l l l}
\toprule
 & &         & & \multicolumn{3}{c}{\textbf{\ul{Tests of Equality}}} \\[10pt]
 & & IV & & PSM & & Kernel \\
\midrule
Num. of Friends & & 0.06 & & -0.69  & & 0.18 \\
\quad \textit{Unadjusted P-Value} & & (0.99)  & & (0.56)  & & (0.92) \\
\quad \textit{Stepdown P-Value} & & (0.98)  & & (0.70)  & & (0.99) \\[3pt]
Doesn't Talk About Activities & & 0.14 & & 0.12  & & 0.08 \\
\quad \textit{Unadjusted P-Value} & & (0.70)  & & (0.25)  & & (0.38) \\
\quad \textit{Stepdown P-Value} & & (0.96)  & & (0.66)  & & (0.88) \\[3pt]
Doesn't Talk About School & & 0.08 & & 0.13  & & 0.02 \\
\quad \textit{Unadjusted P-Value} & & (0.81)  & & (0.22)  & & (0.81) \\
\quad \textit{Stepdown P-Value} & & (0.96)  & & (0.66)  & & (0.99) \\[3pt]
Volunteers & & -0.17 & & -0.05  & & -0.01 \\
\quad \textit{Unadjusted P-Value} & & (0.46)  & & (0.52)  & & (0.84) \\
\quad \textit{Stepdown P-Value} & & (0.93)  & & (0.70)  & & (0.99) \\[3pt]
\bottomrule
\end{tabular}

\begin{tablenotes}
Note: The column labeled IV presents second stage IV estimates corresponding to Equation~\eqref{eq:second-stage}. Below each estimated coefficient is an unadjusted p-value and a stepdown p-value. The columns labeled PSM and Kernel report analogous estimates generated using the corresponding methodologies. ***, **, and * indicate significance of the coefficients at the 1\%, 5\%, and 10\% levels respectively. $\dagger \dagger \dagger$, $\dagger \dagger$, and $\dagger$ denote significance at the 1\%, 5\%, and 10\% level respectively for the $z$-test of the null that no difference exists between the IV estimates and the estimates generated by the corresponding alternative methodologies.
\end{tablenotes}
\end{threeparttable}
}
\end{table}

\begin{table}[H]
\centering
\scalebox{.8}{
\begin{threeparttable}
	\caption{Second Stage IV Estimation Results for Health Outcomes, Comparison to Non-RA preschools within Reggio Emilia}\label{iv-H-adol}
	\begin{tabular}{L{9cm} L{1cm} l L{1cm} l l l}
\toprule
 & &         & & \multicolumn{3}{c}{\textbf{\ul{Tests of Equality}}} \\[10pt]
 & & IV & & PSM & & Kernel \\
\midrule
Not Obese & & -0.01 & & -0.07  & & -0.07 \\
\quad \textit{Unadjusted P-Value} & & (0.96)  & & (0.10)  & & (0.15) \\
\quad \textit{Stepdown P-Value} & & (0.96)  & & (0.41)  & & (0.55) \\[3pt]
Not Overweight & & -0.13 & & -0.03  & & 0.01 \\
\quad \textit{Unadjusted P-Value} & & (0.30)  & & (0.42)  & & (0.84) \\
\quad \textit{Stepdown P-Value} & & (0.78)  & & (0.77)  & & (0.99) \\[3pt]
Ever Suspended from School & & 0.14 & & 0.03  & & 0.02 \\
\quad \textit{Unadjusted P-Value} & & (0.29)  & & (0.32)  & & (0.66) \\
\quad \textit{Stepdown P-Value} & & (0.78)  & & (0.77)  & & (0.98) \\[3pt]
Health is Good & & 0.17 & & 0.05  & & 0.02 \\
\quad \textit{Unadjusted P-Value} & & (0.48)  & & (0.50)  & & (0.82) \\
\quad \textit{Stepdown P-Value} & & (0.84)  & & (0.77)  & & (0.99) \\[3pt]
Number of Sick Days & & -0.10 & & -0.01  & & -0.03 \\
\quad \textit{Unadjusted P-Value} & & (0.79)  & & (0.89)  & & (0.83) \\
\quad \textit{Stepdown P-Value} & & (0.96)  & & (0.90)  & & (0.99) \\[3pt]
\bottomrule
\end{tabular}

\begin{tablenotes}
Note: The column labeled IV presents second stage IV estimates corresponding to Equation~\eqref{eq:second-stage}. Below each estimated coefficient is an unadjusted p-value and a stepdown p-value. The columns labeled PSM and Kernel report analogous estimates generated using the corresponding methodologies. ***, **, and * indicate significance of the coefficients at the 1\%, 5\%, and 10\% levels respectively. $\dagger \dagger \dagger$, $\dagger \dagger$, and $\dagger$ denote significance at the 1\%, 5\%, and 10\% level respectively for the $z$-test of the null that no difference exists between the IV estimates and the estimates generated by the corresponding alternative methodologies.
\end{tablenotes}
\end{threeparttable}
}
\end{table}

\begin{table}[H]
\centering
\scalebox{.8}{
\begin{threeparttable}
	\caption{Second Stage IV Estimation Results for Behavioral Outcomes, Comparison to Non-RA preschools within Reggio Emilia}\label{iv-B-adol}
	\begin{tabular}{L{9cm} L{1cm} l L{1cm} l l l}
\toprule
 & &         & & \multicolumn{3}{c}{\textbf{\ul{Tests of Equality}}} \\[10pt]
 & & IV & & PSM & & Kernel \\
\midrule
Bothered by Migrants & & -0.57 & & 0.22 $\bm{\dagger$} & & 0.25 $\bm{\dagger$} \\
\quad \textit{Unadjusted P-Value} & & (0.14)  & & (0.09) * & & (0.06) * \\
\quad \textit{Stepdown P-Value} & & (0.49)  & & (0.46)  & & (0.27) \\[3pt]
Trust Score & & -1.17 & & 0.09  & & 0.13 \\
\quad \textit{Unadjusted P-Value} & & (0.16)  & & (0.71)  & & (0.57) \\
\quad \textit{Stepdown P-Value} & & (0.49)  & & (0.99)  & & (0.98) \\[3pt]
Not Excited to Learn & & 0.13 & & -0.00  & & 0.01 \\
\quad \textit{Unadjusted P-Value} & & (0.32)  & & (0.96)  & & (0.74) \\
\quad \textit{Stepdown P-Value} & & (0.55)  & & (0.99)  & & (0.98) \\[3pt]
Problems Sitting Still & & 0.17 & & 0.05  & & 0.01 \\
\quad \textit{Unadjusted P-Value} & & (0.25)  & & (0.27)  & & (0.70) \\
\quad \textit{Stepdown P-Value} & & (0.55)  & & (0.82)  & & (0.98) \\[3pt]
Go To School & & -0.30 & & -0.01 $\bm{\dagger$} & & -0.00 $\bm{\dagger$} \\
\quad \textit{Unadjusted P-Value} & & (0.05) * & & (0.76)  & & (0.96) \\
\quad \textit{Stepdown P-Value} & & (0.34)  & & (0.99)  & & (0.98) \\[3pt]
How Much Child Likes School & & -0.95 & & -0.04  & & -0.08 \\
\quad \textit{Unadjusted P-Value} & & (0.10)  & & (0.74)  & & (0.55) \\
\quad \textit{Stepdown P-Value} & & (0.47)  & & (0.99)  & & (0.98) \\[3pt]
Days of Sport (Weekly) & & -0.44 & & -0.32  & & -0.66 \\
\quad \textit{Unadjusted P-Value} & & (0.60)  & & (0.33)  & & (0.03) ** \\
\quad \textit{Stepdown P-Value} & & (0.55)  & & (0.85)  & & (0.20) \\[3pt]
\bottomrule
\end{tabular}

\begin{tablenotes}
Note: The column labeled IV presents second stage IV estimates corresponding to Equation~\eqref{eq:second-stage}. Below each estimated coefficient is an unadjusted p-value and a stepdown p-value. The columns labeled PSM and Kernel report analogous estimates generated using the corresponding methodologies. ***, **, and * indicate significance of the coefficients at the 1\%, 5\%, and 10\% levels respectively. $\dagger \dagger \dagger$, $\dagger \dagger$, and $\dagger$ denote significance at the 1\%, 5\%, and 10\% level respectively for the $z$-test of the null that no difference exists between the IV estimates and the estimates generated by the corresponding alternative methodologies.
\end{tablenotes}
\end{threeparttable}
}
\end{table}



