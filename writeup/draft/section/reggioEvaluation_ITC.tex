\subsection{Estimating Effects of Infant-Toddler Centers}
The Reggio Approach includes two stages of early childhood interventions: (i) Infant Toddler Centers between ages 0-3 and (ii) Preschool between ages 3-6. Table \ref{tab:sample-asilo} shows the infant-toddler center attendance by each cohort, city, and school type. This shows that there are only small number of people in the adult cohorts who attended infant-toddler centers, whereas more than half of our sample in children and adolescent cohorts have attended some type of infant-toddler centers.

\begin{table}[H]
\centering
\scalebox{0.9}{
\begin{threeparttable}
	\caption{Tabulation of Infant-Toddler Center Attendance by Cohort, City, and School Type}\label{tab:sample-asilo}
	\begin{tabular}{l*{7}{c}}
\toprule
		&	\mc{6}{c}{Reggio Emilia: 1,486}													\\	\midrule
		&	None	&	Muni	&	Reli	&	Priv	&	Muni-Affi	&	Other	\\	\midrule		
\textbf{Children	}&		&		&		&		&		&		\\	
\quad Italians	&	115	&	109	&	28	&	6	&	51	&	0	\\			
\quad Migrants		&	58	&	24	&	2	&	0	&	20	&	3	\\			
\textbf{Adolescents}		&	129	&	112	&	10	&	3	&	36	&	3	\\			
\textbf{Adults 30s}		&	210	&	53	&	2	&	3	&	1	&	7	\\			
\textbf{Adults 40s}		&	241	&	31	&	0	&	0	&	0	&	5	\\			
\textbf{Adults 50s}		&	194	&	0	&	1	&	0	&	0	&	1	\\	\midrule		
		&	\mc{6}{c}{ Parma: 1,211}											\\	\midrule		
		&	None	&	Muni	&	Reli	&	Priv	&	Muni-Affi	&	Other	\\	\midrule	
\textbf{Children}&		&		&		&		&		&		\\		
\quad Italians		&	98	&	99	&	7	&	15	&	48	&	21	\\			
\quad Migrants		&	24	&	23	&	1	&	0	&	9	&	1	\\			
\textbf{Adolescents}		&	126	&	74	&	10	&	11	&	25	&	2	\\			
\textbf{Adults 30s}		&	187	&	31	&	8	&	6	&	11	&	4	\\			
\textbf{Adults 40s}		&	222	&	0	&	2	&	0	&	10	&	16	\\			
\textbf{Adults 50s}		&	85	&	0	&	4	&	0	&	0	&	13	\\	\midrule		
		&	\mc{6}{c}{Padova: 1,322}											\\	\midrule		
		&	None	&	Muni	&	Reli	&	Priv	&	Muni-Affi	&	Other	\\	\midrule		
\textbf{Children}&		&		&		&		&		&		\\		
\quad Italians		&	143	&	48	&	26	&	40	&	19	&	1	\\			
\quad Migrants		&	57	&	44	&	3	&	5	&	0	&	1	\\			
\textbf{Adolescents}		&	209	&	52	&	8	&	0	&	6	&	1	\\			
\textbf{Adults 30s}		&	220	&	19	&	5	&	3	&	0	&	0	\\			
\textbf{Adults 40s}		&	225	&	0	&	7	&	0	&	1	&	17	\\			
\textbf{Adults 50s}		&	133	&	0	&	6	&	0	&	0	&	0	\\			


\bottomrule
\end{tabular}


\begin{tablenotes}
Note: This table shows the sample size by city, cohort, and school type. We separate migrants and children for clarity in this table even though they are in the same birth cohort (year of birth: 2006). None: no preschool; Muni.: municipal preschool; Relig.: religious preschool; Priv.: private preschool. Muni-Affi: municipal-affiliated preschool; Other: uncategorized preschool.
\end{tablenotes}
\end{threeparttable}
}
\end{table}



Table~\ref{tab:cases-treat} shows the four possible combinations of interventions that a child could potentially receive, where 0 indicates not attending and 1 indicates attending. It is important to note that the 1 case only includes those who attended Municipal institutions, and 0 includes those who who didn't attend any childcare as well as those who attended non-Municipal childcare.

\begin{table}[H]
\caption{Possible Cases of Treatment} \label{tab:cases-treat}
\begin{tabular}{C{1.8cm} R{0.7cm} C{2cm} C{2cm}}

		& & \multicolumn{2}{c}{Preschool (Ages 3-6)} \\
		& & 0 & 1 \\ \cline{3-4}
        								 &  & \multicolumn{1}{|c|}{} & \multicolumn{1}{c|}{} \\
        							& 0 & \multicolumn{1}{|c|}{(0,0)} & \multicolumn{1}{c|}{(0,1)} \\
        				ITC				&  & \multicolumn{1}{|c|}{} & \multicolumn{1}{c|}{} \\ \cline{3-4}
                        (Age 0-3)  		&  & \multicolumn{1}{|c|}{} & \multicolumn{1}{c|}{} \\
        								& 1 & \multicolumn{1}{|c|}{(1,0)} & \multicolumn{1}{c|}{(1,1)} \\
        								&  & \multicolumn{1}{|c|}{} & \multicolumn{1}{c|}{} \\ \cline{3-4}
\end{tabular}
\begin{flushleft}
\footnotesize{Note:} We only consider municipal ITCs (infant-toddler-centers, ages 0-3) and preschools (ages 3-6). (0,0): did not attend any municipal school for both ages 0-3 and 3-6; (1,0): attended a municipal school for ages 0-3 but did \textit{not} attend for ages 3-6; (0,1): did \textit{not} attend a municipal school for ages 0-3 but did attend for ages 3-6; (1,1): attended a municipal school for both ages 0-3 and 3-6.
\end{flushleft}
\end{table}

There are two main methods to test the effect of attending infant-toddler centers. The first is to compare people who did not attend infant-toddler care or preschool with people who only attended municipal infant-toddler care. Using the notation in Table~\ref{tab:cases-treat}, this comparison is between (0,0) and (1,0). The second method is to compare people who only attended municipal preschool with people who attended both municipal infant-toddler centers and preschools. That is, to compare (0,1) and (1,1). The hypotheses are formally written as
\begin{eqnarray}
H_1: &  Y_{0,0} = Y_{1,0} \\
H_2: &  Y_{0,1} = Y_{1,1}
\end{eqnarray}
\noindent where $Y_{i,j}$ is the outcome of the individuals who attended $i \in \{0,1\}$ infant-toddler care and $j \in \{0,1\}$ preschool.

A possible estimation strategy is to limit the sample to Reggio and a specific cohort constrained to the comparison groups needed according to the hypotheses above. To test $H_1$, we estimate the following regression equation:
\begin{eqnarray}
Y_{i}^{c,h} & = & \alpha + \beta_{0}R_i^{ITC} + \mathbf{X}_i \bm{\gamma} + \varepsilon_{i}^{Reggio,h}, \\ \nonumber
& \forall & i \in \text{ \{People in Reggio and cohort $h$ and in group (0,0) or (1,0)\}}
\end{eqnarray}
where $R_i^{ITC}$ is the indicator for attending municipal infant-toddler center and $\mathbf{X_i}$ is the vector of baseline variables for individual $i$. Likewise, to test $H_2$:
\begin{eqnarray}
Y_{i}^{c,h} & = & \alpha + \beta_{0}R_i^{ITC} + \mathbf{X}_i \bm{\gamma} + \varepsilon_{i}^{Reggio,h}, \\ \nonumber
& \forall & i \in \text{ \{People in Reggio and cohort $h$ and in group (0,1) or (1,1)\}.}
\end{eqnarray}

One caveat of this analysis is that it uses a limited sample size. In our data, these hypotheses cannot be tested under this strategy for many groups. Table~\ref{tab:num-group-2} shows the number of individuals available for each group necessary for analysis using this strategy. It is impossible to test $H_1$ in our data, because there are almost no people who attended municipal infant-toddler care without attending preschool (the group (1,0)). While it is possible to test $H_2$ for several groups, the number of observations for the group (1,1) is small for the adult cohorts.

In Table \ref{tab:num-group-2}, the groups subject to our estimation are highlighted. Based on the available number of individuals in each cell and the history of the foundation date of municipal infant-toddler care for Reggio, we decide to test $H_2$ for the highlighted groups.

%\begin{table}[H] \caption{Number of Individuals in Each Group} \label{tab:num-group-2}
%\scalebox{0.77}{
%\begin{tabular}{l|ccccc|ccccc|ccccc}
%\toprule
%			& 		\multicolumn{5}{c}{\textbf{Reggio}}		& 	\multicolumn{5}{|c|}{\textbf{Parma}}	& 			\multicolumn{5}{c}{\textbf{Padova}}				\\
%			& (0,0) & (1,0) & (0,1) & (1,1) & Total & (0,0) & (1,0) & (0,1) & (1,1) & Total  & (0,0) & (1,0) & (0,1) & (1,1) & Total \\ \midrule
%Child		& 2 & 0 & \cellcolor{blue!25}46 & \cellcolor{blue!25}117 & \textbf{311} & 5 & 1 & \cellcolor{blue!25}35 & \cellcolor{blue!25}100 & \textbf{291} & 2 & 0 & \cellcolor{blue!25}31 & \cellcolor{blue!25}36 & \textbf{278} \\
%Migrant		& 4 & 0	& 24 & 26 & \textbf{110} & 4 & 0 & 12 & 23 & \textbf{58} & 5 & 0 & 18 & 16 & \textbf{113} \\
%Adolescent 	& 7 & 0 & \cellcolor{blue!25}45 &	\cellcolor{blue!25}116 & \textbf{300} & 4 & 0 & \cellcolor{blue!25}49 & \cellcolor{blue!25}61 & \textbf{254} & 1 & 0 & \cellcolor{blue!25}55 & \cellcolor{blue!25}37 & \textbf{282} \\
%Age-30		& 57 & 0 & \cellcolor{blue!25}95 &	\cellcolor{blue!25}53 & \textbf{280} & 43 & 0 & \cellcolor{blue!25}64 & \cellcolor{blue!25}29 & \textbf{251} & 47 & 0 & 25 & 9 & \textbf{251} \\
%Age-40		& 80 & 0 & \cellcolor{blue!25}97 &	\cellcolor{blue!25}28 & \textbf{285} & 115 &00 & 0 & 0 & \textbf{254} & 75 & 0 & 0 & 0 & \textbf{252} \\ \bottomrule
%\end{tabular}}
%\end{table}


\begin{table}[H] \caption{Number of Individuals in Each Group} \label{tab:num-group-2}
\scalebox{0.77}{
\begin{tabular}{lccccc}
\toprule
			& 		\multicolumn{5}{c}{\textbf{Reggio}}		\\
			& (0,0) & (1,0) & (0,1) & (1,1) & Total \\ \midrule
Child		& 6 & 0 & \cellcolor{blue!25}66 & \cellcolor{blue!25}94 & \textbf{419} \\
Adolescent 	& 7 & 0 & \cellcolor{blue!25}39 &	\cellcolor{blue!25}93 & \textbf{299} \\
Age-30		& 57 & 0 & \cellcolor{blue!25}110 &	\cellcolor{blue!25}47 & \textbf{280} \\
Age-40		& 80 & 0 & \cellcolor{blue!25}90 &	\cellcolor{blue!25}26 & \textbf{285} \\ \bottomrule
\end{tabular}}
\end{table}
