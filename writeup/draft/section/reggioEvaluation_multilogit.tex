\subsection{Differences in Selection across Cities}
We estimate multinomial logit models for each city using choice of preschool type as the dependent variable, and a vector of baseline characteristics as independent variables. We then compare the estimated marginal effects for each baseline characteristic to investigate whether a common set of baseline characteristics determines selection into municipal preschools across the three cities. This analysis highlights characteristics that determine selection into the municipal schools relative to preschool alternatives available within the same city. Substantial differences in selection across cities would suggest that selection into municipal schools differ by city. This could be a result of different parental perceptions of quality, the available slots, or other factors determining demand for municipal schools relative to alternative options.

We define the choices available to parents in each city as follows:
\begin{equation}
S_i = \quad
\begin{cases}
0 \quad \text{if \quad (\textit{i} did not attend any preschool)} \\
1 \quad  \text{if \quad (\textit{i} attended an alternative preschool)} \\
2 \quad \text{if \quad (\textit{i} attended a municipal preschool)}  \\
\end{cases}
\end{equation}

\noindent We then estimate the following multinomial logit specification separately for each city:

\begin{equation} \label{eq:log-odds-ratio}
\nu_r = \text{log} \left(\frac{\Pr(S_i=r)}{\Pr(S_i=2)}\right) = \beta_0^r +  \bm{X}_i\bm{\beta}^r + \epsilon_i^r \quad \forall \quad r \in \{0,1\}
\end{equation}

\noindent where $\bm{X}_i$ is a vector of baseline characteristics including child's gender, number of child's siblings, religiosity of caregiver, mothers education level, and indicators for whether parents were born in province.

The predicted log-odds ratio for each alternative option, $\hat{\nu}_r$, from (\ref{eq:log-odds-ratio}) can be used to construct the estimated probability of attending each school type $r$, $\hat{\pi}_{i,r}$, as follows:

\begin{equation}
\hat{\pi}_{i,r} = \frac{\text{exp}(\hat{\nu}_{r,i})}{\sum\limits_{j=0}^{2} \text{exp}(\hat{\nu}_{j,i})} \quad \forall  r \in \{0,1,2\}.
\end{equation}

Finally, we estimate the impact of each characteristic $x_c$ of $\bm{X}$ on the propensity to attend a school type $r$, $\pi_{r}$, by calculating the partial derivative $\partial \pi_{r}/\partial x_c$. These results are reported in Appendix~\ref{appendix:mlogit}. Comparisons across coefficients across cities are given in Tables~\ref{mlogit_all_chi-ado} and~\ref{mlogit_all}.

\textbf{[JJH: Group, I asked for comparisons across the \ul{coefficients} of the choice probabilities across cities and an F-test for equality, i.e. $\beta^r$ same in all cities? Test equality.]}

\textbf{[We are working on this.]}


% ==============================================================================%
\subsection{Estimation Results}\label{appendix:mlogit}
% ==============================================================================%



\begin{table}[H]
\centering
\caption{Multinomial Logit, Child and Adolescent Cohorts, Reggio Emilia} \label{mlogit-chi-ado-RE}
\begin{adjustbox}{width=\textwidth}
\begin{threeparttable}
{
\def\sym#1{\ifmmode^{#1}\else\(^{#1}\)\fi}
\begin{tabular}{l*{6}{c}}
\toprule
			& \multicolumn{3}{c}{Children} & \multicolumn{3}{c}{Adolescents} \\
                    &\multicolumn{1}{c}{None}&\multicolumn{1}{c}{Other}&\multicolumn{1}{c}{Municipal}&\multicolumn{1}{c}{None}&\multicolumn{1}{c}{Other}&\multicolumn{1}{c}{Municipal}\\
\midrule
Male                &       -0.01         &        0.00         &        0.00         &        0.02         &        0.01         &       -0.03         \\
                    &      (0.01)         &      (0.05)         &      (0.05)         &      (0.02)         &      (0.06)         &      (0.06)         \\
\addlinespace
Low birthweight     &       -0.18         &       -0.13         &        0.31         &       -0.01         &       -0.10         &        0.11         \\
                    &     (43.39)         &     (20.93)         &     (22.46)         &      (0.05)         &      (0.15)         &      (0.15)         \\
\addlinespace
Premature birth     &       -0.17         &        0.05         &        0.12         &        0.01         &        0.21         &       -0.22         \\
                    &     (36.10)         &     (17.41)         &     (18.69)         &      (0.05)         &      (0.14)         &      (0.14)         \\
\addlinespace
Mom born in province&        0.18         &       -0.12         &       -0.06         &        0.02         &       -0.17\sym{**} &        0.15\sym{*}  \\
                    &     (23.41)         &     (11.29)         &     (12.12)         &      (0.02)         &      (0.06)         &      (0.06)         \\
\addlinespace
Dad born in province&        0.17         &       -0.09         &       -0.09         &       -0.03         &        0.02         &        0.00         \\
                    &     (15.96)         &      (7.70)         &      (8.26)         &      (0.02)         &      (0.06)         &      (0.06)         \\
\addlinespace
Mom Max Edu: University&       -0.18         &        0.05         &        0.12         &       -0.89\sym{**} &        0.58\sym{***}&        0.31         \\
                    &     (26.20)         &     (12.64)         &     (13.56)         &      (0.30)         &      (0.15)         &      (0.19)         \\
\addlinespace
Dad Max Edu: University&       -0.32         &        0.15         &        0.17         &        0.04         &       -0.16\sym{*}  &        0.12         \\
                    &     (27.95)         &     (13.48)         &     (14.47)         &      (0.02)         &      (0.08)         &      (0.08)         \\
\addlinespace
Has 2 siblings      &        0.02         &        0.11         &       -0.13         &       -0.01         &        0.01         &       -0.00         \\
                    &      (0.01)         &      (0.07)         &      (0.07)         &      (0.03)         &      (0.07)         &      (0.07)         \\
\addlinespace
Has more than 2 siblings&        0.02         &       -0.12         &        0.10         &        0.02         &       -0.06         &        0.04         \\
                    &      (0.02)         &      (0.09)         &      (0.09)         &      (0.02)         &      (0.10)         &      (0.10)         \\
\addlinespace
Caregiver is Catholic&        0.33         &       -0.02         &       -0.30         &       -0.05         &        0.25         &       -0.20         \\
                    &     (43.83)         &     (21.14)         &     (22.69)         &      (0.04)         &      (0.30)         &      (0.30)         \\
\addlinespace
Caregiver is Muslim &        0.32         &       -0.10         &       -0.22         &       -0.89\sym{**} &        0.56         &        0.33         \\
                    &     (43.83)         &     (21.14)         &     (22.69)         &      (0.30)         &      (0.45)         &      (0.45)         \\
\addlinespace
Caregiver is religious&       -0.35         &        0.22         &        0.13         &        0.05         &        0.01         &       -0.06         \\
                    &     (43.83)         &     (21.14)         &     (22.69)         &      (0.04)         &      (0.31)         &      (0.31)         \\
\addlinespace
Mother: born outside of Italy&        0.36         &       -0.11         &       -0.25         &        0.26\sym{**} &        4.83\sym{***}&       -5.09\sym{***}\\
                    &     (28.34)         &     (13.67)         &     (14.67)         &      (0.10)         &      (0.39)         &      (0.39)         \\
\midrule
Observations        &         421         &         421         &         421         &         300         &         300         &         300         \\
\bottomrule
\end{tabular}
}


\begin{tablenotes}
\footnotesize\raggedright{Note: This table shows the results from a multinomial logit that uses baseline characteristics to predict enrollment in municipal preschool, other preschool, or no preschool. The columns titled ``None'' display the marginal effects and standard errors of attending no preschool. Similarly, the columns titled ``Other'' display the same estimates for attending a non-municipal preschool and those titled ``Municipal" display estimates for attending a municipal school. Standard errors are reported in parentheses. Stars show statistical significance as follows: * $p < 0.05$, ** $p < 0.01$, *** $p < 0.001$.}
\end{tablenotes}
\end{threeparttable}
\end{adjustbox}
\end{table}

\begin{table}[H]
\centering
\caption{Multinomial Logit, Adult Cohorts, Reggio Emilia} \label{mlogit-adult-RE}
\begin{adjustbox}{width=\textwidth}
\begin{threeparttable}
{
\def\sym#1{\ifmmode^{#1}\else\(^{#1}\)\fi}
\begin{tabular}{l*{6}{c}}
\toprule
& \multicolumn{3}{c}{Adults 30s} &  \multicolumn{3}{c}{Adults 40s} \\
                    &\multicolumn{1}{c}{None}&\multicolumn{1}{c}{Other}&\multicolumn{1}{c}{Municipal}&\multicolumn{1}{c}{None}&\multicolumn{1}{c}{Other}&\multicolumn{1}{c}{Municipal}\\
\midrule
Male                &     -0.0696         &     -0.0375         &       0.107         &     -0.0468         &      0.0196         &      0.0272         \\
                    &    (0.0456)         &    (0.0533)         &    (0.0587)         &    (0.0482)         &    (0.0528)         &    (0.0559)         \\
\addlinespace
Mom born in province&      0.0545         &     -0.0124         &     -0.0421         &      -0.246\sym{***}&      0.0262         &       0.220\sym{**} \\
                    &    (0.0696)         &    (0.0754)         &    (0.0855)         &    (0.0479)         &    (0.0749)         &    (0.0823)         \\
\addlinespace
Dad born in province&     -0.0119         &     -0.0337         &      0.0456         &     -0.0796         &     -0.0632         &       0.143\sym{*}  \\
                    &    (0.0716)         &    (0.0804)         &    (0.0926)         &    (0.0551)         &    (0.0649)         &    (0.0721)         \\
\addlinespace
Has 2 siblings      &      0.0537         &    -0.00344         &     -0.0502         &      0.0563         &     -0.0387         &     -0.0176         \\
                    &    (0.0544)         &    (0.0682)         &    (0.0746)         &    (0.0587)         &    (0.0682)         &    (0.0716)         \\
\addlinespace
Has more than 2 siblings&      0.0789         &    -0.00920         &     -0.0697         &      0.0373         &     -0.0354         &    -0.00190         \\
                    &    (0.0657)         &    (0.0890)         &    (0.0975)         &    (0.0637)         &    (0.0746)         &    (0.0791)         \\
\addlinespace
Caregiver was religious&       0.108\sym{*}  &      0.0614         &      -0.169\sym{**} &      0.0640         &      0.0568         &      -0.121\sym{*}  \\
                    &    (0.0491)         &    (0.0547)         &    (0.0587)         &    (0.0500)         &    (0.0551)         &    (0.0567)         \\
\addlinespace
Mom Max Edu: Middle School&      -1.120         &      -2.337         &       3.457         &      -1.414         &       2.829         &      -1.415         \\
                    &     (853.8)         &     (526.0)         &     (928.7)         &     (57.44)         &     (160.9)         &     (103.5)         \\
\addlinespace
Mom Max Edu: High School&       0.893         &      -3.002         &       2.109         &      -1.147         &       2.772         &      -1.626         \\
                    &     (844.7)         &     (524.3)         &     (925.1)         &     (57.44)         &     (160.9)         &     (103.5)         \\
\addlinespace
Mom Max Edu: University&       0.997         &      -3.134         &       2.137         &      -1.099         &       2.766         &      -1.667         \\
                    &     (844.7)         &     (524.3)         &     (925.1)         &     (57.44)         &     (160.9)         &     (103.5)         \\
\midrule
Observations        &         280         &         280         &         280         &         285         &         285         &         285         \\
\bottomrule
\end{tabular}
}

\begin{tablenotes}
\footnotesize\raggedright{Note: This table shows the results from a multinomial logit that uses baseline characteristics to predict enrollment in municipal preschool, other preschool, or no preschool. The columns titled ``None'' display the marginal effects and standard errors of attending no preschool. Similarly, the columns titled ``Other'' display the same estimates for attending a non-municipal preschool and those titled ``Municipal" display estimates for attending a municipal school. Standard errors are reported in parentheses. Stars show statistical significance as follows: * $p < 0.05$, ** $p < 0.01$, *** $p < 0.001$.}
\end{tablenotes}
\end{threeparttable}
\end{adjustbox}
\end{table}

\begin{table}[H]
\centering
\caption{Multinomial Logit, Child and Adolescent Cohorts, Parma} \label{mlogit-chi-ado-PR}
\begin{adjustbox}{width=\textwidth}
\begin{threeparttable}
{
\def\sym#1{\ifmmode^{#1}\else\(^{#1}\)\fi}
\begin{tabular}{l*{6}{c}}
\toprule
& \multicolumn{3}{c}{Children} & \multicolumn{3}{c}{Adolescents} \\
                    &\multicolumn{1}{c}{None}&\multicolumn{1}{c}{Other}&\multicolumn{1}{c}{Municipal}&\multicolumn{1}{c}{None}&\multicolumn{1}{c}{Other}&\multicolumn{1}{c}{Municipal}\\
\midrule
Male                &       0.019         &      -0.031         &       0.013         &       0.043         &       0.126\sym{*}  &      -0.168\sym{***}\\
                    &     (0.019)         &     (0.040)         &     (0.036)         &     (0.029)         &     (0.055)         &     (0.048)         \\
\addlinespace
Low birthweight     &      -0.231         &       2.066         &      -1.835         &       0.191         &      -0.223         &       0.032         \\
                    &    (91.650)         &   (330.847)         &   (331.171)         &    (12.003)         &    (10.753)         &     (1.257)         \\
\addlinespace
Premature birth     &      -0.290         &       0.229         &       0.061         &      -0.142         &       0.290         &      -0.148         \\
                    &    (84.928)         &    (71.780)         &    (13.150)         &    (12.003)         &    (10.753)         &     (1.257)         \\
\addlinespace
Mom born in province&       0.021         &      -0.063         &       0.042         &      -0.022         &      -0.037         &       0.059         \\
                    &     (0.021)         &     (0.048)         &     (0.045)         &     (0.016)         &     (0.052)         &     (0.050)         \\
\addlinespace
Dad born in province&      -0.045         &       0.026         &       0.019         &      -0.394         &       0.508         &      -0.114         \\
                    &     (0.026)         &     (0.048)         &     (0.043)         &   (827.426)         &   (741.203)         &    (86.210)         \\
\addlinespace
Mother max. edu.: less than middle school&      -0.054         &       0.134         &      -0.080         &      -0.229         &       0.241         &      -0.011         \\
                    &     (0.031)         &     (0.084)         &     (0.080)         &  (1869.958)         &  (1675.097)         &   (194.833)         \\
\addlinespace
Mom Max Edu: Middle School&      -0.291         &       0.232         &       0.059         &      -0.253         &       0.372         &      -0.119         \\
                    &    (99.555)         &    (84.143)         &    (15.415)         &  (2106.111)         &  (1886.641)         &   (219.438)         \\
\addlinespace
Mom Max Edu: High School&      -0.022         &       0.040         &      -0.018         &       0.025         &       0.027         &      -0.052         \\
                    &     (0.021)         &     (0.044)         &     (0.041)         &     (0.017)         &     (0.054)         &     (0.052)         \\
\addlinespace
Father max. edu.: less than middle school&       0.041         &       0.028         &      -0.069         &      -0.275         &       0.075         &       0.200         \\
                    &     (0.026)         &     (0.079)         &     (0.076)         &  (1965.284)         &  (1760.489)         &   (204.765)         \\
\addlinespace
Father max. edu.: middle school&      -0.273         &       0.311         &      -0.038         &      -0.213         &      -0.024         &       0.237         \\
                    &    (84.300)         &    (71.250)         &    (13.053)         &  (3720.300)         &  (3332.622)         &   (387.622)         \\
\addlinespace
Father max. edu.: high school&       0.032         &      -0.079         &       0.047         &      -0.007         &      -0.054         &       0.061         \\
                    &     (0.022)         &     (0.044)         &     (0.040)         &     (0.022)         &     (0.062)         &     (0.059)         \\
\addlinespace
Has 2 siblings      &      -0.006         &      -0.032         &       0.037         &       0.037         &       0.122         &      -0.159\sym{*}  \\
                    &     (0.021)         &     (0.046)         &     (0.042)         &     (0.019)         &     (0.077)         &     (0.075)         \\
\addlinespace
Has more than 2 siblings&      -0.351         &       0.131         &       0.219         &      -0.156         &       0.255         &      -0.099         \\
                    &    (53.325)         &    (45.070)         &     (8.257)         &  (4308.627)         &  (3859.641)         &   (448.920)         \\
\addlinespace
Caregiver is Catholic&      -0.011         &       0.032         &      -0.022         &       0.110         &      -2.037         &       1.927         \\
                    &     (0.031)         &     (0.087)         &     (0.083)         &   (108.173)         &    (96.901)         &    (11.273)         \\
\addlinespace
Caregiver is Muslim&       0.080\sym{*}  &      -0.136         &       0.056         &                &                &                \\
                    &     (0.032)         &     (0.106)         &     (0.103)         &                 &                 &               \\
\addlinespace
Caregiver was religious&      -0.008         &      -0.091         &       0.098         &      -0.130         &       2.081         &      -1.951         \\
                    &     (0.037)         &     (0.109)         &     (0.105)         &   (108.173)         &    (96.901)         &    (11.272)         \\
\addlinespace
Mother: born outside of Italy&      -0.003         &       0.009         &      -0.006         &      -0.307         &       3.232         &      -2.925         \\
                    &     (0.025)         &     (0.079)         &     (0.076)         & (51159.733)         & (68562.821)         & (51716.262)         \\
\midrule
Observations        &         349         &         349         &         349         &         254         &         254         &         254         \\
\bottomrule
\end{tabular}
}

\begin{tablenotes}
\footnotesize\raggedright{Note: This table shows the results from a multinomial logit that uses baseline characteristics to predict enrollment in municipal preschool, other preschool, or no preschool. The columns titled ``None'' display the marginal effects and standard errors of attending no preschool. Similarly, the columns titled ``Other'' display the same estimates for attending a non-municipal preschool and those titled ``Municipal" display estimates for attending a municipal school. Standard errors are reported in parentheses. Stars show statistical significance as follows: * $p < 0.05$, ** $p < 0.01$, *** $p < 0.001$.}
\end{tablenotes}
\end{threeparttable}
\end{adjustbox}
\end{table}

\begin{table}[H]
\centering
\caption{Multinomial Logit, Adult Cohorts, Parma} \label{mlogit-adult-PR}
\begin{threeparttable}
{
\def\sym#1{\ifmmode^{#1}\else\(^{#1}\)\fi}
\begin{tabular}{l*{3}{c}}
\toprule
& \multicolumn{3}{c}{Adults 30s} \\
                    &\multicolumn{1}{c}{None}&\multicolumn{1}{c}{Other}&\multicolumn{1}{c}{Municipal}\\
\midrule
Male                &       0.012         &      -0.101         &       0.089         \\
                    &     (0.047)         &     (0.060)         &     (0.054)         \\
\addlinespace
Mom born in province&       0.061         &      -0.026         &      -0.035         \\
                    &     (0.055)         &     (0.069)         &     (0.060)         \\
\addlinespace
Dad born in province&       0.046         &      -0.039         &      -0.008         \\
                    &     (0.060)         &     (0.077)         &     (0.067)         \\
\addlinespace
Has 2 siblings      &       0.119\sym{*}  &      -0.078         &      -0.041         \\
                    &     (0.057)         &     (0.072)         &     (0.063)         \\
\addlinespace
Has more than 2 siblings&       0.195\sym{***}&      -0.186\sym{*}  &      -0.009         \\
                    &     (0.058)         &     (0.079)         &     (0.070)         \\
\addlinespace
Caregiver was religious&       0.074         &      -0.092         &       0.019         \\
                    &     (0.061)         &     (0.072)         &     (0.062)         \\
\addlinespace
Mom Max Edu: Middle School&       0.058         &      -0.142         &       0.084         \\
                    &     (0.092)         &     (0.120)         &     (0.097)         \\
\addlinespace
Mom Max Edu: High School&       0.104\sym{*}  &      -0.057         &      -0.047         \\
                    &     (0.048)         &     (0.068)         &     (0.061)         \\
\addlinespace
Mom Max Edu: University&               &              &               \\
                    &                 &                  &                 \\
\midrule
Observations        &         251         &         251         &         251         \\
\bottomrule
\end{tabular}
}

\begin{tablenotes}
\footnotesize\raggedright{Note: This table shows the results from a multinomial logit that uses baseline characteristics to predict enrollment in municipal preschool, other preschool, or no preschool. The columns titled ``None'' display the marginal effects and standard errors of attending no preschool. Similarly, the columns titled ``Other'' display the same estimates for attending a non-municipal preschool and those titled ``Municipal" display estimates for attending a municipal school. Standard errors are reported in parentheses. Stars show statistical significance as follows: * $p < 0.05$, ** $p < 0.01$, *** $p < 0.001$.}
\end{tablenotes}
\end{threeparttable}
\end{table}

\begin{table}[H]
\centering
\caption{Multinomial Logit, Child and Adolescent Cohorts, Padova} \label{mlogit-chi-ado-PD}
\begin{adjustbox}{width=\textwidth}
\begin{threeparttable}
{
\def\sym#1{\ifmmode^{#1}\else\(^{#1}\)\fi}
\begin{tabular}{l*{6}{c}}
\toprule
                    &\multicolumn{1}{c}{None}&\multicolumn{1}{c}{Other}&\multicolumn{1}{c}{Municipal}&\multicolumn{1}{c}{None}&\multicolumn{1}{c}{Other}&\multicolumn{1}{c}{Municipal}\\
\midrule
Male                &        0.01         &       -0.04         &        0.03         &        0.00         &        0.02         &       -0.02         \\
                    &      (0.02)         &      (0.04)         &      (0.04)         &      (0.00)         &      (0.08)         &      (0.05)         \\
\addlinespace
Low birthweight     &       -0.24         &        0.02         &        0.22         &        0.00         &       -0.14         &        0.14         \\
                    &     (47.43)         &     (32.33)         &     (15.11)         &      (0.00)         &      (0.14)         &      (0.13)         \\
\addlinespace
Premature birth     &       -0.24         &        0.23         &        0.01         &        0.00         &       -0.09         &        0.09         \\
                    &     (44.14)         &     (30.09)         &     (14.06)         &      (0.00)         &      (0.12)         &      (0.11)         \\
\addlinespace
Mom at least uni. &       -0.02         &       -0.00         &        0.02         &       -0.00         &        0.01         &       -0.01         \\
                    &      (0.02)         &      (0.05)         &      (0.05)         &      (0.00)         &      (0.09)         &      (0.06)         \\
\addlinespace
Income more than 50,000 &        0.02         &       -0.01         &       -0.01         &        0.00         &        0.10         &       -0.10         \\
                    &      (0.03)         &      (0.07)         &      (0.07)         &      (0.00)         &      (0.14)         &      (0.09)         \\
\addlinespace
Caregiver is Catholic&       -0.25         &        0.13         &        0.12         &        0.00         &       -0.13         &        0.13         \\
                    &     (25.97)         &     (17.70)         &      (8.28)         &      (0.00)         &      (0.09)         &      (0.07)         \\
\addlinespace
Caregiver is Catholic and very faithful&        0.24         &       -0.08         &       -0.16         &       -0.00         &        0.05         &       -0.05         \\
                    &     (25.97)         &     (17.70)         &      (8.28)         &      (0.00)         &      (0.08)         &      (0.06)         \\
\addlinespace
Mom born in province&       -0.26         &        0.26         &       -0.01         &        0.00         &        0.12         &       -0.12         \\
                    &     (20.73)         &     (14.13)         &      (6.60)         &      (0.00)         &      (0.10)         &      (0.06)         \\
\addlinespace
Migrant         &        0.01         &       -0.02         &        0.01         &                     &                     &                     \\
                    &      (0.02)         &      (0.07)         &      (0.06)         &                     &                     &                     \\
\addlinespace
Has 2 siblings      &        0.01         &       -0.14\sym{**} &        0.13\sym{*}  &       -0.00         &        0.06         &       -0.06         \\
                    &      (0.02)         &      (0.05)         &      (0.05)         &      (0.00)         &      (0.11)         &      (0.08)         \\
\addlinespace
Has more than 2 siblings&        0.01         &       -0.18\sym{*}  &        0.18\sym{*}  &        0.00         &        0.17         &       -0.17         \\
                    &      (0.02)         &      (0.07)         &      (0.07)         &      (0.00)         &      (0.37)         &      (0.23)         \\
\midrule
Observations        &         391         &         391         &         391         &         282         &         282         &         282         \\
\bottomrule
\end{tabular}
}

\begin{tablenotes}
\footnotesize\raggedright{Note: This table shows the results from a multinomial logit that uses baseline characteristics to predict enrollment in municipal preschool, other preschool, or no preschool. The columns titled ``None'' display the marginal effects and standard errors of attending no preschool. Similarly, the columns titled ``Other'' display the same estimates for attending a non-municipal preschool and those titled ``Municipal" display estimates for attending a municipal school. Standard errors are reported in parentheses. Stars show statistical significance as follows: * $p < 0.05$, ** $p < 0.01$, *** $p < 0.001$.}
\end{tablenotes}
\end{threeparttable}
\end{adjustbox}
\end{table}


\begin{table}[H]
\centering
\caption{Multinomial Logit, Adult Cohorts, Padova} \label{mlogit-adult-PD}
\begin{threeparttable}
{
\def\sym#1{\ifmmode^{#1}\else\(^{#1}\)\fi}
\begin{tabular}{l*{3}{c}}
\toprule
	& \multicolumn{3}{c}{Adults 30s} \\
                    &\multicolumn{1}{c}{None}&\multicolumn{1}{c}{Other}&\multicolumn{1}{c}{Municipal}\\
\midrule
Male                &       0.124\sym{*}  &      -0.116\sym{*}  &    -0.00847         \\
                    &    (0.0500)         &    (0.0574)         &    (0.0387)         \\
\addlinespace
Mom born in province&    0.000238         &      0.0210         &     -0.0213         \\
                    &    (0.0540)         &    (0.0634)         &    (0.0429)         \\
\addlinespace
Dad born in province&      0.0838         &     -0.0932         &     0.00938         \\
                    &    (0.0640)         &    (0.0711)         &    (0.0470)         \\
\addlinespace
Has 2 siblings      &       0.140\sym{**} &      -0.161\sym{*}  &      0.0214         \\
                    &    (0.0535)         &    (0.0636)         &    (0.0445)         \\
\addlinespace
Has more than 2 siblings&       0.109         &     -0.0951         &     -0.0142         \\
                    &    (0.0627)         &    (0.0766)         &    (0.0565)         \\
\addlinespace
Caregiver was religious&      0.0563         &     -0.0250         &     -0.0312         \\
                    &    (0.0571)         &    (0.0652)         &    (0.0423)         \\
\addlinespace
Mom Max Edu: Middle School&       1.634         &      -1.276         &      -0.357         \\
                    &     (66.89)         &     (57.64)         &     (9.249)         \\
\addlinespace
Mom Max Edu: High School&       1.696         &      -1.370         &      -0.326         \\
                    &     (66.89)         &     (57.64)         &     (9.248)         \\
\addlinespace
Mom Max Edu: University&       1.641         &      -1.303         &      -0.338         \\
                    &     (66.89)         &     (57.64)         &     (9.248)         \\
\midrule
Observations        &         251         &         251         &         251         \\
\bottomrule
\end{tabular}
}
\begin{tablenotes}
\footnotesize\raggedright{Note: This table shows the results from a multinomial logit that uses baseline characteristics to predict enrollment in municipal preschool, other preschool, or no preschool. The columns titled ``None'' display the marginal effects and standard errors of attending no preschool. Similarly, the columns titled ``Other'' display the same estimates for attending a non-municipal preschool and those titled ``Municipal" display estimates for attending a municipal school. Standard errors are reported in parentheses. Stars show statistical significance as follows: * $p < 0.05$, ** $p < 0.01$, *** $p < 0.001$.}
\end{tablenotes}
\end{threeparttable}
\end{table}

\begin{table}[H]
\centering
\caption{Multinomial Logit, Child and Adolescent Cohorts,  All Cities} \label{mlogit_all_chi-ado}
\begin{threeparttable}
{
\def\sym#1{\ifmmode^{#1}\else\(^{#1}\)\fi}
\begin{tabular}{l*{6}{c}}
\toprule
                    &\multicolumn{1}{c}{None}&\multicolumn{1}{c}{Other}&\multicolumn{1}{c}{Municipal}&\multicolumn{1}{c}{None}&\multicolumn{1}{c}{Other}&\multicolumn{1}{c}{Municipal}\\
\midrule
City                &        0.00         &        0.13\sym{***}&       -0.13\sym{***}&       -0.01         &        0.13\sym{***}&       -0.12\sym{***}\\
                    &      (0.00)         &      (0.02)         &      (0.02)         &      (0.01)         &      (0.02)         &      (0.02)         \\
\addlinespace
Male                &        0.01         &        0.00         &       -0.01         &        0.02         &        0.02         &       -0.03         \\
                    &      (0.01)         &      (0.03)         &      (0.03)         &      (0.01)         &      (0.03)         &      (0.03)         \\
\addlinespace
Low birthweight     &       -0.22         &        0.03         &        0.19         &        0.01         &       -0.14         &        0.12         \\
                    &     (10.59)         &      (6.50)         &      (4.09)         &      (0.02)         &      (0.09)         &      (0.09)         \\
\addlinespace
Premature birth     &       -0.22         &        0.18         &        0.05         &        0.01         &       -0.01         &       -0.00         \\
                    &     (10.69)         &      (6.56)         &      (4.13)         &      (0.02)         &      (0.07)         &      (0.07)         \\
\addlinespace
Mom born in province&       -0.01         &        0.02         &       -0.01         &       -0.00         &       -0.03         &        0.04         \\
                    &      (0.01)         &      (0.04)         &      (0.04)         &      (0.01)         &      (0.04)         &      (0.04)         \\
\addlinespace
Dad born in province&       -0.02         &        0.01         &        0.01         &       -0.02         &        0.07\sym{*}  &       -0.06         \\
                    &      (0.01)         &      (0.04)         &      (0.03)         &      (0.01)         &      (0.04)         &      (0.04)         \\
\addlinespace
Mom Max Edu: University&        0.00         &       -0.04         &        0.04         &       -0.03         &        0.02         &        0.01         \\
                    &      (0.01)         &      (0.03)         &      (0.03)         &      (0.02)         &      (0.04)         &      (0.04)         \\
\addlinespace
Father max. edu.: university&       -0.00         &        0.03         &       -0.02         &        0.02         &       -0.01         &       -0.01         \\
                    &      (0.01)         &      (0.04)         &      (0.04)         &      (0.01)         &      (0.04)         &      (0.04)         \\
\addlinespace
Has 2 siblings      &        0.01         &       -0.02         &        0.02         &        0.01         &        0.06         &       -0.06         \\
                    &      (0.01)         &      (0.04)         &      (0.04)         &      (0.01)         &      (0.04)         &      (0.04)         \\
\addlinespace
Has more than 2 siblings&       -0.00         &       -0.12\sym{*}  &        0.12\sym{*}  &        0.01         &        0.01         &       -0.02         \\
                    &      (0.01)         &      (0.05)         &      (0.05)         &      (0.01)         &      (0.07)         &      (0.07)         \\
\addlinespace
Caregiver is Catholic&       -0.02         &        0.08         &       -0.06         &       -0.02         &        0.19         &       -0.17         \\
                    &      (0.02)         &      (0.06)         &      (0.06)         &      (0.02)         &      (0.18)         &      (0.18)         \\
\addlinespace
dv: Caregiver's religion is Islam&        0.01         &        0.04         &       -0.05         &       -0.32         &        0.52         &       -0.21         \\
                    &      (0.01)         &      (0.06)         &      (0.06)         &   (5378.57)         &   (2970.82)         &   (2408.61)         \\
\addlinespace
Caregiver was religious&       -0.00         &       -0.00         &        0.00         &        0.02         &       -0.14         &        0.11         \\
                    &      (0.02)         &      (0.07)         &      (0.06)         &      (0.02)         &      (0.19)         &      (0.19)         \\
\addlinespace
Mother: born outside of Italy&       -0.00         &        0.01         &       -0.00         &        0.14\sym{***}&        5.07\sym{***}&       -5.21\sym{***}\\
                    &      (0.01)         &      (0.05)         &      (0.05)         &      (0.04)         &      (0.28)         &      (0.28)         \\
\midrule
Observations        &        1161         &        1161         &        1161         &         836         &         836         &         836         \\
\bottomrule
\end{tabular}
}

\begin{tablenotes}
\footnotesize\raggedright{Note: This table shows the results from a multinomial logit that uses baseline characteristics to predict enrollment in municipal preschool, other preschool, or no preschool. The columns titled ``None'' display the marginal effects and standard errors of attending no preschool. Similarly, the columns titled ``Other'' display the same estimates for attending a non-municipal preschool and those titled ``Municipal" display estimates for attending a municipal school. Standard errors are reported in parentheses. Stars show statistical significance as follows: * $p < 0.05$, ** $p < 0.01$, *** $p < 0.001$.}
\end{tablenotes}
\end{threeparttable}
\end{table}


\begin{table}[H]
\centering
\caption{Multinomial Logit, Adult Cohorts,  All Cities} \label{mlogit_all}
\begin{threeparttable}
{
\def\sym#1{\ifmmode^{#1}\else\(^{#1}\)\fi}
\begin{tabular}{l*{6}{c}}
\toprule
                    &\multicolumn{1}{c}{None}&\multicolumn{1}{c}{Other}&\multicolumn{1}{c}{Municipal}&\multicolumn{1}{c}{None}&\multicolumn{1}{c}{Other}&\multicolumn{1}{c}{Municipal}\\
\midrule
City                &       -0.01         &        0.20\sym{***}&       -0.19\sym{***}&        0.49         &        0.83         &       -1.32         \\
                    &      (0.02)         &      (0.02)         &      (0.02)         &     (24.73)         &     (33.93)         &     (58.66)         \\
\addlinespace
Male                &        0.02         &       -0.06         &        0.05         &        0.01         &       -0.02         &        0.01         \\
                    &      (0.03)         &      (0.03)         &      (0.03)         &      (0.03)         &      (0.03)         &      (0.02)         \\
\addlinespace
Mom born in province&        0.04         &        0.03         &       -0.07\sym{*}  &       -0.12\sym{**} &        0.04         &        0.07\sym{*}  \\
                    &      (0.03)         &      (0.04)         &      (0.04)         &      (0.04)         &      (0.04)         &      (0.03)         \\
\addlinespace
Dad born in province&        0.05         &       -0.04         &       -0.01         &       -0.00         &       -0.01         &        0.01         \\
                    &      (0.04)         &      (0.04)         &      (0.04)         &      (0.04)         &      (0.04)         &      (0.03)         \\
\addlinespace
Has 2 siblings      &        0.11\sym{***}&       -0.13\sym{***}&        0.02         &        0.12\sym{**} &       -0.11\sym{**} &       -0.01         \\
                    &      (0.03)         &      (0.04)         &      (0.04)         &      (0.04)         &      (0.04)         &      (0.03)         \\
\addlinespace
Has more than 2 siblings&        0.14\sym{***}&       -0.15\sym{**} &        0.01         &        0.14\sym{***}&       -0.16\sym{***}&        0.02         \\
                    &      (0.04)         &      (0.05)         &      (0.04)         &      (0.04)         &      (0.04)         &      (0.03)         \\
\addlinespace
Caregiver was religious&        0.09\sym{**} &       -0.02         &       -0.07\sym{*}  &        0.05         &       -0.03         &       -0.03         \\
                    &      (0.03)         &      (0.04)         &      (0.03)         &      (0.04)         &      (0.04)         &      (0.02)         \\
\addlinespace
Mom Max Edu: Middle School&        1.78         &       -1.11         &       -0.67         &       -0.21         &        0.21         &        0.00         \\
                    &     (88.59)         &     (52.62)         &     (35.97)         &      (0.13)         &      (0.14)         &      (0.06)         \\
\addlinespace
Mom Max Edu: High School&        1.84         &       -1.11         &       -0.73         &       -0.12         &        0.15         &       -0.03         \\
                    &     (88.59)         &     (52.62)         &     (35.97)         &      (0.13)         &      (0.14)         &      (0.06)         \\
\addlinespace
Mom Max Edu: University&        1.82         &       -1.14         &       -0.68         &       -0.08         &        0.12         &       -0.05         \\
                    &     (88.59)         &     (52.62)         &     (35.97)         &      (0.13)         &      (0.14)         &      (0.06)         \\
\midrule
Observations        &         782         &         782         &         782         &         791         &         791         &         791         \\
\bottomrule
\end{tabular}
}

\begin{tablenotes}
\footnotesize\raggedright{Note: This table shows the results from a multinomial logit that uses baseline characteristics to predict enrollment in municipal preschool, other preschool, or no preschool. The columns titled ``None'' display the marginal effects and standard errors of attending no preschool. Similarly, the columns titled ``Other'' display the same estimates for attending a non-municipal preschool and those titled ``Municipal" display estimates for attending a municipal school. Standard errors are reported in parentheses. Stars show statistical significance as follows: * $p < 0.05$, ** $p < 0.01$, *** $p < 0.001$.}
\end{tablenotes}
\end{threeparttable}
\end{table}

