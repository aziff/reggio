\subsection{Differences in Selection across Cities}
We estimate multinomial logit models for each city using choice of preschool type as the dependent variable, and a vector of baseline characteristics as independent variables. We then compare the estimated marginal effects for each baseline characteristic to investigate whether a common set of baseline characteristics determines selection into municipal preschools across the three cities. This analysis highlights characteristics that determine selection into the municipal schools relative to preschool alternatives available within the same city. Substantial differences in selection across cities would suggest that selection into municipal schools differ by city. This could be a result of different parental perceptions of quality, the available slots, or other factors determining demand for municipal schools relative to alternative options.

We define the choices available to parents in each city as follows:
\begin{equation}
S_i = \quad
\begin{cases}
0 \quad \text{if \quad (\textit{i} did not attend any preschool)} \\
1 \quad  \text{if \quad (\textit{i} attended an alternative preschool)} \\
2 \quad \text{if \quad (\textit{i} attended a municipal preschool)}  \\
\end{cases}
\end{equation}

\noindent We then estimate the following multinomial logit specification separately for each city:

\begin{equation} \label{eq:log-odds-ratio}
\nu_r = \text{log} \left(\frac{\Pr(S_i=r)}{\Pr(S_i=2)}\right) = \beta_0^r +  \bm{X}_i\bm{\beta}^r + \epsilon_i^r \quad \forall \quad r \in \{0,1\}
\end{equation}

\noindent where $\bm{X}_i$ is a vector of baseline characteristics including child's gender, number of child's siblings, religiosity of caregiver, mothers education level, and indicators for whether parents were born in province.

The predicted log-odds ratio for each alternative option, $\hat{\nu}_r$, from (\ref{eq:log-odds-ratio}) can be used to construct the estimated probability of attending each school type $r$, $\hat{\pi}_{i,r}$, as follows:

\begin{equation}
\hat{\pi}_{i,r} = \frac{\text{exp}(\hat{\nu}_{r,i})}{\sum\limits_{j=0}^{2} \text{exp}(\hat{\nu}_{j,i})} \quad \forall  r \in \{0,1,2\}.
\end{equation}

Finally, we estimate the impact of each characteristic $x_c$ of $\bm{X}$ on the propensity to attend a school type $r$, $\pi_{r}$, by calculating the partial derivative $\partial \pi_{r}/\partial x_c$. These results are reported in Appendix~\ref{appendix:mlogit}. Comparisons across coefficients across cities are given in Tables~\ref{mlogit_coeff_child} through~\ref{mlogit_coeff_asilo_age40}.

\textbf{[JJH: Group, I asked for comparisons across the \ul{coefficients} of the choice probabilities across cities and an F-test for equality, i.e. $\beta^r$ same in all cities? Test equality.][We have added tables to display the coefficients and test the difference between the interaction terms of Parma and Padova. The differences are much more pronounced when looking at selection into infant-toddler care between the two cities, relative to Reggio Emilia.]}


% ==============================================================================%
\subsection{Estimation Results}\label{appendix:mlogit}
% ==============================================================================%



\begin{table}[H]
\centering
\caption{Multinomial Logit, Child and Adolescent Cohorts, Reggio Emilia} \label{mlogit-chi-ado-RE}
\begin{adjustbox}{width=\textwidth}
\begin{threeparttable}
{
\def\sym#1{\ifmmode^{#1}\else\(^{#1}\)\fi}
\begin{tabular}{l*{6}{c}}
\toprule
			& \multicolumn{3}{c}{Children} & \multicolumn{3}{c}{Adolescents} \\
                    &\multicolumn{1}{c}{None}&\multicolumn{1}{c}{Other}&\multicolumn{1}{c}{Municipal}&\multicolumn{1}{c}{None}&\multicolumn{1}{c}{Other}&\multicolumn{1}{c}{Municipal}\\
\midrule
Male                &       -0.01         &        0.00         &        0.00         &        0.02         &        0.01         &       -0.03         \\
                    &      (0.01)         &      (0.05)         &      (0.05)         &      (0.02)         &      (0.06)         &      (0.06)         \\
\addlinespace
Low birthweight     &       -0.18         &       -0.13         &        0.31         &       -0.01         &       -0.10         &        0.11         \\
                    &     (43.39)         &     (20.93)         &     (22.46)         &      (0.05)         &      (0.15)         &      (0.15)         \\
\addlinespace
Premature birth     &       -0.17         &        0.05         &        0.12         &        0.01         &        0.21         &       -0.22         \\
                    &     (36.10)         &     (17.41)         &     (18.69)         &      (0.05)         &      (0.14)         &      (0.14)         \\
\addlinespace
Mom born in province&        0.18         &       -0.12         &       -0.06         &        0.02         &       -0.17\sym{**} &        0.15\sym{*}  \\
                    &     (23.41)         &     (11.29)         &     (12.12)         &      (0.02)         &      (0.06)         &      (0.06)         \\
\addlinespace
Dad born in province&        0.17         &       -0.09         &       -0.09         &       -0.03         &        0.02         &        0.00         \\
                    &     (15.96)         &      (7.70)         &      (8.26)         &      (0.02)         &      (0.06)         &      (0.06)         \\
\addlinespace
Mom Max Edu: University&       -0.18         &        0.05         &        0.12         &       -0.89\sym{**} &        0.58\sym{***}&        0.31         \\
                    &     (26.20)         &     (12.64)         &     (13.56)         &      (0.30)         &      (0.15)         &      (0.19)         \\
\addlinespace
Dad Max Edu: University&       -0.32         &        0.15         &        0.17         &        0.04         &       -0.16\sym{*}  &        0.12         \\
                    &     (27.95)         &     (13.48)         &     (14.47)         &      (0.02)         &      (0.08)         &      (0.08)         \\
\addlinespace
Has 2 siblings      &        0.02         &        0.11         &       -0.13         &       -0.01         &        0.01         &       -0.00         \\
                    &      (0.01)         &      (0.07)         &      (0.07)         &      (0.03)         &      (0.07)         &      (0.07)         \\
\addlinespace
Has more than 2 siblings&        0.02         &       -0.12         &        0.10         &        0.02         &       -0.06         &        0.04         \\
                    &      (0.02)         &      (0.09)         &      (0.09)         &      (0.02)         &      (0.10)         &      (0.10)         \\
\addlinespace
Caregiver is Catholic&        0.33         &       -0.02         &       -0.30         &       -0.05         &        0.25         &       -0.20         \\
                    &     (43.83)         &     (21.14)         &     (22.69)         &      (0.04)         &      (0.30)         &      (0.30)         \\
\addlinespace
Caregiver is Muslim &        0.32         &       -0.10         &       -0.22         &       -0.89\sym{**} &        0.56         &        0.33         \\
                    &     (43.83)         &     (21.14)         &     (22.69)         &      (0.30)         &      (0.45)         &      (0.45)         \\
\addlinespace
Caregiver is religious&       -0.35         &        0.22         &        0.13         &        0.05         &        0.01         &       -0.06         \\
                    &     (43.83)         &     (21.14)         &     (22.69)         &      (0.04)         &      (0.31)         &      (0.31)         \\
\addlinespace
Mother: born outside of Italy&        0.36         &       -0.11         &       -0.25         &        0.26\sym{**} &        4.83\sym{***}&       -5.09\sym{***}\\
                    &     (28.34)         &     (13.67)         &     (14.67)         &      (0.10)         &      (0.39)         &      (0.39)         \\
\midrule
Observations        &         421         &         421         &         421         &         300         &         300         &         300         \\
\bottomrule
\end{tabular}
}


\begin{tablenotes}
\footnotesize\raggedright{Note: This table shows the marginal effects from a multinomial logit that uses baseline characteristics to predict enrollment in municipal preschool, other preschool, or no preschool. The columns titled ``None'' display the marginal effects and standard errors of attending no preschool. Similarly, the columns titled ``Other'' display the same estimates for attending a non-municipal preschool and those titled ``Municipal" display estimates for attending a municipal school. Standard errors are reported in parentheses. Stars show statistical significance as follows: * $p < 0.05$, ** $p < 0.01$, *** $p < 0.001$.}
\end{tablenotes}
\end{threeparttable}
\end{adjustbox}
\end{table}

\begin{table}[H]
\centering
\caption{Multinomial Logit, Adult Cohorts, Reggio Emilia} \label{mlogit-adult-RE}
\begin{adjustbox}{width=\textwidth}
\begin{threeparttable}
{
\def\sym#1{\ifmmode^{#1}\else\(^{#1}\)\fi}
\begin{tabular}{l*{6}{c}}
\toprule
 &\multicolumn{3}{c}{Adults 30s}&\multicolumn{3}{c}{Adults 40s} \\
                    &\multicolumn{1}{c}{None}&\multicolumn{1}{c}{Other}&\multicolumn{1}{c}{Municipal}&\multicolumn{1}{c}{None}&\multicolumn{1}{c}{Other}&\multicolumn{1}{c}{Municipal}\\
\midrule
Male                &       -0.07         &       -0.04         &        0.11         &       -0.05         &        0.01         &        0.03         \\
                    &      (0.05)         &      (0.05)         &      (0.06)         &      (0.05)         &      (0.05)         &      (0.06)         \\
\addlinespace
Mom born in province&        0.05         &       -0.00         &       -0.05         &       -0.25\sym{***}&        0.04         &        0.21\sym{*}  \\
                    &      (0.07)         &      (0.08)         &      (0.09)         &      (0.05)         &      (0.08)         &      (0.09)         \\
\addlinespace
Dad born in province&       -0.01         &       -0.02         &        0.04         &       -0.08         &       -0.04         &        0.11         \\
                    &      (0.07)         &      (0.08)         &      (0.09)         &      (0.06)         &      (0.07)         &      (0.07)         \\
\addlinespace
Has 2 siblings      &        0.05         &       -0.01         &       -0.04         &        0.06         &       -0.08         &        0.02         \\
                    &      (0.05)         &      (0.07)         &      (0.08)         &      (0.06)         &      (0.07)         &      (0.07)         \\
\addlinespace
Has more than 2 siblings&        0.08         &       -0.06         &       -0.02         &        0.04         &       -0.07         &        0.03         \\
                    &      (0.07)         &      (0.09)         &      (0.10)         &      (0.06)         &      (0.08)         &      (0.08)         \\
\addlinespace
Caregiver was religious&        0.11\sym{*}  &        0.07         &       -0.18\sym{**} &        0.06         &        0.03         &       -0.09         \\
                    &      (0.05)         &      (0.05)         &      (0.06)         &      (0.05)         &      (0.06)         &      (0.06)         \\
\addlinespace
Mom Max Edu: Middle School&       -1.05         &       -2.21         &        3.27         &       -1.38         &        2.84         &       -1.46         \\
                    &    (560.60)         &    (350.42)         &    (615.62)         &     (36.11)         &    (103.47)         &     (67.36)         \\
\addlinespace
Mom Max Edu: High School&        0.85         &       -2.91         &        2.06         &       -1.11         &        2.77         &       -1.65         \\
                    &    (553.74)         &    (349.14)         &    (612.93)         &     (36.11)         &    (103.47)         &     (67.36)         \\
\addlinespace
Mom Max Edu: University&        0.95         &       -3.06         &        2.11         &       -1.07         &        2.75         &       -1.68         \\
                    &    (553.74)         &    (349.14)         &    (612.93)         &     (36.11)         &    (103.47)         &     (67.36)         \\
\midrule
Observations        &         280         &         280         &         280         &         285         &         285         &         285         \\
\bottomrule
\end{tabular}
}

\begin{tablenotes}
\footnotesize\raggedright{Note: This table shows the marginal effects from a multinomial logit that uses baseline characteristics to predict enrollment in municipal preschool, other preschool, or no preschool. The columns titled ``None'' display the marginal effects and standard errors of attending no preschool. Similarly, the columns titled ``Other'' display the same estimates for attending a non-municipal preschool and those titled ``Municipal" display estimates for attending a municipal school. Standard errors are reported in parentheses. Stars show statistical significance as follows: * $p < 0.05$, ** $p < 0.01$, *** $p < 0.001$.}
\end{tablenotes}
\end{threeparttable}
\end{adjustbox}
\end{table}

\begin{table}[H]
\centering
\caption{Multinomial Logit, Child and Adolescent Cohorts, Parma} \label{mlogit-chi-ado-PR}
\begin{adjustbox}{width=\textwidth}
\begin{threeparttable}
{
\def\sym#1{\ifmmode^{#1}\else\(^{#1}\)\fi}
\begin{tabular}{l*{6}{c}}
\toprule
& \multicolumn{3}{c}{Children} & \multicolumn{3}{c}{Adolescents} \\
                    &\multicolumn{1}{c}{None}&\multicolumn{1}{c}{Other}&\multicolumn{1}{c}{Municipal}&\multicolumn{1}{c}{None}&\multicolumn{1}{c}{Other}&\multicolumn{1}{c}{Municipal}\\
\midrule
Male                &        0.02         &       -0.03         &        0.01         &        0.01         &        0.15\sym{**} &       -0.16\sym{***}\\
                    &      (0.02)         &      (0.04)         &      (0.04)         &      (0.02)         &      (0.05)         &      (0.05)         \\
\addlinespace
Low birthweight     &       -0.22         &        2.00         &       -1.78         &        0.21         &       -0.24         &        0.03         \\
                    &     (60.54)         &    (207.66)         &    (207.44)         &     (15.49)         &     (12.97)         &      (2.52)         \\
\addlinespace
Premature birth     &       -0.28         &        0.22         &        0.06         &       -0.19         &        0.33         &       -0.14         \\
                    &     (54.48)         &     (45.90)         &      (8.58)         &     (15.49)         &     (12.97)         &      (2.52)         \\
\addlinespace
Mom born in province&        0.02         &       -0.07         &        0.05         &       -0.02         &       -0.02         &        0.04         \\
                    &      (0.02)         &      (0.05)         &      (0.05)         &      (0.02)         &      (0.05)         &      (0.05)         \\
\addlinespace
Dad born in province&       -0.04         &        0.00         &        0.04         &       -0.37         &        0.43         &       -0.06         \\
                    &      (0.03)         &      (0.05)         &      (0.04)         &     (21.70)         &     (18.18)         &      (3.53)         \\
\addlinespace
Mom Max Edu: University&        0.04         &       -0.06         &        0.02         &       -0.00         &       -0.05         &        0.05         \\
                    &      (0.02)         &      (0.05)         &      (0.04)         &      (0.02)         &      (0.06)         &      (0.05)         \\
\addlinespace
Dad Max Edu: University&       -0.01         &        0.05         &       -0.03         &        0.01         &        0.06         &       -0.08         \\
                    &      (0.02)         &      (0.05)         &      (0.04)         &      (0.02)         &      (0.06)         &      (0.06)         \\
\addlinespace
Has 2 siblings      &       -0.01         &       -0.03         &        0.05         &        0.02         &        0.14         &       -0.17\sym{*}  \\
                    &      (0.02)         &      (0.05)         &      (0.04)         &      (0.02)         &      (0.08)         &      (0.08)         \\
\addlinespace
Has more than 2 siblings&       -0.34         &        0.13         &        0.20         &       -0.17         &        0.24         &       -0.07         \\
                    &     (31.11)         &     (26.21)         &      (4.90)         &     (46.49)         &     (38.94)         &      (7.56)         \\
\addlinespace
Caregiver is Catholic&       -0.02         &        0.03         &       -0.01         &        0.14         &       -2.06         &        1.92         \\
                    &      (0.03)         &      (0.09)         &      (0.08)         &     (92.65)         &    (578.27)         &    (583.95)         \\
\addlinespace
Caregiver is Muslim &        0.08\sym{**} &       -0.15         &        0.07         &                 &                 &                 \\
                    &      (0.03)         &      (0.10)         &      (0.10)         &                  &                  &                  \\
\addlinespace
Caregiver is religious&        0.01         &       -0.10         &        0.09         &       -0.17         &        2.11         &       -1.94         \\
                    &      (0.03)         &      (0.11)         &      (0.10)         &     (92.65)         &    (578.27)         &    (583.95)         \\
\addlinespace
Mom born outside of Italy&       -0.01         &        0.01         &       -0.00         &       -0.21         &        2.07         &       -1.86         \\
                    &      (0.03)         &      (0.08)         &      (0.08)         &    (105.49)         &    (508.01)         &    (509.92)         \\
\midrule
Observations        &         349         &         349         &         349         &         254         &         254         &         254         \\
\bottomrule
\end{tabular}
}

\begin{tablenotes}
\footnotesize\raggedright{Note: This table shows the marginal effects from a multinomial logit that uses baseline characteristics to predict enrollment in municipal preschool, other preschool, or no preschool. The columns titled ``None'' display the marginal effects and standard errors of attending no preschool. Similarly, the columns titled ``Other'' display the same estimates for attending a non-municipal preschool and those titled ``Municipal" display estimates for attending a municipal school. Standard errors are reported in parentheses. Stars show statistical significance as follows: * $p < 0.05$, ** $p < 0.01$, *** $p < 0.001$.}
\end{tablenotes}
\end{threeparttable}
\end{adjustbox}
\end{table}

\begin{table}[H]
\centering
\caption{Multinomial Logit, Adult Cohorts, Parma} \label{mlogit-adult-PR}
\begin{threeparttable}
{
\def\sym#1{\ifmmode^{#1}\else\(^{#1}\)\fi}
\begin{tabular}{l*{3}{c}}
\toprule
                    &\multicolumn{1}{c}{None}&\multicolumn{1}{c}{Other}&\multicolumn{1}{c}{Municipal}\\
\midrule
Male                &        0.01         &       -0.07         &        0.05         \\
                    &      (0.05)         &      (0.06)         &      (0.06)         \\
\addlinespace
Mom up to high school &        0.04         &        0.17         &       -0.22         \\
                    &      (0.09)         &      (0.13)         &      (0.12)         \\
\addlinespace
Mom at least uni.  &       -0.06         &        0.15         &       -0.09         \\
                    &      (0.09)         &      (0.13)         &      (0.11)         \\
\addlinespace
Caregiver was religious&        0.07         &       -0.14\sym{*}  &        0.07         \\
                    &      (0.06)         &      (0.07)         &      (0.07)         \\
\addlinespace
Mom born in province&        0.06         &        0.06         &       -0.11         \\
                    &      (0.05)         &      (0.07)         &      (0.06)         \\
\addlinespace
Dad born in province&        0.05         &        0.03         &       -0.07         \\
                    &      (0.06)         &      (0.08)         &      (0.07)         \\
\addlinespace
Has 2 siblings      &        0.12\sym{*}  &       -0.10         &       -0.02         \\
                    &      (0.06)         &      (0.07)         &      (0.07)         \\
\addlinespace
Has more than 2 siblings&        0.20\sym{***}&       -0.25\sym{**} &        0.05         \\
                    &      (0.06)         &      (0.08)         &      (0.08)         \\
\midrule
Observations        &         251         &         251         &         251         \\
\bottomrule
\end{tabular}
}

\begin{tablenotes}
\footnotesize\raggedright{Note: This table shows the marginal effects from a multinomial logit that uses baseline characteristics to predict enrollment in municipal preschool, other preschool, or no preschool. The columns titled ``None'' display the marginal effects and standard errors of attending no preschool. Similarly, the columns titled ``Other'' display the same estimates for attending a non-municipal preschool and those titled ``Municipal" display estimates for attending a municipal school. Standard errors are reported in parentheses. Stars show statistical significance as follows: * $p < 0.05$, ** $p < 0.01$, *** $p < 0.001$.}
\end{tablenotes}
\end{threeparttable}
\end{table}

\begin{table}[H]
\centering
\caption{Multinomial Logit, Child and Adolescent Cohorts, Padova} \label{mlogit-chi-ado-PD}
\begin{adjustbox}{width=\textwidth}
\begin{threeparttable}
\input{../../output/mlogit_Padova_chi-ado_ready.tex}
\begin{tablenotes}
\footnotesize\raggedright{Note: This table shows the marginal effects from a multinomial logit that uses baseline characteristics to predict enrollment in municipal preschool, other preschool, or no preschool. The columns titled ``None'' display the marginal effects and standard errors of attending no preschool. Similarly, the columns titled ``Other'' display the same estimates for attending a non-municipal preschool and those titled ``Municipal" display estimates for attending a municipal school. Standard errors are reported in parentheses. Stars show statistical significance as follows: * $p < 0.05$, ** $p < 0.01$, *** $p < 0.001$.}
\end{tablenotes}
\end{threeparttable}
\end{adjustbox}
\end{table}


\begin{table}[H]
\centering
\caption{Multinomial Logit, Adult Cohorts, Padova} \label{mlogit-adult-PD}
\begin{threeparttable}
{
\def\sym#1{\ifmmode^{#1}\else\(^{#1}\)\fi}
\begin{tabular}{l*{3}{c}}
\toprule
& \multicolumn{3}{c}{Adults 30s} \\
                    &\multicolumn{1}{c}{None}&\multicolumn{1}{c}{Other}&\multicolumn{1}{c}{Municipal}\\
\midrule
Male                &        0.12\sym{*}  &       -0.12\sym{*}  &       -0.01         \\
                    &      (0.05)         &      (0.06)         &      (0.04)         \\
\addlinespace
Mom born in province&        0.00         &        0.02         &       -0.02         \\
                    &      (0.05)         &      (0.06)         &      (0.04)         \\
\addlinespace
Dad born in province&        0.08         &       -0.09         &        0.01         \\
                    &      (0.06)         &      (0.07)         &      (0.05)         \\
\addlinespace
Has 2 siblings      &        0.14\sym{**} &       -0.16\sym{*}  &        0.02         \\
                    &      (0.05)         &      (0.06)         &      (0.04)         \\
\addlinespace
Has more than 2 siblings&        0.11         &       -0.10         &       -0.01         \\
                    &      (0.06)         &      (0.08)         &      (0.06)         \\
\addlinespace
Caregiver was religious&        0.06         &       -0.03         &       -0.03         \\
                    &      (0.06)         &      (0.07)         &      (0.04)         \\
\addlinespace
Mom Max Edu: Middle School&        1.63         &       -1.28         &       -0.36         \\
                    &     (66.89)         &     (57.64)         &      (9.25)         \\
\addlinespace
Mom Max Edu: High School&        1.70         &       -1.37         &       -0.33         \\
                    &     (66.89)         &     (57.64)         &      (9.25)         \\
\addlinespace
Mom Max Edu: University&        1.64         &       -1.30         &       -0.34         \\
                    &     (66.89)         &     (57.64)         &      (9.25)         \\
\midrule
Observations        &         251         &         251         &         251         \\
\bottomrule
\end{tabular}
}

\begin{tablenotes}
\footnotesize\raggedright{Note: This table shows the marginal effects from a multinomial logit that uses baseline characteristics to predict enrollment in municipal preschool, other preschool, or no preschool. The columns titled ``None'' display the marginal effects and standard errors of attending no preschool. Similarly, the columns titled ``Other'' display the same estimates for attending a non-municipal preschool and those titled ``Municipal" display estimates for attending a municipal school. Standard errors are reported in parentheses. Stars show statistical significance as follows: * $p < 0.05$, ** $p < 0.01$, *** $p < 0.001$.}
\end{tablenotes}
\end{threeparttable}
\end{table}

\begin{table}[H]
\centering
\caption{Multinomial Logit, Child Cohort, Preschool, All Cities} \label{mlogit_coeff_child}
\begin{threeparttable}
\begin{tabular}{l c c c}
\toprule
& None & Other \\
\midrule
Male &    -0.356 &    -0.014 \\
Male $\times$ Parma &     0.871 &     0.271 \\
Male $\times$ Padova &     0.985 &    -0.200 \\
Low birthweight &   -14.463 & -0.923* \\
Low birthweight $\times$ Parma &     1.348 & 1.320* \\
Low birthweight $\times$ Padova &    -0.296 &     0.046 \\
Premature &   -13.632 &    -0.109 \\
Premature $\times$ Parma &    -0.376 &     0.262 \\
Premature $\times$ Padova &    -0.471 &     0.506 \\
Mom at least uni. &   -14.711 &    -0.262 \\
Mom at least uni. $\times$ Parma &    15.616 &     0.213 \\
Mom at least uni. $\times$ Padova &    13.537 &     0.168 \\
Income more than 50,000 &   -14.240 &    -0.056 \\
Income more than 50,000 $\times$ Parma &    -0.899 &     0.273 \\
Income more than 50,000 $\times$ Padova &    15.606 &     0.066 \\
Catholic caregiver &   -15.130 &     0.467 \\
Catholic caregiver $\times$ Parma &    14.466 &    -0.492 \\
Catholic caregiver $\times$ Padova &     0.182 &    -0.703 \\
Relig. Catholic caregiver &    14.849 &     0.354 \\
Relig. Catholic caregiver $\times$ Parma &   -15.962 &    -0.042 \\
Relig. Catholic caregiver $\times$ Padova &    -0.041 &     0.164 \\
Mom born in province &    15.378 &    -0.173 \\
Mom born in province $\times$ Parma &   -16.094 &     0.182 \\
Mom born in province $\times$ Padova &   -30.343 & 0.702* \\
Migrant &    15.398 & 0.516* \\
Migrant $\times$ Parma &   -14.705 & -0.823* \\
Migrant $\times$ Padova &   -14.956 &    -0.607 \\
At least 2 siblings &     1.122 & 0.471* \\
At least 2 siblings $\times$ Parma &    -1.012 & -0.708* \\
At least 2 siblings $\times$ Padova &    -1.140 & -1.250* \\
More than 2 siblings &     0.799 &    -0.448 \\
More than 2 siblings $\times$ Parma &   -16.321 &     0.239 \\
More than 2 siblings $\times$ Padova &    -1.205 &    -0.623 \\
\bottomrule
\end{tabular}
% This mlogit converged? 1
% This file is generated using reggio/script/multinomial-logit/analysis_mlogit_interaction.do

\begin{tablenotes}
\footnotesize\raggedright{Note: This table shows the coefficients from a multinomial logit that uses baseline characteristics to predict enrollment in non-municipal preschool or no preschool, relative to enrollment in municipal preschool. The column titled ``None'' displays the coefficients of attending no preschool. Similarly, the column titled ``Other'' displays the same coefficients for attending a non-municipal preschool. Estimates with a * indicate significance at the 10\% level. Bolded  estimates indicate that the interaction terms of Parma and Padova are significantly different at the 10\% level.}
\end{tablenotes}
\end{threeparttable}
\end{table}


\begin{table}[H]
\centering
\caption{Multinomial Logit, Adolescent Cohort, Preschool,  All Cities} \label{mlogit_coeff_adol}
\begin{threeparttable}
\begin{tabular}{l c c c}
\toprule
& None & Other \\
\midrule
Male &     0.912 &     0.087 \\
Male $\times$ Parma &    -0.604 &     0.124 \\
Male $\times$ Padova &    27.412 &     0.032 \\
Low birthweight &    -0.074 &    -0.562 \\
Low birthweight $\times$ Parma &     2.153 &    -0.096 \\
Low birthweight $\times$ Padova &    27.475 &    -0.139 \\
Premature &     0.327 &     0.866 \\
Premature $\times$ Parma &    -0.718 &    -1.238 \\
Premature $\times$ Padova &    27.075 &    -1.326 \\
Mom at least uni. &   -15.581 &     0.326 \\
Mom at least uni. $\times$ Parma &    16.059 & -0.766* \\
Mom at least uni. $\times$ Padova &     7.737 &    -0.288 \\
Income at least 50,000 &   -24.989 &    -0.276 \\
Income at least 50,000 $\times$ Parma &    -3.126 &     0.435 \\
Income at least 50,000 $\times$ Padova &    22.148 &     0.772 \\
Catholic caregiver &     0.351 & 0.717* \\
Catholic caregiver $\times$ Parma &    -1.555 &    -0.893 \\
Catholic caregiver $\times$ Padova &     1.126 & -1.375* \\
Relig. Catholic caregiver &    -0.258 & 0.759* \\
Relig. Catholic caregiver $\times$ Parma &     0.310 &    -0.560 \\
Relig. Catholic caregiver $\times$ Padova &    -9.333 &    -0.483 \\
Mom born in province &     0.217 & -0.658* \\
Mom born in province $\times$ Parma &    -2.029 &     0.651 \\
Mom born in province $\times$ Padova &    -1.886 & 1.243* \\
Migrant caregiver & 37.675* & 33.271* \\
Migrant caregiver $\times$ Parma &   -45.715 &    -7.621 \\
Migrant caregiver $\times$ Padova & 0.000* & 0.000* \\
At least 2 siblings &    -0.736 &    -0.060 \\
At least 2 siblings $\times$ Parma &     2.818 &     0.663 \\
At least 2 siblings $\times$ Padova &   -31.218 &     0.375 \\
More than 2 siblings &     1.163 &    -0.295 \\
More than 2 siblings $\times$ Parma &   -24.273 &     0.364 \\
More than 2 siblings $\times$ Padova &    19.109 &     1.140 \\
\bottomrule
\end{tabular}
% This mlogit converged? 0
% This file is generated using reggio/script/multinomial-logit/analysis_mlogit_interaction.do

\begin{tablenotes}
\footnotesize\raggedright{Note: This table shows the coefficients from a multinomial logit that uses baseline characteristics to predict enrollment in non-municipal preschool or no preschool, relative to enrollment in municipal preschool. The column titled ``None'' displays the coefficients of attending no preschool. Similarly, the column titled ``Other'' displays the same coefficients for attending a non-municipal preschool. Estimates with a * indicate significance at the 10\% level. Bolded  estimates indicate that the interaction terms of Parma and Padova are significantly different at the 10\% level.}
\end{tablenotes}
\end{threeparttable}
\end{table}


\begin{table}[H]
\centering
\caption{Multinomial Logit, Age-30 Cohort, Preschool, All Cities} \label{mlogit_coeff_age30}
\begin{threeparttable}
\begin{tabular}{l c c c}
\toprule
& None & Other \\
\midrule
Male & -0.610* &    -0.371 \\
Male $\times$ Parma &     0.532 &     0.047 \\
Male $\times$ Padova & 1.541* &     0.436 \\
Mom up to high school &    12.574 &    -0.598 \\
Mom up to high school $\times$ Parma &   -11.645 & 1.656* \\
Mom up to high school $\times$ Padova &   -12.038 &     0.566 \\
Mom at least uni. &    12.997 & -1.357* \\
Mom at least uni. $\times$ Parma &   -13.132 & 1.954* \\
Mom at least uni. $\times$ Padova &   -12.751 &     1.443 \\
Religious caregiver & 0.967* & 0.637* \\
Religious caregiver $\times$ Parma &    -0.726 & \textbf{-1.165*} \\
Religious caregiver $\times$ Padova &    -0.096 & \textbf{   -0.155} \\
Mom born in province &     0.486 &     0.329 \\
Mom born in province $\times$ Parma &     0.218 &     0.145 \\
Mom born in province $\times$ Padova &    -0.396 &    -0.223 \\
Dad born in province &    -0.145 &    -0.188 \\
Dad born in province $\times$ Parma &     0.652 &     0.463 \\
Dad born in province $\times$ Padova &     0.361 &    -0.237 \\
At least 2 siblings &     0.309 &    -0.175 \\
At least 2 siblings $\times$ Parma &     0.467 &    -0.010 \\
At least 2 siblings $\times$ Padova &     0.229 &    -0.368 \\
More than 2 siblings &     0.578 &     0.095 \\
More than 2 siblings $\times$ Parma &     0.496 &    -0.823 \\
More than 2 siblings $\times$ Padova &     0.031 &    -0.268 \\
\bottomrule
\end{tabular}
% This mlogit converged? 1
% This file is generated using reggio/script/multinomial-logit/analysis_mlogit_interaction.do

\begin{tablenotes}
\footnotesize\raggedright{Note: This table shows the coefficients from a multinomial logit that uses baseline characteristics to predict enrollment in non-municipal preschool or no preschool, relative to enrollment in municipal preschool. The column titled ``None'' displays the coefficients of attending no preschool. Similarly, the column titled ``Other'' displays the same coefficients for attending a non-municipal preschool. Estimates with a * indicate significance at the 10\% level. Bolded  estimates indicate that the interaction terms of Parma and Padova are significantly different at the 10\% level.}
\end{tablenotes}
\end{threeparttable}
\end{table}

\begin{table}[H]
\centering
\caption{Multinomial Logit, Age-40 Cohort, Preschool,  All Cities} \label{mlogit_coeff_age40}
\begin{threeparttable}
\begin{tabular}{l c c c}
\toprule
& None & Other \\
\midrule
Male &    -0.226 &    -0.047 \\
Male $\times$ Parma &     0.325 &    -0.012 \\
Male $\times$ Padova &     0.446 &    -0.079 \\
Mom up to high school & 0.954* &     0.355 \\
Mom up to high school $\times$ Parma &    -1.018 &    -0.315 \\
Mom up to high school $\times$ Padova &    -0.448 &    -0.594 \\
Mom at least uni. & 1.300* &     0.435 \\
Mom at least uni. $\times$ Parma &    -1.543 &    -0.287 \\
Mom at least uni. $\times$ Padova &    -0.430 &    -0.900 \\
Religious caregiver & 0.639* &     0.252 \\
Religious caregiver $\times$ Parma &    -0.329 &    -0.439 \\
Religious caregiver $\times$ Padova &    -0.960 &    -0.062 \\
Mom born in province & -1.756* &    -0.323 \\
Mom born in province $\times$ Parma &     1.525 &     0.451 \\
Mom born in province $\times$ Padova &     1.889 &     0.252 \\
Dad born in province &    -0.394 &    -0.036 \\
Dad born in province $\times$ Parma &     0.777 &    -0.189 \\
Dad born in province $\times$ Padova &     0.443 &     0.010 \\
At least 2 siblings &     0.424 &    -0.000 \\
At least 2 siblings $\times$ Parma &     0.109 &    -0.313 \\
At least 2 siblings $\times$ Padova &    -0.124 &    -0.168 \\
More than 2 siblings &     0.114 &    -0.321 \\
More than 2 siblings $\times$ Parma &     0.698 &    -0.142 \\
More than 2 siblings $\times$ Padova &     0.230 &     0.124 \\
\bottomrule
\end{tabular}
% This mlogit converged? 1
% This file is generated using reggio/script/multinomial-logit/analysis_mlogit_interaction.do

\begin{tablenotes}
\footnotesize\raggedright{Note: This table shows the coefficients from a multinomial logit that uses baseline characteristics to predict enrollment in non-municipal preschool or no preschool, relative to enrollment in municipal preschool. The column titled ``None'' displays the coefficients of attending no preschool. Similarly, the column titled ``Other'' displays the same coefficients for attending a non-municipal preschool. Estimates with a * indicate significance at the 10\% level. Bolded  estimates indicate that the interaction terms of Parma and Padova are significantly different at the 10\% level.}
\end{tablenotes}
\end{threeparttable}
\end{table}

\begin{table}[H]
\centering
\caption{Multinomial Logit, Child Cohort,  Infant-toddler Care, All Cities} \label{mlogit_coeff_asilo_child}
\begin{threeparttable}
\begin{tabular}{l c c c}
\toprule
& None & Other \\
\midrule
Male & -0.539* &    -0.401 \\
Male $\times$ Parma &     0.569 &     0.486 \\
Male $\times$ Padova &     0.324 &     0.619 \\
Low birthweight &    -0.395 &    -0.931 \\
Low birthweight $\times$ Parma & \textbf{    1.415} &     0.884 \\
Low birthweight $\times$ Padova & \textbf{   -0.284} &     0.672 \\
Premature & -1.198* &     0.300 \\
Premature $\times$ Parma &     0.540 &    -0.513 \\
Premature $\times$ Padova &     0.279 & -1.533* \\
Mom at least uni. & -1.005* &    -0.123 \\
Mom at least uni. $\times$ Parma &     0.277 &    -0.556 \\
Mom at least uni. $\times$ Padova &     0.044 &     0.198 \\
Income at least 50,000 &    -0.215 &     0.419 \\
Income at least 50,000 $\times$ Parma &     0.240 &     0.063 \\
Income at least 50,000 $\times$ Padova &    -0.343 &    -0.423 \\
Catholic caregiver &    -0.140 &    -0.416 \\
Catholic caregiver $\times$ Parma &     0.356 & 1.071* \\
Catholic caregiver $\times$ Padova &     0.118 & 1.196* \\
Relig. Catholic caregiver &     0.205 & 0.807* \\
Relig. Catholic caregiver $\times$ Parma &    -0.355 & -1.497* \\
Relig. Catholic caregiver $\times$ Padova &    -0.351 &    -0.623 \\
Mom born in province & 0.714* &     0.019 \\
Mom born in province $\times$ Parma & \textbf{-0.756*} &    -0.298 \\
Mom born in province $\times$ Padova & \textbf{    0.303} &    -0.286 \\
Migrant & 0.959* &     0.506 \\
Migrant $\times$ Parma & -1.127* & -1.374* \\
Migrant $\times$ Padova & -1.379* & -2.123* \\
At least 2 siblings &     0.526 &     0.128 \\
At least 2 siblings $\times$ Parma & -1.513* & -1.096* \\
At least 2 siblings $\times$ Padova & -1.408* &    -0.797 \\
More than 2 siblings &     0.014 &    -0.451 \\
More than 2 siblings $\times$ Parma &    -0.567 & \textbf{    0.311} \\
More than 2 siblings $\times$ Padova & -1.385* & \textbf{   -1.346} \\
\bottomrule
\end{tabular}
% This mlogit converged? 1
% This file is generated using reggio/script/multinomial-logit/analysis_mlogit_interaction.do

\begin{tablenotes}
\footnotesize\raggedright{Note: This table shows the coefficients from a multinomial logit that uses baseline characteristics to predict enrollment in non-municipal infant-toddler care or no infant-toddler care, relative to enrollment in municipal infant-toddler care. The column titled ``None'' displays the coefficients of attending no preschool. Similarly, the column titled ``Other'' displays the same coefficients for attending a non-municipal infant-toddler care. Estimates with a * indicate significance at the 10\% level. Bolded  estimates indicate that the interaction terms of Parma and Padova are significantly different at the 10\% level.}
\end{tablenotes}
\end{threeparttable}
\end{table}


\begin{table}[H]
\centering
\caption{Multinomial Logit, Adolescent Cohort,  Infant-toddler Care,  All Cities} \label{mlogit_coeff_asilo_adol}
\begin{threeparttable}
\begin{tabular}{l c c c}
\toprule
& None & Other \\
\midrule
Male &    -0.081 &     0.095 \\
Male $\times$ Parma &     0.174 &     0.805 \\
Male $\times$ Padova &    -0.037 &     0.258 \\
Low birthweight &     0.014 &    -0.190 \\
Low birthweight $\times$ Parma &     0.091 &     0.269 \\
Low birthweight $\times$ Padova &    -0.655 &   -14.027 \\
Premature &     0.626 &     1.265 \\
Premature $\times$ Parma &    -0.941 &    -1.287 \\
Premature $\times$ Padova &    -0.587 &    -0.894 \\
Mom at least uni. &    -0.368 &     0.250 \\
Mom at least uni. $\times$ Parma & \textbf{    0.467} &     0.171 \\
Mom at least uni. $\times$ Padova & \textbf{   -0.502} &    -0.595 \\
Income at least 50,000 & -0.537* &     0.038 \\
Income at least 50,000 $\times$ Parma &     0.135 &    -0.545 \\
Income at least 50,000 $\times$ Padova &     0.621 &    -0.445 \\
Catholic caregiver &     0.145 & 0.879* \\
Catholic caregiver $\times$ Parma & \textbf{    0.603} & \textbf{   -1.143} \\
Catholic caregiver $\times$ Padova & \textbf{   -0.754} & \textbf{-3.896*} \\
Relig. Catholic caregiver &     0.285 &    -0.359 \\
Relig. Catholic caregiver $\times$ Parma & \textbf{   -0.766} & \textbf{    0.661} \\
Relig. Catholic caregiver $\times$ Padova & \textbf{    0.677} & \textbf{2.852*} \\
Mom born in province & -0.526* &    -0.121 \\
Mom born in province $\times$ Parma & 0.885* &    -0.002 \\
Mom born in province $\times$ Padova & 1.028* &    -0.259 \\
Migrant caregiver &     0.011 &     0.331 \\
Migrant caregiver $\times$ Parma &     1.051 &   -14.985 \\
Migrant caregiver $\times$ Padova & 0.000* & 0.000* \\
At least 2 siblings &     0.446 &    -0.461 \\
At least 2 siblings $\times$ Parma & \textbf{   -0.364} & \textbf{    0.934} \\
At least 2 siblings $\times$ Padova & \textbf{-1.326*} & \textbf{   -1.538} \\
More than 2 siblings &    -0.075 &    -0.018 \\
More than 2 siblings $\times$ Parma &    -0.762 &    -0.857 \\
More than 2 siblings $\times$ Padova &     0.323 &   -14.337 \\
\bottomrule
\end{tabular}
% This mlogit converged? 1
% This file is generated using reggio/script/multinomial-logit/analysis_mlogit_interaction.do

\begin{tablenotes}
\footnotesize\raggedright{Note: This table shows the coefficients from a multinomial logit that uses baseline characteristics to predict enrollment in non-municipal infant-toddler care or no infant-toddler care, relative to enrollment in municipal infant-toddler care. The column titled ``None'' displays the coefficients of attending no preschool. Similarly, the column titled ``Other'' displays the same coefficients for attending a non-municipal infant-toddler care. Estimates with a * indicate significance at the 10\% level. Bolded  estimates indicate that the interaction terms of Parma and Padova are significantly different at the 10\% level.}
\end{tablenotes}
\end{threeparttable}
\end{table}


\begin{table}[H]
\centering
\caption{Multinomial Logit, Age-30 Cohort,  Infant-toddler Care,  All Cities} \label{mlogit_coeff_asilo_age30}
\begin{threeparttable}
\begin{tabular}{l c c c}
\toprule
& None & Other \\
\midrule
Male &     0.007 &     0.445 \\
Male $\times$ Parma &    -0.007 &    -0.565 \\
Male $\times$ Padova &    -0.276 &    -0.966 \\
Mom up to high school &   -16.818 &   -18.484 \\
Mom up to high school $\times$ Parma &    17.921 &    19.941 \\
Mom up to high school $\times$ Padova &    16.125 &    16.561 \\
Mom at least uni. &   -17.213 &   -19.770 \\
Mom at least uni. $\times$ Parma &    18.651 &    21.676 \\
Mom at least uni. $\times$ Padova &    16.737 &    19.724 \\
Religious caregiver & 0.576* &     0.390 \\
Religious caregiver $\times$ Parma &     0.314 & \textbf{    1.037} \\
Religious caregive $\times$ Padova &    -0.261 & \textbf{   -0.685} \\
Mom born in province & 1.005* & 1.957* \\
Mom born in province $\times$ Parma &    -0.424 & -2.295* \\
Mom born in province $\times$ Padova &    -0.450 &    -1.411 \\
Dad born in province &    -0.444 &    -0.664 \\
Dad born in province $\times$ Parma &     0.349 &     0.069 \\
Dad born in province $\times$ Padova &     0.881 &     0.985 \\
At least 2 siblings &     0.003 &     0.226 \\
At least 2 siblings $\times$ Parma & 2.459* &     0.922 \\
At least 2 siblings $\times$ Padova & 2.283* &     1.338 \\
More than 2 siblings & 2.154* &   -12.849 \\
More than 2 siblings $\times$ Parma &    -0.395 &    12.490 \\
More than 2 siblings $\times$ Padova &    13.631 &    13.167 \\
\bottomrule
\end{tabular}
% This mlogit converged? 1
% This file is generated using reggio/script/multinomial-logit/analysis_mlogit_interaction.do

\begin{tablenotes}
\footnotesize\raggedright{Note: This table shows the coefficients from a multinomial logit that uses baseline characteristics to predict enrollment in non-municipal infant-toddler care or no infant-toddler care, relative to enrollment in municipal infant-toddler care. The column titled ``None'' displays the coefficients of attending no preschool. Similarly, the column titled ``Other'' displays the same coefficients for attending a non-municipal infant-toddler care. Estimates with a * indicate significance at the 10\% level. Bolded  estimates indicate that the interaction terms of Parma and Padova are significantly different at the 10\% level.}
\end{tablenotes}
\end{threeparttable}
\end{table}

\begin{table}[H]
\centering
\caption{Multinomial Logit, Age-40 Cohort,  Infant-toddler Care,  All Cities} \label{mlogit_coeff_asilo_age40}
\begin{threeparttable}
\begin{tabular}{l c c c}
\toprule
& None & Other \\
\midrule
Male &    -0.329 &    -0.890 \\
Male $\times$ Parma &     0.342 &     0.732 \\
Male $\times$ Padova &     0.302 &     1.181 \\
Mom up to high school & 0.956* &     0.132 \\
Mom up to high school $\times$ Parma &    -0.893 &    -0.739 \\
Mom up to high school $\times$ Padova &    -0.951 &    -0.194 \\
Mom at least uni. & 1.383* &     1.497 \\
Mom at least uni. $\times$ Parma &    -1.324 &    -2.092 \\
Mom at least uni. $\times$ Padova &    -1.369 &    -1.663 \\
Religious caregiver &     0.201 &    -0.531 \\
Religious caregiver $\times$ Parma &    -0.180 &     0.293 \\
Religious caregiver $\times$ Padova &    -0.233 &     0.874 \\
Mom born in province &    -0.276 &    -0.090 \\
Mom born in province $\times$ Parma &     0.194 &     1.110 \\
Mom born in province $\times$ Padova &     0.278 &     0.102 \\
Dad born in province & -1.951* & -2.301* \\
Dad born in province $\times$ Parma &     2.001 &     1.670 \\
Dad born in province $\times$ Padova &     1.993 &     1.855 \\
At least 2 siblings &     0.905 &     0.653 \\
At least 2 siblings $\times$ Parma &    -0.828 &    -1.479 \\
At least 2 siblings $\times$ Padova &    -0.787 &    -1.857 \\
More than 2 siblings & 1.960* &   -14.402 \\
More than 2 siblings $\times$ Parma &    -1.845 &    13.086 \\
More than 2 siblings $\times$ Padova &    -1.832 &    13.045 \\
\bottomrule
\end{tabular}
% This mlogit converged? 1
% This file is generated using reggio/script/multinomial-logit/analysis_mlogit_interaction.do

\begin{tablenotes}
\footnotesize\raggedright{Note: This table shows the coefficients from a multinomial logit that uses baseline characteristics to predict enrollment in non-municipal infant-toddler care or no infant-toddler care, relative to enrollment in municipal infant-toddler care. The column titled ``None'' displays the coefficients of attending no preschool. Similarly, the column titled ``Other'' displays the same coefficients for attending a non-municipal infant-toddler care. Estimates with a * indicate significance at the 10\% level. Bolded  estimates indicate that the interaction terms of Parma and Padova are significantly different at the 10\% level.}
\end{tablenotes}
\end{threeparttable}
\end{table}


