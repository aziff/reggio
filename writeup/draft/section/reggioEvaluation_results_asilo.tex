

\subsection{Estimation Results for Infant-Toddler Centers}

% ========================================================================= %
% CHILD COHORT

Tables \ref{ols-M-child-reg-nopres-asilo} to \ref{ols-M-adult40-reg-nopres-asilo} show the estimation results for Reggio Approach infant-toddler centers based on the previous section. 

For children cohort, Reggio Approach infant-toddler centers have significantly positive effect on IQ scores and obesity relative to no infant-toddler centers in Reggio. For adolescents cohort, Reggio Approach infant-toddler centers do not have clear affect relative to no infant-toddler centers. 

For adult 30s cohort, Reggio Approach infant-toddler centers have significantly negative effect on IQ score, IQ factor, high school grade, and locus of control. Moreover, Reggio Approach infant-toddler centers have significantly positive effect on hours worked per week, marriage, and obesity. For adult 40s cohort, Reggio Approach has significantly negative effect on IQ score, high school grades, and marriage. However, Reggio Approach is shown to have positive effects on obesity. 


\begin{table}[H] \caption{Estimation Results for Main Outcomes, Comparison to No Infant-Toddler Centers, Child Cohort} \label{ols-M-child-reg-nopres-asilo}
\scalebox{0.8}{\begin{tabular}{l c c c}
\toprule
 & NoneIt & BICIt & FullIt \\
\midrule
IQ Score &      0.05 & \textbf{      0.06 } &      0.03 \\
& (     0.03 ) & (     0.03 ) & (     0.04 ) \\
& \textit{ 228 } & \textit{ 228 } & \textit{ 228 } \\
IQ Factor & \textbf{      0.27 } & \textbf{      0.30 } &      0.19 \\
& (     0.14 ) & (     0.15 ) & (     0.15 ) \\
& \textit{ 228 } & \textit{ 228 } & \textit{ 228 } \\
SDQ Composite - Child &      0.17 &      0.31 &      0.72 \\
& (     0.63 ) & (     0.62 ) & (     0.71 ) \\
& \textit{ 228 } & \textit{ 228 } & \textit{ 228 } \\
Obese & \textbf{     -0.22 } & \textbf{     -0.23 } & \textbf{     -0.16 } \\
& (     0.07 ) & (     0.07 ) & (     0.08 ) \\
& \textit{ 228 } & \textit{ 228 } & \textit{ 228 } \\
Overweight &      0.01 &      0.02 &     -0.02 \\
& (     0.05 ) & (     0.05 ) & (     0.05 ) \\
& \textit{ 228 } & \textit{ 228 } & \textit{ 228 } \\
Health is Good &     -0.03 &     -0.06 &      0.02 \\
& (     0.07 ) & (     0.07 ) & (     0.08 ) \\
& \textit{ 228 } & \textit{ 228 } & \textit{ 228 } \\
Not Excited to Learn &     -0.01 &      0.01 &      0.01 \\
& (     0.03 ) & (     0.03 ) & (     0.03 ) \\
& \textit{ 228 } & \textit{ 228 } & \textit{ 228 } \\
Problems Sitting Still &     -0.07 &     -0.09 &     -0.09 \\
& (     0.06 ) & (     0.06 ) & (     0.06 ) \\
& \textit{ 228 } & \textit{ 228 } & \textit{ 228 } \\
How Much Child Likes School &      0.12 &      0.12 &      0.08 \\
& (     0.09 ) & (     0.09 ) & (     0.10 ) \\
& \textit{ 228 } & \textit{ 228 } & \textit{ 228 } \\
\bottomrule
\end{tabular}
}
\vspace{1ex} \\
\footnotesize\raggedright{Note: This table shows the estimates of the coefficient for attending Reggio Approach infant-toddler centers from multiple methods. We compare Reggio Approach people with people who attended no infant-toddler center. Column title indicates the corresponding control set and and model. ``NoneIt'' refers to the OLS estimate with no control variables. ``BICIt'' refers to the OLS estimate with controls selected by Bayesian Information Criterion (BIC) and additional controls for caregiver's religion. ``FullIt'' refers to the OLS estimate with the full set of controls. Robust standard errors are reported in parentheses. Bold number shows that the estimate is statistically significant at the 10\% level. Number of observations used in estimation is reported in italic.}

\end{table}

\begin{table}[H] \caption{Estimation Results for Main Outcomes, Comparison to No Infant-Toddler Centers, Adolescent Cohort} \label{ols-M-adol-reg-nopres-asilo}
\scalebox{0.8}{\begin{tabular}{l c c c}
\toprule
 & None & Bic & Full \\
\midrule
IQ Factor & 0.17 & 0.16 & 0.10 \\
& (0.18) & (0.18) & (0.17) \\
& \textit{ 204 } & \textit{ 204 } & \textit{ 204 } \\
SDQ Composite - Child & \textbf{ 1.29 } & 0.87 & 0.85 \\
& (0.78) & (0.81) & (0.86) \\
& \textit{ 202 } & \textit{ 202 } & \textit{ 202 } \\
SDQ Composite & -0.52 & -1.04 & -1.10 \\
& (0.88) & (0.88) & (0.98) \\
& \textit{ 201 } & \textit{ 201 } & \textit{ 201 } \\
Depression Score - positive & -0.90 & -1.23 & -1.38 \\
& (1.11) & (1.10) & (1.13) \\
& \textit{ 198 } & \textit{ 198 } & \textit{ 198 } \\
Locus of Control - positive & -0.12 & -0.15 & \textbf{ -0.19 } \\
& (0.12) & (0.12) & (0.12) \\
& \textit{ 200 } & \textit{ 200 } & \textit{ 200 } \\
Not Obese & 0.06 & 0.09 & 0.07 \\
& (0.07) & (0.07) & (0.06) \\
& \textit{ 204 } & \textit{ 204 } & \textit{ 204 } \\
Not Overweight & 0.04 & 0.04 & 0.03 \\
& (0.05) & (0.05) & (0.05) \\
& \textit{ 204 } & \textit{ 204 } & \textit{ 204 } \\
Health is Good & -0.04 & -0.06 & -0.05 \\
& (0.08) & (0.09) & (0.09) \\
& \textit{ 204 } & \textit{ 204 } & \textit{ 204 } \\
Go To School & 0.02 & 0.01 & -0.01 \\
& (0.04) & (0.04) & (0.04) \\
& \textit{ 204 } & \textit{ 204 } & \textit{ 204 } \\
How Much Child Likes School & 0.00 & -0.02 & -0.06 \\
& (0.17) & (0.18) & (0.19) \\
& \textit{ 195 } & \textit{ 195 } & \textit{ 195 } \\
Days of Sport (Weekly) & \textbf{ 0.54 } & 0.49 & 0.46 \\
& (0.36) & (0.38) & (0.39) \\
& \textit{ 198 } & \textit{ 198 } & \textit{ 198 } \\
Num. of Friends & \textbf{ 4.38 } & \textbf{ 4.18 } & \textbf{ 4.01 } \\
& (1.32) & (1.33) & (1.50) \\
& \textit{ 189 } & \textit{ 189 } & \textit{ 189 } \\
Volunteers & 0.05 & 0.02 & 0.01 \\
& (0.08) & (0.08) & (0.08) \\
& \textit{ 204 } & \textit{ 204 } & \textit{ 204 } \\
Trust Score & 0.13 & 0.09 & 0.01 \\
& (0.25) & (0.26) & (0.28) \\
& \textit{ 201 } & \textit{ 201 } & \textit{ 201 } \\
\bottomrule
\end{tabular}
}
\vspace{1ex} \\
\footnotesize\raggedright{Note: This table shows the estimates of the coefficient for attending Reggio Approach infant-toddler centers from multiple methods. We compare Reggio Approach people with people who attended no infant-toddler center. Column title indicates the corresponding control set and and model. ``None'' refers to the OLS estimate with no control variables. ``BIC'' refers to the OLS estimate with controls selected by Bayesian Information Criterion (BIC) and additional controls for caregiver's religion. ``Full'' refers to the OLS estimate with the full set of controls. Robust standard errors are reported in parentheses. Bold number shows that the estimate is statistically significant at the 10\% level. Number of observations used in estimation is reported in italic.}
\end{table}



\begin{table}[H] \caption{Estimation Results for Main Outcomes, Comparison to No Infant-Toddler Centers, Adult 30s Cohort} \label{ols-M-adult30-reg-nopres-asilo}
\scalebox{0.75}{\begin{tabular}{l c c c}
\toprule
 & None & BIC & Full \\
\midrule
IQ Factor & -0.06 & -0.13 & -0.15 \\
& (0.12) & (0.12) & (0.12) \\
& \textit{ 215 } & \textit{ 215 } & \textit{ 215 } \\
Graduate from High School & -0.05 & -0.04 & -0.00 \\
& (0.05) & (0.05) & (0.05) \\
& \textit{ 215 } & \textit{ 215 } & \textit{ 215 } \\
High School Grade & \textbf{ -7.63 } & \textbf{ -6.72 } & \textbf{ -6.18 } \\
& (2.01) & (1.86) & (1.95) \\
& \textit{ 162 } & \textit{ 162 } & \textit{ 162 } \\
High School Grade (Standardized) & \textbf{ -5.61 } & \textbf{ -4.76 } & \textbf{ -4.62 } \\
& (1.72) & (1.73) & (1.79) \\
& \textit{ 162 } & \textit{ 162 } & \textit{ 162 } \\
Max Edu: University & -0.01 & -0.02 & -0.03 \\
& (0.06) & (0.06) & (0.06) \\
& \textit{ 215 } & \textit{ 215 } & \textit{ 215 } \\
Employed & \textbf{ 0.04 } & \textbf{ 0.05 } & 0.04 \\
& (0.03) & (0.03) & (0.03) \\
& \textit{ 215 } & \textit{ 215 } & \textit{ 215 } \\
Hours Worked Per Week & \textbf{ 2.91 } & \textbf{ 3.61 } & \textbf{ 3.00 } \\
& (1.60) & (1.66) & (1.77) \\
& \textit{ 191 } & \textit{ 191 } & \textit{ 191 } \\
Married or Cohabitating & \textbf{ 0.18 } & \textbf{ 0.18 } & \textbf{ 0.14 } \\
& (0.07) & (0.07) & (0.07) \\
& \textit{ 215 } & \textit{ 215 } & \textit{ 215 } \\
Not Obese & \textbf{ 0.15 } & \textbf{ 0.11 } & \textbf{ 0.10 } \\
& (0.05) & (0.05) & (0.06) \\
& \textit{ 215 } & \textit{ 215 } & \textit{ 215 } \\
Not Overweight & 0.06 & 0.05 & 0.06 \\
& (0.06) & (0.05) & (0.06) \\
& \textit{ 215 } & \textit{ 215 } & \textit{ 215 } \\
Locus of Control - positive & -0.14 & -0.13 & -0.08 \\
& (0.11) & (0.10) & (0.11) \\
& \textit{ 212 } & \textit{ 212 } & \textit{ 212 } \\
Depression Score - positive & 0.72 & 0.18 & 0.03 \\
& (0.78) & (0.76) & (0.79) \\
& \textit{ 214 } & \textit{ 214 } & \textit{ 214 } \\
Ever Voted for Municipal & \textbf{ 0.14 } & 0.04 & 0.05 \\
& (0.07) & (0.06) & (0.06) \\
& \textit{ 210 } & \textit{ 210 } & \textit{ 210 } \\
Ever Voted for Regional & \textbf{ 0.15 } & 0.07 & 0.07 \\
& (0.07) & (0.06) & (0.06) \\
& \textit{ 210 } & \textit{ 210 } & \textit{ 210 } \\
\bottomrule
\end{tabular}
}
\vspace{1ex} \\
\footnotesize\raggedright{Note: This table shows the estimates of the coefficient for attending Reggio Approach infant-toddler centers from multiple methods. We compare Reggio Approach people with people who attended no infant-toddler center. Column title indicates the corresponding control set and and model. For age-30 cohort, the columns are as follows: ``None30'' refers to the OLS estimate with no control variables. ``BIC30'' refers to the OLS estimate with controls selected by Bayesian Information Criterion (BIC) and additional controls for caregiver's religion. ``Full30'' refers to the OLS estimate with the full set of controls. Column titles are analogous for age-40 cohort. Robust standard errors are reported in parentheses. Bold number shows that the estimate is statistically significant at the 10\% level. Number of observations used in estimation is reported in italic.}
\end{table}


\begin{table}[H] \caption{Estimation Results for Main Outcomes, Comparison to No Infant-Toddler Centers, Adult 40s Cohort} \label{ols-M-adult40-reg-nopres-asilo}
\scalebox{0.75}{\begin{tabular}{l c c c}
\toprule
 & None40 & BIC40 & Full40 \\
\midrule
IQ Score & \textbf{     -0.12 } & \textbf{     -0.14 } & \textbf{     -0.15 } \\
& (     0.04 ) & (     0.04 ) & (     0.04 ) \\
& \textit{ 206 } & \textit{ 206 } & \textit{ 206 } \\
IQ Factor & \textbf{     -0.22 } & \textbf{     -0.28 } & \textbf{     -0.31 } \\
& (     0.11 ) & (     0.12 ) & (     0.12 ) \\
& \textit{ 206 } & \textit{ 206 } & \textit{ 206 } \\
Graduate from High School &     -0.05 &     -0.07 &     -0.05 \\
& (     0.05 ) & (     0.06 ) & (     0.06 ) \\
& \textit{ 206 } & \textit{ 206 } & \textit{ 206 } \\
High School Grade & \textbf{     -9.32 } & \textbf{     -9.35 } & \textbf{     -8.19 } \\
& (     2.14 ) & (     2.12 ) & (     2.39 ) \\
& \textit{ 157 } & \textit{ 157 } & \textit{ 157 } \\
High School Grade (Standardized) & \textbf{     -4.53 } & \textbf{     -4.80 } & \textbf{     -3.95 } \\
& (     1.85 ) & (     1.93 ) & (     2.07 ) \\
& \textit{ 154 } & \textit{ 154 } & \textit{ 154 } \\
Max Edu: University &     -0.01 &     -0.05 &     -0.08 \\
& (     0.05 ) & (     0.05 ) & (     0.05 ) \\
& \textit{ 206 } & \textit{ 206 } & \textit{ 206 } \\
Employed &     -0.01 &     -0.03 &     -0.01 \\
& (     0.03 ) & (     0.03 ) & (     0.03 ) \\
& \textit{ 206 } & \textit{ 206 } & \textit{ 206 } \\
Hours Worked Per Week &     -0.00 &      0.39 &      1.47 \\
& (     1.69 ) & (     1.80 ) & (     2.01 ) \\
& \textit{ 190 } & \textit{ 190 } & \textit{ 190 } \\
Married or Cohabitating & \textbf{     -0.25 } & \textbf{     -0.23 } & \textbf{     -0.24 } \\
& (     0.06 ) & (     0.07 ) & (     0.07 ) \\
& \textit{ 206 } & \textit{ 206 } & \textit{ 206 } \\
Obese & \textbf{     -0.21 } & \textbf{     -0.17 } & \textbf{     -0.17 } \\
& (     0.06 ) & (     0.06 ) & (     0.06 ) \\
& \textit{ 206 } & \textit{ 206 } & \textit{ 206 } \\
Overweight &     -0.06 &     -0.09 &     -0.09 \\
& (     0.06 ) & (     0.07 ) & (     0.07 ) \\
& \textit{ 206 } & \textit{ 206 } & \textit{ 206 } \\
Locus of Control - positive & \textbf{     -0.26 } &     -0.17 &     -0.13 \\
& (     0.12 ) & (     0.13 ) & (     0.14 ) \\
& \textit{ 204 } & \textit{ 204 } & \textit{ 204 } \\
Depression Score - positive &     -0.58 &     -0.99 &     -0.97 \\
& (     0.78 ) & (     0.81 ) & (     0.86 ) \\
& \textit{ 206 } & \textit{ 206 } & \textit{ 206 } \\
Ever Voted for Municipal &      0.01 &     -0.07 &     -0.02 \\
& (     0.07 ) & (     0.06 ) & (     0.07 ) \\
& \textit{ 196 } & \textit{ 196 } & \textit{ 196 } \\
Ever Voted for Regional &      0.00 &     -0.07 &     -0.02 \\
& (     0.07 ) & (     0.06 ) & (     0.07 ) \\
& \textit{ 196 } & \textit{ 196 } & \textit{ 196 } \\
\bottomrule
\end{tabular}
}
\vspace{1ex} \\
\footnotesize\raggedright{Note: This table shows the estimates of the coefficient for attending Reggio Approach infant-toddler centers from multiple methods. We compare Reggio Approach people with people who attended no infant-toddler center. Column title indicates the corresponding control set and and model. For age-30 cohort, the columns are as follows: ``None40'' refers to the OLS estimate with no control variables. ``BIC40'' refers to the OLS estimate with controls selected by Bayesian Information Criterion (BIC) and additional controls for caregiver's religion. ``Full40'' refers to the OLS estimate with the full set of controls. Robust standard errors are reported in parentheses. Bold number shows that the estimate is statistically significant at the 10\% level. Number of observations used in estimation is reported in italic.}
\end{table}








