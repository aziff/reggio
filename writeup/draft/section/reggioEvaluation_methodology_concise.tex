Since no single analytic approach is best, we consider several methodologies to evaluate the effect of the Reggio Approach using the survey data described in Section~\ref{sec:data}. These methodologies require different identifying assumptions and leverage different control groups. An effect robustly estimated across these methodologies would provide strong evidence in favor of the validity of any such conclusion that might emerge.

Our analysis is carried out in two dimensions. First, we compare the Reggio Approach with other childcare systems within the city of Reggio Emilia. Section~\ref{sec:within-city-analysis} presents various methodologies used to estimate the treatment effects of the Reggio Approach with a restriction of the sample to individuals within the city of Reggio Emilia. Second, we estimate the effect of Reggio Approach relative to other childcare systems across cities. Section~\ref{sec:across-city-analysis} presents methodologies used for the across-city analysis.

The Reggio Approach includes interventions at two different age ranges: (i) infant-toddler centers between ages 0-3, and (ii) preschool between ages 3-6. We note that our analysis of the infant-toddler centers is more limited compared to our preschool analysis because attendance of infant-toddler centers was very low in the adult cohorts. The data limitations are especially problematic for across-city comparisons as infant-toddler centers in Parma and Padova had relatively poorer provision, operations, and program quality for the older cohorts \footnote{Among adults in Padova and Parma, only the age 30 cohorts were exposed to municipal infant-toddler centers.}. Even though we focus on preschool, we present our method to analyze the effect of the Reggio Approach infant-toddler care in Section~\ref{subsubsection:itc}.

\textbf{[JJH: Then this is a great -- clean -- comparison. Group, we should feature more.][We now include an analysis of infant-toddler care.]}. 

\textbf{[JJH: Can we evaluate this within cohorts?]} \textbf{[Team: We provide these within-cohort evaluations in the appendix. We have changed the text in this paragraph to make this clearer. Would you prefer to have it in the main paper?] [JJH: Yes, if possible.][We moved these results to the main paper.]}

\subsection{Within-City Analysis} \label{sec:within-city-analysis}

\subsubsection{Framework to Evaluate Preschool} 
\label{subsubsection:OLS-Preschool}

We perform several within-Reggio Emilia comparisons using OLS and matching models. We compare individuals from Reggio Emilia who attended a Reggio Approach preschool to those in Reggio Emilia who attended (i) any other type of preschool (state, religious, municipal-affiliated, and other), (ii) no preschool at all, (iii) state preschool, and (iv) religious preschool. We only present estimates of the first two comparisons in the main paper to focus on the main hypotheses of the effectiveness of the Reggio Approach. The estimates of comparisons to specific school types are reported in Appendix~\ref{app:comparison-reli-stat} and summarized in Section~\ref{sec:result}. \textbf{[JJH: Why? Please present in text.][We edited this to explain why we have other comparisons in the appendix. We also added to the description of results to summarize the results.]} 

For the children cohort (age 5), it is not possible to compare Reggio Approach preschools with no preschool because the sample of individuals who didn't attend preschool is small (See Table~\ref{tab:sample}). For the age-50 cohort, we compare against other adult cohorts because the age-50 cohort preceded the Reggio Approach. \textbf{[JJH: But this gives us a good baseline no-treatment approach.][We now include this analysis.]}

The basic OLS takes the form,

\begin{equation}
	Y_i = \alpha_0 + \alpha_1 D_i + \bm{X}_i \bm{\gamma} + \varepsilon_i
	\label{eq:ra-v-none}
\end{equation}
where $i$ indexes individuals, $D_i$ is an indicator for whether individual $i$ attended municipal preschool, $\bm{X}_i$ is a vector of baseline control variables, and $\varepsilon_i$ is a random disturbance. Estimations for three specifications for $\bm{X}_i$ are reported: (i) no baseline control, (ii) baseline variables selected by the Bayesian Information Criterion (BIC)\footnote{Since the set of baseline variables are different for child, adolescent, and adult cohorts, we use separate model selections. For \emph{child cohort}, the \emph{a priori} designated control variables are male, CAPI, infant-toddler center attendance, and migrant indicators, and the BIC-selected variables are (i) mother graduated university, (2) family owns house, and (3) family income 10,000-25,000. For the \emph{adolescent cohort}, the fixed variables are male, CAPI, infant-toddler center attendance indicators and BIC-selected variables are (i) high school is father's maximum education, (ii) university is father's maximum education, and (iii) caregiver is catholic and faithful. For \emph{adult cohorts}, the fixed variables are male and CAPI indicators, and BIC-selected variables are (i) university is father's maximum education and (ii) number of siblings.} in addition to indicators for male, CAPI, and infant-toddler center attendance, and (iii) the full set of available baseline variables. In Equation~\eqref{eq:ra-v-none}, $\alpha_1$ represents the mean differences in outcomes between the Reggio Approach and the other preschool types in Reggio Emilia, conditional on $\bm{X}$. Under the assumption that, conditional on $\bm{X}$, there is no systematic selection of individuals into the treatment $D_i$, this parameter estimates the causal treatment effect of the Reggio Approach on outcome $Y$.

In order to complement the OLS analysis, we also estimate two matching models: (i) a propensity score matching model that implements nearest-neighbor matching on an estimated propensity score based on a BIC-selected set of observed baseline characteristics $\boldsymbol{X}_i$ and (ii) a  matching model using Epanechnikov kernel weight and $\boldsymbol{X}_i$. These matching models are versions of non-parametric OLS and condition on the same set of $\bm{X}$ variables as OLS. These approaches match people who attended Reggio Approach preschools with people who did not attend Reggio Approach preschools based on similarities in observed baseline characteristics. 

The average treatment effect (ATE) under the assumption for propensity score matching is written as:
\begin{equation} \label{eq:ATE-PSM}
E[Y(1)-Y(0)] = E\bigg[E\big[Y_i|D_i=1, \hat{\pi(\boldsymbol{X}_i)}\big] - E\big[Y_i|D_i=0, \hat{\pi(\boldsymbol{X}_i)}\big]\bigg].
\end{equation}
where the propensity score $\hat{\pi({\boldsymbol{X}_i})} = Pr(D_i=1|\boldsymbol{X}_i)$ is predicted for each individual $i$ using the estimated coefficients obtained from a probit model. We average over sample $\bm{X}$ to evaluate the average treatment effect.

The $k$-nearest neighbor matching estimator is defined as
\begin{equation} \label{eq:PSM-estimator}
\widehat{E[Y(1)-Y(0)]_{PSM}} = \frac{1}{n} \sum_{i=1}^{n} (2D_i -1)(Y_i - \frac{1}{M}\sum_{j \in \mathcal{J}_M(i)}Y_j )
\end{equation}
where $M$ is a fixed number of matches per individual based on the propensity score and $\mathcal{J}_M(i)$ is a set of matches for individual $i$.\footnote{We specify $M = 3$ in our analysis.} The kernel matching estimator constructs a match for each treated individual using the weighted average over multiple people in the comparison group based on Mahalanobis distance and Epanechnikov kernel weight. The standard errors for both nearest neighbor matching estimator and the kernel matching estimator are derived by \cite{Abadie_Imbens_2006_Econometrica}. \textbf{[JJH: Why did we not do kernel matching?][We have added kernel matching.]} We compare the robustness of the estimates across methods in the results section.

\textbf{[Team: Our discussion of effects on outcomes in the results and discussion sections only takes into account outcomes that are robust across different regressions.  We have added text in the results section to make this more explicit.] [JJH: What does ``robust'' mean? OLS and propensity score should have the same outcomes.][Yes, we use the same outcomes. We highlight the robustness of the estimates across methods in the results section and have included kernel density matching for additional robustness checking.]}

\subsubsection{Framework to Evaluate Infant-toddler Care}
\label{subsubsection:itc}

We analyze the effectiveness of Reggio Approach infant-toddler care within the city of Reggio Emilia accounting for subsequent preschool experiences. Table~\ref{tab:cases-treat} shows the four possible combinations of interventions that a child could receive, where 1 indicates attending a municipal program and 0 indicates not attending any childcare or attending a non-municipal program.

\begin{table}[H]
\caption{Possible Cases of Treatment} \label{tab:cases-treat}
\begin{tabular}{C{1.8cm} R{0.7cm} C{2cm} C{2cm}}

		& & \multicolumn{2}{c}{Preschool (Ages 3-6)} \\
		& & 0 & 1 \\ \cline{3-4}
        								 &  & \multicolumn{1}{|c|}{} & \multicolumn{1}{c|}{} \\
        							& 0 & \multicolumn{1}{|c|}{(0,0)} & \multicolumn{1}{c|}{(0,1)} \\
        				ITC				&  & \multicolumn{1}{|c|}{} & \multicolumn{1}{c|}{} \\ \cline{3-4}
                        (Age 0-3)  		&  & \multicolumn{1}{|c|}{} & \multicolumn{1}{c|}{} \\
        								& 1 & \multicolumn{1}{|c|}{(1,0)} & \multicolumn{1}{c|}{(1,1)} \\
        								&  & \multicolumn{1}{|c|}{} & \multicolumn{1}{c|}{} \\ \cline{3-4}
\end{tabular}
\begin{flushleft}
\footnotesize{Note:} We only consider municipal infant-toddler-centers (ages 0-3) and preschools (ages 3-6). (0,0): did not attend any municipal school for both ages 0-3 and 3-6; (1,0): attended a municipal school for ages 0-3 but did \textit{not} attend for ages 3-6; (0,1): did \textit{not} attend a municipal school for ages 0-3 but did attend for ages 3-6; (1,1): attended a municipal school for both ages 0-3 and 3-6.
\end{flushleft}
\end{table}

There are two main methods to test the effect of attending infant-toddler centers. The first is to compare people who did not attend infant-toddler care or preschool with people who only attended municipal infant-toddler care. Using the notation in Table~\ref{tab:cases-treat}, this comparison is between (0,0) and (1,0). The second method is to compare people who only attended municipal preschool with people who attended both municipal infant-toddler centers and preschools. That is, to compare (0,1) and (1,1). The hypotheses are formally written as
\begin{eqnarray}
H_1: &  Y_{0,0} = Y_{1,0} &\text{\quad Effect of infant-toddler care with no subsequent preschool}\\
H_2: &  Y_{0,1} = Y_{1,1} &\text{\quad Effect of infant-toddler care with subsequent preschool}
\end{eqnarray}
\noindent where $Y_{i,j}$ is the outcome of the individuals who attended $i \in \{0,1\}$ infant-toddler care and $j \in \{0,1\}$ preschool.

For each of the two hypotheses above, we limit the sample to include only those individuals from Reggio Emilia who received the treatment combinations that are relevant to testing the hypothesis in question. Furthermore, we restrict the sample to include only one cohort at a time because we test cohort-specific hypotheses. To test these hypotheses, we estimate $\beta_{0}$ in the following equation:
\begin{equation}
Y_{i}^{c,h} = \alpha + \beta_{0}R_i^{ITC} + \mathbf{X}_i \bm{\gamma} + \varepsilon_{i}^{Reggio,h}
\end{equation}

where $R_i^{ITC}$ is the indicator for attending municipal infant-toddler center and $\mathbf{X}_i$ is the vector of baseline variables for individual $i$. To test $H_1$, we would estimate $\beta_0$ on a sample consisting of all individuals from cohort $h$ in Reggio Emilia who received either the (0,0) or (1,0) combination of childcare. We remind the reader that (0,0) and (1,0) is composed of those individuals who did not attend preschool. To test $H_2$, we would estimate $\beta_0$ for all cohort-$h$ individuals in Reggio Emilia who were in groups (0,1) or (1,1). 

One caveat for the analysis of infant-toddler centers is that it relies on small samples. As a result, these hypotheses cannot be tested for many groups. Table~\ref{tab:num-group-2} shows the number of individuals available in each group necessary for this strategy. It is impossible to test $H_1$ in our data, because there are almost no individuals who attended municipal infant-toddler care without attending preschool (group (1,0)). While it is possible to test $H_2$ for several groups, the number of observations for the group (1,1) is small for the adult cohorts. Table~\ref{tab:num-group-2} highlights the groups that we use for estimation. 

\begin{table}[H] \caption{Number of Individuals in Each Group} \label{tab:num-group-2}
\scalebox{0.77}{
\begin{tabular}{lccccc}
\toprule
			& 		\multicolumn{5}{c}{\textbf{Reggio}}		\\
			& (0,0) & (1,0) & (0,1) & (1,1) & Total \\ \midrule
Child		& 6 & 0 & \cellcolor{blue!25}66 & \cellcolor{blue!25}94 & \textbf{419} \\
Adolescent 	& 7 & 0 & \cellcolor{blue!25}39 &	\cellcolor{blue!25}93 & \textbf{299} \\
Age-30		& 57 & 0 & \cellcolor{blue!25}110 &	\cellcolor{blue!25}47 & \textbf{280} \\
Age-40		& 80 & 0 & \cellcolor{blue!25}90 &	\cellcolor{blue!25}26 & \textbf{285} \\ \bottomrule
\end{tabular}}
\end{table}

Similar to Section \ref{subsubsection:OLS-Preschool}, we also estimate (i) a propensity score matching model that implements nearest-neighbor matching on an estimated propensity score based on a BIC-selected set of observed baseline characteristics $\boldsymbol{X}_i$ and (ii) a matching model using Epanechnikov kernel weight and $\boldsymbol{X}_i$, in addition to OLS analysis for infant-toddler centers. 

%\subsubsection{Propensity Score Matching}  \label{subsubsection:psm}

%In order to complement the OLS analysis, we also estimate a propensity score matching model that implements nearest-neighbor matching on an estimated propensity score based on a BIC-selected set of observed baseline characteristics $\boldsymbol{X}_i$. Propensity score analysis is a version of non-parametric OLS and conditions on the same set of $\bm{X}$ variables as OLS. This approach is based on the assumption that two individuals with a very similar propensity score have the same level of unobservable characteristics, so that selection into treatment is determined solely by the estimated propensity score. \textbf{[Team: Our discussion of effects on outcomes in the results and discussion sections only takes into account outcomes that are robust across different regressions.  We have added text in the results section to make this more explicit.] [JJH: What does ``robust'' mean? OLS and propensity score should have the same outcomes.][Yes, we use the same outcomes. We highlight the robustness of the estimates across methods in the results section and have included kernel density matching for additional robustness checking.]} 

\subsection{Across-City Comparisons} \label{sec:across-city-analysis}
\subsubsection{Difference-in-Difference}  \label{subsubsection:DID}

\textbf{[JJH: Age-50 cohort comparisons isolate cross-city effects since no program was in place anywhere.][We now include this analysis.]}

We first estimate a difference-in-difference (DiD) model that allows for cross-city comparisons of municipal preschools while controlling for permanent differences in characteristics across cities. Similar to the OLS model, we estimate the parameters separately for each cohort. We only estimate parameters across cohorts when considering the age-50 cohort. This is because there was no program in place in any of the cities, allowing us to compare the other adult cohorts to a those who did not have the opportunity to participate in early childhood education. In the main paper, we present comparisons between municipal schools and (i) all other types of preschools pooled together, and (ii) no preschool. We present comparisons to specific school types in Appendix~\ref{app:comparison-reli-stat} and summarize the results in Section~\ref{sec:result}. \textbf{[JJH: We should summarize in text.][We have added summaries of these results in the results section.]}. 

For the age-40 cohort, we compare individuals who attended Reggio Approach preschools with those in Parma or Padova who attended any type of preschool. This is because municipal childcare systems were not available in Parma and Padova for the age-40 cohort. \textbf{[JJH: (?) Why is this different from 30 -- it's Reggio vs any preschool.][For the younger cohorts, we compare the difference between the municipal program and alternative preschools across cities. We also compare the difference between the municipal program and not attending preschool across cities. However, at age 40, Parma and Padova did not have municipal systems. For this cohort, we compare: (Reggio Approach - no preschool in Reggio Emilia) - (Other preschools in Parma - no preschool in Parma). We do the same for Padova. We edited this paragraph to try to clarify this difference.]}

To illustrate, we present the comparison between between Reggio Emilia and Parma for those who either attended municipal preschool or no preschool at all. The estimation equation for this case as follows:
\begin{eqnarray}  \label{eq:specific2}
Y_i & = \beta_0 + \beta_1 Reggio_i + \beta_2 D_i + \beta_3 Reggio_i * D_i + \bm{X}_i \bm{\delta} + \bm{X}_i \bm{D}_i \bm{\gamma} + \epsilon_i
\end{eqnarray}
\noindent \textbf{[JJH: You might want to interact $X_i D_i$ as well.][We have included the interaction]} where $Reggio_i$ is the indicator for individual $i$ having attended preschool in Reggio Emilia and $D_i$ is the indicator for attending municipal preschool. $\beta_3$ is interpreted as the difference that remains between municipal schools in Reggio Emilia and Parma after adjusting for permanent differences between the two cities. In other words, $\beta_3$ is the DiD treatment effect estimator that amounts to (Reggio Emilia municipal - Parma municipal) - (Reggio Emilia none - Parma none), where the first difference captures the unadjusted (for permanent city differences) difference between municipal preschools in each city, and the second difference captures the permanent differences between the two cities. Analogous interpretations are applied to DiD comparisons between Reggio Emilia and Padova and comparisons between municipal schools and other school types. This approach is valid under the assumption that the same regressors that account for selection into municipal preschools is comparable in the three cities, and the difference in the outcomes between municipal and non-municipal schools would have been the same in all three cities in the absence of the Reggio Approach.

\subsubsection{Matching}

The DiD model presented in Section \ref{subsubsection:DID} estimates the effect of municipal preschools relative to other types of preschool or no preschool across cities. However, selection into municipal preschools in Parma and Padova may not be analogous to selection into Reggio Approach preschools. In order to complement the DiD analysis, we estimate a propensity score matching model and a kernel matching model using Epanechnikov kernel weight to match people who attended the Reggio Approach preschools with people in Parma or Padova who attended (i) all types of preschools pooled together, including municipal preschools, or (ii) no preschool.

To illustrate, the comparison group for the matching models is limited to (i) people in Reggio who attended Reggio Approach preschools and (ii) people in Parma who attended any preschool. The purpose is to match Reggio Approach people with people who have similar propensity scores but have attended preschool in Parma. We assume that the latter group is similar to the Reggio Approach people except that they are not exposed to the Reggio Approach. By comparing the outcomes across the matches, the propensity score matching model estimates the effect of the Reggio Approach. Analogous interpretations are applied to comparisons for different control group specifications, including people in Padova. \textbf{[JJH: I thought we also had IV and selection bias results -- what happened?]} \textbf{[Team: We do not report an IV approach in this paper for the following reasons:
\begin{enumerate*}[label=(\roman*)]
\item The IV approach in the second paper --which focused on Children and Adolescents-- used distance to school, mother's place of birth, whether mother attended childcare, and Reggio Score (approximation of scores that would have been assigned by Reggio Approach based on background characteristics) as instruments. However, the first-stage was very weak.
\item The distance variable is not an appropriate instrument for adult cohorts because distance is calculated from residence at time of interview which could be different than residence when individual was attending childcare.
\item We explored cost as an instrument, but weren't able to find the schedule of costs for adult cohorts.
\end{enumerate*}
] [JJH: But report the IV estimates in an appendix and test if they are different from propensity score -- use the likely valid instruments.][We have included this analysis in Appendix~\ref{sec:iv}.]} 