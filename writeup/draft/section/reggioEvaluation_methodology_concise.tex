Since no single analytic approach is best suited for this particular treatment evaluation, we consider several methodologies to evaluate the effect of the Reggio Approach, using the survey data based on background information on different school systems presented in Section \ref{sec:eceexperiences} and data availability presented in Section \ref{sec:data}. All these methodologies require different identification assumptions and leverage different control groups. An effect robustly estimated across these methodologies will provide strong evidence in favor of the validity of the result.

Our analysis is carried out in two dimensions. First, we compare the Reggio Approach with other childcare systems within Reggio Emilia. Section \ref{sec:within-city-analysis} presents various methodologies used to estimate the treatment effects of the Reggio Approach with a restriction of sample to individuals in Reggio Emilia. Second, we estimate the effect of Reggio Approach relative to other childcare systems across cities. Section \ref{sec:across-city-analysis} presents methodologies used for the across-city analysis.

Although the Reggio Approach includes (i) infant-toddler center between ages 0-3 and (ii) preschool between ages 3-6, we focus on the evaluation of the preschool intervention at ages 3-6 in the main paper. This is because too few people attended infant-toddler centers in the adult cohorts,\footnote{See Appendix Table \ref{tab:sample-asilo} for the distribution of individuals who attended each type of infant-toddler centers.} and virtually no other type of center provided child-care for children younger than three.
%our data is not appropriate to obtain estimates for infant-toddler centers due to small sample size of adults who attended infant-toddler centers\footnote{See Appendix Table \ref{tab:sample-asilo} for the distribution of individuals who attended each type of infant-toddler centers.}. \textbf{[YKK: Sylvi will add more reasons why we do not analyze asilo on its own in the main paper. YKK will estimate the treatment effects for 0-6 Reggio Approach for younger cohorts this weekend, because Reggio Approach emphasizes their approach being 0-6 as a whole.]} We present our strategy for evaluating ITC education and the corresponding results in Appendix~\ref{sec:ITC}.\footnote{The estimation results show that Reggio Approach ITCs do not have significant effects for children and adolescent cohorts, and significantly negative effects on cognitive skill and high school grade for adult cohorts (See Tables \ref{ols-M-child-reg-nopres-asilo} - \ref{ols-M-adult40-reg-nopres-asilo}).}


\subsection{Within-City Analysis} \label{sec:within-city-analysis}
\subsubsection{Baseline OLS} \label{subsubsection:OLS}
We perform several within-Reggio Emilia comparisons using a baseline OLS model. We compare individuals from Reggio Emilia who attended a Reggio Approach preschool to those in Reggio Emilia who attended (i) any type of preschool (state, religious, municipal-affiliated, and other), (ii) no preschool, (iii) state preschool, and (iv) religious preschool. We only present estimates of the first and second comparisons in the main paper. Estimates from the other comparisons are reported in Appendix~\ref{app:comparison-reli-stat}. %We do not compare Reggio Approach preschools with municipal-affiliated or other types of preschools due to small sample size (See Table \ref{tab:sample}). 
For the children cohort, it is not possible to compare Reggio Approach preschools with no preschool because of small sample size for those who did not attend any preschool (See Table \ref{tab:sample}). We also exclude the cohort of adults in their 50s because the Reggio Approach was not available at that time. 

The basic OLS takes the form, 

\begin{equation}
	Y_i = \alpha_0 + \alpha_1 D_i + \bm{X}_i\gamma + \varepsilon_i
	\label{eq:ra-v-none}
\end{equation}
where $i$ indexes individuals having attended preschool in Reggio Emilia, $D_i$ is an indicator for whether individual $i$ attended municipal preschool, $\bm{X}_i$ is a vector of baseline control variables, and $\varepsilon_i$ is a random disturbance. Estimations for three specifications for $\bm{X}_i$ are reported: (i) no baseline control, (ii) baseline variables selected by the Bayesian Information Criterion (BIC)\footnote{These variables are: number of siblings, an indicator if the mother's maximum education was middle school, an indicator if the father's maximum education is university, and two indicators for the number of siblings (one indicating having two or more siblings, the other indicating having three or more siblings).} in addition to indicators for male, CAPI, and infant-toddler center attendance, and (iii) full set of all available baseline variables. In Equation~\ref{eq:ra-v-none}, $\alpha_1$ represents the mean differences in outcomes between the Reggio Approach and the other preschool types in Reggio Emilia, conditional on $X$. Under the assumption that, conditional on $X$, there is no systematic selection of individuals into the treatment $D_i$, this parameter estimates the causal treatment effect of the Reggio Approach on outcome $Y$.

\subsubsection{Propensity Score Matching}  \label{subsubsection:psm}

In order to complement the OLS analysis, we also estimate a propensity score matching model that implements nearest-neighbor matching on an estimated propensity score based on a BIC-selected set of observed baseline characteristics $\boldsymbol{X}_i$. This approach is based on the assumption that two individuals with a very similar propensity score have the same level of unobservable characteristics, so that selection into treatment is determined solely by the estimated propensity score. %The benefits of propensity score matching over regression is that it is nonparametric. Furthermore, the use of the propensity score allows us to reduce the dimensionality of the matching exercise to a single dimension.

%Let $Y_{1,i}$ and $Y_{0,i}$ represent counterfactual outcomes for individual, $i$, where treatment status is fixed to $D_i=1$ and $D_i=0$ respectively. Under this framework, the realized outcome $Y_i$ is summarized by,
%\begin{equation}\label{eq:roy}
%Y_i = (1-D_i)Y_{0,i} + D_iY_{1,i}.
%\end{equation}
The average treatment effect (ATE) under the assumption for propensity score matching is written as:
\begin{equation} \label{eq:ATE-PSM}
E[Y(1)-Y(0)] = E[E[Y_i|D_i=1, \hat{\pi(\boldsymbol{X}_i)}] - E[Y_i|D_i=0, \hat{\pi(\boldsymbol{X}_i)}]].
\end{equation}
where the propensity score $\hat{\pi({\boldsymbol{X}_i})} = Pr(D_i=1|\boldsymbol{X}_i)$ is predicted for each individual $i$ using the estimated coefficients obtained from a probit model.

The propensity score matching estimator is defined as
\begin{equation} \label{eq:PSM-estimator}
\widehat{E[Y(1)-Y(0)]_{PSM}} = \frac{1}{n} \sum_{i=1}^{n} (2D_i -1)(Y_i - \frac{1}{M}\sum_{j \in \mathcal{J}_M(i)}Y_j )
\end{equation}
where $M$ is a fixed number of matches per individual based on the propensity score and $\mathcal{J}_M(i)$ is a set of matches for individual $i$. The standard errors for this estimator are derived by \cite{Abadie_Imbens_2006_Econometrica}.

\subsubsection{Augmented IPW} \label{subsubsection:aipw}

As another way of complementing the OLS estimates, we estimate an Augmented Inverse Propensity Weighted (AIPW) estimator. As in propensity score matching, we assume that individuals select into treatment status $D_i$ based on a BIC-selected set of observed baseline characteristics $\boldsymbol{X}_i$. Under the assumption of strongly ignorable selection into treatment given $\boldsymbol{X}_i$, and the existence of an estimated propensity score $\hat{\pi({\boldsymbol{X}_i})}$ such that $0<\hat{\pi({\boldsymbol{X}_i})}<1$, the AIPW estimator for the average treatment effect $E[Y(1)-Y(0)]$ is defined as:

\begin{align}\label{eq:AIPW}
E[\widehat{Y(1)-Y(0)]}_{AIPW} = \frac{1}{n} \sum_{i=1}^{n} \bigg \{ \bigg[ & \overbrace{\frac{D_i Y_i}{\hat{\pi(\boldsymbol{X}_i)}} - \frac{(1-D_i)Y_i}{1-\hat{\pi(\boldsymbol{X}_i)}}}^{IPTW} \bigg]- \frac{D_i - \hat{\pi}(\boldsymbol{X}_i)}{\hat{\pi}(\boldsymbol{X}_i) (1-\hat{\pi}(\boldsymbol{X}_i))} \nonumber \\[10pt]
& \bigg[ (1-\hat{\pi}(\boldsymbol{X}_i)) E[\hat{Y_i}|D_i=1,\boldsymbol{X}_i] + \hat{\pi}(\boldsymbol{X}_i) E[\hat{Y_i}|D_i=0,\boldsymbol{X}_i] \bigg] \bigg \}
\end{align}

Note that the first part of the AIPW estimator equals to the Inverse Propensity of Treatment Weighting (IPTW) estimator, which adjusts for selection on observable by assigning a higher weight to individuals with characteristics $\boldsymbol{X}_i$ if these individuals are less likely to have been in treatment group $D = 1$ and assigning a higher weight to individuals with characteristics $\boldsymbol{X}_i$ if these individuals are less likely to have been in treatment group $D = 0$. The AIPW estimator further addresses the problem of instability associated with the IPTW estimator near $\hat{\pi(\boldsymbol{X}_i)} = 0$ or $1$, by adjusting the weighted estimator with a weighted average of $\hat{Y_i}$ obtained from an OLS regression. For $D_i = 0$, as the IPTW estimate becomes large and negative near $\hat{\pi(\boldsymbol{X}_i)} = 1$, the adjustment term becomes large and positive, thereby, stabilizing the estimator at these values. Analogously, for $D = 1$, as the IPTW estimate becomes large and positive near $\hat{\pi(\boldsymbol{X}_i)} = 0$, the adjustment term becomes large and negative.\footnote{\citet{Tsiatis_2006_Semiparametric-Theory} and \citet{Glynn-Quinn_2010_Political-Analysis} show that the AIPW estimator is an unbiased estimator of the Average Treatment Effect if the propensity score is known. Further, they show that the estimator is consistent if either the propensity score model or the outcome prediction model in the adjustment term is correctly specified. Due to this last property, they refer to the AIPW as a doubly robust estimator.}  Hence, the AIPW is a way to adjust the potential instability of IPTW.


\subsection{Across-City Comparison} \label{sec:across-city-analysis}
\subsubsection{Difference-in-Difference}  \label{subsubsection:DID}
Next, we estimate a difference-in-difference (DiD) model that allows for cross-city comparisons of municipal preschools while controlling for permanent differences in characteristics between cities. Similar to the OLS model, we estimate the parameters separately for each cohort. In the main paper, we present comparisons between municipal schools and (i) all other types of preschools pooled together or (ii) no preschool. We present comparisons to specific school types in Appendices~\ref{app:comparison-reli-stat}. %We exclude age-50 cohort as municipal childcare systems were not available for all three cities. %%PB This is the same as before, no need to repeat 
For age-40 cohort, we only compare people who attended Reggio Approach preschools with people in Parma or Padova who attended any type of preschools because municipal childcare systems were not available in both Parma and Padova. 

To illustrate, we present the comparison between between Reggio Emilia and Parma for those who either attended municipal preschool or no preschool at all. We can write the estimation equation for this case as follows:
\begin{eqnarray}  \label{eq:specific2}
Y_i & = \beta_0 + \beta_1 Reggio_i + \beta_2 D_i + \beta_3 Reggio_i * D_i + \bm{X}_i\delta + \epsilon_i
\end{eqnarray}
\noindent where $Reggio_i$ is the indicator for individual $i$ having attended preschool in Reggio Emilia and $D_i$ is the indicator for attending municipal preschool. $\beta_3$ is interpreted as the difference that remains between municipal schools in Reggio Emilia and Parma after adjusting for permanent differences between the two cities. In other words, $\beta_3$ is the DiD treatment effect estimator that amounts to (Reggio Emilia municipal - Parma municipal) - (Reggio Emilia none - Parma none), where the first difference captures the unadjusted (for permanent city differences) difference between municipal preschools in each city, and the second difference captures the permanent differences between the two cities. Analogous interpretations are applied to DiD comparisons between Reggio Emilia and Padova and comparisons between municipal schools and other school types. This approach is valid under the assumption that the selection into municipal preschools is comparable in the three cities, and the difference in the outcomes between municipal and non-municipal schools would have been the same in all three cities in the absence of the Reggio Approach.

\subsubsection{Propensity Score Matching}

The DiD model presented in Section \ref{subsubsection:DID} estimates the effect of municipal preschools relative to other types of preschool or no preschool across cities. However, selection into municipal preschools in Parma and Padova may not be analogous to selection into Reggio Approach preschools. In order to complement the DiD analysis, we estimate the propensity score matching model to match people who attended the Reggio Approach preschools with people in Parma or Padova who attended (i) all types of preschools pooled together, including municipal preschools, or (ii) no preschool. 

To illustrate, the propensity score matching estimator for the comparison between Reggio Approach and all types of pooled together in Parma is formally written as:
\begin{equation} \label{eq:PSM-estimator-across-city}
\widehat{E[Y(1)-Y(0)]_{PSM}} = \frac{1}{n} \sum_{i=1}^{n} (2D_i -1)(Y_i - \frac{1}{M}\sum_{j \in \mathcal{J}_M(i)}Y_j )\} 
\end{equation}
where $i$ is limited to (i) people in Reggio who attended Reggio Approach preschools and (ii) people in Parma who attended any preschool. $D_i$ is an indicator for attending a Reggio Approach preschool. Note that Equation \ref{eq:PSM-estimator-across-city} is analogous to Equation \ref{eq:PSM-estimator}. The purpose is to match Reggio Approach people with people who have similar propensity scores but have attended preschool in Parma. We assume that the latter group is similar to the Reggio Approach people except that they are not exposed to the Reggio Approach. By comparing the outcomes across the matches, the propensity score matching model estimates the effect of the Reggio Approach. Analogous interpretations are applied to comparisons for different control group specifications.