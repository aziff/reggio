The Reggio Approach includes two stages of early childhood education: (i) infant-toddler center between ages 0-3 and (ii) preschool between ages 3-6. Table~\ref{tab:cases-treat} shows the four possible combinations of interventions that a child could potentially receive, where 0 indicates not attending and 1 indicates attending. It is important to note that the 1 case only includes those who attended municipal institutions, and 0 includes those who who didn't attend any childcare as well as those who attended non-municipal childcare.

\begin{table}[H]
\caption{Possible Cases of Treatment} \label{tab:cases-treat}
\begin{tabular}{C{1.8cm} R{0.7cm} C{2cm} C{2cm}}
  
		& & \multicolumn{2}{c}{Preschool (Ages 3-6)} \\
		& & 0 & 1 \\ \cline{3-4}            
        								 &  & \multicolumn{1}{|c|}{} & \multicolumn{1}{c|}{} \\
        							& 0 & \multicolumn{1}{|c|}{(0,0)} & \multicolumn{1}{c|}{(0,1)} \\ 
        				ITC				&  & \multicolumn{1}{|c|}{} & \multicolumn{1}{c|}{} \\ \cline{3-4}
                        (Age 0-3)  		&  & \multicolumn{1}{|c|}{} & \multicolumn{1}{c|}{} \\
        								& 1 & \multicolumn{1}{|c|}{(1,0)} & \multicolumn{1}{c|}{(1,1)} \\ 
        								&  & \multicolumn{1}{|c|}{} & \multicolumn{1}{c|}{} \\ \cline{3-4}
\end{tabular}
\begin{flushleft}
\footnotesize{Note: We only consider municipal ITCs (infant-toddler centers, ages 0-3) and preschools (ages 3-6). (0,0): did not attend any municipal school for both ages 0-3 and 3-6; (1,0): attended a municipal school for ages 0-3 but did \textit{not} attend for ages 3-6; (0,1): did \textit{not} attend a municipal school for ages 0-3 but did attend for ages 3-6; (1,1): attended a municipal school for both ages 0-3 and 3-6.}
\end{flushleft}
\end{table}

%The distribution of individuals who experienced each combination in our data is reported in Table \ref{tab:num-group} for by cohort and city. It is immediately apparent that the (1,0) case, in which a child attends municipal ITC but not the municipal Preschool, is virtually non-existent over time and across cities.  The (0,0) case, in which a child receives neither intervention, undergoes a sharp decline in frequency between the adult and younger cohorts across cities.

Because it is uncommon for people in the adult cohorts to have enrolled in infant-toddler centers, we have only small number of individuals among adult cohorts who attended infant-toddler centers (See Appendix Table \ref{tab:sample-asilo} for the distribution of individuals who attended each type of infant-toddler centers). We present our strategy for evaluating ITC education and the corresponding results in Appendix~\ref{sec:ITC}.
The estimation results show that Reggio Approach ITCs do not have significant effects for children and adolescent cohorts, and significantly negative effects on cognitive skill and high school grade for adult cohorts (See Tables \ref{ols-M-child-reg-nopres-asilo} - \ref{ols-M-adult40-reg-nopres-asilo}). We focus the remainder of this paper on the evaluation of the preschool intervention at ages 3-6 controlling the effects of attending ITCs. 
%\subsection{Estimating Effects of Infant-Toddler Centers}
%There are two main methods to test the effect of attending infant-toddler centers. The first is to compare people who did not attend infant-toddler care or preschool with people who only attended municipal infant-toddler care. Using the notation in Table~\ref{tab:cases-treat}, this comparison is between (0,0) and (1,0). The second method is to compare people who only attended municipal preschool with people who attended both municipal infant-toddler centers and preschools. That is, to compare (0,1) and (1,1). The hypotheses are formally written as
%\begin{eqnarray}
%H_1: &  Y_{0,0} = Y_{1,0} \\ 
%H_2: &  Y_{0,1} = Y_{1,1} 
%\end{eqnarray}
%\noindent where $Y_{i,j}$ is the outcome of the individuals who attended $i \in \{0,1\}$ infant-toddler care and $j \in \{0,1\}$ preschool.
%
%A possible estimation strategy is to limit the sample to a specific city and a specific cohort constrained to the comparison groups needed according to the hypotheses above. To test $H_1$, we estimate the following regression equation:
%\begin{eqnarray}
%Y_{i}^{c,h} & = & \alpha + \beta_{0}R_i^{ITC} + \mathbf{X}_i\gamma + \varepsilon_{i}^{c,h}, \\ \nonumber
%& \forall & i \in \text{ \{People in city $c$ and cohort $h$ and in group (0,0) or (1,0)\}}
%\end{eqnarray}
%where $R_i^{ITC}$ is the indicator for attending municipal infant-toddler center and $\mathbf{X_i}$ is the vector of baseline variables for individual $i$. Likewise, to test $H_2$:
%\begin{eqnarray}
%Y_{i}^{c,h} & = & \alpha + \beta_{0}R_i^{ITC} + \mathbf{X}_i\gamma + \varepsilon_{i}^{c,h}, \\ \nonumber
%& \forall & i \in \text{ \{People in city $c$ and cohort $h$ and in group (0,1) or (1,1)\}.}
%\end{eqnarray}
%
%One caveat of this analysis is that it uses a limited sample size. In our data, these hypotheses cannot be tested under this strategy for many groups. Table~\ref{tab:num-group} shows the number of individuals available for each group necessary for analysis using this strategy. It is impossible to test $H_1$ in our data, because there are almost no people who attended municipal infant-toddler care without attending preschool (the group (1,0)). While it is possible to test $H_2$ for several groups, the number of observations for the group (1,1) is small for the adult cohorts. 
%
%In Table \ref{tab:num-group}, the groups subject to our estimation are highlighted. Based on the available number of individuals in each cell and the history of the foundation date of municipal infant-toddler care for each city, we decide to test $H_2$ for the highlighted groups.
%
%\begin{table}[H] \caption{Number of Individuals in Each Group} \label{tab:num-group}
%\scalebox{0.77}{
%\begin{tabular}{l|ccccc|ccccc|ccccc}
%\toprule
%			& 		\multicolumn{5}{c}{\textbf{Reggio}}		& 	\multicolumn{5}{|c|}{\textbf{Parma}}	& 			\multicolumn{5}{c}{\textbf{Padova}}				\\
%			& (0,0) & (1,0) & (0,1) & (1,1) & Total & (0,0) & (1,0) & (0,1) & (1,1) & Total  & (0,0) & (1,0) & (0,1) & (1,1) & Total \\ \midrule
%Child		& 2 & 0 & \cellcolor{blue!25}46 & \cellcolor{blue!25}117 & \textbf{311} & 5 & 1 & \cellcolor{blue!25}35 & \cellcolor{blue!25}100 & \textbf{291} & 2 & 0 & \cellcolor{blue!25}31 & \cellcolor{blue!25}36 & \textbf{278} \\
%Migrant		& 4 & 0	& 24 & 26 & \textbf{110} & 4 & 0 & 12 & 23 & \textbf{58} & 5 & 0 & 18 & 16 & \textbf{113} \\
%Adolescent 	& 7 & 0 & \cellcolor{blue!25}45 &	\cellcolor{blue!25}116 & \textbf{300} & 4 & 0 & \cellcolor{blue!25}49 & \cellcolor{blue!25}61 & \textbf{254} & 1 & 0 & \cellcolor{blue!25}55 & \cellcolor{blue!25}37 & \textbf{282} \\
%Age-30		& 57 & 0 & \cellcolor{blue!25}95 &	\cellcolor{blue!25}53 & \textbf{280} & 43 & 0 & \cellcolor{blue!25}64 & \cellcolor{blue!25}29 & \textbf{251} & 47 & 0 & 25 & 9 & \textbf{251} \\
%Age-40		& 80 & 0 & \cellcolor{blue!25}97 &	\cellcolor{blue!25}28 & \textbf{285} & 115 & 1 & 35 & 16 & \textbf{254} & 75 & 0 & 25 & 2 & \textbf{252} \\
%Age-50		& 146 & 0 &	8 & 0 & \textbf{200} & 71 & 0 & 4 & 8 & \textbf{103} & 55 & 0 & 11 & 0 & \textbf{146} \\ \bottomrule
%\end{tabular}}
%\end{table}


\subsection{Estimating Effects of Preschools}
We adopt multiple approaches to evaluate the effects of the Reggio Approach preschool intervention. In Section~\ref{subsubsection:OLS}, we describe an OLS model that allows us to compare the outcomes of those who attended municipal preschool in Reggio Emilia to those from Reggio Emilia who did not attend municipal preschool. In Section~\ref{subsubsection:DID}, we present a difference-in-differences (DiD) approach that allows us to make cross-city comparisons for those who attended  municipal preschool in each city, while controlling for permanent differences between cities. In Section~\ref{subsubsection:psm}, we describe a propensity score matching model to calculate treatment effects nonparametrically. In Section~\ref{subsubsection:aipw}, we present an AIPW model that weights individuals by their propensity to attend municipal preschool. In Appendix~\ref{sec:multi-logit} presents a multinomial logit model to test for the presence of selection into treatment.

\subsubsection{Baseline OLS} \label{subsubsection:OLS}
We perform several within-Reggio Emilia comparisons using a baseline OLS model. We compare individuals from Reggio Emilia who attended a Reggio Approach preschool to those in Reggio Emilia who attended (i) any type of preschool, (ii) no preschool, (iii) state preschool, and (iv) religious preschool. We only present estimates of the first and second comparisons in the main paper. Estimates from the other comparisons are reported in Appendix~\ref{app:comparison-reli-stat}. We exclude the cohort of adults in their 50s because the Reggio Approach was not available at that time.

The basic OLS takes the form, 

\begin{equation}
	Y_i = \alpha_0 + \alpha_1 D_i + \bm{X}_i\gamma + \varepsilon_i
	\label{eq:ra-v-none}
\end{equation}
where $i$ indexes individuals having attended preschool in Reggio Emilia, $D_i$ is an indicator for whether individual $i$ attended municipal preschool, $\bm{X}_i$ is a vector of baseline control variables\footnote{These variables are: gender, whether the individual attended infant-toddler center, whether the individual took the computer-assisted (CAPI), number of siblings, an indicator if the mother's maximum education was middle school, an indicator if the father's maximum education is university, and two indicators for the number of siblings (one indicating having two or more siblings, the other indicating having three or more siblings).} selected based on the Bayesian Information Criterion (BIC), and $\varepsilon_i$ is a random disturbance. In Equation~\ref{eq:ra-v-none}, $\alpha_i$ is the OLS treatment effect parameter.

\subsubsection{Difference-in-difference}  \label{subsubsection:DID}
Next, we estimate a difference-in-difference (DiD) model that allows for cross-city comparisons of municipal preschools while controlling for permanent differences in characteristics between cities. Similar to the OLS model, we estimate the parameters separately for each cohort. In the main paper, we present comparisons between municipal schools and (i) all other types of preschools pooled together or (ii) no preschool. We present comparisons to specific school types in Appendices~\ref{app:comparison-reli-stat}.

To illustrate, we present the comparison between between Reggio Emilia and Parma for those who either attended municipal preschool or no preschool at all. We can write the estimation equation for this case as follows:
\begin{eqnarray}  \label{eq:specific2}
Y_i & = \beta_0 + \beta_1 Reggio_i + \beta_2 D_i + \beta_3 Reggio_i * D_i + \bm{X}_i\delta + \epsilon_i
\end{eqnarray}
\noindent where $Reggio_i$ is the indicator for individual $i$ having attended preschool in Reggio Emilia and $\beta_3$ is interpreted as the difference that remains between municipal schools in Reggio Emilia and Parma after adjusting for permanent differences between the two cities. In other words, $\beta_3$ is the DiD treatment effect estimator that amounts to (Reggio Emilia municipal - Parma municipal) - (Reggio Emilia none - Parma none), where the first difference captures the unadjusted (for permanent city differences) difference between municipal preschools in each city, and the second difference captures the permanent differences between the two cities. Analogous interpretations are applied to DiD comparisons between Reggio Emilia and Padova and comparisons between municipal schools and other school types.


\subsubsection{Propensity Score Matching}  \label{subsubsection:psm}

We estimate a propensity score matching model that implements nearest-neighbor matching on an estimated propensity score based on a set of observed baseline characteristics $\boldsymbol{X}_i$. This approach is based on the assumption that adjusting for the propensity score is sufficient to eliminate all confounding. The benefits of propensity score matching over regression is that it is nonparametric and it reduces the dimensionality of matching to a single dimension.

Let $Y_{1,i}$ and $Y_{0,i}$ represent counterfactual outcomes for individual, $i$, where treatment status is fixed to $D_i=1$ and $D_i=0$ respectively. Under this framework, the realized outcome $Y_i$ is summarized by,
\begin{equation}\label{eq:roy}
Y_i = (1-D_i)Y_{0,i} + D_iY_{1,i}.
\end{equation}
The average treatment effect under the assumption for propensity score matching is written as:
\begin{equation} \label{eq:ATE-PSM}
E[Y(1)-Y(0)] = E[E[Y_i|D=1, \hat{\pi(\boldsymbol{X}_i)}] - E[Y_i|D=0, \hat{\pi(\boldsymbol{X}_i)}]].
\end{equation}
where the propensity score $\hat{\pi({\boldsymbol{X}_i})} = Pr(D_i=1|\boldsymbol{X}_i)$ is predicted for each individual $i$ using the estimated coefficients obtained from a probit model.

The propensity score matching estimator is defined as
\begin{equation} \label{eq:PSM-estimator}
\widehat{E[Y(1)-Y(0)]_{PSM}} = \frac{1}{n} \sum_{i=1}^{n} (2D_i -1)(Y_i - \frac{1}{M}\sum_{j \in \mathcal{J}_M(i)}Y_j )
\end{equation}
where $M$ is a fixed number of matches per individual and $\mathcal{J}_M(i)$ is a set of matches for individual $i$. The standard errors for this estimator are derived by \cite{Abadie_Imbens_2006_Econometrica}.

\subsubsection{Augmented IPW} \label{subsubsection:aipw}

We estimate an Augmented Inverse Propensity Weighted (AIPW) estimator in order to account for the possibility that regression estimator for ATE may be poorly estimated when the observed values of baseline variables are not similar for the treatment and control groups.
%\citep{Glynn_2010}. 

%To begin, consider the following two treatment statuses,
%\begin{equation}\label{eq:cases-d}
%D_i = \quad
%\begin{cases}
%1 \quad if \quad &  R_i = 0 \text{\quad 		($i$ from Reggio Emilia but did not attend Reggio Approach)} \\
%0 \quad if \quad &  R_i \neq 0 \text{\quad 	($i$ from Reggio Emilia and attended the Reggio Approach).} \\
%\end{cases}
%\end{equation}
%~\\


We assume that individuals select into treatment status $D_i$ based on a set of observed baseline characteristics $\boldsymbol{X}_i$. Under the assumption of strongly ignorable selection into treatment given $\boldsymbol{X}_i$, and the existence of an estimated propensity score $\hat{\pi({\boldsymbol{X}_i})}$ such that $0<\hat{\pi({\boldsymbol{X}_i})}<1$, we can use standard Inverse Propensity of Treatment Weighting (IPTW) to estimate the average treatment effect $E[Y(1)-Y(0)]$,

\begin{equation}\label{eq:IPW}
\widehat{E[Y(1)-Y(0)]_{IPW}} = \frac{1}{n} \sum_{i=1}^{n} \left \{\frac{D_i Y_i}{\hat{\pi(\boldsymbol{X}_i)}} - \frac{(1-D_i)Y_i}{1-\hat{\pi(\boldsymbol{X}_i)}}\right \}
\end{equation}

In calculating $\widehat{E[Y(1)]}$, the IPTW estimator assigns a higher weight to individuals with characteristics $\boldsymbol{X}_i$ if these individuals are less likely to have been in treatment group $D = 1$. Analogously, in calculating $\widehat{E[Y(0)]}$,  the IPTW estimator assigns a higher weight to individuals with characteristics $\boldsymbol{X}_i$ if these individuals are less likely to have been in treatment group $D = 0$. In this sense, the estimator adjusts for selection on observables by balancing the estimation of expected outcomes on $\boldsymbol{X}$.

An issue with the simple IPTW estimator is that the estimates become unstable as the estimated propensities, $\hat{\pi(\boldsymbol{X}_i)}$, tend to 0 or 1. We use the following Augmented IPW (AIPW) estimator to correct for this issue. 
\begin{align}\label{eq:AIPW}
E[\widehat{Y(1)-Y(0)]}_{AIPW} = \frac{1}{n} \sum_{i=1}^{n} \bigg \{ \bigg[ & \overbrace{\frac{D_i Y_i}{\hat{\pi(\boldsymbol{X}_i)}} - \frac{(1-D_i)Y_i}{1-\hat{\pi(\boldsymbol{X}_i)}}}^{IPTW} \bigg]- \frac{D_i - \hat{\pi}(\boldsymbol{X}_i)}{\hat{\pi}(\boldsymbol{X}_i) (1-\hat{\pi}(\boldsymbol{X}_i))} \nonumber \\[10pt]
& \bigg[ (1-\hat{\pi}(\boldsymbol{X}_i)) E[\hat{Y_i}|D_i=1,\boldsymbol{X}_i] + \hat{\pi}(\boldsymbol{X}_i) E[\hat{Y_i}|D_i=0,\boldsymbol{X}_i] \bigg] \bigg \}
\end{align}

The AIPW estimator in Equation (\ref{eq:AIPW}) addresses the problem of instability associated with the IPTW estimator near $\hat{\pi(\boldsymbol{X}_i)} = 0$ or $1$, by adjusting the weighted estimator with a weighted average of $\hat{Y_i}$ obtained from an OLS regression. For $D_i = 0$, as the IPTW estimate becomes large and negative near $\hat{\pi(\boldsymbol{X}_i)} = 1$, the adjustment term becomes large and positive, thereby, stabilizing the estimator at these values. Analogously, for $D = 1$, as the IPTW estimate becomes large and positive near $\hat{\pi(\boldsymbol{X}_i)} = 0$, the adjustment term becomes large and negative.\footnote{\citet{Tsiatis_2006_Semiparametric-Theory} and \citet{Glynn-Quinn_2010_Political-Analysis} show that the AIPW estimator is an unbiased estimator of the Average Treatment Effect if the propensity score is known. Further, they show that the estimator is consistent if either the propensity score model or the outcome prediction model in the adjustment term is correctly specified. Due to this last property, they refer to the AIPW as a doubly robust estimator.}  Hence, the AIPW is a way to adjust the potential instability of IPTW.

%\subsubsection{AIPW Using Multinomial Control Selection Model}
%\label{sec:aipw-mlogit}
%An issue with the AIPW estimator in Equation (\ref{eq:AIPW}) is that the control group, $D_i=0$, includes all individuals who did not attend Reggio Municipal schools. As a result, we are unable to account for potential heterogeneity in treatment effects with respect to control group individuals who attended different types of preschools. In the rest of this subsection, we present an extension of the two-outcome AIPW that takes into account the non-binary nature of treatment statuses in our sample. 
%
%To account for this heterogeneity, we can start by defining a new variable for treatment status,
%
%\begin{equation}
%T_i = \quad
%\begin{cases}
%0 \quad \text{if \quad (\textit{i} from Reggio Emilia and attended no preschool)} \\
%1 \quad  \text{if \quad (\textit{i} from Reggio Emilia and attended other preschool)} \\
%2 \quad \text{if \quad (\textit{i} from Reggio Emilia and attended Reggio Approach)}  \\
%\end{cases}
%\end{equation}
%~\\ ~\\
%Let $Y_{j,i}$ denote the counterfactual outcome for individual $i$ when treatment status $T_i$ is fixed to $j$ $\forall \,  j \, \in \{0,1,2\}$. Under this framework, individual $i$'s potential outcomes can be represented as,
%\begin{equation}
%Y_i = \sum_{j=0}^{2}(\mathbb{1}_{T_i = j}) Y_{j,i}
%\end{equation}
%\noindent Where $\mathbb{1}_{T_i=j}$ is an indicator that equals one if the treatment status of individual $i$ is set to $j$.
%
%\noindent The average treatment effect with respect to each control group, $j$, can be defined,
%\begin{equation}
%ATE_j = E[Y(2) - Y(j)] \quad \forall j \in \{0,1\}
%\end{equation}
%Thus, the IPW estimator of the treatment effect with respect to each control group, $j$, can be defined as, 
%
%\begin{equation}\label{eq:IPWmulti}
%\widehat{IPW}_j=\widehat{E[Y(2)-Y(j)]}_{IPW} = \frac{1}{n} \sum_{i=1}^{n} \left \{\frac{(\mathbb{1}_{T_i=2}) Y_i}{\hat{\Pi_2}(\boldsymbol{X}_i)} - \frac{(\mathbb{1}_{T_i=j})Y_i}{\hat{\Pi_j}(\boldsymbol{X}_i)}\right \}
%\end{equation}
%~\\
%\noindent Where $\hat{\Pi_j}(\boldsymbol{X}_i) = Pr(T_i=j|\boldsymbol{X}_i) \quad \forall \, j \in \{0,1,2\}$, is predicted for each individual, $i$, using the estimated coefficients obtained from a multinomial logit model. 
%
%Similarly, we may define the AIPW estimator of the treatment effect with respect to each control group, $j$, as,
%
%\begin{align}\label{eq:AIPWmulti}
%\widehat{AIPW_{j}} = E[\widehat{Y(2)-Y(j)]}_{AIPW} =& \frac{1}{n} \sum_{i=1}^{n}\bigg \{ \bigg[ \overbrace{ \frac{(\mathbb{1}_{T_i=2}) Y_i}{\hat{\Pi_2}(\boldsymbol{X}_i)} - \frac{(\mathbb{1}_{T_i=j})Y_i}{\hat{\Pi_j}(\boldsymbol{X}_i)}}^{IPTW}  \bigg] - \frac{(\mathbb{1}_{T_i=2}) - \hat{\Pi}_2(\boldsymbol{X}_i)}{\hat{\Pi}_2(\boldsymbol{X}_i) \, \hat{\Pi}_j(\boldsymbol{X}_i)} \nonumber \\[10pt]
%& \bigg[ \hat{\Pi}_j(\boldsymbol{X}_i) \cdot E[\hat{Y_i}|T_i=2,\boldsymbol{X}_i] +\hat{\Pi}_2(\boldsymbol{X}_i) \cdot E[\hat{Y_i}|T_i=j,\boldsymbol{X}_i] \bigg] \bigg \}
%\end{align}

