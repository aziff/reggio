Since no single analytic approach is best, we consider several methodologies to evaluate the effect of the Reggio Approach, using the survey data based on background information on different school systems discussed in Section \ref{sec:ece-italy} and the data on curricula in Section \ref{sec:data}. These methodologies require different identifying assumptions and leverage different control groups. An effect robustly estimated across these methodologies would provide strong evidence in favor of the validity of any such conclusion that might emerge.

Our analysis is carried out in two dimensions. First, we compare the Reggio Approach with other childcare systems within the city of Reggio Emilia. Section \ref{sec:within-city-analysis} presents various methodologies used to estimate the treatment effects of the Reggio Approach with a restriction of the sample to individuals within the city of Reggio Emilia. \textbf{[JJH: Isn't Reggio a town and a province?]} \textbf{[Team: Yes, it is both a city/municipality and a province. We now clarify in the previous sentences that we are referring to the city of Reggio Emilia when we speak of Reggio Emilia.]} Second, we estimate the effect of Reggio Approach relative to other childcare systems across cities. Section \ref{sec:across-city-analysis} presents methodologies used for the across-city analysis.

The Reggio Approach includes interventions at two different age ranges: (i) infant-toddler centers between ages 0-3, and (ii) preschool between ages 3-6. We note that our analysis of the infant-toddler centers is more limited compared to our preschool analysis because attendance of infant-toddler centers was very low in the adult cohorts,\footnote{See Appendix Table \ref{tab:sample-asilo} for the distribution of individuals who attended each type of infant-toddler centers.} and virtually no other type of center provided child-care for children younger than three \textbf{[JJH: Then this is a great -- clean -- comparison. Group, we should feature more.]}. Given these data limitations, we focus the main paper on the evaluation of preschool intervention between ages 3 to 6, and present cohort-specific evaluations of  infant-toddler centers between ages 0 to 3 in Appendix \ref{sec:ITC}. \textbf{[JJH: Can we evaluate this within cohorts?]} \textbf{[Team: We provide these within-cohort evaluations in the appendix. We have changed the text in this paragraph to make this clearer. Would you prefer to have it in the main paper?] [JJH: Yes, if possible.]}
%our data is not appropriate to obtain estimates for infant-toddler centers due to small sample size of adults who attended infant-toddler centers\footnote{See Appendix Table \ref{tab:sample-asilo} for the distribution of individuals who attended each type of infant-toddler centers.}. \textbf{[YKK: Sylvi will add more reasons why we do not analyze asilo on its own in the main paper. YKK will estimate the treatment effects for 0-6 Reggio Approach for younger cohorts this weekend, because Reggio Approach emphasizes their approach being 0-6 as a whole.]} We present our strategy for evaluating ITC education and the corresponding results in Appendix~\ref{sec:ITC}.\footnote{The estimation results show that Reggio Approach ITCs do not have significant effects for children and adolescent cohorts, and significantly negative effects on cognitive skill and high school grade for adult cohorts (See Tables \ref{ols-M-child-reg-nopres-asilo} - \ref{ols-M-adult40-reg-nopres-asilo}).}


\subsection{Within-City Analysis} \label{sec:within-city-analysis}
\subsubsection{Baseline OLS} \label{subsubsection:OLS}
We perform several within-Reggio Emilia comparisons using a baseline OLS model. We compare individuals from Reggio Emilia who attended a Reggio Approach preschool to those in Reggio Emilia who attended (i) any other type of preschool (state, religious, municipal-affiliated, and other), (ii) no preschool at all, (iii) state preschool, and (iv) religious preschool. We only present estimates of the first and second comparisons in the main paper. Estimates from the other comparisons are reported in Appendix~\ref{app:comparison-reli-stat}. \textbf{[JJH: Why? Please present in text.]} %We do not compare Reggio Approach preschools with municipal-affiliated or other types of preschools due to small sample size (See Table \ref{tab:sample}).
For the children cohort (age 5), it is not possible to compare Reggio Approach preschools with no preschool because of small sample size for those who did not attend any preschool (See Table \ref{tab:sample}). We also exclude the cohort of adults in their 50s because the Reggio Approach was not available at that time. \textbf{[JJH: But this gives us a good baseline no-treatment approach.]}

The basic OLS takes the form,

\begin{equation}
	Y_i = \alpha_0 + \alpha_1 D_i + \bm{X}_i \bm{\gamma} + \varepsilon_i
	\label{eq:ra-v-none}
\end{equation}
where $i$ indexes individuals having attended any preschool in Reggio Emilia, $D_i$ is an indicator for whether individual $i$ attended municipal preschool, $\bm{X}_i$ is a vector of baseline control variables, and $\varepsilon_i$ is a random disturbance. Estimations for three specifications for $\bm{X}_i$ are reported: (i) no baseline control, (ii) baseline variables selected by the Bayesian Information Criterion (BIC)\footnote{Since the set of baseline variables are different for child, adolescent, and adult cohorts, we use separate model selections. For \emph{child cohort}, the \emph{a priori} designated control variables are male, CAPI, ITC attendance, and migrant indicators, and the BIC-selected variables are (i) mother graduated university, (2) family owns house, and (3) family income 10,000-25,000. For the \emph{adolescent cohort}, the fixed variables are male, CAPI, ITC attendance indicators and BIC-selected variables are (i) high school is father's maximum education, (ii) university is father's maximum education, and (iii) caregiver is catholic and faithful. For \emph{adult cohorts}, the fixed variables are male and CAPI indicators, and BIC-selected variables are (i) university is father's maximum education and (ii) number of siblings.} in addition to indicators for male, CAPI, and infant-toddler center attendance, and (iii) full set of all available baseline variables. In Equation~\ref{eq:ra-v-none}, $\alpha_1$ represents the mean differences in outcomes between the Reggio Approach and the other preschool types in Reggio Emilia, conditional on $X$. Under the assumption that, conditional on $X$, there is no systematic selection of individuals into the treatment $D_i$, this parameter estimates the causal treatment effect of the Reggio Approach on outcome $Y$.

\subsubsection{Propensity Score Matching}  \label{subsubsection:psm}

In order to complement the OLS analysis, we also estimate a propensity score matching model that implements nearest-neighbor matching on an estimated propensity score based on a BIC-selected set of observed baseline characteristics $\boldsymbol{X}_i$. Propensity score analysis is a version of non-parametric OLS and conditions on the same set of $\bm{X}$ variables as in OLS. This approach is based on the assumption that two individuals with a very similar propensity score have the same level of unobservable characteristics, so that selection into treatment is determined solely by the estimated propensity score. \textbf{[Team: Our discussion of effects on outcomes in the results and discussion sections only takes into account outcomes that are robust across different regressions.  We have added text in the results section to make this more explicit.] [JJH: What does ``robust'' mean? OLS and propensity score should have the same outcomes.]} %The benefits of propensity score matching over regression is that it is nonparametric. Furthermore, the use of the propensity score allows us to reduce the dimensionality of the matching exercise to a single dimension.

%Let $Y_{1,i}$ and $Y_{0,i}$ represent counterfactual outcomes for individual, $i$, where treatment status is fixed to $D_i=1$ and $D_i=0$ respectively. Under this framework, the realized outcome $Y_i$ is summarized by,
%\begin{equation}\label{eq:roy}
%Y_i = (1-D_i)Y_{0,i} + D_iY_{1,i}.
%\end{equation}
The average treatment effect (ATE) under the assumption for propensity score matching is written as:
\begin{equation} \label{eq:ATE-PSM}
E[Y(1)-Y(0)] = E[E[Y_i|D_i=1, \hat{\pi(\boldsymbol{X}_i)}] - E[Y_i|D_i=0, \hat{\pi(\boldsymbol{X}_i)}]].
\end{equation}
where the propensity score $\hat{\pi({\boldsymbol{X}_i})} = Pr(D_i=1|\boldsymbol{X}_i)$ is predicted for each individual $i$ using the estimated coefficients obtained from a probit model. We average over sample $\bm{X}$ to evaluate the average treatment effect.

The propensity score matching estimator is defined as
\begin{equation} \label{eq:PSM-estimator}
\widehat{E[Y(1)-Y(0)]_{PSM}} = \frac{1}{n} \sum_{i=1}^{n} (2D_i -1)(Y_i - \frac{1}{M}\sum_{j \in \mathcal{J}_M(i)}Y_j )
\end{equation}
where $M$ is a fixed number of matches per individual based on the propensity score and $\mathcal{J}_M(i)$ is a set of matches for individual $i$. The standard errors for this estimator are derived by \cite{Abadie_Imbens_2006_Econometrica}. \textbf{[JJH: Why did we not do kernel matching?]}

\subsubsection{Augmented IPW} \label{subsubsection:aipw}

\textbf{[JJH: Very ad hoc.]}

As another way of testing the OLS estimates, we estimate an Augmented Inverse Propensity Weighted (AIPW) estimator. As in propensity score matching, we assume that individuals select into treatment status $D_i$ based on a BIC-selected set of observed baseline characteristics $\boldsymbol{X}_i$. Under the assumption of strongly ignorable selection into treatment given $\boldsymbol{X}_i$, and the existence of an estimated propensity score $\hat{\pi({\boldsymbol{X}_i})}$ such that $0<\hat{\pi({\boldsymbol{X}_i})}<1$, the AIPW estimator for the average treatment effect $E[Y(1)-Y(0)]$ is defined as:

\begin{align}\label{eq:AIPW}
E[\widehat{Y(1)-Y(0)]}_{AIPW} &= \frac{1}{n} \sum_{i=1}^{n} \bigg \{ \bigg[ \overbrace{\frac{D_i Y_i}{\hat{\pi(\boldsymbol{X}_i)}} - \frac{(1-D_i)Y_i}{1-\hat{\pi(\boldsymbol{X}_i)}}}^{IPTW} \bigg] \\[10pt]
& - \frac{D_i - \hat{\pi}(\boldsymbol{X}_i)}{\hat{\pi}(\boldsymbol{X}_i) (1-\hat{\pi}(\boldsymbol{X}_i))} \nonumber \bigg[ (1-\hat{\pi}(\boldsymbol{X}_i)) E[\hat{Y_i}|D_i=1,\boldsymbol{X}_i] + \hat{\pi}(\boldsymbol{X}_i) E[\hat{Y_i}|D_i=0,\boldsymbol{X}_i] \bigg] \bigg \}
\end{align}

\textbf{[JJH: This term must multiply $[\;] E$ (term) $=0$?? So you add a mean zero term.]}

\noindent \textbf{[JJH: ? Seems wrong -- this weighting I need to see. I honestly do not know this method -- Can I see it? ]} \textbf{[Team: The weight $\left[-\frac{D_i - \hat{\pi}(\boldsymbol{X}_i)}{\hat{\pi}(\boldsymbol{X}_i) (1-\hat{\pi}(\boldsymbol{X}_i))}\right]$ is meant to stabilize the IPTW term by offsetting large values of the IPTW term near $\hat{\pi(\boldsymbol{X}_i)} = 0$ or $1$ with large values of the weight (which takes the opposite sign of the IPTW term): \textbf{[JJH: This looks very ad hoc? Can I see a rigorous justification for this? Does it affect results much?]} \\
\begin{align*}
	\text{Weight} = \left[-\frac{D_i - \hat{\pi}(\boldsymbol{X}_i)}{\hat{\pi}(\boldsymbol{X}_i) (1-\hat{\pi}(\boldsymbol{X}_i))}\right] =
	 \begin{cases}
		-\frac{1}{\hat{\pi}(\boldsymbol{X}_i)}, &\text{if  } D_i=1 \\
		\frac{1}{1-\hat{\pi}(\boldsymbol{X}_i)}, &\text{if  } D_i=0
	\end{cases} \\[10pt]
	\text{IPTW}= \left[\frac{D_i Y_i}{\hat{\pi(\boldsymbol{X}_i)}} - \frac{(1-D_i)Y_i}{1-\hat{\pi(\boldsymbol{X}_i)}}\right]=
	 \begin{cases}
		\frac{Y_i}{\hat{\pi}(\boldsymbol{X}_i)}, &\text{if  } D_i=1 \\
		-\frac{Y_i}{1-\hat{\pi}(\boldsymbol{X}_i)}, &\text{if  } D_i=0
	\end{cases}
\end{align*}
]}\\

Note that the first part of the AIPW estimator equals to the Inverse Propensity of Treatment Weighting (IPTW) estimator, which adjusts for selection on observable by assigning a higher weight to individuals with characteristics $\boldsymbol{X}_i$ if these individuals are less likely to have been in treatment group $D = 1$ and assigning a higher weight to individuals with characteristics $\boldsymbol{X}_i$ if these individuals are less likely to have been in treatment group $D = 0$. The AIPW estimator further addresses the problem of instability associated with the IPTW estimator near $\hat{\pi(\boldsymbol{X}_i)} = 0$ or $1$, by adjusting the weighted estimator with a weighted average of $\hat{Y_i}$ obtained from an OLS regression. For $D_i = 0$, as the IPTW estimate becomes large and negative near $\hat{\pi(\boldsymbol{X}_i)} = 1$, \textbf{[JJH: Do we ever have this problem?]} the adjustment term becomes large and positive, thereby, stabilizing the estimator at these values. Analogously, for $D = 1$, as the IPTW estimate becomes large and positive near $\hat{\pi(\boldsymbol{X}_i)} = 0$, the adjustment term becomes large and negative.\footnote{\citet{Tsiatis_2006_Semiparametric-Theory} and \citet{Glynn-Quinn_2010_Political-Analysis} \textbf{[JJH: I need these papers!] [Typist: Glynn and Quinn (2010) has been printed for you again and Tsiatis (2006) is a book which was checked out and given to you on Monday 2/6. We have also printed Ch.13 for you per the researchers' request.]} show that the AIPW estimator is an unbiased estimator of the Average Treatment Effect if the propensity score is known \textbf{[JJH: Which it is not!]}. Further, they show that the estimator is consistent if either the propensity score model or the outcome prediction model in the adjustment term is correctly specified. Due to this last property, they refer to the AIPW as a doubly robust estimator.}  Hence, the AIPW is a way to adjust the potential instability of IPTW. \textbf{[JJH: We estimate $p$! This is extremely \emph{ad hoc}! Can you give me a reference or proof?]}


\subsection{Across-City Comparisons} \label{sec:across-city-analysis}
\subsubsection{Difference-in-Difference}  \label{subsubsection:DID}

\textbf{[JJH: Age-50 cohort comparisons isolate cross-city effects since no program was in place anywhere.]}

We first estimate a difference-in-difference (DiD) model that allows for cross-city comparisons of municipal preschools while controlling for permanent differences in characteristics across cities. Similar to the OLS model, we estimate the parameters separately for each cohort. In the main paper, we present comparisons between municipal schools and (i) all other types of preschools pooled together or (ii) no preschool. We present comparisons to specific school types in Appendices~\ref{app:comparison-reli-stat} \textbf{[JJH: We should summarize in text.]}. %We exclude age-50 cohort as municipal childcare systems were not available for all three cities. %%PB This is the same as before, no need to repeat
For the age-40 cohort, we only compare people who attended Reggio Approach preschools with people in Parma or Padova who attended any type of preschools because municipal childcare systems were not available in both Parma and Padova. \textbf{[JJH: (?) Why is this different from 30 -- it's Reggio vs any preschool.]}

To illustrate, we present the comparison between between Reggio Emilia and Parma for those who either attended municipal preschool or no preschool at all. The estimation equation for this case as follows:
\begin{eqnarray}  \label{eq:specific2}
Y_i & = \beta_0 + \beta_1 Reggio_i + \beta_2 D_i + \beta_3 Reggio_i * D_i + \bm{X}_i \bm{\delta} + \epsilon_i
\end{eqnarray}
\noindent \textbf{[JJH: You might want to interact $X_i D_i$ as well.]} where $Reggio_i$ is the indicator for individual $i$ having attended preschool in Reggio Emilia and $D_i$ is the indicator for attending municipal preschool. $\beta_3$ is interpreted as the difference that remains between municipal schools in Reggio Emilia and Parma after adjusting for permanent differences between the two cities. In other words, $\beta_3$ is the DiD treatment effect estimator that amounts to (Reggio Emilia municipal - Parma municipal) - (Reggio Emilia none - Parma none), where the first difference captures the unadjusted (for permanent city differences) difference between municipal preschools in each city, and the second difference captures the permanent differences between the two cities. Analogous interpretations are applied to DiD comparisons between Reggio Emilia and Padova and comparisons between municipal schools and other school types. This approach is valid under the assumption that the same regressors account for selection into municipal preschools is comparable in the three cities, and the difference in the outcomes between municipal and non-municipal schools would have been the same in all three cities in the absence of the Reggio Approach.

\subsubsection{Propensity Score Matching}

The DiD model presented in Section \ref{subsubsection:DID} estimates the effect of municipal preschools relative to other types of preschool or no preschool across cities. However, selection into municipal preschools in Parma and Padova may not be analogous to selection into Reggio Approach preschools. In order to complement the DiD analysis, we estimate a propensity score matching model to match people who attended the Reggio Approach preschools with people in Parma or Padova who attended (i) all types of preschools pooled together, including municipal preschools, or (ii) no preschool.

To illustrate, the propensity score matching estimator for the comparison between Reggio Approach and all types of pooled together in Parma is formally written as Equation \ref{eq:PSM-estimator}, except that now $i$ is limited to (i) people in Reggio who attended Reggio Approach preschools and (ii) people in Parma who attended any preschool. $D_i$ is an indicator for attending a Reggio Approach preschool. The purpose is to match Reggio Approach people with people who have similar propensity scores but have attended preschool in Parma. We assume that the latter group is similar to the Reggio Approach people except that they are not exposed to the Reggio Approach. By comparing the outcomes across the matches, the propensity score matching model estimates the effect of the Reggio Approach. Analogous interpretations are applied to comparisons for different control group specifications. \textbf{[JJH: I thought we also had IV and selection bias results -- what happened?]} \textbf{[Team: We do not report an IV approach in this paper for the following reasons:
\begin{enumerate*}[label=(\roman*)]
\item The IV approach in the second paper --which focused on Children and Adolescents-- used distance to school, mother's place of birth, whether mother attended childcare, and Reggio Score (approximation of scores that would have been assigned by Reggio Approach based on background characteristics) as instruments. However, the first-stage was very weak.
\item The distance variable is not an appropriate instrument for adult cohorts because distance is calculated from residence at time of interview which could be different than residence when individual was attending childcare.
\item We explored cost as an instrument, but weren't able to find the schedule of costs for adult cohorts.
\end{enumerate*}
] [JJH: But report the IV estimates in an appendix and test if they are different from propensity score -- use the likely valid instruments.]} 