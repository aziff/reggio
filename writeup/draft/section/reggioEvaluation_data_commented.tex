\subsection{The Survey Data Collection}

Respondents were sampled from the population registries of the cities based on their year of birth. The sample was then restricted to those individuals living in the same city in which they were raised. All cohorts, except the youngest one, are restricted to individuals who are Italian citizens. In contrast, the youngest cohort includes an oversampling of immigrant children.\footnote{In the adult cohorts there was no immigrant who was preschool age in the same school in which they live. In the adolescent cohort, the number was immigrant born was extremely small.} The sample from Reggio Emilia, across all cohorts, includes an oversampling of those who attended municipal schools, as this is our treatment group.

Of the reference sample, 7,176 individuals were randomly selected. Of these, 4,019 completed interviews, resulting in a response rate of 56\%.\footnote{We have very limited information on those who refused. Thus, we are unable to adjust for this high non-responsive rate.} Table~\ref{tab:sample-response} provides an overview of the birth years for the different cohorts, the counts of the full sample, and the response rate. The most common reasons for non-response were that nobody was home when the surveying agency solicited and sharp refusals.

\begin{table}[H]
\centering
\begin{threeparttable}
	\caption{Description of the Full Sample and Response Rates}\label{tab:sample-response}
	\begin{tabular}{l c c c c c c}
\toprule
Cohort & Birth year(s) & Age at interview & Reggio Emilia & Parma & Padova & \textbf{Total} \\			
\midrule
\textbf{Children} &  &  &  & &  &  \\ 
\quad Italians & 2006 & 7 & 311 & 291& 278 & 880 \\
			&&	& \textit{50.0\%} &  \textit{62.7\%} &  \textit{50.0\%} &  \textit{53.6\%} \\
\quad Migrants & 2006 & 7 & 110 & 58 & 113 & 281 \\
			&& 	&  \textit{53.1\%} &  \textit{49.2\%} &  \textit{63.1\%} &  \textit{55.8\%} \\
\textbf{Adolescents} & 1994 & 19 & 300 & 254 & 282 & 836 \\
			&& 	&  \textit{57.1\%} &  \textit{58.5\%} &  \textit{55.5\%} &  \textit{57.0\%} \\
\textbf{Adults 30s} & 1980-1981 & 32 & 280 & 251 & 251 & 782 \\
			&& 	&  \textit{58.3\%} &  \textit{58.2\%} &  \textit{57.4\%} &  \textit{57.9\%} \\
\textbf{Adults 40s} & 1969-1970 & 43 & 285 & 254 & 252 & 791 \\
			&& 	&  \textit{59.3\%} &  \textit{56.3\%} &  \textit{53.8\%} &  \textit{56.0\%}\\
\textbf{Adults 50s} & 1954-1959 & 54-60 & 200 & 103 & 146 & 449 \\
			&& 	&  \textit{52.2\%} &  \textit{63.6\%} &  \textit{55.6\%}  &  \textit{55.6\%}\\
\midrule
\textbf{Total}	& 				& & 1,486 & 1,211 & 1,322 & 4,019 \\
			&&				& \textit{55.1\%} &  \textit{58.8\%} &  \textit{55.0\%} & \textit{56.0\%} \\
\bottomrule
\end{tabular}

\begin{tablenotes}
\footnotesize
Note: The response rates for each city and cohort are in italics. They are the the ratio of interviews to total valid contacts. Valid contacts are the sum of: completed interviews, sharp refusal, no person present, talked with a relative, left paper questionnaire but never returned, interview began but not completed. The age at interview is an approximation given there is some variation in the interview date and birth year within each cohort. In analysis, we combine the Italian and migrant subsamples of the child cohort and control for migrant status.
\end{tablenotes}
\end{threeparttable}
\end{table}

Tables~\ref{tab:sample-asilo} and \ref{tab:sample} provide a detailed tabulation of the sample by city, cohort, and school type for both infant-toddler care and preschool attendance. They show that the number of people who do not attend any preschool and infant-toddler center decreases over time. Whereas the majority of individuals from the age-50 cohort did not attend any infant-toddler care or preschool, there are few such cases in the child and adolescent cohorts. These tables also show that the proportion of individuals attending municipal infant-toddler centers and preschools is higher in Reggio Emilia than in the other cities.\footnote{This is due to the construction of the sample.} Note that the Reggio Approach preschools were not available for the age-50 cohort. \textbf{[JJH: We need a definition of muni-affiliated -- not previously discussed or defined.][Team: We added the definition of muni-affiliated in the background section.]}

\begin{table}[H]
\centering
\scalebox{0.85}{
\begin{threeparttable}
	\caption{Tabulation of Infant-Toddler Care Attendance by Cohort, City, and School Type}\label{tab:sample-asilo}
	\begin{tabular}{l*{7}{c}}
\toprule
		&	\mc{6}{c}{Reggio Emilia: 1,486}													\\	\midrule
		&	None	&	Muni	&	Reli	&	Priv	&	Muni-Affi	&	Other	\\	\midrule		
\textbf{Children	}&		&		&		&		&		&		\\	
\quad Italians	&	115	&	109	&	28	&	6	&	51	&	0	\\			
\quad Migrants		&	58	&	24	&	2	&	0	&	20	&	3	\\			
\textbf{Adolescents}		&	129	&	112	&	10	&	3	&	36	&	3	\\			
\textbf{Adults 30s}		&	210	&	53	&	2	&	3	&	1	&	7	\\			
\textbf{Adults 40s}		&	241	&	31	&	0	&	0	&	0	&	5	\\			
\textbf{Adults 50s}		&	194	&	0	&	1	&	0	&	0	&	1	\\	\midrule		
		&	\mc{6}{c}{ Parma: 1,211}											\\	\midrule		
		&	None	&	Muni	&	Reli	&	Priv	&	Muni-Affi	&	Other	\\	\midrule	
\textbf{Children}&		&		&		&		&		&		\\		
\quad Italians		&	98	&	99	&	7	&	15	&	48	&	21	\\			
\quad Migrants		&	24	&	23	&	1	&	0	&	9	&	1	\\			
\textbf{Adolescents}		&	126	&	74	&	10	&	11	&	25	&	2	\\			
\textbf{Adults 30s}		&	187	&	31	&	8	&	6	&	11	&	4	\\			
\textbf{Adults 40s}		&	222	&	0	&	2	&	0	&	10	&	16	\\			
\textbf{Adults 50s}		&	85	&	0	&	4	&	0	&	0	&	13	\\	\midrule		
		&	\mc{6}{c}{Padova: 1,322}											\\	\midrule		
		&	None	&	Muni	&	Reli	&	Priv	&	Muni-Affi	&	Other	\\	\midrule		
\textbf{Children}&		&		&		&		&		&		\\		
\quad Italians		&	143	&	48	&	26	&	40	&	19	&	1	\\			
\quad Migrants		&	57	&	44	&	3	&	5	&	0	&	1	\\			
\textbf{Adolescents}		&	209	&	52	&	8	&	0	&	6	&	1	\\			
\textbf{Adults 30s}		&	220	&	19	&	5	&	3	&	0	&	0	\\			
\textbf{Adults 40s}		&	225	&	0	&	7	&	0	&	1	&	17	\\			
\textbf{Adults 50s}		&	133	&	0	&	6	&	0	&	0	&	0	\\			


\bottomrule
\end{tabular}


\begin{tablenotes}
Note: This table shows the sample size by city, cohort, and school type. We separate migrants and children for clarity in this table even though they are in the same birth cohort (year of birth: 2006). None: did not enroll in formal childcare; Muni.: municipal preschool; Relig.: religious preschool; Priv.: private preschool. Muni-Affi: municipal-affiliated preschool; Other: uncategorized preschool.
\end{tablenotes}
\end{threeparttable}
}
\end{table}

\begin{table}[H]
\centering
\scalebox{0.85}{
\begin{threeparttable}
	\caption{Tabulation of Preschool Attendence by Cohort, City, and School Type}\label{tab:sample}
	\begin{tabular}{l*{8}{c}}
\toprule
	&	\mc{7}{c}{Reggio Emilia: 1,486}													\\	\midrule
	&	None	&	Muni	&	State	&	Reli	&	Priv	&	Muni-Affi	&	Other	\\	\midrule
Children	&	2	&	159	&	44	&	92	&	5	&	7	&	1	\\	
Migrants	&	4	&	47	&	38	&	14	&	1	&	3	&	1	\\	
Adolescents	&	7	&	151	&	22	&	98	&	6	&	13	&	0	\\	
Adults 30s	&	57	&	138	&	31	&	40	&	1	&	4	&	8	\\	
Adults 40s	&	80	&	87	&	14	&	52	&	5	&	1	&	43	\\	
Adults 50s	&	147	&	0	&	0	&	29	&	2	&	0	&	20	\\	\midrule
	&	\mc{7}{c}{ Parma: 1,211}													\\	\midrule
	&	None	&	Muni	&	State	&	Reli	&	Priv	&	Muni-Affi	&	Other	\\	\midrule
Children	&	5	&	105	&	42	&	74	&	8	&	52	&	0	\\	
Migrants	&	4	&	25	&	12	&	3	&	6	&	7	&	0	\\	
Adolescents	&	4	&	100	&	52	&	77	&	6	&	5	&	2	\\	
Adults 30s	&	44	&	85	&	56	&	51	&	5	&	4	&	3	\\	
Adults 40s	&	116	&	0	&	0	&	55	&	1	&	4	&	73	\\	
Adults 50s	&	72	&	0	&	0	&	11	&	0	&	10	&	9	\\	\midrule
	&	\mc{7}{c}{Padova: 1,322}													\\	\midrule
	&	None	&	Muni	&	State	&	Reli	&	Priv	&	Muni-Affi	&	Other	\\	\midrule
Children	&	2	&	58	&	45	&	141	&	12	&	19	&	0	\\	
Migrants	&	5	&	33	&	46	&	23	&	1	&	0	&	4	\\	
Adolescents	&	1	&	84	&	46	&	132	&	6	&	7	&	2	\\	
Adults 30s	&	47	&	27	&	27	&	140	&	1	&	7	&	0	\\	
Adults 40s	&	75	&	0	&	0	&	126	&	0	&	10	&	39	\\	
Adults 50s	&	57	&	0	&	0	&	72	&	2	&	6	&	3	\\	

\bottomrule
\end{tabular}


\begin{tablenotes}
Note: This table shows the sample size by city, cohort, and school type. We separate migrants and children for clarity in this table even though they are in the same birth cohort (year of birth: 2006). None: no preschool; Muni.: municipal preschool;  State: state preschool; Relig.: religious preschool; Priv.: private preschool. Muni-Affi: municipal-affiliated preschool; Other: uncategorized preschool.
\end{tablenotes}
\end{threeparttable}
}
\end{table}

The structure of the cohorts allows us to study the effects of the Reggio Approach at different stages throughout the life cycle. The children in the youngest cohort were interviewed when they entered primary school, the adolescent cohort was interviewed when they complete compulsory schooling, and the adult cohorts capture different points of adulthood to measure key outcomes such as engagement in the labor market, health, and family decisions. Although this cohort structure allows us to study the evolution of the program, the other preschools also evolved making it challenging to compare the outcomes from the Reggio Approach with those from a stable control group. Our investigation in Section~\ref{sec:ece-italy} of the early childhood education landscape helps characterize the comparison group over time.

Restricting the sample to individuals living in the same city in which they were raised is necessary in order to compare individuals who had the \textit{opportunity} to attend the different types of preschool. Table~\ref{tab:immigration}, based on population registry data, presents the proportion of the population who were born in Italy, of Italian citizenship, and still resident in that town of birth. For all cohorts, the immigration rates are very similar for all three cities. Both treatment and control cities share a similar economic and labor market history. Nonetheless, it is worth noting that embedded in our sample selection is the potential bias due to the fact that one of the effects of preschool might be a higher propensity to emigrate.\footnote{\citet{Gertler_Heckman_etal_2014_Science} show that one important benefit of the Jamaica early childhood intervention was on emigration to more prosperous countries.} In general, higher skilled individuals are more mobile. This does not necessarily bias treatment effects because migration patterns are uniform across regions.

\begin{table}[H]
\centering
\begin{threeparttable}
	\caption{Percentage of People Living in the Same City Since Birth}\label{tab:immigration}
	\begin{tabular}{ l c c c c }
\toprule
\textbf{Cohort} & \textbf{Reggio (\%)} & \textbf{Parma (\%)} & \textbf{Padova (\%)} & \textbf{Total (\%)}\\
\midrule
Italian Children born in 2006 (Cohort V)   & 61.3  & 70.2  & 65.1  & 65.2 \\[0.2em]
Adolescents born in 1994 (Cohort IV)       & 58.1  & 63.0  & 64.4  & 61.9 \\[0.2em]
Adults born in 1980-81 (Cohort III)        & 26.5  & 27.5  & 32.6  & 29.0 \\[0.2em]
Adults born in 1969-70 (Cohort II)         & 27.9  & 31.6  & 31.9  & 30.6 \\[0.2em]
Adults born in 1954-59 (Cohort I)          & 28.8  & 27.9  & 31.4  & 29.5 \\[0.2em]
\midrule
\textit{Total}         & \textit{32.3\%}  & \textit{32.5\%}  & \textit{35.2\%} & \textit{33.5\%} \\
\bottomrule
\end{tabular}

\begin{tablenotes}
\footnotesize
Note: This table presents the percentage of people living in same city since birth. This  shows the reference sample who satified the selection criteria (born in the city of residence and of Italian citizenship) as a percentage of the total number of names given by the population registries.
\end{tablenotes}
\end{threeparttable}
\end{table}

%\subsection{The Questionnaire Design}

In order to evaluate the effect of the Reggio Approach on a broad set of domains, we designed a questionnaire surveying various outcomes and dimensions of life success. Respondents were asked about family composition, fertility, labor force participation, income, schooling, cognitive ability, social and emotional skills, health and healthy habits, social capital, interpersonal ties, as well as attitudes on migrants. Three age-specific questionnaires were designed, piloted, and fielded: one for the Italian and immigrant child cohorts, one for the adolescent cohort, and one for the adult cohorts. The parents of the children and adolescents were also administered a questionnaire.\footnote{The questionnaire was piloted in the city of Bergamo with a sample from every cohort. A second pilot was conducted in Reggio Emilia, Parma, and Padova on a subsample of adults. The questionnaires were subsequently tested and refined to the final version, which lasts approximately 40 minutes for the adults, and 1 hour for the children and the adolescents.} 