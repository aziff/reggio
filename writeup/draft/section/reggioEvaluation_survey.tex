

\subsection{Overview of Survey}
Below is a list of the administrative and pedagogical components that we inquire about. Components with a * next to them are present in the Reggio Approach. Components with a $^o$ are omitted. We omit these components because we received feedback from survey respondents that those questions were interpreted differently than originally drafted in the English version. These components were assembled based on published information of the Reggio Approach, and confirmed by expert scholars with firsthand knowledge of the Reggio Approach and early childhood programs in northern Italy.\footnote{See \citet{Edwards-etal-eds_1998_Hundred-Languages} and \citet{Corsaro_2008_Policy-Practice}.}

\begin{itemize}
 \item Administrative components
 \begin{itemize}
 	\item All teachers graduated from a teacher training institution, in accordance with national guidelines.$^o$
 	\item Full-time educative coordinators, with a university degree in psychology or education, were hired by the school system.*
 	\item Educative coordinators met biweekly with educative staff to provide mentoring and professional development.*
 	\item Kitchen staff participated in professional development and routine trainings with teachers.*
 	\item Janitorial staff participated in professional development and routine trainings with teachers.*
 	\item Teachers participated in professional development with teachers from other school systems (e.g. municipal and private Catholic).*
 	\item Schools were open daily for 8 hours.*
 	\item Schools offered extended hours for working families.*
 	\item Scheduled work hours are set aside weekly for teachers to engage families.*
 	\item Scheduled work hours are set aside weekly for teachers to document children's work.*
 	\item Scheduled work hours are set aside weekly for teachers to participate in professional development.*
 	\item Priority of enrollment is given to economically disadvantaged families.*
 	\item Priority of enrollment is given to single-parent families.*
 	\item Priority of enrollment is given to children with disabilities.*
 	\item Schools received funding from public sources.*
	\item Schools received equitable funding from public sources.$^o$
	\item Schools acquired ``paritaria'' status from the state.*$^o$
 \end{itemize}
 \item Pedagogical components
 \begin{itemize}
 	\item Daily activities were implemented by following a predefined program to guide children in acquiring knowledge of specific concepts.
 	\item Classrooms were homogenous in age.*
 	\item Two co-teachers were assigned to the same group of children. Continuity of care provided by keeping at least one teacher with the same group from year to year.*
 	\item A full-time, on-site teacher with specific training or experience in the fine arts helped educators design creative learning activities.*
 	\item Fine arts were used as a tool to help children learn.*
 	\item Children participate in religious teaching.
 	\item Teachers document children's learning in portfolios.*
 	\item The design of the school environment emphasizes open spaces, natural lighting, and the use of natural materials for furniture.*
 	\item The school environment included a dedicated room where children from different classrooms work individually or in small groups.*
 	\item An on-site kitchen was used daily to prepare meals.*$^o$
 	\item Project-based learning with unlimited timelines shapes the educational program.*
 	\item Academic theories of psychology and early childhood education influenced educational approaches.*
 	\item Early childhood practices endorsed by Agazzi, Froebl, and/or Montessori influenced the daily program.
 	\item Early childhood practices promoted by Loris Malaguzzi influenced the daily program.*
 	\item The educational program is designed to promote good morals of family life, and is based on love of family and the homeland.
 	\item Parental boards or advisory groups were encouraged and active participants in school culture.*
	\item Transitions between schools were supported by teacher visits to homes or scheduled visits for children to new schools.$^o$
 \end{itemize}
 \end{itemize}
 
The following information comes from a variety of sources: \citet{Reggio-Admin-data_1966-2006, Reggio-Annual-Journals_1994-2011, Padova-Admin-Data_1964-2011} and results from the historical survey \citep{CEHD_2016_Historical-Analysis}.
 
\subsection{Full Survey}
\includepdf[pages=-]{section/CEHD-ECE-Italy_SurveyQuestionnaire-ENGLISH_2016-10-24_sk}

