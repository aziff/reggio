
Because there are two stages of early childhood interventions, (i) ages 0-3 and (ii) ages 3-6, it is important to consider both when estimating treatment effects of either intervention on later outcomes. Table~\ref{tab:cases-treat} shows the possible cases of receiving early childhood intervention in our data, where 0 indicates not attending and 1 indicates attending. At this stage, we limit the type of infant-toddler centers and preschools to municipal only. 

\begin{table}[H]
\caption{Possible Cases of Treatment} \label{tab:cases-treat}
\begin{tabular}{C{1.8cm} R{0.7cm} C{2cm} C{2cm}}
  
		& & \multicolumn{2}{c}{Preschool (Ages 3-6)} \\
		& & 0 & 1 \\ \cline{3-4}            
        								 &  & \multicolumn{1}{|c|}{} & \multicolumn{1}{c|}{} \\
        							& 0 & \multicolumn{1}{|c|}{(0,0)} & \multicolumn{1}{c|}{(0,1)} \\ 
        				ITC				&  & \multicolumn{1}{|c|}{} & \multicolumn{1}{c|}{} \\ \cline{3-4}
                        (Age 0-3)  		&  & \multicolumn{1}{|c|}{} & \multicolumn{1}{c|}{} \\
        								& 1 & \multicolumn{1}{|c|}{(1,0)} & \multicolumn{1}{c|}{(1,1)} \\ 
        								&  & \multicolumn{1}{|c|}{} & \multicolumn{1}{c|}{} \\ \cline{3-4}
\end{tabular}
\begin{flushleft}
\footnotesize{Note:} We only consider municipal ITCs (infant-toddler-centers, ages 0-3) and preschools (ages 3-6). (0,0): did not attend any municipal school for both ages 0-3 and 3-6; (1,0): attended a municipal school for ages 0-3 but did \textit{not} attend for ages 3-6; (0,1): did \textit{not} attend a municipal school for ages 0-3 but did attend for ages 3-6; (1,1): attended a municipal school for both ages 0-3 and 3-6.
\end{flushleft}
\end{table}

\subsection{Estimating Effects of Infant-Toddler Centers}
There are two main methods to test the effect of attending infant-toddler centers. The first is to compare people who did not attend infant-toddler care or preschool with people who only attended municipal infant-toddler care. Using the notation in Table~\ref{tab:cases-treat}, this comparison is between (0,0) and (1,0). The second method is to compare people who only attended municipal preschool with people who attended both municipal infant-toddler centers and preschools. That is, to compare (0,1) and (1,1). The hypotheses are formally written as
\begin{eqnarray}
H_1: &  Y_{0,0} = Y_{1,0} \\ 
H_2: &  Y_{0,1} = Y_{1,1} 
\end{eqnarray}
\noindent where $Y_{i,j}$ is the outcome of the individuals who attended $i \in \{0,1\}$ infant-toddler care and $j \in \{0,1\}$ preschool.

A possible estimation strategy is to limit the sample to a specific city and a specific cohort constrained to the comparison groups needed according to the hypotheses above. To test $H_1$, we estimate the following regression equation:
\begin{eqnarray}
Y_{i}^{c,h} & = & \alpha + \beta_{0}R_i^{ITC} + \mathbf{X}_i\gamma + \varepsilon_{i}^{c,h}, \\ \nonumber
& \forall & i \in \text{ \{People in city $c$ and cohort $h$ and in group (0,0) or (1,0)\}}
\end{eqnarray}
where $R_i^{ITC}$ is the indicator for attending municipal infant-toddler center and $\mathbf{X_i}$ is the vector of baseline variables for individual $i$. Likewise, to test $H_2$:
\begin{eqnarray}
Y_{i}^{c,h} & = & \alpha + \beta_{0}R_i^{ITC} + \mathbf{X}_i\gamma + \varepsilon_{i}^{c,h}, \\ \nonumber
& \forall & i \in \text{ \{People in city $c$ and cohort $h$ and in group (0,1) or (1,1)\}.}
\end{eqnarray}

One caveat of this analysis is that it uses a limited sample size. In our data, these hypotheses cannot be tested under this strategy for many groups. Table~\ref{tab:num-group} shows the number of individuals available for each group necessary for analysis using this strategy. It is impossible to test $H_1$ in our data, because there are almost no people who attended municipal infant-toddler care without attending preschool (the group (1,0)). While it is possible to test $H_2$ for several groups, the number of observations for the group (1,1) is small for the adult cohorts. 

In Table \ref{tab:num-group}, the groups subject to our estimation are highlighted. Based on the available number of individuals in each cell and the history of the foundation date of municipal infant-toddler care for each city, we decide to test $H_2$ for the highlighted groups.

\begin{table}[H] \caption{Number of Individuals in Each Group} \label{tab:num-group}
\scalebox{0.77}{
\begin{tabular}{l|ccccc|ccccc|ccccc}
\toprule
			& 		\multicolumn{5}{c}{\textbf{Reggio}}		& 	\multicolumn{5}{|c|}{\textbf{Parma}}	& 			\multicolumn{5}{c}{\textbf{Padova}}				\\
			& (0,0) & (1,0) & (0,1) & (1,1) & Total & (0,0) & (1,0) & (0,1) & (1,1) & Total  & (0,0) & (1,0) & (0,1) & (1,1) & Total \\ \midrule
Child		& 2 & 0 & \cellcolor{blue!25}46 & \cellcolor{blue!25}117 & \textbf{311} & 5 & 1 & \cellcolor{blue!25}35 & \cellcolor{blue!25}100 & \textbf{291} & 2 & 0 & \cellcolor{blue!25}31 & \cellcolor{blue!25}36 & \textbf{278} \\
Migrant		& 4 & 0	& 24 & 26 & \textbf{110} & 4 & 0 & 12 & 23 & \textbf{58} & 5 & 0 & 18 & 16 & \textbf{113} \\
Adolescent 	& 7 & 0 & \cellcolor{blue!25}45 &	\cellcolor{blue!25}116 & \textbf{300} & 4 & 0 & \cellcolor{blue!25}49 & \cellcolor{blue!25}61 & \textbf{254} & 1 & 0 & \cellcolor{blue!25}55 & \cellcolor{blue!25}37 & \textbf{282} \\
Age-30		& 57 & 0 & \cellcolor{blue!25}95 &	\cellcolor{blue!25}53 & \textbf{280} & 43 & 0 & \cellcolor{blue!25}64 & \cellcolor{blue!25}29 & \textbf{251} & 47 & 0 & 25 & 9 & \textbf{251} \\
Age-40		& 80 & 0 & \cellcolor{blue!25}97 &	\cellcolor{blue!25}28 & \textbf{285} & 115 & 1 & 35 & 16 & \textbf{254} & 75 & 0 & 25 & 2 & \textbf{252} \\
Age-50		& 146 & 0 &	8 & 0 & \textbf{200} & 71 & 0 & 4 & 8 & \textbf{103} & 55 & 0 & 11 & 0 & \textbf{146} \\ \bottomrule
\end{tabular}}
\end{table}


\subsection{Estimating Effects of Preschools}
\subsubsection{Baseline OLS}
We perform several comparisons using a baseline OLS model. We compare individuals from Reggio Emilia who attended a Reggio Approach preschool to those in Reggio Emilia who attended (i) any type of preschool, (ii) no preschool, (iii) state preschool, (iv) religious preschool, and (v) no preschool. In the main paper, we only present estimates of the first comparison. Estimates from the other comparisons are reported in Appendices~\ref{appendix:no-preschool} through~\ref{appendix:religious}. We exclude the cohort of adults in their 50s because the Reggio Approach was not available at that time. We also exclude individuals in Parma and Padova from this analysis.

Let $i$ index individuals in Reggio Emilia and let $R_i^{P}$ indicate whether that individual attended a municipal preschool. Given we restrict this model to those in Reggio Emilia, $R_i^{P} = 1$ implies the individual attended the Reggio Approach. We select a vector, $\bm{X}_i$, of baseline control variables with the lowest BIC to account for family background.\footnote{These variables are: gender, whether the individual took the computer-assisted (CAPI), number of siblings, an indicator if the mother's maximum education was middle school, an indicator if the father's maximum education is university, and two indicators for the number of siblings (one indicating having two or more siblings, the other indicating having three or more siblings).} For both cohorts, we estimate $\beta$ in the simple model

\begin{equation}
	Y_i = \alpha_0 + \alpha_1 R_i^{P} + \bm{X}_i\gamma + \varepsilon_i
	\label{eq:ra-v-none}
\end{equation}

\noindent where we assume $\varepsilon_i$ to be a random disturbance. 

\subsubsection{Difference-in-Difference}
Although the baseline OLS allows for a basic understanding of the relationship between the Reggio Approach and relevant outcomes, it does not include any city-level effect. This difference-in-difference model accounts for city-level effects by comparing the difference in outcomes across cities for different preschool types after fixing the cohort. 

We restrict the analysis to individuals in the age-30 and age-40 cohorts. We present comparisons to specific school types in Appendices~\ref{appendix:no-preschool} through~\ref{appendix:religious}. In the main paper, we present comparisons between municipal schools and all other types of preschools pooled together.

As an example, we only consider age-30 individuals from Reggio Emilia and Parma who either attended municipal preschool or did not attend any preschool. We can write the estimation equation for this group as:
\begin{eqnarray}  \label{eq:specific2}
Y_i & = \beta_0 + \beta_1 Reggio_i + \beta_2 R_i^P + \beta_3 Reggio_i * R_i^P + \bm{X}_i\delta + \epsilon_i 
\end{eqnarray}
where $Reggio_i$ is the indicator for $i$ living in Reggio Emilia. In this regression, $\beta_3$ has an interpretation of (Reggio Municipal - Parma Municipal) - (Reggio None - Parma None). The first difference shows the difference between outcomes of those who attended municipal preschools in Reggio Emilia and those who attended municipal preschools in Parma. However, there can be inherent differences between individuals in Reggio Emilia and those in Parma for the age-30 cohort. The second difference is intended to capture the inherent difference in outcomes between two cities and to eliminate the city effect. An analogous interpretation is applied to difference-in-difference comparisons between Reggio Emilia and Padova and considering individuals in the age-40 cohort. 


\subsubsection{Augemented IPW}

The baseline OLS and difference-in-difference models do not account for selection into the Reggio Approach. We estimate an Augmented Inverse Propensity Weighted (AIPW) estimator in order to account for this.   

To begin, consider the following two treatment statuses,

\begin{equation}\label{eq:cases-d}
D_i = \quad
\begin{cases}
1 \quad if \quad &  R_i = 1\text{\quad 		(Attended Reggio Municipal)} \\
0 \quad if \quad &  R_i \neq 0 \text{\quad 	(From Reggio but did not attend Municipal)} \\
\end{cases}
\end{equation}
~\\
Let $Y_{1,i}$ and $Y_{0,i}$ represent counterfactual outcomes for individual, $i$, where treatment status is fixed to $D_i=1$ and $D_i=0$ respectively. Under this framework, the realized outcome $Y_i$ is summarized by,
\begin{equation}\label{eq:roy}
Y_i = (1-D_i)Y_{0,i} + D_iY_{1,i}.
\end{equation}

Next, assume that individuals select into treatment status $D_i$ based on a a set of observed baseline characteristics $\boldsymbol{X}$. Under the assumption of strongly ignorable selection into treatment given $\boldsymbol{X}$, and the existence of an estimated propensity score $\hat{\pi({\boldsymbol{X_i}})} = Pr(D=1|\boldsymbol{X_i})$ such that $0<\hat{\pi({\boldsymbol{X_i}})}<1$, we can use standard Inverse Propensity of Treatment Weighting (IPTW) to estimate the average treatment effect $E[Y(1)-Y(0)]$,

\begin{equation}\label{eq:IPW}
\widehat{E[Y(1)-Y(0)]_{IPW}} = \frac{1}{n} \sum_{i=1}^{n} \left \{\frac{D_i Y_i}{\hat{\pi(\boldsymbol{X_i})}} - \frac{(1-D_i)Y_i}{1-\hat{\pi(\boldsymbol{X_i})}}\right \}
\end{equation}

\noindent Where $\hat{\pi({\boldsymbol{X_i}})} = Pr(D=1|\boldsymbol{X_i})$ is predicted for each individual $i$ using the estimated coefficients obtained from a probit model.

In calculating $\widehat{E[Y(1)]}$, the IPTW estimator assigns a higher weight to individuals with characteristics $\boldsymbol{X_i}$ if these individuals are less likely to have been in treatment group $D = 1$. Analogously, in calculating $\widehat{E[Y(0)]}$,  the IPTW estimator assigns a higher weight to individuals with characteristics $\boldsymbol{X_i}$ if these individuals are less likely to have been in treatment group $D = 0$. In this sense, the estimator adjusts for selection on observables by \textit{balancing} the estimation of expected outcomes on $\boldsymbol{X}$.

An issue with the simple IPTW estimator is that the estimates become unstable as the estimated propensities, $\hat{\pi(\boldsymbol{X_i})}$, tend to 0 or 1. We use the following Augmented IPW (AIPW) estimator to correct for this issue. 
\begin{align}\label{eq:AIPW}
E[\widehat{Y(1)-Y(0)]}_{AIPW} = \frac{1}{n} \sum_{i=1}^{n} \bigg \{ \bigg[ & \overbrace{\frac{D_i Y_i}{\hat{\pi(\boldsymbol{X_i})}} - \frac{(1-D_i)Y_i}{1-\hat{\pi(\boldsymbol{X_i})}}}^{IPTW} \bigg]- \frac{D_i - \hat{\pi}(\boldsymbol{X_i})}{\hat{\pi}(\boldsymbol{X_i}) (1-\hat{\pi}(\boldsymbol{X_i}))} \nonumber \\[10pt]
& \bigg[ (1-\hat{\pi}(\boldsymbol{X_i})) E[\hat{Y_i}|D_i=1,\boldsymbol{X_i}] + \hat{\pi}(\boldsymbol{X_i})) E[\hat{Y_i}|D_i=0,\boldsymbol{X_i}] \bigg] \bigg \}
\end{align}

The AIPW estimator in Equation (\ref{eq:AIPW}) addresses the problem of instability associated with the IPTW estimator near $\hat{\pi(\boldsymbol{X_i})} = 0$ or $1$, by adjusting the weighted estimator with a weighted average of $\hat{Y_i}$ obtained from an OLS regression. For $D_i = 0$, as the IPTW estimate becomes large and negative near $\hat{\pi(\boldsymbol{X_i})} = 1$, the adjustment term becomes large and positive, thereby, \textit{stabilizing} the estimator at these values. Analogously, for $D = 1$, as the IPTW estimate becomes large and positive near $\hat{\pi(\boldsymbol{X_i})} = 0$, the adjustment term becomes large and negative.

\subsubsection{Next Steps: AIPW using multinomial control selection model}
An issue with the AIPW estimator in Equation (\ref{eq:AIPW}) is that the control group, $D_i=0$, includes all individuals who did not attend Reggio Municipal schools. As a result, we are unable to account for potential heterogeneity in treatment effects with respect to control group individuals who attended different types of preschools. In the rest of this subsection, we present a preliminary framework for capturing this heterogeneity of effects. \textbf{The work is preliminary and we haven't tested the estimators for properties including unbiasedness.}

To account for this heterogeneity, we can start by defining a new variable for treatment status,

\begin{equation}
T_i = \quad
\begin{cases}
0 \quad \text{if \quad (\textit{i} attended Reggio Municipal)} \\
1 \quad  \text{if \quad (\textit{i} From Reggio and attended no pre-school)} \\
2 \quad \text{if \quad (\textit{i} attended Reggio State)}  \\
3 \quad \text{if \quad (\textit{i} attended Reggio Religious)}  \\
\end{cases}
\end{equation}
~\\ ~\\
Let $Y_{j,i}$ denote the counterfactual outcome for individual $i$ when treatment status $T_i$ is fixed to $j$ $\forall \,  j \, \in \{0,1,2,3\}$. Under this framework, individual $i$'s potential outcomes can be represented as,
\begin{equation}
Y_i = \sum_{j=0}^{3}(\mathbb{1}_{T_i = j}) Y_{j,i}
\end{equation}
\noindent Where $\mathbb{1}_{T_i=j}$ is an indicator that equals one if the treatment status of individual $i$ is set to $j$.

\noindent The average treatment effect with respect to each control group, $j$, can be defined,
\begin{equation}
ATE_j = Y(0) - Y(j) \quad \forall j \in \{1,2,3\}
\end{equation}
Thus, the IPW estimator of the treatment effect with respect to each control group, $j$, can be defined as, 

\begin{equation}\label{eq:IPWmulti}
\widehat{IPW}_j=\widehat{E[Y(0)-Y(j)]}_{IPW} = \frac{1}{n} \sum_{i=1}^{n} \left \{\frac{(\mathbb{1}_{T_i=0}) Y_i}{\hat{\Pi_0}(\boldsymbol{X_i})} - \frac{(\mathbb{1}_{T_i=j})Y_i}{\hat{\Pi_j}(\boldsymbol{X_i})}\right \}
\end{equation}
~\\
\noindent Where $\hat{\Pi_j}(\boldsymbol{X_i}) = Pr(T_i=j|\boldsymbol{X_i}) \quad \forall \, j \in \{0,1,2,3\}$, is predicted for each individual, $i$, using the estimated coefficients obtained from a multinomial logit model. 

Similarly, we may define the AIPW estimator of the treatment effect with respect to each control group, $j$, as,

\begin{align}\label{eq:AIPWmulti}
\widehat{AIPW_{j}} = E[\widehat{Y(1)-Y(j)]}_{AIPW} =& \frac{1}{n} \sum_{i=1}^{n}\bigg \{ \bigg[ \overbrace{ \frac{(\mathbb{1}_{T_i=0}) Y_i}{\hat{\Pi_0}(\boldsymbol{X_i})} - \frac{(\mathbb{1}_{T_i=j})Y_i}{\hat{\Pi_j}(\boldsymbol{X_i})}}^{IPTW}  \bigg]- \frac{(\mathbb{1}_{T_i=0}) \hat{\Pi}_0(\boldsymbol{X_i}) - (\mathbb{1}_{T_i=j}) \hat{\Pi}_j(\boldsymbol{X_i})}{\hat{\Pi}_0(\boldsymbol{X_i}) \, \hat{\Pi}_j(\boldsymbol{X_i})} \nonumber \\[10pt]
& \bigg[ \hat{\Pi}_j(\boldsymbol{X_i}) \cdot E[\hat{Y_i}|T_i=0,\boldsymbol{X_i}] +\hat{\Pi}_0(\boldsymbol{X_i}) \cdot E[\hat{Y_i}|T_i=j,\boldsymbol{X_i}] \bigg] \bigg \}
\end{align}