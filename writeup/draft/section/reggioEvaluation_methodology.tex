
Because there are two stages of early childhood interventions, (i) ages 0-3 and (ii) ages 3-6, it is important to consider both when estimating treatment effects of either intervention on later outcomes. Table~\ref{tab:cases-treat} shows the possible cases of receiving early childhood intervention in our data, where 0 indicates not attending and 1 indicates attending. At this stage, we limit the type of infant-toddler centers and preschools to municipal only. 

\begin{table}[H]
\caption{Possible Cases of Treatment} \label{tab:cases-treat}
\begin{tabular}{C{1.8cm} R{0.7cm} C{2cm} C{2cm}}
  
		& & \multicolumn{2}{c}{Preschool (Ages 3-6)} \\
		& & 0 & 1 \\ \cline{3-4}            
        								 &  & \multicolumn{1}{|c|}{} & \multicolumn{1}{c|}{} \\
        							& 0 & \multicolumn{1}{|c|}{(0,0)} & \multicolumn{1}{c|}{(0,1)} \\ 
        				ITC				&  & \multicolumn{1}{|c|}{} & \multicolumn{1}{c|}{} \\ \cline{3-4}
                        (Age 0-3)  		&  & \multicolumn{1}{|c|}{} & \multicolumn{1}{c|}{} \\
        								& 1 & \multicolumn{1}{|c|}{(1,0)} & \multicolumn{1}{c|}{(1,1)} \\ 
        								&  & \multicolumn{1}{|c|}{} & \multicolumn{1}{c|}{} \\ \cline{3-4}
\end{tabular}
\begin{flushleft}
\footnotesize{Note:} We only consider municipal ITCs (infant-toddler-centers, ages 0-3) and preschools (ages 3-6). (0,0): did not attend any municipal school for both ages 0-3 and 3-6; (1,0): attended a municipal school for ages 0-3 but did \textit{not} attend for ages 3-6; (0,1): did \textit{not} attend a municipal school for ages 0-3 but did attend for ages 3-6; (1,1): attended a municipal school for both ages 0-3 and 3-6.
\end{flushleft}
\end{table}

\subsection{Estimating Effects of Infant-Toddler Centers}
There are two main methods to test the effect of attending infant-toddler centers. The first is to compare people who did not attend infant-toddler care or preschool with people who only attended municipal infant-toddler care. Using the notation in Table~\ref{tab:cases-treat}, this comparison is between (0,0) and (1,0). The second method is to compare people who only attended municipal preschool with people who attended both municipal infant-toddler centers and preschools. That is, to compare (0,1) and (1,1). The hypotheses are formally written as
\begin{eqnarray}
H_1: &  Y_{0,0} = Y_{1,0} \\ 
H_2: &  Y_{0,1} = Y_{1,1} 
\end{eqnarray}
\noindent where $Y_{i,j}$ is the outcome of the individuals who attended $i \in \{0,1\}$ infant-toddler care and $j \in \{0,1\}$ preschool.

A possible estimation strategy is to limit the sample to a specific city and a specific cohort constrained to the comparison groups needed according to the hypotheses above. To test $H_1$, we estimate the following regression equation:
\begin{eqnarray}
Y_{i}^{c,h} & = & \alpha + \beta_{0}R_i^{ITC} + \mathbf{X}_i\gamma + \varepsilon_{i}^{c,h}, \\ \nonumber
& \forall & i \in \text{ \{People in city $c$ and cohort $h$ and in group (0,0) or (1,0)\}}
\end{eqnarray}
where $R_i^{ITC}$ is the indicator for attending municipal infant-toddler center and $\mathbf{X_i}$ is the vector of baseline variables for individual $i$. Likewise, to test $H_2$:
\begin{eqnarray}
Y_{i}^{c,h} & = & \alpha + \beta_{0}R_i^{ITC} + \mathbf{X}_i\gamma + \varepsilon_{i}^{c,h}, \\ \nonumber
& \forall & i \in \text{ \{People in city $c$ and cohort $h$ and in group (0,1) or (1,1)\}.}
\end{eqnarray}

One caveat of this analysis is that it uses a limited sample size. In our data, these hypotheses cannot be tested under this strategy for many groups. Table~\ref{tab:num-group} shows the number of individuals available for each group necessary for analysis using this strategy. It is impossible to test $H_1$ in our data, because there are almost no people who attended municipal infant-toddler care without attending preschool (the group (1,0)). While it is possible to test $H_2$ for several groups, the number of observations for the group (1,1) is small for the adult cohorts. 

\begin{table}[H] \caption{Number of Individuals in Each Group} \label{tab:num-group}
\scalebox{0.77}{
\begin{tabular}{l|ccccc|ccccc|ccccc}
\toprule
			& 		\multicolumn{5}{c}{\textbf{Reggio}}		& 	\multicolumn{5}{|c|}{\textbf{Parma}}	& 			\multicolumn{5}{c}{\textbf{Padova}}				\\
			& (0,0) & (1,0) & (0,1) & (1,1) & Total & (0,0) & (1,0) & (0,1) & (1,1) & Total  & (0,0) & (1,0) & (0,1) & (1,1) & Total \\ \midrule
Child		& 2 & 0 & 46 & 117 & \textbf{311} & 5 & 1 & 35 & 100 & \textbf{291} & 2 & 0 & 31 & 36 & \textbf{278} \\
Migrant		& 4 & 0	& 24 & 26 & \textbf{110} & 4 & 0 & 12 & 23 & \textbf{58} & 5 & 0 & 18 & 16 & \textbf{113} \\
Adolescent 	& 7 & 0 & 45 &	116 & \textbf{300} & 4 & 0 & 49 & 61 & \textbf{254} & 1 & 0 & 55 & 37 & \textbf{282} \\
Age-30		& 57 & 0 & \cellcolor{blue!25}95 &	\cellcolor{blue!25}53 & \textbf{280} & 43 & 0 & \cellcolor{blue!25}64 & \cellcolor{blue!25}29 & \textbf{251} & 47 & 0 & 25 & 9 & \textbf{251} \\
Age-40		& 80 & 0 & \cellcolor{blue!25}97 &	\cellcolor{blue!25}28 & \textbf{285} & 115 & 1 & 35 & 16 & \textbf{254} & 75 & 0 & 25 & 2 & \textbf{252} \\
Age-50		& 146 & 0 &	8 & 0 & \textbf{200} & 71 & 0 & 4 & 8 & \textbf{103} & 55 & 0 & 11 & 0 & \textbf{146} \\ \bottomrule
\end{tabular}}
\end{table}

In Table \ref{tab:num-group}, the groups subject to our estimation are highlighted. Since there are more outcomes for adult cohorts than for younger cohorts, we first focus on analyzing the effect of infant-toddler care on the adult cohorts. Based on the available number of individuals in each cell and the history of the foundation date of municipal infant-toddler care for each city, we decide to test $H_2$ for the following groups:
\begin{itemize}
\item Age-30 individuals in Reggio who did not attend any infant-toddler center and attended municipal preschool (the group (0,1)) \textbf{vs.} Age-30 individuals in Reggio who attended both municipal infant-toddler center and municipal preschool (the group (1,1))
\item Age-40 individuals in Reggio who did not attend no infant-toddler center and attended municipal preschool (the group (0,1)) \textbf{vs.} Age-40 individuals in Reggio who attended both municipal infant-toddler center and municipal preschool (the group (1,1))
\end{itemize}

These two comparisons show effects of the Reggio-Approach infant-toddler care for each age cohort. In this draft, we only include the comparisons for individuals in Reggio Emilia. 

\subsection{Estimating Effects of Preschools}
\subsubsection{Baseline OLS}
We perform several comparisons using a baseline OLS model. We compare individuals from Reggio Emilia who attended a Reggio Approach preschool to those in Reggio Emilia who attended (i) any type of preschool, (ii) no preschool, (iii) state preschool, (iv) religious preschool, and (v) no preschool. In the main paper, we only present estimates of the first comparison. Estimates from the other comparisons are reported in Appendices~\ref{appendix:no-preschool} through~\ref{appendix:religious}. We exclude the cohort of adults in their 50s because the Reggio Approach was not available at that time. We also exclude individuals in Parma and Padova from this analysis.

Let $i$ index individuals in Reggio Emilia and let $R_i^{P}$ indicate whether that individual attended a municipal preschool. Given we restrict this model to those in Reggio Emilia, $R_i^{P} = 1$ implies the individual attended the Reggio Approach. We select a vector, $\bm{X}_i$, of baseline control variables with the lowest BIC to account for family background.\footnote{These variables are: gender, whether the individual took the computer-assisted (CAPI), number of siblings, an indicator if the mother's maximum education was middle school, an indicator if the father's maximum education is university, and two indicators for the number of siblings (one indicating having two or more siblings, the other indicating having three or more siblings).} For both cohorts, we estimate $\beta$ in the simple model

\begin{equation}
	Y_i = \alpha_0 + \alpha_1 R_i^{P} + \bm{X}_i\gamma + \varepsilon_i
	\label{eq:ra-v-none}
\end{equation}

\noindent where we assume $\varepsilon_i$ to be a random disturbance. 

\subsubsection{Difference-in-Difference}
Although the baseline OLS allows for a basic understanding of the relationship between the Reggio Approach and relevant outcomes, it does not include any city-level effect. This difference-in-difference model accounts for city-level effects by comparing the difference in outcomes across cities for different preschool types after fixing the cohort. 

We restrict the analysis to individuals in the age-30 and age-40 cohorts. We present comparisons to specific school types in Appendices~\ref{appendix:no-preschool} through~\ref{appendix:religious}. In the main paper, we present comparisons between municipal schools and all other types of preschools pooled together.

As an example, we only consider age-30 individuals from Reggio Emilia and Parma who either attended municipal preschool or did not attend any preschool. We can write the estimation equation for this group as:
\begin{eqnarray}  \label{eq:specific2}
Y_i & = \beta_0 + \beta_1 Reggio_i + \beta_2 R_i^P + \beta_3 Reggio_i * R_i^P + \bm{X}_i\delta + \epsilon_i 
\end{eqnarray}
where $Reggio_i$ is the indicator for $i$ living in Reggio Emilia. In this regression, $\beta_3$ has an interpretation of (Reggio Municipal - Parma Municipal) - (Reggio None - Parma None). The first difference shows the difference between outcomes of those who attended municipal preschools in Reggio Emilia and those who attended municipal preschools in Parma. However, there can be inherent differences between individuals in Reggio Emilia and those in Parma for the age-30 cohort. The second difference is intended to capture the inherent difference in outcomes between two cities and to eliminate the city effect. An analogous interpretation is applied to difference-in-difference comparisons between Reggio Emilia and Padova and considering individuals in the age-40 cohort. 

\subsubsection{Propensity Score Matching}
The baseline OLS and difference-in-difference models do not account for selection into the Reggio Approach. In order to account for this, we use propensity score matching.

For each cohort, we fit a probit using only individuals in Reggio Emilia. 

\begin{equation}
	R_i^{P} = \gamma_0 + \bm{X}_i\gamma + \upsilon_i
\end{equation}

Using these estimated coefficients, we predict $\hat{R}_i^{P}$ for individuals in Parma and Padova, as well as those in Reggio Emilia. Using this predicted $\hat{R}_i$, we define the estimated inverse propensity to enroll in the Reggio Approach, $\hat{\omega}_i$,

\begin{align}
	\hat{\omega}_i &= 
	\begin{cases}
		\frac{1}{\hat{R}_i^{P}} & P_i = 1  \\
		\frac{1}{1-\hat{R}_i^{P}} & P_i = 0,
	\end{cases}
\end{align}

\noindent where $P_i$ indicate the subject attended any preschool type in any city. We fit \eqref{eq:ra-v-none} weighing the individuals with $\omega_i$. We include $P_i$ in the model. This allows us to capture the effect of the Reggio Approach compared to no alternative preschool as well as the additional effect beyond the available alternatives. In Section~\ref{sec:results}, we present the additional effects. That is, using $\hat{\omega}_i$ weights, we present estimates of $\psi_1$ from

\begin{equation}
	Y_i = \psi_0 + \psi_1 R_i^{P} + \psi_2 P_i + \bm{X}_i \psi + \varepsilon'_i.
\end{equation}


