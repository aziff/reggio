The sample is a subset of the individuals in Reggio Emilia, Parma, and Padova who were born in the year ranges of the five cohorts.  These individuals were collected from the population registries in each of the cities. The sample was then restricted to those individuals living in the same city in which they were raised. All cohorts except the youngest one comprised only individuals who are Italian citizens. In contrast, the youngest cohort includes an oversampling of immigrant children.\footnote{In the adult cohorts there was no immigrant who was born in the any of the three cities and still lived there. In the adolescent cohort, the number was immigrant born was extremely small.} The sample from Reggio Emilia, across all cohorts, includes an oversampling of those who attended municipal schools, as this is the treatment group.

Of the reference sample, 7,109 individuals were randomly selected. Of these, 4,019 completed interviews, resulting in a response rate of 56.5\%. Table~\ref{tab:sample-city-cohort} provides an overview of the birth years for the different cohorts and the counts of the full sample. Table~\ref{tab:sample} provides a detailed tabulation of the sample by city, cohort, and school type.

\begin{table}[H]
\centering
\begin{threeparttable}
	\caption{Description of the Full Sample by Cohort and City}\label{tab:sample-city-cohort}
	\begin{tabular}{l c c c c c c}
\toprule
Cohort & Birth year(s) & Age at interview & \mc{4}{c}{Count} \\
\cmidrule{4-7}
 & 		&						& Reggio Emilia & Parma & Padova & \textbf{Total} \\
\midrule
\textbf{Children} &  &  &  & &  &  \\ 
\quad Italians & 2006 & 6 & 311 & 291& 278 & 880 \\
\quad Migrants & 2006 & 6 & 110 & 58 & 113 & 281 \\
\textbf{Adolescents} & 1994 & 19 & 300 & 254 & 282 & 836 \\
\textbf{Adults 30s} & 1980-1981 & 32 & 280 & 251 & 251 & 782 \\
\textbf{Adults 40s} & 1969-1970 & 43 & 285 & 254 & 252 & 791 \\
\textbf{Adults 50s} & 1954-1959 & 54-60 & 200 & 103 & 146 & 449 \\
\midrule
\textbf{Total}	& 				& & 1,486 & 1,211 & 1,322 & 4,019 \\
\bottomrule
\end{tabular}
\begin{tablenotes}
\footnotesize
Note: This table presents the number of individuals in the full sample. The age at interview is approximate given there is some variation in the interview date and birth year within each cohort.
\end{tablenotes}
\end{threeparttable}
\end{table}

\begin{table}[H]
\centering
\scalebox{0.7}{
\begin{threeparttable}
	\caption{Tabulation by Cohort, City, and School Type}\label{tab:sample}
	\begin{tabular}{l*{15}{c}}
\toprule
            &\mc{5}{c}{Reggio Emilia: 1,471}   &     \mc{5}{c}{ Parma: 1,198}       &      \mc{5}{c}{Padova: 1,305}      \\
           \cmidrule(lr){2-6} \cmidrule(lr){7-11} \cmidrule(lr){12-16} 
Cohort      &        None&       Muni.&       State&      Relig.&       Priv.&        None&       Muni.&       State&      Relig.&       Priv.&        None&       Muni.&       State&      Relig.&       Priv.\\
\midrule
Children    &           2&         166&          45&          92&           5&           6&         154&          43&        77&           9&           2&          82&          40&         141&          12\\
Migrants    &           4&          52&          37&          14&           1&           4&          35&          10&          	3&           6&           5&          36&          47&          23&           1\\
Adolescents &           7&         166&          22&          96&           6&           4&         116&          43&     82&           6&           1&          93&          47&         131&           6\\
Adults 30    &          57&         149&          31&          40&           1&          44&          98&          51&        50&          5&       47&         35&         26&            140&    1 \\
Adults 40    &          80&         128&          17&          52&           5&         116&          52&          26&       55&          1&        75&      27 &            24&            123&   \\
Adults 50    &         147&           9&          10&          28&           2&          72&          12&           7&          11&            &        57&      11 &           2 &           68&    2 \\
\midrule
	       &         297&         670&         162&         322&          20&         246&         467&   180&    278&          27&      187&      284 &            186&        626& 22  \\
\bottomrule
\end{tabular}


\begin{tablenotes}
Note: This table shows the sample size by city, cohort, and school type. These numbers do not include individuals with an unidentified preschool type. In total, there are 45 individuals with unidentified preschool type. We separate migrants and children for clarity in this table even though they are in the same birth cohort (year of birth: 2006). None: no preschool; Muni.: municipal preschool;  State: state preschool; Relig.: religious preschool; Priv.: private preschool.
\end{tablenotes}
\end{threeparttable}
}
\end{table}

The structure of the cohorts allows us to study the effects of the Reggio Approach at different points throughout the life cycle. The youngest cohort of children were interviewed when they entered primary school, the adolescent cohort when they ended compulsory schooling, and the adult cohorts capture different points of adulthood to measure key outcomes such as engagement in the labor market, health, and family decisions. This cohort structure also allows us to evaluate the Reggio Approach compared to the alternative early childhood experiences over time.

Separate questionnaires were administered to the children, adolescents, and adults. as well as to the caregivers of the children and adolescents. The questionnaires include items about early childhood experiences, family structure, education, interaction with non-Italians (or with Italians in the case of the migrant children), and measures of cognitive and social-emotional skills. The questionnaires for adults additionally included items about occupation, income, health, and life satisfaction. 

Table \ref{table:summaryStat_baseline} presents summary statistics for baseline variables by cohort and city. As mentioned above, certain baseline variables are missing for the adult cohorts due to differences in questionnaires administered to adults and children. The table illustrates differences and similarities in parental, caregiver and family characteristics across cities as well as over time.

\begin{landscape}
\singlespace
\setlength{\tabcolsep}{2pt}
\begin{center}
\scriptsize{
\begin{longtable}{L{6cm} c c c p{.5cm} c c c p{.5cm} c c c p{.5cm} c c c p{.5cm} c c c}
\hline\multicolumn{20}{L{24cm}}{\textbf{Note:} Means are reported for each variable by cohort and city. Standard Deviations are reported in italics below each mean estimate. A . denotes that the variable is not defined for a specific cohort.}
\endfoot
\caption{Summary statistics for baseline variables by cohort and city} \label{table:summaryStat_baseline} \\
\hline
& \multicolumn{3}{c}{\textbf{Children}} & & \multicolumn{3}{c}{\textbf{Adolescents}} & & \multicolumn{3}{c}{\textbf{Adults 30}} & & \multicolumn{3}{c}{\textbf{Adults 40}} & & \multicolumn{3}{c}{\textbf{Adults 50}}\\
& \scriptsize{Reggio} & \scriptsize{Parma}& \scriptsize{Padova} & & \scriptsize{Reggio} & \scriptsize{Parma}& \scriptsize{Padova} & & \scriptsize{Reggio} & \scriptsize{Parma}& \scriptsize{Padova} & & \scriptsize{Reggio} & \scriptsize{Parma}& \scriptsize{Padova} & & \scriptsize{Reggio} & \scriptsize{Parma}& \scriptsize{Padova}\\
\hline \\[.2em] \endhead
 \quad Male & 0.54 &      0.56 &      0.52 & &      0.43 &      0.44 &      0.48 & &      0.60 &      0.53 &      0.55 & &      0.54 &      0.49 &      0.48 & &      0.47 &      0.38 &      0.46 \\*
 \quad & $\mathit{     0.50}$ & $\mathit{     0.50}$ & $\mathit{     0.50}$ & & $\mathit{     0.50}$ & $\mathit{     0.50}$ & $\mathit{     0.50}$ & & $\mathit{     0.49}$ & $\mathit{     0.50}$ & $\mathit{     0.50}$ & & $\mathit{     0.50}$ & $\mathit{     0.50}$ & $\mathit{     0.50}$ & & $\mathit{     0.50}$ & $\mathit{     0.49}$ & $\mathit{     0.50}$ \\[.2em]
 \quad Low birthweight & 0.08 &      0.07 &      0.05 & &      0.05 &      0.06 &      0.05 & &         . &         . &         . & &         . &         . &         . & &         . &         . &         . \\*
 \quad & $\mathit{     0.27}$ & $\mathit{     0.25}$ & $\mathit{     0.21}$ & & $\mathit{     0.23}$ & $\mathit{     0.24}$ & $\mathit{     0.21}$ & & $\mathit{        .}$ & $\mathit{        .}$ & $\mathit{        .}$ & & $\mathit{        .}$ & $\mathit{        .}$ & $\mathit{        .}$ & & $\mathit{        .}$ & $\mathit{        .}$ & $\mathit{        .}$ \\[.2em]
 \quad Premature birth & 0.10 &      0.08 &      0.07 & &      0.06 &      0.10 &      0.07 & &         . &         . &         . & &         . &         . &         . & &         . &         . &         . \\*
 \quad & $\mathit{     0.30}$ & $\mathit{     0.26}$ & $\mathit{     0.25}$ & & $\mathit{     0.24}$ & $\mathit{     0.30}$ & $\mathit{     0.25}$ & & $\mathit{        .}$ & $\mathit{        .}$ & $\mathit{        .}$ & & $\mathit{        .}$ & $\mathit{        .}$ & $\mathit{        .}$ & & $\mathit{        .}$ & $\mathit{        .}$ & $\mathit{        .}$ \\[.2em]
 \quad CAPI & 0.55 &      0.43 &      0.47 & &      0.43 &      0.55 &      0.49 & &      0.57 &      0.40 &      0.35 & &      0.63 &      0.34 &      0.35 & &      0.45 &      0.35 &      0.28 \\*
 \quad & $\mathit{     0.50}$ & $\mathit{     0.50}$ & $\mathit{     0.50}$ & & $\mathit{     0.50}$ & $\mathit{     0.50}$ & $\mathit{     0.50}$ & & $\mathit{     0.50}$ & $\mathit{     0.49}$ & $\mathit{     0.48}$ & & $\mathit{     0.48}$ & $\mathit{     0.48}$ & $\mathit{     0.48}$ & & $\mathit{     0.50}$ & $\mathit{     0.48}$ & $\mathit{     0.45}$ \\[.2em]
 \quad Born to teenaged mother & 0.00 &      0.01 &      0.01 & &      0.01 &      0.02 &      0.00 & &      1.00 &      1.00 &      1.00 & &      1.00 &      1.00 &      1.00 & &      1.00 &      1.00 &      1.00 \\*
 \quad & $\mathit{     0.00}$ & $\mathit{     0.08}$ & $\mathit{     0.08}$ & & $\mathit{     0.11}$ & $\mathit{     0.12}$ & $\mathit{     0.00}$ & & $\mathit{     0.00}$ & $\mathit{     0.00}$ & $\mathit{     0.00}$ & & $\mathit{     0.00}$ & $\mathit{     0.00}$ & $\mathit{     0.00}$ & & $\mathit{     0.00}$ & $\mathit{     0.00}$ & $\mathit{     0.00}$ \\[.2em]
 \quad Mom born in province & 0.51 &      0.60 &      0.69 & &      0.68 &      0.68 &      0.78 & &      0.84 &      0.70 &      0.71 & &      0.80 &      0.74 &      0.63 & &      0.78 &      0.80 &      0.76 \\*
 \quad & $\mathit{     0.50}$ & $\mathit{     0.49}$ & $\mathit{     0.46}$ & & $\mathit{     0.47}$ & $\mathit{     0.47}$ & $\mathit{     0.41}$ & & $\mathit{     0.36}$ & $\mathit{     0.46}$ & $\mathit{     0.46}$ & & $\mathit{     0.40}$ & $\mathit{     0.44}$ & $\mathit{     0.48}$ & & $\mathit{     0.42}$ & $\mathit{     0.40}$ & $\mathit{     0.43}$ \\[.2em]
 \quad Mom Max Edu: Low & 0.17 &      0.07 &      0.10 & &      0.16 &      0.11 &      0.14 & &      0.00 &      0.00 &      0.00 & &      0.02 &      0.00 &      0.00 & &      0.01 &      0.04 &      0.01 \\*
 \quad & $\mathit{     0.38}$ & $\mathit{     0.25}$ & $\mathit{     0.30}$ & & $\mathit{     0.36}$ & $\mathit{     0.31}$ & $\mathit{     0.35}$ & & $\mathit{     0.06}$ & $\mathit{     0.00}$ & $\mathit{     0.00}$ & & $\mathit{     0.13}$ & $\mathit{     0.00}$ & $\mathit{     0.06}$ & & $\mathit{     0.10}$ & $\mathit{     0.19}$ & $\mathit{     0.12}$ \\[.2em]
 \quad Mom Max Edu: Middle School & 0.08 &      0.05 &      0.09 & &      0.09 &      0.10 &      0.11 & &      0.03 &      0.07 &      0.10 & &      0.19 &      0.24 &      0.23 & &      0.41 &      0.55 &      0.64 \\*
 \quad & $\mathit{     0.27}$ & $\mathit{     0.23}$ & $\mathit{     0.29}$ & & $\mathit{     0.29}$ & $\mathit{     0.30}$ & $\mathit{     0.31}$ & & $\mathit{     0.18}$ & $\mathit{     0.25}$ & $\mathit{     0.30}$ & & $\mathit{     0.39}$ & $\mathit{     0.43}$ & $\mathit{     0.42}$ & & $\mathit{     0.49}$ & $\mathit{     0.50}$ & $\mathit{     0.48}$ \\[.2em]
 \quad Mom Max Edu: High School & 0.45 &      0.41 &      0.45 & &      0.48 &      0.44 &      0.43 & &      0.42 &      0.30 &      0.35 & &      0.47 &      0.35 &      0.35 & &      0.36 &      0.26 &      0.18 \\*
 \quad & $\mathit{     0.50}$ & $\mathit{     0.49}$ & $\mathit{     0.50}$ & & $\mathit{     0.50}$ & $\mathit{     0.50}$ & $\mathit{     0.50}$ & & $\mathit{     0.49}$ & $\mathit{     0.46}$ & $\mathit{     0.48}$ & & $\mathit{     0.50}$ & $\mathit{     0.48}$ & $\mathit{     0.48}$ & & $\mathit{     0.48}$ & $\mathit{     0.44}$ & $\mathit{     0.39}$ \\[.2em]
 \quad Mom Max Edu: University & 0.28 &      0.46 &      0.36 & &      0.25 &      0.33 &      0.29 & &      0.55 &      0.63 &      0.54 & &      0.31 &      0.41 &      0.41 & &      0.22 &      0.15 &      0.15 \\*
 \quad & $\mathit{     0.45}$ & $\mathit{     0.50}$ & $\mathit{     0.48}$ & & $\mathit{     0.43}$ & $\mathit{     0.47}$ & $\mathit{     0.46}$ & & $\mathit{     0.50}$ & $\mathit{     0.48}$ & $\mathit{     0.50}$ & & $\mathit{     0.46}$ & $\mathit{     0.49}$ & $\mathit{     0.49}$ & & $\mathit{     0.42}$ & $\mathit{     0.35}$ & $\mathit{     0.36}$ \\[.2em]
 \quad Born to teenaged father & 0.00 &      0.00 &      0.01 & &      0.00 &      0.01 &      0.00 & &      1.00 &      1.00 &      1.00 & &      1.00 &      1.00 &      1.00 & &      1.00 &      1.00 &      1.00 \\*
 \quad & $\mathit{     0.00}$ & $\mathit{     0.06}$ & $\mathit{     0.08}$ & & $\mathit{     0.00}$ & $\mathit{     0.09}$ & $\mathit{     0.06}$ & & $\mathit{     0.00}$ & $\mathit{     0.00}$ & $\mathit{     0.00}$ & & $\mathit{     0.00}$ & $\mathit{     0.00}$ & $\mathit{     0.00}$ & & $\mathit{     0.00}$ & $\mathit{     0.00}$ & $\mathit{     0.00}$ \\[.2em]
 \quad Father born in province & 0.52 &      0.59 &      0.64 & &      0.58 &      0.61 &      0.73 & &      0.87 &      0.79 &      0.76 & &      0.78 &      0.85 &      0.73 & &      0.84 &      0.65 &      0.82 \\*
 \quad & $\mathit{     0.50}$ & $\mathit{     0.49}$ & $\mathit{     0.48}$ & & $\mathit{     0.49}$ & $\mathit{     0.49}$ & $\mathit{     0.44}$ & & $\mathit{     0.34}$ & $\mathit{     0.41}$ & $\mathit{     0.43}$ & & $\mathit{     0.42}$ & $\mathit{     0.36}$ & $\mathit{     0.44}$ & & $\mathit{     0.37}$ & $\mathit{     0.48}$ & $\mathit{     0.38}$ \\[.2em]
 \quad Dad Max Edu: Low & 0.23 &      0.12 &      0.09 & &      0.19 &      0.16 &      0.14 & &         . &         . &         . & &      0.02 &      0.00 &      0.00 & &      0.01 &      0.04 &      0.02 \\*
 \quad & $\mathit{     0.42}$ & $\mathit{     0.33}$ & $\mathit{     0.29}$ & & $\mathit{     0.40}$ & $\mathit{     0.37}$ & $\mathit{     0.35}$ & & $\mathit{        .}$ & $\mathit{        .}$ & $\mathit{        .}$ & & $\mathit{     0.14}$ & $\mathit{     0.06}$ & $\mathit{     0.06}$ & & $\mathit{     0.07}$ & $\mathit{     0.19}$ & $\mathit{     0.14}$ \\[.2em]
 \quad Dad Max Edu: Middle School & 0.08 &      0.10 &      0.09 & &      0.09 &      0.07 &      0.10 & &      0.03 &      0.08 &      0.10 & &      0.19 &      0.22 &      0.14 & &      0.35 &      0.55 &      0.52 \\*
 \quad & $\mathit{     0.27}$ & $\mathit{     0.30}$ & $\mathit{     0.28}$ & & $\mathit{     0.28}$ & $\mathit{     0.26}$ & $\mathit{     0.30}$ & & $\mathit{     0.17}$ & $\mathit{     0.27}$ & $\mathit{     0.30}$ & & $\mathit{     0.39}$ & $\mathit{     0.41}$ & $\mathit{     0.35}$ & & $\mathit{     0.48}$ & $\mathit{     0.50}$ & $\mathit{     0.50}$ \\[.2em]
 \quad Dad Max Edu: High School & 0.35 &      0.36 &      0.42 & &      0.40 &      0.36 &      0.39 & &      0.37 &      0.32 &      0.32 & &      0.45 &      0.33 &      0.27 & &      0.36 &      0.19 &      0.17 \\*
 \quad & $\mathit{     0.48}$ & $\mathit{     0.48}$ & $\mathit{     0.49}$ & & $\mathit{     0.49}$ & $\mathit{     0.48}$ & $\mathit{     0.49}$ & & $\mathit{     0.48}$ & $\mathit{     0.47}$ & $\mathit{     0.47}$ & & $\mathit{     0.50}$ & $\mathit{     0.47}$ & $\mathit{     0.44}$ & & $\mathit{     0.48}$ & $\mathit{     0.40}$ & $\mathit{     0.38}$ \\[.2em]
 \quad Dad Max Edu: University & 0.24 &      0.35 &      0.30 & &      0.19 &      0.26 &      0.28 & &      0.60 &      0.60 &      0.57 & &      0.33 &      0.45 &      0.58 & &      0.27 &      0.20 &      0.27 \\*
 \quad & $\mathit{     0.43}$ & $\mathit{     0.48}$ & $\mathit{     0.46}$ & & $\mathit{     0.39}$ & $\mathit{     0.44}$ & $\mathit{     0.45}$ & & $\mathit{     0.49}$ & $\mathit{     0.49}$ & $\mathit{     0.50}$ & & $\mathit{     0.47}$ & $\mathit{     0.50}$ & $\mathit{     0.49}$ & & $\mathit{     0.45}$ & $\mathit{     0.40}$ & $\mathit{     0.45}$ \\[.2em]
 \quad Has 1 sibling & 0.52 &      0.46 &      0.50 & &      0.52 &      0.46 &      0.55 & &      0.36 &      0.33 &      0.45 & &      0.36 &      0.38 &      0.35 & &      0.28 &      0.26 &      0.39 \\*
 \quad & $\mathit{     0.50}$ & $\mathit{     0.50}$ & $\mathit{     0.50}$ & & $\mathit{     0.50}$ & $\mathit{     0.50}$ & $\mathit{     0.50}$ & & $\mathit{     0.48}$ & $\mathit{     0.47}$ & $\mathit{     0.50}$ & & $\mathit{     0.48}$ & $\mathit{     0.49}$ & $\mathit{     0.48}$ & & $\mathit{     0.45}$ & $\mathit{     0.44}$ & $\mathit{     0.49}$ \\[.2em]
 \quad Has 2 siblings & 0.14 &      0.21 &      0.17 & &      0.18 &      0.20 &      0.15 & &      0.24 &      0.33 &      0.25 & &      0.27 &      0.33 &      0.35 & &      0.30 &      0.27 &      0.22 \\*
 \quad & $\mathit{     0.35}$ & $\mathit{     0.41}$ & $\mathit{     0.37}$ & & $\mathit{     0.38}$ & $\mathit{     0.40}$ & $\mathit{     0.36}$ & & $\mathit{     0.43}$ & $\mathit{     0.47}$ & $\mathit{     0.43}$ & & $\mathit{     0.44}$ & $\mathit{     0.47}$ & $\mathit{     0.48}$ & & $\mathit{     0.46}$ & $\mathit{     0.45}$ & $\mathit{     0.42}$ \\[.2em]
 \quad Has more than 2 siblings & 0.06 &      0.04 &      0.05 & &      0.09 &      0.06 &      0.02 & &      0.13 &      0.20 &      0.17 & &      0.20 &      0.19 &      0.26 & &      0.34 &      0.39 &      0.35 \\*
 \quad & $\mathit{     0.23}$ & $\mathit{     0.19}$ & $\mathit{     0.21}$ & & $\mathit{     0.29}$ & $\mathit{     0.24}$ & $\mathit{     0.16}$ & & $\mathit{     0.33}$ & $\mathit{     0.40}$ & $\mathit{     0.37}$ & & $\mathit{     0.40}$ & $\mathit{     0.39}$ & $\mathit{     0.44}$ & & $\mathit{     0.48}$ & $\mathit{     0.49}$ & $\mathit{     0.48}$ \\[.2em]
 \quad Caregiver was Catholic & 0.77 &      0.83 &      0.79 & &      0.75 &      0.86 &      0.73 & &         . &         . &         . & &         . &         . &         . & &         . &         . &         . \\*
 \quad & $\mathit{     0.42}$ & $\mathit{     0.37}$ & $\mathit{     0.41}$ & & $\mathit{     0.43}$ & $\mathit{     0.35}$ & $\mathit{     0.44}$ & & $\mathit{        .}$ & $\mathit{        .}$ & $\mathit{        .}$ & & $\mathit{        .}$ & $\mathit{        .}$ & $\mathit{        .}$ & & $\mathit{        .}$ & $\mathit{        .}$ & $\mathit{        .}$ \\[.2em]
 \quad Caregiver was faithful and Catholic & 0.47 &      0.52 &      0.51 & &      0.45 &      0.54 &      0.45 & &         . &         . &         . & &         . &         . &         . & &         . &         . &         . \\*
 \quad & $\mathit{     0.50}$ & $\mathit{     0.50}$ & $\mathit{     0.50}$ & & $\mathit{     0.50}$ & $\mathit{     0.50}$ & $\mathit{     0.50}$ & & $\mathit{        .}$ & $\mathit{        .}$ & $\mathit{        .}$ & & $\mathit{        .}$ & $\mathit{        .}$ & $\mathit{        .}$ & & $\mathit{        .}$ & $\mathit{        .}$ & $\mathit{        .}$ \\[.2em]
 \quad Caregiver owned house & 0.58 &      0.71 &      0.66 & &      0.84 &      0.81 &      0.77 & &         . &         . &         . & &         . &         . &         . & &         . &         . &         . \\*
 \quad & $\mathit{     0.49}$ & $\mathit{     0.46}$ & $\mathit{     0.48}$ & & $\mathit{     0.37}$ & $\mathit{     0.39}$ & $\mathit{     0.42}$ & & $\mathit{        .}$ & $\mathit{        .}$ & $\mathit{        .}$ & & $\mathit{        .}$ & $\mathit{        .}$ & $\mathit{        .}$ & & $\mathit{        .}$ & $\mathit{        .}$ & $\mathit{        .}$ \\[.2em]
 \quad Caregiver was a migrant & 0.07 &      0.02 &      0.02 & &      0.01 &      0.02 &      0.00 & &         . &         . &         . & &         . &         . &         . & &         . &         . &         . \\*
 \quad & $\mathit{     0.26}$ & $\mathit{     0.14}$ & $\mathit{     0.15}$ & & $\mathit{     0.11}$ & $\mathit{     0.12}$ & $\mathit{     0.00}$ & & $\mathit{        .}$ & $\mathit{        .}$ & $\mathit{        .}$ & & $\mathit{        .}$ & $\mathit{        .}$ & $\mathit{        .}$ & & $\mathit{        .}$ & $\mathit{        .}$ & $\mathit{        .}$ \\[.2em]
 \quad Caregiver Income: 5,000 euros or less & 0.01 &      0.02 &      0.03 & &      0.00 &      0.02 &      0.04 & &         . &         . &         . & &         . &         . &         . & &         . &         . &         . \\*
 \quad & $\mathit{     0.11}$ & $\mathit{     0.15}$ & $\mathit{     0.18}$ & & $\mathit{     0.06}$ & $\mathit{     0.15}$ & $\mathit{     0.19}$ & & $\mathit{        .}$ & $\mathit{        .}$ & $\mathit{        .}$ & & $\mathit{        .}$ & $\mathit{        .}$ & $\mathit{        .}$ & & $\mathit{        .}$ & $\mathit{        .}$ & $\mathit{        .}$ \\[.2em]
 \quad Caregiver Income: 5,001-10,000 euros & 0.01 &      0.02 &      0.01 & &      0.01 &      0.01 &      0.01 & &         . &         . &         . & &         . &         . &         . & &         . &         . &         . \\*
 \quad & $\mathit{     0.11}$ & $\mathit{     0.13}$ & $\mathit{     0.12}$ & & $\mathit{     0.10}$ & $\mathit{     0.09}$ & $\mathit{     0.08}$ & & $\mathit{        .}$ & $\mathit{        .}$ & $\mathit{        .}$ & & $\mathit{        .}$ & $\mathit{        .}$ & $\mathit{        .}$ & & $\mathit{        .}$ & $\mathit{        .}$ & $\mathit{        .}$ \\[.2em]
 \quad Caregiver Income: 10,001-25,000 euros & 0.17 &      0.19 &      0.15 & &      0.18 &      0.18 &      0.10 & &         . &         . &         . & &         . &         . &         . & &         . &         . &         . \\*
 \quad & $\mathit{     0.38}$ & $\mathit{     0.39}$ & $\mathit{     0.36}$ & & $\mathit{     0.39}$ & $\mathit{     0.38}$ & $\mathit{     0.30}$ & & $\mathit{        .}$ & $\mathit{        .}$ & $\mathit{        .}$ & & $\mathit{        .}$ & $\mathit{        .}$ & $\mathit{        .}$ & & $\mathit{        .}$ & $\mathit{        .}$ & $\mathit{        .}$ \\[.2em]
 \quad Caregiver Income: 25,001-50,000 euros & 0.32 &      0.41 &      0.32 & &      0.32 &      0.29 &      0.24 & &         . &         . &         . & &         . &         . &         . & &         . &         . &         . \\*
 \quad & $\mathit{     0.47}$ & $\mathit{     0.49}$ & $\mathit{     0.47}$ & & $\mathit{     0.47}$ & $\mathit{     0.46}$ & $\mathit{     0.43}$ & & $\mathit{        .}$ & $\mathit{        .}$ & $\mathit{        .}$ & & $\mathit{        .}$ & $\mathit{        .}$ & $\mathit{        .}$ & & $\mathit{        .}$ & $\mathit{        .}$ & $\mathit{        .}$ \\[.2em]
 \quad Caregiver Income: 50,001-100,000 euros & 0.19 &      0.19 &      0.13 & &      0.24 &      0.24 &      0.11 & &         . &         . &         . & &         . &         . &         . & &         . &         . &         . \\*
 \quad & $\mathit{     0.40}$ & $\mathit{     0.39}$ & $\mathit{     0.34}$ & & $\mathit{     0.43}$ & $\mathit{     0.43}$ & $\mathit{     0.31}$ & & $\mathit{        .}$ & $\mathit{        .}$ & $\mathit{        .}$ & & $\mathit{        .}$ & $\mathit{        .}$ & $\mathit{        .}$ & & $\mathit{        .}$ & $\mathit{        .}$ & $\mathit{        .}$ \\[.2em]
 \quad Caregiver Income: 100,001-250,000 euros & 0.02 &      0.02 &      0.03 & &      0.04 &      0.03 &      0.02 & &         . &         . &         . & &         . &         . &         . & &         . &         . &         . \\*
 \quad & $\mathit{     0.15}$ & $\mathit{     0.14}$ & $\mathit{     0.17}$ & & $\mathit{     0.20}$ & $\mathit{     0.17}$ & $\mathit{     0.16}$ & & $\mathit{        .}$ & $\mathit{        .}$ & $\mathit{        .}$ & & $\mathit{        .}$ & $\mathit{        .}$ & $\mathit{        .}$ & & $\mathit{        .}$ & $\mathit{        .}$ & $\mathit{        .}$ \\[.2em]
 \quad Caregiver Income: > 250,000 euros & . &         . &         . & &      0.00 &      0.00 &      0.00 & &         . &         . &         . & &         . &         . &         . & &         . &         . &         . \\*
 \quad & $\mathit{        .}$ & $\mathit{        .}$ & $\mathit{        .}$ & & $\mathit{     0.06}$ & $\mathit{     0.00}$ & $\mathit{     0.00}$ & & $\mathit{        .}$ & $\mathit{        .}$ & $\mathit{        .}$ & & $\mathit{        .}$ & $\mathit{        .}$ & $\mathit{        .}$ & & $\mathit{        .}$ & $\mathit{        .}$ & $\mathit{        .}$ \\[.2em]
 \quad Caregiver was religious & 0.85 &      0.87 &      0.80 & &      0.77 &      0.87 &      0.74 & &      0.50 &      0.73 &      0.72 & &      0.50 &      0.75 &      0.75 & &      0.64 &      0.71 &      0.77 \\*
 \quad & $\mathit{     0.36}$ & $\mathit{     0.34}$ & $\mathit{     0.40}$ & & $\mathit{     0.42}$ & $\mathit{     0.34}$ & $\mathit{     0.44}$ & & $\mathit{     0.50}$ & $\mathit{     0.45}$ & $\mathit{     0.45}$ & & $\mathit{     0.50}$ & $\mathit{     0.43}$ & $\mathit{     0.44}$ & & $\mathit{     0.48}$ & $\mathit{     0.46}$ & $\mathit{     0.42}$ \\[.2em]
 ~\\[-.5em]
\hline
\end{longtable}
}
\end{center}

\end{landscape}

\subsection{Characteristics of Reggio Emilia, Parma, and Padova}

Parma and Padova are similar to Reggio Emilia along several characteristics, but have different early childhood education offerings.\footnote{See Section~\ref{sec:eceexperiences} for a discussion of the differences in early childhood education in the three cities.} All three cities are in Northern Italy; Reggio Emilia and Parma are in the same region of Emilia Romagna, Padova is in the adjacent region of Veneto. Tables~\ref{table:demo-employ} and~\ref{table:demo-other} describe the cities along demographic characteristics. Reggio Emilia and Parma, in addition to being geographically close are socially and economically similar.

The three cities are similar in employment for different industries, proportion of individuals renting property, and marital status. In 2011, Padova had more individuals with post-secondary degrees compared with Reggio Emilia and Parma. Starting in 1991, the aging index has been higher in Parma than in Reggio Emilia. Starting in 2001, Padova's aging index overtook that of both Reggio Emilia and Parma. A higher aging index indicates there are more individuals over 59 years of age per one hundred individuals under 15 years of age, i.e., an older population with low birth rates.

\begin{landscape}
\begin{table}[ht!]
\begin{center}
\scriptsize{
	\caption{Proportion of Individuals in Different Employment and Industry Categories} \label{table:demo-employ}
	
\begin{tabular}{L{6.5cm} *{3}{*{5}{c} c}}
\hline \\[-7pt]
& \multicolumn{5}{c}{\textbf{Reggio Emilia}} & & \multicolumn{5}{c}{\textbf{Parma}} & & \multicolumn{5}{c}{\textbf{Padova}} \\[3pt]
& \textbf{1971} & \textbf{1981} & \textbf{1991} & \textbf{2001} & \textbf{2011} & & \textbf{1971} & \textbf{1981} & \textbf{1991} & \textbf{2001} & \textbf{2011} & & \textbf{1971} & \textbf{1981} & \textbf{1991} & \textbf{2001} & \textbf{2011} \\[3pt]
\hline \\
\textbf{Employment}\\
\quad Employed (B) & 0.48 & 0.51 & 0.49 & 0.53 & 0.53 & & 0.47 & 0.49 & 0.49 & 0.50 & 0.53 & & 0.45 & 0.46 & 0.45 & 0.47 & 0.49 & \\ 
\quad Employed (F) & 0.28 & 0.37 & 0.38 & 0.43 & 0.46 & & 0.26 & 0.34 & 0.37 & 0.41 & 0.46 & & 0.24 & 0.30 & 0.32 & 0.37 & 0.42 & \\ 
\quad Employed (M) & 0.70 & 0.66 & 0.61 & 0.63 & 0.62 & & 0.70 & 0.66 & 0.62 & 0.60 & 0.60 & & 0.69 & 0.64 & 0.60 & 0.59 & 0.57 & \\[5pt] 
\quad Unemployed (B) & 0.01 & 0.01 & 0.02 & 0.02 & 0.06 & & 0.02 & 0.02 & 0.01 & 0.02 & 0.03 & & 0.02 & 0.01 & 0.02 & 0.03 & 0.04 & \\ 
\quad Unemployed (F) & 0.01 & 0.01 & 0.02 & 0.03 & 0.06 & & 0.01 & 0.02 & 0.01 & 0.02 & 0.03 & & 0.01 & 0.01 & 0.02 & 0.03 & 0.04 & \\ 
\quad Unemployed (M) & 0.02 & 0.01 & 0.02 & 0.02 & 0.05 & & 0.02 & 0.01 & 0.01 & 0.02 & 0.03 & & 0.02 & 0.02 & 0.03 & 0.03 & 0.04 & \\[5pt] 
\quad Homemaker (B) & 0.26 & 0.17 & 0.13 & 0.11 & 0.06 & & 0.28 & 0.20 & 0.16 & 0.12 & 0.07 & & 0.32 & 0.25 & 0.21 & 0.16 & 0.09 & \\ 
\quad Homemaker (F) & 0.50 & 0.33 & 0.25 & 0.20 & 0.11 & & 0.53 & 0.37 & 0.30 & 0.22 & 0.12 & & 0.59 & 0.47 & 0.38 & 0.30 & 0.16 & \\ 
\quad Homemaker (M) & 0.00 & 0.00 & 0.00 & 0.00 & 0.00 & & 0.00 & 0.00 & 0.00 & 0.00 & 0.00 & & 0.00 & 0.00 & 0.00 & 0.00 & 0.01 & \\[5pt] 
\quad Pensioner (B) & 0.15 & 0.21 & 0.23 & 0.24 & 0.25 & & 0.15 & 0.19 & 0.21 & 0.24 & 0.27 & & 0.11 & 0.13 & 0.16 & 0.21 & 0.26 & \\ 
\quad Pensioner (F) & 0.13 & 0.20 & 0.22 & 0.23 & 0.27 & & 0.13 & 0.17 & 0.19 & 0.22 & 0.28 & & 0.07 & 0.09 & 0.12 & 0.18 & 0.27 & \\ 
\quad Pensioner (M) & 0.17 & 0.22 & 0.24 & 0.25 & 0.23 & & 0.18 & 0.21 & 0.23 & 0.26 & 0.25 & & 0.15 & 0.17 & 0.20 & 0.26 & 0.25 & \\[5pt] 
\quad Student (B) & 0.07 & 0.07 & 0.08 & 0.06 & 0.06 & & 0.07 & 0.08 & 0.09 & 0.07 & 0.07 & & 0.09 & 0.11 & 0.11 & 0.08 & 0.08 & \\ 
\quad Student (F) & 0.06 & 0.07 & 0.08 & 0.05 & 0.06 & & 0.06 & 0.08 & 0.08 & 0.06 & 0.06 & & 0.07 & 0.10 & 0.10 & 0.07 & 0.07 & \\ 
\quad Student (M) & 0.08 & 0.08 & 0.08 & 0.06 & 0.07 & & 0.08 & 0.09 & 0.09 & 0.07 & 0.07 & & 0.11 & 0.13 & 0.12 & 0.08 & 0.08 & \\[5pt] 
\quad Other (B) & 0.03 & 0.03 & 0.05 & 0.05 & 0.04 & & 0.02 & 0.02 & 0.04 & 0.05 & 0.04 & & 0.02 & 0.03 & 0.05 & 0.05 & 0.05 & \\ 
\quad Other (F) & 0.03 & 0.02 & 0.05 & 0.05 & 0.04 & & 0.02 & 0.02 & 0.04 & 0.05 & 0.04 & & 0.02 & 0.02 & 0.05 & 0.05 & 0.04 & \\ 
\quad Other (M) & 0.04 & 0.03 & 0.04 & 0.04 & 0.04 & & 0.02 & 0.03 & 0.04 & 0.05 & 0.04 & & 0.03 & 0.04 & 0.05 & 0.05 & 0.05 & \\[5pt] 
\textbf{Industry}\\
\quad Agriculture, Forestry And Fishing (B) &   . & 0.08 & 0.04 & 0.04 & 0.04 & &   . & 0.05 & 0.02 & 0.02 & 0.03 & &   . & 0.01 & 0.01 & 0.01 & 0.01 & \\ 
\quad Agriculture, Forestry And Fishing (F) &   . & 0.06 & 0.03 & 0.03 & 0.02 & &   . & 0.04 & 0.01 & 0.02 & 0.02 & &   . & 0.01 & 0.01 & 0.01 & 0.01 & \\ 
\quad Agriculture, Forestry And Fishing (M) &   . & 0.10 & 0.05 & 0.04 & 0.05 & &   . & 0.05 & 0.03 & 0.03 & 0.04 & &   . & 0.02 & 0.01 & 0.01 & 0.02 & \\[5pt] 
\quad Finance, Professional, Scientific, Admin (B) &   . & 0.07 & 0.11 & 0.11 & 0.14 & &   . & 0.08 & 0.13 & 0.14 & 0.17 & &   . & 0.09 & 0.15 & 0.17 & 0.19 & \\ 
\quad Finance, Professional, Scientific, Admin (F) &   . & 0.06 & 0.12 & 0.12 & 0.15 & &   . & 0.07 & 0.15 & 0.14 & 0.18 & &   . & 0.08 & 0.15 & 0.17 & 0.19 & \\ 
\quad Finance, Professional, Scientific, Admin (M) &   . & 0.07 & 0.10 & 0.11 & 0.13 & &   . & 0.08 & 0.12 & 0.13 & 0.16 & &   . & 0.09 & 0.15 & 0.17 & 0.20 & \\[5pt] 
\quad Trade, Hotels And Restaurants  (B) &   . & 0.19 & 0.20 & 0.19 & 0.18 & &   . & 0.20 & 0.19 & 0.18 & 0.17 & &   . & 0.26 & 0.23 & 0.20 & 0.16 & \\ 
\quad Trade, Hotels And Restaurants  (F) &   . & 0.20 & 0.21 & 0.21 & 0.20 & &   . & 0.21 & 0.21 & 0.20 & 0.18 & &   . & 0.24 & 0.21 & 0.19 & 0.16 & \\ 
\quad Trade, Hotels And Restaurants  (M) &   . & 0.18 & 0.19 & 0.18 & 0.16 & &   . & 0.19 & 0.18 & 0.17 & 0.15 & &   . & 0.26 & 0.23 & 0.20 & 0.17 & \\[5pt] 
\quad Transport, Storage, Info, Communication  (B) &   . & 0.05 & 0.04 & 0.04 & 0.06 & &   . & 0.05 & 0.05 & 0.04 & 0.06 & &   . & 0.06 & 0.05 & 0.05 & 0.07 & \\ 
\quad Transport, Storage, Info, Communication  (F) &   . & 0.02 & 0.03 & 0.02 & 0.03 & &   . & 0.02 & 0.03 & 0.02 & 0.04 & &   . & 0.03 & 0.03 & 0.03 & 0.04 & \\ 
\quad Transport, Storage, Info, Communication  (M) &   . & 0.06 & 0.05 & 0.05 & 0.07 & &   . & 0.07 & 0.06 & 0.05 & 0.08 & &   . & 0.08 & 0.07 & 0.06 & 0.09 & \\[5pt] 
\quad Other Activities  (B) &   . & 0.24 & 0.23 & 0.25 & 0.28 & &   . & 0.25 & 0.25 & 0.28 & 0.31 & &   . & 0.32 & 0.32 & 0.35 & 0.37 & \\ 
\quad Other Activities  (F) &   . & 0.36 & 0.34 & 0.38 & 0.43 & &   . & 0.39 & 0.36 & 0.41 & 0.44 & &   . & 0.47 & 0.44 & 0.47 & 0.51 & \\ 
\quad Other Activities  (M) &   . & 0.16 & 0.16 & 0.15 & 0.15 & &   . & 0.17 & 0.18 & 0.19 & 0.20 & &   . & 0.24 & 0.25 & 0.27 & 0.25 & \\[5pt] 
\hline \\[-7pt]
\multicolumn{19}{L{24cm}}{\textbf{Note:} This table presents the percentage of individuals in different employment and industry categories within each city during each of the 5 listed years. Percentages are reported for females (F), males (M), and both genders (B) combined. The percentages are calculated using the total number of individuals above age 15 for the denominator. Data were collected from ISTAT and regional agencies.}
\end{tabular}

}
\end{center}
\end{table}
\end{landscape}

\begin{landscape}
\begin{table}[ht!]
\begin{center}
\scriptsize{
	\caption{Proportion of Individuals in Different Education, Rental, and Marital Categories} \label{table:demo-other}
	
\begin{tabular}{L{5cm} *{3}{*{5}{c} c}}
\hline \\[-7pt]
& \multicolumn{5}{c}{\textbf{Reggio Emilia}} & & \multicolumn{5}{c}{\textbf{Parma}} & & \multicolumn{5}{c}{\textbf{Padova}} \\[3pt]
& \textbf{1971} & \textbf{1981} & \textbf{1991} & \textbf{2001} & \textbf{2011} & & \textbf{1971} & \textbf{1981} & \textbf{1991} & \textbf{2001} & \textbf{2011} & & \textbf{1971} & \textbf{1981} & \textbf{1991} & \textbf{2001} & \textbf{2011} \\[3pt]
\hline \\
~\\[-4pt]
\textbf{Education}\\
\quad $<$ Primary (B) & 0.27 & 0.15 & 0.10 & 0.08 & 0.07 & & 0.28 & 0.14 & 0.09 & 0.07 & 0.06 & & 0.23 & 0.13 & 0.08 & 0.06 & 0.06 & \\ 
\quad $<$ Primary (F) & 0.31 & 0.17 & 0.11 & 0.09 & 0.08 & & 0.32 & 0.16 & 0.10 & 0.08 & 0.07 & & 0.26 & 0.14 & 0.09 & 0.07 & 0.06 & \\ 
\quad $<$ Primary (M) & 0.23 & 0.13 & 0.08 & 0.07 & 0.07 & & 0.23 & 0.12 & 0.07 & 0.06 & 0.06 & & 0.20 & 0.11 & 0.06 & 0.06 & 0.06 & \\[5pt] 
\quad Primary (B) & 0.45 & 0.43 & 0.34 & 0.26 & 0.18 & & 0.43 & 0.41 & 0.32 & 0.24 & 0.18 & & 0.41 & 0.35 & 0.27 & 0.21 & 0.17 & \\ 
\quad Primary (F) & 0.44 & 0.45 & 0.37 & 0.28 & 0.21 & & 0.42 & 0.43 & 0.35 & 0.27 & 0.20 & & 0.42 & 0.39 & 0.31 & 0.25 & 0.20 & \\ 
\quad Primary (M) & 0.46 & 0.40 & 0.31 & 0.22 & 0.16 & & 0.43 & 0.38 & 0.28 & 0.21 & 0.15 & & 0.39 & 0.31 & 0.22 & 0.17 & 0.13 & \\[5pt] 
\quad Lower Secondary (B) & 0.16 & 0.24 & 0.27 & 0.27 & 0.27 & & 0.16 & 0.24 & 0.28 & 0.25 & 0.25 & & 0.20 & 0.26 & 0.28 & 0.25 & 0.23 & \\ 
\quad Lower Secondary (F) & 0.14 & 0.21 & 0.23 & 0.23 & 0.24 & & 0.15 & 0.21 & 0.25 & 0.23 & 0.22 & & 0.19 & 0.24 & 0.26 & 0.23 & 0.21 & \\ 
\quad Lower Secondary (M) & 0.17 & 0.26 & 0.31 & 0.31 & 0.31 & & 0.18 & 0.26 & 0.31 & 0.28 & 0.27 & & 0.22 & 0.28 & 0.31 & 0.27 & 0.24 & \\[5pt] 
\quad High School (B) & 0.10 & 0.15 & 0.24 & 0.30 & 0.33 & & 0.10 & 0.16 & 0.24 & 0.30 & 0.32 & & 0.12 & 0.19 & 0.27 & 0.30 & 0.31 & \\ 
\quad High School (F) & 0.09 & 0.14 & 0.24 & 0.29 & 0.33 & & 0.09 & 0.16 & 0.24 & 0.29 & 0.31 & & 0.10 & 0.17 & 0.25 & 0.29 & 0.30 & \\ 
\quad High School (M) & 0.11 & 0.16 & 0.24 & 0.30 & 0.33 & & 0.11 & 0.17 & 0.25 & 0.31 & 0.33 & & 0.13 & 0.20 & 0.28 & 0.32 & 0.33 & \\[5pt] 
\quad Post Secondary Degree (B) & 0.02 & 0.04 & 0.06 & 0.10 & 0.14 & & 0.03 & 0.05 & 0.07 & 0.14 & 0.19 & & 0.04 & 0.07 & 0.11 & 0.17 & 0.24 & \\ 
\quad Post Secondary Degree (F) & 0.02 & 0.03 & 0.05 & 0.10 & 0.15 & & 0.02 & 0.04 & 0.06 & 0.13 & 0.20 & & 0.03 & 0.05 & 0.09 & 0.16 & 0.23 & \\ 
\quad Post Secondary Degree (M) & 0.03 & 0.05 & 0.07 & 0.10 & 0.13 & & 0.04 & 0.06 & 0.09 & 0.14 & 0.19 & & 0.06 & 0.09 & 0.13 & 0.19 & 0.24 & \\[5pt] 
~\\[-4pt]
\textbf{Rental Status}\\
%\quad Other (B) & 0.05 & 0.06 & 0.07 & 0.08 & 0.09 & & 0.05 & 0.05 & 0.06 & 0.08 & 0.08 & & 0.04 & 0.04 & 0.05 & 0.06 & 0.08 & \\ 
%\quad Owned (B) & 0.41 & 0.53 & 0.63 & 0.68 & 0.67 & & 0.34 & 0.46 & 0.58 & 0.66 & 0.67 & & 0.39 & 0.47 & 0.62 & 0.69 & 0.70 & \\ 
\quad Rented (B) & 0.53 & 0.41 & 0.30 & 0.23 & 0.23 & & 0.61 & 0.49 & 0.35 & 0.26 & 0.25 & & 0.58 & 0.49 & 0.33 & 0.25 & 0.23 & \\ 
~\\[-4pt]
\textbf{Marital Status}\\
\quad Divorced (B) &   . & 0.02 & 0.02 & 0.04 & 0.06 & &   . & 0.02 & 0.03 & 0.04 & 0.06 & &   . & 0.02 & 0.03 & 0.04 & 0.06 & \\ 
\quad Married (B) & 0.52 & 0.52 & 0.51 & 0.49 & 0.44 & & 0.53 & 0.53 & 0.52 & 0.50 & 0.43 & & 0.48 & 0.48 & 0.48 & 0.47 & 0.43 & \\ 
\quad Never Married (B) & 0.40 & 0.37 & 0.37 & 0.38 & 0.42 & & 0.39 & 0.37 & 0.36 & 0.37 & 0.41 & & 0.46 & 0.43 & 0.41 & 0.40 & 0.41 & \\ 
\quad Widowed (B) & 0.08 & 0.09 & 0.09 & 0.09 & 0.08 & & 0.08 & 0.09 & 0.10 & 0.10 & 0.09 & & 0.07 & 0.07 & 0.09 & 0.09 & 0.09 & \\ 
~\\[-4pt]
\textbf{Population Metrics}\\
\quad Aging Index (B) & 69.49 & 101.51 & 171.58 & 155.22 & 131.09 & & 63.32 & 99.35 & 192.66 & 210.50 & 184.46 & & 44.27 & 73.08 & 160.67 & 202.58 & 205.18 & \\ 
\quad Dependency Ratio (B) & 46.34 & 41.05 & 46.98 & 51.69 & 54.17 & & 47.05 & 47.92 & 43.79 & 50.36 & 56.70 & & 51.97 & 45.65 & 40.58 & 50.29 & 59.41 & \\ 
%\quad Eatio (B) & 1.68 & 3.22 & 4.51 & 3.55 & 3.11 & & 1.51 & 3.34 & 5.24 & 5.00 & 4.35 & & 1.08 & 2.50 & 4.25 & 4.93 & 5.02 & \\ 
\hline \\[-7pt]
\multicolumn{19}{L{24cm}}{\textbf{Note:} This table presents the percentage of individuals in different education, rental and marital categories within each city during each of the 5 listed years. Percentages are reported for females (F), males (M), and both genders (B) combined. The percentages are calculated using the total number of individuals above age 15 for the denominator. Data were collected from ISTAT and regional agencies. Aging Index: number of people older than 59 years old per one hundred people younger than 15 years; Dependency Ratio: number of people older than 64 or younger than 15 divided by the number of people between 15 and 64 years old.}
\end{tabular}


}
\end{center}
\end{table}
\end{landscape}




