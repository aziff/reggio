\subsection{The Survey Data Collection}

Respondents were sampled from the population registries of the cities based on their year of birth. The sample was then restricted to those individuals living in the same city in which they were raised. All cohorts, except the youngest one, are restricted to individuals who are Italian citizens. In contrast, the youngest cohort includes an oversampling of immigrant children.\footnote{In the adult cohorts there was no immigrant who was preschool age in the same school in which they live. In the adolescent cohort, the number was immigrant born was extremely small.} The sample from Reggio Emilia, across all cohorts, includes an oversampling of those who attended municipal schools, as this is considered the treatment group.

Of the reference sample, 7,176 individuals were randomly selected. Of these, 4,019 completed interviews, resulting in a response rate of 56\%. \textbf{[JJH: Atrocious -- Do we ever correct?][We do not correct because we have very limited information on those who refused.]} Table~\ref{tab:sample-response} provides an overview of the birth years for the different cohorts, the counts of the full sample, and the response rate. The most common reasons for non-response were that nobody was home when the surveying agency solicited and sharp refusals. 

\begin{table}[H]
\centering
\begin{threeparttable}
	\caption{Description of the Full Sample and Response Rates}\label{tab:sample-response}
	\begin{tabular}{l c c c c c c}
\toprule
Cohort & Birth year(s) & Age at interview & Reggio Emilia & Parma & Padova & \textbf{Total} \\			
\midrule
\textbf{Children} &  &  &  & &  &  \\ 
\quad Italians & 2006 & 7 & 311 & 291& 278 & 880 \\
			&&	& \textit{50.0\%} &  \textit{62.7\%} &  \textit{50.0\%} &  \textit{53.6\%} \\
\quad Migrants & 2006 & 7 & 110 & 58 & 113 & 281 \\
			&& 	&  \textit{53.1\%} &  \textit{49.2\%} &  \textit{63.1\%} &  \textit{55.8\%} \\
\textbf{Adolescents} & 1994 & 19 & 300 & 254 & 282 & 836 \\
			&& 	&  \textit{57.1\%} &  \textit{58.5\%} &  \textit{55.5\%} &  \textit{57.0\%} \\
\textbf{Adults 30s} & 1980-1981 & 32 & 280 & 251 & 251 & 782 \\
			&& 	&  \textit{58.3\%} &  \textit{58.2\%} &  \textit{57.4\%} &  \textit{57.9\%} \\
\textbf{Adults 40s} & 1969-1970 & 43 & 285 & 254 & 252 & 791 \\
			&& 	&  \textit{59.3\%} &  \textit{56.3\%} &  \textit{53.8\%} &  \textit{56.0\%}\\
\textbf{Adults 50s} & 1954-1959 & 54-60 & 200 & 103 & 146 & 449 \\
			&& 	&  \textit{52.2\%} &  \textit{63.6\%} &  \textit{55.6\%}  &  \textit{55.6\%}\\
\midrule
\textbf{Total}	& 				& & 1,486 & 1,211 & 1,322 & 4,019 \\
			&&				& \textit{55.1\%} &  \textit{58.8\%} &  \textit{55.0\%} & \textit{56.0\%} \\
\bottomrule
\end{tabular}

\begin{tablenotes}
\footnotesize
Note: The response rates for each city and cohort are in italics. They are the the ratio of interviews to total valid contacts. Valid contacts are the sum of: completed interviews, sharp refusal, no person present, talked with a relative, left paper questionnaire but never returned, interview began but not completed. The age at interview is an approximation given there is some variation in the interview date and birth year within each cohort. In analysis, we combine the Italian and migrant subsamples of the child cohort and control for migrant status.
\end{tablenotes}
\end{threeparttable}
\end{table}

\textbf{[JJH: What are the response rates by location?]}\textbf{[Table~\ref{tab:sample-response} now includes this.]}

Table~\ref{tab:sample} provides a detailed tabulation of the sample by city, cohort, and school type. It shows that the number of people who do not attend any preschool decreases over time. Whereas the majority of individuals from the age-50 cohort did not attend any preschool, there are few such cases in the child and adolescent cohorts. Table \ref{tab:sample} also shows that the proportion of individual attending municipal preschools is higher in Reggio Emilia than in the other cities.\footnote{This is due to the construction of the sample.} Note that the Reggio Approach preschools were not available for the age-50 cohort.

\begin{table}[H]
\centering
\scalebox{0.85}{
\begin{threeparttable}
	\caption{Tabulation of Preschool Attendence by Cohort, City, and School Type}\label{tab:sample}
	\begin{tabular}{l*{8}{c}}
\toprule
	&	\mc{7}{c}{Reggio Emilia: 1,486}													\\	\midrule
	&	None	&	Muni	&	State	&	Reli	&	Priv	&	Muni-Affi	&	Other	\\	\midrule
Children	&	2	&	159	&	44	&	92	&	5	&	7	&	1	\\	
Migrants	&	4	&	47	&	38	&	14	&	1	&	3	&	1	\\	
Adolescents	&	7	&	151	&	22	&	98	&	6	&	13	&	0	\\	
Adults 30s	&	57	&	138	&	31	&	40	&	1	&	4	&	8	\\	
Adults 40s	&	80	&	87	&	14	&	52	&	5	&	1	&	43	\\	
Adults 50s	&	147	&	0	&	0	&	29	&	2	&	0	&	20	\\	\midrule
	&	\mc{7}{c}{ Parma: 1,211}													\\	\midrule
	&	None	&	Muni	&	State	&	Reli	&	Priv	&	Muni-Affi	&	Other	\\	\midrule
Children	&	5	&	105	&	42	&	74	&	8	&	52	&	0	\\	
Migrants	&	4	&	25	&	12	&	3	&	6	&	7	&	0	\\	
Adolescents	&	4	&	100	&	52	&	77	&	6	&	5	&	2	\\	
Adults 30s	&	44	&	85	&	56	&	51	&	5	&	4	&	3	\\	
Adults 40s	&	116	&	0	&	0	&	55	&	1	&	4	&	73	\\	
Adults 50s	&	72	&	0	&	0	&	11	&	0	&	10	&	9	\\	\midrule
	&	\mc{7}{c}{Padova: 1,322}													\\	\midrule
	&	None	&	Muni	&	State	&	Reli	&	Priv	&	Muni-Affi	&	Other	\\	\midrule
Children	&	2	&	58	&	45	&	141	&	12	&	19	&	0	\\	
Migrants	&	5	&	33	&	46	&	23	&	1	&	0	&	4	\\	
Adolescents	&	1	&	84	&	46	&	132	&	6	&	7	&	2	\\	
Adults 30s	&	47	&	27	&	27	&	140	&	1	&	7	&	0	\\	
Adults 40s	&	75	&	0	&	0	&	126	&	0	&	10	&	39	\\	
Adults 50s	&	57	&	0	&	0	&	72	&	2	&	6	&	3	\\	

\bottomrule
\end{tabular}


\begin{tablenotes}
Note: This table shows the sample size by city, cohort, and school type. We separate migrants and children for clarity in this table even though they are in the same birth cohort (year of birth: 2006). None: no preschool; Muni.: municipal preschool;  State: state preschool; Relig.: religious preschool; Priv.: private preschool. Muni-Affi: municipal-affiliated preschool; Other: uncategorized preschool.
\end{tablenotes}
\end{threeparttable}
}
\end{table}

The structure of the cohorts allows us to study the effects of the Reggio Approach at different points throughout the life cycle. The children in the youngest cohort were interviewed when they entered primary school, the adolescent cohort when they ended compulsory schooling, and the adult cohorts capture different points of adulthood to measure key outcomes such as engagement in the labor market, health, and family decisions. Although this cohort structure allows us to study the evolution of the program, the other preschools also evolved making it challenging to compare the Reggio Approach to a consistent group. Our investigation in Section~\ref{sec:ece-italy} of the early childhood education landscape helps characterize the comparison group over time. \textbf{[JJH: But utterly screws up quality -- constant comparisons.]}\textbf{[Sentence qualified to acknowledge challenge.]}

Restricting the sample to individuals living in the same city in which they were raised is necessary in order to compare individuals who had the \textit{opportunity} to attend the different types of preschool. Due to the the municipal structure of the population registries, it is not feasible to sample all individuals born in Reggio Emilia, Parma or Padova and now living in other cities or abroad. If the Reggio Approach has a treatment effect on migration out of Reggio Emilia, then our sample might oversample individuals with characteristics that are not predictive of emigration. Table~\ref{tab:immigration} presents the proportion of people who were born in Italy, of Italian citizenship, and then still resident in that town of birth, out of the total number of names given by the population registries. For all cohorts, the immigration rates are very similar for all three cities. This reinforces our choice of treatment and control cities, which share a similar economic and labor market history. Nonetheless, it is worth noting that embedded in our sample selection is the potential bias due to the fact that one of the effects of preschool might be a higher propensity to emigrate. It may be that higher skilled individuals migrate. This would result in an under-estimation of treatment effects. \textbf{[JJH: and the better people leave -- what is the out-migration rate of all Italian?][We double checked that it is not possible to find this information unless we look for the migration rate out of Italy to other countries. The information of migration out of each city is not readily available for other cities.]}

\begin{table}[H]
\centering
\begin{threeparttable}
	\caption{Percentage of People Living in the Same City Since Birth}\label{tab:immigration}
	\begin{table}[ht!]
\caption{\textbf{Percentage of people living in the same city since birth, by cohort}}
\label{tab:SameCity}
\vspace{-5mm}
\begin{center}
\begin{tabular}{ l c c c c }
\hline\hline
\textbf{Cohort} & \textbf{Reggio (\%)} & \textbf{Parma (\%)} & \textbf{Padova (\%)} & \textbf{Total (\%)}\\
\hline
Italian Children born in 2006 (Cohort V)   & 61.3  & 70.2  & 65.1  & 65.2 \\[0.2em]
Adolescents born in 1994 (Cohort IV)       & 58.1  & 63.0  & 64.4  & 61.9 \\[0.2em]
Adults born in 1980-81 (Cohort III)        & 26.5  & 27.5  & 32.6  & 29.0 \\[0.2em]
Adults born in 1969-70 (Cohort II)         & 27.9  & 31.6  & 31.9  & 30.6 \\[0.2em]
Adults born in 1954-59 (Cohort I)          & 28.8  & 27.9  & 31.4  & 29.5 \\[0.2em]
\hline
\textit{Total}         & \textit{32.3\%}  & \textit{32.5\%}  & \textit{35.2\%} & \textit{33.5\%} \\
\hline
\end{tabular}
\end{center}
\footnotesize{{\bfseries Notes:} Reference sample who satisfied the selection criteria (born in the city of residence and of Italian citizenship) as a percentage of the total number of names given by the population registries, broken down by City and Cohort. Source: authors calculations on data provided by the population registries.}
\end{table}

\begin{tablenotes}
\footnotesize
Note: This table presents the percentage of people living in same city since birth. This  shows the reference sample who satified the selection criteria (born in the city of residence and of Italian citizenship) as a percentage of the total number of names given by the population registries.
\end{tablenotes}
\end{threeparttable}
\end{table}

%\subsection{The Questionnaire Design}

In order to allow us an evaluation of the potential effect of the Reggio Approach on a broad set of domains, we designed a questionnaire surveying various outcomes and dimensions of life success. Respondents were asked about family composition, fertility, labor force participation, income, schooling, cognitive ability, social and emotional skills, health and healthy habits, social capital, interpersonal ties, as well as attitudes on migrants. Three age-specific questionnaires were designed, piloted, and fielded: one for the Italian and immigrant child cohorts, one for the adolescent cohort, and one for the adult cohorts.\footnote{By means of a first pilot in the city of Bergamo with a sample from every cohort, and a second pilot in Reggio Emilia, Parma, and Padova on a subsample of adults, the questionnaires were tested and refined to the final version, which lasts approximately 40 minutes for the adults, and 1 hour for the children and the adolescents.} \textbf{[JJH: This is very non-specific. We need to distinguish: (a) Diffusion, (b) Common founder effects.]}