\subsection{The Selection of Cities}
%Section \ref{sec:eceexperiences} shows that the childcare systems within and outside Reggio Emilia share the Reggio Approach components to varying degrees. Because other childcare options may be similar to the Reggio Approach to some extent and may have their own high-quality components, it is difficult to evaluate the clean effects of the Reggio Approach.
%Our survey data collection is designed to evaluate the impact of the Reggio Approach on life-time social and economic outcomes comparing it to other early childhood programs that might share similar components. Note that our comparison is not to determine whether the Reggio Approach is of higher or lower quality relative to other programs, but rather to understand how these programs that vary across school types, cities, and over time might have differential impact of life-time outcomes.

We collect data on five cohorts of individuals born and raised in Reggio Emilia, Parma, and Padova. Parma and Padova have been chosen in addition to Reggio Emilia to represent appropriate controls: they are close enough to Reggio Emilia in terms of size, geographic, demographic, and socio-economic structure, but they did not experience its unique approach to early childhood education.\footnote{Other Italian cities were taken into consideration, notably Brescia, Livorno, Modena, Perugia, Piacenza, Prato, and Ravenna. Parma and Padova were the two cities that best fulfilled our comparability and sample requirements.} Parma is a neighboring city 30 km to the west of Reggio Emilia, it belongs to the same administrative region of Emilia-Romagna, and shares with Reggio a common background that forged the city's political and economic system, and shaped its culture and social capital; yet Parma has witnessed a different approach to public investment in early childhood education \textbf{[JJH: How different? We need a table -- the previous discussion was not all that different.]}. On the contrary, Padova lies 150 km away from Reggio Emilia, in the richer north-eastern region of Veneto, and was more influenced by the history and culture of Venice. Yet, Padova, just as Reggio Emilia, is a middle-size Italian city with a thriving \textbf{[JJH: Evidence?]} migrant population, facing similar political and economic issues, with a different social background but similar resources which can be devoted to early childhood education.

We collect demographic data to explore the similarities among the three cities. Reggio Emilia has a population of roughly 173,000, Parma of 188,000, and Padova of 210,000 in 2013.\footnote{The population size in December 2013 was 172,525 in Reggio Emilia, 187,938 in Parma, and 209,678 in Padova. Source: ISTAT, \url{http://www.demo.istat.it/}} Appendix Figure \ref{fig:population} shows the similar rising trends in population in three cities. Also their economic resources are comparable: Reggio Emilia has an average per-capita income of 25,226 euros, Parma of 28,437, and Padova of 29,915 in 2011.\footnote{Source: Finance Minister, taxable income for 2011.}
%See \url{http://www.comuni-italiani.it/statistiche/redditir2011.html}
Even more importantly, these three cities face analogous fertility dynamics and, consequently, potential demand for child-care services. Figure \ref{fig:population} shows that three cities observe similar trend in birth rate and natural rate. Trends in migration are also similar in three cities, which is shown in Appendix Figure \ref{fig:emigr-immigr}. Although emigration rate is highest in Padova and net migration rate is highest in Reggio for most of the years, general trends in emigration and immigration in similar in all cities. Especially trends in foreign migration are almost identical in three cities.

%Figure \ref{fig:enrollment} explores the similarities and difference in enrollment statistics in three cities. Note that enrollment data is not available for Parma except for the year 2010. It is shown that Padova has the highest rate of attending preschool. It is also shown that Reggio Emilia and Padova observed the increase in preschool enrollment; from the 1990s, almost all children of age 3-5 are enrolled in preschool. The figures also show that the percentage of enrollment in municipal preschools increased during the 1970s and the 1980s in Reggio Emilia and Padova, but decreased after the 1990s in Reggio Emilia. The percentage enrolled in state preschools increased over time in both Reggio Emilia and Padova. In contrast, the percentage enrolled in private preschools decreased over time in Reggio Emilia and Padova. For Parma, the enrollment statistics in 2010 are similar to those of Reggio Emilia.
