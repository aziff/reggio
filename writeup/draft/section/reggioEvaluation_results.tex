We present the estimates of the methods described above for a handful of key outcomes. See Appendix~\ref{sec:results} for estimates of more outcomes. In the child cohort, the Reggio Approach increased SDQ scores when compared to children who attended other preschools (OLS). This result gets more positive after controlling for more background characteristics. These results do not persist when comparing with children in Parma and Padova (difference-in-difference). Using AIPW to address issues of selection provides similar results as those of the OLS specification. The children are slightly more likely to be overweight and have worse health, although these estimates are not significant for all specifications. The results compared to no preschools are difficult to interpret for the child and adolescent cohorts considering few individuals in these cohorts stayed at home for preschool.

The effect of the Reggio Approach in the adolescent cohort is similar to that of the children cohort in that the subjects are more likely to be obese. This maintains across OLS specifications and the AIPW specification. Socially positive outcomes that are consistent include going to school, that is not dropping out of school, and decreased depression. However, the adolescents liked school less, especially when compared to their peers in Parma.

Adults in the age-40 cohort similarly showed lower levels of depression. For both adult cohorts, individuals worked more. This is one of the more consistent estimates, given it holds significance when comparing to those who attended no preschool and across both adult cohorts listed above. The adults are more likely to be employed, which is tied to this outcome of hours working. 

% ========================================================================= %
% CHILD COHORT

\begin{landscape}

\begin{table}[H] \caption{Estimation Results for Main Outcomes, Comparison to Preschools, Child Cohort} \label{ols-M-child-reg-pres}
\scalebox{0.8}{\begin{tabular}{l c c c c c c}
\toprule
 & None & BIC & Full & AIPW & DidPm & DidPv \\
\midrule
IQ Score &     -0.02 &     -0.03 &     -0.03 &     -0.02 &      0.03 &     -0.01 \\
& (     0.02 ) & (     0.02 ) & (     0.02 ) & (     0.02 ) & (     0.04 ) & (     0.03 ) \\
& \textit{ 410 } & \textit{ 410 } & \textit{ 410 } & \textit{ 410 } & \textit{ 762 } & \textit{ 790 } \\
IQ Factor &     -0.07 &     -0.13 &     -0.12 &     -0.12 &      0.14 &      0.03 \\
& (     0.10 ) & (     0.10 ) & (     0.10 ) & (     0.10 ) & (     0.17 ) & (     0.15 ) \\
& \textit{ 410 } & \textit{ 410 } & \textit{ 410 } & \textit{ 410 } & \textit{ 762 } & \textit{ 790 } \\
SDQ Composite - Child & \textbf{      0.70 } & \textbf{      0.96 } & \textbf{      1.21 } & \textbf{     0.93} &      0.10 & \textbf{      1.18 } \\
& (     0.47 ) & (     0.46 ) & (     0.45 ) & (     0.46 ) & (     0.84 ) & (     0.75 ) \\
& \textit{ 409 } & \textit{ 409 } & \textit{ 409 } & \textit{ 409 } & \textit{ 761 } & \textit{ 789 } \\
Obese &     -0.00 &      0.04 &      0.03 &      0.05 &      0.03 &     -0.05 \\
& (     0.05 ) & (     0.05 ) & (     0.05 ) & (     0.04 ) & (     0.08 ) & (     0.07 ) \\
& \textit{ 410 } & \textit{ 410 } & \textit{ 410 } & \textit{ 410 } & \textit{ 762 } & \textit{ 790 } \\
Overweight & \textbf{      0.05 } &      0.03 &      0.05 &      0.03 &     -0.06 & \textbf{      0.09 } \\
& (     0.03 ) & (     0.03 ) & (     0.04 ) & (     0.03 ) & (     0.07 ) & (     0.05 ) \\
& \textit{ 410 } & \textit{ 410 } & \textit{ 410 } & \textit{ 410 } & \textit{ 762 } & \textit{ 790 } \\
Health is Good &     -0.05 &     -0.03 &     -0.01 &     -0.02 &      0.08 &     -0.05 \\
& (     0.05 ) & (     0.05 ) & (     0.05 ) & (     0.05 ) & (     0.09 ) & (     0.07 ) \\
& \textit{ 409 } & \textit{ 409 } & \textit{ 409 } & \textit{ 409 } & \textit{ 761 } & \textit{ 788 } \\
Not Excited to Learn &     -0.00 &      0.00 &     -0.00 &     -0.00 &      0.03 &     -0.03 \\
& (     0.02 ) & (     0.02 ) & (     0.02 ) & (     0.02 ) & (     0.04 ) & (     0.04 ) \\
& \textit{ 410 } & \textit{ 410 } & \textit{ 410 } & \textit{ 410 } & \textit{ 762 } & \textit{ 790 } \\
Problems Sitting Still &      0.02 &      0.03 &      0.00 &      0.02 &     -0.08 &     -0.04 \\
& (     0.03 ) & (     0.03 ) & (     0.03 ) & (     0.03 ) & (     0.07 ) & (     0.05 ) \\
& \textit{ 410 } & \textit{ 410 } & \textit{ 410 } & \textit{ 410 } & \textit{ 762 } & \textit{ 790 } \\
How Much Child Likes School &      0.02 &      0.00 &      0.04 &     -0.01 &      0.07 &      0.10 \\
& (     0.06 ) & (     0.06 ) & (     0.06 ) & (     0.06 ) & (     0.10 ) & (     0.09 ) \\
& \textit{ 408 } & \textit{ 408 } & \textit{ 408 } & \textit{ 408 } & \textit{ 758 } & \textit{ 788 } \\
\bottomrule
\end{tabular}
}
\vspace{1ex} \\
\footnotesize\raggedright{Note: This table shows the estimates of the coefficient for attending Reggio Approach preschools from multiple methods. We compare Reggio Approach people with people who attended other preschools. Column title indicates the corresponding control set and and model. ``NoneIt'' refers to the OLS estimate with no control variables. ``BICIt'' refers to the OLS estimate with controls selected by Bayesian Information Criterion (BIC) and additional controls for caregiver's religion. ``FullIt'' refers to the OLS estimate with the full set of controls. ``DidPmIt'' refers to the difference-in-difference estimate of (Reggio Muni - Parma Muni) - (Reggio None - Parma None). ``DidPvIt'' refers to the difference-in-difference estimate of (Reggio Muni - Padova Muni) - (Reggio None - Padova None). ``AIPWIt" refers to AIPW estimate for comparing Reggio Approach children with children in Reggio who attended other type of preschool. Robust standard errors are reported in parentheses. Bold number shows that the estimate is statistically significant at the 10\% level. Number of observations used in estimation is reported in italic.}

\end{table}


\begin{table}[H] \caption{Estimation Results for Main Outcomes, Comparison to No Preschools, Child Cohort} \label{ols-M-child-reg-nopres}
\scalebox{0.8}{\begin{tabular}{l c c c c c c}
\toprule
 & None & BIC & Full & AIPW & DidPm & DidPv \\
\midrule
IQ Score &     -0.09 & \textbf{     -0.11 } &     -0.14 &      0.05 &      0.03 & \textbf{     -0.30 } \\
& (     0.06 ) & (     0.07 ) & (     0.10 ) & (     0.12 ) & (     0.04 ) & (     0.08 ) \\
& \textit{ 216 } & \textit{ 216 } & \textit{ 216 } & \textit{ 216 } & \textit{ 762 } & \textit{ 321 } \\
IQ Factor &     -0.26 &     -0.41 &     -0.48 &     -0.11 &      0.14 & \textbf{     -1.17 } \\
& (     0.31 ) & (     0.37 ) & (     0.47 ) & (     0.44 ) & (     0.17 ) & (     0.43 ) \\
& \textit{ 216 } & \textit{ 216 } & \textit{ 216 } & \textit{ 216 } & \textit{ 762 } & \textit{ 321 } \\
SDQ Composite - Child &     -0.36 &     -0.52 &     -0.55 &      2.59 &      0.10 & \textbf{     -3.35 } \\
& (     1.74 ) & (     1.79 ) & (     1.58 ) & (     2.35 ) & (     0.84 ) & (     2.12 ) \\
& \textit{ 216 } & \textit{ 216 } & \textit{ 216 } & \textit{ 216 } & \textit{ 761 } & \textit{ 321 } \\
Obese &      0.17 & \textbf{      0.27 } & \textbf{      0.32 } & \textbf{     0.75} &      0.03 &     -0.04 \\
& (     0.16 ) & (     0.17 ) & (     0.19 ) & (     0.20 ) & (     0.08 ) & (     0.21 ) \\
& \textit{ 216 } & \textit{ 216 } & \textit{ 216 } & \textit{ 216 } & \textit{ 762 } & \textit{ 321 } \\
Overweight &     -0.02 &     -0.01 &     -0.09 &      0.07 &     -0.06 &      0.04 \\
& (     0.15 ) & (     0.17 ) & (     0.20 ) & (     0.20 ) & (     0.07 ) & (     0.21 ) \\
& \textit{ 216 } & \textit{ 216 } & \textit{ 216 } & \textit{ 216 } & \textit{ 762 } & \textit{ 321 } \\
Health is Good & \textbf{      0.51 } & \textbf{      0.58 } & \textbf{      0.73 } & \textbf{     0.75} &      0.08 & \textbf{      0.74 } \\
& (     0.16 ) & (     0.16 ) & (     0.15 ) & (     0.28 ) & (     0.09 ) & (     0.16 ) \\
& \textit{ 216 } & \textit{ 216 } & \textit{ 216 } & \textit{ 216 } & \textit{ 761 } & \textit{ 321 } \\
Not Excited to Learn &     -0.12 &     -0.09 &     -0.10 &     -0.08 &      0.03 &      0.24 \\
& (     0.15 ) & (     0.15 ) & (     0.14 ) & (     0.16 ) & (     0.04 ) & (     0.24 ) \\
& \textit{ 216 } & \textit{ 216 } & \textit{ 216 } & \textit{ 216 } & \textit{ 762 } & \textit{ 321 } \\
Problems Sitting Still & \textbf{      0.14 } & \textbf{      0.24 } & \textbf{      0.25 } & \textbf{     0.14} &     -0.08 &      0.03 \\
& (     0.02 ) & (     0.07 ) & (     0.07 ) & (     0.06 ) & (     0.07 ) & (     0.06 ) \\
& \textit{ 216 } & \textit{ 216 } & \textit{ 216 } & \textit{ 216 } & \textit{ 762 } & \textit{ 321 } \\
How Much Child Likes School &      0.14 &     -0.06 &      0.04 & \textbf{     0.55} &      0.07 &     -0.19 \\
& (     0.32 ) & (     0.31 ) & (     0.26 ) & (     0.34 ) & (     0.10 ) & (     0.48 ) \\
& \textit{ 216 } & \textit{ 216 } & \textit{ 216 } & \textit{ 216 } & \textit{ 758 } & \textit{ 321 } \\
\bottomrule
\end{tabular}
}
\vspace{1ex} \\
\footnotesize\raggedright{Note: This table shows the estimates of the coefficient for attending Reggio Approach preschools from multiple methods. We compare Reggio Approach people with people who attended no preschool. Column title indicates the corresponding control set and and model. ``NoneIt'' refers to the OLS estimate with no control variables. ``BICIt'' refers to the OLS estimate with controls selected by Bayesian Information Criterion (BIC) and additional controls for caregiver's religion. ``FullIt'' refers to the OLS estimate with the full set of controls. ``DidPmIt'' refers to the difference-in-difference estimate of (Reggio Muni - Parma Muni) - (Reggio None - Parma None). ``DidPvIt'' refers to the difference-in-difference estimate of (Reggio Muni - Padova Muni) - (Reggio None - Padova None). ``AIPWIt" refers to AIPW estimate for comparing Reggio Approach children with children in Reggio who did not attend any preschool. Robust standard errors are reported in parentheses. Bold number shows that the estimate is statistically significant at the 10\% level. Number of observations used in estimation is reported in italic.}

\end{table}


\begin{table}[H] \caption{Estimation Results for Main Outcomes, Comparison to Preschools, Adolescent Cohort} \label{ols-M-adol-reg-pres}
\scalebox{0.8}{\begin{tabular}{l c c c c c c}
\toprule
 & None & Bic & Full & AIPW & DidPm & DidPv \\
\midrule
IQ Score &     -0.03 &     -0.04 &      0.00 &     -0.03 &     -0.04 &     -0.05 \\
& (     0.03 ) & (     0.03 ) & (     0.03 ) & (     0.04 ) & (     0.05 ) & (     0.05 ) \\
& \textit{ 286 } & \textit{ 286 } & \textit{ 286 } & \textit{ 286 } & \textit{ 530 } & \textit{ 561 } \\
IQ Factor &     -0.12 & \textbf{     -0.17 } &     -0.02 &     -0.13 &     -0.06 &     -0.20 \\
& (     0.10 ) & (     0.11 ) & (     0.11 ) & (     0.12 ) & (     0.17 ) & (     0.17 ) \\
& \textit{ 286 } & \textit{ 286 } & \textit{ 286 } & \textit{ 286 } & \textit{ 530 } & \textit{ 561 } \\
SDQ Composite - Child &     -0.03 &     -0.16 &      0.31 &      0.21 &      0.50 &     -0.60 \\
& (     0.59 ) & (     0.70 ) & (     0.62 ) & (     0.70 ) & (     1.09 ) & (     0.80 ) \\
& \textit{ 286 } & \textit{ 286 } & \textit{ 286 } & \textit{ 286 } & \textit{ 530 } & \textit{ 556 } \\
SDQ Composite &      0.89 &      0.90 &      0.69 &      0.45 & \textbf{      1.62 } &      0.68 \\
& (     0.63 ) & (     0.71 ) & (     0.72 ) & (     0.77 ) & (     1.06 ) & (     0.94 ) \\
& \textit{ 284 } & \textit{ 284 } & \textit{ 284 } & \textit{ 284 } & \textit{ 526 } & \textit{ 557 } \\
Depression Score - positive & \textbf{      1.38 } & \textbf{      1.75 } & \textbf{      1.71 } &      1.24 & \textbf{      3.14 } & \textbf{      1.86 } \\
& (     0.78 ) & (     0.88 ) & (     0.91 ) & (     0.93 ) & (     1.18 ) & (     1.14 ) \\
& \textit{ 279 } & \textit{ 279 } & \textit{ 279 } & \textit{ 279 } & \textit{ 512 } & \textit{ 550 } \\
Locus of Control - positive &      0.03 &      0.05 &      0.04 &     -0.01 & \textbf{     -0.25 } &      0.11 \\
& (     0.09 ) & (     0.10 ) & (     0.09 ) & (     0.10 ) & (     0.15 ) & (     0.13 ) \\
& \textit{ 283 } & \textit{ 283 } & \textit{ 283 } & \textit{ 283 } & \textit{ 522 } & \textit{ 556 } \\
Obese & \textbf{      0.08 } & \textbf{      0.11 } & \textbf{      0.09 } & \textbf{     0.10} &      0.00 &      0.09 \\
& (     0.04 ) & (     0.05 ) & (     0.04 ) & (     0.04 ) & (     0.07 ) & (     0.07 ) \\
& \textit{ 286 } & \textit{ 286 } & \textit{ 286 } & \textit{ 286 } & \textit{ 530 } & \textit{ 561 } \\
Overweight &     -0.01 &      0.00 &      0.00 &      0.01 &     -0.08 &      0.02 \\
& (     0.02 ) & (     0.03 ) & (     0.02 ) & (     0.03 ) & (     0.06 ) & (     0.03 ) \\
& \textit{ 286 } & \textit{ 286 } & \textit{ 286 } & \textit{ 286 } & \textit{ 530 } & \textit{ 561 } \\
Health is Good &      0.06 &      0.07 & \textbf{      0.09 } &      0.06 &      0.11 & \textbf{      0.13 } \\
& (     0.06 ) & (     0.06 ) & (     0.06 ) & (     0.07 ) & (     0.10 ) & (     0.09 ) \\
& \textit{ 285 } & \textit{ 285 } & \textit{ 285 } & \textit{ 285 } & \textit{ 529 } & \textit{ 560 } \\
Go To School &      0.03 &      0.01 &      0.03 &      0.01 &     -0.00 &      0.04 \\
& (     0.02 ) & (     0.02 ) & (     0.03 ) & (     0.02 ) & (     0.03 ) & (     0.03 ) \\
& \textit{ 286 } & \textit{ 286 } & \textit{ 286 } & \textit{ 286 } & \textit{ 530 } & \textit{ 561 } \\
How Much Child Likes School &     -0.12 &     -0.11 & \textbf{     -0.18 } &     -0.00 &      0.08 &     -0.12 \\
& (     0.11 ) & (     0.13 ) & (     0.12 ) & (     0.14 ) & (     0.19 ) & (     0.16 ) \\
& \textit{ 273 } & \textit{ 273 } & \textit{ 273 } & \textit{ 273 } & \textit{ 508 } & \textit{ 543 } \\
Days of Sport (Weekly) & \textbf{     -0.41 } & \textbf{     -0.43 } &     -0.29 &     -0.46 & \textbf{     -0.84 } &     -0.48 \\
& (     0.23 ) & (     0.26 ) & (     0.26 ) & (     0.26 ) & (     0.38 ) & (     0.34 ) \\
& \textit{ 280 } & \textit{ 280 } & \textit{ 280 } & \textit{ 280 } & \textit{ 516 } & \textit{ 536 } \\
\bottomrule
\end{tabular}
}
\vspace{1ex} \\
\footnotesize\raggedright{Note: This table shows the estimates of the coefficient for attending Reggio Approach preschools from multiple methods. We compare Reggio Approach people with people who attended other preschools. Column title indicates the corresponding control set and and model. ``None'' refers to the OLS estimate with no control variables. ``BIC'' refers to the OLS estimate with controls selected by Bayesian Information Criterion (BIC) and additional controls for caregiver's religion. ``Full'' refers to the OLS estimate with the full set of controls. ``DidPm'' refers to the difference-in-difference estimate of (Reggio Muni - Parma Muni) - (Reggio None - Parma None). ``DidPv'' refers to the difference-in-difference estimate of (Reggio Muni - Padova Muni) - (Reggio None - Padova None). ``AIPW" refers to AIPW estimate for comparing Reggio Approach children with children in Reggio who attended other type of preschool. Robust standard errors are reported in parentheses. Bold number shows that the estimate is statistically significant at the 10\% level. Number of observations used in estimation is reported in italic.}
\end{table}

\begin{table}[H] \caption{Estimation Results for Main Outcomes, Comparison to No Preschools, Adolescent Cohort} \label{ols-M-adol-reg-nopres}
\scalebox{0.8}{\begin{tabular}{l c c c c c c}
\toprule
 & None & BIC & Full & DidPm & DidPv & AIPW \\
\midrule
SDQ Composite - Child &      0.62 &      1.61 &      2.93 &      0.77 &      1.24 &     -0.17 \\
& (     1.61) & (     1.85) & (     2.47) & (     1.68) & (     1.92) & (     1.68) \\
SDQ Composite & \textbf{     -2.57 } &     -1.42 &     -0.91 & \textbf{     -2.36 } &     -2.67 &     -1.11 \\
& (     1.73) & (     2.06) & (     2.01) & (     1.60) & (     2.42) & (     2.66) \\
Depression Score - positive & \textbf{     -3.25 } & \textbf{     -2.26 } & \textbf{     -3.81 } & \textbf{     -3.23 } &     -1.85 &     -1.31 \\
& (     1.49) & (     1.13) & (     1.49) & (     1.71) & (     2.81) & (     1.75) \\
Obese &     -0.10 &     -0.08 &     -0.32 &     -0.16 & \textbf{     -0.37 } &     -0.14 \\
& (     0.17) & (     0.17) & (     0.23) & (     0.13) & (     0.16) & (     0.23) \\
Overweight &     -0.11 &     -0.11 &     -0.17 &     -0.14 &     -0.11 &     -0.11 \\
& (     0.13) & (     0.12) & (     0.16) & (     0.12) & (     0.10) & (     0.16) \\
Health is Good &     -0.02 &      0.04 & \textbf{     -0.30 } &     -0.10 & \textbf{      0.41 } &      0.19 \\
& (     0.18) & (     0.18) & (     0.12) & (     0.16) & (     0.14) & (     0.28) \\
Go To School & \textbf{      0.26 } & \textbf{      0.23 } & \textbf{      0.33 } & \textbf{      0.13 } & \textbf{      0.21 } &      0.09 \\
& (     0.17) & (     0.14) & (     0.14) & (     0.08) & (     0.12) & (     0.19) \\
How Much Child Likes School &     -0.29 &     -0.18 &     -0.18 & \textbf{     -0.58 } &     -0.44 &      0.56 \\
& (     0.41) & (     0.41) & (     0.38) & (     0.29) & (     0.53) & (     0.63) \\
Trust Score &     -0.33 &     -0.48 & \textbf{     -1.11 } &     -0.68 &     -0.57 &      0.28 \\
& (     0.50) & (     0.48) & (     0.48) & (     0.52) & (     0.66) & (     1.16) \\
Days of Sport (Weekly) &     -0.21 &     -0.03 &     -0.45 & \textbf{      1.05 } &      1.49 &      0.31 \\
& (     0.84) & (     0.84) & (     1.03) & (     0.70) & (     1.09) & (     0.99) \\
\bottomrule
\end{tabular}
}
\vspace{1ex} \\
\footnotesize\raggedright{Note: This table shows the estimates of the coefficient for attending Reggio Approach preschools from multiple methods. We compare Reggio Approach people with people who attended no preschool. Column title indicates the corresponding control set and and model. ``None'' refers to the OLS estimate with no control variables. ``BIC'' refers to the OLS estimate with controls selected by Bayesian Information Criterion (BIC) and additional controls for caregiver's religion. ``Full'' refers to the OLS estimate with the full set of controls. ``DidPm'' refers to the difference-in-difference estimate of (Reggio Muni - Parma Muni) - (Reggio None - Parma None). ``DidPv'' refers to the difference-in-difference estimate of (Reggio Muni - Padova Muni) - (Reggio None - Padova None).  ``AIPW" refers to AIPW estimate for comparing Reggio Approach children with children in Reggio who did not attend any preschool. Robust standard errors are reported in parentheses. Bold number shows that the estimate is statistically significant at the 10\% level. Number of observations used in estimation is reported in italic.}
\end{table}




\begin{table}[H] \caption{Estimation Results for Main Outcomes, Comparison to Preschools, Adult Cohorts} \label{ols-M-adult-reg-pres}
\scalebox{0.75}{\begin{tabular}{l c c c c c c c c c c}
\toprule
 & None30 & BIC30 & Full30 & DidPm30 & DidPv30 & AIPW30 & None40 & BIC40 & Full40 & AIPW40 \\
\midrule
Graduate from High School &     -0.04 &     -0.03 &     -0.05 & \textbf{     -0.17 } &      0.08 &     -0.06 & \textbf{      0.13 } &      0.09 & \textbf{      0.11 } &      0.01 \\
& (     0.05 ) & (     0.05 ) & (     0.05 ) & (     0.06 ) & (     0.06 ) & (     0.05 ) & (     0.07 ) & (     0.07 ) & (     0.07 ) & (     0.06 ) \\
& \textit{ 153 } & \textit{ 153 } & \textit{ 153 } & \textit{ 299 } & \textit{ 326 } & \textit{ 153 } & \textit{ 161 } & \textit{ 161 } & \textit{ 161 } & \textit{ 161 } \\
Max Edu: University &      0.04 &      0.04 &      0.03 & \textbf{     -0.18 } &     -0.17 &      0.02 &      0.08 &      0.05 &      0.03 &      0.05 \\
& (     0.07 ) & (     0.07 ) & (     0.07 ) & (     0.08 ) & (     0.12 ) & (     0.07 ) & (     0.06 ) & (     0.05 ) & (     0.05 ) & (     0.06 ) \\
& \textit{ 153 } & \textit{ 153 } & \textit{ 153 } & \textit{ 299 } & \textit{ 326 } & \textit{ 153 } & \textit{ 161 } & \textit{ 161 } & \textit{ 161 } & \textit{ 161 } \\
Employed &     -0.02 &     -0.02 &     -0.01 &     -0.03 &      0.02 &     -0.02 &      0.01 &      0.01 &      0.01 &      0.02 \\
& (     0.03 ) & (     0.04 ) & (     0.04 ) & (     0.06 ) & (     0.08 ) & (     0.03 ) & (     0.03 ) & (     0.03 ) & (     0.04 ) & (     0.03 ) \\
& \textit{ 153 } & \textit{ 153 } & \textit{ 153 } & \textit{ 299 } & \textit{ 326 } & \textit{ 153 } & \textit{ 161 } & \textit{ 161 } & \textit{ 161 } & \textit{ 161 } \\
Hours Worked Per Week &      1.13 &      0.40 &      1.22 &      0.16 &     -0.12 &      0.53 &     -0.86 &     -0.98 &     -1.25 &     -0.10 \\
& (     1.90 ) & (     2.04 ) & (     2.08 ) & (     2.87 ) & (     3.44 ) & (     2.43 ) & (     1.90 ) & (     2.03 ) & (     2.14 ) & (     1.89 ) \\
& \textit{ 123 } & \textit{ 123 } & \textit{ 123 } & \textit{ 266 } & \textit{ 292 } & \textit{ 123 } & \textit{ 146 } & \textit{ 146 } & \textit{ 146 } & \textit{ 146 } \\
Married or Cohabitating &      0.08 &      0.04 &      0.04 &     -0.01 &     -0.14 &      0.02 &      0.02 &      0.01 &      0.01 &      0.00 \\
& (     0.08 ) & (     0.08 ) & (     0.08 ) & (     0.08 ) & (     0.12 ) & (     0.08 ) & (     0.07 ) & (     0.07 ) & (     0.07 ) & (     0.07 ) \\
& \textit{ 153 } & \textit{ 153 } & \textit{ 153 } & \textit{ 299 } & \textit{ 326 } & \textit{ 153 } & \textit{ 161 } & \textit{ 161 } & \textit{ 161 } & \textit{ 161 } \\
Obese &     -0.02 &      0.04 &      0.00 &     -0.08 &     -0.04 &      0.02 &      0.03 &     -0.01 &     -0.04 &     -0.05 \\
& (     0.07 ) & (     0.06 ) & (     0.06 ) & (     0.07 ) & (     0.10 ) & (     0.06 ) & (     0.07 ) & (     0.07 ) & (     0.07 ) & (     0.09 ) \\
& \textit{ 153 } & \textit{ 153 } & \textit{ 153 } & \textit{ 299 } & \textit{ 326 } & \textit{ 153 } & \textit{ 161 } & \textit{ 161 } & \textit{ 161 } & \textit{ 161 } \\
Overweight &      0.04 &     -0.01 &     -0.01 & \textbf{      0.14 } &     -0.02 &     -0.01 &     -0.04 &     -0.03 &     -0.02 &      0.05 \\
& (     0.07 ) & (     0.06 ) & (     0.06 ) & (     0.08 ) & (     0.09 ) & (     0.07 ) & (     0.07 ) & (     0.07 ) & (     0.07 ) & (     0.07 ) \\
& \textit{ 153 } & \textit{ 153 } & \textit{ 153 } & \textit{ 299 } & \textit{ 326 } & \textit{ 153 } & \textit{ 161 } & \textit{ 161 } & \textit{ 161 } & \textit{ 161 } \\
Locus of Control - positive &      0.11 &      0.04 &      0.06 &      0.14 &     -0.19 &      0.04 &      0.13 &      0.14 &      0.11 &      0.08 \\
& (     0.12 ) & (     0.11 ) & (     0.12 ) & (     0.19 ) & (     0.19 ) & (     0.11 ) & (     0.13 ) & (     0.14 ) & (     0.14 ) & (     0.14 ) \\
& \textit{ 149 } & \textit{ 149 } & \textit{ 149 } & \textit{ 286 } & \textit{ 315 } & \textit{ 149 } & \textit{ 158 } & \textit{ 158 } & \textit{ 158 } & \textit{ 158 } \\
Depression Score - positive &      0.46 &     -0.53 &     -0.24 &     -0.39 &     -0.72 &     -0.63 &      0.57 & \textbf{      1.34 } &      0.99 &      0.99 \\
& (     1.05 ) & (     0.69 ) & (     0.67 ) & (     1.04 ) & (     1.52 ) & (     0.63 ) & (     0.91 ) & (     0.85 ) & (     0.89 ) & (     0.93 ) \\
& \textit{ 151 } & \textit{ 151 } & \textit{ 151 } & \textit{ 297 } & \textit{ 321 } & \textit{ 151 } & \textit{ 158 } & \textit{ 158 } & \textit{ 158 } & \textit{ 158 } \\
Ever Voted for Municipal &      0.02 &      0.00 &      0.01 &     -0.07 & \textbf{     -0.27 } &      0.01 &     -0.05 &      0.08 &      0.07 & \textbf{     0.10} \\
& (     0.08 ) & (     0.06 ) & (     0.07 ) & (     0.05 ) & (     0.09 ) & (     0.06 ) & (     0.08 ) & (     0.07 ) & (     0.07 ) & (     0.07 ) \\
& \textit{ 151 } & \textit{ 151 } & \textit{ 151 } & \textit{ 295 } & \textit{ 314 } & \textit{ 151 } & \textit{ 153 } & \textit{ 153 } & \textit{ 153 } & \textit{ 153 } \\
Ever Voted for Regional &     -0.01 &     -0.04 &     -0.03 &     -0.07 & \textbf{     -0.37 } &     -0.04 &     -0.03 &      0.09 &      0.08 & \textbf{     0.11} \\
& (     0.08 ) & (     0.07 ) & (     0.07 ) & (     0.05 ) & (     0.09 ) & (     0.06 ) & (     0.08 ) & (     0.07 ) & (     0.07 ) & (     0.08 ) \\
& \textit{ 151 } & \textit{ 151 } & \textit{ 151 } & \textit{ 295 } & \textit{ 314 } & \textit{ 151 } & \textit{ 153 } & \textit{ 153 } & \textit{ 153 } & \textit{ 153 } \\
\bottomrule
\end{tabular}
}
\vspace{1ex} \\
\footnotesize\raggedright{Note: This table shows the estimates of the coefficient for attending Reggio Approach preschools from multiple methods. We compare Reggio Approach people with people who attended other preschools.  Column title indicates the corresponding control set and and model. For age-30 cohort, the columns are as follows: ``None30'' refers to the OLS estimate with no control variables. ``BIC30'' refers to the OLS estimate with controls selected by Bayesian Information Criterion (BIC) and additional controls for caregiver's religion. ``Full30'' refers to the OLS estimate with the full set of controls. ``DidPm30'' refers to the difference-in-difference estimate of (Reggio Muni - Parma Muni) - (Reggio None - Parma None). ``DidPv30'' refers to the difference-in-difference estimate of (Reggio Muni - Padova Muni) - (Reggio None - Padova None).  ``AIPW30" refers to AIPW estimate for comparing Reggio Approach children with children in Reggio who attended other type of preschool. Column titles are analogous for age-40 cohort. Robust standard errors are reported in parentheses. Bold number shows that the estimate is statistically significant at the 10\% level. Number of observations used in estimation is reported in italic.}
\end{table}

\begin{table}[H] \caption{Estimation Results for Main Outcomes, Comparison to No Preschools, Adult Cohorts} \label{ols-M-adult-reg-nopres}
\scalebox{0.75}{\begin{tabular}{l c c c c c c c c c c}
\toprule
 & None30 & BIC30 & Full30 & DidPm30 & DidPv30 & AIPW30 & None40 & BIC40 & Full40 & AIPW40 \\
\midrule
Graduate from High School &     -0.02 &      0.05 &      0.05 & \textbf{     -0.17 } &      0.06 &      0.06 &     -0.07 &     -0.01 &     -0.06 &     -0.01 \\
& (     0.05 ) & (     0.05 ) & (     0.05 ) & (     0.08 ) & (     0.07 ) & (     0.06 ) & (     0.05 ) & (     0.05 ) & (     0.06 ) & (     0.06 ) \\
& \textit{ 151 } & \textit{ 151 } & \textit{ 151 } & \textit{ 232 } & \textit{ 217 } & \textit{ 151 } & \textit{ 170 } & \textit{ 170 } & \textit{ 170 } & \textit{ 170 } \\
Max Edu: University &     -0.06 &      0.00 &     -0.01 & \textbf{     -0.17 } &      0.10 &      0.01 &      0.01 &      0.06 & \textbf{      0.11 } &      0.06 \\
& (     0.07 ) & (     0.08 ) & (     0.08 ) & (     0.09 ) & (     0.13 ) & (     0.07 ) & (     0.06 ) & (     0.06 ) & (     0.06 ) & (     0.06 ) \\
& \textit{ 151 } & \textit{ 151 } & \textit{ 151 } & \textit{ 232 } & \textit{ 217 } & \textit{ 151 } & \textit{ 170 } & \textit{ 170 } & \textit{ 170 } & \textit{ 170 } \\
Employed &      0.05 &      0.03 &      0.06 &     -0.04 &     -0.00 &      0.02 & \textbf{      0.06 } & \textbf{      0.08 } &      0.05 & \textbf{     0.08} \\
& (     0.05 ) & (     0.05 ) & (     0.04 ) & (     0.07 ) & (     0.09 ) & (     0.04 ) & (     0.04 ) & (     0.03 ) & (     0.03 ) & (     0.05 ) \\
& \textit{ 151 } & \textit{ 151 } & \textit{ 151 } & \textit{ 232 } & \textit{ 217 } & \textit{ 151 } & \textit{ 169 } & \textit{ 169 } & \textit{ 169 } & \textit{ 169 } \\
Hours Worked Per Week & \textbf{      7.93 } & \textbf{      5.10 } & \textbf{      6.27 } &     -1.18 &     -0.35 &      4.28 & \textbf{      5.71 } & \textbf{      7.29 } & \textbf{      7.39 } & \textbf{     7.24} \\
& (     2.74 ) & (     3.00 ) & (     2.90 ) & (     3.19 ) & (     3.90 ) & (     3.03 ) & (     2.42 ) & (     2.39 ) & (     2.60 ) & (     2.23 ) \\
& \textit{ 124 } & \textit{ 124 } & \textit{ 124 } & \textit{ 205 } & \textit{ 190 } & \textit{ 124 } & \textit{ 151 } & \textit{ 151 } & \textit{ 151 } & \textit{ 151 } \\
Married or Cohabitating &      0.00 &     -0.09 &     -0.10 &      0.04 &     -0.04 &     -0.08 &      0.02 &      0.01 &      0.05 &      0.04 \\
& (     0.08 ) & (     0.08 ) & (     0.09 ) & (     0.11 ) & (     0.14 ) & (     0.08 ) & (     0.07 ) & (     0.07 ) & (     0.08 ) & (     0.08 ) \\
& \textit{ 151 } & \textit{ 151 } & \textit{ 151 } & \textit{ 232 } & \textit{ 217 } & \textit{ 151 } & \textit{ 170 } & \textit{ 170 } & \textit{ 170 } & \textit{ 170 } \\
Obese &     -0.01 & \textbf{      0.10 } & \textbf{      0.10 } &      0.02 & \textbf{     -0.23 } & \textbf{     0.10} & \textbf{     -0.14 } &     -0.08 &     -0.01 &     -0.07 \\
& (     0.07 ) & (     0.06 ) & (     0.07 ) & (     0.08 ) & (     0.12 ) & (     0.06 ) & (     0.07 ) & (     0.08 ) & (     0.08 ) & (     0.08 ) \\
& \textit{ 151 } & \textit{ 151 } & \textit{ 151 } & \textit{ 232 } & \textit{ 217 } & \textit{ 151 } & \textit{ 170 } & \textit{ 170 } & \textit{ 170 } & \textit{ 170 } \\
Overweight &      0.05 &     -0.04 &     -0.01 &      0.06 &      0.01 &     -0.03 &      0.03 &     -0.04 &     -0.07 &     -0.06 \\
& (     0.07 ) & (     0.06 ) & (     0.06 ) & (     0.10 ) & (     0.10 ) & (     0.06 ) & (     0.07 ) & (     0.07 ) & (     0.07 ) & (     0.08 ) \\
& \textit{ 151 } & \textit{ 151 } & \textit{ 151 } & \textit{ 232 } & \textit{ 217 } & \textit{ 151 } & \textit{ 170 } & \textit{ 170 } & \textit{ 170 } & \textit{ 170 } \\
Locus of Control - positive &      0.08 &     -0.10 &     -0.11 & \textbf{      0.46 } &     -0.03 &     -0.09 &      0.14 & \textbf{      0.20 } & \textbf{      0.28 } & \textbf{     0.27} \\
& (     0.14 ) & (     0.13 ) & (     0.13 ) & (     0.22 ) & (     0.22 ) & (     0.15 ) & (     0.13 ) & (     0.13 ) & (     0.14 ) & (     0.14 ) \\
& \textit{ 148 } & \textit{ 148 } & \textit{ 148 } & \textit{ 221 } & \textit{ 206 } & \textit{ 148 } & \textit{ 165 } & \textit{ 165 } & \textit{ 165 } & \textit{ 165 } \\
Depression Score - positive & \textbf{      1.57 } &     -0.34 &     -0.35 &     -0.67 &      1.02 &     -0.22 & \textbf{      2.25 } & \textbf{      2.23 } & \textbf{      2.10 } & \textbf{     2.47} \\
& (     1.01 ) & (     0.88 ) & (     0.91 ) & (     1.25 ) & (     1.74 ) & (     0.86 ) & (     0.92 ) & (     0.95 ) & (     1.07 ) & (     1.15 ) \\
& \textit{ 149 } & \textit{ 149 } & \textit{ 149 } & \textit{ 230 } & \textit{ 214 } & \textit{ 149 } & \textit{ 168 } & \textit{ 168 } & \textit{ 168 } & \textit{ 168 } \\
Ever Voted for Municipal & \textbf{      0.19 } &      0.07 &      0.07 &      0.02 &      0.06 &      0.06 & \textbf{      0.19 } & \textbf{      0.12 } &      0.11 & \textbf{     0.14} \\
& (     0.08 ) & (     0.07 ) & (     0.07 ) & (     0.06 ) & (     0.11 ) & (     0.08 ) & (     0.08 ) & (     0.08 ) & (     0.08 ) & (     0.07 ) \\
& \textit{ 148 } & \textit{ 148 } & \textit{ 148 } & \textit{ 228 } & \textit{ 207 } & \textit{ 148 } & \textit{ 153 } & \textit{ 153 } & \textit{ 153 } & \textit{ 153 } \\
Ever Voted for Regional & \textbf{      0.14 } &      0.02 &      0.03 &      0.01 &     -0.07 &      0.02 & \textbf{      0.20 } & \textbf{      0.14 } & \textbf{      0.13 } & \textbf{     0.16} \\
& (     0.08 ) & (     0.07 ) & (     0.07 ) & (     0.06 ) & (     0.10 ) & (     0.08 ) & (     0.08 ) & (     0.08 ) & (     0.08 ) & (     0.09 ) \\
& \textit{ 148 } & \textit{ 148 } & \textit{ 148 } & \textit{ 228 } & \textit{ 207 } & \textit{ 148 } & \textit{ 153 } & \textit{ 153 } & \textit{ 153 } & \textit{ 153 } \\
\bottomrule
\end{tabular}
}
\vspace{1ex} \\
\footnotesize\raggedright{Note: This table shows the estimates of the coefficient for attending Reggio Approach preschools from multiple methods. We compare Reggio Approach people with people who attended no preschool. Column title indicates the corresponding control set and and model. For age-30 cohort, the columns are as follows: ``None30'' refers to the OLS estimate with no control variables. ``BIC30'' refers to the OLS estimate with controls selected by Bayesian Information Criterion (BIC) and additional controls for caregiver's religion. ``Full30'' refers to the OLS estimate with the full set of controls. ``DidPm30'' refers to the difference-in-difference estimate of (Reggio Muni - Parma Muni) - (Reggio None - Parma None). ``DidPv30'' refers to the difference-in-difference estimate of (Reggio Muni - Padova Muni) - (Reggio None - Padova None).  ``AIPW30" refers to AIPW estimate for comparing Reggio Approach children with children in Reggio who did not attend any preschool. Column titles are analogous for age-40 cohort. Robust standard errors are reported in parentheses. Bold number shows that the estimate is statistically significant at the 10\% level. Number of observations used in estimation is reported in italic.}
\end{table}

\end{landscape}
