We present the estimates of the methods described above for a handful of key outcomes that do not have a severe missing value problem and have enough variation across individuals. For the children and adolescents cohorts, we only compare Reggio Approach people with people who went to other types of preschool, because there are less than 10 people in no preschool group for those cohorts. For adult cohorts, we compare Reggio Approach people with people who went to other types of preschool and people who did not attend any preschool, respectively.

In the child cohort, the Reggio Approach increased SDQ scores when compared to children who attended other preschools (OLS). This result gets more positive after controlling for more background characteristics. These results do not persist when comparing with children in Parma and Padova (difference-in-difference). Using AIPW to address issues of selection provides similar results as those of the OLS specification. The children are slightly more likely to be overweight and have worse health, although these estimates are not significant for all specifications. Other main outcomes do not show significant effect.

In the adolescent cohort, the Reggio Approach people are significantly less likely to be depressed according to OLS and AIPW results. However, DiD estimate with Parma shows that Reggio Approach people are more likely to be depressed compared to municipal preschool people in Parma. The Reggio Approach people are more likely to be obese than people who attended other types of preschool, and the estimate on obesity is consistent across most of the methods. Some methods show that Reggio Approach people are less likely to like school and less likely to be involved in sport activities. 

In the adult cohorts, there are differences when we compare Reggio Approach people to other preschool group and to no preschool group. The comparison with no preschool group shows much more statistically significant estimates. In the comparison with the other preschool group, the only outcome that shows some significance across different methods is high school graduation rate in age-40 cohort. OLS estimate shows that the Reggio Approach age-40 people are more likely to graduate from high school than other group. 

In the comparison with no preschool group, Reggio Approach people in age-40 group are more likely to be employed. Moreover, Reggio Approach people are significantly more likely to work more during the week than other groups for both age-30 and age-40 groups. OLS estimates show that the Reggio Approach people are significantly more likely to be obese when compared to no preschool group in Reggio. Age-40 Reggio Approach people have significant positive effects on locus of control, depression score, and voting behaviors. 

% ========================================================================= %
% CHILD COHORT

\begin{landscape}

\begin{table}[H] \caption{Estimation Results for Main Outcomes, Comparison to Non-RA Preschools, Child Cohort} \label{ols-M-child-reg-pres}
\scalebox{0.8}{\begin{tabular}{l c c c c c c c}
\toprule
 & NoneIt & BICIt & FullIt & DidPmIt & DidPvIt & AIPWnoneIt & AIPWpresIt \\
\midrule
SDQ Composite - Child &      0.64 & \textbf{      1.03 } & \textbf{      1.18 } &      0.80 &     -0.33 & \textbf{     2.27} & \textbf{     1.02} \\
& (     0.47) & (     0.46) & (     0.45) & (     0.71) & (     0.58) & (     2.46) & (     0.42) \\
Obese &      0.00 &      0.02 &      0.04 &      0.00 &      0.06 &      0.07 &      0.02 \\
& (     0.05) & (     0.05) & (     0.05) & (     0.06) & (     0.06) & (     0.19) & (     0.04) \\
Overweight & \textbf{      0.05 } &      0.04 &      0.05 & \textbf{      0.10 } &     -0.04 &      0.07 & \textbf{     0.05} \\
& (     0.03) & (     0.03) & (     0.04) & (     0.06) & (     0.04) & (     0.20) & (     0.04) \\
Health is Good &     -0.04 &     -0.03 &     -0.01 &     -0.11 &      0.00 & \textbf{     0.41} &     -0.00 \\
& (     0.05) & (     0.05) & (     0.05) & (     0.08) & (     0.05) & (     0.20) & (     0.03) \\
Not Excited to Learn &     -0.00 &      0.00 &     -0.00 &     -0.03 &      0.03 &     -0.08 &     -0.01 \\
& (     0.02) & (     0.02) & (     0.02) & (     0.03) & (     0.03) & (     0.19) & (     0.02) \\
Problems Sitting Still &      0.03 &      0.02 &      0.00 & \textbf{      0.10 } &      0.06 & \textbf{     0.14} &      0.01 \\
& (     0.03) & (     0.03) & (     0.03) & (     0.06) & (     0.04) & (     0.03) & (     0.03) \\
How Much Child Likes School &      0.01 &      0.03 &      0.03 &     -0.06 &     -0.07 & \textbf{     0.56} &      0.04 \\
& (     0.06) & (     0.06) & (     0.06) & (     0.08) & (     0.07) & (     0.32) & (     0.06) \\
\bottomrule
\end{tabular}
}
\vspace{1ex} \\
\footnotesize\raggedright{Note: This table shows the estimates of the coefficient for attending Reggio Approach preschools from multiple methods. We compare Reggio Approach people with people who attended other preschools. Column title indicates the corresponding control set and and model. ``NoneIt'' refers to the OLS estimate with no control variables. ``BICIt'' refers to the OLS estimate with controls selected by Bayesian Information Criterion (BIC) and additional controls for caregiver's religion. ``FullIt'' refers to the OLS estimate with the full set of controls. ``DidPmIt'' refers to the difference-in-difference estimate of (Reggio Muni - Parma Muni) - (Reggio None - Parma None). ``DidPvIt'' refers to the difference-in-difference estimate of (Reggio Muni - Padova Muni) - (Reggio None - Padova None). ``AIPWIt" refers to AIPW estimate for comparing Reggio Approach children with children in Reggio who attended other type of preschool. Robust standard errors are reported in parentheses. Bold number shows that the estimate is statistically significant at the 10\% level. Number of observations used in estimation is reported in italic.}

\end{table}


\begin{table}[H] \caption{Estimation Results for Main Outcomes, Comparison to Non-RA Preschools, Adolescent Cohort} \label{ols-M-adol-reg-pres}
\scalebox{0.8}{\begin{tabular}{l c c c c c c}
\toprule
 & None & Bic & Full & DidPm & DidPv & AIPW \\
\midrule
SDQ Composite - Child &      0.00 &      0.02 &      0.35 &     -0.32 &      0.44 &      0.32 \\
& (     0.58 ) & (     0.57 ) & (     0.61 ) & (     0.76 ) & (     0.48 ) & (     0.55 ) \\
SDQ Composite &      0.71 &      0.52 &      0.56 &     -0.56 &     -0.09 &      0.20 \\
& (     0.62 ) & (     0.64 ) & (     0.70 ) & (     0.77 ) & (     0.64 ) & (     0.58 ) \\
Depression Score - positive & \textbf{      1.13 } &      1.16 & \textbf{      1.43 } & \textbf{     -1.40 } &     -0.54 & \textbf{     0.77} \\
& (     0.77 ) & (     0.83 ) & (     0.89 ) & (     0.82 ) & (     0.74 ) & (     0.80 ) \\
Obese & \textbf{      0.07 } & \textbf{      0.08 } & \textbf{      0.08 } &      0.07 &     -0.05 & \textbf{     0.08} \\
& (     0.04 ) & (     0.04 ) & (     0.04 ) & (     0.05 ) & (     0.05 ) & (     0.04 ) \\
Overweight &     -0.01 &     -0.01 &     -0.01 &      0.05 &     -0.02 &     -0.00 \\
& (     0.02 ) & (     0.02 ) & (     0.03 ) & (     0.04 ) & (     0.01 ) & (     0.02 ) \\
Health is Good &      0.05 & \textbf{      0.08 } &      0.08 &      0.00 &     -0.03 & \textbf{     0.08} \\
& (     0.06 ) & (     0.06 ) & (     0.06 ) & (     0.07 ) & (     0.06 ) & (     0.06 ) \\
Go To School & \textbf{      0.04 } & \textbf{      0.04 } & \textbf{      0.05 } & \textbf{      0.03 } &     -0.01 & \textbf{     0.05} \\
& (     0.02 ) & (     0.03 ) & (     0.03 ) & (     0.01 ) & (     0.02 ) & (     0.03 ) \\
How Much Child Likes School &     -0.13 & \textbf{     -0.18 } & \textbf{     -0.19 } & \textbf{     -0.23 } &     -0.08 &     -0.10 \\
& (     0.11 ) & (     0.12 ) & (     0.12 ) & (     0.15 ) & (     0.11 ) & (     0.12 ) \\
Trust Score &     -0.00 &     -0.10 &     -0.02 &     -0.31 &      0.07 &     -0.08 \\
& (     0.18 ) & (     0.18 ) & (     0.19 ) & (     0.23 ) & (     0.18 ) & (     0.16 ) \\
Days of Sport (Weekly) & \textbf{     -0.40 } &     -0.31 &     -0.29 &      0.35 &      0.20 &     -0.33 \\
& (     0.23 ) & (     0.24 ) & (     0.26 ) & (     0.28 ) & (     0.24 ) & (     0.24 ) \\
\bottomrule
\end{tabular}
}
\vspace{1ex} \\
\footnotesize\raggedright{Note: This table shows the estimates of the coefficient for attending Reggio Approach preschools from multiple methods. We compare Reggio Approach people with people who attended other preschools. Column title indicates the corresponding control set and and model. ``None'' refers to the OLS estimate with no control variables. ``BIC'' refers to the OLS estimate with controls selected by Bayesian Information Criterion (BIC) and additional controls for caregiver's religion. ``Full'' refers to the OLS estimate with the full set of controls. ``DidPm'' refers to the difference-in-difference estimate of (Reggio Muni - Parma Muni) - (Reggio None - Parma None). ``DidPv'' refers to the difference-in-difference estimate of (Reggio Muni - Padova Muni) - (Reggio None - Padova None). ``AIPW" refers to AIPW estimate for comparing Reggio Approach children with children in Reggio who attended other type of preschool. Robust standard errors are reported in parentheses. Bold number shows that the estimate is statistically significant at the 10\% level. Number of observations used in estimation is reported in italic.}
\end{table}




\begin{table}[H] \caption{Estimation Results for Main Outcomes, Comparison to Non-RA Preschools, Adult Cohorts} \label{ols-M-adult-reg-pres}
\scalebox{0.75}{\begin{tabular}{l c c c c c c c c c c}
\toprule
 & None30 & BIC30 & Full30 & DidPm30 & DidPv30 & AIPW30 & None40 & BIC40 & Full40 & AIPW40 \\
\midrule
Graduate from High School &     -0.04 &     -0.03 &     -0.05 & \textbf{     -0.17 } &      0.08 &     -0.06 & \textbf{      0.13 } &      0.09 & \textbf{      0.11 } &      0.01 \\
& (     0.05 ) & (     0.05 ) & (     0.05 ) & (     0.06 ) & (     0.06 ) & (     0.05 ) & (     0.07 ) & (     0.07 ) & (     0.07 ) & (     0.06 ) \\
& \textit{ 153 } & \textit{ 153 } & \textit{ 153 } & \textit{ 299 } & \textit{ 326 } & \textit{ 153 } & \textit{ 161 } & \textit{ 161 } & \textit{ 161 } & \textit{ 161 } \\
Max Edu: University &      0.04 &      0.04 &      0.03 & \textbf{     -0.18 } &     -0.17 &      0.02 &      0.08 &      0.05 &      0.03 &      0.05 \\
& (     0.07 ) & (     0.07 ) & (     0.07 ) & (     0.08 ) & (     0.12 ) & (     0.07 ) & (     0.06 ) & (     0.05 ) & (     0.05 ) & (     0.06 ) \\
& \textit{ 153 } & \textit{ 153 } & \textit{ 153 } & \textit{ 299 } & \textit{ 326 } & \textit{ 153 } & \textit{ 161 } & \textit{ 161 } & \textit{ 161 } & \textit{ 161 } \\
Employed &     -0.02 &     -0.02 &     -0.01 &     -0.03 &      0.02 &     -0.02 &      0.01 &      0.01 &      0.01 &      0.02 \\
& (     0.03 ) & (     0.04 ) & (     0.04 ) & (     0.06 ) & (     0.08 ) & (     0.03 ) & (     0.03 ) & (     0.03 ) & (     0.04 ) & (     0.03 ) \\
& \textit{ 153 } & \textit{ 153 } & \textit{ 153 } & \textit{ 299 } & \textit{ 326 } & \textit{ 153 } & \textit{ 161 } & \textit{ 161 } & \textit{ 161 } & \textit{ 161 } \\
Hours Worked Per Week &      1.13 &      0.40 &      1.22 &      0.16 &     -0.12 &      0.53 &     -0.86 &     -0.98 &     -1.25 &     -0.10 \\
& (     1.90 ) & (     2.04 ) & (     2.08 ) & (     2.87 ) & (     3.44 ) & (     2.43 ) & (     1.90 ) & (     2.03 ) & (     2.14 ) & (     1.89 ) \\
& \textit{ 123 } & \textit{ 123 } & \textit{ 123 } & \textit{ 266 } & \textit{ 292 } & \textit{ 123 } & \textit{ 146 } & \textit{ 146 } & \textit{ 146 } & \textit{ 146 } \\
Married or Cohabitating &      0.08 &      0.04 &      0.04 &     -0.01 &     -0.14 &      0.02 &      0.02 &      0.01 &      0.01 &      0.00 \\
& (     0.08 ) & (     0.08 ) & (     0.08 ) & (     0.08 ) & (     0.12 ) & (     0.08 ) & (     0.07 ) & (     0.07 ) & (     0.07 ) & (     0.07 ) \\
& \textit{ 153 } & \textit{ 153 } & \textit{ 153 } & \textit{ 299 } & \textit{ 326 } & \textit{ 153 } & \textit{ 161 } & \textit{ 161 } & \textit{ 161 } & \textit{ 161 } \\
Obese &     -0.02 &      0.04 &      0.00 &     -0.08 &     -0.04 &      0.02 &      0.03 &     -0.01 &     -0.04 &     -0.05 \\
& (     0.07 ) & (     0.06 ) & (     0.06 ) & (     0.07 ) & (     0.10 ) & (     0.06 ) & (     0.07 ) & (     0.07 ) & (     0.07 ) & (     0.09 ) \\
& \textit{ 153 } & \textit{ 153 } & \textit{ 153 } & \textit{ 299 } & \textit{ 326 } & \textit{ 153 } & \textit{ 161 } & \textit{ 161 } & \textit{ 161 } & \textit{ 161 } \\
Overweight &      0.04 &     -0.01 &     -0.01 & \textbf{      0.14 } &     -0.02 &     -0.01 &     -0.04 &     -0.03 &     -0.02 &      0.05 \\
& (     0.07 ) & (     0.06 ) & (     0.06 ) & (     0.08 ) & (     0.09 ) & (     0.07 ) & (     0.07 ) & (     0.07 ) & (     0.07 ) & (     0.07 ) \\
& \textit{ 153 } & \textit{ 153 } & \textit{ 153 } & \textit{ 299 } & \textit{ 326 } & \textit{ 153 } & \textit{ 161 } & \textit{ 161 } & \textit{ 161 } & \textit{ 161 } \\
Locus of Control - positive &      0.11 &      0.04 &      0.06 &      0.14 &     -0.19 &      0.04 &      0.13 &      0.14 &      0.11 &      0.08 \\
& (     0.12 ) & (     0.11 ) & (     0.12 ) & (     0.19 ) & (     0.19 ) & (     0.11 ) & (     0.13 ) & (     0.14 ) & (     0.14 ) & (     0.14 ) \\
& \textit{ 149 } & \textit{ 149 } & \textit{ 149 } & \textit{ 286 } & \textit{ 315 } & \textit{ 149 } & \textit{ 158 } & \textit{ 158 } & \textit{ 158 } & \textit{ 158 } \\
Depression Score - positive &      0.46 &     -0.53 &     -0.24 &     -0.39 &     -0.72 &     -0.63 &      0.57 & \textbf{      1.34 } &      0.99 &      0.99 \\
& (     1.05 ) & (     0.69 ) & (     0.67 ) & (     1.04 ) & (     1.52 ) & (     0.63 ) & (     0.91 ) & (     0.85 ) & (     0.89 ) & (     0.93 ) \\
& \textit{ 151 } & \textit{ 151 } & \textit{ 151 } & \textit{ 297 } & \textit{ 321 } & \textit{ 151 } & \textit{ 158 } & \textit{ 158 } & \textit{ 158 } & \textit{ 158 } \\
Ever Voted for Municipal &      0.02 &      0.00 &      0.01 &     -0.07 & \textbf{     -0.27 } &      0.01 &     -0.05 &      0.08 &      0.07 & \textbf{     0.10} \\
& (     0.08 ) & (     0.06 ) & (     0.07 ) & (     0.05 ) & (     0.09 ) & (     0.06 ) & (     0.08 ) & (     0.07 ) & (     0.07 ) & (     0.07 ) \\
& \textit{ 151 } & \textit{ 151 } & \textit{ 151 } & \textit{ 295 } & \textit{ 314 } & \textit{ 151 } & \textit{ 153 } & \textit{ 153 } & \textit{ 153 } & \textit{ 153 } \\
Ever Voted for Regional &     -0.01 &     -0.04 &     -0.03 &     -0.07 & \textbf{     -0.37 } &     -0.04 &     -0.03 &      0.09 &      0.08 & \textbf{     0.11} \\
& (     0.08 ) & (     0.07 ) & (     0.07 ) & (     0.05 ) & (     0.09 ) & (     0.06 ) & (     0.08 ) & (     0.07 ) & (     0.07 ) & (     0.08 ) \\
& \textit{ 151 } & \textit{ 151 } & \textit{ 151 } & \textit{ 295 } & \textit{ 314 } & \textit{ 151 } & \textit{ 153 } & \textit{ 153 } & \textit{ 153 } & \textit{ 153 } \\
\bottomrule
\end{tabular}
}
\vspace{1ex} \\
\footnotesize\raggedright{Note: This table shows the estimates of the coefficient for attending Reggio Approach preschools from multiple methods. We compare Reggio Approach people with people who attended other preschools.  Column title indicates the corresponding control set and and model. For age-30 cohort, the columns are as follows: ``None30'' refers to the OLS estimate with no control variables. ``BIC30'' refers to the OLS estimate with controls selected by Bayesian Information Criterion (BIC) and additional controls for caregiver's religion. ``Full30'' refers to the OLS estimate with the full set of controls. ``DidPm30'' refers to the difference-in-difference estimate of (Reggio Muni - Parma Muni) - (Reggio None - Parma None). ``DidPv30'' refers to the difference-in-difference estimate of (Reggio Muni - Padova Muni) - (Reggio None - Padova None).  ``AIPW30" refers to AIPW estimate for comparing Reggio Approach children with children in Reggio who attended other type of preschool. Column titles are analogous for age-40 cohort. Robust standard errors are reported in parentheses. Bold number shows that the estimate is statistically significant at the 10\% level. Number of observations used in estimation is reported in italic.}
\end{table}

\begin{table}[H] \caption{Estimation Results for Main Outcomes, Comparison to No Preschools, Adult Cohorts} \label{ols-M-adult-reg-nopres}
\scalebox{0.75}{\begin{tabular}{l c c c c c c c c c c}
\toprule
 & None30 & BIC30 & Full30 & DidPm30 & DidPv30 & AIPW30 & None40 & BIC40 & Full40 & AIPW40 \\
\midrule
Graduate from High School &     -0.02 &      0.05 &      0.05 & \textbf{     -0.17 } &      0.07 &      0.06 &     -0.07 &     -0.01 &     -0.06 &     -0.01 \\
& (     0.05 ) & (     0.05 ) & (     0.05 ) & (     0.08 ) & (     0.08 ) & (     0.06 ) & (     0.05 ) & (     0.05 ) & (     0.06 ) & (     0.06 ) \\
Max Edu: University &     -0.06 &      0.00 &     -0.01 & \textbf{     -0.17 } &      0.12 &      0.01 &      0.01 &      0.06 & \textbf{      0.11 } & \textbf{     0.06} \\
& (     0.07 ) & (     0.08 ) & (     0.08 ) & (     0.09 ) & (     0.13 ) & (     0.07 ) & (     0.06 ) & (     0.06 ) & (     0.06 ) & (     0.05 ) \\
Employed &      0.05 &      0.03 &      0.06 &     -0.05 &     -0.01 &      0.02 & \textbf{      0.06 } & \textbf{      0.08 } &      0.05 & \textbf{     0.08} \\
& (     0.05 ) & (     0.05 ) & (     0.04 ) & (     0.07 ) & (     0.09 ) & (     0.04 ) & (     0.04 ) & (     0.03 ) & (     0.03 ) & (     0.05 ) \\
Hours Worked Per Week & \textbf{      6.36 } & \textbf{      5.52 } & \textbf{      5.44 } &      0.20 &     -0.51 & \textbf{     4.84} & \textbf{      3.67 } & \textbf{      3.68 } & \textbf{      5.27 } & \textbf{     3.50} \\
& (     2.02 ) & (     2.13 ) & (     2.05 ) & (     2.14 ) & (     1.62 ) & (     2.12 ) & (     1.97 ) & (     1.91 ) & (     2.12 ) & (     1.98 ) \\
Married or Cohabitating &      0.00 &     -0.09 &     -0.10 &      0.03 &     -0.03 &     -0.08 &      0.02 &      0.01 &      0.05 &      0.04 \\
& (     0.08 ) & (     0.08 ) & (     0.09 ) & (     0.11 ) & (     0.14 ) & (     0.06 ) & (     0.07 ) & (     0.07 ) & (     0.08 ) & (     0.07 ) \\
Obese &     -0.01 & \textbf{      0.10 } & \textbf{      0.10 } &      0.03 & \textbf{     -0.21 } & \textbf{     0.10} & \textbf{     -0.14 } &     -0.08 &     -0.01 &     -0.07 \\
& (     0.07 ) & (     0.06 ) & (     0.07 ) & (     0.08 ) & (     0.13 ) & (     0.06 ) & (     0.07 ) & (     0.08 ) & (     0.08 ) & (     0.07 ) \\
Overweight &      0.05 &     -0.04 &     -0.01 &      0.07 &      0.02 &     -0.03 &      0.03 &     -0.04 &     -0.07 &     -0.06 \\
& (     0.07 ) & (     0.06 ) & (     0.06 ) & (     0.10 ) & (     0.11 ) & (     0.07 ) & (     0.07 ) & (     0.07 ) & (     0.07 ) & (     0.07 ) \\
Locus of Control - positive &      0.08 &     -0.10 &     -0.11 & \textbf{      0.47 } &     -0.04 &     -0.09 &      0.14 & \textbf{      0.20 } & \textbf{      0.28 } & \textbf{     0.27} \\
& (     0.14 ) & (     0.13 ) & (     0.13 ) & (     0.22 ) & (     0.23 ) & (     0.12 ) & (     0.13 ) & (     0.13 ) & (     0.14 ) & (     0.12 ) \\
& \textbf{      1.57 } &     -0.34 &     -0.35 &     -0.77 &      0.92 &     -0.22 & \textbf{      2.25 } & \textbf{      2.23 } & \textbf{      2.10 } & \textbf{     2.47} \\
& (     1.01 ) & (     0.88 ) & (     0.91 ) & (     1.25 ) & (     1.82 ) & (     0.87 ) & (     0.92 ) & (     0.95 ) & (     1.07 ) & (     0.99 ) \\
Ever Voted for Municipal & \textbf{      0.19 } &      0.07 &      0.07 &      0.03 &      0.08 &      0.06 & \textbf{      0.19 } & \textbf{      0.12 } &      0.11 & \textbf{     0.14} \\
& (     0.08 ) & (     0.07 ) & (     0.07 ) & (     0.06 ) & (     0.11 ) & (     0.06 ) & (     0.08 ) & (     0.08 ) & (     0.08 ) & (     0.10 ) \\
Ever Voted for Regional & \textbf{      0.14 } &      0.02 &      0.03 &      0.02 &     -0.06 &      0.02 & \textbf{      0.20 } & \textbf{      0.14 } & \textbf{      0.13 } & \textbf{     0.16} \\
& (     0.08 ) & (     0.07 ) & (     0.07 ) & (     0.06 ) & (     0.11 ) & (     0.08 ) & (     0.08 ) & (     0.08 ) & (     0.08 ) & (     0.09 ) \\
\bottomrule
\end{tabular}
}
\vspace{1ex} \\
\footnotesize\raggedright{Note: This table shows the estimates of the coefficient for attending Reggio Approach preschools from multiple methods. We compare Reggio Approach people with people who attended no preschool. Column title indicates the corresponding control set and and model. For age-30 cohort, the columns are as follows: ``None30'' refers to the OLS estimate with no control variables. ``BIC30'' refers to the OLS estimate with controls selected by Bayesian Information Criterion (BIC) and additional controls for caregiver's religion. ``Full30'' refers to the OLS estimate with the full set of controls. ``DidPm30'' refers to the difference-in-difference estimate of (Reggio Muni - Parma Muni) - (Reggio None - Parma None). ``DidPv30'' refers to the difference-in-difference estimate of (Reggio Muni - Padova Muni) - (Reggio None - Padova None).  ``AIPW30" refers to AIPW estimate for comparing Reggio Approach children with children in Reggio who did not attend any preschool. Column titles are analogous for age-40 cohort. Robust standard errors are reported in parentheses. Bold number shows that the estimate is statistically significant at the 10\% level. Number of observations used in estimation is reported in italic.}
\end{table}

\end{landscape}
