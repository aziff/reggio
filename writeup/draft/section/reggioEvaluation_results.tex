
Tables \ref{ols-E-reg} to \ref{ols-S-reg} show the OLS estimates of the effect of the Reggio Approach preschools for people in Reggio Emilia who attended Reggio Approach preschools compared to those who did not attend preschool at all. We do not include the age-50 cohort in this analysis, as the Reggio Approach did not exist for that cohort. We perform analysis using (i) no control, (ii) 6 controls selected by Bayesian Information Criterion (BIC), and (iii) the full set of controls that include baseline variables available for the adult cohorts. 

Table \ref{ols-E-reg} shows statistically significant effects of the Reggio Approach on high school grade for age-30 cohort. However, it should be noted that the high school grade measure in the data is not standardized across different schools. 


\begin{table}[H] \caption{OLS Results for Cognitive and Education, Municipal vs. None, Reggio Emilia} \label{ols-E-reg}
{
\def\sym#1{\ifmmode^{#1}\else\(^{#1}\)\fi}
\begin{tabular}{l*{6}{c}}
\toprule
            &\multicolumn{1}{c}{(1)}&\multicolumn{1}{c}{(2)}&\multicolumn{1}{c}{(3)}&\multicolumn{1}{c}{(4)}&\multicolumn{1}{c}{(5)}&\multicolumn{1}{c}{(6)}\\
            &\multicolumn{1}{c}{None30}&\multicolumn{1}{c}{BIC30}&\multicolumn{1}{c}{Full30}&\multicolumn{1}{c}{None40}&\multicolumn{1}{c}{BIC40}&\multicolumn{1}{c}{Full40}\\
\midrule
IQ Factor   &     0.00406         &      -0.220         &      -0.293         &     -0.0747         &     -0.0638         &      -0.162         \\
            &     (0.149)         &     (0.156)         &     (0.161)         &     (0.120)         &     (0.142)         &     (0.170)         \\
\addlinespace
High School Grade&       4.051\sym{*}  &       5.042\sym{*}  &       3.969         &       0.690         &       2.034         &       1.398         \\
            &     (1.961)         &     (2.130)         &     (2.401)         &     (1.395)         &     (1.618)         &     (2.232)         \\
\addlinespace
University Grade&       2.390         &      0.0387         &       0.638         &      -0.826         &      -0.222         &      -1.926         \\
            &     (2.233)         &     (2.552)         &     (2.840)         &     (2.433)         &     (2.751)         &     (5.638)         \\
\addlinespace
Graduate from High School&     -0.0424         &      0.0301         &     0.00954         &      -0.170\sym{***}&     -0.0525         &     -0.0315         \\
            &    (0.0502)         &    (0.0495)         &    (0.0577)         &    (0.0502)         &    (0.0542)         &    (0.0669)         \\
\addlinespace
Max Edu: University&      -0.120         &     -0.0673         &     -0.0744         &     -0.0188         &      0.0528         &      0.0572         \\
            &    (0.0670)         &    (0.0684)         &    (0.0716)         &    (0.0535)         &    (0.0608)         &    (0.0624)         \\
\addlinespace
Max Edu: Graduate School&     0.00671         &     0.00747         &     0.00700         &           0         &           0         &           0         \\
            &   (0.00672)         &   (0.00769)         &   (0.00744)         &         (.)         &         (.)         &         (.)         \\
\bottomrule
\end{tabular}
}

\vspace{1ex} \\
\footnotesize\raggedright{Note: This table shows the OLS estimates for attending Reggio Approach schools for people in Reggio Emilia who attended Reggio Approach preschools or no preschool at all. Column title indicates the age group and control set used in each regression corresponding to the column. ``None30'' refers to the regression with only age-30 cohort and with no control variables. ``BIC30'' refers to the regression with only age-30 cohort and with controls selected by Bayesian Information Criterion (BIC). ``Full30'' refers to the regression with only age-30 cohort and with the full set of controls. Analogous meanings applied to the age-40 cohort. Robust standard errors are reported in parentheses. Stars show statistical significance as follows. * p < 0.05, ** p < 0.01, *** p < 0.001.}
\end{table}

Table \ref{ols-W-reg} shows consistently significant positive effects of the Reggio Approach on hours worked per week. Moreover, there are significant effects on the indicator for the lowest household income category for the age-40 cohort. Note that the measures for subject labor income, including ``Monthly Wage'' presented in the table, have very few observations. Household income suffers less from the missing value problem.

\begin{table}[H] \caption{OLS Results for Employment and Income, Municipal vs. None, Reggio Emilia} \label{ols-W-reg}
\scalebox{0.92}{
{
\def\sym#1{\ifmmode^{#1}\else\(^{#1}\)\fi}
\begin{tabular}{l*{6}{c}}
\toprule
            &\multicolumn{1}{c}{(1)}&\multicolumn{1}{c}{(2)}&\multicolumn{1}{c}{(3)}&\multicolumn{1}{c}{(4)}&\multicolumn{1}{c}{(5)}&\multicolumn{1}{c}{(6)}\\
            &\multicolumn{1}{c}{None30}&\multicolumn{1}{c}{BIC30}&\multicolumn{1}{c}{Full30}&\multicolumn{1}{c}{None40}&\multicolumn{1}{c}{BIC40}&\multicolumn{1}{c}{Full40}\\
\midrule
Employed    &      0.0717         &      0.0526         &      0.0588         &      0.0652         &      0.0574\sym{*}  &    -0.00160         \\
            &    (0.0435)         &    (0.0432)         &    (0.0458)         &    (0.0348)         &    (0.0287)         &    (0.0356)         \\
\addlinespace
Self-Employed&     -0.0566         &      -0.102         &     -0.0994         &      0.0461         &      0.0126         &     0.00948         \\
            &    (0.0591)         &    (0.0631)         &    (0.0686)         &    (0.0487)         &    (0.0556)         &    (0.0763)         \\
\addlinespace
Hours Worked Per Week&       6.476\sym{**} &       6.134\sym{**} &       6.180\sym{**} &       4.736\sym{*}  &       3.648         &       4.713\sym{*}  \\
            &     (1.987)         &     (2.110)         &     (2.028)         &     (1.880)         &     (1.891)         &     (2.354)         \\
\addlinespace
Monthly Wage&      -188.4\sym{*}  &      -179.1\sym{*}  &      -192.6         &      -849.4         &     -1088.6         &       243.9         \\
            &     (74.31)         &     (88.37)         &     (118.0)         &     (699.5)         &     (758.5)         &    (1080.8)         \\
\addlinespace
Income: 5,000 Euros of Less&       0.148\sym{***}&       0.157\sym{***}&       0.173\sym{***}&     -0.0125         &    -0.00656         &     -0.0134         \\
            &    (0.0292)         &    (0.0350)         &    (0.0421)         &    (0.0125)         &   (0.00635)         &    (0.0133)         \\
\addlinespace
Income: 5,001-10,000 Euros&     -0.0284         &     -0.0220         &     -0.0278         &           0         &           0         &           0         \\
            &    (0.0254)         &    (0.0249)         &    (0.0246)         &         (.)         &         (.)         &         (.)         \\
\addlinespace
Income: 10,001-25,000 Euros&      -0.199\sym{**} &      -0.158         &      -0.217\sym{*}  &     -0.0531         &     -0.0284         &     -0.0338         \\
            &    (0.0760)         &    (0.0838)         &    (0.0897)         &    (0.0672)         &    (0.0728)         &    (0.0896)         \\
\addlinespace
Income: 25,001-50,000 Euros&      0.0741         &      0.0332         &      0.0728         &      0.0797         &      0.0464         &     -0.0259         \\
            &    (0.0780)         &    (0.0874)         &    (0.0882)         &    (0.0707)         &    (0.0754)         &    (0.0930)         \\
\addlinespace
Income: 50,001-100,000 Euros&     0.00518         &    -0.00974         &   -0.000807         &      0.0250         &      0.0365         &      0.0525         \\
            &    (0.0294)         &    (0.0306)         &    (0.0214)         &    (0.0303)         &    (0.0340)         &    (0.0317)         \\
\addlinespace
Income: 100,001-250,000 Euros&           0         &           0         &           0         &     -0.0391         &     -0.0479         &      0.0205         \\
            &         (.)         &         (.)         &         (.)         &    (0.0303)         &    (0.0299)         &    (0.0335)         \\
\addlinespace
Income: More than 250,000 Euros&           0         &           0         &           0         &           0         &           0         &           0         \\
            &         (.)         &         (.)         &         (.)         &         (.)         &         (.)         &         (.)         \\
\bottomrule
\end{tabular}
}
}
\vspace{1ex} \\
\footnotesize\raggedright{Note: This table shows the OLS estimates for attending Reggio Approach schools for people in Reggio Emilia who attended Reggio Approach preschools or no preschool at all. Column title indicates the age group and control set used in each regression corresponding to the column. ``None30'' refers to the regression with only age-30 cohort and with no control variables. ``BIC30'' refers to the regression with only age-30 cohort and with controls selected by Bayesian Information Criterion (BIC). ``Full30'' refers to the regression with only age-30 cohort and with the full set of controls. Analogous meanings applied to the age-40 cohort. Robust standard errors are reported in parentheses. Stars show statistical significance as follows. * p < 0.05, ** p < 0.01, *** p < 0.001.}
\end{table}

Table \ref{ols-L-reg} shows significant effects of the Reggio Approach on the status of being divorced for the age-30 cohort. Other results do not show any significant trend. 

\begin{table}[H] \caption{OLS Results for Living Environment, Municipal vs. None, Reggio Emilia} \label{ols-L-reg}
{
\def\sym#1{\ifmmode^{#1}\else\(^{#1}\)\fi}
\begin{tabular}{l*{6}{c}}
\toprule
            &\multicolumn{1}{c}{(1)}&\multicolumn{1}{c}{(2)}&\multicolumn{1}{c}{(3)}&\multicolumn{1}{c}{(4)}&\multicolumn{1}{c}{(5)}&\multicolumn{1}{c}{(6)}\\
            &\multicolumn{1}{c}{None30}&\multicolumn{1}{c}{BIC30}&\multicolumn{1}{c}{Full30}&\multicolumn{1}{c}{None40}&\multicolumn{1}{c}{BIC40}&\multicolumn{1}{c}{Full40}\\
\midrule
Married or Cohabitating&      0.0652         &     -0.0665         &     -0.0129         &     0.00781         &     -0.0164         &      0.0201         \\
            &    (0.0754)         &    (0.0810)         &    (0.0827)         &    (0.0618)         &    (0.0703)         &    (0.0965)         \\
\addlinespace
Divorced    &      0.0268\sym{*}  &      0.0409\sym{*}  &      0.0192         &     -0.0312         &     -0.0113         &      0.0373         \\
            &    (0.0133)         &    (0.0204)         &    (0.0130)         &    (0.0453)         &    (0.0478)         &    (0.0615)         \\
\addlinespace
Num. of Children in House&      0.0223         &      0.0172         &      0.0418         &      0.0563         &     -0.0530         &     -0.0707         \\
            &    (0.0519)         &    (0.0566)         &    (0.0603)         &    (0.0846)         &    (0.0979)         &     (0.106)         \\
\addlinespace
Own House   &    -0.00177         &      0.0830         &       0.153         &     -0.0859         &     -0.0160         &     -0.0294         \\
            &    (0.0773)         &    (0.0845)         &    (0.0899)         &    (0.0642)         &    (0.0707)         &    (0.0812)         \\
\addlinespace
Live With Parents&     -0.0613         &      -0.102         &     -0.0876         &     -0.0266         &     -0.0375         &     -0.0413         \\
            &    (0.0570)         &    (0.0531)         &    (0.0544)         &    (0.0279)         &    (0.0254)         &    (0.0297)         \\
\bottomrule
\end{tabular}
}

\vspace{1ex} \\
\footnotesize\raggedright{Note: This table shows the OLS estimates for attending Reggio Approach schools for people in Reggio Emilia who attended Reggio Approach preschools or no preschool at all. Column title indicates the age group and control set used in each regression corresponding to the column. ``None30'' refers to the regression with only age-30 cohort and with no control variables. ``BIC30'' refers to the regression with only age-30 cohort and with controls selected by Bayesian Information Criterion (BIC). ``Full30'' refers to the regression with only age-30 cohort and with the full set of controls. Analogous meanings applied to the age-40 cohort. Robust standard errors are reported in parentheses. Stars show statistical significance as follows. * p < 0.05, ** p < 0.01, *** p < 0.001.}
\end{table}

Table \ref{ols-H-reg} show that the significantly increasing effects of the Reggio Approach on numbers of days sick for the past month at the time of the interview for the age-30 cohort. Moreover, the results show the significant decreasing effects on whether the respondent has ever been suspended from school. 

\begin{table}[H] \caption{OLS Results for Health, Municipal vs. None, Reggio Emilia} \label{ols-H-reg}
{
\def\sym#1{\ifmmode^{#1}\else\(^{#1}\)\fi}
\begin{tabular}{l*{2}{c}}
\hline\hline
            &\multicolumn{1}{c}{(1)}&\multicolumn{1}{c}{(2)}\\
            &\multicolumn{1}{c}{Muni30}&\multicolumn{1}{c}{Muni40}\\
\hline
Tried Marijuana&      0.0916         &      0.0864         \\
            &    (0.0563)         &    (0.0507)         \\
[1em]
Smokes      &           0         &           0         \\
            &         (.)         &         (.)         \\
[1em]
Num. of Cigarettes Per Day&       2.431         &      -0.372         \\
            &     (1.348)         &     (1.666)         \\
[1em]
BMI         &       0.620         &      -0.220         \\
            &     (0.412)         &     (0.482)         \\
[1em]
Good Health &       0.129         &       0.126         \\
            &    (0.0775)         &     (0.103)         \\
[1em]
Num. of Days Sick Past Month&       0.285\sym{***}&      0.0400         \\
            &    (0.0848)         &    (0.0872)         \\
[1em]
Engaged in A Fight&           0         &           0         \\
            &         (.)         &         (.)         \\
[1em]
Drove Under Influence&           0         &           0         \\
            &         (.)         &         (.)         \\
[1em]
Ever Suspended from School&      -0.141\sym{**} &     -0.0159         \\
            &    (0.0521)         &    (0.0545)         \\
[1em]
Age At First Drink&      -0.518         &      -0.367         \\
            &     (1.238)         &     (1.396)         \\
\hline\hline
\multicolumn{3}{l}{\footnotesize \specialcell{\underline{Note:} This table shows the OLS estimates for people in Reggio who attended municipal preschools or none. \\                                                                         Standard errors are reported in parenthesis. Stars show statistical significance as follows: \\                                                                         * p < 0.05, ** p < 0.01, *** p < 0.001.}}\\
\end{tabular}
}

\vspace{1ex} \\
\footnotesize\raggedright{Note: This table shows the OLS estimates for attending Reggio Approach schools for people in Reggio Emilia who attended Reggio Approach preschools or no preschool at all. Column title indicates the age group and control set used in each regression corresponding to the column. ``None30'' refers to the regression with only age-30 cohort and with no control variables. ``BIC30'' refers to the regression with only age-30 cohort and with controls selected by Baysian Information Criterion (BIC). ``Full30'' refers to the regression with only age-30 cohort and with the full set of controls. Analogous meanings applied to the age-40 cohort. Robust standard errors are reported in parentheses. Stars show statistical significance as follows. * p < 0.05, ** p < 0.01, *** p < 0.001.}
\end{table}

Table \ref{ols-N-reg} shows no consistently significant trends between the age-30 cohort and age-40 cohort. For the age-30 cohort, there are significantly decreasing effects on the optimistic look in life. Moreover, there are significantly increasing effects on whether the respondent is likely to do the same to someone who puts him/her in a difficult situation and to someone who insults him/her.

For the age-40 cohort, results show the significantly decreasing effects on depression for the age-30 cohort. There are also significantly positive effects on satisfaction with income and with work for the age-40 cohort. However, there are significantly positive effects on whether the respondent thinks work is source of stress. 

\begin{table}[H] \caption{OLS Results for Non-cognitive, Municipal vs. None, Reggio Emilia} \label{ols-N-reg}
{
\def\sym#1{\ifmmode^{#1}\else\(^{#1}\)\fi}
\begin{tabular}{l*{6}{c}}
\toprule
            &\multicolumn{1}{c}{(1)}&\multicolumn{1}{c}{(2)}&\multicolumn{1}{c}{(3)}&\multicolumn{1}{c}{(4)}&\multicolumn{1}{c}{(5)}&\multicolumn{1}{c}{(6)}\\
            &\multicolumn{1}{c}{None30}&\multicolumn{1}{c}{BIC30}&\multicolumn{1}{c}{Full30}&\multicolumn{1}{c}{None40}&\multicolumn{1}{c}{BIC40}&\multicolumn{1}{c}{Full40}\\
\midrule
Locus of Control - positive&      0.0713         &      -0.133         &      -0.122         &       0.171         &       0.162         &       0.199         \\
            &     (0.127)         &     (0.125)         &     (0.127)         &     (0.119)         &     (0.137)         &     (0.165)         \\
\addlinespace
Depression Score - positive&       1.337         &      -0.678         &      -0.251         &       2.399\sym{**} &       1.848         &       2.225\sym{*}  \\
            &     (0.920)         &     (0.849)         &     (0.883)         &     (0.872)         &     (0.962)         &     (1.085)         \\
\addlinespace
Stress      &       0.184         &       0.102         &      0.0645         &       0.255\sym{*}  &       0.201         &       0.134         \\
            &     (0.111)         &     (0.108)         &     (0.126)         &     (0.104)         &     (0.112)         &     (0.150)         \\
\addlinespace
Work is Source of Stress&      0.0933         &      0.0639         &      0.0222         &       0.162         &       0.179\sym{*}  &       0.249\sym{*}  \\
            &    (0.0994)         &     (0.102)         &    (0.0983)         &    (0.0857)         &    (0.0882)         &     (0.101)         \\
\addlinespace
Satisfied with Income&       0.240         &       0.272         &       0.221         &       0.269\sym{*}  &       0.300\sym{*}  &       0.306         \\
            &     (0.142)         &     (0.153)         &     (0.152)         &     (0.128)         &     (0.147)         &     (0.161)         \\
\addlinespace
Satisfied with Work&       0.122         &       0.126         &       0.125         &       0.353\sym{**} &       0.315\sym{**} &       0.248         \\
            &     (0.138)         &     (0.149)         &     (0.146)         &     (0.114)         &     (0.122)         &     (0.145)         \\
\addlinespace
Satisfied with Health&     -0.0871         &      -0.180         &      -0.191         &      0.0578         &      0.0169         &    0.000877         \\
            &     (0.108)         &     (0.110)         &     (0.135)         &    (0.0700)         &    (0.0857)         &     (0.102)         \\
\addlinespace
Satisfied with Family&      0.0616         &     -0.0277         &      0.0122         &       0.227\sym{*}  &       0.142         &      0.0969         \\
            &     (0.134)         &     (0.143)         &     (0.148)         &     (0.116)         &     (0.137)         &     (0.157)         \\
\addlinespace
Optimistic Look in Life&      -0.173\sym{*}  &      -0.197\sym{*}  &      -0.127         &     -0.0632         &    -0.00647         &       0.105         \\
            &    (0.0746)         &    (0.0771)         &    (0.0874)         &    (0.0716)         &    (0.0816)         &     (0.108)         \\
\addlinespace
Return Favor&      0.0610         &     -0.0535         &     -0.0901         &      0.0891         &     0.00774         &     -0.0505         \\
            &     (0.159)         &     (0.154)         &     (0.192)         &     (0.128)         &     (0.149)         &     (0.195)         \\
\addlinespace
Put Someone in Difficulty&       0.615\sym{***}&       0.688\sym{***}&       0.642\sym{**} &      0.0109         &      -0.119         &       0.154         \\
            &     (0.178)         &     (0.191)         &     (0.205)         &     (0.163)         &     (0.191)         &     (0.213)         \\
\addlinespace
Help Someone Kind To Me&     -0.0164         &     -0.0848         &      -0.146         &       0.102         &      0.0696         &     0.00196         \\
            &     (0.111)         &     (0.110)         &     (0.123)         &    (0.0940)         &    (0.0977)         &     (0.124)         \\
\addlinespace
Insult Back &       0.387\sym{*}  &       0.524\sym{**} &       0.719\sym{***}&      -0.369\sym{*}  &      -0.272         &     -0.0920         \\
            &     (0.176)         &     (0.169)         &     (0.169)         &     (0.158)         &     (0.164)         &     (0.209)         \\
\bottomrule
\end{tabular}
}

\vspace{1ex} \\
\footnotesize\raggedright{Note: This table shows the OLS estimates for attending Reggio Approach schools for people in Reggio Emilia who attended Reggio Approach preschools or no preschool at all. Column title indicates the age group and control set used in each regression corresponding to the column. ``None30'' refers to the regression with only age-30 cohort and with no control variables. ``BIC30'' refers to the regression with only age-30 cohort and with controls selected by Bayesian Information Criterion (BIC). ``Full30'' refers to the regression with only age-30 cohort and with the full set of controls. Analogous meanings applied to the age-40 cohort. Robust standard errors are reported in parentheses. Stars show statistical significance as follows. * p < 0.05, ** p < 0.01, *** p < 0.001.}
\end{table}

Table \ref{ols-S-reg} shows consistently significant and negative effects of the Reggio Approach on whether the respondent does volunteer work for both the age-30 and age-40 cohorts. Moreover, for the age-40 cohort, there are significantly negative effects on respondents having migrant friends and significantly positive effects on whether respondents have ever voted for municipal and regional elections. For the age-30 cohort, there are significantly negative effects on whether respondents have ever voted for national elections.

\begin{table}[H] \caption{OLS Results for Social Behavior, Municipal vs. None, Reggio Emilia} \label{ols-S-reg}
{
\def\sym#1{\ifmmode^{#1}\else\(^{#1}\)\fi}
\begin{tabular}{l*{6}{c}}
\toprule
            &\multicolumn{1}{c}{(1)}&\multicolumn{1}{c}{(2)}&\multicolumn{1}{c}{(3)}&\multicolumn{1}{c}{(4)}&\multicolumn{1}{c}{(5)}&\multicolumn{1}{c}{(6)}\\
            &\multicolumn{1}{c}{None30}&\multicolumn{1}{c}{BIC30}&\multicolumn{1}{c}{Full30}&\multicolumn{1}{c}{None40}&\multicolumn{1}{c}{BIC40}&\multicolumn{1}{c}{Full40}\\
\midrule
Favorable to Migrants&      -0.070         &      -0.062         &       -0.12         &     -0.0078         &       0.027         &      -0.013         \\
            &      (0.08)         &      (0.09)         &      (0.10)         &      (0.09)         &      (0.09)         &      (0.12)         \\
\addlinespace
Number of Friends&       -1.46         &       -1.41         &       -1.53         &       -2.07\sym{*}  &       -0.82         &        0.13         \\
            &      (1.36)         &      (1.66)         &      (1.55)         &      (0.92)         &      (0.94)         &      (1.47)         \\
\addlinespace
Has Migrant Friends&       0.041         &       0.044         &     -0.0055         &       -0.13\sym{*}  &       -0.11         &       -0.18\sym{*}  \\
            &      (0.07)         &      (0.07)         &      (0.08)         &      (0.06)         &      (0.07)         &      (0.09)         \\
\addlinespace
Volunteers  &       -0.15\sym{**} &       -0.15\sym{**} &       -0.16\sym{**} &       -0.15\sym{**} &      -0.088         &       -0.21\sym{**} \\
            &      (0.06)         &      (0.05)         &      (0.06)         &      (0.05)         &      (0.06)         &      (0.08)         \\
\addlinespace
Child Eats Meal with Fam&        0.33         &      -0.096         &       0.011         &       0.095         &       0.078         &        0.13         \\
            &      (0.21)         &      (0.17)         &      (0.25)         &      (0.14)         &      (0.15)         &      (0.20)         \\
\addlinespace
Ever Voted for Municipal&        0.27\sym{***}&        0.11         &        0.11         &        0.30\sym{***}&       0.099         &        0.18\sym{*}  \\
            &      (0.08)         &      (0.06)         &      (0.07)         &      (0.07)         &      (0.08)         &      (0.08)         \\
\addlinespace
Ever Voted for Regional&        0.21\sym{**} &       0.099         &       0.088         &        0.30\sym{***}&        0.12         &        0.23\sym{**} \\
            &      (0.08)         &      (0.07)         &      (0.08)         &      (0.07)         &      (0.08)         &      (0.09)         \\
\addlinespace
Ever Voted for National&       -0.10\sym{*}  &       -0.13\sym{**} &      -0.090         &       0.094         &       0.030         &       0.078         \\
            &      (0.05)         &      (0.05)         &      (0.05)         &      (0.06)         &      (0.06)         &      (0.08)         \\
\bottomrule
\end{tabular}
}

\vspace{1ex} \\
\footnotesize\raggedright{Note: This table shows the OLS estimates for attending Reggio Approach schools for people in Reggio Emilia who attended Reggio Approach preschools or no preschool at all. Column title indicates the age group and control set used in each regression corresponding to the column. ``None30'' refers to the regression with only age-30 cohort and with no control variables. ``BIC30'' refers to the regression with only age-30 cohort and with controls selected by Bayesian Information Criterion (BIC). ``Full30'' refers to the regression with only age-30 cohort and with the full set of controls. Analogous meanings applied to the age-40 cohort. Robust standard errors are reported in parentheses. Stars show statistical significance as follows. * p < 0.05, ** p < 0.01, *** p < 0.001.}
\end{table}


