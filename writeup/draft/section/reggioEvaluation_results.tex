We present the estimates of the methods described above for a handful of key outcomes.\footnote{We choose outcomes that are economically significant,  outcomes that have limited missing values, and outcomes with sufficient variation across individuals. Results on more outcomes are reported in Appendix \ref{appsec:extended-outcome}. \textbf{[JJH: All outcomes. We need to summarize in text what we find in the appendix.]}}\footnote{A brief description of the outcomes is as follows: We rescale noncognitive outcomes, including SDQ (Strengths and Difficulties Questionnaire) score, Locus of Control, and Depression score, so that the higher value has a more socially positive meaning; \textbf{SDQ Composite - Child} is reported by mother, and \textbf{SDQ Composite} is self-reported; \textbf{IQ Score} is measured using Raven's Progressive Matrices; \textbf{How Much Child Likes School} is a single question with three answers, where 1 means ``A little", 2 means ``So so", and 3 means ``A lot";  \textbf{High School Grade} has the maximum scoring of 100; since the mean and variance is not always the same, we standardize the high school grade for each city, cohort, and high school type based on our data to have mean zero and unit variance; All the other measures reported in the estimation results are binary indicators.} Overall, very few outcomes show statistically significant treatment effects that are robust across different estimation procedures for the Reggio Approach. The strongest results are shown when we compare the Reggio Approach with no preschool for the age-40 cohort. Accounting for multiple hypotheses testing using the step down procedure of \citet{Romano_Wolf_2005_E}, step down $p$-values are reported in appendix \ref{appsec:extended-outcome-sd}. We note that after the step down procedure is applied, there is a loss of statistical significance. The remainder of the paper focuses on non-step down results.

In the child cohort, as shown in Table \ref{ols-M-child-reg-pres}, the Reggio Approach increased the positive-SDQ (Strengths and Difficulties Questionnaire) scores when compared to children who attended other preschools (OLS) \footnote{The SDQ is a widely-used scale inquiring about emotional symptoms, conduct problems, hyperactivity/inattention, peer relationships problems, and pro-social behavior \citep{Goodman_1997_JCPP}. For ease of interpretation, we have converted the SDQ score such that higher values correspond to more positive outcomes.}. This result becomes more positive after controlling for more background characteristics, or when comparing with children in Padova.\footnote{The estimated coefficient is still positive, but not statistically significant, when comparing with children in Parma (DiD).} Using PSM and AIPW to address issues of selection provides similar results as those of the OLS specification. The children are slightly more likely to be overweight and have worse health, although these estimates are not statistically significant for all specifications. The other main outcomes do not show significant effects.

In the adolescent cohort, shown in Table \ref{ols-M-adol-reg-pres}, adolescents who attended the Reggio Approach are significantly less likely to be depressed according to within Reggio Emilia analyses and DiD estimates with Parma and Padova. However, the propensity score matching with Parma and Padova adolescents do not show significant effect on depression. The Reggio Approach individuals are more likely to be obese than individuals who attended other types of preschool, and the estimate on obesity is consistent across most of the methods. Some methods show that Reggio Approach individuals are less likely to be involved in sport activities, which is consistent with the increase in obesity.

In the adult cohorts (age-40), there are differences when we compare Reggio Approach individuals to other preschool group and to no preschool group. The comparison with no preschool group, shown in tables \ref{ols-M-adult30-reg-nopres} and \ref{ols-M-adult40-reg-nopres}, shows many more statistically significant estimates. In the comparison with the other preschool group, shown in tables \ref{ols-M-adult30-reg-pres} and \ref{ols-M-adult40-reg-pres}, the only outcomes that shows some significance across different methods is high school graduation and voting behavior in the age-40 cohort. OLS estimate shows that the Reggio Approach age-40 individuals are more likely to graduate from high school than other groups within Reggio Emilia. Propensity score matching estimates show that the Reggio Approach age-40 individuals are more likely to have voted for regional election than people in Parma and Padova who attended preschool.

In the comparison with no preschool group, Reggio Approach individuals in age-40 group are more likely to be employed. Moreover, Reggio Approach people are significantly more likely to work more hours than other groups for both the age-30 and age-40 groups. OLS estimates show that the Reggio Approach people are significantly more likely to be obese when compared to no preschool group in Reggio Emilia, which aligns with the adolescent results. The age-40 Reggio Approach individuals have significant and positive effects on locus of control, the depression score, and voting behaviors across many methodologies.

We refer the reader to Appendix \ref{sec:results} for comparisons of the Reggio Approach with individual school types. We note that the estimated effects of the Reggio Approach on outcomes from these disaggregated comparisons vary depending on the comparison group being used as different school types exhibited different degrees of similarity to the Reggio Approach at different points of time (as evidenced by the results of our survey reported in Appendix \ref{sec:survey}). The disaggregated nature of these comparisons makes the results suitable to juxtaposition against the survey results that highlight differences in program components between the different types of schools. \textbf{[JJH: ? Summarize this mass of results please -- this is too vague.]}

% ========================================================================= %
% CHILD COHORT


\begin{table}[H] \caption{Estimation Results for Main Outcomes, Comparison to Non-RA Preschools, Child Cohort} \label{ols-M-child-reg-pres}
\scalebox{0.7}{\begin{tabular}{l c c c c c c c}
\toprule
 & NoneIt & BICIt & FullIt & DidPmIt & DidPvIt & AIPWnoneIt & AIPWpresIt \\
\midrule
SDQ Composite - Child &      0.64 & \textbf{      1.03 } & \textbf{      1.18 } &      0.80 &     -0.33 & \textbf{     2.27} & \textbf{     1.02} \\
& (     0.47) & (     0.46) & (     0.45) & (     0.71) & (     0.58) & (     2.46) & (     0.42) \\
Obese &      0.00 &      0.02 &      0.04 &      0.00 &      0.06 &      0.07 &      0.02 \\
& (     0.05) & (     0.05) & (     0.05) & (     0.06) & (     0.06) & (     0.19) & (     0.04) \\
Overweight & \textbf{      0.05 } &      0.04 &      0.05 & \textbf{      0.10 } &     -0.04 &      0.07 & \textbf{     0.05} \\
& (     0.03) & (     0.03) & (     0.04) & (     0.06) & (     0.04) & (     0.20) & (     0.04) \\
Health is Good &     -0.04 &     -0.03 &     -0.01 &     -0.11 &      0.00 & \textbf{     0.41} &     -0.00 \\
& (     0.05) & (     0.05) & (     0.05) & (     0.08) & (     0.05) & (     0.20) & (     0.03) \\
Not Excited to Learn &     -0.00 &      0.00 &     -0.00 &     -0.03 &      0.03 &     -0.08 &     -0.01 \\
& (     0.02) & (     0.02) & (     0.02) & (     0.03) & (     0.03) & (     0.19) & (     0.02) \\
Problems Sitting Still &      0.03 &      0.02 &      0.00 & \textbf{      0.10 } &      0.06 & \textbf{     0.14} &      0.01 \\
& (     0.03) & (     0.03) & (     0.03) & (     0.06) & (     0.04) & (     0.03) & (     0.03) \\
How Much Child Likes School &      0.01 &      0.03 &      0.03 &     -0.06 &     -0.07 & \textbf{     0.56} &      0.04 \\
& (     0.06) & (     0.06) & (     0.06) & (     0.08) & (     0.07) & (     0.32) & (     0.06) \\
\bottomrule
\end{tabular}
}
\vspace{1ex} \\
\footnotesize\raggedright{Note: This table shows the estimates of the coefficient for attending Reggio Approach preschools from multiple methods. We compare Reggio Approach individuals with those who attended other preschools. Column title indicates the corresponding control set and and model. \textbf{None} = OLS estimate with no control variables. \textbf{BIC} = OLS estimate with controls selected by Bayesian Information Criterion (BIC) and additional controls for male indicator, migrant indicator, and ITC attendance indicator. \textbf{Full} = OLS estimate with the full set of controls. \textbf{PSM} =  propensity score matching estimation. \textbf{AIPW} = augmented inverse propensity weighting estimation. \textbf{DidPm} = difference-in-difference estimate of (Reggio Muni - Parma Muni) - (Reggio Other - Parma Other). \textbf{PSMPm} = propensity score matching between Reggio Approach people and people who attended Parma preschools. \textbf{DidPv} = difference-in-difference estimate of (Reggio Muni - Padova Muni) - (Reggio Other - Padova Other). \textbf{PSMPv} = propensity score matching between Reggio Approach people and people who attended Padova preschools. Robust standard errors are reported in parentheses. Bold number shows that the estimate is statistically significant at the 10\% level \textbf{[JJH: Use 10\% level and 5\% as well. 15\%.]}. Number of observations used in estimation is reported in italic.}

\end{table}

\textbf{[JJH: What cohort label? Compare with OLS as well. Also report results for multiple hypothesis testing in each cohort.]}

\textbf{[JJH: We really need to correct for multiple hypothesis testing -- step down or Bonferroni.]}

\textbf{[JJH: Label cohorts!]}

\begin{table}[H] \caption{Estimation Results for Main Outcomes, Comparison to Non-RA Preschools, Adolescent Cohort} \label{ols-M-adol-reg-pres}
\scalebox{0.66}{\begin{tabular}{l c c c c c c}
\toprule
 & None & Bic & Full & DidPm & DidPv & AIPW \\
\midrule
SDQ Composite - Child &      0.00 &      0.02 &      0.35 &     -0.32 &      0.44 &      0.32 \\
& (     0.58 ) & (     0.57 ) & (     0.61 ) & (     0.76 ) & (     0.48 ) & (     0.55 ) \\
SDQ Composite &      0.71 &      0.52 &      0.56 &     -0.56 &     -0.09 &      0.20 \\
& (     0.62 ) & (     0.64 ) & (     0.70 ) & (     0.77 ) & (     0.64 ) & (     0.58 ) \\
Depression Score - positive & \textbf{      1.13 } &      1.16 & \textbf{      1.43 } & \textbf{     -1.40 } &     -0.54 & \textbf{     0.77} \\
& (     0.77 ) & (     0.83 ) & (     0.89 ) & (     0.82 ) & (     0.74 ) & (     0.80 ) \\
Obese & \textbf{      0.07 } & \textbf{      0.08 } & \textbf{      0.08 } &      0.07 &     -0.05 & \textbf{     0.08} \\
& (     0.04 ) & (     0.04 ) & (     0.04 ) & (     0.05 ) & (     0.05 ) & (     0.04 ) \\
Overweight &     -0.01 &     -0.01 &     -0.01 &      0.05 &     -0.02 &     -0.00 \\
& (     0.02 ) & (     0.02 ) & (     0.03 ) & (     0.04 ) & (     0.01 ) & (     0.02 ) \\
Health is Good &      0.05 & \textbf{      0.08 } &      0.08 &      0.00 &     -0.03 & \textbf{     0.08} \\
& (     0.06 ) & (     0.06 ) & (     0.06 ) & (     0.07 ) & (     0.06 ) & (     0.06 ) \\
Go To School & \textbf{      0.04 } & \textbf{      0.04 } & \textbf{      0.05 } & \textbf{      0.03 } &     -0.01 & \textbf{     0.05} \\
& (     0.02 ) & (     0.03 ) & (     0.03 ) & (     0.01 ) & (     0.02 ) & (     0.03 ) \\
How Much Child Likes School &     -0.13 & \textbf{     -0.18 } & \textbf{     -0.19 } & \textbf{     -0.23 } &     -0.08 &     -0.10 \\
& (     0.11 ) & (     0.12 ) & (     0.12 ) & (     0.15 ) & (     0.11 ) & (     0.12 ) \\
Trust Score &     -0.00 &     -0.10 &     -0.02 &     -0.31 &      0.07 &     -0.08 \\
& (     0.18 ) & (     0.18 ) & (     0.19 ) & (     0.23 ) & (     0.18 ) & (     0.16 ) \\
Days of Sport (Weekly) & \textbf{     -0.40 } &     -0.31 &     -0.29 &      0.35 &      0.20 &     -0.33 \\
& (     0.23 ) & (     0.24 ) & (     0.26 ) & (     0.28 ) & (     0.24 ) & (     0.24 ) \\
\bottomrule
\end{tabular}
}
\vspace{1ex} \\
\footnotesize\raggedright{Note: This table shows the estimates of the coefficient for attending Reggio Approach preschools from multiple methods. We compare Reggio Approach individuals with those who attended other preschools. Column title indicates the corresponding control set and and model. \textbf{None} = OLS estimate with no control variables. \textbf{BIC} = OLS estimate with controls selected by Bayesian Information Criterion (BIC) and additional controls for male indicator and ITC attendance indicator. \textbf{Full} = OLS estimate with the full set of controls. \textbf{PSM} =  propensity score matching estimation. \textbf{AIPW} = augmented inverse propensity weighting estimation. \textbf{DidPm} = difference-in-difference estimate of (Reggio Muni - Parma Muni) - (Reggio Other - Parma Other). \textbf{PSMPm} = propensity score matching between Reggio Approach people and people who attended Parma preschools. \textbf{DidPv} = difference-in-difference estimate of (Reggio Muni - Padova Muni) - (Reggio Other - Padova Other). \textbf{PSMPv} = propensity score matching between Reggio Approach people and people who attended Padova preschools. Robust standard errors are reported in parentheses. Bold number shows that the estimate is statistically significant at the 15\% level. Number of observations used in estimation is reported in italic.}
\end{table}




\begin{table}[H] \caption{Estimation Results for Main Outcomes, Comparison to Non-RA Preschools, Adult-30 Cohorts} \label{ols-M-adult30-reg-pres}
\scalebox{0.65}{\begin{tabular}{l c c c c c c}
\toprule
 & None & BIC & Full & AIPW & DidPm & DidPv \\
\midrule
IQ Score &     -0.04 &     -0.07 &     -0.05 &     -0.07 & \textbf{     -0.16 } &     -0.00 \\
& (     0.06 ) & (     0.05 ) & (     0.05 ) & (     0.06 ) & (     0.06 ) & (     0.08 ) \\
& \textit{ 153 } & \textit{ 153 } & \textit{ 153 } & \textit{ 153 } & \textit{ 299 } & \textit{ 326 } \\
IQ Factor &     -0.15 & \textbf{     -0.24 } &     -0.17 &     -0.22 & \textbf{     -0.50 } &     -0.08 \\
& (     0.16 ) & (     0.15 ) & (     0.15 ) & (     0.14 ) & (     0.18 ) & (     0.24 ) \\
& \textit{ 153 } & \textit{ 153 } & \textit{ 153 } & \textit{ 153 } & \textit{ 299 } & \textit{ 326 } \\
Graduate from High School &     -0.04 &     -0.03 &     -0.05 &     -0.06 & \textbf{      0.14 } & \textbf{     -0.11 } \\
& (     0.05 ) & (     0.05 ) & (     0.05 ) & (     0.04 ) & (     0.08 ) & (     0.07 ) \\
& \textit{ 153 } & \textit{ 153 } & \textit{ 153 } & \textit{ 153 } & \textit{ 299 } & \textit{ 326 } \\
High School Grade &      2.23 &      1.96 &      2.10 & \textbf{     2.26} &      2.57 &      2.28 \\
& (     1.55 ) & (     1.49 ) & (     1.58 ) & (     1.62 ) & (     3.34 ) & (     3.79 ) \\
& \textit{ 117 } & \textit{ 117 } & \textit{ 117 } & \textit{ 117 } & \textit{ 246 } & \textit{ 253 } \\
Max Edu: University &      0.04 &      0.04 &      0.03 &      0.02 & \textbf{      0.22 } & \textbf{      0.22 } \\
& (     0.07 ) & (     0.07 ) & (     0.07 ) & (     0.07 ) & (     0.10 ) & (     0.14 ) \\
& \textit{ 153 } & \textit{ 153 } & \textit{ 153 } & \textit{ 153 } & \textit{ 299 } & \textit{ 326 } \\
Employed &     -0.02 &     -0.02 &     -0.01 &     -0.02 &     -0.01 &     -0.04 \\
& (     0.03 ) & (     0.04 ) & (     0.04 ) & (     0.04 ) & (     0.07 ) & (     0.09 ) \\
& \textit{ 153 } & \textit{ 153 } & \textit{ 153 } & \textit{ 153 } & \textit{ 299 } & \textit{ 326 } \\
Hours Worked Per Week &      1.13 &      0.40 &      1.22 &      0.53 &      0.18 &      0.65 \\
& (     1.90 ) & (     2.04 ) & (     2.08 ) & (     2.22 ) & (     3.48 ) & (     3.97 ) \\
& \textit{ 123 } & \textit{ 123 } & \textit{ 123 } & \textit{ 123 } & \textit{ 266 } & \textit{ 292 } \\
Married or Cohabitating &      0.08 &      0.04 &      0.04 &      0.02 &      0.07 &      0.19 \\
& (     0.08 ) & (     0.08 ) & (     0.08 ) & (     0.09 ) & (     0.11 ) & (     0.14 ) \\
& \textit{ 153 } & \textit{ 153 } & \textit{ 153 } & \textit{ 153 } & \textit{ 299 } & \textit{ 326 } \\
Obese &     -0.02 &      0.04 &      0.00 &      0.02 &      0.09 &      0.05 \\
& (     0.07 ) & (     0.06 ) & (     0.06 ) & (     0.07 ) & (     0.09 ) & (     0.12 ) \\
& \textit{ 153 } & \textit{ 153 } & \textit{ 153 } & \textit{ 153 } & \textit{ 299 } & \textit{ 326 } \\
Overweight &      0.04 &     -0.01 &     -0.01 &     -0.01 &     -0.14 &      0.01 \\
& (     0.07 ) & (     0.06 ) & (     0.06 ) & (     0.06 ) & (     0.10 ) & (     0.11 ) \\
& \textit{ 153 } & \textit{ 153 } & \textit{ 153 } & \textit{ 153 } & \textit{ 299 } & \textit{ 326 } \\
Locus of Control - positive &      0.11 &      0.04 &      0.06 &      0.04 &     -0.11 &      0.24 \\
& (     0.12 ) & (     0.11 ) & (     0.12 ) & (     0.11 ) & (     0.22 ) & (     0.22 ) \\
& \textit{ 149 } & \textit{ 149 } & \textit{ 149 } & \textit{ 149 } & \textit{ 286 } & \textit{ 315 } \\
Depression Score - positive &      0.46 &     -0.53 &     -0.24 &     -0.63 &      0.17 &      0.45 \\
& (     1.05 ) & (     0.69 ) & (     0.67 ) & (     0.67 ) & (     1.25 ) & (     1.70 ) \\
& \textit{ 151 } & \textit{ 151 } & \textit{ 151 } & \textit{ 151 } & \textit{ 297 } & \textit{ 321 } \\
Ever Voted for Municipal &      0.02 &      0.00 &      0.01 &      0.01 &      0.06 & \textbf{      0.28 } \\
& (     0.08 ) & (     0.06 ) & (     0.07 ) & (     0.06 ) & (     0.08 ) & (     0.11 ) \\
& \textit{ 151 } & \textit{ 151 } & \textit{ 151 } & \textit{ 151 } & \textit{ 295 } & \textit{ 314 } \\
Ever Voted for Regional &     -0.01 &     -0.04 &     -0.03 &     -0.04 &      0.03 & \textbf{      0.34 } \\
& (     0.08 ) & (     0.07 ) & (     0.07 ) & (     0.07 ) & (     0.08 ) & (     0.11 ) \\
& \textit{ 151 } & \textit{ 151 } & \textit{ 151 } & \textit{ 151 } & \textit{ 295 } & \textit{ 314 } \\
\bottomrule
\end{tabular}
}
\vspace{1ex} \\
\footnotesize\raggedright{Note: This table shows the estimates of the coefficient for attending Reggio Approach preschools from multiple methods. We compare Reggio Approach individuals with those who attended other preschools. Column title indicates the corresponding control set and and model. \textbf{None} = OLS estimate with no control variables. \textbf{BIC} = OLS estimate with controls selected by Bayesian Information Criterion (BIC) and additional controls for male indicator and ITC attendance indicator. \textbf{Full} = OLS estimate with the full set of controls. \textbf{PSM} =  propensity score matching estimation. \textbf{AIPW} = augmented inverse propensity weighting estimation. \textbf{DidPm} = difference-in-difference estimate of (Reggio Muni - Parma Muni) - (Reggio Other - Parma Other). \textbf{PSMPm} = propensity score matching between Reggio Approach people and people who attended Parma preschools. \textbf{DidPv} = difference-in-difference estimate of (Reggio Muni - Padova Muni) - (Reggio Other - Padova Other). \textbf{PSMPv} = propensity score matching between Reggio Approach people and people who attended Padova preschools. Robust standard errors are reported in parentheses. Bold number shows that the estimate is statistically significant at the 15\% level. Number of observations used in estimation is reported in italic.}
\end{table}

\begin{table}[H] \caption{Estimation Results for Main Outcomes, Comparison to No Preschools, Age-30 Cohorts} \label{ols-M-adult30-reg-nopres}
\scalebox{0.65}{\begin{tabular}{l c c c c c c c c c}
\toprule
 & None & BIC & Full & PSM & AIPW & DidPm & PSMPm & DidPv & PSMPv \\
\midrule
IQ Factor & 0.14 & 0.03 & -0.05 & 0.15 & 0.05 & -0.24 & \textbf{-0.57} & -0.11 & \textbf{-0.28} \\
& (0.16) & (0.15) & (0.16) & (0.19) & (0.17) & (0.22) & (0.18) & (0.27) & (0.13) \\
& \textit{ 167 } & \textit{ 167 } & \textit{ 167 } & \textit{ 167 } & \textit{ 167 } & \textit{ 252 } & \textit{ 153 } & \textit{ 233 } & \textit{ 157 } \\
Graduate from High School & -0.03 & 0.02 & 0.03 & 0.03 & 0.03 & 0.12 & 0.00 & -0.05 & -0.01 \\
& (0.05) & (0.05) & (0.05) & (0.07) & (0.06) & (0.09) & (0.09) & (0.09) & (0.05) \\
& \textit{ 167 } & \textit{ 167 } & \textit{ 167 } & \textit{ 167 } & \textit{ 167 } & \textit{ 252 } & \textit{ 153 } & \textit{ 233 } & \textit{ 157 } \\
High School Grade & \textbf{ 4.54 } & \textbf{ 4.98 } & \textbf{ 4.62 } & \textbf{5.57} & \textbf{5.90} & 0.35 & \textbf{12.70} & 3.16 & \textbf{3.68} \\
& (2.01) & (2.13) & (2.26) & (1.98) & (2.49) & (4.46) & (2.56) & (4.19) & (2.19) \\
& \textit{ 123 } & \textit{ 123 } & \textit{ 123 } & \textit{ 123 } & \textit{ 123 } & \textit{ 194 } & \textit{ 118 } & \textit{ 176 } & \textit{ 118 } \\
High School Grade (Standardized) & \textbf{ 6.39 } & \textbf{ 6.88 } & \textbf{ 6.54 } & \textbf{6.91} & \textbf{7.63} & 4.50 & \textbf{4.87} & 6.16 & -0.79 \\
& (2.25) & (2.39) & (2.52) & (2.27) & (2.52) & (3.85) & (2.23) & (4.82) & (2.46) \\
& \textit{ 123 } & \textit{ 123 } & \textit{ 123 } & \textit{ 123 } & \textit{ 123 } & \textit{ 192 } & \textit{ 117 } & \textit{ 175 } & \textit{ 118 } \\
Max Edu: University & -0.07 & -0.03 & -0.04 & -0.02 & -0.01 & 0.01 & \textbf{-0.16} & -0.15 & 0.03 \\
& (0.07) & (0.07) & (0.07) & (0.08) & (0.07) & (0.12) & (0.08) & (0.15) & (0.07) \\
& \textit{ 167 } & \textit{ 167 } & \textit{ 167 } & \textit{ 167 } & \textit{ 167 } & \textit{ 252 } & \textit{ 153 } & \textit{ 233 } & \textit{ 157 } \\
Employed & 0.04 & 0.02 & 0.04 & 0.05 & 0.01 & \textbf{ 0.14 } & -0.02 & 0.03 & 0.08 \\
& (0.05) & (0.05) & (0.05) & (0.05) & (0.03) & (0.09) & (0.03) & (0.10) & (0.08) \\
& \textit{ 167 } & \textit{ 167 } & \textit{ 167 } & \textit{ 167 } & \textit{ 167 } & \textit{ 252 } & \textit{ 153 } & \textit{ 233 } & \textit{ 157 } \\
Hours Worked Per Week & \textbf{ 6.84 } & \textbf{ 4.30 } & \textbf{ 5.16 } & 2.80 & \textbf{3.57} & \textbf{ 9.35 } & 1.75 & 5.25 & 2.77 \\
& (2.73) & (2.76) & (2.80) & (2.94) & (2.69) & (4.39) & (3.52) & (4.97) & (3.14) \\
& \textit{ 140 } & \textit{ 140 } & \textit{ 140 } & \textit{ 140 } & \textit{ 140 } & \textit{ 223 } & \textit{ 134 } & \textit{ 206 } & \textit{ 138 } \\
Married or Cohabitating & -0.01 & -0.08 & -0.10 & -0.05 & -0.07 & -0.09 & 0.04 & -0.01 & -0.05 \\
& (0.08) & (0.08) & (0.08) & (0.09) & (0.09) & (0.13) & (0.11) & (0.16) & (0.10) \\
& \textit{ 167 } & \textit{ 167 } & \textit{ 167 } & \textit{ 167 } & \textit{ 167 } & \textit{ 252 } & \textit{ 153 } & \textit{ 233 } & \textit{ 157 } \\
Not Obese & -0.00 & -0.06 & -0.06 & -0.09 & -0.07 & 0.03 & \textbf{-0.24} & \textbf{ -0.25 } & 0.10 \\
& (0.07) & (0.06) & (0.06) & (0.07) & (0.07) & (0.11) & (0.05) & (0.14) & (0.09) \\
& \textit{ 167 } & \textit{ 167 } & \textit{ 167 } & \textit{ 167 } & \textit{ 167 } & \textit{ 252 } & \textit{ 153 } & \textit{ 233 } & \textit{ 157 } \\
Not Overweight & -0.07 & 0.01 & -0.02 & 0.03 & 0.01 & -0.00 & 0.15 & 0.00 & -0.07 \\
& (0.07) & (0.06) & (0.06) & (0.08) & (0.06) & (0.12) & (0.10) & (0.12) & (0.06) \\
& \textit{ 167 } & \textit{ 167 } & \textit{ 167 } & \textit{ 167 } & \textit{ 167 } & \textit{ 252 } & \textit{ 153 } & \textit{ 233 } & \textit{ 157 } \\
Locus of Control - positive & 0.07 & -0.05 & -0.08 & -0.11 & -0.03 & -0.15 & \textbf{0.63} & 0.04 & -0.02 \\
& (0.14) & (0.13) & (0.14) & (0.12) & (0.11) & (0.25) & (0.16) & (0.27) & (0.20) \\
& \textit{ 163 } & \textit{ 163 } & \textit{ 163 } & \textit{ 163 } & \textit{ 163 } & \textit{ 239 } & \textit{ 144 } & \textit{ 221 } & \textit{ 148 } \\
Depression Score - positive & 1.26 & -0.04 & -0.20 & 0.37 & 0.07 & 0.74 & -0.93 & -0.53 & -0.38 \\
& (0.97) & (0.85) & (0.91) & (0.97) & (0.95) & (1.57) & (1.59) & (1.90) & (1.00) \\
& \textit{ 165 } & \textit{ 165 } & \textit{ 165 } & \textit{ 165 } & \textit{ 165 } & \textit{ 250 } & \textit{ 152 } & \textit{ 230 } & \textit{ 156 } \\
Ever Voted for Municipal & 0.10 & 0.03 & 0.04 & -0.07 & 0.02 & -0.05 & \textbf{0.22} & -0.01 & \textbf{0.27} \\
& (0.08) & (0.06) & (0.06) & (0.09) & (0.05) & (0.09) & (0.09) & (0.13) & (0.09) \\
& \textit{ 164 } & \textit{ 164 } & \textit{ 164 } & \textit{ 164 } & \textit{ 164 } & \textit{ 248 } & \textit{ 152 } & \textit{ 223 } & \textit{ 152 } \\
Ever Voted for Regional & 0.05 & -0.02 & -0.01 & -0.09 & -0.02 & -0.05 & \textbf{0.23} & 0.06 & \textbf{0.22} \\
& (0.08) & (0.07) & (0.07) & (0.09) & (0.07) & (0.09) & (0.09) & (0.13) & (0.09) \\
& \textit{ 164 } & \textit{ 164 } & \textit{ 164 } & \textit{ 164 } & \textit{ 164 } & \textit{ 248 } & \textit{ 152 } & \textit{ 223 } & \textit{ 152 } \\
\bottomrule
\end{tabular}
}
\vspace{1ex} \\
\footnotesize\raggedright{Note: This table shows the estimates of the coefficient for attending Reggio Approach preschools from multiple methods. We compare Reggio Approach individuals with those who attended other preschools. Column title indicates the corresponding control set and and model. \textbf{None} = OLS estimate with no control variables. \textbf{BIC} = OLS estimate with controls selected by Bayesian Information Criterion (BIC) and additional controls for male indicator and ITC attendance indicator. \textbf{Full} = OLS estimate with the full set of controls. \textbf{PSM} =  propensity score matching estimation. \textbf{AIPW} = augmented inverse propensity weighting estimation. \textbf{DidPm} = difference-in-difference estimate of (Reggio Muni - Parma Muni) - (Reggio None - Parma None). \textbf{PSMPm} = propensity score matching between Reggio Approach people and Parma people who attended no preschool. \textbf{DidPv} = difference-in-difference estimate of (Reggio Muni - Padova Muni) - (Reggio None - Padova None). \textbf{PSMPv} = propensity score matching between Reggio Approach people and Padova people who attended no preschool. Robust standard errors are reported in parentheses. Bold number shows that the estimate is statistically significant at the 15\% level. Number of observations used in estimation is reported in italic.}
\end{table}

\textbf{[JJH: Do step down. Define comparisons more clearly.]}

\begin{table}[H] \caption{Estimation Results for Main Outcomes, Comparison to Non-RA Preschools, Age-40 Cohorts} \label{ols-M-adult40-reg-pres}
\scalebox{0.65}{\begin{tabular}{l c c c c c c c}
\toprule
 & None & BIC & Full & PSM & AIPW & PSMPm & PSMPv \\
\midrule
IQ Factor & -0.15 & -0.12 & -0.14 & -0.11 & -0.18 & \textbf{-0.30} & \textbf{-0.25} \\
& (0.12) & (0.11) & (0.11) & (0.12) & (0.10) & (0.12) & (0.14) \\
& \textit{ 159 } & \textit{ 159 } & \textit{ 159 } & \textit{ 159 } & \textit{ 159 } & \textit{ 197 } & \textit{ 239 } \\
Graduate from High School & \textbf{ 0.13 } & \textbf{ 0.10 } & \textbf{ 0.12 } & 0.09 & 0.02 & 0.05 & 0.05 \\
& (0.07) & (0.07) & (0.07) & (0.07) & (0.06) & (0.05) & (0.04) \\
& \textit{ 159 } & \textit{ 159 } & \textit{ 159 } & \textit{ 159 } & \textit{ 159 } & \textit{ 197 } & \textit{ 239 } \\
High School Grade & -0.66 & -0.09 & 0.36 & -0.84 & 0.53 & \textbf{3.74} & \textbf{5.91} \\
& (1.56) & (1.65) & (1.71) & (1.64) & (1.78) & (1.81) & (1.67) \\
& \textit{ 117 } & \textit{ 117 } & \textit{ 117 } & \textit{ 117 } & \textit{ 117 } & \textit{ 161 } & \textit{ 188 } \\
High School Grade (Standardized) & -1.13 & -0.17 & 0.36 & 0.74 & 0.65 & -1.50 & 1.65 \\
& (2.07) & (2.22) & (2.28) & (2.40) & (2.74) & (1.77) & (1.96) \\
& \textit{ 116 } & \textit{ 116 } & \textit{ 116 } & \textit{ 116 } & \textit{ 116 } & \textit{ 159 } & \textit{ 188 } \\
Max Edu: University & 0.07 & 0.05 & 0.03 & 0.01 & 0.04 & \textbf{-0.15} & \textbf{-0.12} \\
& (0.06) & (0.05) & (0.05) & (0.07) & (0.06) & (0.07) & (0.06) \\
& \textit{ 159 } & \textit{ 159 } & \textit{ 159 } & \textit{ 159 } & \textit{ 159 } & \textit{ 197 } & \textit{ 239 } \\
Employed & 0.01 & 0.01 & 0.01 & 0.03 & 0.03 & -0.00 & \textbf{0.07} \\
& (0.03) & (0.04) & (0.04) & (0.04) & (0.04) & (0.03) & (0.03) \\
& \textit{ 159 } & \textit{ 159 } & \textit{ 159 } & \textit{ 159 } & \textit{ 159 } & \textit{ 197 } & \textit{ 239 } \\
Hours Worked Per Week & -0.90 & -1.17 & -1.28 & -1.71 & -0.08 & 0.22 & \textbf{5.21} \\
& (1.93) & (2.12) & (2.20) & (1.92) & (1.65) & (1.73) & (1.77) \\
& \textit{ 144 } & \textit{ 144 } & \textit{ 144 } & \textit{ 144 } & \textit{ 144 } & \textit{ 179 } & \textit{ 226 } \\
Married or Cohabitating & 0.03 & 0.02 & 0.02 & 0.01 & 0.01 & 0.10 & \textbf{0.11} \\
& (0.07) & (0.07) & (0.07) & (0.07) & (0.07) & (0.07) & (0.07) \\
& \textit{ 159 } & \textit{ 159 } & \textit{ 159 } & \textit{ 159 } & \textit{ 159 } & \textit{ 197 } & \textit{ 239 } \\
Not Obese & -0.04 & 0.02 & 0.04 & 0.03 & 0.06 & -0.07 & -0.08 \\
& (0.07) & (0.07) & (0.07) & (0.08) & (0.06) & (0.07) & (0.07) \\
& \textit{ 159 } & \textit{ 159 } & \textit{ 159 } & \textit{ 159 } & \textit{ 159 } & \textit{ 197 } & \textit{ 239 } \\
Not Overweight & 0.05 & 0.03 & 0.03 & -0.01 & -0.04 & 0.01 & -0.03 \\
& (0.07) & (0.07) & (0.07) & (0.07) & (0.07) & (0.07) & (0.06) \\
& \textit{ 159 } & \textit{ 159 } & \textit{ 159 } & \textit{ 159 } & \textit{ 159 } & \textit{ 197 } & \textit{ 239 } \\
Locus of Control - positive & 0.13 & 0.14 & 0.11 & 0.19 & 0.07 & 0.16 & -0.00 \\
& (0.14) & (0.14) & (0.14) & (0.17) & (0.15) & (0.14) & (0.13) \\
& \textit{ 156 } & \textit{ 156 } & \textit{ 156 } & \textit{ 156 } & \textit{ 156 } & \textit{ 192 } & \textit{ 231 } \\
Depression Score - positive & 0.56 & \textbf{ 1.37 } & 1.09 & 1.28 & 1.00 & -0.71 & -0.31 \\
& (0.92) & (0.84) & (0.89) & (0.90) & (0.86) & (0.85) & (0.84) \\
& \textit{ 156 } & \textit{ 156 } & \textit{ 156 } & \textit{ 156 } & \textit{ 156 } & \textit{ 197 } & \textit{ 237 } \\
Ever Voted for Municipal & -0.07 & 0.07 & 0.06 & 0.09 & \textbf{0.10} & \textbf{0.14} & 0.07 \\
& (0.08) & (0.07) & (0.07) & (0.06) & (0.06) & (0.07) & (0.07) \\
& \textit{ 151 } & \textit{ 151 } & \textit{ 151 } & \textit{ 151 } & \textit{ 151 } & \textit{ 191 } & \textit{ 218 } \\
Ever Voted for Regional & -0.05 & 0.08 & 0.07 & 0.10 & \textbf{0.10} & \textbf{0.27} & \textbf{0.14} \\
& (0.08) & (0.07) & (0.07) & (0.07) & (0.08) & (0.07) & (0.07) \\
& \textit{ 151 } & \textit{ 151 } & \textit{ 151 } & \textit{ 151 } & \textit{ 151 } & \textit{ 191 } & \textit{ 218 } \\
\bottomrule
\end{tabular}
}
\vspace{1ex} \\
\footnotesize\raggedright{Note: This table shows the estimates of the coefficient for attending Reggio Approach preschools from multiple methods. We compare Reggio Approach individuals with those who attended other preschools. Column title indicates the corresponding control set and and model. \textbf{None} = OLS estimate with no control variables. \textbf{BIC} = OLS estimate with controls selected by Bayesian Information Criterion (BIC) and additional controls for male indicator and ITC attendance indicator. \textbf{Full} = OLS estimate with the full set of controls. \textbf{PSM} =  propensity score matching estimation. \textbf{AIPW} = augmented inverse propensity weighting estimation. \textbf{PSMPm} = propensity score matching between Reggio Approach people and Parma people who attended no preschool.  \textbf{PSMPv} = propensity score matching between Reggio Approach people and Padova people who attended no preschool. Robust standard errors are reported in parentheses. DiD estimates is not available for this cohort due to unavailability of municipal preschool systems in Parma and Padova. Bold number shows that the estimate is statistically significant at the 15\% level. Number of observations used in estimation is reported in italic.}
\end{table}

\begin{table}[H] \caption{Estimation Results for Main Outcomes, Comparison to No Preschools, Age-40 Cohorts} \label{ols-M-adult40-reg-nopres}
\scalebox{0.65}{\begin{tabular}{l c c c c c c}
\toprule
 & None & BIC & Full & AIPW & DidPm & DidPv \\
\midrule
IQ Score &     -0.03 &     -0.02 &     -0.01 &     -0.02 &      0.04 &     -0.04 \\
& (     0.05 ) & (     0.05 ) & (     0.06 ) & (     0.05 ) & (     0.06 ) & (     0.05 ) \\
& \textit{ 170 } & \textit{ 170 } & \textit{ 170 } & \textit{ 170 } & \textit{ 382 } & \textit{ 375 } \\
IQ Factor &      0.01 &      0.01 &      0.04 &      0.00 &      0.18 &      0.09 \\
& (     0.13 ) & (     0.14 ) & (     0.16 ) & (     0.13 ) & (     0.16 ) & (     0.15 ) \\
& \textit{ 170 } & \textit{ 170 } & \textit{ 170 } & \textit{ 170 } & \textit{ 382 } & \textit{ 375 } \\
Graduate from High School &     -0.07 &     -0.01 &     -0.06 &     -0.01 &      0.03 &     -0.09 \\
& (     0.05 ) & (     0.05 ) & (     0.06 ) & (     0.06 ) & (     0.07 ) & (     0.07 ) \\
& \textit{ 170 } & \textit{ 170 } & \textit{ 170 } & \textit{ 170 } & \textit{ 382 } & \textit{ 375 } \\
High School Grade &      0.59 &      1.65 &      1.77 &      1.51 & \textbf{     -4.02 } &      2.46 \\
& (     1.51 ) & (     1.59 ) & (     1.86 ) & (     1.66 ) & (     2.50 ) & (     2.33 ) \\
& \textit{ 135 } & \textit{ 135 } & \textit{ 135 } & \textit{ 135 } & \textit{ 311 } & \textit{ 297 } \\
Max Edu: University &      0.01 &      0.06 & \textbf{      0.11 } &      0.06 &     -0.11 &     -0.06 \\
& (     0.06 ) & (     0.06 ) & (     0.06 ) & (     0.07 ) & (     0.08 ) & (     0.09 ) \\
& \textit{ 170 } & \textit{ 170 } & \textit{ 170 } & \textit{ 170 } & \textit{ 382 } & \textit{ 375 } \\
Employed & \textbf{      0.06 } & \textbf{      0.08 } &      0.05 & \textbf{     0.08} &      0.05 & \textbf{      0.09 } \\
& (     0.04 ) & (     0.03 ) & (     0.03 ) & (     0.04 ) & (     0.05 ) & (     0.06 ) \\
& \textit{ 169 } & \textit{ 169 } & \textit{ 169 } & \textit{ 169 } & \textit{ 381 } & \textit{ 374 } \\
Hours Worked Per Week & \textbf{      5.71 } & \textbf{      7.29 } & \textbf{      7.39 } & \textbf{     7.24} & \textbf{      4.92 } & \textbf{      7.19 } \\
& (     2.42 ) & (     2.39 ) & (     2.60 ) & (     2.95 ) & (     2.80 ) & (     3.07 ) \\
& \textit{ 151 } & \textit{ 151 } & \textit{ 151 } & \textit{ 151 } & \textit{ 358 } & \textit{ 355 } \\
Married or Cohabitating &      0.02 &      0.01 &      0.05 &      0.04 &     -0.10 &     -0.13 \\
& (     0.07 ) & (     0.07 ) & (     0.08 ) & (     0.07 ) & (     0.09 ) & (     0.10 ) \\
& \textit{ 170 } & \textit{ 170 } & \textit{ 170 } & \textit{ 170 } & \textit{ 382 } & \textit{ 375 } \\
Obese & \textbf{     -0.14 } &     -0.08 &     -0.01 &     -0.07 & \textbf{     -0.26 } &     -0.03 \\
& (     0.07 ) & (     0.08 ) & (     0.08 ) & (     0.07 ) & (     0.09 ) & (     0.10 ) \\
& \textit{ 170 } & \textit{ 170 } & \textit{ 170 } & \textit{ 170 } & \textit{ 382 } & \textit{ 375 } \\
Overweight &      0.03 &     -0.04 &     -0.07 &     -0.06 &     -0.01 &      0.04 \\
& (     0.07 ) & (     0.07 ) & (     0.07 ) & (     0.07 ) & (     0.09 ) & (     0.08 ) \\
& \textit{ 170 } & \textit{ 170 } & \textit{ 170 } & \textit{ 170 } & \textit{ 382 } & \textit{ 375 } \\
Locus of Control - positive &      0.14 & \textbf{      0.20 } & \textbf{      0.28 } & \textbf{     0.27} &      0.12 & \textbf{      0.31 } \\
& (     0.13 ) & (     0.13 ) & (     0.14 ) & (     0.14 ) & (     0.17 ) & (     0.17 ) \\
& \textit{ 165 } & \textit{ 165 } & \textit{ 165 } & \textit{ 165 } & \textit{ 364 } & \textit{ 357 } \\
Depression Score - positive & \textbf{      2.25 } & \textbf{      2.23 } & \textbf{      2.10 } & \textbf{     2.47} &      0.20 & \textbf{      2.26 } \\
& (     0.92 ) & (     0.95 ) & (     1.07 ) & (     0.99 ) & (     1.13 ) & (     1.18 ) \\
& \textit{ 168 } & \textit{ 168 } & \textit{ 168 } & \textit{ 168 } & \textit{ 380 } & \textit{ 371 } \\
Ever Voted for Municipal & \textbf{      0.19 } & \textbf{      0.12 } &      0.11 & \textbf{     0.14} &     -0.02 &     -0.09 \\
& (     0.08 ) & (     0.08 ) & (     0.08 ) & (     0.09 ) & (     0.10 ) & (     0.10 ) \\
& \textit{ 153 } & \textit{ 153 } & \textit{ 153 } & \textit{ 153 } & \textit{ 365 } & \textit{ 340 } \\
Ever Voted for Regional & \textbf{      0.20 } & \textbf{      0.14 } & \textbf{      0.13 } & \textbf{     0.16} &      0.05 &     -0.10 \\
& (     0.08 ) & (     0.08 ) & (     0.08 ) & (     0.08 ) & (     0.09 ) & (     0.10 ) \\
& \textit{ 153 } & \textit{ 153 } & \textit{ 153 } & \textit{ 153 } & \textit{ 365 } & \textit{ 340 } \\
\bottomrule
\end{tabular}
}
\vspace{1ex} \\
\footnotesize\raggedright{Note: This table shows the estimates of the coefficient for attending Reggio Approach preschools from multiple methods. We compare Reggio Approach individuals with those who attended other preschools. Column title indicates the corresponding control set and and model. \textbf{None} = OLS estimate with no control variables. \textbf{BIC} = OLS estimate with controls selected by Bayesian Information Criterion (BIC) and additional controls for male indicator and ITC attendance indicator. \textbf{Full} = OLS estimate with the full set of controls. \textbf{PSM} =  propensity score matching estimation. \textbf{AIPW} = augmented inverse propensity weighting estimation. \textbf{DidPm} = difference-in-difference estimate of (Reggio Muni - Parma Other) - (Reggio None - Parma None). \textbf{PSMPm} = propensity score matching between Reggio Approach people and Parma people who attended no preschool. \textbf{DidPv} = difference-in-difference estimate of (Reggio Muni - Padova Other) - (Reggio None - Padova None). \textbf{PSMPv} = propensity score matching between Reggio Approach people and Padova people who attended no preschool. Robust standard errors are reported in parentheses. Bold number shows that the estimate is statistically significant at the 15\% level. Number of observations used in estimation is reported in italic.}
\end{table}

Tables \ref{ols-M-child-reg-nopres-asilo} to \ref{ols-M-adult40-reg-nopres-asilo} show the estimation results for Reggio Approach infant-toddler centers based on the previous section. 

For children cohort, Reggio Approach infant-toddler centers have significantly positive effect on IQ factor and obesity relative to no infant-toddler centers in Reggio. For adolescents cohort, Reggio Approach infant-toddler centers do not have clear affect relative to no infant-toddler centers. 

For adult 30s cohort, Reggio Approach infant-toddler centers have significantly negative effect on IQ score, IQ factor, high school grade, and locus of control. Moreover, Reggio Approach infant-toddler centers have significantly positive effect on hours worked per week, marriage, and obesity. For adult 40s cohort, Reggio Approach has significantly negative effect on IQ score, high school grades, and marriage. However, Reggio Approach is shown to have positive effects on obesity. 


\begin{table}[H] \caption{Estimation Results for Main Outcomes, Comparison to No Infant-Toddler Centers, Child Cohort} \label{ols-M-child-reg-nopres-asilo}
\scalebox{0.8}{\begin{tabular}{l c c c}
\toprule
 & NoneIt & BICIt & FullIt \\
\midrule
IQ Score &      0.05 & \textbf{      0.06 } &      0.03 \\
& (     0.03 ) & (     0.03 ) & (     0.04 ) \\
& \textit{ 228 } & \textit{ 228 } & \textit{ 228 } \\
IQ Factor & \textbf{      0.27 } & \textbf{      0.30 } &      0.19 \\
& (     0.14 ) & (     0.15 ) & (     0.15 ) \\
& \textit{ 228 } & \textit{ 228 } & \textit{ 228 } \\
SDQ Composite - Child &      0.17 &      0.31 &      0.72 \\
& (     0.63 ) & (     0.62 ) & (     0.71 ) \\
& \textit{ 228 } & \textit{ 228 } & \textit{ 228 } \\
Obese & \textbf{     -0.22 } & \textbf{     -0.23 } & \textbf{     -0.16 } \\
& (     0.07 ) & (     0.07 ) & (     0.08 ) \\
& \textit{ 228 } & \textit{ 228 } & \textit{ 228 } \\
Overweight &      0.01 &      0.02 &     -0.02 \\
& (     0.05 ) & (     0.05 ) & (     0.05 ) \\
& \textit{ 228 } & \textit{ 228 } & \textit{ 228 } \\
Health is Good &     -0.03 &     -0.06 &      0.02 \\
& (     0.07 ) & (     0.07 ) & (     0.08 ) \\
& \textit{ 228 } & \textit{ 228 } & \textit{ 228 } \\
Not Excited to Learn &     -0.01 &      0.01 &      0.01 \\
& (     0.03 ) & (     0.03 ) & (     0.03 ) \\
& \textit{ 228 } & \textit{ 228 } & \textit{ 228 } \\
Problems Sitting Still &     -0.07 &     -0.09 &     -0.09 \\
& (     0.06 ) & (     0.06 ) & (     0.06 ) \\
& \textit{ 228 } & \textit{ 228 } & \textit{ 228 } \\
How Much Child Likes School &      0.12 &      0.12 &      0.08 \\
& (     0.09 ) & (     0.09 ) & (     0.10 ) \\
& \textit{ 228 } & \textit{ 228 } & \textit{ 228 } \\
\bottomrule
\end{tabular}
}
\vspace{1ex} \\
\footnotesize\raggedright{Note: This table shows the estimates of the coefficient for attending Reggio Approach infant-toddler centers from multiple methods. We compare Reggio Approach people with people who attended no infant-toddler center. Column title indicates the corresponding control set and and model.  \textbf{None} = OLS estimate with no control variables. \textbf{BIC} = OLS estimate with controls selected by Bayesian Information Criterion (BIC) and additional controls for male indicator and ITC attendance indicator. \textbf{Full} = OLS estimate with the full set of controls. \textbf{PSM} =  propensity score matching estimation. Bold number shows that the estimate is statistically significant at the 15\% level. Number of observations used in estimation is reported in italic.}

\end{table}

\begin{table}[H] \caption{Estimation Results for Main Outcomes, Comparison to No Infant-Toddler Centers, Adolescent Cohort} \label{ols-M-adol-reg-nopres-asilo}
\scalebox{0.8}{\begin{tabular}{l c c c}
\toprule
 & None & Bic & Full \\
\midrule
IQ Factor & 0.17 & 0.16 & 0.10 \\
& (0.18) & (0.18) & (0.17) \\
& \textit{ 204 } & \textit{ 204 } & \textit{ 204 } \\
SDQ Composite - Child & \textbf{ 1.29 } & 0.87 & 0.85 \\
& (0.78) & (0.81) & (0.86) \\
& \textit{ 202 } & \textit{ 202 } & \textit{ 202 } \\
SDQ Composite & -0.52 & -1.04 & -1.10 \\
& (0.88) & (0.88) & (0.98) \\
& \textit{ 201 } & \textit{ 201 } & \textit{ 201 } \\
Depression Score - positive & -0.90 & -1.23 & -1.38 \\
& (1.11) & (1.10) & (1.13) \\
& \textit{ 198 } & \textit{ 198 } & \textit{ 198 } \\
Locus of Control - positive & -0.12 & -0.15 & \textbf{ -0.19 } \\
& (0.12) & (0.12) & (0.12) \\
& \textit{ 200 } & \textit{ 200 } & \textit{ 200 } \\
Not Obese & 0.06 & 0.09 & 0.07 \\
& (0.07) & (0.07) & (0.06) \\
& \textit{ 204 } & \textit{ 204 } & \textit{ 204 } \\
Not Overweight & 0.04 & 0.04 & 0.03 \\
& (0.05) & (0.05) & (0.05) \\
& \textit{ 204 } & \textit{ 204 } & \textit{ 204 } \\
Health is Good & -0.04 & -0.06 & -0.05 \\
& (0.08) & (0.09) & (0.09) \\
& \textit{ 204 } & \textit{ 204 } & \textit{ 204 } \\
Go To School & 0.02 & 0.01 & -0.01 \\
& (0.04) & (0.04) & (0.04) \\
& \textit{ 204 } & \textit{ 204 } & \textit{ 204 } \\
How Much Child Likes School & 0.00 & -0.02 & -0.06 \\
& (0.17) & (0.18) & (0.19) \\
& \textit{ 195 } & \textit{ 195 } & \textit{ 195 } \\
Days of Sport (Weekly) & \textbf{ 0.54 } & 0.49 & 0.46 \\
& (0.36) & (0.38) & (0.39) \\
& \textit{ 198 } & \textit{ 198 } & \textit{ 198 } \\
Num. of Friends & \textbf{ 4.38 } & \textbf{ 4.18 } & \textbf{ 4.01 } \\
& (1.32) & (1.33) & (1.50) \\
& \textit{ 189 } & \textit{ 189 } & \textit{ 189 } \\
Volunteers & 0.05 & 0.02 & 0.01 \\
& (0.08) & (0.08) & (0.08) \\
& \textit{ 204 } & \textit{ 204 } & \textit{ 204 } \\
Trust Score & 0.13 & 0.09 & 0.01 \\
& (0.25) & (0.26) & (0.28) \\
& \textit{ 201 } & \textit{ 201 } & \textit{ 201 } \\
\bottomrule
\end{tabular}
}
\vspace{1ex} \\
\footnotesize\raggedright{Note: This table shows the estimates of the coefficient for attending Reggio Approach infant-toddler centers from multiple methods. We compare Reggio Approach people with people who attended no infant-toddler center. Column title indicates the corresponding control set and and model.  \textbf{None} = OLS estimate with no control variables. \textbf{BIC} = OLS estimate with controls selected by Bayesian Information Criterion (BIC) and additional controls for male indicator and ITC attendance indicator. \textbf{Full} = OLS estimate with the full set of controls. \textbf{PSM} =  propensity score matching estimation. Bold number shows that the estimate is statistically significant at the 15\% level. Number of observations used in estimation is reported in italic.}
\end{table}



\begin{table}[H] \caption{Estimation Results for Main Outcomes, Comparison to No Infant-Toddler Centers, Adult 30s Cohort} \label{ols-M-adult30-reg-nopres-asilo}
\scalebox{0.75}{\begin{tabular}{l c c c}
\toprule
 & None & BIC & Full \\
\midrule
IQ Factor & -0.06 & -0.13 & -0.15 \\
& (0.12) & (0.12) & (0.12) \\
& \textit{ 215 } & \textit{ 215 } & \textit{ 215 } \\
Graduate from High School & -0.05 & -0.04 & -0.00 \\
& (0.05) & (0.05) & (0.05) \\
& \textit{ 215 } & \textit{ 215 } & \textit{ 215 } \\
High School Grade & \textbf{ -7.63 } & \textbf{ -6.72 } & \textbf{ -6.18 } \\
& (2.01) & (1.86) & (1.95) \\
& \textit{ 162 } & \textit{ 162 } & \textit{ 162 } \\
High School Grade (Standardized) & \textbf{ -5.61 } & \textbf{ -4.76 } & \textbf{ -4.62 } \\
& (1.72) & (1.73) & (1.79) \\
& \textit{ 162 } & \textit{ 162 } & \textit{ 162 } \\
Max Edu: University & -0.01 & -0.02 & -0.03 \\
& (0.06) & (0.06) & (0.06) \\
& \textit{ 215 } & \textit{ 215 } & \textit{ 215 } \\
Employed & \textbf{ 0.04 } & \textbf{ 0.05 } & 0.04 \\
& (0.03) & (0.03) & (0.03) \\
& \textit{ 215 } & \textit{ 215 } & \textit{ 215 } \\
Hours Worked Per Week & \textbf{ 2.91 } & \textbf{ 3.61 } & \textbf{ 3.00 } \\
& (1.60) & (1.66) & (1.77) \\
& \textit{ 191 } & \textit{ 191 } & \textit{ 191 } \\
Married or Cohabitating & \textbf{ 0.18 } & \textbf{ 0.18 } & \textbf{ 0.14 } \\
& (0.07) & (0.07) & (0.07) \\
& \textit{ 215 } & \textit{ 215 } & \textit{ 215 } \\
Not Obese & \textbf{ 0.15 } & \textbf{ 0.11 } & \textbf{ 0.10 } \\
& (0.05) & (0.05) & (0.06) \\
& \textit{ 215 } & \textit{ 215 } & \textit{ 215 } \\
Not Overweight & 0.06 & 0.05 & 0.06 \\
& (0.06) & (0.05) & (0.06) \\
& \textit{ 215 } & \textit{ 215 } & \textit{ 215 } \\
Locus of Control - positive & -0.14 & -0.13 & -0.08 \\
& (0.11) & (0.10) & (0.11) \\
& \textit{ 212 } & \textit{ 212 } & \textit{ 212 } \\
Depression Score - positive & 0.72 & 0.18 & 0.03 \\
& (0.78) & (0.76) & (0.79) \\
& \textit{ 214 } & \textit{ 214 } & \textit{ 214 } \\
Ever Voted for Municipal & \textbf{ 0.14 } & 0.04 & 0.05 \\
& (0.07) & (0.06) & (0.06) \\
& \textit{ 210 } & \textit{ 210 } & \textit{ 210 } \\
Ever Voted for Regional & \textbf{ 0.15 } & 0.07 & 0.07 \\
& (0.07) & (0.06) & (0.06) \\
& \textit{ 210 } & \textit{ 210 } & \textit{ 210 } \\
\bottomrule
\end{tabular}
}
\vspace{1ex} \\
\footnotesize\raggedright{Note: This table shows the estimates of the coefficient for attending Reggio Approach infant-toddler centers from multiple methods. We compare Reggio Approach people with people who attended no infant-toddler center. Column title indicates the corresponding control set and and model.  \textbf{None} = OLS estimate with no control variables. \textbf{BIC} = OLS estimate with controls selected by Bayesian Information Criterion (BIC) and additional controls for male indicator and ITC attendance indicator. \textbf{Full} = OLS estimate with the full set of controls. \textbf{PSM} =  propensity score matching estimation. Bold number shows that the estimate is statistically significant at the 15\% level. Number of observations used in estimation is reported in italic.}
\end{table}


\begin{table}[H] \caption{Estimation Results for Main Outcomes, Comparison to No Infant-Toddler Centers, Adult 40s Cohort} \label{ols-M-adult40-reg-nopres-asilo}
\scalebox{0.75}{\begin{tabular}{l c c c}
\toprule
 & None40 & BIC40 & Full40 \\
\midrule
IQ Score & \textbf{     -0.12 } & \textbf{     -0.14 } & \textbf{     -0.15 } \\
& (     0.04 ) & (     0.04 ) & (     0.04 ) \\
& \textit{ 206 } & \textit{ 206 } & \textit{ 206 } \\
IQ Factor & \textbf{     -0.22 } & \textbf{     -0.28 } & \textbf{     -0.31 } \\
& (     0.11 ) & (     0.12 ) & (     0.12 ) \\
& \textit{ 206 } & \textit{ 206 } & \textit{ 206 } \\
Graduate from High School &     -0.05 &     -0.07 &     -0.05 \\
& (     0.05 ) & (     0.06 ) & (     0.06 ) \\
& \textit{ 206 } & \textit{ 206 } & \textit{ 206 } \\
High School Grade & \textbf{     -9.32 } & \textbf{     -9.35 } & \textbf{     -8.19 } \\
& (     2.14 ) & (     2.12 ) & (     2.39 ) \\
& \textit{ 157 } & \textit{ 157 } & \textit{ 157 } \\
High School Grade (Standardized) & \textbf{     -4.53 } & \textbf{     -4.80 } & \textbf{     -3.95 } \\
& (     1.85 ) & (     1.93 ) & (     2.07 ) \\
& \textit{ 154 } & \textit{ 154 } & \textit{ 154 } \\
Max Edu: University &     -0.01 &     -0.05 &     -0.08 \\
& (     0.05 ) & (     0.05 ) & (     0.05 ) \\
& \textit{ 206 } & \textit{ 206 } & \textit{ 206 } \\
Employed &     -0.01 &     -0.03 &     -0.01 \\
& (     0.03 ) & (     0.03 ) & (     0.03 ) \\
& \textit{ 206 } & \textit{ 206 } & \textit{ 206 } \\
Hours Worked Per Week &     -0.00 &      0.39 &      1.47 \\
& (     1.69 ) & (     1.80 ) & (     2.01 ) \\
& \textit{ 190 } & \textit{ 190 } & \textit{ 190 } \\
Married or Cohabitating & \textbf{     -0.25 } & \textbf{     -0.23 } & \textbf{     -0.24 } \\
& (     0.06 ) & (     0.07 ) & (     0.07 ) \\
& \textit{ 206 } & \textit{ 206 } & \textit{ 206 } \\
Obese & \textbf{     -0.21 } & \textbf{     -0.17 } & \textbf{     -0.17 } \\
& (     0.06 ) & (     0.06 ) & (     0.06 ) \\
& \textit{ 206 } & \textit{ 206 } & \textit{ 206 } \\
Overweight &     -0.06 &     -0.09 &     -0.09 \\
& (     0.06 ) & (     0.07 ) & (     0.07 ) \\
& \textit{ 206 } & \textit{ 206 } & \textit{ 206 } \\
Locus of Control - positive & \textbf{     -0.26 } &     -0.17 &     -0.13 \\
& (     0.12 ) & (     0.13 ) & (     0.14 ) \\
& \textit{ 204 } & \textit{ 204 } & \textit{ 204 } \\
Depression Score - positive &     -0.58 &     -0.99 &     -0.97 \\
& (     0.78 ) & (     0.81 ) & (     0.86 ) \\
& \textit{ 206 } & \textit{ 206 } & \textit{ 206 } \\
Ever Voted for Municipal &      0.01 &     -0.07 &     -0.02 \\
& (     0.07 ) & (     0.06 ) & (     0.07 ) \\
& \textit{ 196 } & \textit{ 196 } & \textit{ 196 } \\
Ever Voted for Regional &      0.00 &     -0.07 &     -0.02 \\
& (     0.07 ) & (     0.06 ) & (     0.07 ) \\
& \textit{ 196 } & \textit{ 196 } & \textit{ 196 } \\
\bottomrule
\end{tabular}
}
\vspace{1ex} \\
\footnotesize\raggedright{Note: This table shows the estimates of the coefficient for attending Reggio Approach infant-toddler centers from multiple methods. We compare Reggio Approach people with people who attended no infant-toddler center. Column title indicates the corresponding control set and and model.  \textbf{None} = OLS estimate with no control variables. \textbf{BIC} = OLS estimate with controls selected by Bayesian Information Criterion (BIC) and additional controls for male indicator and ITC attendance indicator. \textbf{Full} = OLS estimate with the full set of controls. \textbf{PSM} =  propensity score matching estimation. Bold number shows that the estimate is statistically significant at the 15\% level. Number of observations used in estimation is reported in italic.}
\end{table}
