We present the estimates of the methods described above for a handful of key outcomes.\footnote{We choose outcomes that are economically significant,  outcomes that have limited missing values, and outcomes with sufficient variation across individuals. Results on the full set of outcomes are reported in Appendix~\ref{appsec:extended-outcome}. \textbf{[JJH: All outcomes. We need to summarize in text what we find in the appendix.][We have edited this section to include a summary of the results in the appendix.}}\footnote{A brief description of the outcomes is as follows: We rescale non-cognitive outcomes, including SDQ (Strengths and Difficulties Questionnaire) score, Locus of Control, and Depression score, so that the higher value has a more socially positive meaning; \textbf{SDQ Composite - Child} is reported by mother, and \textbf{SDQ Composite} is self-reported; \textbf{IQ Score} is measured using Raven's Progressive Matrices; \textbf{How Much Child Likes School} is a single question with three answers, where 1 means ``A little", 2 means ``So so", and 3 means ``A lot";  \textbf{High School Grade} has the maximum scoring of 100; since the mean and variance is not always the same, we standardize the high school grade for each city, cohort, and high school type based on our data to have mean zero and unit variance; All the other measures reported in the estimation results are binary indicators.} We first present the results from the analysis of infant-toddler care. The results are not consistently significant with some negative effects appearing for the younger cohorts. We then present the results from the analysis of preschool. Although these results are stronger than those from the infant-toddler care, very few outcomes show statistically significant treatment effects that are robust across different estimation procedures. The strongest results are from the comparison of Reggio Approach preschool against no preschool for the age-40 cohort. 

Tables~\ref{ols-M-child-reg-nopres-asilo} to~\ref{ols-M-adult40-reg-nopres-asilo} show the estimation results for Reggio Approach infant-toddler care. In the child cohort, Reggio Approach infant-toddler centers had significantly positive effect on IQ and obesity relative to no infant-toddler care in Reggio Emilia. In the adolescent cohort, Reggio Approach infant-toddler care did not have a clear effect relative to no infant-toddler care. In the age-30 cohort, Reggio Approach infant-toddler care had a significantly negative effect on IQ, high school grade, and locus of control. Moreover, Reggio Approach infant-toddler centers had a significantly positive effect on hours worked per week, marriage, and obesity. In the age-40 cohort, Reggio Approach also had a significantly negative effect on IQ score, high school grades, and marriage. A positive effect was found for obesity. 

\begin{table}[H] \caption{Estimation Results for Main Outcomes, Comparison to No Infant-Toddler Care, Child Cohort} \label{ols-M-child-reg-nopres-asilo}
\scalebox{0.8}{\begin{tabular}{l c c c}
\toprule
 & NoneIt & BICIt & FullIt \\
\midrule
IQ Score &      0.05 & \textbf{      0.06 } &      0.03 \\
& (     0.03 ) & (     0.03 ) & (     0.04 ) \\
& \textit{ 228 } & \textit{ 228 } & \textit{ 228 } \\
IQ Factor & \textbf{      0.27 } & \textbf{      0.30 } &      0.19 \\
& (     0.14 ) & (     0.15 ) & (     0.15 ) \\
& \textit{ 228 } & \textit{ 228 } & \textit{ 228 } \\
SDQ Composite - Child &      0.17 &      0.31 &      0.72 \\
& (     0.63 ) & (     0.62 ) & (     0.71 ) \\
& \textit{ 228 } & \textit{ 228 } & \textit{ 228 } \\
Obese & \textbf{     -0.22 } & \textbf{     -0.23 } & \textbf{     -0.16 } \\
& (     0.07 ) & (     0.07 ) & (     0.08 ) \\
& \textit{ 228 } & \textit{ 228 } & \textit{ 228 } \\
Overweight &      0.01 &      0.02 &     -0.02 \\
& (     0.05 ) & (     0.05 ) & (     0.05 ) \\
& \textit{ 228 } & \textit{ 228 } & \textit{ 228 } \\
Health is Good &     -0.03 &     -0.06 &      0.02 \\
& (     0.07 ) & (     0.07 ) & (     0.08 ) \\
& \textit{ 228 } & \textit{ 228 } & \textit{ 228 } \\
Not Excited to Learn &     -0.01 &      0.01 &      0.01 \\
& (     0.03 ) & (     0.03 ) & (     0.03 ) \\
& \textit{ 228 } & \textit{ 228 } & \textit{ 228 } \\
Problems Sitting Still &     -0.07 &     -0.09 &     -0.09 \\
& (     0.06 ) & (     0.06 ) & (     0.06 ) \\
& \textit{ 228 } & \textit{ 228 } & \textit{ 228 } \\
How Much Child Likes School &      0.12 &      0.12 &      0.08 \\
& (     0.09 ) & (     0.09 ) & (     0.10 ) \\
& \textit{ 228 } & \textit{ 228 } & \textit{ 228 } \\
\bottomrule
\end{tabular}
}
\vspace{1ex} \\
\footnotesize\raggedright{Note: This table shows the estimates of the coefficient for attending Reggio Approach infant-toddler centers from multiple methods. We compare Reggio Approach people with people who attended no infant-toddler center. Column title indicates the corresponding control set and and model.  \textbf{None} = OLS estimate with no control variables. \textbf{BIC} = OLS estimate with controls selected by Bayesian Information Criterion (BIC) and additional controls for male indicator and ITC attendance indicator. \textbf{Full} = OLS estimate with the full set of controls. \textbf{PSM} =  propensity score matching estimation. Bold number shows that the estimate is statistically significant at the 15\% level. Number of observations used in estimation is reported in italic.}
\end{table}

\begin{table}[H] \caption{Estimation Results for Main Outcomes, Comparison to No Infant-Toddler Care, Adolescent Cohort} \label{ols-M-adol-reg-nopres-asilo}
\scalebox{0.8}{\begin{tabular}{l c c c}
\toprule
 & None & Bic & Full \\
\midrule
IQ Score &      0.05 &     -0.07 &      0.02 \\
& (     0.05 ) & (     0.18 ) & (     0.05 ) \\
& \textit{ 185 } & \textit{ 185 } & \textit{ 185 } \\
IQ Factor &      0.16 &     -0.53 &      0.08 \\
& (     0.18 ) & (     0.48 ) & (     0.18 ) \\
& \textit{ 185 } & \textit{ 185 } & \textit{ 185 } \\
SDQ Composite - Child & \textbf{      1.42 } &     -1.36 &      1.21 \\
& (     0.79 ) & (     5.83 ) & (     0.82 ) \\
& \textit{ 183 } & \textit{ 183 } & \textit{ 183 } \\
SDQ Composite &     -0.61 &     -2.26 &     -0.99 \\
& (     0.90 ) & (     4.68 ) & (     0.99 ) \\
& \textit{ 183 } & \textit{ 183 } & \textit{ 183 } \\
Depression Score - positive &     -0.94 &      2.95 &     -1.44 \\
& (     1.13 ) & (     8.27 ) & (     1.14 ) \\
& \textit{ 181 } & \textit{ 181 } & \textit{ 181 } \\
Locus of Control - positive &     -0.10 & \textbf{     -1.61 } &     -0.17 \\
& (     0.12 ) & (     0.39 ) & (     0.12 ) \\
& \textit{ 182 } & \textit{ 182 } & \textit{ 182 } \\
Obese &     -0.06 & \textbf{      0.40 } &     -0.07 \\
& (     0.07 ) & (     0.11 ) & (     0.06 ) \\
& \textit{ 185 } & \textit{ 185 } & \textit{ 185 } \\
Overweight &     -0.06 &      0.08 &     -0.06 \\
& (     0.04 ) & (     0.09 ) & (     0.04 ) \\
& \textit{ 185 } & \textit{ 185 } & \textit{ 185 } \\
Health is Good &     -0.02 &      0.21 &     -0.04 \\
& (     0.09 ) & (     0.59 ) & (     0.09 ) \\
& \textit{ 185 } & \textit{ 185 } & \textit{ 185 } \\
Go To School &      0.03 &     -0.08 &     -0.00 \\
& (     0.04 ) & (     0.07 ) & (     0.04 ) \\
& \textit{ 185 } & \textit{ 185 } & \textit{ 185 } \\
How Much Child Likes School &     -0.01 &      2.39 &     -0.09 \\
& (     0.17 ) & (     2.24 ) & (     0.19 ) \\
& \textit{ 179 } & \textit{ 179 } & \textit{ 179 } \\
Days of Sport (Weekly) &      0.55 &      2.97 &      0.56 \\
& (     0.37 ) & (     2.48 ) & (     0.39 ) \\
& \textit{ 180 } & \textit{ 180 } & \textit{ 180 } \\
\bottomrule
\end{tabular}
}
\vspace{1ex} \\
\footnotesize\raggedright{Note: This table shows the estimates of the coefficient for attending Reggio Approach infant-toddler centers from multiple methods. We compare Reggio Approach people with people who attended no infant-toddler center. Column title indicates the corresponding control set and and model.  \textbf{None} = OLS estimate with no control variables. \textbf{BIC} = OLS estimate with controls selected by Bayesian Information Criterion (BIC) and additional controls for male indicator and ITC attendance indicator. \textbf{Full} = OLS estimate with the full set of controls. \textbf{PSM} =  propensity score matching estimation. Bold number shows that the estimate is statistically significant at the 15\% level. Number of observations used in estimation is reported in italic.}
\end{table}

\begin{table}[H] \caption{Estimation Results for Main Outcomes, Comparison to No Infant-Toddler Care, Age-30 Cohort} \label{ols-M-adult30-reg-nopres-asilo}
\scalebox{0.75}{\begin{tabular}{l c c c}
\toprule
 & None & BIC & Full \\
\midrule
IQ Factor & -0.06 & -0.13 & -0.15 \\
& (0.12) & (0.12) & (0.12) \\
& \textit{ 215 } & \textit{ 215 } & \textit{ 215 } \\
Graduate from High School & -0.05 & -0.04 & -0.00 \\
& (0.05) & (0.05) & (0.05) \\
& \textit{ 215 } & \textit{ 215 } & \textit{ 215 } \\
High School Grade & \textbf{ -7.63 } & \textbf{ -6.72 } & \textbf{ -6.18 } \\
& (2.01) & (1.86) & (1.95) \\
& \textit{ 162 } & \textit{ 162 } & \textit{ 162 } \\
High School Grade (Standardized) & \textbf{ -5.61 } & \textbf{ -4.76 } & \textbf{ -4.62 } \\
& (1.72) & (1.73) & (1.79) \\
& \textit{ 162 } & \textit{ 162 } & \textit{ 162 } \\
Max Edu: University & -0.01 & -0.02 & -0.03 \\
& (0.06) & (0.06) & (0.06) \\
& \textit{ 215 } & \textit{ 215 } & \textit{ 215 } \\
Employed & \textbf{ 0.04 } & \textbf{ 0.05 } & 0.04 \\
& (0.03) & (0.03) & (0.03) \\
& \textit{ 215 } & \textit{ 215 } & \textit{ 215 } \\
Hours Worked Per Week & \textbf{ 2.91 } & \textbf{ 3.61 } & \textbf{ 3.00 } \\
& (1.60) & (1.66) & (1.77) \\
& \textit{ 191 } & \textit{ 191 } & \textit{ 191 } \\
Married or Cohabitating & \textbf{ 0.18 } & \textbf{ 0.18 } & \textbf{ 0.14 } \\
& (0.07) & (0.07) & (0.07) \\
& \textit{ 215 } & \textit{ 215 } & \textit{ 215 } \\
Not Obese & \textbf{ 0.15 } & \textbf{ 0.11 } & \textbf{ 0.10 } \\
& (0.05) & (0.05) & (0.06) \\
& \textit{ 215 } & \textit{ 215 } & \textit{ 215 } \\
Not Overweight & 0.06 & 0.05 & 0.06 \\
& (0.06) & (0.05) & (0.06) \\
& \textit{ 215 } & \textit{ 215 } & \textit{ 215 } \\
Locus of Control - positive & -0.14 & -0.13 & -0.08 \\
& (0.11) & (0.10) & (0.11) \\
& \textit{ 212 } & \textit{ 212 } & \textit{ 212 } \\
Depression Score - positive & 0.72 & 0.18 & 0.03 \\
& (0.78) & (0.76) & (0.79) \\
& \textit{ 214 } & \textit{ 214 } & \textit{ 214 } \\
Ever Voted for Municipal & \textbf{ 0.14 } & 0.04 & 0.05 \\
& (0.07) & (0.06) & (0.06) \\
& \textit{ 210 } & \textit{ 210 } & \textit{ 210 } \\
Ever Voted for Regional & \textbf{ 0.15 } & 0.07 & 0.07 \\
& (0.07) & (0.06) & (0.06) \\
& \textit{ 210 } & \textit{ 210 } & \textit{ 210 } \\
\bottomrule
\end{tabular}
}
\vspace{1ex} \\
\footnotesize\raggedright{Note: This table shows the estimates of the coefficient for attending Reggio Approach infant-toddler centers from multiple methods. We compare Reggio Approach people with people who attended no infant-toddler center. Column title indicates the corresponding control set and and model.  \textbf{None} = OLS estimate with no control variables. \textbf{BIC} = OLS estimate with controls selected by Bayesian Information Criterion (BIC) and additional controls for male indicator and ITC attendance indicator. \textbf{Full} = OLS estimate with the full set of controls. \textbf{PSM} =  propensity score matching estimation. Bold number shows that the estimate is statistically significant at the 15\% level. Number of observations used in estimation is reported in italic.}
\end{table}

\begin{table}[H] \caption{Estimation Results for Main Outcomes, Comparison to No Infant-Toddler Care, Age-40 Cohort} \label{ols-M-adult40-reg-nopres-asilo}
\scalebox{0.75}{\begin{tabular}{l c c c}
\toprule
 & None40 & BIC40 & Full40 \\
\midrule
IQ Score & \textbf{     -0.12 } & \textbf{     -0.14 } & \textbf{     -0.15 } \\
& (     0.04 ) & (     0.04 ) & (     0.04 ) \\
& \textit{ 206 } & \textit{ 206 } & \textit{ 206 } \\
IQ Factor & \textbf{     -0.22 } & \textbf{     -0.28 } & \textbf{     -0.31 } \\
& (     0.11 ) & (     0.12 ) & (     0.12 ) \\
& \textit{ 206 } & \textit{ 206 } & \textit{ 206 } \\
Graduate from High School &     -0.05 &     -0.07 &     -0.05 \\
& (     0.05 ) & (     0.06 ) & (     0.06 ) \\
& \textit{ 206 } & \textit{ 206 } & \textit{ 206 } \\
High School Grade & \textbf{     -9.32 } & \textbf{     -9.35 } & \textbf{     -8.19 } \\
& (     2.14 ) & (     2.12 ) & (     2.39 ) \\
& \textit{ 157 } & \textit{ 157 } & \textit{ 157 } \\
High School Grade (Standardized) & \textbf{     -4.53 } & \textbf{     -4.80 } & \textbf{     -3.95 } \\
& (     1.85 ) & (     1.93 ) & (     2.07 ) \\
& \textit{ 154 } & \textit{ 154 } & \textit{ 154 } \\
Max Edu: University &     -0.01 &     -0.05 & \textbf{     -0.08 } \\
& (     0.05 ) & (     0.05 ) & (     0.05 ) \\
& \textit{ 206 } & \textit{ 206 } & \textit{ 206 } \\
Employed &     -0.01 &     -0.03 &     -0.01 \\
& (     0.03 ) & (     0.03 ) & (     0.03 ) \\
& \textit{ 206 } & \textit{ 206 } & \textit{ 206 } \\
Hours Worked Per Week &     -0.00 &      0.39 &      1.47 \\
& (     1.69 ) & (     1.80 ) & (     2.01 ) \\
& \textit{ 190 } & \textit{ 190 } & \textit{ 190 } \\
Married or Cohabitating & \textbf{     -0.25 } & \textbf{     -0.23 } & \textbf{     -0.24 } \\
& (     0.06 ) & (     0.07 ) & (     0.07 ) \\
& \textit{ 206 } & \textit{ 206 } & \textit{ 206 } \\
Obese & \textbf{     -0.21 } & \textbf{     -0.17 } & \textbf{     -0.17 } \\
& (     0.06 ) & (     0.06 ) & (     0.06 ) \\
& \textit{ 206 } & \textit{ 206 } & \textit{ 206 } \\
Overweight &     -0.06 &     -0.09 &     -0.09 \\
& (     0.06 ) & (     0.07 ) & (     0.07 ) \\
& \textit{ 206 } & \textit{ 206 } & \textit{ 206 } \\
Locus of Control - positive & \textbf{     -0.26 } &     -0.17 &     -0.13 \\
& (     0.12 ) & (     0.13 ) & (     0.14 ) \\
& \textit{ 204 } & \textit{ 204 } & \textit{ 204 } \\
Depression Score - positive &     -0.58 &     -0.99 &     -0.97 \\
& (     0.78 ) & (     0.81 ) & (     0.86 ) \\
& \textit{ 206 } & \textit{ 206 } & \textit{ 206 } \\
Ever Voted for Municipal &      0.01 &     -0.07 &     -0.02 \\
& (     0.07 ) & (     0.06 ) & (     0.07 ) \\
& \textit{ 196 } & \textit{ 196 } & \textit{ 196 } \\
Ever Voted for Regional &      0.00 &     -0.07 &     -0.02 \\
& (     0.07 ) & (     0.06 ) & (     0.07 ) \\
& \textit{ 196 } & \textit{ 196 } & \textit{ 196 } \\
\bottomrule
\end{tabular}
}
\vspace{1ex} \\
\footnotesize\raggedright{Note: This table shows the estimates of the coefficient for attending Reggio Approach infant-toddler centers from multiple methods. We compare Reggio Approach people with people who attended no infant-toddler center. Column title indicates the corresponding control set and and model.  \textbf{None} = OLS estimate with no control variables. \textbf{BIC} = OLS estimate with controls selected by Bayesian Information Criterion (BIC) and additional controls for male indicator and ITC attendance indicator. \textbf{Full} = OLS estimate with the full set of controls. \textbf{PSM} =  propensity score matching estimation. Bold number shows that the estimate is statistically significant at the 15\% level. Number of observations used in estimation is reported in italic.}
\end{table}

We now discuss the results from the analysis of preschool.\footnote{Appendix~\ref{sec:results} includes more estimates including comparisons to specific school types and additional outcomes.} In the child cohort (Table \ref{ols-M-child-reg-pres}), the Reggio Approach increased the SDQ (Strengths and Difficulties Questionnaire) scores when compared to children who attended other preschools (OLS)\footnote{The SDQ is a widely-used scale inquiring about emotional symptoms, conduct problems, hyperactivity/inattention, peer relationships problems, and pro-social behavior \citep{Goodman_1997_JCPP}. For ease of interpretation, we have converted the SDQ score such that higher values correspond to more positive outcomes.}. This result becomes more positive after controlling for more background characteristics and when comparing with children in Padova.\footnote{The estimated coefficient is still positive, but not statistically significant, when comparing with children in Parma (DiD).} When we consider the sub-scales of the SDQ as outcomes, the results are positive and significant for the emotional symptomns, positive conduct, and pro-social tests while not significant on the hyperactivity and peer problems tests (see Table~\ref{combined_child_CN_Other}). 

Using propensity score matching to address issues of selection provides similar results as those of the OLS specification. The children were slightly more likely to be overweight and have worse health, although these estimates are not statistically significant for all specifications. The other main outcomes do not show significant effects. 

When we compare the Reggio Approach individuals in the child cohort to those who attended religious schools (Table~\ref{ols-M-child-reg-reli}), the Reggio Approach individuals had lower IQ scores and were more obese both within Reggio Emilia and in comparison to the other cities. Compared with the state schools (Table~\ref{ols-M-child-reg-state}), Reggio Approach children had higher IQ scores except in comparison to Parma. The SDQ score was positive when compared with Padova, but not as positive for within Reggio Emilia as was seen when comparing to all non-Reggio Approach schools.

In the adolescent cohort (Table~\ref{ols-M-adol-reg-pres}), adolescents who attended the Reggio Approach were significantly less likely to be depressed according to analyses done within Reggio Emilia and DiD estimates with Parma and Padova. However, the propensity score matching with Parma and Padova adolescents did not show significant effect on depression. The Reggio Approach individuals were more likely to be obese than individuals who attended other types of preschool, and the estimate on obesity is consistent across most of the methods. Some methods show that Reggio Approach individuals were less likely to be involved in sport activities, which is consistent with the increase in obesity. Other outcomes did not have consistently significant results, except for being more bothered by migrants than others in Reggio Emilia (Table~\ref{ols-S-adol-reg-reli}). 

In comparison to adolescents who attended religious schools (Table~\ref{ols-M-adol-reg-reli}) the IQ scores are lower for the Reggio Approach adolescents. This is consistent with the results for the child cohort. The SDQ score, capturing social-emotional skills, is higher both when considering the summary score and the individual sub-scales. Similar to the main specification, the adolescents had lower depression scores and higher obesity rates. There are fewer significant outcomes when comparing the Reggio Approach adolescents with those who attended state schools (Table~\ref{ols-M-adol-reg-stat}). Additionally, those that are significant are negative: SDQ scores were lower and adolescents reported less exercise and fewer friends.

In the adult cohorts, the results differ depending on the comparison group. The comparison with no preschool, shown in Tables~\ref{ols-M-adult30-reg-nopres} and~\ref{ols-M-adult40-reg-nopres}, shows many more statistically significant estimates. In the comparison with the other preschools, shown in Tables~\ref{ols-M-adult30-reg-pres} and~\ref{ols-M-adult40-reg-pres}, the only outcomes that show some significance across different methods is high school graduation and voting behavior in the age-40 cohort. The OLS estimates show that the Reggio Approach individuals in the age-40 cohort are more likely to graduate from high school than others within Reggio Emilia. Propensity score matching estimates show that the Reggio Approach age-40 individuals are more likely to have voted for regional election than people in Parma and Padova who attended preschool.

In the age-30 cohort, Reggio Approach individuals had worse health along certain outcomes compared with others in Reggio Emilia who did not attend any preschool (Table~\ref{ols-H-adult30-reg-none}). This is seen in reporting more cigarettes per day and more sick days in the past months. Compared with those attended other preschools in Reggio Emilia, Reggio Approach adults were less satisfied with their health and more optimistic (Table~\ref{ols-H-adult30-reg-other}). These two estimates flip directions when comparing against those in Reggio Emilia who did not attend any preschool. 

In comparison to those who attended religious schools (Tables~\ref{ols-M-adult30-reg-reli} and~\ref{ols-M-adult40-reg-reli}), age-30 and age-40 adults had lower IQ scores. This is similarly seen in the child and adolescent cohorts when comparing to individuals from religious schools. Individuals in the age-30 cohort also had lower employment levels than those who attended religious schools within Reggio Emilia. Similar to the child and adolescent cohorts, the results flip directions in comparison to state schools (Table~\ref{ols-M-adult30-reg-stat}). More results are positive in the comparison to state schools than the comparison to religious schools. Some examples include lower obesity and more positive locus of control.

In the comparison with no preschool, Reggio Approach individuals in the age-40 cohort were more likely to be employed. Moreover, Reggio Approach individuals were significantly more likely to work more hours than other groups in both the age-30 and age-40 cohorts. The age-40 cohort was more stressed from work in comparison to both no preschool and other preschools, but also reported being more satisfied with work and their income than those in Parma and Padova (Tables~\ref{ols-W-adult40-reg-other} and~\ref{ols-W-adult40-reg-none}). 

The OLS estimates show that the Reggio Approach individuals were significantly more likely to be obese when compared to no preschool in Reggio Emilia, which aligns with the adolescent results. The age-40 Reggio Approach individuals had significant and positive effects on locus of control, the depression score, and voting behaviors across many methodologies. 

\textbf{[JJH: ? Summarize this mass of results please -- this is too vague.][We have added more summary of the results in the appendix.]}

% ========================================================================= %
% CHILD COHORT


\begin{table}[H] \caption{Estimation Results for Main Outcomes, Comparison to Non-RA Preschools, Child Cohort} \label{ols-M-child-reg-pres}
\scalebox{0.7}{\begin{tabular}{l c c c c c c c c c c c}
\toprule
 & None & BIC & Full & PSMR & KMR & DidPm & PSMPm & KMPm & DidPv & PSMPv & KMPv \\
\midrule
IQ Factor & -0.13 & -0.20 & -0.19 & -0.21 & -0.15 & -0.03 & -0.34 & -0.39 & -0.14 & -0.21 & -0.25 \\
\quad \textit{Unadjusted P-Value} & (0.22) & (0.06)** & (0.06)** & (0.05)** & (0.20) & (0.83) & (0.00)*** & (0.00)*** & (0.43) & (0.04)*** & (0.03)*** \\
\quad \textit{Stepdown P-Value} & (0.85) & (0.42) & (0.33) & (0.37) & (0.81) & (0.99) & (0.03)*** & (0.00)*** & (0.94) & (0.29) & (0.19) \\
SDQ Composite - Child & 1.59 & 1.47 & 2.14 & 1.39 & 1.13 & 0.62 & 0.30 & 0.24 & 1.91 & 0.75 & 0.71 \\
\quad \textit{Unadjusted P-Value} & (0.00)*** & (0.01)*** & (0.00)*** & (0.01)*** & (0.06)** & (0.43) & (0.52) & (0.60) & (0.03)*** & (0.17) & (0.16) \\
\quad \textit{Stepdown P-Value} & (0.04)*** & (0.07)** & (0.00)*** & (0.15) & (0.45) & (0.99) & (0.93) & (0.96) & (0.22) & (0.67) & (0.58) \\
Not Obese & -0.04 & -0.07 & -0.08 & -0.08 & -0.06 & -0.01 & -0.13 & -0.16 & 0.02 & -0.06 & -0.06 \\
\quad \textit{Unadjusted P-Value} & (0.47) & (0.16) & (0.14)* & (0.16) & (0.28) & (0.84) & (0.01)*** & (0.00)*** & (0.83) & (0.29) & (0.23) \\
\quad \textit{Stepdown P-Value} & (0.98) & (0.63) & (0.59) & (0.71) & (0.84) & (0.99) & (0.08)** & (0.01)*** & (0.96) & (0.69) & (0.65) \\
Not Overweight & -0.02 & -0.01 & -0.02 & 0.00 & -0.01 & -0.02 & 0.05 & 0.02 & -0.04 & -0.04 & -0.04 \\
\quad \textit{Unadjusted P-Value} & (0.54) & (0.87) & (0.64) & (0.99) & (0.79) & (0.76) & (0.18) & (0.53) & (0.44) & (0.26) & (0.24) \\
\quad \textit{Stepdown P-Value} & (0.98) & (0.63) & (0.92) & (0.99) & (0.99) & (0.99) & (0.75) & (0.96) & (0.94) & (0.69) & (0.65) \\
Health is Good & -0.02 & -0.00 & 0.01 & -0.02 & -0.03 & 0.07 & 0.07 & 0.04 & -0.01 & -0.03 & -0.09 \\
\quad \textit{Unadjusted P-Value} & (0.78) & (0.99) & (0.87) & (0.70) & (0.64) & (0.43) & (0.16) & (0.39) & (0.93) & (0.55) & (0.06)** \\
\quad \textit{Stepdown P-Value} & (0.98) & (0.99) & (0.92) & (0.99) & (0.99) & (0.99) & (0.75) & (0.96) & (0.96) & (0.69) & (0.35) \\
Not Excited to Learn & -0.01 & -0.00 & -0.01 & 0.00 & -0.00 & -0.00 & -0.02 & -0.02 & -0.04 & -0.03 & -0.02 \\
\quad \textit{Unadjusted P-Value} & (0.60) & (0.84) & (0.69) & (0.92) & (0.99) & (0.95) & (0.41) & (0.28) & (0.31) & (0.22) & (0.41) \\
\quad \textit{Stepdown P-Value} & (0.98) & (0.63) & (0.92) & (0.99) & (0.99) & (0.99) & (0.93) & (0.92) & (0.83) & (0.69) & (0.65) \\
Problems Sitting Still & -0.00 & 0.01 & -0.03 & 0.02 & 0.02 & -0.08 & -0.01 & -0.01 & -0.08 & 0.05 & -0.00 \\
\quad \textit{Unadjusted P-Value} & (0.90) & (0.78) & (0.51) & (0.71) & (0.63) & (0.16) & (0.71) & (0.85) & (0.20) & (0.24) & (0.90) \\
\quad \textit{Stepdown P-Value} & (0.98) & (0.63) & (0.92) & (0.99) & (0.99) & (0.77) & (0.93) & (0.96) & (0.76) & (0.69) & (0.92) \\
How Much Child Likes School & 0.14 & 0.11 & 0.15 & 0.10 & 0.11 & 0.24 & -0.03 & -0.04 & 0.29 & 0.28 & 0.33 \\
\quad \textit{Unadjusted P-Value} & (0.05)*** & (0.11)* & (0.04)*** & (0.19) & (0.15)* & (0.01)*** & (0.58) & (0.45) & (0.01)*** & (0.00)*** & (0.00)*** \\
\quad \textit{Stepdown P-Value} & (0.29) & (0.59) & (0.25) & (0.71) & (0.76) & (0.11) & (0.93) & (0.96) & (0.08)** & (0.00)*** & (0.00)*** \\
Num. of Friends & -0.30 & -0.42 & -0.35 & -0.36 & -0.38 & -0.19 & -0.36 & -0.34 & -0.22 & -1.54 & -1.57 \\
\quad \textit{Unadjusted P-Value} & (0.23) & (0.09)** & (0.18) & (0.15) & (0.15)* & (0.74) & (0.26) & (0.27) & (0.79) & (0.00)*** & (0.00)*** \\
\quad \textit{Stepdown P-Value} & (0.85) & (0.52) & (0.64) & (0.71) & (0.76) & (0.99) & (0.83) & (0.92) & (0.96) & (0.00)*** & (0.00)*** \\
Candy Game: Willing to Share Candies & 0.01 & 0.00 & 0.03 & -0.03 & 0.01 & 0.01 & -0.03 & -0.01 & 0.02 & -0.06 & -0.04 \\
\quad \textit{Unadjusted P-Value} & (0.70) & (0.90) & (0.39) & (0.44) & (0.89) & (0.77) & (0.41) & (0.63) & (0.65) & (0.11)* & (0.14)* \\
\quad \textit{Stepdown P-Value} & (0.98) & (0.99) & (0.89) & (0.96) & (0.99) & (0.99) & (0.93) & (0.96) & (0.95) & (0.59) & (0.58) \\
\bottomrule
\end{tabular}
}
\vspace{1ex} \\
\footnotesize\raggedright{Note: This table shows the estimates of the coefficient for attending Reggio Approach preschools from multiple methods. We compare Reggio Approach individuals with those who attended other preschools. Column title indicates the corresponding control set and and model. \textbf{None} = OLS estimate with no control variables. \textbf{BIC} = OLS estimate with controls selected by Bayesian Information Criterion (BIC) and additional controls for male indicator, migrant indicator, and ITC attendance indicator. \textbf{Full} = OLS estimate with the full set of controls. \textbf{PSM} =  propensity score matching estimation. \textbf{AIPW} = augmented inverse propensity weighting estimation. \textbf{DidPm} = difference-in-difference estimate of (Reggio Muni - Parma Muni) - (Reggio Other - Parma Other). \textbf{PSMPm} = propensity score matching between Reggio Approach people and people who attended Parma preschools. \textbf{DidPv} = difference-in-difference estimate of (Reggio Muni - Padova Muni) - (Reggio Other - Padova Other). \textbf{PSMPv} = propensity score matching between Reggio Approach people and people who attended Padova preschools. Robust standard errors are reported in parentheses. Bold number shows that the estimate is statistically significant at the 10\% level \textbf{[JJH: Use 10\% level and 5\% as well. 15\%.]}. Number of observations used in estimation is reported in italic.}

\end{table}

\textbf{[JJH: What cohort label? Compare with OLS as well. Also report results for multiple hypothesis testing in each cohort.]}

\textbf{[JJH: We really need to correct for multiple hypothesis testing -- step down or Bonferroni.]}

\textbf{[JJH: Label cohorts!]}

\begin{table}[H] \caption{Estimation Results for Main Outcomes, Comparison to Non-RA Preschools, Adolescent Cohort} \label{ols-M-adol-reg-pres}
\scalebox{0.66}{\begin{tabular}{l c c c c c c c}
\toprule
 & None & BIC & Full & PSM & DidPm & DidPv \\
\midrule
IQ Factor & -0.12 & \textbf{ -0.15 } & -0.03 & -0.06 & -0.16 & \textbf{ -0.25 } \\
\quad \textit{Unadjusted P-Value} & (0.22) & (0.15) & (0.78) & (0.53) & (0.21) & (0.15) \\
\quad \textit{Stepdown P-Value} & (0.88) & (0.84) & (0.98) & (0.99) & (0.87) & (0.84) \\
SDQ Composite - Child & 0.01 & 0.18 & 0.37 & -0.56 & -0.41 & -0.52 \\
\quad \textit{Unadjusted P-Value} & (0.98) & (0.80) & (0.55) & (0.49) & (0.64) & (0.51) \\
\quad \textit{Stepdown P-Value} & (0.99) & (0.99) & (0.97) & (0.99) & (0.98) & (0.96) \\
SDQ Composite & 0.90 & \textbf{ 1.03 } & 0.72 & 1.02 & 0.90 & 0.71 \\
\quad \textit{Unadjusted P-Value} & (0.15) & (0.14) & (0.32) & (0.22) & (0.32) & (0.45) \\
\quad \textit{Stepdown P-Value} & (0.82) & (0.82) & (0.90) & (0.94) & (0.88) & (0.96) \\
Depression Score - positive & \textbf{ 1.46 } & \textbf{ 2.39 } & \textbf{ 1.81 } & \textbf{ 2.24 } & \textbf{ 2.21 } & \textbf{ 2.19 } \\
\quad \textit{Unadjusted P-Value} & (0.06) & (0.01) & (0.05) & (0.03) & (0.03) & (0.05) \\
\quad \textit{Stepdown P-Value} & (0.58) & (0.09) & (0.31) & (0.36) & (0.36) & (0.49) \\
Locus of Control - positive & 0.03 & 0.10 & 0.04 & 0.07 & \textbf{ -0.25 } & 0.13 \\
\quad \textit{Unadjusted P-Value} & (0.68) & (0.27) & (0.63) & (0.52) & (0.07) & (0.29) \\
\quad \textit{Stepdown P-Value} & (0.98) & (0.94) & (0.97) & (0.99) & (0.55) & (0.93) \\
Not Obese & \textbf{ -0.08 } & \textbf{ -0.11 } & \textbf{ -0.09 } & \textbf{ -0.07 } & 0.02 & -0.09 \\
\quad \textit{Unadjusted P-Value} & (0.04) & (0.03) & (0.03) & (0.10) & (0.73) & (0.21) \\
\quad \textit{Stepdown P-Value} & (0.53) & (0.26) & (0.33) & (0.73) & (0.98) & (0.89) \\
Not Overweight & 0.01 & -0.02 & -0.00 & -0.03 & \textbf{ 0.08 } & -0.03 \\
\quad \textit{Unadjusted P-Value} & (0.75) & (0.58) & (0.98) & (0.42) & (0.03) & (0.35) \\
\quad \textit{Stepdown P-Value} & (0.99) & (0.99) & (0.98) & (0.98) & (0.21) & (0.94) \\
Health is Good & 0.06 & 0.07 & 0.09 & 0.05 & 0.10 & \textbf{ 0.13 } \\
\quad \textit{Unadjusted P-Value} & (0.32) & (0.28) & (0.15) & (0.50) & (0.23) & (0.13) \\
\quad \textit{Stepdown P-Value} & (0.94) & (0.94) & (0.80) & (0.99) & (0.87) & (0.81) \\
Go To School & 0.03 & 0.01 & 0.03 & -0.01 & 0.03 & 0.04 \\
\quad \textit{Unadjusted P-Value} & (0.22) & (0.78) & (0.22) & (0.76) & (0.33) & (0.22) \\
\quad \textit{Stepdown P-Value} & (0.88) & (0.99) & (0.84) & (0.99) & (0.88) & (0.89) \\
How Much Child Likes School & -0.11 & -0.05 & -0.17 & -0.04 & -0.08 & -0.09 \\
\quad \textit{Unadjusted P-Value} & (0.33) & (0.67) & (0.17) & (0.74) & (0.64) & (0.57) \\
\quad \textit{Stepdown P-Value} & (0.94) & (0.99) & (0.84) & (0.99) & (0.98) & (0.96) \\
Days of Sport (Weekly) & \textbf{ -0.43 } & \textbf{ -0.56 } & -0.33 & -0.32 & \textbf{ -0.67 } & \textbf{ -0.58 } \\
\quad \textit{Unadjusted P-Value} & (0.06) & (0.04) & (0.20) & (0.33) & (0.04) & (0.09) \\
\quad \textit{Stepdown P-Value} & (0.58) & (0.37) & (0.84) & (0.98) & (0.38) & (0.69) \\
Num. of Friends & -0.76 & -0.57 & -0.35 & -0.69 & \textbf{ -2.71 } & -0.69 \\
\quad \textit{Unadjusted P-Value} & (0.54) & (0.59) & (0.76) & (0.56) & (0.15) & (0.74) \\
\quad \textit{Stepdown P-Value} & (0.98) & (0.99) & (0.98) & (0.99) & (0.81) & (0.96) \\
Volunteers & -0.02 & 0.01 & 0.04 & -0.05 & -0.03 & -0.06 \\
\quad \textit{Unadjusted P-Value} & (0.71) & (0.92) & (0.50) & (0.52) & (0.70) & (0.51) \\
\quad \textit{Stepdown P-Value} & (0.99) & (0.99) & (0.97) & (0.99) & (0.98) & (0.96) \\
Trust Score & 0.03 & 0.06 & 0.04 & 0.09 & 0.36 & -0.06 \\
\quad \textit{Unadjusted P-Value} & (0.85) & (0.76) & (0.83) & (0.71) & (0.17) & (0.83) \\
\quad \textit{Stepdown P-Value} & (0.99) & (0.99) & (0.98) & (0.99) & (0.82) & (0.96) \\
\bottomrule
\end{tabular}
}
\vspace{1ex} \\
\footnotesize\raggedright{Note: This table shows the estimates of the coefficient for attending Reggio Approach preschools from multiple methods. We compare Reggio Approach individuals with those who attended other preschools. Column title indicates the corresponding control set and and model. \textbf{None} = OLS estimate with no control variables. \textbf{BIC} = OLS estimate with controls selected by Bayesian Information Criterion (BIC) and additional controls for male indicator and ITC attendance indicator. \textbf{Full} = OLS estimate with the full set of controls. \textbf{PSM} =  propensity score matching estimation. \textbf{DidPm} = difference-in-difference estimate of (Reggio Muni - Parma Muni) - (Reggio Other - Parma Other). \textbf{PSMPm} = propensity score matching between Reggio Approach people and people who attended Parma preschools. \textbf{DidPv} = difference-in-difference estimate of (Reggio Muni - Padova Muni) - (Reggio Other - Padova Other). \textbf{PSMPv} = propensity score matching between Reggio Approach people and people who attended Padova preschools. Robust standard errors are reported in parentheses. Bold number shows that the estimate is statistically significant at the 15\% level. Number of observations used in estimation is reported in italic.}
\end{table}


\begin{table}[H] \caption{Estimation Results for Main Outcomes, Comparison to Non-RA Preschools, Age-30 Cohort} \label{ols-M-adult30-reg-pres}
\scalebox{0.65}{\begin{tabular}{l c c c c c c c}
\toprule
 & None & BIC & Full & PSM & DidPm & DidPv \\
\midrule
IQ Factor & 0.01 & -0.01 & 0.04 & -0.12 & -0.29 & 0.13 \\
\quad \textit{Unadjusted P-Value} & (0.95) & (0.92) & (0.77) & (0.58) & (0.11) & (0.56) \\
\quad \textit{Stepdown P-Value} & (0.99) & (0.99) & (0.96) & (0.89) & (0.84) & (0.99) \\
Graduate from High School & -0.05 & -0.04 & -0.06 & -0.04 & 0.04 & -0.12 \\
\quad \textit{Unadjusted P-Value} & (0.31) & (0.38) & (0.23) & (0.44) & (0.57) & (0.09) \\
\quad \textit{Stepdown P-Value} & (0.99) & (0.99) & (0.54) & (0.89) & (0.99) & (0.92) \\
High School Grade & 1.05 & 0.56 & 0.66 & 1.40 & 1.70 & 0.94 \\
\quad \textit{Unadjusted P-Value} & (0.49) & (0.71) & (0.67) & (0.40) & (0.63) & (0.80) \\
\quad \textit{Stepdown P-Value} & (0.99) & (0.99) & (0.89) & (0.89) & (0.99) & (0.99) \\
High School Grade (Standardized) & 2.78 & 1.96 & 2.01 & 2.73 & 3.81 & 2.61 \\
\quad \textit{Unadjusted P-Value} & (0.15) & (0.33) & (0.28) & (0.13) & (0.18) & (0.54) \\
\quad \textit{Stepdown P-Value} & (0.87) & (0.99) & (0.62) & (0.89) & (0.92) & (0.99) \\
Max Edu: University & 0.02 & 0.01 & 0.00 & -0.03 & 0.08 & 0.19 \\
\quad \textit{Unadjusted P-Value} & (0.76) & (0.89) & (1.00) & (0.71) & (0.48) & (0.16) \\
\quad \textit{Stepdown P-Value} & (0.99) & (0.99) & (0.99) & (0.89) & (0.99) & (0.89) \\
Employed & -0.03 & -0.03 & -0.02 & -0.02 & 0.12 & -0.05 \\
\quad \textit{Unadjusted P-Value} & (0.39) & (0.43) & (0.64) & (0.56) & (0.11) & (0.60) \\
\quad \textit{Stepdown P-Value} & (0.99) & (0.99) & (0.84) & (0.89) & (0.62) & (0.99) \\
Hours Worked Per Week & -0.02 & 0.19 & 0.63 & 0.64 & 3.80 & 0.29 \\
\quad \textit{Unadjusted P-Value} & (0.99) & (0.93) & (0.77) & (0.85) & (0.29) & (0.94) \\
\quad \textit{Stepdown P-Value} & (0.99) & (0.99) & (0.96) & (0.99) & (0.95) & (0.99) \\
Married or Cohabitating & 0.08 & 0.06 & 0.05 & -0.02 & 0.15 & 0.20 \\
\quad \textit{Unadjusted P-Value} & (0.29) & (0.46) & (0.52) & (0.85) & (0.18) & (0.16) \\
\quad \textit{Stepdown P-Value} & (0.98) & (0.99) & (0.78) & (0.99) & (0.92) & (0.90) \\
Not Obese & 0.01 & 0.00 & 0.03 & -0.03 & 0.01 & -0.03 \\
\quad \textit{Unadjusted P-Value} & (0.87) & (0.95) & (0.61) & (0.76) & (0.94) & (0.81) \\
\quad \textit{Stepdown P-Value} & (0.99) & (0.99) & (0.83) & (0.99) & (0.99) & (0.99) \\
Not Overweight & -0.06 & -0.01 & -0.02 & 0.04 & 0.07 & -0.04 \\
\quad \textit{Unadjusted P-Value} & (0.40) & (0.89) & (0.81) & (0.58) & (0.46) & (0.70) \\
\quad \textit{Stepdown P-Value} & (0.99) & (0.99) & (0.98) & (0.89) & (0.99) & (0.99) \\
Locus of Control - positive & 0.11 & 0.08 & 0.06 & 0.07 & 0.35 & 0.27 \\
\quad \textit{Unadjusted P-Value} & (0.40) & (0.49) & (0.59) & (0.60) & (0.11) & (0.22) \\
\quad \textit{Stepdown P-Value} & (0.99) & (0.99) & (0.83) & (0.89) & (0.70) & (0.97) \\
Depression Score - positive & 0.16 & -0.03 & 0.04 & -0.29 & 1.12 & 0.60 \\
\quad \textit{Unadjusted P-Value} & (0.87) & (0.97) & (0.96) & (0.74) & (0.40) & (0.72) \\
\quad \textit{Stepdown P-Value} & (0.99) & (0.99) & (0.99) & (0.99) & (0.98) & (0.99) \\
Volunteers & 0.11 & 0.10 & 0.11 & 0.10 & -0.05 & -0.03 \\
\quad \textit{Unadjusted P-Value} & (0.00) & (0.00) & (0.00) & (0.00) & (0.63) & (0.82) \\
\quad \textit{Stepdown P-Value} & (0.15) & (0.23) & (0.28) & (0.01) & (0.99) & (0.99) \\
Ever Voted for Municipal & -0.07 & -0.03 & -0.02 & 0.04 & -0.05 & 0.24 \\
\quad \textit{Unadjusted P-Value} & (0.36) & (0.66) & (0.77) & (0.51) & (0.57) & (0.03) \\
\quad \textit{Stepdown P-Value} & (0.99) & (0.99) & (0.95) & (0.89) & (0.99) & (0.68) \\
Ever Voted for Regional & -0.11 & -0.08 & -0.07 & -0.02 & -0.06 & 0.29 \\
\quad \textit{Unadjusted P-Value} & (0.18) & (0.23) & (0.29) & (0.71) & (0.49) & (0.01) \\
\quad \textit{Stepdown P-Value} & (0.90) & (0.99) & (0.62) & (0.89) & (0.99) & (0.39) \\
Num. of Friends & 0.73 & 0.62 & 0.86 & 1.25 & 3.37 & 1.26 \\
\quad \textit{Unadjusted P-Value} & (0.45) & (0.60) & (0.53) & (0.52) & (0.04) & (0.51) \\
\quad \textit{Stepdown P-Value} & (0.99) & (0.99) & (0.75) & (0.89) & (0.66) & (0.99) \\
Trust Score & 0.38 & 0.28 & 0.34 & 0.20 & 0.58 & 0.36 \\
\quad \textit{Unadjusted P-Value} & (0.07) & (0.20) & (0.13) & (0.59) & (0.15) & (0.30) \\
\quad \textit{Stepdown P-Value} & (0.69) & (0.98) & (0.41) & (0.89) & (0.84) & (0.98) \\
\bottomrule
\end{tabular}
}
\vspace{1ex} \\
\footnotesize\raggedright{Note: This table shows the estimates of the coefficient for attending Reggio Approach preschools from multiple methods. We compare Reggio Approach individuals with those who attended other preschools. Column title indicates the corresponding control set and and model. \textbf{None} = OLS estimate with no control variables. \textbf{BIC} = OLS estimate with controls selected by Bayesian Information Criterion (BIC) and additional controls for male indicator and ITC attendance indicator. \textbf{Full} = OLS estimate with the full set of controls. \textbf{PSM} =  propensity score matching estimation. \textbf{AIPW} = augmented inverse propensity weighting estimation. \textbf{DidPm} = difference-in-difference estimate of (Reggio Muni - Parma Muni) - (Reggio Other - Parma Other). \textbf{PSMPm} = propensity score matching between Reggio Approach people and people who attended Parma preschools. \textbf{DidPv} = difference-in-difference estimate of (Reggio Muni - Padova Muni) - (Reggio Other - Padova Other). \textbf{PSMPv} = propensity score matching between Reggio Approach people and people who attended Padova preschools. Robust standard errors are reported in parentheses. Bold number shows that the estimate is statistically significant at the 15\% level. Number of observations used in estimation is reported in italic.}
\end{table}

\begin{table}[H] \caption{Estimation Results for Main Outcomes, Comparison to No Preschools, Age-30 Cohort} \label{ols-M-adult30-reg-nopres}
\scalebox{0.65}{\begin{tabular}{l c c c c c c}
\toprule
 & None & BIC & Full & AIPW & DidPm & DidPv \\
\midrule
IQ Score &      0.01 &     -0.06 &     -0.07 &     -0.04 & \textbf{     -0.16 } &     -0.10 \\
& (     0.06 ) & (     0.05 ) & (     0.06 ) & (     0.05 ) & (     0.07 ) & (     0.10 ) \\
& \textit{ 151 } & \textit{ 151 } & \textit{ 151 } & \textit{ 151 } & \textit{ 232 } & \textit{ 217 } \\
IQ Factor &     -0.00 &     -0.22 & \textbf{     -0.28 } &     -0.17 & \textbf{     -0.51 } &     -0.32 \\
& (     0.17 ) & (     0.16 ) & (     0.16 ) & (     0.12 ) & (     0.22 ) & (     0.29 ) \\
& \textit{ 151 } & \textit{ 151 } & \textit{ 151 } & \textit{ 151 } & \textit{ 232 } & \textit{ 217 } \\
Graduate from High School &     -0.02 &      0.05 &      0.05 &      0.06 & \textbf{      0.22 } &     -0.03 \\
& (     0.05 ) & (     0.05 ) & (     0.05 ) & (     0.06 ) & (     0.09 ) & (     0.09 ) \\
& \textit{ 151 } & \textit{ 151 } & \textit{ 151 } & \textit{ 151 } & \textit{ 232 } & \textit{ 217 } \\
High School Grade & \textbf{      5.64 } & \textbf{      6.59 } & \textbf{      6.12 } & \textbf{     7.11} &      0.97 &      4.52 \\
& (     2.04 ) & (     2.26 ) & (     2.41 ) & (     2.11 ) & (     4.52 ) & (     4.29 ) \\
& \textit{ 110 } & \textit{ 110 } & \textit{ 110 } & \textit{ 110 } & \textit{ 174 } & \textit{ 163 } \\
Max Edu: University &     -0.06 &      0.00 &     -0.01 &      0.01 &      0.17 &     -0.12 \\
& (     0.07 ) & (     0.08 ) & (     0.08 ) & (     0.07 ) & (     0.12 ) & (     0.15 ) \\
& \textit{ 151 } & \textit{ 151 } & \textit{ 151 } & \textit{ 151 } & \textit{ 232 } & \textit{ 217 } \\
Employed &      0.05 &      0.03 &      0.06 &      0.02 &      0.07 &      0.04 \\
& (     0.05 ) & (     0.05 ) & (     0.04 ) & (     0.04 ) & (     0.08 ) & (     0.10 ) \\
& \textit{ 151 } & \textit{ 151 } & \textit{ 151 } & \textit{ 151 } & \textit{ 232 } & \textit{ 217 } \\
Hours Worked Per Week & \textbf{      7.93 } & \textbf{      5.10 } & \textbf{      6.27 } & \textbf{     4.28} & \textbf{      7.29 } &      6.82 \\
& (     2.74 ) & (     3.00 ) & (     2.90 ) & (     2.79 ) & (     4.24 ) & (     5.01 ) \\
& \textit{ 124 } & \textit{ 124 } & \textit{ 124 } & \textit{ 124 } & \textit{ 205 } & \textit{ 190 } \\
Married or Cohabitating &      0.00 &     -0.09 &     -0.10 &     -0.08 &     -0.12 &      0.00 \\
& (     0.08 ) & (     0.08 ) & (     0.09 ) & (     0.09 ) & (     0.14 ) & (     0.16 ) \\
& \textit{ 151 } & \textit{ 151 } & \textit{ 151 } & \textit{ 151 } & \textit{ 232 } & \textit{ 217 } \\
Obese &     -0.01 & \textbf{      0.10 } & \textbf{      0.10 } & \textbf{     0.10} &      0.06 & \textbf{      0.28 } \\
& (     0.07 ) & (     0.06 ) & (     0.07 ) & (     0.06 ) & (     0.10 ) & (     0.14 ) \\
& \textit{ 151 } & \textit{ 151 } & \textit{ 151 } & \textit{ 151 } & \textit{ 232 } & \textit{ 217 } \\
Overweight &      0.05 &     -0.04 &     -0.01 &     -0.03 &     -0.11 &     -0.02 \\
& (     0.07 ) & (     0.06 ) & (     0.06 ) & (     0.06 ) & (     0.12 ) & (     0.12 ) \\
& \textit{ 151 } & \textit{ 151 } & \textit{ 151 } & \textit{ 151 } & \textit{ 232 } & \textit{ 217 } \\
Locus of Control - positive &      0.08 &     -0.10 &     -0.11 &     -0.09 & \textbf{     -0.51 } &      0.04 \\
& (     0.14 ) & (     0.13 ) & (     0.13 ) & (     0.14 ) & (     0.26 ) & (     0.26 ) \\
& \textit{ 148 } & \textit{ 148 } & \textit{ 148 } & \textit{ 148 } & \textit{ 221 } & \textit{ 206 } \\
Depression Score - positive & \textbf{      1.57 } &     -0.34 &     -0.35 &     -0.22 &      0.41 &     -0.51 \\
& (     1.01 ) & (     0.88 ) & (     0.91 ) & (     0.92 ) & (     1.52 ) & (     1.95 ) \\
& \textit{ 149 } & \textit{ 149 } & \textit{ 149 } & \textit{ 149 } & \textit{ 230 } & \textit{ 214 } \\
Ever Voted for Municipal & \textbf{      0.19 } &      0.07 &      0.07 &      0.06 &      0.06 &      0.04 \\
& (     0.08 ) & (     0.07 ) & (     0.07 ) & (     0.07 ) & (     0.09 ) & (     0.13 ) \\
& \textit{ 148 } & \textit{ 148 } & \textit{ 148 } & \textit{ 148 } & \textit{ 228 } & \textit{ 207 } \\
Ever Voted for Regional & \textbf{      0.14 } &      0.02 &      0.03 &      0.02 &      0.02 &      0.11 \\
& (     0.08 ) & (     0.07 ) & (     0.07 ) & (     0.08 ) & (     0.09 ) & (     0.13 ) \\
& \textit{ 148 } & \textit{ 148 } & \textit{ 148 } & \textit{ 148 } & \textit{ 228 } & \textit{ 207 } \\
\bottomrule
\end{tabular}
}
\vspace{1ex} \\
\footnotesize\raggedright{Note: This table shows the estimates of the coefficient for attending Reggio Approach preschools from multiple methods. We compare Reggio Approach individuals with those who attended other preschools. Column title indicates the corresponding control set and and model. \textbf{None} = OLS estimate with no control variables. \textbf{BIC} = OLS estimate with controls selected by Bayesian Information Criterion (BIC) and additional controls for male indicator and ITC attendance indicator. \textbf{Full} = OLS estimate with the full set of controls. \textbf{PSM} =  propensity score matching estimation. \textbf{DidPm} = difference-in-difference estimate of (Reggio Muni - Parma Muni) - (Reggio None - Parma None). \textbf{PSMPm} = propensity score matching between Reggio Approach people and Parma people who attended no preschool. \textbf{DidPv} = difference-in-difference estimate of (Reggio Muni - Padova Muni) - (Reggio None - Padova None). \textbf{PSMPv} = propensity score matching between Reggio Approach people and Padova people who attended no preschool. Robust standard errors are reported in parentheses. Bold number shows that the estimate is statistically significant at the 15\% level. Number of observations used in estimation is reported in italic.}
\end{table}


\begin{table}[H] \caption{Estimation Results for Main Outcomes, Comparison to Non-RA Preschools, Age-40 Cohort} \label{ols-M-adult40-reg-pres}
\scalebox{0.65}{\begin{tabular}{l c c c c}
\toprule
 & None & BIC & Full & AIPW \\
\midrule
IQ Score & \textbf{     -0.07 } & \textbf{     -0.07 } & \textbf{     -0.07 } &     -0.09 \\
& (     0.04 ) & (     0.04 ) & (     0.04 ) & (     0.04 ) \\
& \textit{ 159 } & \textit{ 159 } & \textit{ 159 } & \textit{ 159 } \\
IQ Factor &     -0.15 &     -0.12 &     -0.14 &     -0.18 \\
& (     0.12 ) & (     0.12 ) & (     0.11 ) & (     0.13 ) \\
& \textit{ 159 } & \textit{ 159 } & \textit{ 159 } & \textit{ 159 } \\
Graduate from High School & \textbf{      0.13 } & \textbf{      0.10 } & \textbf{      0.12 } &      0.02 \\
& (     0.07 ) & (     0.07 ) & (     0.07 ) & (     0.06 ) \\
& \textit{ 159 } & \textit{ 159 } & \textit{ 159 } & \textit{ 159 } \\
High School Grade &     -0.66 &      0.07 &      0.36 &      0.63 \\
& (     1.56 ) & (     1.65 ) & (     1.71 ) & (     1.58 ) \\
& \textit{ 117 } & \textit{ 117 } & \textit{ 117 } & \textit{ 117 } \\
High School Grade (Standardized) &     -1.13 &      0.02 &      0.36 &      0.74 \\
& (     2.07 ) & (     2.23 ) & (     2.28 ) & (     2.49 ) \\
& \textit{ 116 } & \textit{ 116 } & \textit{ 116 } & \textit{ 116 } \\
Max Edu: University &      0.07 &      0.05 &      0.03 &      0.04 \\
& (     0.06 ) & (     0.05 ) & (     0.05 ) & (     0.05 ) \\
& \textit{ 159 } & \textit{ 159 } & \textit{ 159 } & \textit{ 159 } \\
Employed &      0.01 &      0.01 &      0.01 &      0.03 \\
& (     0.03 ) & (     0.04 ) & (     0.04 ) & (     0.04 ) \\
& \textit{ 159 } & \textit{ 159 } & \textit{ 159 } & \textit{ 159 } \\
Hours Worked Per Week &     -0.90 &     -1.02 &     -1.28 &      0.02 \\
& (     1.93 ) & (     2.09 ) & (     2.20 ) & (     1.83 ) \\
& \textit{ 144 } & \textit{ 144 } & \textit{ 144 } & \textit{ 144 } \\
Married or Cohabitating &      0.03 &      0.02 &      0.02 &      0.01 \\
& (     0.07 ) & (     0.07 ) & (     0.07 ) & (     0.07 ) \\
& \textit{ 159 } & \textit{ 159 } & \textit{ 159 } & \textit{ 159 } \\
Obese &      0.04 &     -0.01 &     -0.04 &     -0.05 \\
& (     0.07 ) & (     0.07 ) & (     0.07 ) & (     0.08 ) \\
& \textit{ 159 } & \textit{ 159 } & \textit{ 159 } & \textit{ 159 } \\
Overweight &     -0.05 &     -0.03 &     -0.03 &      0.04 \\
& (     0.07 ) & (     0.07 ) & (     0.07 ) & (     0.07 ) \\
& \textit{ 159 } & \textit{ 159 } & \textit{ 159 } & \textit{ 159 } \\
Locus of Control - positive &      0.13 &      0.14 &      0.11 &      0.07 \\
& (     0.14 ) & (     0.14 ) & (     0.14 ) & (     0.12 ) \\
& \textit{ 156 } & \textit{ 156 } & \textit{ 156 } & \textit{ 156 } \\
Depression Score - positive &      0.56 & \textbf{      1.43 } &      1.09 & \textbf{     1.10} \\
& (     0.92 ) & (     0.85 ) & (     0.89 ) & (     0.83 ) \\
& \textit{ 156 } & \textit{ 156 } & \textit{ 156 } & \textit{ 156 } \\
Ever Voted for Municipal &     -0.07 &      0.07 &      0.06 & \textbf{     0.10} \\
& (     0.08 ) & (     0.07 ) & (     0.07 ) & (     0.07 ) \\
& \textit{ 151 } & \textit{ 151 } & \textit{ 151 } & \textit{ 151 } \\
Ever Voted for Regional &     -0.05 &      0.08 &      0.07 & \textbf{     0.10} \\
& (     0.08 ) & (     0.07 ) & (     0.07 ) & (     0.07 ) \\
& \textit{ 151 } & \textit{ 151 } & \textit{ 151 } & \textit{ 151 } \\
\bottomrule
\end{tabular}
}
\vspace{1ex} \\
\footnotesize\raggedright{Note: This table shows the estimates of the coefficient for attending Reggio Approach preschools from multiple methods. We compare Reggio Approach individuals with those who attended other preschools. Column title indicates the corresponding control set and and model. \textbf{None} = OLS estimate with no control variables. \textbf{BIC} = OLS estimate with controls selected by Bayesian Information Criterion (BIC) and additional controls for male indicator and ITC attendance indicator. \textbf{Full} = OLS estimate with the full set of controls. \textbf{PSM} =  propensity score matching estimation. \textbf{AIPW} = augmented inverse propensity weighting estimation. \textbf{PSMPm} = propensity score matching between Reggio Approach people and Parma people who attended no preschool.  \textbf{PSMPv} = propensity score matching between Reggio Approach people and Padova people who attended no preschool. Robust standard errors are reported in parentheses. DiD estimates is not available for this cohort due to unavailability of municipal preschool systems in Parma and Padova. Bold number shows that the estimate is statistically significant at the 15\% level. Number of observations used in estimation is reported in italic.}
\end{table}

\begin{table}[H] \caption{Estimation Results for Main Outcomes, Comparison to No Preschools, Age-40 Cohort} \label{ols-M-adult40-reg-nopres}
\scalebox{0.65}{\begin{tabular}{l c c c c c c c c c}
\toprule
 & None & BIC & Full & PSM & AIPW & DidPm & PSMPm & DidPv & PSMPv \\
\midrule
IQ Factor & 0.01 & 0.02 & 0.04 & 0.13 & 0.02 & \textbf{ 0.24 } & \textbf{-0.44} & 0.09 & \textbf{-0.46} \\
& (0.13) & (0.14) & (0.16) & (0.14) & (0.14) & (0.17) & (0.14) & (0.15) & (0.13) \\
& \textit{ 170 } & \textit{ 170 } & \textit{ 170 } & \textit{ 170 } & \textit{ 170 } & \textit{ 357 } & \textit{ 205 } & \textit{ 375 } & \textit{ 165 } \\
Graduate from High School & -0.07 & -0.04 & -0.06 & -0.07 & -0.04 & 0.00 & 0.01 & -0.09 & 0.07 \\
& (0.05) & (0.05) & (0.06) & (0.06) & (0.05) & (0.07) & (0.04) & (0.07) & (0.06) \\
& \textit{ 170 } & \textit{ 170 } & \textit{ 170 } & \textit{ 170 } & \textit{ 170 } & \textit{ 357 } & \textit{ 205 } & \textit{ 375 } & \textit{ 165 } \\
High School Grade & 0.59 & 1.13 & 1.77 & 1.53 & 1.13 & \textbf{ -5.03 } & \textbf{9.07} & 2.42 & \textbf{4.35} \\
& (1.51) & (1.55) & (1.86) & (1.61) & (1.60) & (2.77) & (1.72) & (2.34) & (1.98) \\
& \textit{ 135 } & \textit{ 135 } & \textit{ 135 } & \textit{ 135 } & \textit{ 135 } & \textit{ 289 } & \textit{ 165 } & \textit{ 297 } & \textit{ 130 } \\
High School Grade (Standardized) & 0.43 & 0.81 & 0.94 & 1.54 & 0.76 & -2.94 & 0.54 & 2.65 & -0.28 \\
& (1.89) & (1.93) & (2.40) & (1.97) & (1.55) & (2.47) & (1.65) & (2.69) & (2.46) \\
& \textit{ 135 } & \textit{ 135 } & \textit{ 135 } & \textit{ 135 } & \textit{ 135 } & \textit{ 284 } & \textit{ 161 } & \textit{ 297 } & \textit{ 130 } \\
Max Edu: University & 0.01 & 0.05 & \textbf{ 0.11 } & 0.03 & 0.05 & -0.10 & 0.04 & -0.06 & 0.00 \\
& (0.06) & (0.06) & (0.06) & (0.06) & (0.06) & (0.08) & (0.06) & (0.09) & (0.07) \\
& \textit{ 170 } & \textit{ 170 } & \textit{ 170 } & \textit{ 170 } & \textit{ 170 } & \textit{ 357 } & \textit{ 205 } & \textit{ 375 } & \textit{ 165 } \\
Employed & \textbf{ 0.06 } & \textbf{ 0.05 } & 0.05 & 0.06 & \textbf{0.05} & 0.03 & 0.00 & \textbf{ 0.09 } & 0.03 \\
& (0.04) & (0.04) & (0.03) & (0.04) & (0.04) & (0.05) & (0.03) & (0.06) & (0.03) \\
& \textit{ 169 } & \textit{ 169 } & \textit{ 169 } & \textit{ 169 } & \textit{ 169 } & \textit{ 356 } & \textit{ 205 } & \textit{ 374 } & \textit{ 165 } \\
Hours Worked Per Week & \textbf{ 5.71 } & \textbf{ 6.51 } & \textbf{ 7.39 } & \textbf{7.43} & \textbf{6.46} & \textbf{ 4.67 } & 1.55 & \textbf{ 7.06 } & \textbf{4.22} \\
& (2.42) & (2.40) & (2.60) & (2.54) & (2.57) & (2.89) & (1.63) & (3.08) & (2.03) \\
& \textit{ 151 } & \textit{ 151 } & \textit{ 151 } & \textit{ 151 } & \textit{ 151 } & \textit{ 333 } & \textit{ 192 } & \textit{ 355 } & \textit{ 153 } \\
Married or Cohabitating & 0.02 & -0.01 & 0.05 & -0.00 & -0.01 & -0.06 & \textbf{0.19} & -0.14 & \textbf{0.23} \\
& (0.07) & (0.07) & (0.08) & (0.08) & (0.06) & (0.10) & (0.07) & (0.10) & (0.09) \\
& \textit{ 170 } & \textit{ 170 } & \textit{ 170 } & \textit{ 170 } & \textit{ 170 } & \textit{ 357 } & \textit{ 205 } & \textit{ 375 } & \textit{ 165 } \\
Not Obese & \textbf{ 0.14 } & \textbf{ 0.11 } & 0.01 & 0.12 & \textbf{0.11} & \textbf{ 0.23 } & \textbf{-0.22} & 0.03 & -0.04 \\
& (0.07) & (0.07) & (0.08) & (0.08) & (0.06) & (0.09) & (0.07) & (0.10) & (0.10) \\
& \textit{ 170 } & \textit{ 170 } & \textit{ 170 } & \textit{ 170 } & \textit{ 170 } & \textit{ 357 } & \textit{ 205 } & \textit{ 375 } & \textit{ 165 } \\
Not Overweight & -0.03 & 0.03 & 0.07 & 0.05 & 0.03 & 0.02 & 0.07 & -0.04 & 0.02 \\
& (0.07) & (0.07) & (0.07) & (0.06) & (0.05) & (0.09) & (0.06) & (0.08) & (0.09) \\
& \textit{ 170 } & \textit{ 170 } & \textit{ 170 } & \textit{ 170 } & \textit{ 170 } & \textit{ 357 } & \textit{ 205 } & \textit{ 375 } & \textit{ 165 } \\
Locus of Control - positive & 0.14 & \textbf{ 0.23 } & \textbf{ 0.28 } & \textbf{0.26} & \textbf{0.23} & 0.17 & \textbf{0.24} & \textbf{ 0.31 } & -0.11 \\
& (0.13) & (0.13) & (0.14) & (0.13) & (0.11) & (0.18) & (0.13) & (0.17) & (0.17) \\
& \textit{ 165 } & \textit{ 165 } & \textit{ 165 } & \textit{ 165 } & \textit{ 165 } & \textit{ 340 } & \textit{ 193 } & \textit{ 357 } & \textit{ 156 } \\
Depression Score - positive & \textbf{ 2.25 } & \textbf{ 2.24 } & \textbf{ 2.10 } & \textbf{2.90} & \textbf{2.24} & 0.12 & 0.63 & \textbf{ 2.26 } & -0.81 \\
& (0.92) & (0.94) & (1.07) & (0.98) & (0.89) & (1.18) & (0.82) & (1.18) & (0.91) \\
& \textit{ 168 } & \textit{ 168 } & \textit{ 168 } & \textit{ 168 } & \textit{ 168 } & \textit{ 355 } & \textit{ 205 } & \textit{ 371 } & \textit{ 165 } \\
Ever Voted for Municipal & \textbf{ 0.19 } & \textbf{ 0.15 } & 0.11 & \textbf{0.17} & \textbf{0.16} & -0.03 & \textbf{0.35} & -0.09 & \textbf{0.36} \\
& (0.08) & (0.08) & (0.08) & (0.10) & (0.08) & (0.10) & (0.07) & (0.10) & (0.07) \\
& \textit{ 153 } & \textit{ 153 } & \textit{ 153 } & \textit{ 153 } & \textit{ 153 } & \textit{ 340 } & \textit{ 199 } & \textit{ 340 } & \textit{ 153 } \\
Ever Voted for Regional & \textbf{ 0.20 } & \textbf{ 0.16 } & \textbf{ 0.13 } & \textbf{0.18} & \textbf{0.17} & 0.03 & \textbf{0.43} & -0.10 & \textbf{0.39} \\
& (0.08) & (0.08) & (0.08) & (0.10) & (0.07) & (0.10) & (0.07) & (0.10) & (0.10) \\
& \textit{ 153 } & \textit{ 153 } & \textit{ 153 } & \textit{ 153 } & \textit{ 153 } & \textit{ 340 } & \textit{ 199 } & \textit{ 340 } & \textit{ 153 } \\
\bottomrule
\end{tabular}
}
\vspace{1ex} \\
\footnotesize\raggedright{Note: This table shows the estimates of the coefficient for attending Reggio Approach preschools from multiple methods. We compare Reggio Approach individuals with those who attended other preschools. Column title indicates the corresponding control set and and model. \textbf{None} = OLS estimate with no control variables. \textbf{BIC} = OLS estimate with controls selected by Bayesian Information Criterion (BIC) and additional controls for male indicator and ITC attendance indicator. \textbf{Full} = OLS estimate with the full set of controls. \textbf{PSM} =  propensity score matching estimation. \textbf{AIPW} = augmented inverse propensity weighting estimation. \textbf{DidPm} = difference-in-difference estimate of (Reggio Muni - Parma Other) - (Reggio None - Parma None). \textbf{PSMPm} = propensity score matching between Reggio Approach people and Parma people who attended no preschool. \textbf{DidPv} = difference-in-difference estimate of (Reggio Muni - Padova Other) - (Reggio None - Padova None). \textbf{PSMPv} = propensity score matching between Reggio Approach people and Padova people who attended no preschool. Robust standard errors are reported in parentheses. Bold number shows that the estimate is statistically significant at the 15\% level. Number of observations used in estimation is reported in italic.}
\end{table}

