Evidence from seminal experiments in early childhood interventions, such as the Perry Preschool Program, demonstrates the potential for early childhood education to improve life-cycle outcomes of disadvantaged individuals \citep{Heckman_Moon_etal_2010_QE, Elango_Hojman_etal_2016_Early-Edu}. There are many early childhood interventions that are widely replicated with little empirical evidence of their effectiveness. The Reggio Approach is one such intervention. Since its inception in 1963 in Reggio Emilia, Italy, the Reggio Approach has been implemented in the municipal schools of Reggio Emilia, as well as replicated internationally.\footnote{The official \href{http://www.reggiochildren.it/network/?lang=en}{Reggio Children International Network} is present in 33 countries worldwide. Many other preschools around the world are ``inspired'' by the Reggio Approach but they are not officially part of these network.}

This paper presents an evaluation of the Reggio Approach, which includes infant-toddler centers (ages 0-3) and preschools (ages 3-6). The Reggio Approach is administrated through the municipal government of Reggio Emilia. Other early childhood education options include state preschools and private infant-toddler centers and preschools. We have collected data on individuals who have attended each type of early childhood education, as well as those who were informally cared for outside of a center setting. 

Our sample includes individuals across five age cohorts: three cohorts of adults, one cohort of adolescents, and one cohort of children in their first year of elementary school. The individuals are not only from Reggio Emilia, but also from Padova and Parma, two cities that are similar to Reggio Emilia along several dimensions but have different preschool systems. 

Given the non-experimental nature of the program, evaluating the Reggio Approach presents several challenges. The Reggio Approach preschool system was introduced in 1963, grew over the decades, and now has infrastructure to teach educators about the Reggio Approach. Therefore, the evaluation strategy has to account for potential changes in treatment over time, the lack of well defined control group, and the potential spillover of the Reggio Approach into other programs attended by individuals in the data. 

% Findings from analysis

In Section~\ref{sec:eceexperiences}, we highlight the similarities and differences between the Reggio Approach and the other early childhood education experiences in the three cities. Section~\ref{sec:data} describes the data in more detail and discusses the outcome variables that are available for the different cohorts. We also present an overview of the demographic aspects of Reggio Emilia, Parma, and Padova that contextualize the different approaches to early childhood education. Section~\ref{sec:methodology} discusses the methodology we employ to produce the results discussed in Section~\ref{sec:results}. 
