The Reggio Children Approach to Early Childhood Education (the Reggio Approach) is a birth to age-6 universal early childhood program developed in Reggio Emilia, Italy  in the early 1960s. The Reggio Approach, strongly influenced by Loris Malaguzzi, has developed a more progressive vision of the child as an individual with rights and potential. It has been a source of inspiration for hundreds of early childhood centers in many countries around the world\footnote{The official \href{http://www.reggiochildren.it/network/?lang=en}{Reggio Children International Network} is present in 33 countries worldwide.} and has received several prizes. Despite its worldwide recognition, the Reggio Approach has never been formally evaluated, and there is little empirical evidence of its effectiveness. 

This paper presents an evaluation of the Reggio Approach. The study utilizes non-experimental data on individuals from five different age cohorts -- three cohorts of adults, one cohort of adolescents, and one cohort of children in their first year of elementary school -- and three different cities -- Reggio Emilia, Parma, and Padova. The individuals in each city are exposed to one of four different early childhood experiences: Municipal, State, Private, or none. The Reggio Approach is delivered by the Municipal institutions of Reggio Emilia, and we thus, define the treatment group to include those individuals who attended a Municipal institution in Reggio Emilia. Our basic strategy involves comparing the outcomes of those who attended Municipal institutions in Reggio Emilia (treatment group) to those who experienced other combinations of city-school experiences (control group).

The evaluation of the Reggio Approach faces several challenges stemming from the nature of the data. Firstly, the non-experimental nature of the data gives rise to concerns about selection of individuals into treatment. In the absence of a randomized controlled trial, we employ a number of econometric techniques to control for this potential selection. Secondly, the prevalence of other high-quality childcare programs in Northern Italy combined with the large propensity of individuals in our sample to attend childcare programs leads to heterogeneity in the experiences of the control group.\footnote{In the late 20th century, Northern Italy witnessed a rise in local early childhood investments many of which were influenced by Malaguzzi as well as other respected early childhood experts \citep{OECD_2001_Italy-Country-Note}. This rise in quality of childcare alternatives was accompanied by an increase in the preschool attendance rate of Italian children aged 3-6 years from 50\% in the 1960s to 96\% in the 1990s \citep{Hohnerlein_2015_Development-and-Diffusion}.} Given that a portion of our control group substitute the Reggio Approach with other good quality childcare programs while the remainder do not, we need to qualify our inference of the results as follows: (i) When comparing individuals who attended Reggio Approach with those who attended other childcare programs, the estimates should be interpreted as capturing the benefits stemming from attending the Reggio Approach relative to other institutions, and (ii) When comparing individuals who attended Reggio Approach with those who didn't attend any childcare, the estimates are to be interpreted as capturing the benefits of attending Reggio relative to not attending any childcare.

In contextualizing our findings, it is essential to understand the heterogeneity in early childhood experiences across school types, cities, and cohorts. Towards this end, section \ref{sec:ece-italy} presents key findings from an extensive review of the literature as well as results from a novel survey we undertook to quantify differences in administrative and curricular components between city-school type combinations. The survey is especially helpful because it allows us to track the evolution of these differences over cohorts. In summary, results of the survey show that non-Reggio Approach schools have historically shared some characteristics with Reggio Approach schools, and that the overlap of these characteristics increases over time. Given the overlap in these characteristics, it is reasonable to expect our comparisons with individuals who attended alternative programs to yield lower treatment effects than our comparisons with individuals who attended no program.

The results differ across age cohorts and across comparisons with different control groups. With the exception of some noncognitive outcomes, we do not find any consistently significant positive effects of the Reggio Approach on the children and adolescent cohorts. Our strongest results emerge in the adult 40 cohort when comparing Reggio Approach individuals with those from Reggio Emilia who did not attend preschool. Positive and significant effects are estimated for employment, noncognitive skills, and voting behavior. These effects disappear when we compare Reggio Approach individuals with those who attended alternative preschools within the city. The lack of positive and significant results persist when making comparisons against those who attended municipal programs in other cities, especially Padova.\footnote{This is consistent with historical information about the lower availability of alternative preschools at this time and the unavailability of the municipal system in Padova before the age-30 cohorts.} These results suggest that attending Reggio Approach preschools offer distinct advantages relative to not attending any preschools. However, the benefits are less pronounced when comparing the Reggio Approach to other types of preschools. 

The remainder of the document is structured as follows. Section \ref{sec:ece-italy} describes the Reggio Approach and availability of other childcare programs of the three cities drawing from historical records. Section \ref{sec:ece-comparison} compare the components of the Reggio Approach and other programs based on an original survey administered to officials across the different school systems. Section \ref{sec:data} describes the research design, including the selection of cities, the survey data collection, and the questionnaires. Section \ref{sec:methodology} presents the methods used to estimate the treatment effects of the Reggio Approach. Section \ref{sec:result} presents estimation results. Section \ref{sec:discussion} discusses the results in the context of historical information on different childcare programs. 

