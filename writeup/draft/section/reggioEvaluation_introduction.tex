Evidence from seminal experiments in early childhood interventions, such as the Perry Preschool Program, demonstrates the potential for early childhood education to improve life-cycle outcomes of disadvantaged children.\footnote{\citet{Heckman_Moon_etal_2010_QE, Elango_Hojman_etal_2016_Early-Edu}.} There are many early childhood interventions that are widely replicated with little empirical evidence of their effectiveness. The Reggio Approach is one such intervention. Since its inception in 1963 in Reggio Emilia, Italy, the Reggio Approach has been implemented in the municipal schools of Reggio Emilia, as well as replicated internationally.\footnote{The official \href{http://www.reggiochildren.it/network/?lang=en}{Reggio Children International Network} is present in 33 countries worldwide. Many other preschools around the world are ``inspired'' by the Reggio Approach but they are not officially part of these network.}

This memo presents a discussion of the evaluation of the Reggio Approach, which includes infant-toddler centers (ages 0-3) and preschools (ages 3-6). The Reggio Approach is administrated through the municipal government of Reggio Emilia. Other early childhood education options include state preschools and private infant-toddler centers and preschools. We have collected data on individuals who have attended each type of early childhood education, as well as those who were informally cared for outside of a center setting. Our sample includes individuals across five age cohorts: three cohorts of adults, one cohort of adolescents, and one cohort of children in their first year of elementary school. The individuals are not only from Reggio Emilia, but also from Padova and Parma, two cities that are similar to Reggio Emilia along several dimensions but have different preschool systems. 

This memo focuses on a discussion of the evaluation in order to highlight the methodological issues arising from the data collection. Much additional work was done to understand the evolutions of the different preschool programs. The appendix offers more detail into the specifics of the data, programs, evaluation, and results.

