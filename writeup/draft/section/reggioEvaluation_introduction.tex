The Reggio Approach is a birth to age-6 universal early childhood program implemented in Reggio Emilia, Italy in the early 1960s. It is based on a vision of the child as an individual with rights and potential. It has been a source of inspiration for hundreds of early childhood centers around the world\footnote{The official \href{http://www.reggiochildren.it/network/?lang=en}{Reggio Children International Network} is present in 33 countries worldwide.} and Reggio schools have been awarded prizes.\footnote{Examples include the Danish LEGO Prize (1992), the Kohl Foundation of Chicago award (1993), the Hans Christian Anderson Prize (1994), the Mediterranean Association of International Schools award (1994), the award from the French city of Blois (2001).} Despite its widespread recognition, the Reggio Approach has never been formally evaluated and there is no rigorous empirical evidence of its effects on children's life-cycle outcomes.

This paper presents an evaluation of the Reggio Approach using non-experimental comparison groups constructed from data on individuals from five different age cohorts (three cohorts of adults, one cohort of adolescents, and one cohort of children in their first year of elementary school) and three different cities (Reggio Emilia, Parma, and Padova). Although Parma and Padova are geographically close to Reggio Emilia and similar along economic and demographic measures, they have somewhat different preschool systems as described below. The individuals in each city are exposed to one of four different early childhood experiences: municipal, state, religious, or none. The Reggio Approach is delivered through the municipal early childhood schools of Reggio Emilia. Our evaluation strategy involves comparing the outcomes of those who attended municipal institutions in Reggio Emilia (treatment group) to those who experienced other preschool school types either in Reggio Emilia or in Parma and Padova (control group).

The evaluation of the Reggio Approach faces several challenges. First, the non-experimental nature of the data raises concerns about selective differences in the samples compared. We employ a number of econometric techniques in an attempt to control for selection problems. Second, there is a prevalence of other high-quality childcare programs in Northern Italy that enroll many youth. In the late 20th century, Northern Italy witnessed a rise in local early childhood programs many of which were influenced by Loris Malaguzzi as well as other respected early childhood experts \citep{OECD_2001_Italy-Country-Note}. This rise in quality of childcare alternatives was accompanied by an increase in the preschool attendance rate of Italian children aged 3-6 years from 50\% in the 1960s to 96\% in the 1990s \citep{Hohnerlein_2015_Development-and-Diffusion}. The common influences across areas poses serious problems for any analysis based on comparison groups across cities in the region. \textbf{[JJH: But how much for non-Reggio against nothing?]} The evidence of common preschool practices in Northern Italy is consistent with two interpretations: (i) that a common influence was at work (e.g., Malaguzzi); or (ii) that the Reggio program was unique, but its essential elements diffused rapidly across towns and alternative schools within the same towns. We make the following counterfactual comparisons. (i) Comparing individuals who attended Reggio Approach programs with those who attended other center-based programs within and across Reggio. These estimates capture the benefits of attending the Reggio Approach programs relative to other center-based programs. (ii) Comparing individuals who attended Reggio Approach schools with those who did not attend any center-based program. These estimates capture the benefits of attending Reggio Approach programs relative to staying at home or receiving informal childcare outside of a center setting.

In contextualizing our findings, it is essential to understand the heterogeneity in early childhood experiences across school types, cities, and cohorts. Towards this end, Section~\ref{sec:ece-italy} presents key findings from an extensive review of the literature as well as results from a survey we undertook to quantify differences in administrative and pedagogical components among the different school types in the three cities. The survey allows us to track the evolution of these differences across cities and schools within cities over time to align with the age cohorts in the sample. Results from the survey show that non-Reggio-Approach schools have historically shared many of the same characteristics with Reggio Approach schools, and that commonalities of these characteristics increase over time. Given the overlap in these characteristics, it is reasonable to expect comparisons with individuals who attended alternative programs to yield lower treatment effects than our comparisons with individuals who attended no program.

The results differ across age cohorts and across comparisons with different control groups. With the exception of some non-cognitive outcomes, we do not find any consistently significant positive effects of the Reggio Approach on the children and adolescent cohorts. Our strongest favorable comparisons emerge in the adult 40 cohort when comparing Reggio Approach individuals with those from Reggio Emilia who did not attend preschool. Positive and significant effects are estimated for employment, non-cognitive skills, and voting behavior. 

These effects disappear when we compare Reggio Approach attendees with those who attended alternative preschools within the city. The lack of positive and significant results persist when we make comparisons against those who attended municipal programs in other cities, especially Padova.\footnote{This is consistent with historical information about the lower availability of alternative preschools at this time and the unavailability of the municipal system in Padova before the age-30 cohorts.} These results suggest that attending Reggio Approach preschools offer distinct advantages relative to not attending center-based programs. However, the benefits are less pronounced when comparing the Reggio Approach to other types of preschools. \textbf{[JJH: Compare the comparisons each against the null.]}

The remainder of the document is structured as follows. Section \ref{sec:ece-italy} describes the Reggio Approach and other childcare programs of the three cities drawing from historical records and an original survey administered to officials across the different school systems. Section \ref{sec:data} describes the research design, including the selection of cities, the survey data collection, and the questionnaires. Section \ref{sec:methodology} presents the methods used to estimate the treatment effects of the Reggio Approach. Section \ref{sec:result} presents estimation results. Section \ref{sec:discussion} discusses the results in the context of historical information on different childcare programs.

