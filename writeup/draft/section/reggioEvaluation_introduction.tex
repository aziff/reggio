Although evidence from seminal experiments in early childhood education demonstrates the potential for early childhood education to improve life-cycle outcomes, there are many early childhood interventions implemented with little empirical evidence of their effectiveness.\footnote{For an example of an analysis of the Perry Preschool Program, see \citet{Heckman_Moon_etal_2010_QE}. See \citet{Elango_Hojman_etal_2016_Early-Edu} for an overview of the literature evaluating early childhood education.} The Reggio Approach is one such intervention. Since its inception in 1963 in Reggio Emilia, Italy, the Reggio Approach has been implemented in the municipal schools of Reggio Emilia, as well as replicated internationally.\footnote{The official \href{http://www.reggiochildren.it/network/?lang=en}{Reggio Children International Network} is present in 33 countries worldwide.}

This paper presents an evaluation the Reggio Approach.\footnote{Although the Reggio Approach includes infant-toddler centers (ages 0-3) and preschools (ages 3-6), we focus our evaluation on the preschools. See Appendix~\ref{sec:asilo_results} for an analysis of the infant-toddler centers.} The Reggio Approach is administrated through the municipal government of Reggio Emilia. Other early childhood education options include state preschools and religious infant-toddler centers and preschools. 

We have collected data on individuals who have attended each type of early childhood education, as well as those who were informally cared for outside of a center setting. Our sample includes individuals across five age cohorts: three cohorts of adults, one cohort of adolescents, and one cohort of children in their first year of elementary school. The individuals are not only from Reggio Emilia, but also from Padova and Parma, two cities that are similar to Reggio Emilia along several dimensions but have different preschool systems. 

In addition to describing the programs (Section~\ref{sec:eceexperiences}), data (Section~\ref{sec:data}), methods (Section~\ref{sec:methodology}), and results (Section~\ref{sec:result}), we discuss challenges with collecting quality data and the historical context making construction of a suitable comparison group difficult. We complement discussion of our empirical results with uniquely collated historical records and an original survey administered to officials in the different school systems.

We find the strongest results comparing adults in their 40s who attended the Reggio Approach with those in Reggio Emilia who did not attend any preschool. These results are also seen when comparing to those who attended the municipal programs in other cities, especially Padova. This is consistent with historical information about the lower availability of alternative preschools at this time and the municipal system in Padova, which formed after the age-40 cohort was preschool age.
