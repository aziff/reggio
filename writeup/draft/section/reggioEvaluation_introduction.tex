
The Reggio Children Approach to Early Childhood Education (the Reggio Approach) is an educational philosophy developed in Reggio Emilia by Loris Malaguzzi in the mid-1900s. This approach has been a source of inspiration for hundreds of early childhood centers that have been created in many countries and in a wide variety of contexts.\footnote{The official \href{http://www.reggiochildren.it/network/?lang=en}{Reggio Children International Network} is present in 33 countries worldwide.} Despite its worldwide recognition, the Reggio Approach has never been formally evaluated, and there is little empirical evidence of its effectiveness. 

The evaluation of the Reggio Approach faces several challenges, given the non-experimental nature of the program and prevalence of other high-quality childcare programs. The northern Italy in the 20th century experienced the development of the early childhood programs that are influenced not only by Malaguzzi but also by other educators. Evidence shows that northern Italy has high-quality, universal early childhood services including the Reggio Approach \citep{OECD_2001_Italy-Country-Note}. Moreover, the take-up rate of childcare in Italy has been increasingly high. The preschool attendence rate of Italian children aged 3-6 years has increased from 50\% in the 1960s to 96\% in the 1990s \citep{Hohnerlein_2015_Development-and-Diffusion}. Therefore, children who are not enrolled in the Reggio Approach preschools have the option of attending other well-established childcare, which makes it difficult to evaluate the effect of the Reggio Approach \textit{per se}. 

This paper presents an evaluation of the Reggio Approach accounting for the challenges mentioned above. Our goal is to compare the Reggio Approach to other programs and understand the similarities and differences between the Reggio Approach and other programs. We have collected data on individuals who have attended each type of early childhood education, as well as those who were informally cared for outside of a center setting. Our sample includes individuals across five age cohorts: three cohorts of adults, one cohort of adolescents, and one cohort of children in their first year of elementary school. The individuals are not only from Reggio Emilia, but also from Padova and Parma, two cities that are similar to Reggio Emilia along several dimensions but have different preschool systems.  

In addition to describing the programs (Section~\ref{sec:eceexperiences}), data (Section~\ref{sec:data}), methods (Section~\ref{sec:methodology}), and results (Section~\ref{sec:result}), we discuss challenges with collecting quality data as well as the historical context that has made construction of a suitable comparison group difficult. We complement discussion of our empirical results with uniquely collated historical records and an original survey administered to officials across the different school systems.

%We find the strongest results comparing adults in their 40s who attended the Reggio Approach with those in Reggio Emilia who did not attend any preschool. These results persist when we make comparisons against those who attended municipal programs in other cities, especially Padova. This is consistent with historical information about the lower availability of alternative preschools at this time and the unavailability of the municipal system in Padova before the age-30 cohorts.

%Although evidence from seminal experiments in early childhood education demonstrates the potential for early childhood education to improve life-cycle outcomes, there are many early childhood interventions implemented with little empirical evidence of their effectiveness.\footnote{For an example of an analysis of the Perry Preschool Program, see \citet{Heckman_Moon_etal_2010_QE}. See \citet{Elango_Hojman_etal_2016_Early-Edu} for an overview of the literature evaluating early childhood education.} The Reggio Approach is one such intervention. Since its inception in 1963 in Reggio Emilia, Italy, the Reggio Approach has been implemented in the municipal schools of Reggio Emilia, as well as replicated internationally.\footnote{The official \href{http://www.reggiochildren.it/network/?lang=en}{Reggio Children International Network} is present in 33 countries worldwide.}

%This paper presents an evaluation the Reggio Approach.\footnote{Although the Reggio Approach includes infant-toddler centers (ages 0-3) and preschools (ages 3-6), we focus our evaluation on the preschools. See Appendix~\ref{sec:asilo_results} for an analysis of the infant-toddler centers.} The Reggio Approach is administrated through the municipal government of Reggio Emilia. Other early childhood education options include state preschools and religious infant-toddler centers and preschools. 

%We have collected data on individuals who have attended each type of early childhood education, as well as those who were informally cared for outside of a center setting. Our sample includes individuals across five age cohorts: three cohorts of adults, one cohort of adolescents, and one cohort of children in their first year of elementary school. The individuals are not only from Reggio Emilia, but also from Padova and Parma, two cities that are similar to Reggio Emilia along several dimensions but have different preschool systems. 

%In addition to describing the programs (Section~\ref{sec:eceexperiences}), data (Section~\ref{sec:data}), methods (Section~\ref{sec:methodology}), and results (Section~\ref{sec:result}), we discuss challenges with collecting quality data as well as the historical context that has made construction of a suitable comparison group difficult. We complement discussion of our empirical results with uniquely collated historical records and an original survey administered to officials across the different school systems.

%We find the strongest results comparing adults in their 40s who attended the Reggio Approach with those in Reggio Emilia who did not attend any preschool. These results persist when we make comparisons against those who attended municipal programs in other cities, especially Padova. This is consistent with historical information about the lower availability of alternative preschools at this time and the unavailability of the municipal system in Padova before the age-30 cohorts.
