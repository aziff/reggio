
The Reggio Children Approach to Early Childhood Education (the Reggio Approach) is a birth to age-6 early childhood program developed in Reggio Emilia, Italy by Loris Malaguzzi in the early 1960s. This approach has been a source of inspiration for hundreds of early childhood centers in many countries around the world.\footnote{The official \href{http://www.reggiochildren.it/network/?lang=en}{Reggio Children International Network} is present in 33 countries worldwide.} Despite its worldwide recognition, the Reggio Approach has never been formally evaluated, and there is little empirical evidence of its effectiveness. 

The evaluation of the Reggio Approach faces several challenges given the non-experimental nature of the program, and the prevalence of other high-quality childcare programs in Northern Italy. First of all, there has not been any randomized experiment on the Reggio Approach. Furthermore, almost all children in Norhtern Italy after the 1990s attended some form of preschool, making it difficult to identify a reliable comparison group. The preschool attendance rate of Italian children aged 3-6 years increased from 50\% in the 1960s to 96\% in the 1990s \citep{Hohnerlein_2015_Development-and-Diffusion}. In the late 20th century, Northern Italy witnessed a further rise in local early childhood investments including high-quality universal educational programs and services. These programs were influenced by Malaguzzi as well as other respected early childhood experts \citep{OECD_2001_Italy-Country-Note}. Therefore, children who were not enrolled in the Reggio Approach preschools had the option of attending other established childcare.  The combination of all these factors makes it difficult to evaluate the effect of the Reggio Approach \textit{per se}. 

This paper presents an evaluation of the Reggio Approach trying to account for the challenges mentioned above. We compare the Reggio Approach to other programs (e.g. state and religious programs) and study the similarities and differences between these programs. We collected data on individuals who have attended each type of early childhood education, as well as those who were informally cared for outside of a center setting. Our sample includes individuals across five age cohorts: three cohorts of adults, one cohort of adolescents, and one cohort of children in their first year of elementary school. These individuals are sampled not only from Reggio Emilia, but also from Padova and Parma, two cities that are similar to Reggio Emilia along several dimensions but were not home to the birth and development of the Reggio Approach system.

Since our study comprises of individuals who attended different school types in different cities during different time periods, we describe in greater detail the Italian early childcare systems, highlighting the similarities and differences of administrative and pedagogical components of the Reggio Approach and other childcare systems over time. Section \ref{sec:eceexperiences} describes the childcare programs of these three cities drawing from uniquely collated historical records and an original survey administered to officials across the different school systems. Section \ref{sec:data} describes the research design, including the selection of cities, the survey data collection, and the questionnaires. Section \ref{sec:methodology} presents the methods used to estimate the treatment effects of the Reggio Approach. Section \ref{sec:result} presents estimation results. Section \ref{sec:discussion} discusses the results in the context of historical information on different childcare programs. 

The results differ across age cohorts and across comparisons with different control groups. Children and adolescents cohorts find no consistently significant positive effects of the Reggio Approach except for some noncognitive outcomes. We find the strongest results comparing adults in their 40s who attended the Reggio Approach with those in Reggio Emilia who did not attend any preschool. Significantly positive effects are shown on employment, noncognitive skills, and voting behavior with this comparison. However, these effects disappear when we compare adults in their 40s with those who attended other types of preschool. These results persist when we make comparisons against those who attended municipal programs in other cities, especially Padova. This is consistent with historical information about the lower availability of alternative preschools at this time and the unavailability of the municipal system in Padova before the age-30 cohorts.
