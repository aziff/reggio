\begin{landscape}

\subsection{Summary Statistics of Outcomes}
\singlespace
\setlength{\tabcolsep}{2pt}
\begin{center}
\scriptsize{
\begin{longtable}{L{5cm} c c c p{.5cm} c c c p{.5cm} c c c p{.5cm} c c c p{.5cm} c c c}
\hline\multicolumn{20}{L{24cm}}{\textbf{Note:} Means are reported for each variable by cohort and city. Standard Deviations are reported in italics below each mean estimate. A . denotes that the variable is not defined for a specific cohort.}
\endfoot
\caption{Summary statistics for outcome variables by cohort and city} \label{table:summaryStat_outcomes} \\
\hline
& \multicolumn{3}{c}{\textbf{Children}} & & \multicolumn{3}{c}{\textbf{Adolescents}} & & \multicolumn{3}{c}{\textbf{Adults 30}} & & \multicolumn{3}{c}{\textbf{Adults 40}} & & \multicolumn{3}{c}{\textbf{Adults 50}}\\
& \scriptsize{Reggio} & \scriptsize{Parma}& \scriptsize{Padova} & & \scriptsize{Reggio} & \scriptsize{Parma}& \scriptsize{Padova} & & \scriptsize{Reggio} & \scriptsize{Parma}& \scriptsize{Padova} & & \scriptsize{Reggio} & \scriptsize{Parma}& \scriptsize{Padova} & & \scriptsize{Reggio} & \scriptsize{Parma}& \scriptsize{Padova}\\
\hline \\[.2em] \endhead
\textbf{Cognitive} \\[.6em]
 \quad IQ Factor & -0.07 &      0.26 &      0.01 & &      0.15 &      0.16 &     -0.30 & &     -0.19 &      0.47 &      0.43 & &      0.09 &      0.45 &      0.31 & &      0.58 &      0.32 &      0.45 \\*
 \quad & $\mathit{     0.93}$ & $\mathit{     0.79}$ & $\mathit{     1.08}$ & & $\mathit{     0.85}$ & $\mathit{     0.67}$ & $\mathit{     1.19}$ & & $\mathit{     0.93}$ & $\mathit{     0.54}$ & $\mathit{     0.62}$ & & $\mathit{     0.80}$ & $\mathit{     0.58}$ & $\mathit{     0.76}$ & & $\mathit{     0.40}$ & $\mathit{     0.57}$ & $\mathit{     0.48}$ \\[.2em]
 \quad IQ Score & 0.60 &      0.69 &      0.64 & &      0.77 &      0.76 &      0.65 & &      0.57 &      0.78 &      0.78 & &      0.66 &      0.77 &      0.72 & &      0.84 &      0.71 &      0.77 \\*
 \quad & $\mathit{     0.22}$ & $\mathit{     0.18}$ & $\mathit{     0.24}$ & & $\mathit{     0.25}$ & $\mathit{     0.21}$ & $\mathit{     0.34}$ & & $\mathit{     0.33}$ & $\mathit{     0.21}$ & $\mathit{     0.22}$ & & $\mathit{     0.29}$ & $\mathit{     0.22}$ & $\mathit{     0.25}$ & & $\mathit{     0.15}$ & $\mathit{     0.20}$ & $\mathit{     0.19}$ \\[.2em]
 ~\\[-.5em]
\textbf{NonCognitive} \\[.6em]
 \quad SDQ Composite - Child & 32.23 &     33.18 &     32.32 & &     32.60 &     32.79 &     32.78 & &         . &         . &         . & &         . &         . &         . & &         . &         . &         . \\*
 \quad & $\mathit{     4.87}$ & $\mathit{     4.46}$ & $\mathit{     4.73}$ & & $\mathit{     4.97}$ & $\mathit{     4.98}$ & $\mathit{     4.36}$ & & $\mathit{        .}$ & $\mathit{        .}$ & $\mathit{        .}$ & & $\mathit{        .}$ & $\mathit{        .}$ & $\mathit{        .}$ & & $\mathit{        .}$ & $\mathit{        .}$ & $\mathit{        .}$ \\[.2em]
 \quad SDQ Pro-social - Child & 2.15 &      2.17 &      2.22 & &      2.35 &      2.26 &      2.53 & &         . &         . &         . & &         . &         . &         . & &         . &         . &         . \\*
 \quad & $\mathit{     1.79}$ & $\mathit{     1.76}$ & $\mathit{     1.82}$ & & $\mathit{     1.91}$ & $\mathit{     1.76}$ & $\mathit{     1.71}$ & & $\mathit{        .}$ & $\mathit{        .}$ & $\mathit{        .}$ & & $\mathit{        .}$ & $\mathit{        .}$ & $\mathit{        .}$ & & $\mathit{        .}$ & $\mathit{        .}$ & $\mathit{        .}$ \\[.2em]
 \quad SDQ Peer problems - Child & 8.93 &      8.82 &      8.72 & &      8.68 &      8.64 &      8.77 & &         . &         . &         . & &         . &         . &         . & &         . &         . &         . \\*
 \quad & $\mathit{     1.32}$ & $\mathit{     1.47}$ & $\mathit{     1.57}$ & & $\mathit{     1.57}$ & $\mathit{     1.40}$ & $\mathit{     1.41}$ & & $\mathit{        .}$ & $\mathit{        .}$ & $\mathit{        .}$ & & $\mathit{        .}$ & $\mathit{        .}$ & $\mathit{        .}$ & & $\mathit{        .}$ & $\mathit{        .}$ & $\mathit{        .}$ \\[.2em]
 \quad SDQ Hyper - Child & 6.74 &      7.28 &      6.86 & &      7.80 &      7.95 &      7.48 & &         . &         . &         . & &         . &         . &         . & &         . &         . &         . \\*
 \quad & $\mathit{     2.30}$ & $\mathit{     2.17}$ & $\mathit{     2.19}$ & & $\mathit{     1.97}$ & $\mathit{     2.04}$ & $\mathit{     1.95}$ & & $\mathit{        .}$ & $\mathit{        .}$ & $\mathit{        .}$ & & $\mathit{        .}$ & $\mathit{        .}$ & $\mathit{        .}$ & & $\mathit{        .}$ & $\mathit{        .}$ & $\mathit{        .}$ \\[.2em]
 \quad SDQ Emotional - Child & 8.22 &      8.51 &      8.35 & &      7.65 &      7.64 &      7.99 & &         . &         . &         . & &         . &         . &         . & &         . &         . &         . \\*
 \quad & $\mathit{     1.81}$ & $\mathit{     1.61}$ & $\mathit{     1.61}$ & & $\mathit{     2.06}$ & $\mathit{     2.13}$ & $\mathit{     1.73}$ & & $\mathit{        .}$ & $\mathit{        .}$ & $\mathit{        .}$ & & $\mathit{        .}$ & $\mathit{        .}$ & $\mathit{        .}$ & & $\mathit{        .}$ & $\mathit{        .}$ & $\mathit{        .}$ \\[.2em]
 \quad SDQ Conduct - Child & 8.35 &      8.56 &      8.39 & &      8.46 &      8.56 &      8.56 & &         . &         . &         . & &         . &         . &         . & &         . &         . &         . \\*
 \quad & $\mathit{     1.49}$ & $\mathit{     1.44}$ & $\mathit{     1.46}$ & & $\mathit{     1.43}$ & $\mathit{     1.52}$ & $\mathit{     1.43}$ & & $\mathit{        .}$ & $\mathit{        .}$ & $\mathit{        .}$ & & $\mathit{        .}$ & $\mathit{        .}$ & $\mathit{        .}$ & & $\mathit{        .}$ & $\mathit{        .}$ & $\mathit{        .}$ \\[.2em]
 \quad SDQ Composite & . &         . &         . & &     30.77 &     31.78 &     31.09 & &         . &         . &         . & &         . &         . &         . & &         . &         . &         . \\*
 \quad & $\mathit{        .}$ & $\mathit{        .}$ & $\mathit{        .}$ & & $\mathit{     5.33}$ & $\mathit{     4.90}$ & $\mathit{     5.15}$ & & $\mathit{        .}$ & $\mathit{        .}$ & $\mathit{        .}$ & & $\mathit{        .}$ & $\mathit{        .}$ & $\mathit{        .}$ & & $\mathit{        .}$ & $\mathit{        .}$ & $\mathit{        .}$ \\[.2em]
 \quad SDQ Pro-social & . &         . &         . & &      2.37 &      2.19 &      2.78 & &         . &         . &         . & &         . &         . &         . & &         . &         . &         . \\*
 \quad & $\mathit{        .}$ & $\mathit{        .}$ & $\mathit{        .}$ & & $\mathit{     1.76}$ & $\mathit{     1.75}$ & $\mathit{     1.82}$ & & $\mathit{        .}$ & $\mathit{        .}$ & $\mathit{        .}$ & & $\mathit{        .}$ & $\mathit{        .}$ & $\mathit{        .}$ & & $\mathit{        .}$ & $\mathit{        .}$ & $\mathit{        .}$ \\[.2em]
 \quad SDQ Peer problems & . &         . &         . & &      8.62 &      8.57 &      8.55 & &         . &         . &         . & &         . &         . &         . & &         . &         . &         . \\*
 \quad & $\mathit{        .}$ & $\mathit{        .}$ & $\mathit{        .}$ & & $\mathit{     1.47}$ & $\mathit{     1.31}$ & $\mathit{     1.61}$ & & $\mathit{        .}$ & $\mathit{        .}$ & $\mathit{        .}$ & & $\mathit{        .}$ & $\mathit{        .}$ & $\mathit{        .}$ & & $\mathit{        .}$ & $\mathit{        .}$ & $\mathit{        .}$ \\[.2em]
 \quad SDQ Hyper & . &         . &         . & &      6.80 &      7.45 &      6.85 & &         . &         . &         . & &         . &         . &         . & &         . &         . &         . \\*
 \quad & $\mathit{        .}$ & $\mathit{        .}$ & $\mathit{        .}$ & & $\mathit{     2.14}$ & $\mathit{     2.14}$ & $\mathit{     1.99}$ & & $\mathit{        .}$ & $\mathit{        .}$ & $\mathit{        .}$ & & $\mathit{        .}$ & $\mathit{        .}$ & $\mathit{        .}$ & & $\mathit{        .}$ & $\mathit{        .}$ & $\mathit{        .}$ \\[.2em]
 \quad SDQ Emotional & . &         . &         . & &      7.17 &      7.40 &      7.46 & &         . &         . &         . & &         . &         . &         . & &         . &         . &         . \\*
 \quad & $\mathit{        .}$ & $\mathit{        .}$ & $\mathit{        .}$ & & $\mathit{     2.24}$ & $\mathit{     2.17}$ & $\mathit{     2.17}$ & & $\mathit{        .}$ & $\mathit{        .}$ & $\mathit{        .}$ & & $\mathit{        .}$ & $\mathit{        .}$ & $\mathit{        .}$ & & $\mathit{        .}$ & $\mathit{        .}$ & $\mathit{        .}$ \\[.2em]
 \quad SDQ Conduct & . &         . &         . & &      8.18 &      8.36 &      8.23 & &         . &         . &         . & &         . &         . &         . & &         . &         . &         . \\*
 \quad & $\mathit{        .}$ & $\mathit{        .}$ & $\mathit{        .}$ & & $\mathit{     1.59}$ & $\mathit{     1.57}$ & $\mathit{     1.58}$ & & $\mathit{        .}$ & $\mathit{        .}$ & $\mathit{        .}$ & & $\mathit{        .}$ & $\mathit{        .}$ & $\mathit{        .}$ & & $\mathit{        .}$ & $\mathit{        .}$ & $\mathit{        .}$ \\[.2em]
 \quad Depression Score - positive & . &         . &         . & &     37.14 &     37.89 &     38.61 & &     37.80 &     39.21 &     38.94 & &     38.83 &     39.51 &     39.10 & &     37.62 &     38.04 &     35.79 \\*
 \quad & $\mathit{        .}$ & $\mathit{        .}$ & $\mathit{        .}$ & & $\mathit{     6.51}$ & $\mathit{     5.03}$ & $\mathit{     5.95}$ & & $\mathit{     5.83}$ & $\mathit{     5.92}$ & $\mathit{     5.55}$ & & $\mathit{     5.87}$ & $\mathit{     5.30}$ & $\mathit{     5.61}$ & & $\mathit{     5.16}$ & $\mathit{     4.69}$ & $\mathit{     6.06}$ \\[.2em]
 \quad Locus of Control - positive & . &         . &         . & &      0.06 &     -0.15 &      0.07 & &      0.09 &     -0.23 &      0.26 & &      0.15 &     -0.11 &      0.15 & &      0.12 &     -0.40 &     -0.06 \\*
 \quad & $\mathit{        .}$ & $\mathit{        .}$ & $\mathit{        .}$ & & $\mathit{     0.71}$ & $\mathit{     0.82}$ & $\mathit{     0.73}$ & & $\mathit{     0.74}$ & $\mathit{     0.97}$ & $\mathit{     0.79}$ & & $\mathit{     0.82}$ & $\mathit{     0.87}$ & $\mathit{     0.82}$ & & $\mathit{     0.83}$ & $\mathit{     0.91}$ & $\mathit{     0.91}$ \\[.2em]
 \quad Stress & . &         . &         . & &      2.89 &      3.08 &      2.85 & &      3.40 &      3.15 &      3.21 & &      3.27 &      3.20 &      3.25 & &      3.29 &      3.26 &      2.92 \\*
 \quad & $\mathit{        .}$ & $\mathit{        .}$ & $\mathit{        .}$ & & $\mathit{     0.87}$ & $\mathit{     0.73}$ & $\mathit{     0.82}$ & & $\mathit{     0.69}$ & $\mathit{     0.81}$ & $\mathit{     0.75}$ & & $\mathit{     0.69}$ & $\mathit{     0.77}$ & $\mathit{     0.75}$ & & $\mathit{     0.73}$ & $\mathit{     0.79}$ & $\mathit{     0.78}$ \\[.2em]
 \quad Work is Source of Stress & . &         . &         . & &         . &         . &         . & &      0.65 &      0.55 &      0.59 & &      0.59 &      0.48 &      0.57 & &      0.35 &      0.53 &      0.48 \\*
 \quad & $\mathit{        .}$ & $\mathit{        .}$ & $\mathit{        .}$ & & $\mathit{        .}$ & $\mathit{        .}$ & $\mathit{        .}$ & & $\mathit{     0.48}$ & $\mathit{     0.50}$ & $\mathit{     0.49}$ & & $\mathit{     0.49}$ & $\mathit{     0.50}$ & $\mathit{     0.50}$ & & $\mathit{     0.48}$ & $\mathit{     0.50}$ & $\mathit{     0.50}$ \\[.2em]
 \quad Satisfied with Income & . &         . &         . & &         . &         . &         . & &      3.53 &      3.10 &      3.46 & &      3.56 &      3.23 &      3.45 & &      3.30 &      3.11 &      3.29 \\*
 \quad & $\mathit{        .}$ & $\mathit{        .}$ & $\mathit{        .}$ & & $\mathit{        .}$ & $\mathit{        .}$ & $\mathit{        .}$ & & $\mathit{     0.84}$ & $\mathit{     0.91}$ & $\mathit{     0.82}$ & & $\mathit{     0.86}$ & $\mathit{     0.83}$ & $\mathit{     0.84}$ & & $\mathit{     0.80}$ & $\mathit{     0.94}$ & $\mathit{     0.99}$ \\[.2em]
 \quad Satisfied with Work & . &         . &         . & &         . &         . &         . & &      3.87 &      3.40 &      3.71 & &      3.95 &      3.62 &      3.57 & &      3.62 &      3.60 &      3.46 \\*
 \quad & $\mathit{        .}$ & $\mathit{        .}$ & $\mathit{        .}$ & & $\mathit{        .}$ & $\mathit{        .}$ & $\mathit{        .}$ & & $\mathit{     0.80}$ & $\mathit{     1.09}$ & $\mathit{     0.90}$ & & $\mathit{     0.76}$ & $\mathit{     0.83}$ & $\mathit{     0.98}$ & & $\mathit{     0.80}$ & $\mathit{     0.83}$ & $\mathit{     1.02}$ \\[.2em]
 \quad Satisfied with Health & . &         . &         . & &      4.19 &      4.15 &      4.24 & &      4.20 &      4.16 &      4.08 & &      4.14 &      3.91 &      3.96 & &      3.79 &      3.42 &      3.58 \\*
 \quad & $\mathit{        .}$ & $\mathit{        .}$ & $\mathit{        .}$ & & $\mathit{     0.91}$ & $\mathit{     0.70}$ & $\mathit{     0.88}$ & & $\mathit{     0.69}$ & $\mathit{     0.64}$ & $\mathit{     0.65}$ & & $\mathit{     0.53}$ & $\mathit{     0.72}$ & $\mathit{     0.63}$ & & $\mathit{     0.67}$ & $\mathit{     0.76}$ & $\mathit{     1.08}$ \\[.2em]
 \quad Satisfied with Family & . &         . &         . & &      4.14 &      4.10 &      4.21 & &      3.90 &      3.83 &      4.02 & &      4.01 &      3.92 &      3.88 & &      3.87 &      3.80 &      3.94 \\*
 \quad & $\mathit{        .}$ & $\mathit{        .}$ & $\mathit{        .}$ & & $\mathit{     0.88}$ & $\mathit{     0.81}$ & $\mathit{     0.84}$ & & $\mathit{     0.85}$ & $\mathit{     0.83}$ & $\mathit{     0.86}$ & & $\mathit{     0.84}$ & $\mathit{     0.75}$ & $\mathit{     0.82}$ & & $\mathit{     0.91}$ & $\mathit{     0.73}$ & $\mathit{     0.97}$ \\[.2em]
 \quad Optimistic Look in Life & . &         . &         . & &      0.67 &      0.69 &      0.56 & &      0.55 &      0.55 &      0.61 & &      0.59 &      0.30 &      0.46 & &      0.20 &      0.19 &      0.23 \\*
 \quad & $\mathit{        .}$ & $\mathit{        .}$ & $\mathit{        .}$ & & $\mathit{     0.47}$ & $\mathit{     0.46}$ & $\mathit{     0.50}$ & & $\mathit{     0.50}$ & $\mathit{     0.50}$ & $\mathit{     0.49}$ & & $\mathit{     0.49}$ & $\mathit{     0.46}$ & $\mathit{     0.50}$ & & $\mathit{     0.40}$ & $\mathit{     0.39}$ & $\mathit{     0.42}$ \\[.2em]
 \quad Return Favor & . &         . &         . & &      4.13 &      4.06 &      4.25 & &      4.20 &      4.37 &      4.30 & &      4.29 &      4.30 &      4.13 & &      4.61 &      4.41 &      4.08 \\*
 \quad & $\mathit{        .}$ & $\mathit{        .}$ & $\mathit{        .}$ & & $\mathit{     1.25}$ & $\mathit{     1.13}$ & $\mathit{     1.04}$ & & $\mathit{     0.89}$ & $\mathit{     0.65}$ & $\mathit{     0.93}$ & & $\mathit{     0.81}$ & $\mathit{     0.69}$ & $\mathit{     1.02}$ & & $\mathit{     0.50}$ & $\mathit{     0.62}$ & $\mathit{     1.26}$ \\[.2em]
 \quad Put Someone in Difficulty & . &         . &         . & &      3.04 &      3.00 &      2.87 & &      3.11 &      2.32 &      2.45 & &      2.79 &      2.48 &      2.38 & &      2.44 &      2.99 &      2.41 \\*
 \quad & $\mathit{        .}$ & $\mathit{        .}$ & $\mathit{        .}$ & & $\mathit{     1.20}$ & $\mathit{     1.14}$ & $\mathit{     1.16}$ & & $\mathit{     1.07}$ & $\mathit{     1.21}$ & $\mathit{     1.29}$ & & $\mathit{     1.15}$ & $\mathit{     1.20}$ & $\mathit{     1.29}$ & & $\mathit{     1.31}$ & $\mathit{     1.15}$ & $\mathit{     1.34}$ \\[.2em]
 \quad Help Someone Kind To Me & . &         . &         . & &      4.05 &      4.02 &      3.95 & &      4.27 &      4.20 &      4.13 & &      4.29 &      4.15 &      4.08 & &      4.39 &      4.40 &      4.02 \\*
 \quad & $\mathit{        .}$ & $\mathit{        .}$ & $\mathit{        .}$ & & $\mathit{     1.14}$ & $\mathit{     1.06}$ & $\mathit{     1.06}$ & & $\mathit{     0.67}$ & $\mathit{     0.54}$ & $\mathit{     0.86}$ & & $\mathit{     0.66}$ & $\mathit{     0.62}$ & $\mathit{     0.89}$ & & $\mathit{     0.51}$ & $\mathit{     0.57}$ & $\mathit{     1.16}$ \\[.2em]
 \quad Insult Back & . &         . &         . & &      3.06 &      3.00 &      2.88 & &      2.91 &      2.67 &      2.69 & &      2.71 &      2.78 &      2.81 & &      2.73 &      3.06 &      2.62 \\*
 \quad & $\mathit{        .}$ & $\mathit{        .}$ & $\mathit{        .}$ & & $\mathit{     1.34}$ & $\mathit{     1.31}$ & $\mathit{     1.23}$ & & $\mathit{     1.00}$ & $\mathit{     1.20}$ & $\mathit{     1.16}$ & & $\mathit{     1.08}$ & $\mathit{     1.19}$ & $\mathit{     1.07}$ & & $\mathit{     1.06}$ & $\mathit{     1.13}$ & $\mathit{     1.26}$ \\[.2em]
 ~\\[-.5em]
\textbf{Social} \\[.6em]
 \quad Num. of Friends & 3.51 &      3.68 &      4.60 & &      1.45 &      1.65 &      1.59 & &         . &         . &         . & &         . &         . &         . & &         . &         . &         . \\*
 \quad & $\mathit{     2.16}$ & $\mathit{     3.74}$ & $\mathit{     4.38}$ & & $\mathit{     0.58}$ & $\mathit{     0.69}$ & $\mathit{     0.65}$ & & $\mathit{        .}$ & $\mathit{        .}$ & $\mathit{        .}$ & & $\mathit{        .}$ & $\mathit{        .}$ & $\mathit{        .}$ & & $\mathit{        .}$ & $\mathit{        .}$ & $\mathit{        .}$ \\[.2em]
 \quad Musical Instrument at Home & 0.48 &      0.45 &      0.54 & &         . &         . &         . & &         . &         . &         . & &         . &         . &         . & &         . &         . &         . \\*
 \quad & $\mathit{     0.50}$ & $\mathit{     0.50}$ & $\mathit{     0.50}$ & & $\mathit{        .}$ & $\mathit{        .}$ & $\mathit{        .}$ & & $\mathit{        .}$ & $\mathit{        .}$ & $\mathit{        .}$ & & $\mathit{        .}$ & $\mathit{        .}$ & $\mathit{        .}$ & & $\mathit{        .}$ & $\mathit{        .}$ & $\mathit{        .}$ \\[.2em]
 \quad Tell Worry at Home & 0.66 &      0.71 &      0.71 & &         . &         . &         . & &         . &         . &         . & &         . &         . &         . & &         . &         . &         . \\*
 \quad & $\mathit{     0.48}$ & $\mathit{     0.46}$ & $\mathit{     0.46}$ & & $\mathit{        .}$ & $\mathit{        .}$ & $\mathit{        .}$ & & $\mathit{        .}$ & $\mathit{        .}$ & $\mathit{        .}$ & & $\mathit{        .}$ & $\mathit{        .}$ & $\mathit{        .}$ & & $\mathit{        .}$ & $\mathit{        .}$ & $\mathit{        .}$ \\[.2em]
 \quad Tell Worry to Teacher & 0.30 &      0.30 &      0.21 & &         . &         . &         . & &         . &         . &         . & &         . &         . &         . & &         . &         . &         . \\*
 \quad & $\mathit{     0.46}$ & $\mathit{     0.46}$ & $\mathit{     0.41}$ & & $\mathit{        .}$ & $\mathit{        .}$ & $\mathit{        .}$ & & $\mathit{        .}$ & $\mathit{        .}$ & $\mathit{        .}$ & & $\mathit{        .}$ & $\mathit{        .}$ & $\mathit{        .}$ & & $\mathit{        .}$ & $\mathit{        .}$ & $\mathit{        .}$ \\[.2em]
 \quad Tell Worry to Friends & 0.19 &      0.18 &      0.21 & &         . &         . &         . & &         . &         . &         . & &         . &         . &         . & &         . &         . &         . \\*
 \quad & $\mathit{     0.39}$ & $\mathit{     0.38}$ & $\mathit{     0.40}$ & & $\mathit{        .}$ & $\mathit{        .}$ & $\mathit{        .}$ & & $\mathit{        .}$ & $\mathit{        .}$ & $\mathit{        .}$ & & $\mathit{        .}$ & $\mathit{        .}$ & $\mathit{        .}$ & & $\mathit{        .}$ & $\mathit{        .}$ & $\mathit{        .}$ \\[.2em]
 \quad Keep Worry to Myself & 0.14 &      0.13 &      0.11 & &         . &         . &         . & &         . &         . &         . & &         . &         . &         . & &         . &         . &         . \\*
 \quad & $\mathit{     0.35}$ & $\mathit{     0.34}$ & $\mathit{     0.32}$ & & $\mathit{        .}$ & $\mathit{        .}$ & $\mathit{        .}$ & & $\mathit{        .}$ & $\mathit{        .}$ & $\mathit{        .}$ & & $\mathit{        .}$ & $\mathit{        .}$ & $\mathit{        .}$ & & $\mathit{        .}$ & $\mathit{        .}$ & $\mathit{        .}$ \\[.2em]
 \quad Num. of Friends & . &         . &         . & &     10.35 &      9.80 &     11.26 & &      8.24 &     11.43 &     10.49 & &      7.40 &      8.83 &      9.86 & &      8.53 &      6.52 &      9.12 \\*
 \quad & $\mathit{        .}$ & $\mathit{        .}$ & $\mathit{        .}$ & & $\mathit{     9.89}$ & $\mathit{    10.01}$ & $\mathit{    11.18}$ & & $\mathit{     6.65}$ & $\mathit{     9.47}$ & $\mathit{     7.42}$ & & $\mathit{     5.54}$ & $\mathit{     6.95}$ & $\mathit{     7.51}$ & & $\mathit{     3.97}$ & $\mathit{     5.05}$ & $\mathit{     7.89}$ \\[.2em]
 \quad Doesn't Talk About Activities & . &         . &         . & &      1.52 &      1.78 &      1.50 & &         . &         . &         . & &         . &         . &         . & &         . &         . &         . \\*
 \quad & $\mathit{        .}$ & $\mathit{        .}$ & $\mathit{        .}$ & & $\mathit{     0.65}$ & $\mathit{     0.75}$ & $\mathit{     0.62}$ & & $\mathit{        .}$ & $\mathit{        .}$ & $\mathit{        .}$ & & $\mathit{        .}$ & $\mathit{        .}$ & $\mathit{        .}$ & & $\mathit{        .}$ & $\mathit{        .}$ & $\mathit{        .}$ \\[.2em]
 \quad Doesn't Talk About School & . &         . &         . & &      1.45 &      1.62 &      1.41 & &         . &         . &         . & &         . &         . &         . & &         . &         . &         . \\*
 \quad & $\mathit{        .}$ & $\mathit{        .}$ & $\mathit{        .}$ & & $\mathit{     0.61}$ & $\mathit{     0.66}$ & $\mathit{     0.60}$ & & $\mathit{        .}$ & $\mathit{        .}$ & $\mathit{        .}$ & & $\mathit{        .}$ & $\mathit{        .}$ & $\mathit{        .}$ & & $\mathit{        .}$ & $\mathit{        .}$ & $\mathit{        .}$ \\[.2em]
 \quad Favorable to Migrants & . &         . &         . & &      2.91 &      3.09 &      2.78 & &      2.99 &      3.01 &      2.70 & &      2.92 &      3.02 &      2.68 & &      2.78 &      2.91 &      2.53 \\*
 \quad & $\mathit{        .}$ & $\mathit{        .}$ & $\mathit{        .}$ & & $\mathit{     0.90}$ & $\mathit{     0.97}$ & $\mathit{     0.78}$ & & $\mathit{     0.50}$ & $\mathit{     0.84}$ & $\mathit{     0.78}$ & & $\mathit{     0.56}$ & $\mathit{     0.73}$ & $\mathit{     0.75}$ & & $\mathit{     0.65}$ & $\mathit{     0.70}$ & $\mathit{     0.81}$ \\[.2em]
 \quad Has Migrant Friends & . &         . &         . & &      0.86 &      0.87 &      0.78 & &      0.70 &      0.76 &      0.69 & &      0.71 &      0.66 &      0.58 & &      0.72 &      0.58 &      0.60 \\*
 \quad & $\mathit{        .}$ & $\mathit{        .}$ & $\mathit{        .}$ & & $\mathit{     0.35}$ & $\mathit{     0.34}$ & $\mathit{     0.42}$ & & $\mathit{     0.46}$ & $\mathit{     0.43}$ & $\mathit{     0.46}$ & & $\mathit{     0.45}$ & $\mathit{     0.47}$ & $\mathit{     0.49}$ & & $\mathit{     0.45}$ & $\mathit{     0.50}$ & $\mathit{     0.49}$ \\[.2em]
 \quad Volunteers & . &         . &         . & &      0.41 &      0.24 &      0.25 & &      0.07 &      0.24 &      0.16 & &      0.12 &      0.21 &      0.12 & &      0.35 &      0.23 &      0.16 \\*
 \quad & $\mathit{        .}$ & $\mathit{        .}$ & $\mathit{        .}$ & & $\mathit{     0.49}$ & $\mathit{     0.43}$ & $\mathit{     0.43}$ & & $\mathit{     0.25}$ & $\mathit{     0.43}$ & $\mathit{     0.37}$ & & $\mathit{     0.33}$ & $\mathit{     0.41}$ & $\mathit{     0.32}$ & & $\mathit{     0.48}$ & $\mathit{     0.42}$ & $\mathit{     0.37}$ \\[.2em]
 \quad Child Eats Meal with Fam & . &         . &         . & &         . &         . &         . & &      1.45 &      1.63 &      1.61 & &      1.51 &      1.71 &      1.59 & &      1.33 &      1.81 &      1.54 \\*
 \quad & $\mathit{        .}$ & $\mathit{        .}$ & $\mathit{        .}$ & & $\mathit{        .}$ & $\mathit{        .}$ & $\mathit{        .}$ & & $\mathit{     0.51}$ & $\mathit{     0.72}$ & $\mathit{     1.02}$ & & $\mathit{     0.73}$ & $\mathit{     0.58}$ & $\mathit{     0.70}$ & & $\mathit{     0.63}$ & $\mathit{     0.67}$ & $\mathit{     0.83}$ \\[.2em]
 \quad Ever Voted for Municipal & . &         . &         . & &         . &         . &         . & &      0.59 &      0.28 &      0.40 & &      0.57 &      0.26 &      0.35 & &      0.28 &      0.23 &      0.40 \\*
 \quad & $\mathit{        .}$ & $\mathit{        .}$ & $\mathit{        .}$ & & $\mathit{        .}$ & $\mathit{        .}$ & $\mathit{        .}$ & & $\mathit{     0.49}$ & $\mathit{     0.45}$ & $\mathit{     0.49}$ & & $\mathit{     0.50}$ & $\mathit{     0.44}$ & $\mathit{     0.48}$ & & $\mathit{     0.45}$ & $\mathit{     0.42}$ & $\mathit{     0.49}$ \\[.2em]
 \quad Ever Voted for Regional & . &         . &         . & &         . &         . &         . & &      0.60 &      0.22 &      0.42 & &      0.59 &      0.20 &      0.29 & &      0.29 &      0.21 &      0.43 \\*
 \quad & $\mathit{        .}$ & $\mathit{        .}$ & $\mathit{        .}$ & & $\mathit{        .}$ & $\mathit{        .}$ & $\mathit{        .}$ & & $\mathit{     0.49}$ & $\mathit{     0.42}$ & $\mathit{     0.50}$ & & $\mathit{     0.49}$ & $\mathit{     0.40}$ & $\mathit{     0.46}$ & & $\mathit{     0.45}$ & $\mathit{     0.41}$ & $\mathit{     0.50}$ \\[.2em]
 \quad Ever Voted for National & . &         . &         . & &         . &         . &         . & &      0.80 &      0.86 &      0.87 & &      0.83 &      0.84 &      0.85 & &      0.81 &      0.74 &      0.92 \\*
 \quad & $\mathit{        .}$ & $\mathit{        .}$ & $\mathit{        .}$ & & $\mathit{        .}$ & $\mathit{        .}$ & $\mathit{        .}$ & & $\mathit{     0.40}$ & $\mathit{     0.35}$ & $\mathit{     0.34}$ & & $\mathit{     0.38}$ & $\mathit{     0.36}$ & $\mathit{     0.36}$ & & $\mathit{     0.40}$ & $\mathit{     0.44}$ & $\mathit{     0.27}$ \\[.2em]
 ~\\[-.5em]
\textbf{Health} \\[.6em]
 \quad Obese & 0.31 &      0.17 &      0.30 & &      0.15 &      0.11 &      0.28 & &      0.23 &      0.23 &      0.20 & &      0.30 &      0.19 &      0.25 & &      0.17 &      0.25 &      0.38 \\*
 \quad & $\mathit{     0.46}$ & $\mathit{     0.37}$ & $\mathit{     0.46}$ & & $\mathit{     0.36}$ & $\mathit{     0.31}$ & $\mathit{     0.45}$ & & $\mathit{     0.42}$ & $\mathit{     0.42}$ & $\mathit{     0.40}$ & & $\mathit{     0.46}$ & $\mathit{     0.39}$ & $\mathit{     0.43}$ & & $\mathit{     0.38}$ & $\mathit{     0.44}$ & $\mathit{     0.49}$ \\[.2em]
 \quad Overweight & 0.13 &      0.16 &      0.08 & &      0.04 &      0.06 &      0.02 & &      0.19 &      0.23 &      0.19 & &      0.25 &      0.29 &      0.21 & &      0.34 &      0.24 &      0.22 \\*
 \quad & $\mathit{     0.33}$ & $\mathit{     0.37}$ & $\mathit{     0.27}$ & & $\mathit{     0.19}$ & $\mathit{     0.23}$ & $\mathit{     0.14}$ & & $\mathit{     0.40}$ & $\mathit{     0.42}$ & $\mathit{     0.39}$ & & $\mathit{     0.43}$ & $\mathit{     0.46}$ & $\mathit{     0.41}$ & & $\mathit{     0.48}$ & $\mathit{     0.43}$ & $\mathit{     0.42}$ \\[.2em]
 \quad Health is Good & 0.68 &      0.63 &      0.75 & &      0.67 &      0.54 &      0.70 & &         . &         . &         . & &         . &         . &         . & &         . &         . &         . \\*
 \quad & $\mathit{     0.47}$ & $\mathit{     0.48}$ & $\mathit{     0.44}$ & & $\mathit{     0.47}$ & $\mathit{     0.50}$ & $\mathit{     0.46}$ & & $\mathit{        .}$ & $\mathit{        .}$ & $\mathit{        .}$ & & $\mathit{        .}$ & $\mathit{        .}$ & $\mathit{        .}$ & & $\mathit{        .}$ & $\mathit{        .}$ & $\mathit{        .}$ \\[.2em]
 \quad Number of Sick Days & 1.57 &      1.40 &      1.52 & &      1.51 &      1.51 &      1.53 & &         . &         . &         . & &         . &         . &         . & &         . &         . &         . \\*
 \quad & $\mathit{     0.84}$ & $\mathit{     0.71}$ & $\mathit{     0.83}$ & & $\mathit{     0.81}$ & $\mathit{     0.73}$ & $\mathit{     0.77}$ & & $\mathit{        .}$ & $\mathit{        .}$ & $\mathit{        .}$ & & $\mathit{        .}$ & $\mathit{        .}$ & $\mathit{        .}$ & & $\mathit{        .}$ & $\mathit{        .}$ & $\mathit{        .}$ \\[.2em]
 \quad Ever Suspended from School & . &         . &         . & &      0.07 &      0.04 &      0.02 & &      0.08 &      0.06 &      0.05 & &      0.06 &      0.05 &      0.05 & &      0.06 &      0.05 &      0.09 \\*
 \quad & $\mathit{        .}$ & $\mathit{        .}$ & $\mathit{        .}$ & & $\mathit{     0.25}$ & $\mathit{     0.19}$ & $\mathit{     0.14}$ & & $\mathit{     0.28}$ & $\mathit{     0.24}$ & $\mathit{     0.22}$ & & $\mathit{     0.24}$ & $\mathit{     0.21}$ & $\mathit{     0.22}$ & & $\mathit{     0.24}$ & $\mathit{     0.22}$ & $\mathit{     0.29}$ \\[.2em]
 \quad Num. of Cigarettes Per Day & . &         . &         . & &      7.52 &      5.26 &      7.00 & &     16.02 &     12.06 &     10.23 & &     15.90 &     13.35 &     11.19 & &     15.98 &     18.67 &      7.96 \\*
 \quad & $\mathit{        .}$ & $\mathit{        .}$ & $\mathit{        .}$ & & $\mathit{     4.52}$ & $\mathit{     2.93}$ & $\mathit{     4.11}$ & & $\mathit{     4.24}$ & $\mathit{     6.74}$ & $\mathit{     5.28}$ & & $\mathit{     5.94}$ & $\mathit{     6.39}$ & $\mathit{     5.31}$ & & $\mathit{     7.43}$ & $\mathit{    10.17}$ & $\mathit{     5.04}$ \\[.2em]
 \quad Tried Marijuana & . &         . &         . & &      0.23 &      0.18 &      0.17 & &      0.20 &      0.17 &      0.20 & &      0.11 &      0.07 &      0.07 & &      0.04 &      0.03 &      0.06 \\*
 \quad & $\mathit{        .}$ & $\mathit{        .}$ & $\mathit{        .}$ & & $\mathit{     0.42}$ & $\mathit{     0.38}$ & $\mathit{     0.38}$ & & $\mathit{     0.40}$ & $\mathit{     0.37}$ & $\mathit{     0.40}$ & & $\mathit{     0.31}$ & $\mathit{     0.26}$ & $\mathit{     0.25}$ & & $\mathit{     0.18}$ & $\mathit{     0.17}$ & $\mathit{     0.24}$ \\[.2em]
 \quad BMI & . &         . &         . & &     21.36 &     21.39 &     21.40 & &     23.35 &     23.29 &     23.39 & &     24.16 &     23.92 &     23.71 & &     24.46 &     23.76 &     25.11 \\*
 \quad & $\mathit{        .}$ & $\mathit{        .}$ & $\mathit{        .}$ & & $\mathit{     2.83}$ & $\mathit{     3.10}$ & $\mathit{     2.56}$ & & $\mathit{     2.26}$ & $\mathit{     2.94}$ & $\mathit{     2.99}$ & & $\mathit{     2.94}$ & $\mathit{     2.78}$ & $\mathit{     2.85}$ & & $\mathit{     2.77}$ & $\mathit{     2.95}$ & $\mathit{     4.19}$ \\[.2em]
 \quad Good Health & . &         . &         . & &      3.98 &      3.69 &      4.05 & &      4.23 &      3.84 &      3.80 & &      3.90 &      3.54 &      3.55 & &      3.28 &      3.10 &      3.16 \\*
 \quad & $\mathit{        .}$ & $\mathit{        .}$ & $\mathit{        .}$ & & $\mathit{     0.86}$ & $\mathit{     0.74}$ & $\mathit{     0.84}$ & & $\mathit{     0.54}$ & $\mathit{     0.65}$ & $\mathit{     0.64}$ & & $\mathit{     0.57}$ & $\mathit{     0.68}$ & $\mathit{     0.73}$ & & $\mathit{     0.65}$ & $\mathit{     0.60}$ & $\mathit{     0.89}$ \\[.2em]
 \quad No Problematic Health Condition & . &         . &         . & &         . &         . &         . & &      0.58 &      0.63 &      0.61 & &      0.59 &      0.48 &      0.56 & &      0.30 &      0.17 &      0.40 \\*
 \quad & $\mathit{        .}$ & $\mathit{        .}$ & $\mathit{        .}$ & & $\mathit{        .}$ & $\mathit{        .}$ & $\mathit{        .}$ & & $\mathit{     0.49}$ & $\mathit{     0.48}$ & $\mathit{     0.49}$ & & $\mathit{     0.49}$ & $\mathit{     0.50}$ & $\mathit{     0.50}$ & & $\mathit{     0.46}$ & $\mathit{     0.38}$ & $\mathit{     0.49}$ \\[.2em]
 \quad Num. of Days Sick Past Month & . &         . &         . & &         . &         . &         . & &      1.31 &      1.13 &      1.16 & &      1.10 &      1.12 &      1.19 & &      1.31 &      1.14 &      1.33 \\*
 \quad & $\mathit{        .}$ & $\mathit{        .}$ & $\mathit{        .}$ & & $\mathit{        .}$ & $\mathit{        .}$ & $\mathit{        .}$ & & $\mathit{     0.56}$ & $\mathit{     0.51}$ & $\mathit{     0.63}$ & & $\mathit{     0.40}$ & $\mathit{     0.54}$ & $\mathit{     0.62}$ & & $\mathit{     0.66}$ & $\mathit{     0.38}$ & $\mathit{     0.91}$ \\[.2em]
 \quad Age At First Drink & . &         . &         . & &     12.63 &     14.09 &     10.25 & &     12.14 &     13.45 &     13.49 & &     11.07 &     14.26 &     13.09 & &     14.29 &     13.92 &     14.36 \\*
 \quad & $\mathit{        .}$ & $\mathit{        .}$ & $\mathit{        .}$ & & $\mathit{     5.64}$ & $\mathit{     4.12}$ & $\mathit{     7.29}$ & & $\mathit{     7.98}$ & $\mathit{     6.08}$ & $\mathit{     6.61}$ & & $\mathit{     8.58}$ & $\mathit{     5.76}$ & $\mathit{     7.43}$ & & $\mathit{     6.31}$ & $\mathit{     7.13}$ & $\mathit{     6.63}$ \\[.2em]
 ~\\[-.5em]
\textbf{Employment} \\[.6em]
 \quad Employed & . &         . &         . & &      0.02 &      0.01 &      0.01 & &      0.95 &      0.89 &      0.89 & &      0.95 &      0.94 &      0.90 & &      0.90 &      0.86 &      0.74 \\*
 \quad & $\mathit{        .}$ & $\mathit{        .}$ & $\mathit{        .}$ & & $\mathit{     0.14}$ & $\mathit{     0.09}$ & $\mathit{     0.10}$ & & $\mathit{     0.21}$ & $\mathit{     0.32}$ & $\mathit{     0.31}$ & & $\mathit{     0.21}$ & $\mathit{     0.24}$ & $\mathit{     0.29}$ & & $\mathit{     0.30}$ & $\mathit{     0.34}$ & $\mathit{     0.44}$ \\[.2em]
 \quad Self-Employed & . &         . &         . & &         . &         . &         . & &      0.13 &      0.06 &      0.09 & &      0.15 &      0.12 &      0.12 & &      0.11 &      0.14 &      0.11 \\*
 \quad & $\mathit{        .}$ & $\mathit{        .}$ & $\mathit{        .}$ & & $\mathit{        .}$ & $\mathit{        .}$ & $\mathit{        .}$ & & $\mathit{     0.33}$ & $\mathit{     0.25}$ & $\mathit{     0.28}$ & & $\mathit{     0.36}$ & $\mathit{     0.32}$ & $\mathit{     0.32}$ & & $\mathit{     0.31}$ & $\mathit{     0.35}$ & $\mathit{     0.31}$ \\[.2em]
 \quad Hours Worked Per Week & . &         . &         . & &         . &         . &         . & &     40.45 &     38.10 &     39.55 & &     41.44 &     39.80 &     36.88 & &     40.42 &     41.31 &     36.99 \\*
 \quad & $\mathit{        .}$ & $\mathit{        .}$ & $\mathit{        .}$ & & $\mathit{        .}$ & $\mathit{        .}$ & $\mathit{        .}$ & & $\mathit{     7.51}$ & $\mathit{     8.40}$ & $\mathit{     8.79}$ & & $\mathit{    10.02}$ & $\mathit{     6.62}$ & $\mathit{    11.52}$ & & $\mathit{     5.66}$ & $\mathit{     6.15}$ & $\mathit{    10.22}$ \\[.2em]
 \quad Monthly Wage & . &         . &         . & &         . &         . &         . & &   1105.89 &    986.03 &   1430.03 & &   2171.12 &   1151.47 &   1506.15 & &   1479.85 &   1461.54 &   1904.24 \\*
 \quad & $\mathit{        .}$ & $\mathit{        .}$ & $\mathit{        .}$ & & $\mathit{        .}$ & $\mathit{        .}$ & $\mathit{        .}$ & & $\mathit{   527.14}$ & $\mathit{   466.55}$ & $\mathit{   656.81}$ & & $\mathit{  3072.33}$ & $\mathit{   566.95}$ & $\mathit{  1448.48}$ & & $\mathit{   461.44}$ & $\mathit{   588.95}$ & $\mathit{  3754.54}$ \\[.2em]
 \quad Income: 5,000 Euros of Less & . &         . &         . & &         . &         . &         . & &      0.11 &      0.02 &      0.02 & &      0.01 &      0.01 &      0.02 & &      0.01 &      0.01 &      0.02 \\*
 \quad & $\mathit{        .}$ & $\mathit{        .}$ & $\mathit{        .}$ & & $\mathit{        .}$ & $\mathit{        .}$ & $\mathit{        .}$ & & $\mathit{     0.31}$ & $\mathit{     0.13}$ & $\mathit{     0.13}$ & & $\mathit{     0.08}$ & $\mathit{     0.09}$ & $\mathit{     0.14}$ & & $\mathit{     0.10}$ & $\mathit{     0.10}$ & $\mathit{     0.14}$ \\[.2em]
 \quad Income: 5,001-10,000 Euros & . &         . &         . & &         . &         . &         . & &      0.01 &      0.01 &      0.01 & &      0.00 &      0.01 &      0.02 & &      0.00 &      0.02 &      0.05 \\*
 \quad & $\mathit{        .}$ & $\mathit{        .}$ & $\mathit{        .}$ & & $\mathit{        .}$ & $\mathit{        .}$ & $\mathit{        .}$ & & $\mathit{     0.10}$ & $\mathit{     0.11}$ & $\mathit{     0.11}$ & & $\mathit{     0.06}$ & $\mathit{     0.09}$ & $\mathit{     0.14}$ & & $\mathit{     0.00}$ & $\mathit{     0.14}$ & $\mathit{     0.21}$ \\[.2em]
 \quad Income: 10,001-25,000 Euros & . &         . &         . & &         . &         . &         . & &      0.32 &      0.45 &      0.39 & &      0.31 &      0.34 &      0.30 & &      0.26 &      0.40 &      0.28 \\*
 \quad & $\mathit{        .}$ & $\mathit{        .}$ & $\mathit{        .}$ & & $\mathit{        .}$ & $\mathit{        .}$ & $\mathit{        .}$ & & $\mathit{     0.47}$ & $\mathit{     0.50}$ & $\mathit{     0.49}$ & & $\mathit{     0.46}$ & $\mathit{     0.48}$ & $\mathit{     0.46}$ & & $\mathit{     0.44}$ & $\mathit{     0.49}$ & $\mathit{     0.45}$ \\[.2em]
 \quad Income: 25,001-50,000 Euros & . &         . &         . & &         . &         . &         . & &      0.53 &      0.43 &      0.49 & &      0.57 &      0.56 &      0.54 & &      0.65 &      0.45 &      0.49 \\*
 \quad & $\mathit{        .}$ & $\mathit{        .}$ & $\mathit{        .}$ & & $\mathit{        .}$ & $\mathit{        .}$ & $\mathit{        .}$ & & $\mathit{     0.50}$ & $\mathit{     0.50}$ & $\mathit{     0.50}$ & & $\mathit{     0.50}$ & $\mathit{     0.50}$ & $\mathit{     0.50}$ & & $\mathit{     0.48}$ & $\mathit{     0.50}$ & $\mathit{     0.50}$ \\[.2em]
 \quad Income: 50,001-100,000 Euros & . &         . &         . & &         . &         . &         . & &      0.03 &      0.09 &      0.08 & &      0.07 &      0.06 &      0.11 & &      0.09 &      0.12 &      0.14 \\*
 \quad & $\mathit{        .}$ & $\mathit{        .}$ & $\mathit{        .}$ & & $\mathit{        .}$ & $\mathit{        .}$ & $\mathit{        .}$ & & $\mathit{     0.18}$ & $\mathit{     0.28}$ & $\mathit{     0.28}$ & & $\mathit{     0.25}$ & $\mathit{     0.24}$ & $\mathit{     0.31}$ & & $\mathit{     0.28}$ & $\mathit{     0.32}$ & $\mathit{     0.35}$ \\[.2em]
 \quad Income: 100,001-250,000 Euros & . &         . &         . & &         . &         . &         . & &      0.00 &      0.01 &      0.00 & &      0.05 &      0.02 &      0.01 & &      0.00 &      0.01 &      0.01 \\*
 \quad & $\mathit{        .}$ & $\mathit{        .}$ & $\mathit{        .}$ & & $\mathit{        .}$ & $\mathit{        .}$ & $\mathit{        .}$ & & $\mathit{     0.00}$ & $\mathit{     0.09}$ & $\mathit{     0.06}$ & & $\mathit{     0.21}$ & $\mathit{     0.12}$ & $\mathit{     0.11}$ & & $\mathit{     0.00}$ & $\mathit{     0.10}$ & $\mathit{     0.08}$ \\[.2em]
 \quad Income: More than 250,000 Euros & . &         . &         . & &         . &         . &         . & &         . &         . &         . & &         . &         . &         . & &      0.00 &      0.00 &      0.01 \\*
 \quad & $\mathit{        .}$ & $\mathit{        .}$ & $\mathit{        .}$ & & $\mathit{        .}$ & $\mathit{        .}$ & $\mathit{        .}$ & & $\mathit{        .}$ & $\mathit{        .}$ & $\mathit{        .}$ & & $\mathit{        .}$ & $\mathit{        .}$ & $\mathit{        .}$ & & $\mathit{     0.00}$ & $\mathit{     0.00}$ & $\mathit{     0.08}$ \\[.2em]
 ~\\[-.5em]
\textbf{Education} \\[.6em]
 \quad High School Grade & . &         . &         . & &     76.44 &     80.71 &     82.63 & &     82.86 &     74.16 &     77.89 & &     83.36 &     74.98 &     78.66 & &     79.89 &     72.47 &     75.91 \\*
 \quad & $\mathit{        .}$ & $\mathit{        .}$ & $\mathit{        .}$ & & $\mathit{    12.13}$ & $\mathit{    12.05}$ & $\mathit{    10.64}$ & & $\mathit{     9.20}$ & $\mathit{    18.32}$ & $\mathit{    14.42}$ & & $\mathit{     8.22}$ & $\mathit{    14.91}$ & $\mathit{    11.25}$ & & $\mathit{     8.80}$ & $\mathit{    14.79}$ & $\mathit{    11.77}$ \\[.2em]
 \quad University Grade & . &         . &         . & &         . &         . &         . & &    100.67 &     99.68 &     99.81 & &     97.17 &    100.73 &     98.48 & &     97.19 &    100.17 &    104.24 \\*
 \quad & $\mathit{        .}$ & $\mathit{        .}$ & $\mathit{        .}$ & & $\mathit{        .}$ & $\mathit{        .}$ & $\mathit{        .}$ & & $\mathit{     6.38}$ & $\mathit{     7.35}$ & $\mathit{     8.38}$ & & $\mathit{     6.49}$ & $\mathit{     7.52}$ & $\mathit{     7.74}$ & & $\mathit{     6.77}$ & $\mathit{     7.12}$ & $\mathit{     6.66}$ \\[.2em]
 \quad Graduate from High School & . &         . &         . & &         . &         . &         . & &      0.88 &      0.89 &      0.88 & &      0.80 &      0.84 &      0.82 & &      0.75 &      0.59 &      0.57 \\*
 \quad & $\mathit{        .}$ & $\mathit{        .}$ & $\mathit{        .}$ & & $\mathit{        .}$ & $\mathit{        .}$ & $\mathit{        .}$ & & $\mathit{     0.33}$ & $\mathit{     0.31}$ & $\mathit{     0.32}$ & & $\mathit{     0.40}$ & $\mathit{     0.37}$ & $\mathit{     0.38}$ & & $\mathit{     0.44}$ & $\mathit{     0.49}$ & $\mathit{     0.50}$ \\[.2em]
 \quad Max Edu: University & . &         . &         . & &         . &         . &         . & &      0.19 &      0.38 &      0.45 & &      0.15 &      0.28 &      0.33 & &      0.11 &      0.12 &      0.23 \\*
 \quad & $\mathit{        .}$ & $\mathit{        .}$ & $\mathit{        .}$ & & $\mathit{        .}$ & $\mathit{        .}$ & $\mathit{        .}$ & & $\mathit{     0.39}$ & $\mathit{     0.49}$ & $\mathit{     0.50}$ & & $\mathit{     0.36}$ & $\mathit{     0.45}$ & $\mathit{     0.47}$ & & $\mathit{     0.31}$ & $\mathit{     0.32}$ & $\mathit{     0.42}$ \\[.2em]
 \quad Max Edu: Graduate School & . &         . &         . & &         . &         . &         . & &      0.00 &      0.03 &      0.09 & &      0.00 &      0.04 &      0.02 & &      0.00 &      0.00 &      0.05 \\*
 \quad & $\mathit{        .}$ & $\mathit{        .}$ & $\mathit{        .}$ & & $\mathit{        .}$ & $\mathit{        .}$ & $\mathit{        .}$ & & $\mathit{     0.06}$ & $\mathit{     0.16}$ & $\mathit{     0.29}$ & & $\mathit{     0.00}$ & $\mathit{     0.19}$ & $\mathit{     0.15}$ & & $\mathit{     0.00}$ & $\mathit{     0.00}$ & $\mathit{     0.21}$ \\[.2em]
 ~\\[-.5em]
\textbf{LivingStatus} \\[.6em]
 \quad Married or Cohabitating & . &         . &         . & &         . &         . &         . & &      0.39 &      0.39 &      0.41 & &      0.74 &      0.61 &      0.63 & &      0.64 &      0.74 &      0.71 \\*
 \quad & $\mathit{        .}$ & $\mathit{        .}$ & $\mathit{        .}$ & & $\mathit{        .}$ & $\mathit{        .}$ & $\mathit{        .}$ & & $\mathit{     0.49}$ & $\mathit{     0.49}$ & $\mathit{     0.49}$ & & $\mathit{     0.44}$ & $\mathit{     0.49}$ & $\mathit{     0.48}$ & & $\mathit{     0.48}$ & $\mathit{     0.44}$ & $\mathit{     0.45}$ \\[.2em]
 \quad Divorced & . &         . &         . & &         . &         . &         . & &      0.03 &      0.01 &      0.02 & &      0.12 &      0.14 &      0.14 & &      0.26 &      0.18 &      0.14 \\*
 \quad & $\mathit{        .}$ & $\mathit{        .}$ & $\mathit{        .}$ & & $\mathit{        .}$ & $\mathit{        .}$ & $\mathit{        .}$ & & $\mathit{     0.16}$ & $\mathit{     0.09}$ & $\mathit{     0.13}$ & & $\mathit{     0.32}$ & $\mathit{     0.35}$ & $\mathit{     0.35}$ & & $\mathit{     0.44}$ & $\mathit{     0.38}$ & $\mathit{     0.35}$ \\[.2em]
 \quad Num. of Children in House & . &         . &         . & &         . &         . &         . & &      0.11 &      0.16 &      0.26 & &      0.60 &      0.63 &      0.64 & &      0.41 &      0.48 &      0.98 \\*
 \quad & $\mathit{        .}$ & $\mathit{        .}$ & $\mathit{        .}$ & & $\mathit{        .}$ & $\mathit{        .}$ & $\mathit{        .}$ & & $\mathit{     0.35}$ & $\mathit{     0.46}$ & $\mathit{     0.59}$ & & $\mathit{     0.64}$ & $\mathit{     0.78}$ & $\mathit{     0.88}$ & & $\mathit{     0.69}$ & $\mathit{     0.68}$ & $\mathit{     0.95}$ \\[.2em]
 \quad Own House & 0.79 &      0.85 &      0.85 & &      0.91 &      0.86 &      0.90 & &      0.58 &      0.61 &      0.73 & &      0.71 &      0.79 &      0.85 & &      0.85 &      0.86 &      0.82 \\*
 \quad & $\mathit{     0.41}$ & $\mathit{     0.36}$ & $\mathit{     0.36}$ & & $\mathit{     0.29}$ & $\mathit{     0.35}$ & $\mathit{     0.30}$ & & $\mathit{     0.49}$ & $\mathit{     0.49}$ & $\mathit{     0.44}$ & & $\mathit{     0.46}$ & $\mathit{     0.41}$ & $\mathit{     0.36}$ & & $\mathit{     0.36}$ & $\mathit{     0.34}$ & $\mathit{     0.39}$ \\[.2em]
 \quad Live With Parents & . &         . &         . & &         . &         . &         . & &      0.14 &      0.22 &      0.34 & &      0.04 &      0.08 &      0.12 & &      0.01 &      0.01 &      0.06 \\*
 \quad & $\mathit{        .}$ & $\mathit{        .}$ & $\mathit{        .}$ & & $\mathit{        .}$ & $\mathit{        .}$ & $\mathit{        .}$ & & $\mathit{     0.35}$ & $\mathit{     0.41}$ & $\mathit{     0.47}$ & & $\mathit{     0.18}$ & $\mathit{     0.28}$ & $\mathit{     0.32}$ & & $\mathit{     0.12}$ & $\mathit{     0.10}$ & $\mathit{     0.24}$ \\[.2em]
 ~\\[-.5em]
\hline
\end{longtable}
}
\end{center}

\end{landscape}

\begin{landscape}
% ==============================================================================%
\subsection{Municipal vs. No Preschool}\label{appendix:no-preschool}
% ==============================================================================% 
\subsubsection{Children}
\begin{table}[H] \caption{OLS and Diff-in-Diff for Cognitive and Noncognitive, Preschools, Children Cohort} \label{ols-E-reg}
\scalebox{0.80}{{
\def\sym#1{\ifmmode^{#1}\else\(^{#1}\)\fi}
\begin{tabular}{l*{12}{c}}
\toprule
            &\multicolumn{1}{c}{(1)}&\multicolumn{1}{c}{(2)}&\multicolumn{1}{c}{(3)}&\multicolumn{1}{c}{(4)}&\multicolumn{1}{c}{(5)}&\multicolumn{1}{c}{(6)}&\multicolumn{1}{c}{(7)}&\multicolumn{1}{c}{(8)}&\multicolumn{1}{c}{(9)}&\multicolumn{1}{c}{(10)}&\multicolumn{1}{c}{(11)}&\multicolumn{1}{c}{(12)}\\
            &\multicolumn{1}{c}{NoneIt}&\multicolumn{1}{c}{BICIt}&\multicolumn{1}{c}{FullIt}&\multicolumn{1}{c}{DidPmIt}&\multicolumn{1}{c}{DidPvIt}&\multicolumn{1}{c}{IPWIt}&\multicolumn{1}{c}{NoneMg}&\multicolumn{1}{c}{BICMg}&\multicolumn{1}{c}{FullMg}&\multicolumn{1}{c}{DidPmMg}&\multicolumn{1}{c}{DidPvMg}&\multicolumn{1}{c}{IPWMg}\\
\midrule
IQ Factor   &      -0.691\sym{***}&      -0.768\sym{***}&      -1.889\sym{***}&      -0.931\sym{***}&      -1.371         &      -0.294\sym{**} &      -0.133         &      -0.457         &     -0.0368         &      -0.215         &      -0.978\sym{*}  &      -0.285         \\
            &     (0.151)         &     (0.112)         &     (0.308)         &     (0.221)         &     (0.919)         &    (0.0911)         &     (0.434)         &     (0.425)         &     (0.499)         &     (0.542)         &     (0.475)         &     (0.218)         \\
\addlinespace
IQ Score    &      -0.133         &      -0.150\sym{***}&      -0.423\sym{***}&      -0.215\sym{**} &      -0.297         &     -0.0823\sym{***}&     -0.0833         &      -0.160         &     -0.0930         &      -0.109         &      -0.286\sym{**} &     -0.0705         \\
            &    (0.0809)         &    (0.0314)         &    (0.0556)         &    (0.0758)         &     (0.178)         &    (0.0218)         &    (0.0906)         &    (0.0820)         &     (0.117)         &     (0.131)         &    (0.0923)         &    (0.0519)         \\
\addlinespace
SDQ Composite - Child&       1.825         &       2.157         &       0.314         &       1.247         &      -2.015         &       0.195         &      -1.701         &      -2.381         &      -1.972         &      -3.127         &      -3.581         &       1.790\sym{*}  \\
            &     (3.575)         &     (2.409)         &     (2.499)         &     (2.946)         &     (3.321)         &     (0.412)         &     (1.747)         &     (2.036)         &     (2.212)         &     (3.106)         &     (2.739)         &     (0.741)         \\
\addlinespace
SDQ Pro-social - Child&      -1.301\sym{***}&      -1.306\sym{***}&      -0.660         &      -2.007\sym{*}  &      -1.021         &     0.00412         &      -0.627         &      -0.678         &      -0.121         &      -0.673         &       0.218         &       0.514         \\
            &     (0.383)         &     (0.394)         &     (0.900)         &     (0.936)         &     (0.530)         &     (0.174)         &     (0.988)         &     (1.277)         &     (1.636)         &     (1.616)         &     (1.748)         &     (0.312)         \\
\addlinespace
SDQ Peer problems - Child&       0.422         &       0.482         &      0.0779         &      -0.475         &     -0.0144         &       0.196         &      -0.348         &      -0.480         &       0.131         &      -0.392         &      -0.605         &       0.403         \\
            &     (1.071)         &     (0.931)         &     (0.822)         &     (1.088)         &     (1.758)         &     (0.118)         &     (0.695)         &     (0.816)         &     (0.765)         &     (1.067)         &     (0.910)         &     (0.263)         \\
\addlinespace
SDQ Hyper - Child&       1.904\sym{**} &       2.092\sym{***}&       0.118         &       2.617\sym{***}&      -0.209         &      -0.107         &     -0.0539         &      -0.257         &      -0.826         &      -1.912         &      -1.358         &      0.0982         \\
            &     (0.732)         &     (0.316)         &     (0.562)         &     (0.601)         &     (1.076)         &     (0.217)         &     (0.891)         &     (0.734)         &     (1.040)         &     (1.515)         &     (1.142)         &     (0.351)         \\
\addlinespace
SDQ Emotional - Child&     -0.0120         &     0.00747         &       0.479         &      -0.670         &      -1.117         &      0.0565         &    -0.00490         &      -0.166         &     -0.0665         &       0.111         &      -0.109         &       0.659\sym{*}  \\
            &     (1.074)         &     (0.754)         &     (0.870)         &     (1.043)         &     (1.336)         &     (0.147)         &     (0.586)         &     (0.576)         &     (0.766)         &     (0.949)         &     (0.916)         &     (0.295)         \\
\addlinespace
SDQ Conduct - Child&      -0.488         &      -0.425         &      -0.361         &      -0.225         &      -0.675         &      0.0490         &      -1.294\sym{***}&      -1.478\sym{***}&      -1.211\sym{**} &      -0.934         &      -1.510         &       0.630\sym{*}  \\
            &     (0.720)         &     (0.470)         &     (0.815)         &     (0.699)         &     (0.635)         &     (0.131)         &     (0.170)         &     (0.338)         &     (0.419)         &     (0.656)         &     (0.949)         &     (0.265)         \\
\bottomrule
\end{tabular}
}
}
\vspace{1ex} \\
\footnotesize\raggedright{Note: This table shows the estimates of the coefficient for attending Reggio Approach preschools from the OLS and difference-in-difference (DiD) models.Column title indicates the corresponding age group, control set, and model. ``NoneIt'' refers to the OLS estimate with only the Italian children cohort and with no control variables. ``BICIt'' refers to the OLS estimate with only the Italian children cohort and with controls selected by Bayesian Information Criterion (BIC) and additional controls for caregiver's religion. ``FullIt'' refers to the OLS estimate with only the Italian children cohort and with the full set of controls. ``DidPmIt'' refers to the difference-in-difference estimate of (Reggio Muni - Parma Muni) - (Reggio None - Parma None) for the Italian children cohort. ``DidPvIt'' refers to the difference-in-difference estimate of (Reggio Muni - Padova Muni) - (Reggio None - Padova None) for the Italian children cohort.  Analogous meanings are applied to the migrant children cohort in columns (7) through (12). Robust standard errors are reported in parentheses. Stars show statistical significance as follows: * $p < 0.05$, ** $p < 0.01$, *** $p < 0.001$.}
\end{table}

%Table \ref{ols-W-reg} shows consistently significant positive effects of the Reggio Approach on hours worked per week. Moreover, there are significant effects on the indicator for the lowest household income category for the age-40 cohort. Note that the measures for subject labor income, including ``Monthly Wage'' presented in the table, have very few observations. Household income suffers less from the missing value problem.

\begin{table}[H] \caption{OLS and Diff-in-Diff for Health and Risk, Preschools, Children Cohort} \label{ols-W-reg}
\scalebox{0.76}{
{
\def\sym#1{\ifmmode^{#1}\else\(^{#1}\)\fi}
\begin{tabular}{l*{10}{c}}
\toprule
            &\multicolumn{1}{c}{(1)}&\multicolumn{1}{c}{(2)}&\multicolumn{1}{c}{(3)}&\multicolumn{1}{c}{(4)}&\multicolumn{1}{c}{(5)}&\multicolumn{1}{c}{(6)}&\multicolumn{1}{c}{(7)}&\multicolumn{1}{c}{(8)}&\multicolumn{1}{c}{(9)}&\multicolumn{1}{c}{(10)}\\
            &\multicolumn{1}{c}{NoneIt}&\multicolumn{1}{c}{BICIt}&\multicolumn{1}{c}{FullIt}&\multicolumn{1}{c}{DidPmIt}&\multicolumn{1}{c}{DidPvIt}&\multicolumn{1}{c}{NoneMg}&\multicolumn{1}{c}{BICMg}&\multicolumn{1}{c}{FullMg}&\multicolumn{1}{c}{DidPmMg}&\multicolumn{1}{c}{DidPvMg}\\
\midrule
Obese       &      -0.175         &      -0.135         &       0.328         &      0.0533         &      -0.492         &       0.392\sym{***}&       0.306\sym{*}  &       0.224         &       0.765\sym{**} &       0.180         \\
            &     (0.358)         &     (0.332)         &     (0.516)         &     (0.410)         &     (0.384)         &    (0.0696)         &     (0.143)         &     (0.151)         &     (0.279)         &     (0.209)         \\
\addlinespace
Overweight  &       0.139\sym{***}&       0.133\sym{***}&     -0.0114         &       0.155         &      0.0984\sym{*}  &     -0.0735         &     -0.0530         &      -0.153         &      -0.267         &     -0.0131         \\
            &    (0.0270)         &    (0.0342)         &    (0.0922)         &     (0.164)         &    (0.0485)         &     (0.227)         &     (0.262)         &     (0.347)         &     (0.318)         &     (0.331)         \\
\addlinespace
Health is Good&       0.175         &       0.176         &       0.467         &       0.125         &       0.369         &       0.706\sym{***}&       0.823\sym{***}&       0.926\sym{***}&       0.701\sym{*}  &       0.877\sym{***}\\
            &     (0.358)         &     (0.289)         &     (0.328)         &     (0.397)         &     (0.338)         &    (0.0650)         &    (0.0903)         &     (0.252)         &     (0.301)         &    (0.0926)         \\
\addlinespace
Number of Sick Days&      -0.916\sym{*}  &      -0.907\sym{***}&      -1.263\sym{***}&      -0.641         &       0.156         &      -0.321         &      -0.452         &     -0.0137         &      -0.819\sym{*}  &      -0.595         \\
            &     (0.362)         &     (0.242)         &     (0.374)         &     (0.376)         &     (1.113)         &     (0.437)         &     (0.312)         &     (0.419)         &     (0.406)         &     (0.429)         \\
\bottomrule
\end{tabular}
}
}
\vspace{1ex} \\
\footnotesize\raggedright{Note: This table shows the estimates of the coefficient for attending Reggio Approach preschools from the OLS and difference-in-difference (DiD) models.Column title indicates the corresponding age group, control set, and model. ``NoneIt'' refers to the OLS estimate with only the Italian children cohort and with no control variables. ``BICIt'' refers to the OLS estimate with only the Italian children cohort and with controls selected by Bayesian Information Criterion (BIC) and additional controls for caregiver's religion. ``FullIt'' refers to the OLS estimate with only the Italian children cohort and with the full set of controls. ``DidPmIt'' refers to the difference-in-difference estimate of (Reggio Muni - Parma Muni) - (Reggio None - Parma None) for the Italian children cohort. ``DidPvIt'' refers to the difference-in-difference estimate of (Reggio Muni - Padova Muni) - (Reggio None - Padova None) for the Italian children cohort.  Analogous meanings are applied to the migrant children cohort in columns (7) through (12). Robust standard errors are reported in parentheses. Stars show statistical significance as follows: * $p < 0.05$, ** $p < 0.01$, *** $p < 0.001$.}
\end{table}

%Table \ref{ols-L-reg} shows significant effects of the Reggio Approach on the status of being divorced for the age-30 cohort. Other results do not show any significant trend. 

\begin{table}[H] \caption{OLS and Diff-in-Diff for Behaviors, Preschools, Children Cohort} \label{ols-L-reg}
\scalebox{0.80}{
{
\def\sym#1{\ifmmode^{#1}\else\(^{#1}\)\fi}
\begin{tabular}{l*{12}{c}}
\toprule
            &\multicolumn{1}{c}{(1)}&\multicolumn{1}{c}{(2)}&\multicolumn{1}{c}{(3)}&\multicolumn{1}{c}{(4)}&\multicolumn{1}{c}{(5)}&\multicolumn{1}{c}{(6)}&\multicolumn{1}{c}{(7)}&\multicolumn{1}{c}{(8)}&\multicolumn{1}{c}{(9)}&\multicolumn{1}{c}{(10)}&\multicolumn{1}{c}{(11)}&\multicolumn{1}{c}{(12)}\\
            &\multicolumn{1}{c}{NoneIt}&\multicolumn{1}{c}{BICIt}&\multicolumn{1}{c}{FullIt}&\multicolumn{1}{c}{DidPmIt}&\multicolumn{1}{c}{DidPvIt}&\multicolumn{1}{c}{PSMIt}&\multicolumn{1}{c}{NoneMg}&\multicolumn{1}{c}{BICMg}&\multicolumn{1}{c}{FullMg}&\multicolumn{1}{c}{DidPmMg}&\multicolumn{1}{c}{DidPvMg}&\multicolumn{1}{c}{PSMMg}\\
\midrule
Not Excited to Learn&      0.0241\sym{*}  &      0.0213         &    -0.00199         &       0.291         &     -0.0385         &     -0.0277         &      -0.152         &      -0.173         &      -0.213         &      -0.424         &       0.347         &     -0.0400         \\
            &    (0.0120)         &    (0.0182)         &    (0.0367)         &     (0.182)         &    (0.0507)         &    (0.0169)         &     (0.225)         &     (0.240)         &     (0.220)         &     (0.279)         &     (0.339)         &    (0.0495)         \\
\addlinespace
Problems Sitting Still&       0.133\sym{***}&       0.128\sym{***}&       0.329\sym{*}  &       0.150         &     -0.0422         &      0.0205         &       0.137\sym{**} &       0.195\sym{*}  &       0.475\sym{**} &       0.179         &      0.0235         &      0.0748         \\
            &    (0.0265)         &    (0.0294)         &     (0.128)         &     (0.161)         &    (0.0632)         &    (0.0306)         &    (0.0491)         &    (0.0906)         &     (0.180)         &     (0.152)         &    (0.0952)         &    (0.0706)         \\
\addlinespace
How Much Child Likes School&       0.669         &       0.664         &       0.419         &       0.542         &       1.389\sym{*}  &       0.145\sym{**} &      -0.221         &      -0.160         &      -0.534         &      -0.755         &      -1.167\sym{*}  &      -0.247\sym{*}  \\
            &     (0.713)         &     (0.634)         &     (0.540)         &     (0.632)         &     (0.589)         &    (0.0554)         &     (0.240)         &     (0.203)         &     (0.360)         &     (0.526)         &     (0.509)         &     (0.106)         \\
\addlinespace
Happy in General&      -0.289         &      -0.237         &      -0.272         &       0.301         &       0.469         &       0.187         &       0.127         &       0.341         &       0.623         &       1.312         &      -1.808         &      -0.314         \\
            &     (0.723)         &     (0.810)         &     (0.932)         &     (0.889)         &     (0.907)         &     (0.162)         &     (0.623)         &     (0.543)         &     (0.987)         &     (1.009)         &     (1.597)         &     (0.279)         \\
\bottomrule
\end{tabular}
}
}
\vspace{1ex} \\
\footnotesize\raggedright{Note: This table shows the estimates of the coefficient for attending Reggio Approach preschools from the OLS and difference-in-difference (DiD) models.Column title indicates the corresponding age group, control set, and model. ``NoneIt'' refers to the OLS estimate with only the Italian children cohort and with no control variables. ``BICIt'' refers to the OLS estimate with only the Italian children cohort and with controls selected by Bayesian Information Criterion (BIC) and additional controls for caregiver's religion. ``FullIt'' refers to the OLS estimate with only the Italian children cohort and with the full set of controls. ``DidPmIt'' refers to the difference-in-difference estimate of (Reggio Muni - Parma Muni) - (Reggio None - Parma None) for the Italian children cohort. ``DidPvIt'' refers to the difference-in-difference estimate of (Reggio Muni - Padova Muni) - (Reggio None - Padova None) for the Italian children cohort.  Analogous meanings are applied to the migrant children cohort in columns (7) through (12). Robust standard errors are reported in parentheses. Stars show statistical significance as follows: * $p < 0.05$, ** $p < 0.01$, *** $p < 0.001$.}
\end{table}

%Table \ref{ols-H-reg} show that the significantly increasing effects of the Reggio Approach on numbers of days sick for the past month at the time of the interview for the age-30 cohort. Moreover, the results show the significant decreasing effects on whether the respondent has ever been suspended from school. 

\begin{table}[H] \caption{OLS and Diff-in-Diff for Social Behaviors, Preschools, Children} \label{ols-H-reg}
\scalebox{0.77}{
{
\def\sym#1{\ifmmode^{#1}\else\(^{#1}\)\fi}
\begin{tabular}{l*{10}{c}}
\toprule
            &\multicolumn{1}{c}{(1)}&\multicolumn{1}{c}{(2)}&\multicolumn{1}{c}{(3)}&\multicolumn{1}{c}{(4)}&\multicolumn{1}{c}{(5)}&\multicolumn{1}{c}{(6)}&\multicolumn{1}{c}{(7)}&\multicolumn{1}{c}{(8)}&\multicolumn{1}{c}{(9)}&\multicolumn{1}{c}{(10)}\\
            &\multicolumn{1}{c}{NoneIt}&\multicolumn{1}{c}{BICIt}&\multicolumn{1}{c}{FullIt}&\multicolumn{1}{c}{DidPmIt}&\multicolumn{1}{c}{DidPvIt}&\multicolumn{1}{c}{NoneMg}&\multicolumn{1}{c}{BICMg}&\multicolumn{1}{c}{FullMg}&\multicolumn{1}{c}{DidPmMg}&\multicolumn{1}{c}{DidPvMg}\\
\midrule
Musical Instrument at Home&     -0.0313         &     -0.0281         &      -0.309         &       0.198         &      -0.607         &       0.188\sym{**} &       0.191\sym{*}  &      0.0837         &     -0.0849         &      0.0771         \\
            &     (0.358)         &     (0.303)         &     (0.343)         &     (0.368)         &     (0.310)         &    (0.0575)         &    (0.0957)         &     (0.155)         &     (0.219)         &     (0.107)         \\
\addlinespace
Tell Worry at Home&      -0.362\sym{***}&      -0.345\sym{**} &      -0.387\sym{*}  &      -0.337         &      -0.573         &      -0.417\sym{***}&      -0.457\sym{***}&     -0.0964         &      -0.211         &      -0.666\sym{*}  \\
            &    (0.0382)         &     (0.117)         &     (0.170)         &     (0.210)         &     (0.385)         &    (0.0726)         &     (0.137)         &     (0.201)         &     (0.200)         &     (0.303)         \\
\addlinespace
Tell Worry to Teacher&       0.337\sym{***}&       0.359\sym{***}&       0.357\sym{*}  &       0.425         &       0.139         &       0.208\sym{***}&       0.160         &      0.0922         &       0.835\sym{***}&       0.223         \\
            &    (0.0376)         &    (0.0981)         &     (0.167)         &     (0.219)         &     (0.113)         &    (0.0598)         &    (0.0856)         &     (0.147)         &     (0.253)         &     (0.251)         \\
\addlinespace
Tell Worry to Friends&       0.188\sym{***}&       0.209\sym{***}&     -0.0475         &       0.451\sym{*}  &     -0.0429         &       0.146\sym{**} &       0.194         &     -0.0189         &       0.207         &     -0.0502         \\
            &    (0.0310)         &    (0.0594)         &     (0.155)         &     (0.188)         &    (0.0887)         &    (0.0520)         &    (0.0997)         &     (0.108)         &     (0.276)         &     (0.105)         \\
\addlinespace
Keep Worry to Myself&       0.125\sym{***}&      0.0853         &      0.0559         &    -0.00588         &       0.529         &       0.208\sym{***}&       0.257\sym{*}  &      0.0445         &       0.322         &       0.385         \\
            &    (0.0263)         &    (0.0652)         &     (0.114)         &    (0.0909)         &     (0.379)         &    (0.0598)         &     (0.116)         &     (0.188)         &     (0.182)         &     (0.255)         \\
\bottomrule
\end{tabular}
}
}
\vspace{1ex} \\
\footnotesize\raggedright{Note: This table shows the estimates of the coefficient for attending Reggio Approach preschools from the OLS and difference-in-difference (DiD) models.Column title indicates the corresponding age group, control set, and model. ``NoneIt'' refers to the OLS estimate with only the Italian children cohort and with no control variables. ``BICIt'' refers to the OLS estimate with only the Italian children cohort and with controls selected by Bayesian Information Criterion (BIC) and additional controls for caregiver's religion. ``FullIt'' refers to the OLS estimate with only the Italian children cohort and with the full set of controls. ``DidPmIt'' refers to the difference-in-difference estimate of (Reggio Muni - Parma Muni) - (Reggio None - Parma None) for the Italian children cohort. ``DidPvIt'' refers to the difference-in-difference estimate of (Reggio Muni - Padova Muni) - (Reggio None - Padova None) for the Italian children cohort.  Analogous meanings are applied to the migrant children cohort in columns (7) through (12). Robust standard errors are reported in parentheses. Stars show statistical significance as follows: * $p < 0.05$, ** $p < 0.01$, *** $p < 0.001$.}
\end{table}
\end{landscape}


\subsubsection{Adolescents}

\begin{table}[H] \caption{OLS and Diff-in-Diff for Cognitive and Noncognitive, Preschools, Adolescent Cohort} \label{ols-E-reg}
\scalebox{0.72}{{
\def\sym#1{\ifmmode^{#1}\else\(^{#1}\)\fi}
\begin{tabular}{l*{6}{c}}
\toprule
            &\multicolumn{1}{c}{(1)}&\multicolumn{1}{c}{(2)}&\multicolumn{1}{c}{(3)}&\multicolumn{1}{c}{(4)}&\multicolumn{1}{c}{(5)}&\multicolumn{1}{c}{(6)}\\
            &\multicolumn{1}{c}{None}&\multicolumn{1}{c}{BIC}&\multicolumn{1}{c}{Full}&\multicolumn{1}{c}{DidPm}&\multicolumn{1}{c}{DidPv}&\multicolumn{1}{c}{PSM}\\
\midrule
IQ Factor   &       0.347         &       0.156         &     -0.0205         &      -0.276         &      -2.221\sym{***}&       0.134         \\
            &     (0.507)         &     (0.543)         &     (0.617)         &     (0.873)         &     (0.525)         &     (0.106)         \\
\addlinespace
IQ Score    &       0.104         &      0.0509         &     -0.0206         &     -0.0228         &      -0.628\sym{***}&      0.0392         \\
            &     (0.144)         &     (0.155)         &     (0.172)         &     (0.255)         &     (0.150)         &    (0.0310)         \\
\addlinespace
SDQ Composite - Child&       0.753         &       1.684         &       2.833         &      -0.262         &      -4.888\sym{**} &       0.208         \\
            &     (1.606)         &     (1.804)         &     (2.453)         &     (2.938)         &     (1.747)         &     (0.632)         \\
\addlinespace
SDQ Pro-social - Child&       0.664         &       0.645         &       0.572         &       1.431         &       0.132         &      -0.116         \\
            &     (0.585)         &     (0.577)         &     (0.871)         &     (0.805)         &     (0.626)         &     (0.208)         \\
\addlinespace
SDQ Peer problems - Child&      0.0993         &       0.259         &      -0.384         &      -0.393         &       0.149         &      -0.207         \\
            &     (0.417)         &     (0.432)         &     (0.549)         &     (0.476)         &     (0.478)         &     (0.162)         \\
\addlinespace
SDQ Hyper - Child&      -0.465         &      -0.350         &       0.204         &       0.187         &      -4.926\sym{***}&       0.366         \\
            &     (0.742)         &     (0.664)         &     (1.039)         &     (0.905)         &     (0.697)         &     (0.305)         \\
\addlinespace
SDQ Emotional - Child&       1.024         &       1.375         &       2.052\sym{*}  &       0.241         &      -0.956         &      -0.162         \\
            &     (0.654)         &     (0.782)         &     (0.893)         &     (1.176)         &     (0.768)         &     (0.220)         \\
\addlinespace
SDQ Conduct - Child&      0.0958         &       0.400         &       0.960         &      -0.297         &       0.809         &       0.205         \\
            &     (0.646)         &     (0.655)         &     (0.878)         &     (1.284)         &     (0.683)         &     (0.192)         \\
\addlinespace
SDQ Composite&      -2.702         &      -1.481         &      -1.045         &      -1.875         &      -4.071\sym{*}  &       0.555         \\
            &     (1.725)         &     (2.049)         &     (1.987)         &     (2.220)         &     (2.069)         &     (0.762)         \\
\addlinespace
SDQ Pro-social&       0.687         &       0.572         &       0.999\sym{*}  &      -0.452         &      -2.672\sym{***}&      -0.432         \\
            &     (0.709)         &     (0.757)         &     (0.451)         &     (0.901)         &     (0.800)         &     (0.229)         \\
\addlinespace
SDQ Peer problems&    -0.00353         &       0.256         &      -0.223         &       0.171         &       0.805         &       0.209         \\
            &     (0.616)         &     (0.651)         &     (0.333)         &     (0.730)         &     (0.692)         &     (0.207)         \\
\addlinespace
SDQ Hyper   &      -1.229         &      -1.067         &      -0.905         &      -0.705         &      -1.460         &      0.0957         \\
            &     (0.735)         &     (0.785)         &     (1.039)         &     (0.911)         &     (0.816)         &     (0.281)         \\
\addlinespace
SDQ Emotional&      -0.908         &      -0.336         &      -0.153         &      -0.563         &      -4.098\sym{***}&     -0.0602         \\
            &     (0.507)         &     (0.509)         &     (0.514)         &     (0.954)         &     (0.572)         &     (0.287)         \\
\addlinespace
SDQ Conduct &      -0.561         &      -0.333         &       0.236         &      -0.779         &       0.683         &       0.311         \\
            &     (0.394)         &     (0.480)         &     (0.570)         &     (0.534)         &     (0.499)         &     (0.193)         \\
\addlinespace
Depression Score - positive&      -3.445\sym{*}  &      -2.433\sym{*}  &      -3.930\sym{**} &      -1.539         &      -10.09\sym{***}&      -0.543         \\
            &     (1.491)         &     (1.131)         &     (1.506)         &     (2.567)         &     (1.546)         &     (0.739)         \\
\bottomrule
\end{tabular}
}
}
\vspace{1ex} \\
\footnotesize\raggedright{Note: This table shows the estimates of the coefficient for attending Reggio Approach preschools from the OLS and difference-in-difference (DiD) models.Column title indicates the corresponding age group, control set, and model. ``None'' refers to the OLS estimate with only the Italian children cohort and with no control variables. ``BIC'' refers to the OLS estimate with only the Italian children cohort and with controls selected by Bayesian Information Criterion (BIC) and additional controls for caregiver's religion. ``Full'' refers to the OLS estimate with only the Italian children cohort and with the full set of controls. ``DidPm'' refers to the difference-in-difference estimate of (Reggio Muni - Parma Muni) - (Reggio None - Parma None) for the Italian children cohort. ``DidPv'' refers to the difference-in-difference estimate of (Reggio Muni - Padova Muni) - (Reggio None - Padova None) for the Italian children cohort.  Robust standard errors are reported in parentheses. Stars show statistical significance as follows: * $p < 0.05$, ** $p < 0.01$, *** $p < 0.001$.}
\end{table}

%Table \ref{ols-W-reg} shows consistently significant positive effects of the Reggio Approach on hours worked per week. Moreover, there are significant effects on the indicator for the lowest household income category for the age-40 cohort. Note that the measures for subject labor income, including ``Monthly Wage'' presented in the table, have very few observations. Household income suffers less from the missing value problem.


\begin{table}[H] \caption{OLS and Diff-in-Diff for Health and Risk, Preschools, Adolescent Cohort} \label{ols-W-reg}
\scalebox{0.76}{
{
\def\sym#1{\ifmmode^{#1}\else\(^{#1}\)\fi}
\begin{tabular}{l*{5}{c}}
\toprule
            &\multicolumn{1}{c}{(1)}&\multicolumn{1}{c}{(2)}&\multicolumn{1}{c}{(3)}&\multicolumn{1}{c}{(4)}&\multicolumn{1}{c}{(5)}\\
            &\multicolumn{1}{c}{None}&\multicolumn{1}{c}{BIC}&\multicolumn{1}{c}{Full}&\multicolumn{1}{c}{DidPm}&\multicolumn{1}{c}{DidPv}\\
\midrule
Obese       &     -0.0937         &     -0.0745         &      -0.316         &     -0.0153         &       0.672\sym{***}\\
            &     (0.175)         &     (0.175)         &     (0.229)         &     (0.263)         &     (0.190)         \\
\addlinespace
Overweight  &      -0.110         &      -0.105         &      -0.166         &      0.0141         &      -0.127         \\
            &     (0.134)         &     (0.122)         &     (0.162)         &     (0.242)         &     (0.131)         \\
\addlinespace
Health is Good&     -0.0322         &      0.0261         &      -0.311\sym{*}  &       0.580\sym{**} &      -0.597\sym{**} \\
            &     (0.176)         &     (0.176)         &     (0.125)         &     (0.206)         &     (0.187)         \\
\addlinespace
Number of Sick Days&       0.110         &     -0.0919         &      -0.183         &      -0.176         &       1.530\sym{***}\\
            &     (0.230)         &     (0.250)         &     (0.241)         &     (0.419)         &     (0.266)         \\
\addlinespace
Ever Suspended from School&     -0.0700         &     -0.0591         &     -0.0541         &     -0.0470         &      -0.148         \\
            &     (0.135)         &     (0.110)         &     (0.139)         &     (0.117)         &     (0.120)         \\
\addlinespace
Num. of Cigarettes Per Day&      -1.679\sym{*}  &      -1.246         &       6.974         &       3.683\sym{*}  &      -1.435         \\
            &     (0.773)         &     (2.676)         &     (7.698)         &     (1.462)         &     (2.078)         \\
\bottomrule
\end{tabular}
}
}
\vspace{1ex} \\
\footnotesize\raggedright{Note: This table shows the estimates of the coefficient for attending Reggio Approach preschools from the OLS and difference-in-difference (DiD) models.Column title indicates the corresponding age group, control set, and model. ``None'' refers to the OLS estimate with only the Italian children cohort and with no control variables. ``BIC'' refers to the OLS estimate with only the Italian children cohort and with controls selected by Bayesian Information Criterion (BIC) and additional controls for caregiver's religion. ``Full'' refers to the OLS estimate with only the Italian children cohort and with the full set of controls. ``DidPm'' refers to the difference-in-difference estimate of (Reggio Muni - Parma Muni) - (Reggio None - Parma None) for the Italian children cohort. ``DidPv'' refers to the difference-in-difference estimate of (Reggio Muni - Padova Muni) - (Reggio None - Padova None) for the Italian children cohort.  Robust standard errors are reported in parentheses. Stars show statistical significance as follows: * $p < 0.05$, ** $p < 0.01$, *** $p < 0.001$.}
\end{table}

%Table \ref{ols-L-reg} shows significant effects of the Reggio Approach on the status of being divorced for the age-30 cohort. Other results do not show any significant trend. 

\begin{table}[H] \caption{OLS and Diff-in-Diff for Behaviors, Preschools, Adolescent Cohort} \label{ols-L-reg}
\scalebox{0.80}{
{
\def\sym#1{\ifmmode^{#1}\else\(^{#1}\)\fi}
\begin{tabular}{l*{6}{c}}
\toprule
            &\multicolumn{1}{c}{(1)}&\multicolumn{1}{c}{(2)}&\multicolumn{1}{c}{(3)}&\multicolumn{1}{c}{(4)}&\multicolumn{1}{c}{(5)}&\multicolumn{1}{c}{(6)}\\
            &\multicolumn{1}{c}{None}&\multicolumn{1}{c}{BIC}&\multicolumn{1}{c}{Full}&\multicolumn{1}{c}{DidPm}&\multicolumn{1}{c}{DidPv}&\multicolumn{1}{c}{IPW}\\
\midrule
Not Excited to Learn&      -0.255         &      -0.244         &      -0.359         &     -0.0497         &      -0.281         &     -0.0221         \\
            &     (0.172)         &     (0.167)         &     (0.224)         &     (0.292)         &     (0.169)         &    (0.0205)         \\
\addlinespace
Problems Sitting Still&     -0.0758         &     -0.0622         &      -0.121         &      0.0985         &      -0.227         &     -0.0397         \\
            &     (0.134)         &     (0.145)         &     (0.214)         &     (0.240)         &     (0.156)         &    (0.0306)         \\
\addlinespace
Go To School&       0.261         &       0.234         &       0.340\sym{*}  &       0.279         &       0.264         &      0.0223         \\
            &     (0.172)         &     (0.139)         &     (0.146)         &     (0.143)         &     (0.147)         &    (0.0196)         \\
\addlinespace
How Much Child Likes School&      -0.277         &      -0.161         &      -0.178         &      0.0521         &      -0.688         &     -0.0222         \\
            &     (0.410)         &     (0.400)         &     (0.380)         &     (0.418)         &     (0.423)         &    (0.0980)         \\
\addlinespace
Bothered by Migrants&       0.110         &      0.0377         &     -0.0962         &       0.601\sym{*}  &      -0.299         &     -0.0326         \\
            &     (0.137)         &     (0.172)         &     (0.237)         &     (0.300)         &     (0.179)         &    (0.0577)         \\
\addlinespace
Trust Score &      -0.362         &      -0.466         &      -1.073\sym{*}  &     -0.0179         &       0.487         &     -0.0413         \\
            &     (0.500)         &     (0.483)         &     (0.511)         &     (0.774)         &     (0.540)         &     (0.196)         \\
\addlinespace
Days of Sport (Weekly)&      -0.168         &     -0.0271         &      -0.536         &      -1.286         &      -3.600\sym{***}&      -0.224         \\
            &     (0.838)         &     (0.851)         &     (1.023)         &     (0.905)         &     (0.902)         &     (0.236)         \\
\bottomrule
\end{tabular}
}
}
\vspace{1ex} \\
\footnotesize\raggedright{Note: This table shows the estimates of the coefficient for attending Reggio Approach preschools from the OLS and difference-in-difference (DiD) models.Column title indicates the corresponding age group, control set, and model. ``None'' refers to the OLS estimate with only the Italian children cohort and with no control variables. ``BIC'' refers to the OLS estimate with only the Italian children cohort and with controls selected by Bayesian Information Criterion (BIC) and additional controls for caregiver's religion. ``Full'' refers to the OLS estimate with only the Italian children cohort and with the full set of controls. ``DidPm'' refers to the difference-in-difference estimate of (Reggio Muni - Parma Muni) - (Reggio None - Parma None) for the Italian children cohort. ``DidPv'' refers to the difference-in-difference estimate of (Reggio Muni - Padova Muni) - (Reggio None - Padova None) for the Italian children cohort.  Robust standard errors are reported in parentheses. Stars show statistical significance as follows: * $p < 0.05$, ** $p < 0.01$, *** $p < 0.001$.}
\end{table}

%Table \ref{ols-H-reg} show that the significantly increasing effects of the Reggio Approach on numbers of days sick for the past month at the time of the interview for the age-30 cohort. Moreover, the results show the significant decreasing effects on whether the respondent has ever been suspended from school. 

\begin{table}[H] \caption{OLS and Diff-in-Diff for Social Behaviors, Preschools, Adolescent Cohort} \label{ols-H-reg}
\scalebox{0.77}{
{
\def\sym#1{\ifmmode^{#1}\else\(^{#1}\)\fi}
\begin{tabular}{l*{5}{c}}
\toprule
            &\multicolumn{1}{c}{(1)}&\multicolumn{1}{c}{(2)}&\multicolumn{1}{c}{(3)}&\multicolumn{1}{c}{(4)}&\multicolumn{1}{c}{(5)}\\
            &\multicolumn{1}{c}{None}&\multicolumn{1}{c}{BIC}&\multicolumn{1}{c}{Full}&\multicolumn{1}{c}{DidPm}&\multicolumn{1}{c}{DidPv}\\
\midrule
Num. of Friends&       3.462\sym{*}  &       3.651\sym{*}  &       1.684         &       4.867         &       3.872\sym{**} \\
            &     (1.348)         &     (1.720)         &     (1.892)         &     (4.870)         &     (1.491)         \\
\addlinespace
Doesn't Talk About Activities&      -0.301         &      -0.336         &      -0.402         &      -0.316         &      -0.895\sym{**} \\
            &     (0.249)         &     (0.262)         &     (0.427)         &     (0.351)         &     (0.293)         \\
\addlinespace
Doesn't Talk About School&      -0.380         &      -0.399         &      -0.557         &      -0.118         &      -0.928\sym{**} \\
            &     (0.287)         &     (0.323)         &     (0.488)         &     (0.425)         &     (0.339)         \\
\bottomrule
\end{tabular}
}
}
\vspace{1ex} \\
\footnotesize\raggedright{Note: This table shows the estimates of the coefficient for attending Reggio Approach preschools from the OLS and difference-in-difference (DiD) models.Column title indicates the corresponding age group, control set, and model. ``None'' refers to the OLS estimate with only the Italian children cohort and with no control variables. ``BIC'' refers to the OLS estimate with only the Italian children cohort and with controls selected by Bayesian Information Criterion (BIC) and additional controls for caregiver's religion. ``Full'' refers to the OLS estimate with only the Italian children cohort and with the full set of controls. ``DidPm'' refers to the difference-in-difference estimate of (Reggio Muni - Parma Muni) - (Reggio None - Parma None) for the Italian children cohort. ``DidPv'' refers to the difference-in-difference estimate of (Reggio Muni - Padova Muni) - (Reggio None - Padova None) for the Italian children cohort.  Robust standard errors are reported in parentheses. Stars show statistical significance as follows: * $p < 0.05$, ** $p < 0.01$, *** $p < 0.001$.}
\end{table}









\subsubsection{Adults}
\begin{landscape}

\begin{table}[H] \caption{OLS and Diff-in-Diff for Cognitive and Education, Preschools, Adult Cohorts} \label{ols-E-reg}
\scalebox{0.80}{{
\def\sym#1{\ifmmode^{#1}\else\(^{#1}\)\fi}
\begin{tabular}{l*{12}{c}}
\toprule
            &\multicolumn{1}{c}{(1)}&\multicolumn{1}{c}{(2)}&\multicolumn{1}{c}{(3)}&\multicolumn{1}{c}{(4)}&\multicolumn{1}{c}{(5)}&\multicolumn{1}{c}{(6)}&\multicolumn{1}{c}{(7)}&\multicolumn{1}{c}{(8)}&\multicolumn{1}{c}{(9)}&\multicolumn{1}{c}{(10)}&\multicolumn{1}{c}{(11)}&\multicolumn{1}{c}{(12)}\\
            &\multicolumn{1}{c}{None30}&\multicolumn{1}{c}{BIC30}&\multicolumn{1}{c}{Full30}&\multicolumn{1}{c}{DidPm30}&\multicolumn{1}{c}{DidPv30}&\multicolumn{1}{c}{IPW30}&\multicolumn{1}{c}{None40}&\multicolumn{1}{c}{BIC40}&\multicolumn{1}{c}{Full40}&\multicolumn{1}{c}{DidPm40}&\multicolumn{1}{c}{DidPv40}&\multicolumn{1}{c}{IPW40}\\
\midrule
IQ Factor   &     0.00113         &      -0.226         &      -0.278         &      -0.457\sym{*}  &      -0.385         &      -0.674\sym{***}&     0.00654         &     0.00890         &      0.0505         &      0.0883         &       0.478\sym{*}  &     -0.0253         \\
            &     (0.167)         &     (0.159)         &     (0.163)         &     (0.213)         &     (0.259)         &     (0.113)         &     (0.130)         &     (0.138)         &     (0.160)         &     (0.168)         &     (0.227)         &     (0.106)         \\
\addlinespace
High School Grade&       5.644\sym{**} &       6.592\sym{**} &       6.123\sym{*}  &      -0.425         &       5.520         &       5.899\sym{***}&       0.834         &       1.879         &       1.970         &      -2.791         &       5.080         &       4.691\sym{**} \\
            &     (2.043)         &     (2.256)         &     (2.415)         &     (4.268)         &     (4.006)         &     (1.397)         &     (1.503)         &     (1.582)         &     (1.861)         &     (3.033)         &     (3.173)         &     (1.470)         \\
\addlinespace
University Grade&       2.940         &       1.194         &       1.314         &       4.224         &      -1.730         &       1.589         &      -0.498         &      -0.580         &      -7.824         &      -4.709         &      -5.877         &      -5.692\sym{**} \\
            &     (2.193)         &     (2.484)         &     (2.921)         &     (3.575)         &     (4.148)         &     (1.557)         &     (2.519)         &     (2.871)         &     (4.962)         &     (3.325)         &     (4.121)         &     (1.754)         \\
\addlinespace
Graduate from High School&     -0.0316         &      0.0387         &      0.0438         &       0.121         &     -0.0262         &     -0.0302         &      -0.108\sym{*}  &     -0.0425         &     -0.0706         &     -0.0657         &      -0.202\sym{*}  &      -0.135         \\
            &    (0.0542)         &    (0.0526)         &    (0.0541)         &    (0.0841)         &    (0.0946)         &    (0.0332)         &    (0.0515)         &    (0.0514)         &    (0.0604)         &    (0.0745)         &    (0.0980)         &    (0.0755)         \\
\addlinespace
Max Edu: University&     -0.0596         &     0.00197         &     -0.0134         &      0.0519         &      -0.168         &      -0.239\sym{***}&      0.0106         &      0.0612         &       0.112         &      -0.240\sym{*}  &     -0.0497         &      -0.241         \\
            &    (0.0737)         &    (0.0753)         &    (0.0780)         &     (0.117)         &     (0.135)         &    (0.0511)         &    (0.0583)         &    (0.0607)         &    (0.0582)         &     (0.106)         &     (0.123)         &     (0.177)         \\
\addlinespace
Max Edu: Graduate School&           0         &           0         &           0         &     -0.0165         &           0         &     -0.0720\sym{***}&           0         &           0         &           0         &     -0.0675         &     -0.0272         &     -0.0107         \\
            &         (.)         &         (.)         &         (.)         &    (0.0164)         &         (.)         &    (0.0153)         &         (.)         &         (.)         &         (.)         &    (0.0468)         &    (0.0419)         &   (0.00560)         \\
\bottomrule
\end{tabular}
}
}
\vspace{1ex} \\
\footnotesize\raggedright{Note: This table shows the estimates of the coefficient for attending Reggio Approach preschools from the OLS and difference-in-difference (DiD) models.Column title indicates the corresponding age group, control set, and model. ``None30'' refers to the OLS estimate with only the age-30 cohort and with no control variables. ``BIC30'' refers to the OLS estimate with only the age-30 cohort and with controls selected by Bayesian Information Criterion (BIC). ``Full30'' refers to the OLS estimate with only the age-30 cohort and with the full set of controls. ``DidPm30'' refers to the difference-in-difference estimate of (Reggio Muni - Parma Muni) - (Reggio None - Parma None) for the age-30 cohort. ``DidPv30'' refers to the difference-in-difference estimate of (Reggio Muni - Padova Muni) - (Reggio None - Padova None) for the age-30 cohort.  Analogous meanings are applied to the age-40 cohort in columns (7) through (12). Robust standard errors are reported in parentheses. Stars show statistical significance as follows: * $p < 0.05$, ** $p < 0.01$, *** $p < 0.001$.}
\end{table}

%Table \ref{ols-W-reg} shows consistently significant positive effects of the Reggio Approach on hours worked per week. Moreover, there are significant effects on the indicator for the lowest household income category for the age-40 cohort. Note that the measures for subject labor income, including ``Monthly Wage'' presented in the table, have very few observations. Household income suffers less from the missing value problem.

\begin{table}[H] \caption{OLS and Diff-in-Diff for Employment and Income, Preschools, Adult Cohorts} \label{ols-W-reg}
\scalebox{0.76}{
{
\def\sym#1{\ifmmode^{#1}\else\(^{#1}\)\fi}
\begin{tabular}{l*{12}{c}}
\toprule
            &\multicolumn{1}{c}{(1)}&\multicolumn{1}{c}{(2)}&\multicolumn{1}{c}{(3)}&\multicolumn{1}{c}{(4)}&\multicolumn{1}{c}{(5)}&\multicolumn{1}{c}{(6)}&\multicolumn{1}{c}{(7)}&\multicolumn{1}{c}{(8)}&\multicolumn{1}{c}{(9)}&\multicolumn{1}{c}{(10)}&\multicolumn{1}{c}{(11)}&\multicolumn{1}{c}{(12)}\\
            &\multicolumn{1}{c}{None30}&\multicolumn{1}{c}{BIC30}&\multicolumn{1}{c}{Full30}&\multicolumn{1}{c}{DidPm30}&\multicolumn{1}{c}{DidPv30}&\multicolumn{1}{c}{PSM30}&\multicolumn{1}{c}{None40}&\multicolumn{1}{c}{BIC40}&\multicolumn{1}{c}{Full40}&\multicolumn{1}{c}{DidPm40}&\multicolumn{1}{c}{DidPv40}&\multicolumn{1}{c}{PSM40}\\
\midrule
Employed    &      0.0526         &      0.0319         &      0.0630         &       0.108         &      0.0200         &      0.0292         &      0.0577         &      0.0787\sym{*}  &      0.0469         &      0.0634         &      0.0229         &     -0.0119         \\
            &    (0.0470)         &    (0.0453)         &    (0.0435)         &    (0.0721)         &    (0.0908)         &    (0.0298)         &    (0.0367)         &    (0.0338)         &    (0.0344)         &    (0.0531)         &    (0.0735)         &    (0.0247)         \\
\addlinespace
Self-Employed&     -0.0753         &      -0.109         &     -0.0884         &     -0.0352         &      -0.101         &     0.00227         &      0.0137         &     0.00706         &     0.00527         &     0.00540         &       0.148\sym{*}  &      0.0362         \\
            &    (0.0614)         &    (0.0647)         &    (0.0675)         &    (0.0930)         &    (0.0807)         &    (0.0298)         &    (0.0499)         &    (0.0530)         &    (0.0563)         &    (0.0840)         &    (0.0682)         &    (0.0307)         \\
\addlinespace
Hours Worked Per Week&       6.275\sym{**} &       5.415\sym{*}  &       5.398\sym{**} &       5.877\sym{*}  &       4.399         &       1.009         &       3.799         &       3.956\sym{*}  &       5.482\sym{**} &       2.206         &       5.587         &       16.00\sym{***}\\
            &     (2.022)         &     (2.128)         &     (2.058)         &     (2.732)         &     (2.593)         &     (1.000)         &     (1.944)         &     (1.877)         &     (2.096)         &     (2.227)         &     (3.634)         &     (1.787)         \\
\addlinespace
Monthly Wage&      -81.65         &      -105.8         &       16.86         &      -214.1         &      -167.2         &      -238.4         &      -598.8         &      -286.8         &       406.2         &      -614.2         &       201.5         &       957.3         \\
            &     (88.62)         &     (132.1)         &     (118.1)         &     (296.3)         &     (190.2)         &     (135.1)         &     (763.0)         &     (759.0)         &     (867.3)         &    (1176.6)         &     (993.7)         &     (518.5)         \\
\addlinespace
Income: 5,000 Euros of Less&      0.0842\sym{**} &      0.0951\sym{**} &      0.0841\sym{**} &      0.0976\sym{**} &      0.0344         &      0.0516\sym{*}  &     -0.0125         &     -0.0211         &     -0.0187         &     -0.0140         &    -0.00844         &     0.00356         \\
            &    (0.0287)         &    (0.0334)         &    (0.0313)         &    (0.0343)         &    (0.0611)         &    (0.0250)         &    (0.0125)         &    (0.0197)         &    (0.0176)         &    (0.0196)         &   (0.00925)         &   (0.00431)         \\
\addlinespace
Income: 5,001-10,000 Euros&     -0.0246         &     -0.0152         &     -0.0164         &      0.0381         &    -0.00559         &     0.00315         &           0         &           0         &           0         &      0.0106         &     -0.0705         &     -0.0102         \\
            &    (0.0267)         &    (0.0268)         &    (0.0202)         &    (0.0421)         &    (0.0295)         &   (0.00929)         &         (.)         &         (.)         &         (.)         &   (0.00892)         &    (0.0477)         &   (0.00627)         \\
\addlinespace
Income: 10,001-25,000 Euros&      -0.200\sym{*}  &      -0.186\sym{*}  &      -0.193\sym{*}  &      -0.287\sym{*}  &     -0.0573         &      -0.193\sym{***}&      -0.123         &      -0.115         &      -0.135         &      0.0555         &     -0.0822         &      -0.146         \\
            &    (0.0809)         &    (0.0834)         &    (0.0885)         &     (0.131)         &     (0.153)         &    (0.0545)         &    (0.0686)         &    (0.0691)         &    (0.0805)         &     (0.109)         &     (0.140)         &     (0.181)         \\
\addlinespace
Income: 25,001-50,000 Euros&       0.112         &       0.101         &       0.113         &       0.184         &      0.0636         &      0.0835         &       0.122         &       0.113         &      0.0301         &     0.00754         &       0.119         &      0.0414         \\
            &    (0.0838)         &    (0.0887)         &    (0.0897)         &     (0.134)         &     (0.156)         &    (0.0595)         &    (0.0741)         &    (0.0722)         &    (0.0845)         &     (0.122)         &     (0.144)         &     (0.186)         \\
\addlinespace
Income: 50,001-100,000 Euros&      0.0281         &     0.00532         &      0.0125         &     -0.0332         &     -0.0351         &      0.0600         &      0.0450         &      0.0407         &      0.0928\sym{*}  &      0.0153         &      0.0799         &      0.0811         \\
            &    (0.0351)         &    (0.0286)         &    (0.0324)         &    (0.0585)         &    (0.0547)         &    (0.0389)         &    (0.0353)         &    (0.0359)         &    (0.0408)         &    (0.0660)         &    (0.0674)         &    (0.0634)         \\
\addlinespace
Income: 100,001-250,000 Euros&           0         &           0         &           0         &           0         &           0         &    -0.00540         &     -0.0316         &     -0.0168         &      0.0305         &     -0.0750         &     -0.0377         &      0.0306         \\
            &         (.)         &         (.)         &         (.)         &         (.)         &         (.)         &   (0.00388)         &    (0.0325)         &    (0.0332)         &    (0.0359)         &    (0.0507)         &    (0.0328)         &    (0.0270)         \\
\addlinespace
Income: More than 250,000 Euros&           0         &           0         &           0         &           0         &           0         &           0         &           0         &           0         &           0         &           0         &           0         &           0         \\
            &         (.)         &         (.)         &         (.)         &         (.)         &         (.)         &         (.)         &         (.)         &         (.)         &         (.)         &         (.)         &         (.)         &         (.)         \\
\bottomrule
\end{tabular}
}
}
\vspace{1ex} \\
\footnotesize\raggedright{Note: This table shows the estimates of the coefficient for attending Reggio Approach preschools from the OLS and difference-in-difference (DiD) models.Column title indicates the corresponding age group, control set, and model. ``None30'' refers to the OLS estimate with only the age-30 cohort and with no control variables. ``BIC30'' refers to the OLS estimate with only the age-30 cohort and with controls selected by Bayesian Information Criterion (BIC). ``Full30'' refers to the OLS estimate with only the age-30 cohort and with the full set of controls. ``DidPm30'' refers to the difference-in-difference estimate of (Reggio Muni - Parma Muni) - (Reggio None - Parma None) for the age-30 cohort. ``DidPv30'' refers to the difference-in-difference estimate of (Reggio Muni - Padova Muni) - (Reggio None - Padova None) for the age-30 cohort.  Analogous meanings are applied to the age-40 cohort in columns (7) through (12). Robust standard errors are reported in parentheses. Stars show statistical significance as follows: * $p < 0.05$, ** $p < 0.01$, *** $p < 0.001$.}
\end{table}

%Table \ref{ols-L-reg} shows significant effects of the Reggio Approach on the status of being divorced for the age-30 cohort. Other results do not show any significant trend. 

\begin{table}[H] \caption{OLS and Diff-in-Diff for Living Environment, Preschools, Adult Cohorts} \label{ols-L-reg}
\scalebox{0.80}{
{
\def\sym#1{\ifmmode^{#1}\else\(^{#1}\)\fi}
\begin{tabular}{l*{10}{c}}
\toprule
            &\multicolumn{1}{c}{(1)}&\multicolumn{1}{c}{(2)}&\multicolumn{1}{c}{(3)}&\multicolumn{1}{c}{(4)}&\multicolumn{1}{c}{(5)}&\multicolumn{1}{c}{(6)}&\multicolumn{1}{c}{(7)}&\multicolumn{1}{c}{(8)}&\multicolumn{1}{c}{(9)}&\multicolumn{1}{c}{(10)}\\
            &\multicolumn{1}{c}{None30}&\multicolumn{1}{c}{BIC30}&\multicolumn{1}{c}{Full30}&\multicolumn{1}{c}{DidPm30}&\multicolumn{1}{c}{DidPv30}&\multicolumn{1}{c}{None40}&\multicolumn{1}{c}{BIC40}&\multicolumn{1}{c}{Full40}&\multicolumn{1}{c}{DidPm40}&\multicolumn{1}{c}{DidPv40}\\
\midrule
Married or Cohabitating&    0.000187         &     -0.0910         &      -0.101         &      -0.121         &    0.000902         &      0.0167         &      0.0101         &      0.0455         &      -0.125         &       0.132         \\
            &    (0.0807)         &    (0.0827)         &    (0.0864)         &     (0.136)         &     (0.161)         &    (0.0662)         &    (0.0688)         &    (0.0802)         &     (0.174)         &     (0.153)         \\
\addlinespace
Divorced    &           0         &           0         &           0         &    0.000297         &           0         &     -0.0583         &     -0.0394         &     -0.0297         &     -0.0626         &    -0.00805         \\
            &         (.)         &         (.)         &         (.)         &    (0.0303)         &         (.)         &    (0.0456)         &    (0.0493)         &    (0.0556)         &     (0.141)         &     (0.117)         \\
\addlinespace
Num. of Children in House&    -0.00952         &     -0.0300         &     -0.0440         &      -0.119         &     -0.0441         &     0.00694         &     -0.0387         &     -0.0416         &      -0.349         &      -0.199         \\
            &    (0.0574)         &    (0.0518)         &    (0.0569)         &    (0.0991)         &     (0.121)         &    (0.0905)         &    (0.0903)         &    (0.0935)         &     (0.296)         &     (0.238)         \\
\addlinespace
Own House   &      0.0274         &      0.0707         &       0.106         &      -0.114         &       0.136         &     -0.0278         &     0.00187         &      0.0112         &      -0.246\sym{**} &      -0.140         \\
            &    (0.0831)         &    (0.0862)         &    (0.0909)         &     (0.135)         &     (0.164)         &    (0.0680)         &    (0.0698)         &    (0.0741)         &    (0.0833)         &     (0.114)         \\
\addlinespace
Live With Parents&     -0.0584         &      -0.129\sym{*}  &      -0.106\sym{*}  &      -0.340\sym{**} &      -0.224         &     -0.0278         &     -0.0263         &     -0.0480         &     -0.0708         &     -0.0967         \\
            &    (0.0607)         &    (0.0589)         &    (0.0538)         &     (0.114)         &     (0.130)         &    (0.0291)         &    (0.0267)         &    (0.0302)         &    (0.1000)         &    (0.0958)         \\
\bottomrule
\end{tabular}
}
}
\vspace{1ex} \\
\footnotesize\raggedright{Note: This table shows the estimates of the coefficient for attending Reggio Approach preschools from the OLS and difference-in-difference (DiD) models.Column title indicates the corresponding age group, control set, and model. ``None30'' refers to the OLS estimate with only the age-30 cohort and with no control variables. ``BIC30'' refers to the OLS estimate with only the age-30 cohort and with controls selected by Bayesian Information Criterion (BIC). ``Full30'' refers to the OLS estimate with only the age-30 cohort and with the full set of controls. ``DidPm30'' refers to the difference-in-difference estimate of (Reggio Muni - Parma Muni) - (Reggio None - Parma None) for the age-30 cohort. ``DidPv30'' refers to the difference-in-difference estimate of (Reggio Muni - Padova Muni) - (Reggio None - Padova None) for the age-30 cohort.  Analogous meanings are applied to the age-40 cohort in columns (7) through (12). Robust standard errors are reported in parentheses. Stars show statistical significance as follows: * $p < 0.05$, ** $p < 0.01$, *** $p < 0.001$.}
\end{table}

%Table \ref{ols-H-reg} show that the significantly increasing effects of the Reggio Approach on numbers of days sick for the past month at the time of the interview for the age-30 cohort. Moreover, the results show the significant decreasing effects on whether the respondent has ever been suspended from school. 

\begin{table}[H] \caption{OLS and Diff-in-Diff for Health and Risk, Preschools, Adult Cohorts} \label{ols-H-reg}
\scalebox{0.77}{
{
\def\sym#1{\ifmmode^{#1}\else\(^{#1}\)\fi}
\begin{tabular}{l*{10}{c}}
\toprule
            &\multicolumn{1}{c}{(1)}&\multicolumn{1}{c}{(2)}&\multicolumn{1}{c}{(3)}&\multicolumn{1}{c}{(4)}&\multicolumn{1}{c}{(5)}&\multicolumn{1}{c}{(6)}&\multicolumn{1}{c}{(7)}&\multicolumn{1}{c}{(8)}&\multicolumn{1}{c}{(9)}&\multicolumn{1}{c}{(10)}\\
            &\multicolumn{1}{c}{None30}&\multicolumn{1}{c}{BIC30}&\multicolumn{1}{c}{Full30}&\multicolumn{1}{c}{DidPm30}&\multicolumn{1}{c}{DidPv30}&\multicolumn{1}{c}{None40}&\multicolumn{1}{c}{BIC40}&\multicolumn{1}{c}{Full40}&\multicolumn{1}{c}{DidPm40}&\multicolumn{1}{c}{DidPv40}\\
\midrule
Tried Marijuana&      0.0737         &      0.0703         &      0.0480         &     0.00251         &     -0.0885         &      0.0671         &      0.0679         &      0.0950         &     -0.0202         &    -0.00850         \\
            &    (0.0569)         &    (0.0557)         &    (0.0559)         &    (0.0758)         &     (0.101)         &    (0.0487)         &    (0.0497)         &    (0.0540)         &    (0.0701)         &    (0.0933)         \\
\addlinespace
Num. of Cigarettes Per Day&       1.063         &       1.459         &       1.925         &     -0.0137         &       2.637         &      -0.300         &      -0.702         &      -0.638         &      -1.444         &       3.320         \\
            &     (1.286)         &     (1.283)         &     (1.736)         &     (2.781)         &     (3.617)         &     (1.659)         &     (1.733)         &     (1.771)         &     (2.849)         &     (2.124)         \\
\addlinespace
BMI         &       0.942\sym{*}  &       0.489         &       0.673         &      -0.301         &       1.155         &     -0.0833         &      -0.364         &      -0.220         &       0.221         &      -0.564         \\
            &     (0.477)         &     (0.429)         &     (0.438)         &     (0.600)         &     (0.847)         &     (0.530)         &     (0.516)         &     (0.501)         &     (0.716)         &     (0.941)         \\
\addlinespace
Obese       &     -0.0105         &       0.102         &      0.0982         &     -0.0323         &       0.243         &      -0.159\sym{*}  &     -0.0962         &     -0.0186         &      -0.378\sym{***}&      -0.272\sym{*}  \\
            &    (0.0739)         &    (0.0646)         &    (0.0670)         &    (0.0998)         &     (0.134)         &    (0.0722)         &    (0.0749)         &    (0.0826)         &     (0.107)         &     (0.133)         \\
\addlinespace
Overweight  &      0.0456         &     -0.0384         &    -0.00838         &     -0.0433         &     -0.0166         &      0.0615         &     -0.0271         &     -0.0695         &       0.156         &       0.148         \\
            &    (0.0664)         &    (0.0627)         &    (0.0626)         &     (0.108)         &     (0.118)         &    (0.0669)         &    (0.0658)         &    (0.0724)         &     (0.107)         &     (0.100)         \\
\addlinespace
Good Health &       0.228\sym{**} &       0.153         &       0.150         &       0.197         &       0.191         &       0.133         &       0.178         &       0.142         &       0.262         &      -0.348         \\
            &    (0.0858)         &    (0.0821)         &    (0.0857)         &     (0.142)         &     (0.170)         &    (0.0871)         &    (0.0962)         &     (0.105)         &     (0.150)         &     (0.221)         \\
\addlinespace
No Problematic Health Condition&      -0.253\sym{***}&      -0.228\sym{**} &      -0.223\sym{**} &      -0.183         &      -0.275         &      0.0238         &      0.0399         &      0.0721         &       0.106         &      -0.176         \\
            &    (0.0766)         &    (0.0808)         &    (0.0836)         &     (0.123)         &     (0.157)         &    (0.0828)         &    (0.0851)         &    (0.0970)         &     (0.132)         &     (0.145)         \\
\addlinespace
Num. of Days Sick Past Month&       0.231\sym{*}  &       0.282\sym{**} &       0.245\sym{*}  &       0.218\sym{*}  &       0.178         &      0.0237         &     0.00461         &     -0.0378         &      0.0438         &     0.00906         \\
            &    (0.0981)         &     (0.104)         &    (0.0956)         &     (0.103)         &     (0.114)         &    (0.0546)         &    (0.0617)         &    (0.0550)         &    (0.0937)         &     (0.129)         \\
\addlinespace
Ever Suspended from School&      -0.116\sym{*}  &      -0.128\sym{*}  &      -0.148\sym{**} &      -0.125         &      -0.172         &     -0.0360         &     -0.0377         &     -0.0209         &     -0.0803         &     -0.0282         \\
            &    (0.0529)         &    (0.0544)         &    (0.0523)         &    (0.0704)         &     (0.103)         &    (0.0390)         &    (0.0428)         &    (0.0569)         &    (0.0752)         &    (0.0474)         \\
\addlinespace
Age At First Drink&       1.482         &      -0.943         &      -1.276         &       0.159         &      -3.824         &       1.058         &      -0.264         &      -0.801         &      -0.199         &      -0.705         \\
            &     (1.438)         &     (1.351)         &     (1.298)         &     (1.841)         &     (2.368)         &     (1.344)         &     (1.369)         &     (1.439)         &     (1.629)         &     (2.217)         \\
\bottomrule
\end{tabular}
}
}
\vspace{1ex} \\
\footnotesize\raggedright{Note: This table shows the estimates of the coefficient for attending Reggio Approach preschools from the OLS and difference-in-difference (DiD) models.Column title indicates the corresponding age group, control set, and model. ``None30'' refers to the OLS estimate with only the age-30 cohort and with no control variables. ``BIC30'' refers to the OLS estimate with only the age-30 cohort and with controls selected by Bayesian Information Criterion (BIC). ``Full30'' refers to the OLS estimate with only the age-30 cohort and with the full set of controls. ``DidPm30'' refers to the difference-in-difference estimate of (Reggio Muni - Parma Muni) - (Reggio None - Parma None) for the age-30 cohort. ``DidPv30'' refers to the difference-in-difference estimate of (Reggio Muni - Padova Muni) - (Reggio None - Padova None) for the age-30 cohort.  Analogous meanings are applied to the age-40 cohort in columns (7) through (12). Robust standard errors are reported in parentheses. Stars show statistical significance as follows: * $p < 0.05$, ** $p < 0.01$, *** $p < 0.001$.}
\end{table}

%Table \ref{ols-N-reg} shows no consistently significant trends between the age-30 cohort and age-40 cohort. For the age-30 cohort, there are significantly decreasing effects on the optimistic look in life. Moreover, there are significantly increasing effects on whether the respondent is likely to do the same to someone who puts him/her in a difficult situation and to someone who insults him/her.

%For the age-40 cohort, results show the significantly decreasing effects on depression for the age-30 cohort. There are also significantly positive effects on satisfaction with income and with work for the age-40 cohort. However, there are significantly positive effects on whether the respondent thinks work is source of stress. 

\begin{table}[H] \caption{OLS and Diff-in-Diff for Non-cognitive, Preschools, Adult Cohorts} \label{ols-N-reg}
\scalebox{0.80}{
{
\def\sym#1{\ifmmode^{#1}\else\(^{#1}\)\fi}
\begin{tabular}{l*{10}{c}}
\toprule
            &\multicolumn{1}{c}{(1)}&\multicolumn{1}{c}{(2)}&\multicolumn{1}{c}{(3)}&\multicolumn{1}{c}{(4)}&\multicolumn{1}{c}{(5)}&\multicolumn{1}{c}{(6)}&\multicolumn{1}{c}{(7)}&\multicolumn{1}{c}{(8)}&\multicolumn{1}{c}{(9)}&\multicolumn{1}{c}{(10)}\\
            &\multicolumn{1}{c}{None30}&\multicolumn{1}{c}{BIC30}&\multicolumn{1}{c}{Full30}&\multicolumn{1}{c}{DidPm30}&\multicolumn{1}{c}{DidPv30}&\multicolumn{1}{c}{None40}&\multicolumn{1}{c}{BIC40}&\multicolumn{1}{c}{Full40}&\multicolumn{1}{c}{DidPm40}&\multicolumn{1}{c}{DidPv40}\\
\midrule
Locus of Control - positive&      0.0820         &      -0.103         &      -0.115         &      -0.512\sym{*}  &      0.0375         &       0.140         &       0.196         &       0.280         &       0.235         &       0.893\sym{***}\\
            &     (0.136)         &     (0.129)         &     (0.133)         &     (0.259)         &     (0.259)         &     (0.132)         &     (0.134)         &     (0.144)         &     (0.361)         &     (0.238)         \\
\addlinespace
Depression Score - positive&       1.572         &      -0.343         &      -0.352         &       0.405         &      -0.507         &       2.249\sym{*}  &       2.226\sym{*}  &       2.103\sym{*}  &       1.457         &       4.714\sym{**} \\
            &     (1.012)         &     (0.877)         &     (0.912)         &     (1.517)         &     (1.952)         &     (0.925)         &     (0.952)         &     (1.065)         &     (2.829)         &     (1.718)         \\
\addlinespace
Stress      &       0.116         &      0.0117         &      0.0225         &      -0.134         &      -0.183         &       0.188         &       0.203         &      0.0591         &      0.0973         &       0.704\sym{***}\\
            &     (0.121)         &     (0.107)         &     (0.113)         &     (0.220)         &     (0.205)         &     (0.113)         &     (0.120)         &     (0.137)         &     (0.276)         &     (0.195)         \\
\addlinespace
Work is Source of Stress&       0.125         &      0.0708         &      0.0521         &       0.500\sym{**} &      0.0764         &       0.167         &       0.200\sym{*}  &       0.233\sym{*}  &       0.547\sym{**} &      0.0799         \\
            &     (0.105)         &    (0.0974)         &    (0.0954)         &     (0.181)         &     (0.248)         &    (0.0908)         &    (0.0861)         &    (0.0962)         &     (0.211)         &     (0.201)         \\
\addlinespace
Satisfied with Income&       0.383\sym{*}  &       0.407\sym{**} &       0.362\sym{*}  &       0.568\sym{*}  &       0.339         &       0.268\sym{*}  &       0.282         &       0.152         &       0.298         &       0.507         \\
            &     (0.150)         &     (0.154)         &     (0.150)         &     (0.243)         &     (0.292)         &     (0.134)         &     (0.145)         &     (0.156)         &     (0.342)         &     (0.281)         \\
\addlinespace
Satisfied with Work&       0.218         &       0.210         &       0.215         &       0.351         &       0.126         &       0.308\sym{*}  &       0.287\sym{*}  &       0.165         &       0.290         &       0.212         \\
            &     (0.143)         &     (0.154)         &     (0.146)         &     (0.267)         &     (0.300)         &     (0.121)         &     (0.125)         &     (0.127)         &     (0.288)         &     (0.273)         \\
\addlinespace
Satisfied with Health&     -0.0366         &     -0.0886         &      -0.112         &      -0.152         &      -0.486\sym{**} &      0.0250         &      0.0122         &     -0.0314         &      -0.281         &     0.00532         \\
            &     (0.122)         &     (0.127)         &     (0.133)         &     (0.160)         &     (0.187)         &    (0.0784)         &    (0.0879)         &    (0.0879)         &     (0.219)         &     (0.233)         \\
\addlinespace
Satisfied with Family&      0.0368         &     -0.0421         &     -0.0250         &     -0.0682         &       0.147         &       0.192         &       0.183         &       0.177         &       0.172         &       0.217         \\
            &     (0.147)         &     (0.150)         &     (0.153)         &     (0.218)         &     (0.323)         &     (0.119)         &     (0.127)         &     (0.150)         &     (0.388)         &     (0.278)         \\
\addlinespace
Optimistic Look in Life&      -0.228\sym{**} &      -0.215\sym{*}  &      -0.195\sym{*}  &      -0.444\sym{**} &      -0.532\sym{***}&      -0.104         &     -0.0551         &      0.0609         &     -0.0532         &     -0.0836         \\
            &    (0.0815)         &    (0.0853)         &    (0.0916)         &     (0.145)         &     (0.136)         &    (0.0776)         &    (0.0794)         &    (0.0842)         &     (0.150)         &     (0.167)         \\
\addlinespace
Positive Reciprocity&     0.00411         &      -0.122         &      -0.121         &      -0.126         &      -0.339         &     -0.0160         &     -0.0117         &     -0.0489         &     -0.0512         &       0.826\sym{*}  \\
            &     (0.121)         &     (0.115)         &     (0.127)         &     (0.156)         &     (0.250)         &     (0.112)         &     (0.112)         &     (0.123)         &     (0.167)         &     (0.322)         \\
\addlinespace
Negative Reciprocity&       0.664\sym{***}&       0.759\sym{***}&       0.786\sym{***}&       1.308\sym{***}&       0.880\sym{**} &      0.0153         &     -0.0389         &     -0.0490         &      -0.163         &      -0.526         \\
            &     (0.165)         &     (0.159)         &     (0.160)         &     (0.277)         &     (0.333)         &     (0.148)         &     (0.152)         &     (0.173)         &     (0.387)         &     (0.304)         \\
\bottomrule
\end{tabular}
}
}
\vspace{1ex} \\
\footnotesize\raggedright{Note: This table shows the estimates of the coefficient for attending Reggio Approach preschools from the OLS and difference-in-difference (DiD) models.Column title indicates the corresponding age group, control set, and model. ``None30'' refers to the OLS estimate with only the age-30 cohort and with no control variables. ``BIC30'' refers to the OLS estimate with only the age-30 cohort and with controls selected by Bayesian Information Criterion (BIC). ``Full30'' refers to the OLS estimate with only the age-30 cohort and with the full set of controls. ``DidPm30'' refers to the difference-in-difference estimate of (Reggio Muni - Parma Muni) - (Reggio None - Parma None) for the age-30 cohort. ``DidPv30'' refers to the difference-in-difference estimate of (Reggio Muni - Padova Muni) - (Reggio None - Padova None) for the age-30 cohort.  Analogous meanings are applied to the age-40 cohort in columns (7) through (12). Robust standard errors are reported in parentheses. Stars show statistical significance as follows: * $p < 0.05$, ** $p < 0.01$, *** $p < 0.001$.}
\end{table}

%Table \ref{ols-S-reg} shows consistently significant and negative effects of the Reggio Approach on whether the respondent does volunteer work for both the age-30 and age-40 cohorts. Moreover, for the age-40 cohort, there are significantly negative effects on respondents having migrant friends and significantly positive effects on whether respondents have ever voted for municipal and regional elections. For the age-30 cohort, there are significantly negative effects on whether respondents have ever voted for national elections.

\begin{table}[H] \caption{OLS and Diff-in-Diff for Social Behavior, Preschools, Adult Cohorts} \label{ols-S-reg}
\scalebox{0.80}{
{
\def\sym#1{\ifmmode^{#1}\else\(^{#1}\)\fi}
\begin{tabular}{l*{12}{c}}
\toprule
            &\multicolumn{1}{c}{(1)}&\multicolumn{1}{c}{(2)}&\multicolumn{1}{c}{(3)}&\multicolumn{1}{c}{(4)}&\multicolumn{1}{c}{(5)}&\multicolumn{1}{c}{(6)}&\multicolumn{1}{c}{(7)}&\multicolumn{1}{c}{(8)}&\multicolumn{1}{c}{(9)}&\multicolumn{1}{c}{(10)}&\multicolumn{1}{c}{(11)}&\multicolumn{1}{c}{(12)}\\
            &\multicolumn{1}{c}{None30}&\multicolumn{1}{c}{BIC30}&\multicolumn{1}{c}{Full30}&\multicolumn{1}{c}{DidPm30}&\multicolumn{1}{c}{DidPv30}&\multicolumn{1}{c}{IPW30}&\multicolumn{1}{c}{None40}&\multicolumn{1}{c}{BIC40}&\multicolumn{1}{c}{Full40}&\multicolumn{1}{c}{DidPm40}&\multicolumn{1}{c}{DidPv40}&\multicolumn{1}{c}{IPW40}\\
\midrule
Favorable to Migrants&     -0.0807         &     -0.0548         &     -0.0857         &     -0.0223         &     0.00769         &       0.192\sym{**} &     -0.0612         &     -0.0127         &   -0.000380         &       0.363\sym{*}  &       0.456\sym{*}  &       0.125         \\
            &    (0.0862)         &    (0.0897)         &    (0.0961)         &     (0.177)         &     (0.229)         &    (0.0697)         &    (0.0907)         &    (0.0922)         &     (0.112)         &     (0.177)         &     (0.228)         &    (0.0855)         \\
\addlinespace
Num. of Friends&      -0.475         &      -0.444         &      -0.429         &       0.821         &       3.555         &      -1.294         &      -0.951         &      -0.519         &       0.813         &       3.139\sym{*}  &       3.903\sym{*}  &       0.755         \\
            &     (1.488)         &     (1.722)         &     (1.960)         &     (2.163)         &     (2.754)         &     (1.024)         &     (1.018)         &     (0.937)         &     (1.491)         &     (1.392)         &     (1.603)         &     (1.500)         \\
\addlinespace
Has Migrant Friends&       0.105         &       0.116         &      0.0808         &       0.118         &      0.0470         &      0.0169         &      -0.120         &     -0.0988         &      -0.114         &      0.0513         &     -0.0608         &       0.332\sym{***}\\
            &    (0.0749)         &    (0.0801)         &    (0.0831)         &     (0.120)         &     (0.148)         &    (0.0540)         &    (0.0655)         &    (0.0677)         &    (0.0779)         &     (0.109)         &     (0.129)         &    (0.0698)         \\
\addlinespace
Volunteers  &      -0.130\sym{*}  &      -0.121\sym{*}  &      -0.102\sym{*}  &      -0.199         &      -0.314\sym{**} &      -0.131\sym{**} &      -0.122\sym{*}  &     -0.0802         &      -0.110         &    -0.00799         &      -0.113         &      0.0866         \\
            &    (0.0583)         &    (0.0496)         &    (0.0511)         &     (0.109)         &     (0.110)         &    (0.0398)         &    (0.0563)         &    (0.0543)         &    (0.0712)         &    (0.0942)         &    (0.0884)         &    (0.0659)         \\
\addlinespace
Child Eats Meal with Fam&       0.405         &      -0.526         &      -0.500\sym{***}&    -0.00829         &      -0.844\sym{**} &      -0.116         &     0.00361         &     -0.0127         &      0.0325         &      -0.553\sym{*}  &     -0.0976         &      -0.135         \\
            &     (0.262)         &     (0.399)         &  (8.07e-14)         &     (0.817)         &     (0.306)         &     (0.276)         &     (0.155)         &     (0.159)         &     (0.190)         &     (0.245)         &     (0.259)         &     (0.135)         \\
\addlinespace
Ever Voted for Municipal&       0.196\sym{*}  &      0.0719         &      0.0712         &     -0.0233         &     -0.0122         &      0.0396         &       0.228\sym{**} &       0.134         &       0.119         &    -0.00155         &     -0.0917         &      0.0686         \\
            &    (0.0838)         &    (0.0702)         &    (0.0683)         &    (0.0896)         &     (0.138)         &    (0.0461)         &    (0.0781)         &    (0.0773)         &    (0.0786)         &     (0.111)         &     (0.131)         &    (0.0640)         \\
\addlinespace
Ever Voted for Regional&       0.141         &      0.0214         &      0.0289         &     -0.0428         &    0.000570         &      0.0488         &       0.232\sym{**} &       0.147         &       0.135         &      0.0879         &     -0.0137         &       0.164\sym{*}  \\
            &    (0.0847)         &    (0.0738)         &    (0.0717)         &    (0.0921)         &     (0.142)         &    (0.0482)         &    (0.0782)         &    (0.0769)         &    (0.0821)         &     (0.104)         &     (0.141)         &    (0.0661)         \\
\addlinespace
Ever Voted for National&      -0.111         &     -0.0866         &     -0.0828         &      -0.207\sym{*}  &       0.140         &     -0.0627         &      0.0965         &      0.0615         &      0.0728         &      0.0212         &       0.229\sym{*}  &      0.0244         \\
            &    (0.0571)         &    (0.0541)         &    (0.0568)         &     (0.102)         &     (0.120)         &    (0.0372)         &    (0.0607)         &    (0.0643)         &    (0.0729)         &    (0.0887)         &    (0.0980)         &    (0.0463)         \\
\bottomrule
\end{tabular}
}
}
\vspace{1ex} \\
\footnotesize\raggedright{Note: This table shows the estimates of the coefficient for attending Reggio Approach preschools from the OLS and difference-in-difference (DiD) models.Column title indicates the corresponding age group, control set, and model. ``None30'' refers to the OLS estimate with only the age-30 cohort and with no control variables. ``BIC30'' refers to the OLS estimate with only the age-30 cohort and with controls selected by Bayesian Information Criterion (BIC). ``Full30'' refers to the OLS estimate with only the age-30 cohort and with the full set of controls. ``DidPm30'' refers to the difference-in-difference estimate of (Reggio Muni - Parma Muni) - (Reggio None - Parma None) for the age-30 cohort. ``DidPv30'' refers to the difference-in-difference estimate of (Reggio Muni - Padova Muni) - (Reggio None - Padova None) for the age-30 cohort.  Analogous meanings are applied to the age-40 cohort in columns (7) through (12). Robust standard errors are reported in parentheses. Stars show statistical significance as follows: * $p < 0.05$, ** $p < 0.01$, *** $p < 0.001$.}
\end{table}









% ==============================================================================%
\subsection{Municipal vs. State}\label{appendix:state}
% ==============================================================================% 
\subsubsection{Children}
\begin{table}[H] \caption{OLS and Diff-in-Diff for Cognitive and Noncognitive, Preschools, Children Cohort} \label{ols-E-reg}
\scalebox{0.80}{{
\def\sym#1{\ifmmode^{#1}\else\(^{#1}\)\fi}
\begin{tabular}{l*{12}{c}}
\toprule
            &\multicolumn{1}{c}{(1)}&\multicolumn{1}{c}{(2)}&\multicolumn{1}{c}{(3)}&\multicolumn{1}{c}{(4)}&\multicolumn{1}{c}{(5)}&\multicolumn{1}{c}{(6)}&\multicolumn{1}{c}{(7)}&\multicolumn{1}{c}{(8)}&\multicolumn{1}{c}{(9)}&\multicolumn{1}{c}{(10)}&\multicolumn{1}{c}{(11)}&\multicolumn{1}{c}{(12)}\\
            &\multicolumn{1}{c}{NoneIt}&\multicolumn{1}{c}{BICIt}&\multicolumn{1}{c}{FullIt}&\multicolumn{1}{c}{DidPmIt}&\multicolumn{1}{c}{DidPvIt}&\multicolumn{1}{c}{PSMIt}&\multicolumn{1}{c}{NoneMg}&\multicolumn{1}{c}{BICMg}&\multicolumn{1}{c}{FullMg}&\multicolumn{1}{c}{DidPmMg}&\multicolumn{1}{c}{DidPvMg}&\multicolumn{1}{c}{PSMMg}\\
\midrule
IQ Factor   &       0.246         &       0.208         &      0.0804         &       0.465\sym{*}  &       0.101         &      -0.294\sym{**} &       0.236         &       0.287         &       0.320         &       0.846\sym{**} &       0.493         &      -0.285         \\
            &     (0.192)         &     (0.179)         &     (0.173)         &     (0.205)         &     (0.291)         &    (0.0911)         &     (0.230)         &     (0.242)         &     (0.248)         &     (0.295)         &     (0.301)         &     (0.218)         \\
\addlinespace
IQ Score    &      0.0474         &      0.0509         &      0.0228         &       0.106\sym{*}  &     0.00805         &     -0.0823\sym{***}&      0.0450         &      0.0576         &      0.0700         &       0.188\sym{**} &      0.0983         &     -0.0705         \\
            &    (0.0436)         &    (0.0422)         &    (0.0416)         &    (0.0491)         &    (0.0647)         &    (0.0218)         &    (0.0525)         &    (0.0530)         &    (0.0566)         &    (0.0711)         &    (0.0678)         &    (0.0519)         \\
\addlinespace
SDQ Composite - Child&       1.916         &       1.750         &       1.957\sym{*}  &       2.263\sym{*}  &       2.157         &       0.195         &      -0.910         &      -1.009         &      -0.455         &      -0.515         &      -1.591         &       1.790\sym{*}  \\
            &     (1.014)         &     (0.900)         &     (0.909)         &     (1.142)         &     (1.241)         &     (0.412)         &     (0.935)         &     (0.928)         &     (0.949)         &     (1.862)         &     (1.512)         &     (0.741)         \\
\addlinespace
SDQ Pro-social - Child&      -0.142         &      0.0693         &     -0.0383         &      -0.256         &       0.665         &     0.00412         &       0.373         &       0.545         &       0.300         &       0.375         &       0.593         &       0.514         \\
            &     (0.295)         &     (0.274)         &     (0.322)         &     (0.416)         &     (0.487)         &     (0.174)         &     (0.371)         &     (0.364)         &     (0.394)         &     (0.790)         &     (0.483)         &     (0.312)         \\
\addlinespace
SDQ Peer problems - Child&       0.308         &       0.192         &       0.286         &       0.216         &       0.214         &       0.196         &     -0.0710         &   -0.000729         &       0.195         &       0.904         &      -0.241         &       0.403         \\
            &     (0.282)         &     (0.257)         &     (0.266)         &     (0.336)         &     (0.363)         &     (0.118)         &     (0.291)         &     (0.311)         &     (0.347)         &     (0.474)         &     (0.427)         &     (0.263)         \\
\addlinespace
SDQ Hyper - Child&       0.722         &       0.804\sym{*}  &       0.791\sym{*}  &       1.084\sym{*}  &       0.438         &      -0.107         &      -0.317         &      -0.450         &      -0.136         &      -0.378         &      -0.756         &      0.0982         \\
            &     (0.420)         &     (0.397)         &     (0.393)         &     (0.551)         &     (0.546)         &     (0.217)         &     (0.481)         &     (0.462)         &     (0.487)         &     (0.958)         &     (0.673)         &     (0.351)         \\
\addlinespace
SDQ Emotional - Child&       0.556         &       0.553         &       0.490         &       0.603         &       0.952\sym{*}  &      0.0565         &      -0.417         &      -0.381         &      -0.457         &      -0.807         &      -0.421         &       0.659\sym{*}  \\
            &     (0.346)         &     (0.337)         &     (0.374)         &     (0.415)         &     (0.468)         &     (0.147)         &     (0.286)         &     (0.291)         &     (0.322)         &     (0.726)         &     (0.549)         &     (0.295)         \\
\addlinespace
SDQ Conduct - Child&       0.330         &       0.202         &       0.391         &       0.361         &       0.552         &      0.0490         &      -0.105         &      -0.178         &     -0.0574         &      -0.234         &      -0.173         &       0.630\sym{*}  \\
            &     (0.299)         &     (0.268)         &     (0.292)         &     (0.352)         &     (0.370)         &     (0.131)         &     (0.273)         &     (0.290)         &     (0.320)         &     (0.654)         &     (0.461)         &     (0.265)         \\
\bottomrule
\end{tabular}
}
}
\vspace{1ex} \\
\footnotesize\raggedright{Note: This table shows the estimates of the coefficient for attending Reggio Approach preschools from the OLS and difference-in-difference (DiD) models.Column title indicates the corresponding age group, control set, and model. ``NoneIt'' refers to the OLS estimate with only the Italian children cohort and with no control variables. ``BICIt'' refers to the OLS estimate with only the Italian children cohort and with controls selected by Bayesian Information Criterion (BIC) and additional controls for caregiver's religion. ``FullIt'' refers to the OLS estimate with only the Italian children cohort and with the full set of controls. ``DidPmIt'' refers to the difference-in-difference estimate of (Reggio Muni - Parma Muni) - (Reggio None - Parma None) for the Italian children cohort. ``DidPvIt'' refers to the difference-in-difference estimate of (Reggio Muni - Padova Muni) - (Reggio None - Padova None) for the Italian children cohort.  Analogous meanings are applied to the migrant children cohort in columns (7) through (12). Robust standard errors are reported in parentheses. Stars show statistical significance as follows: * $p < 0.05$, ** $p < 0.01$, *** $p < 0.001$.}
\end{table}

%Table \ref{ols-W-reg} shows consistently significant positive effects of the Reggio Approach on hours worked per week. Moreover, there are significant effects on the indicator for the lowest household income category for the age-40 cohort. Note that the measures for subject labor income, including ``Monthly Wage'' presented in the table, have very few observations. Household income suffers less from the missing value problem.

\begin{table}[H] \caption{OLS and Diff-in-Diff for Health and Risk, Preschools, Children Cohort} \label{ols-W-reg}
\scalebox{0.76}{
{
\def\sym#1{\ifmmode^{#1}\else\(^{#1}\)\fi}
\begin{tabular}{l*{10}{c}}
\toprule
            &\multicolumn{1}{c}{(1)}&\multicolumn{1}{c}{(2)}&\multicolumn{1}{c}{(3)}&\multicolumn{1}{c}{(4)}&\multicolumn{1}{c}{(5)}&\multicolumn{1}{c}{(6)}&\multicolumn{1}{c}{(7)}&\multicolumn{1}{c}{(8)}&\multicolumn{1}{c}{(9)}&\multicolumn{1}{c}{(10)}\\
            &\multicolumn{1}{c}{NoneIt}&\multicolumn{1}{c}{BICIt}&\multicolumn{1}{c}{FullIt}&\multicolumn{1}{c}{DidPmIt}&\multicolumn{1}{c}{DidPvIt}&\multicolumn{1}{c}{NoneMg}&\multicolumn{1}{c}{BICMg}&\multicolumn{1}{c}{FullMg}&\multicolumn{1}{c}{DidPmMg}&\multicolumn{1}{c}{DidPvMg}\\
\midrule
Obese       &     -0.0778         &     -0.0484         &      0.0128         &      -0.141         &     -0.0201         &     -0.0783         &     -0.0565         &     0.00156         &       0.337         &     -0.0683         \\
            &    (0.0833)         &    (0.0828)         &    (0.0770)         &     (0.112)         &     (0.133)         &     (0.108)         &     (0.107)         &     (0.108)         &     (0.251)         &     (0.159)         \\
\addlinespace
Overweight  &      0.0466         &      0.0312         &      0.0246         &     -0.0326         &     0.00238         &       0.106         &      0.0888         &      0.0744         &      -0.249         &       0.296\sym{*}  \\
            &    (0.0514)         &    (0.0494)         &    (0.0613)         &    (0.0988)         &    (0.0606)         &    (0.0729)         &    (0.0740)         &    (0.0863)         &     (0.190)         &     (0.125)         \\
\addlinespace
Health is Good&     -0.0295         &   -0.000495         &      0.0152         &       0.130         &      0.0328         &      -0.123         &     -0.0842         &      -0.119         &      -0.233         &     -0.0597         \\
            &    (0.0785)         &    (0.0809)         &    (0.0867)         &     (0.133)         &     (0.116)         &    (0.0940)         &     (0.103)         &     (0.108)         &     (0.287)         &     (0.121)         \\
\addlinespace
Number of Sick Days&       0.232         &       0.224         &       0.160         &       0.315         &       0.259         &       0.213         &       0.166         &       0.278         &       0.228         &      0.0862         \\
            &     (0.126)         &     (0.118)         &     (0.133)         &     (0.183)         &     (0.203)         &     (0.146)         &     (0.156)         &     (0.176)         &     (0.349)         &     (0.225)         \\
\bottomrule
\end{tabular}
}
}
\vspace{1ex} \\
\footnotesize\raggedright{Note: This table shows the estimates of the coefficient for attending Reggio Approach preschools from the OLS and difference-in-difference (DiD) models.Column title indicates the corresponding age group, control set, and model. ``NoneIt'' refers to the OLS estimate with only the Italian children cohort and with no control variables. ``BICIt'' refers to the OLS estimate with only the Italian children cohort and with controls selected by Bayesian Information Criterion (BIC) and additional controls for caregiver's religion. ``FullIt'' refers to the OLS estimate with only the Italian children cohort and with the full set of controls. ``DidPmIt'' refers to the difference-in-difference estimate of (Reggio Muni - Parma Muni) - (Reggio None - Parma None) for the Italian children cohort. ``DidPvIt'' refers to the difference-in-difference estimate of (Reggio Muni - Padova Muni) - (Reggio None - Padova None) for the Italian children cohort.  Analogous meanings are applied to the migrant children cohort in columns (7) through (12). Robust standard errors are reported in parentheses. Stars show statistical significance as follows: * $p < 0.05$, ** $p < 0.01$, *** $p < 0.001$.}
\end{table}

%Table \ref{ols-L-reg} shows significant effects of the Reggio Approach on the status of being divorced for the age-30 cohort. Other results do not show any significant trend. 

\begin{table}[H] \caption{OLS and Diff-in-Diff for Behaviors, Preschools, Children Cohort} \label{ols-L-reg}
\scalebox{0.80}{
{
\def\sym#1{\ifmmode^{#1}\else\(^{#1}\)\fi}
\begin{tabular}{l*{10}{c}}
\toprule
            &\multicolumn{1}{c}{(1)}&\multicolumn{1}{c}{(2)}&\multicolumn{1}{c}{(3)}&\multicolumn{1}{c}{(4)}&\multicolumn{1}{c}{(5)}&\multicolumn{1}{c}{(6)}&\multicolumn{1}{c}{(7)}&\multicolumn{1}{c}{(8)}&\multicolumn{1}{c}{(9)}&\multicolumn{1}{c}{(10)}\\
            &\multicolumn{1}{c}{NoneIt}&\multicolumn{1}{c}{BICIt}&\multicolumn{1}{c}{FullIt}&\multicolumn{1}{c}{DidPmIt}&\multicolumn{1}{c}{DidPvIt}&\multicolumn{1}{c}{NoneMg}&\multicolumn{1}{c}{BICMg}&\multicolumn{1}{c}{FullMg}&\multicolumn{1}{c}{DidPmMg}&\multicolumn{1}{c}{DidPvMg}\\
\midrule
Not Excited to Learn&     -0.0441         &     -0.0329         &     -0.0359         &     -0.0249         &     -0.0779         &     -0.0101         &    -0.00118         &     -0.0119         &      -0.155         &     -0.0270         \\
            &    (0.0400)         &    (0.0314)         &    (0.0333)         &    (0.0494)         &    (0.0583)         &    (0.0666)         &    (0.0648)         &    (0.0798)         &    (0.0931)         &    (0.0825)         \\
\addlinespace
Problems Sitting Still&      0.0416         &      0.0173         &      0.0567         &     -0.0180         &      -0.151\sym{*}  &      0.0562         &      0.0542         &      0.0403         &     -0.0662         &     0.00501         \\
            &    (0.0509)         &    (0.0546)         &    (0.0588)         &    (0.0745)         &    (0.0749)         &    (0.0666)         &    (0.0735)         &    (0.0961)         &    (0.0782)         &    (0.0973)         \\
\addlinespace
How Much Child Likes School&       0.169         &       0.174         &       0.120         &       0.216         &       0.383\sym{*}  &      -0.308\sym{*}  &      -0.305\sym{*}  &      -0.273\sym{*}  &      -0.141         &      -0.335\sym{*}  \\
            &     (0.112)         &     (0.111)         &     (0.118)         &     (0.147)         &     (0.174)         &     (0.125)         &     (0.128)         &     (0.136)         &     (0.195)         &     (0.164)         \\
\addlinespace
Happy in General&      -0.107         &     -0.0551         &      0.0105         &       0.205         &      -0.210         &       0.141         &       0.187         &       0.338         &       0.694         &       0.745         \\
            &     (0.301)         &     (0.307)         &     (0.327)         &     (0.378)         &     (0.427)         &     (0.409)         &     (0.425)         &     (0.499)         &     (0.677)         &     (0.509)         \\
\bottomrule
\end{tabular}
}
}
\vspace{1ex} \\
\footnotesize\raggedright{Note: This table shows the estimates of the coefficient for attending Reggio Approach preschools from the OLS and difference-in-difference (DiD) models.Column title indicates the corresponding age group, control set, and model. ``NoneIt'' refers to the OLS estimate with only the Italian children cohort and with no control variables. ``BICIt'' refers to the OLS estimate with only the Italian children cohort and with controls selected by Bayesian Information Criterion (BIC) and additional controls for caregiver's religion. ``FullIt'' refers to the OLS estimate with only the Italian children cohort and with the full set of controls. ``DidPmIt'' refers to the difference-in-difference estimate of (Reggio Muni - Parma Muni) - (Reggio None - Parma None) for the Italian children cohort. ``DidPvIt'' refers to the difference-in-difference estimate of (Reggio Muni - Padova Muni) - (Reggio None - Padova None) for the Italian children cohort.  Analogous meanings are applied to the migrant children cohort in columns (7) through (12). Robust standard errors are reported in parentheses. Stars show statistical significance as follows: * $p < 0.05$, ** $p < 0.01$, *** $p < 0.001$.}
\end{table}

%Table \ref{ols-H-reg} show that the significantly increasing effects of the Reggio Approach on numbers of days sick for the past month at the time of the interview for the age-30 cohort. Moreover, the results show the significant decreasing effects on whether the respondent has ever been suspended from school. 

\begin{table}[H] \caption{OLS and Diff-in-Diff for Social Behaviors, Preschools, Children} \label{ols-H-reg}
\scalebox{0.77}{
{
\def\sym#1{\ifmmode^{#1}\else\(^{#1}\)\fi}
\begin{tabular}{l*{10}{c}}
\toprule
            &\multicolumn{1}{c}{(1)}&\multicolumn{1}{c}{(2)}&\multicolumn{1}{c}{(3)}&\multicolumn{1}{c}{(4)}&\multicolumn{1}{c}{(5)}&\multicolumn{1}{c}{(6)}&\multicolumn{1}{c}{(7)}&\multicolumn{1}{c}{(8)}&\multicolumn{1}{c}{(9)}&\multicolumn{1}{c}{(10)}\\
            &\multicolumn{1}{c}{NoneIt}&\multicolumn{1}{c}{BICIt}&\multicolumn{1}{c}{FullIt}&\multicolumn{1}{c}{DidPmIt}&\multicolumn{1}{c}{DidPvIt}&\multicolumn{1}{c}{NoneMg}&\multicolumn{1}{c}{BICMg}&\multicolumn{1}{c}{FullMg}&\multicolumn{1}{c}{DidPmMg}&\multicolumn{1}{c}{DidPvMg}\\
\midrule
Musical Instrument at Home&       0.105         &      0.0800         &      0.0314         &      -0.119         &       0.147         &    -0.00169         &     -0.0609         &     -0.0812         &     -0.0849         &     -0.0586         \\
            &    (0.0830)         &    (0.0841)         &    (0.0932)         &     (0.135)         &     (0.130)         &    (0.0866)         &    (0.0919)         &     (0.102)         &     (0.246)         &     (0.118)         \\
\addlinespace
Tell Worry at Home&      0.0239         &     -0.0112         &     -0.0828         &     -0.0156         &      0.0999         &       0.178         &       0.121         &       0.160         &    0.000318         &       0.285\sym{*}  \\
            &    (0.0831)         &    (0.0789)         &    (0.0824)         &     (0.130)         &     (0.126)         &     (0.109)         &     (0.112)         &     (0.117)         &     (0.257)         &     (0.145)         \\
\addlinespace
Tell Worry to Teacher&       0.178\sym{**} &       0.181\sym{**} &       0.153\sym{*}  &       0.138         &       0.131         &      0.0191         &     -0.0364         &      0.0295         &     -0.0297         &     -0.0717         \\
            &    (0.0669)         &    (0.0697)         &    (0.0714)         &     (0.114)         &    (0.0972)         &    (0.0881)         &    (0.0896)         &    (0.0962)         &     (0.258)         &     (0.147)         \\
\addlinespace
Tell Worry to Friends&     -0.0398         &     -0.0409         &     -0.0248         &      0.0786         &      -0.119         &      0.0107         &      0.0237         &      0.0612         &     -0.0407         &     -0.0597         \\
            &    (0.0707)         &    (0.0723)         &    (0.0734)         &     (0.101)         &     (0.109)         &    (0.0768)         &    (0.0851)         &    (0.0799)         &     (0.177)         &     (0.124)         \\
\addlinespace
Keep Worry to Myself&     -0.0568         &     -0.0355         &     -0.0163         &      0.0305         &     -0.0539         &      -0.143         &     -0.0712         &      -0.168         &      0.0306         &      -0.200         \\
            &    (0.0641)         &    (0.0561)         &    (0.0589)         &     (0.100)         &    (0.0917)         &    (0.0991)         &     (0.104)         &     (0.107)         &     (0.177)         &     (0.126)         \\
\bottomrule
\end{tabular}
}
}
\vspace{1ex} \\
\footnotesize\raggedright{Note: This table shows the estimates of the coefficient for attending Reggio Approach preschools from the OLS and difference-in-difference (DiD) models.Column title indicates the corresponding age group, control set, and model. ``NoneIt'' refers to the OLS estimate with only the Italian children cohort and with no control variables. ``BICIt'' refers to the OLS estimate with only the Italian children cohort and with controls selected by Bayesian Information Criterion (BIC) and additional controls for caregiver's religion. ``FullIt'' refers to the OLS estimate with only the Italian children cohort and with the full set of controls. ``DidPmIt'' refers to the difference-in-difference estimate of (Reggio Muni - Parma Muni) - (Reggio None - Parma None) for the Italian children cohort. ``DidPvIt'' refers to the difference-in-difference estimate of (Reggio Muni - Padova Muni) - (Reggio None - Padova None) for the Italian children cohort.  Analogous meanings are applied to the migrant children cohort in columns (7) through (12). Robust standard errors are reported in parentheses. Stars show statistical significance as follows: * $p < 0.05$, ** $p < 0.01$, *** $p < 0.001$.}
\end{table}
\end{landscape}


\subsubsection{Adolescents}

\begin{table}[H] \caption{OLS and Diff-in-Diff for Cognitive and Noncognitive, Preschools, Adolescent Cohort} \label{ols-E-reg}
\scalebox{0.72}{{
\def\sym#1{\ifmmode^{#1}\else\(^{#1}\)\fi}
\begin{tabular}{l*{6}{c}}
\toprule
            &\multicolumn{1}{c}{(1)}&\multicolumn{1}{c}{(2)}&\multicolumn{1}{c}{(3)}&\multicolumn{1}{c}{(4)}&\multicolumn{1}{c}{(5)}&\multicolumn{1}{c}{(6)}\\
            &\multicolumn{1}{c}{None}&\multicolumn{1}{c}{BIC}&\multicolumn{1}{c}{Full}&\multicolumn{1}{c}{DidPm}&\multicolumn{1}{c}{DidPv}&\multicolumn{1}{c}{IPW}\\
\midrule
IQ Factor   &      0.0893         &      0.0734         &       0.126         &     -0.0627         &      -0.629         &       0.288\sym{**} \\
            &     (0.228)         &     (0.257)         &     (0.277)         &     (0.273)         &     (0.330)         &    (0.0990)         \\
\addlinespace
IQ Score    &      0.0322         &      0.0209         &      0.0393         &     -0.0536         &      -0.161         &      0.0861\sym{**} \\
            &    (0.0632)         &    (0.0703)         &    (0.0759)         &    (0.0782)         &    (0.0919)         &    (0.0288)         \\
\addlinespace
SDQ Composite - Child&      -1.866\sym{**} &      -1.511\sym{*}  &      -1.599\sym{*}  &      -1.178         &      -2.544\sym{*}  &       0.196         \\
            &     (0.614)         &     (0.696)         &     (0.753)         &     (1.003)         &     (1.023)         &     (0.537)         \\
\addlinespace
SDQ Pro-social - Child&     -0.0505         &      -0.274         &      -0.367         &      -0.574         &      0.0818         &      -0.128         \\
            &     (0.410)         &     (0.408)         &     (0.372)         &     (0.504)         &     (0.517)         &     (0.191)         \\
\addlinespace
SDQ Peer problems - Child&      -0.520         &      -0.447         &      -0.412         &      -1.187\sym{**} &      -0.774         &     -0.0886         \\
            &     (0.270)         &     (0.294)         &     (0.264)         &     (0.373)         &     (0.407)         &     (0.152)         \\
\addlinespace
SDQ Hyper - Child&      -0.275         &      -0.300         &      -0.363         &      -0.409         &      -1.010\sym{*}  &       0.412         \\
            &     (0.320)         &     (0.337)         &     (0.358)         &     (0.470)         &     (0.458)         &     (0.254)         \\
\addlinespace
SDQ Emotional - Child&      -0.738\sym{*}  &      -0.552         &      -0.625         &     -0.0257         &      -0.654         &      -0.196         \\
            &     (0.327)         &     (0.365)         &     (0.394)         &     (0.509)         &     (0.456)         &     (0.201)         \\
\addlinespace
SDQ Conduct - Child&      -0.333         &      -0.212         &      -0.199         &       0.443         &      -0.119         &      0.0642         \\
            &     (0.276)         &     (0.308)         &     (0.329)         &     (0.361)         &     (0.419)         &     (0.165)         \\
\addlinespace
SDQ Composite&      -2.607\sym{**} &      -2.475\sym{*}  &      -2.807\sym{**} &      -1.052         &      -1.688         &      -0.160         \\
            &     (0.963)         &     (1.047)         &     (1.004)         &     (1.329)         &     (1.350)         &     (0.668)         \\
\addlinespace
SDQ Pro-social&     -0.0273         &      -0.200         &      -0.203         &     0.00359         &      -0.764         &      -0.332         \\
            &     (0.369)         &     (0.352)         &     (0.363)         &     (0.468)         &     (0.470)         &     (0.200)         \\
\addlinespace
SDQ Peer problems&      -0.480         &      -0.428         &      -0.528         &      -0.843\sym{*}  &      -0.208         &       0.151         \\
            &     (0.258)         &     (0.273)         &     (0.285)         &     (0.388)         &     (0.410)         &     (0.186)         \\
\addlinespace
SDQ Hyper   &      -1.086\sym{*}  &      -1.043\sym{*}  &      -1.157\sym{*}  &      -0.717         &      -0.962         &     -0.0817         \\
            &     (0.438)         &     (0.472)         &     (0.462)         &     (0.585)         &     (0.544)         &     (0.245)         \\
\addlinespace
SDQ Emotional&      -0.384         &      -0.335         &      -0.411         &       0.634         &      0.0297         &      -0.335         \\
            &     (0.467)         &     (0.421)         &     (0.452)         &     (0.553)         &     (0.553)         &     (0.257)         \\
\addlinespace
SDQ Conduct &      -0.656\sym{**} &      -0.669\sym{*}  &      -0.710\sym{*}  &      -0.127         &      -0.547         &       0.106         \\
            &     (0.249)         &     (0.284)         &     (0.293)         &     (0.366)         &     (0.406)         &     (0.177)         \\
\addlinespace
Depression Score - positive&      -1.838         &      -1.610         &      -1.724         &       0.254         &       1.062         &      -1.347\sym{*}  \\
            &     (1.367)         &     (1.436)         &     (1.511)         &     (1.638)         &     (1.804)         &     (0.685)         \\
\bottomrule
\end{tabular}
}
}
\vspace{1ex} \\
\footnotesize\raggedright{Note: This table shows the estimates of the coefficient for attending Reggio Approach preschools from the OLS and difference-in-difference (DiD) models.Column title indicates the corresponding age group, control set, and model. ``None'' refers to the OLS estimate with only the Italian children cohort and with no control variables. ``BIC'' refers to the OLS estimate with only the Italian children cohort and with controls selected by Bayesian Information Criterion (BIC) and additional controls for caregiver's religion. ``Full'' refers to the OLS estimate with only the Italian children cohort and with the full set of controls. ``DidPm'' refers to the difference-in-difference estimate of (Reggio Muni - Parma Muni) - (Reggio None - Parma None) for the Italian children cohort. ``DidPv'' refers to the difference-in-difference estimate of (Reggio Muni - Padova Muni) - (Reggio None - Padova None) for the Italian children cohort.  Robust standard errors are reported in parentheses. Stars show statistical significance as follows: * $p < 0.05$, ** $p < 0.01$, *** $p < 0.001$.}
\end{table}

%Table \ref{ols-W-reg} shows consistently significant positive effects of the Reggio Approach on hours worked per week. Moreover, there are significant effects on the indicator for the lowest household income category for the age-40 cohort. Note that the measures for subject labor income, including ``Monthly Wage'' presented in the table, have very few observations. Household income suffers less from the missing value problem.


\begin{table}[H] \caption{OLS and Diff-in-Diff for Health and Risk, Preschools, Adolescent Cohort} \label{ols-W-reg}
\scalebox{0.76}{
{
\def\sym#1{\ifmmode^{#1}\else\(^{#1}\)\fi}
\begin{tabular}{l*{6}{c}}
\toprule
            &\multicolumn{1}{c}{(1)}&\multicolumn{1}{c}{(2)}&\multicolumn{1}{c}{(3)}&\multicolumn{1}{c}{(4)}&\multicolumn{1}{c}{(5)}&\multicolumn{1}{c}{(6)}\\
            &\multicolumn{1}{c}{None}&\multicolumn{1}{c}{BIC}&\multicolumn{1}{c}{Full}&\multicolumn{1}{c}{DidPm}&\multicolumn{1}{c}{DidPv}&\multicolumn{1}{c}{PSM}\\
\midrule
Obese       &      -0.109         &     -0.0511         &     -0.0645         &      -0.136         &       0.168         &     -0.0717         \\
            &     (0.104)         &     (0.103)         &    (0.0967)         &     (0.111)         &     (0.132)         &    (0.0413)         \\
\addlinespace
Overweight  &     -0.0110         &     0.00207         &      0.0122         &     -0.0527         &      0.0107         &    -0.00509         \\
            &    (0.0490)         &    (0.0450)         &    (0.0471)         &    (0.0654)         &    (0.0553)         &    (0.0231)         \\
\addlinespace
Health is Good&      0.0761         &       0.111         &       0.121         &     0.00223         &       0.329\sym{*}  &       0.146\sym{**} \\
            &     (0.113)         &     (0.114)         &     (0.127)         &     (0.141)         &     (0.139)         &    (0.0518)         \\
\addlinespace
Number of Sick Days&      0.0299         &      0.0521         &       0.112         &      -0.182         &       0.132         &    -0.00996         \\
            &     (0.158)         &     (0.188)         &     (0.192)         &     (0.228)         &     (0.217)         &    (0.0801)         \\
\addlinespace
Ever Suspended from School&      0.0256         &      0.0362         &      0.0478         &      0.0743         &      0.0551         &      0.0224         \\
            &    (0.0510)         &    (0.0575)         &    (0.0603)         &    (0.0689)         &    (0.0721)         &    (0.0284)         \\
\addlinespace
Num. of Cigarettes Per Day&       1.100         &     -0.0314         &       0.275         &       1.584         &      -2.189         &       2.001         \\
            &     (2.316)         &     (2.699)         &     (3.590)         &     (3.169)         &     (3.333)         &     (1.047)         \\
\bottomrule
\end{tabular}
}
}
\vspace{1ex} \\
\footnotesize\raggedright{Note: This table shows the estimates of the coefficient for attending Reggio Approach preschools from the OLS and difference-in-difference (DiD) models.Column title indicates the corresponding age group, control set, and model. ``None'' refers to the OLS estimate with only the Italian children cohort and with no control variables. ``BIC'' refers to the OLS estimate with only the Italian children cohort and with controls selected by Bayesian Information Criterion (BIC) and additional controls for caregiver's religion. ``Full'' refers to the OLS estimate with only the Italian children cohort and with the full set of controls. ``DidPm'' refers to the difference-in-difference estimate of (Reggio Muni - Parma Muni) - (Reggio None - Parma None) for the Italian children cohort. ``DidPv'' refers to the difference-in-difference estimate of (Reggio Muni - Padova Muni) - (Reggio None - Padova None) for the Italian children cohort.  Robust standard errors are reported in parentheses. Stars show statistical significance as follows: * $p < 0.05$, ** $p < 0.01$, *** $p < 0.001$.}
\end{table}

%Table \ref{ols-L-reg} shows significant effects of the Reggio Approach on the status of being divorced for the age-30 cohort. Other results do not show any significant trend. 

\begin{table}[H] \caption{OLS and Diff-in-Diff for Behaviors, Preschools, Adolescent Cohort} \label{ols-L-reg}
\scalebox{0.80}{
{
\def\sym#1{\ifmmode^{#1}\else\(^{#1}\)\fi}
\begin{tabular}{l*{5}{c}}
\toprule
            &\multicolumn{1}{c}{(1)}&\multicolumn{1}{c}{(2)}&\multicolumn{1}{c}{(3)}&\multicolumn{1}{c}{(4)}&\multicolumn{1}{c}{(5)}\\
            &\multicolumn{1}{c}{None}&\multicolumn{1}{c}{BIC}&\multicolumn{1}{c}{Full}&\multicolumn{1}{c}{DidPm}&\multicolumn{1}{c}{DidPv}\\
\midrule
Not Excited to Learn&     -0.0171         &    -0.00898         &     -0.0106         &     -0.0670         &      0.0332         \\
            &    (0.0486)         &    (0.0552)         &    (0.0591)         &    (0.0547)         &    (0.0740)         \\
\addlinespace
Problems Sitting Still&      0.0671\sym{***}&      0.0916\sym{**} &       0.109\sym{**} &    -0.00679         &       0.200\sym{**} \\
            &    (0.0196)         &    (0.0319)         &    (0.0416)         &    (0.0458)         &    (0.0714)         \\
\addlinespace
Go To School&      0.0708         &      0.0494         &      0.0513         &       0.109         &      0.0480         \\
            &    (0.0655)         &    (0.0679)         &    (0.0696)         &    (0.0692)         &    (0.0764)         \\
\addlinespace
How Much Child Likes School&      -0.277         &      -0.333         &      -0.391         &      -0.172         &      -0.313         \\
            &     (0.249)         &     (0.259)         &     (0.236)         &     (0.296)         &     (0.297)         \\
\addlinespace
Bothered by Migrants&      0.0150         &      0.0313         &      0.0462         &      -0.101         &     -0.0564         \\
            &    (0.0996)         &     (0.108)         &     (0.102)         &     (0.122)         &     (0.130)         \\
\addlinespace
Trust Score &      0.0661         &    -0.00824         &     -0.0513         &       0.589         &      -0.141         \\
            &     (0.327)         &     (0.357)         &     (0.324)         &     (0.439)         &     (0.421)         \\
\addlinespace
Days of Sport (Weekly)&      -0.825\sym{*}  &      -0.976\sym{*}  &      -0.942\sym{*}  &      -1.463\sym{**} &      -0.817         \\
            &     (0.409)         &     (0.433)         &     (0.455)         &     (0.495)         &     (0.539)         \\
\bottomrule
\end{tabular}
}
}
\vspace{1ex} \\
\footnotesize\raggedright{Note: This table shows the estimates of the coefficient for attending Reggio Approach preschools from the OLS and difference-in-difference (DiD) models.Column title indicates the corresponding age group, control set, and model. ``None'' refers to the OLS estimate with only the Italian children cohort and with no control variables. ``BIC'' refers to the OLS estimate with only the Italian children cohort and with controls selected by Bayesian Information Criterion (BIC) and additional controls for caregiver's religion. ``Full'' refers to the OLS estimate with only the Italian children cohort and with the full set of controls. ``DidPm'' refers to the difference-in-difference estimate of (Reggio Muni - Parma Muni) - (Reggio None - Parma None) for the Italian children cohort. ``DidPv'' refers to the difference-in-difference estimate of (Reggio Muni - Padova Muni) - (Reggio None - Padova None) for the Italian children cohort.  Robust standard errors are reported in parentheses. Stars show statistical significance as follows: * $p < 0.05$, ** $p < 0.01$, *** $p < 0.001$.}
\end{table}

%Table \ref{ols-H-reg} show that the significantly increasing effects of the Reggio Approach on numbers of days sick for the past month at the time of the interview for the age-30 cohort. Moreover, the results show the significant decreasing effects on whether the respondent has ever been suspended from school. 

\begin{table}[H] \caption{OLS and Diff-in-Diff for Social Behaviors, Preschools, Adolescent Cohort} \label{ols-H-reg}
\scalebox{0.77}{
{
\def\sym#1{\ifmmode^{#1}\else\(^{#1}\)\fi}
\begin{tabular}{l*{6}{c}}
\toprule
            &\multicolumn{1}{c}{(1)}&\multicolumn{1}{c}{(2)}&\multicolumn{1}{c}{(3)}&\multicolumn{1}{c}{(4)}&\multicolumn{1}{c}{(5)}&\multicolumn{1}{c}{(6)}\\
            &\multicolumn{1}{c}{None}&\multicolumn{1}{c}{BIC}&\multicolumn{1}{c}{Full}&\multicolumn{1}{c}{DidPm}&\multicolumn{1}{c}{DidPv}&\multicolumn{1}{c}{PSM}\\
\midrule
Num. of Friends&      -2.337         &      -2.321         &      -2.628         &      -6.466         &       0.141         &      -0.973         \\
            &     (2.705)         &     (2.692)         &     (2.492)         &     (3.439)         &     (3.524)         &     (1.368)         \\
\addlinespace
Doesn't Talk About Activities&      -0.124         &      -0.120         &      -0.113         &       0.220         &      -0.183         &      -0.194\sym{**} \\
            &     (0.150)         &     (0.164)         &     (0.152)         &     (0.210)         &     (0.193)         &    (0.0727)         \\
\addlinespace
Doesn't Talk About School&      0.0669         &       0.117         &       0.129         &       0.315         &      0.0989         &     -0.0825         \\
            &     (0.135)         &     (0.143)         &     (0.140)         &     (0.183)         &     (0.173)         &    (0.0610)         \\
\bottomrule
\end{tabular}
}
}
\vspace{1ex} \\
\footnotesize\raggedright{Note: This table shows the estimates of the coefficient for attending Reggio Approach preschools from the OLS and difference-in-difference (DiD) models.Column title indicates the corresponding age group, control set, and model. ``None'' refers to the OLS estimate with only the Italian children cohort and with no control variables. ``BIC'' refers to the OLS estimate with only the Italian children cohort and with controls selected by Bayesian Information Criterion (BIC) and additional controls for caregiver's religion. ``Full'' refers to the OLS estimate with only the Italian children cohort and with the full set of controls. ``DidPm'' refers to the difference-in-difference estimate of (Reggio Muni - Parma Muni) - (Reggio None - Parma None) for the Italian children cohort. ``DidPv'' refers to the difference-in-difference estimate of (Reggio Muni - Padova Muni) - (Reggio None - Padova None) for the Italian children cohort.  Robust standard errors are reported in parentheses. Stars show statistical significance as follows: * $p < 0.05$, ** $p < 0.01$, *** $p < 0.001$.}
\end{table}









\subsubsection{Adults}
\begin{landscape}

\begin{table}[H] \caption{OLS and Diff-in-Diff for Cognitive and Education, Preschools, Adult Cohorts} \label{ols-E-reg}
\scalebox{0.80}{{
\def\sym#1{\ifmmode^{#1}\else\(^{#1}\)\fi}
\begin{tabular}{l*{12}{c}}
\toprule
            &\multicolumn{1}{c}{(1)}&\multicolumn{1}{c}{(2)}&\multicolumn{1}{c}{(3)}&\multicolumn{1}{c}{(4)}&\multicolumn{1}{c}{(5)}&\multicolumn{1}{c}{(6)}&\multicolumn{1}{c}{(7)}&\multicolumn{1}{c}{(8)}&\multicolumn{1}{c}{(9)}&\multicolumn{1}{c}{(10)}&\multicolumn{1}{c}{(11)}&\multicolumn{1}{c}{(12)}\\
            &\multicolumn{1}{c}{None30}&\multicolumn{1}{c}{BIC30}&\multicolumn{1}{c}{Full30}&\multicolumn{1}{c}{DidPm30}&\multicolumn{1}{c}{DidPv30}&\multicolumn{1}{c}{PSM30}&\multicolumn{1}{c}{None40}&\multicolumn{1}{c}{BIC40}&\multicolumn{1}{c}{Full40}&\multicolumn{1}{c}{DidPm40}&\multicolumn{1}{c}{DidPv40}&\multicolumn{1}{c}{PSM40}\\
\midrule
IQ Factor   &       0.502\sym{*}  &       0.162         &       0.201         &      -0.112         &     -0.0785         &      -0.674\sym{***}&       0.301         &       0.412         &       0.345         &      -0.115         &     -0.0926         &     -0.0253         \\
            &     (0.235)         &     (0.177)         &     (0.181)         &     (0.246)         &     (0.315)         &     (0.113)         &     (0.243)         &     (0.218)         &     (0.206)         &     (0.306)         &     (0.449)         &     (0.106)         \\
\addlinespace
High School Grade&      -1.478         &      -0.527         &      -0.112         &      -2.817         &       1.693         &       5.899\sym{***}&      -1.358         &      -1.240         &      -1.331         &      -3.649         &       2.888         &       4.691\sym{**} \\
            &     (2.388)         &     (2.127)         &     (2.358)         &     (4.476)         &     (5.069)         &     (1.397)         &     (2.355)         &     (2.835)         &     (2.820)         &     (5.641)         &     (5.230)         &     (1.470)         \\
\addlinespace
University Grade&      -2.798         &      -1.316         &      -0.672         &      -1.680         &       5.272         &       1.589         &       7.118\sym{***}&       3.444         &      -1.936         &       9.176         &       7.580         &      -5.692\sym{**} \\
            &     (1.948)         &     (3.168)         &     (2.207)         &     (3.744)         &     (4.829)         &     (1.557)         &     (1.689)         &     (4.548)         &     (7.720)         &     (5.159)         &     (4.616)         &     (1.754)         \\
\addlinespace
Graduate from High School&      -0.137\sym{***}&     -0.0945\sym{**} &      -0.115\sym{*}  &    -0.00381         &     -0.0997         &     -0.0302         &     -0.0530         &     -0.0885         &      -0.114         &      -0.236         &     -0.0272         &      -0.135         \\
            &    (0.0356)         &    (0.0366)         &    (0.0483)         &    (0.0687)         &    (0.0783)         &    (0.0332)         &     (0.103)         &    (0.0954)         &     (0.104)         &     (0.136)         &     (0.113)         &    (0.0755)         \\
\addlinespace
Max Edu: University&     -0.0833         &     -0.0366         &    -0.00905         &     -0.0627         &      0.0953         &      -0.239\sym{***}&      0.0427         &      0.0328         &      0.0148         &     -0.0878         &       0.274         &      -0.241         \\
            &     (0.106)         &     (0.108)         &     (0.107)         &     (0.147)         &     (0.181)         &    (0.0511)         &     (0.102)         &    (0.0949)         &     (0.106)         &     (0.157)         &     (0.209)         &     (0.177)         \\
\addlinespace
Max Edu: Graduate School&           0         &           0         &           0         &     0.00651         &       0.146         &     -0.0720\sym{***}&           0         &           0         &           0         &     -0.0814         &     -0.0661         &     -0.0107         \\
            &         (.)         &         (.)         &         (.)         &    (0.0349)         &    (0.0772)         &    (0.0153)         &         (.)         &         (.)         &         (.)         &    (0.0460)         &    (0.0643)         &   (0.00560)         \\
\bottomrule
\end{tabular}
}
}
\vspace{1ex} \\
\footnotesize\raggedright{Note: This table shows the estimates of the coefficient for attending Reggio Approach preschools from the OLS and difference-in-difference (DiD) models.Column title indicates the corresponding age group, control set, and model. ``None30'' refers to the OLS estimate with only the age-30 cohort and with no control variables. ``BIC30'' refers to the OLS estimate with only the age-30 cohort and with controls selected by Bayesian Information Criterion (BIC). ``Full30'' refers to the OLS estimate with only the age-30 cohort and with the full set of controls. ``DidPm30'' refers to the difference-in-difference estimate of (Reggio Muni - Parma Muni) - (Reggio None - Parma None) for the age-30 cohort. ``DidPv30'' refers to the difference-in-difference estimate of (Reggio Muni - Padova Muni) - (Reggio None - Padova None) for the age-30 cohort.  Analogous meanings are applied to the age-40 cohort in columns (7) through (12). Robust standard errors are reported in parentheses. Stars show statistical significance as follows: * $p < 0.05$, ** $p < 0.01$, *** $p < 0.001$.}
\end{table}

%Table \ref{ols-W-reg} shows consistently significant positive effects of the Reggio Approach on hours worked per week. Moreover, there are significant effects on the indicator for the lowest household income category for the age-40 cohort. Note that the measures for subject labor income, including ``Monthly Wage'' presented in the table, have very few observations. Household income suffers less from the missing value problem.

\begin{table}[H] \caption{OLS and Diff-in-Diff for Employment and Income, Preschools, Adult Cohorts} \label{ols-W-reg}
\scalebox{0.76}{
{
\def\sym#1{\ifmmode^{#1}\else\(^{#1}\)\fi}
\begin{tabular}{l*{12}{c}}
\toprule
            &\multicolumn{1}{c}{(1)}&\multicolumn{1}{c}{(2)}&\multicolumn{1}{c}{(3)}&\multicolumn{1}{c}{(4)}&\multicolumn{1}{c}{(5)}&\multicolumn{1}{c}{(6)}&\multicolumn{1}{c}{(7)}&\multicolumn{1}{c}{(8)}&\multicolumn{1}{c}{(9)}&\multicolumn{1}{c}{(10)}&\multicolumn{1}{c}{(11)}&\multicolumn{1}{c}{(12)}\\
            &\multicolumn{1}{c}{None30}&\multicolumn{1}{c}{BIC30}&\multicolumn{1}{c}{Full30}&\multicolumn{1}{c}{DidPm30}&\multicolumn{1}{c}{DidPv30}&\multicolumn{1}{c}{PSM30}&\multicolumn{1}{c}{None40}&\multicolumn{1}{c}{BIC40}&\multicolumn{1}{c}{Full40}&\multicolumn{1}{c}{DidPm40}&\multicolumn{1}{c}{DidPv40}&\multicolumn{1}{c}{PSM40}\\
\midrule
Employed    &      0.0343         &      0.0224         &    -0.00738         &       0.128         &       0.105         &      0.0292         &      0.0405         &      0.0215         &      0.0524         &      0.0510         &      0.0575         &     -0.0119         \\
            &    (0.0636)         &    (0.0660)         &    (0.0622)         &    (0.0789)         &    (0.0856)         &    (0.0298)         &    (0.0717)         &    (0.0705)         &    (0.0848)         &    (0.0873)         &    (0.0990)         &    (0.0247)         \\
\addlinespace
Self-Employed&     0.00692         &     -0.0235         &     -0.0282         &     -0.0606         &      0.0576         &     0.00227         &     -0.0866         &      -0.119         &      -0.109         &      -0.185         &     -0.0204         &      0.0362         \\
            &    (0.0693)         &    (0.0653)         &    (0.0696)         &    (0.0759)         &     (0.100)         &    (0.0298)         &     (0.116)         &     (0.121)         &     (0.124)         &     (0.147)         &     (0.152)         &    (0.0307)         \\
\addlinespace
Hours Worked Per Week&       2.942\sym{***}&       2.797\sym{**} &       3.060\sym{**} &       1.740         &       1.218         &       1.009         &      -2.232         &      -1.079         &      -2.736         &      -3.004         &      -4.935         &       16.00\sym{***}\\
            &     (0.814)         &     (0.963)         &     (1.043)         &     (1.944)         &     (1.989)         &     (1.000)         &     (1.839)         &     (2.048)         &     (2.278)         &     (2.282)         &     (5.962)         &     (1.787)         \\
\addlinespace
Monthly Wage&      -10.87         &      -83.34         &       37.44         &      -83.38         &       149.3         &      -238.4         &      -768.4         &      -325.1         &     -1137.5         &     -2837.1         &     -2497.9         &       957.3         \\
            &     (140.0)         &     (167.1)         &     (171.8)         &     (319.5)         &     (332.8)         &     (135.1)         &    (1492.1)         &    (1173.0)         &    (1199.1)         &    (1879.6)         &    (1956.7)         &     (518.5)         \\
\addlinespace
Income: 5,000 Euros of Less&      0.0407         &      0.0291         &      0.0174         &      0.0224         &     -0.0120         &      0.0516\sym{*}  &     -0.0714         &     -0.0693         &     -0.0638         &     -0.0269         &     -0.0709         &     0.00356         \\
            &    (0.0516)         &    (0.0602)         &    (0.0606)         &    (0.0584)         &    (0.0801)         &    (0.0250)         &    (0.0695)         &    (0.0650)         &    (0.0619)         &    (0.0729)         &    (0.0679)         &   (0.00431)         \\
\addlinespace
Income: 5,001-10,000 Euros&      0.0105         &      0.0221         &      0.0338         &      0.0161         &      0.0126         &     0.00315         &           0         &           0         &           0         &           0         &     -0.0357         &     -0.0102         \\
            &    (0.0106)         &    (0.0223)         &    (0.0321)         &    (0.0164)         &    (0.0141)         &   (0.00929)         &         (.)         &         (.)         &         (.)         &         (.)         &    (0.0307)         &   (0.00627)         \\
\addlinespace
Income: 10,001-25,000 Euros&      0.0563         &      0.0891         &      0.0799         &      0.0278         &     0.00590         &      -0.193\sym{***}&      -0.130         &     -0.0674         &     -0.0506         &      0.0887         &      -0.206         &      -0.146         \\
            &    (0.0983)         &     (0.107)         &     (0.107)         &     (0.148)         &     (0.174)         &    (0.0545)         &     (0.136)         &     (0.120)         &     (0.130)         &     (0.186)         &     (0.210)         &     (0.181)         \\
\addlinespace
Income: 25,001-50,000 Euros&      -0.127         &      -0.125         &      -0.122         &     -0.0636         &     -0.0244         &      0.0835         &       0.160         &      0.0731         &       0.106         &    -0.00694         &       0.186         &      0.0414         \\
            &     (0.110)         &     (0.117)         &     (0.123)         &     (0.153)         &     (0.184)         &    (0.0595)         &     (0.143)         &     (0.132)         &     (0.153)         &     (0.193)         &     (0.228)         &     (0.186)         \\
\addlinespace
Income: 50,001-100,000 Euros&      0.0197         &     -0.0153         &    -0.00903         &    -0.00270         &      0.0178         &      0.0600         &      0.0825\sym{**} &      0.0561         &      0.0485         &    0.000694         &       0.143         &      0.0811         \\
            &    (0.0497)         &    (0.0536)         &    (0.0578)         &    (0.0842)         &    (0.0981)         &    (0.0389)         &    (0.0282)         &    (0.0319)         &    (0.0470)         &    (0.0748)         &     (0.108)         &    (0.0634)         \\
\addlinespace
Income: 100,001-250,000 Euros&           0         &           0         &           0         &           0         &           0         &    -0.00540         &     -0.0405         &     0.00747         &     -0.0398         &     -0.0556         &     -0.0170         &      0.0306         \\
            &         (.)         &         (.)         &         (.)         &         (.)         &         (.)         &   (0.00388)         &    (0.0717)         &    (0.0668)         &    (0.0678)         &    (0.0757)         &    (0.0738)         &    (0.0270)         \\
\addlinespace
Income: More than 250,000 Euros&           0         &           0         &           0         &           0         &           0         &           0         &           0         &           0         &           0         &           0         &           0         &           0         \\
            &         (.)         &         (.)         &         (.)         &         (.)         &         (.)         &         (.)         &         (.)         &         (.)         &         (.)         &         (.)         &         (.)         &         (.)         \\
\bottomrule
\end{tabular}
}
}
\vspace{1ex} \\
\footnotesize\raggedright{Note: This table shows the estimates of the coefficient for attending Reggio Approach preschools from the OLS and difference-in-difference (DiD) models.Column title indicates the corresponding age group, control set, and model. ``None30'' refers to the OLS estimate with only the age-30 cohort and with no control variables. ``BIC30'' refers to the OLS estimate with only the age-30 cohort and with controls selected by Bayesian Information Criterion (BIC). ``Full30'' refers to the OLS estimate with only the age-30 cohort and with the full set of controls. ``DidPm30'' refers to the difference-in-difference estimate of (Reggio Muni - Parma Muni) - (Reggio None - Parma None) for the age-30 cohort. ``DidPv30'' refers to the difference-in-difference estimate of (Reggio Muni - Padova Muni) - (Reggio None - Padova None) for the age-30 cohort.  Analogous meanings are applied to the age-40 cohort in columns (7) through (12). Robust standard errors are reported in parentheses. Stars show statistical significance as follows: * $p < 0.05$, ** $p < 0.01$, *** $p < 0.001$.}
\end{table}

%Table \ref{ols-L-reg} shows significant effects of the Reggio Approach on the status of being divorced for the age-30 cohort. Other results do not show any significant trend. 

\begin{table}[H] \caption{OLS and Diff-in-Diff for Living Environment, Preschools, Adult Cohorts} \label{ols-L-reg}
\scalebox{0.80}{
{
\def\sym#1{\ifmmode^{#1}\else\(^{#1}\)\fi}
\begin{tabular}{l*{12}{c}}
\toprule
            &\multicolumn{1}{c}{(1)}&\multicolumn{1}{c}{(2)}&\multicolumn{1}{c}{(3)}&\multicolumn{1}{c}{(4)}&\multicolumn{1}{c}{(5)}&\multicolumn{1}{c}{(6)}&\multicolumn{1}{c}{(7)}&\multicolumn{1}{c}{(8)}&\multicolumn{1}{c}{(9)}&\multicolumn{1}{c}{(10)}&\multicolumn{1}{c}{(11)}&\multicolumn{1}{c}{(12)}\\
            &\multicolumn{1}{c}{None30}&\multicolumn{1}{c}{BIC30}&\multicolumn{1}{c}{Full30}&\multicolumn{1}{c}{DidPm30}&\multicolumn{1}{c}{DidPv30}&\multicolumn{1}{c}{IPW30}&\multicolumn{1}{c}{None40}&\multicolumn{1}{c}{BIC40}&\multicolumn{1}{c}{Full40}&\multicolumn{1}{c}{DidPm40}&\multicolumn{1}{c}{DidPv40}&\multicolumn{1}{c}{IPW40}\\
\midrule
Married or Cohabitating&       0.173         &      0.0529         &     0.00352         &       0.157         &       0.169         &      -0.128\sym{*}  &       0.130         &       0.108         &       0.167         &     -0.0499         &       0.340         &       0.475\sym{***}\\
            &    (0.0937)         &    (0.0937)         &     (0.101)         &     (0.137)         &     (0.173)         &    (0.0523)         &     (0.136)         &     (0.139)         &     (0.143)         &     (0.194)         &     (0.227)         &    (0.0641)         \\
\addlinespace
Divorced    &           0         &           0         &           0         &    -0.00938         &      0.0424         &     -0.0113         &      -0.214         &      -0.188         &      -0.197         &      -0.305         &      -0.467\sym{**} &      0.0327         \\
            &         (.)         &         (.)         &         (.)         &   (0.00983)         &    (0.0401)         &   (0.00591)         &     (0.125)         &     (0.122)         &     (0.120)         &     (0.162)         &     (0.158)         &    (0.0357)         \\
\addlinespace
Num. of Children in House&     0.00778         &     -0.0108         &     -0.0252         &      0.0440         &      0.0253         &      -0.132\sym{**} &       0.210         &       0.143         &       0.120         &      0.0900         &       0.126         &       0.235\sym{**} \\
            &    (0.0715)         &    (0.0662)         &    (0.0571)         &     (0.127)         &     (0.139)         &    (0.0466)         &     (0.175)         &     (0.193)         &     (0.209)         &     (0.303)         &     (0.373)         &    (0.0872)         \\
\addlinespace
Own House   &      -0.226\sym{*}  &      -0.119         &      -0.156         &      -0.231         &       0.119         &      0.0448         &     -0.0236         &     -0.0340         &     -0.0375         &      -0.209         &      0.0329         &       0.316\sym{***}\\
            &    (0.0945)         &    (0.0971)         &     (0.106)         &     (0.137)         &     (0.164)         &    (0.0543)         &     (0.131)         &     (0.122)         &     (0.138)         &     (0.178)         &     (0.180)         &    (0.0718)         \\
\addlinespace
Live With Parents&      0.0288         &     -0.0224         &      0.0141         &      -0.112         &      -0.370\sym{*}  &     -0.0231         &      0.0206         &      0.0257         &      0.0338         &      0.0236         &     -0.0284         &      -0.545\sym{***}\\
            &    (0.0679)         &    (0.0692)         &    (0.0660)         &     (0.116)         &     (0.152)         &    (0.0479)         &    (0.0146)         &    (0.0222)         &    (0.0274)         &    (0.0733)         &     (0.132)         &    (0.0439)         \\
\bottomrule
\end{tabular}
}
}
\vspace{1ex} \\
\footnotesize\raggedright{Note: This table shows the estimates of the coefficient for attending Reggio Approach preschools from the OLS and difference-in-difference (DiD) models.Column title indicates the corresponding age group, control set, and model. ``None30'' refers to the OLS estimate with only the age-30 cohort and with no control variables. ``BIC30'' refers to the OLS estimate with only the age-30 cohort and with controls selected by Bayesian Information Criterion (BIC). ``Full30'' refers to the OLS estimate with only the age-30 cohort and with the full set of controls. ``DidPm30'' refers to the difference-in-difference estimate of (Reggio Muni - Parma Muni) - (Reggio None - Parma None) for the age-30 cohort. ``DidPv30'' refers to the difference-in-difference estimate of (Reggio Muni - Padova Muni) - (Reggio None - Padova None) for the age-30 cohort.  Analogous meanings are applied to the age-40 cohort in columns (7) through (12). Robust standard errors are reported in parentheses. Stars show statistical significance as follows: * $p < 0.05$, ** $p < 0.01$, *** $p < 0.001$.}
\end{table}

%Table \ref{ols-H-reg} show that the significantly increasing effects of the Reggio Approach on numbers of days sick for the past month at the time of the interview for the age-30 cohort. Moreover, the results show the significant decreasing effects on whether the respondent has ever been suspended from school. 

\begin{table}[H] \caption{OLS and Diff-in-Diff for Health and Risk, Preschools, Adult Cohorts} \label{ols-H-reg}
\scalebox{0.77}{
{
\def\sym#1{\ifmmode^{#1}\else\(^{#1}\)\fi}
\begin{tabular}{l*{12}{c}}
\toprule
            &\multicolumn{1}{c}{(1)}&\multicolumn{1}{c}{(2)}&\multicolumn{1}{c}{(3)}&\multicolumn{1}{c}{(4)}&\multicolumn{1}{c}{(5)}&\multicolumn{1}{c}{(6)}&\multicolumn{1}{c}{(7)}&\multicolumn{1}{c}{(8)}&\multicolumn{1}{c}{(9)}&\multicolumn{1}{c}{(10)}&\multicolumn{1}{c}{(11)}&\multicolumn{1}{c}{(12)}\\
            &\multicolumn{1}{c}{None30}&\multicolumn{1}{c}{BIC30}&\multicolumn{1}{c}{Full30}&\multicolumn{1}{c}{DidPm30}&\multicolumn{1}{c}{DidPv30}&\multicolumn{1}{c}{PSM30}&\multicolumn{1}{c}{None40}&\multicolumn{1}{c}{BIC40}&\multicolumn{1}{c}{Full40}&\multicolumn{1}{c}{DidPm40}&\multicolumn{1}{c}{DidPv40}&\multicolumn{1}{c}{PSM40}\\
\midrule
Tried Marijuana&     -0.0819         &      -0.132         &      -0.107         &      -0.174         &   -0.000543         &     -0.0501         &      0.0118         &      0.0869         &      0.0428         &       0.191         &     0.00664         &      0.0852\sym{*}  \\
            &     (0.101)         &    (0.0964)         &    (0.0886)         &     (0.119)         &     (0.153)         &    (0.0403)         &     (0.101)         &     (0.107)         &     (0.105)         &     (0.145)         &     (0.141)         &    (0.0390)         \\
\addlinespace
Num. of Cigarettes Per Day&       1.777         &       4.807         &       4.243         &       4.574         &       0.676         &       4.952\sym{***}&      -3.300         &      -1.022         &       1.763         &      -3.825         &      -2.729         &       4.541\sym{**} \\
            &     (1.835)         &     (2.473)         &     (2.755)         &     (3.264)         &     (4.308)         &     (1.133)         &     (5.026)         &     (6.032)         &     (5.541)         &     (6.421)         &     (6.365)         &     (1.602)         \\
\addlinespace
BMI         &       0.320         &      -0.116         &      -0.235         &      -1.584         &      -0.445         &      0.0863         &       2.086\sym{*}  &       2.412\sym{*}  &       2.935\sym{*}  &       4.424\sym{**} &       2.529         &     -0.0117         \\
            &     (0.707)         &     (0.675)         &     (0.680)         &     (0.864)         &     (1.193)         &     (0.284)         &     (0.955)         &     (0.970)         &     (1.463)         &     (1.348)         &     (1.605)         &     (0.460)         \\
\addlinespace
Obese       &      -0.226\sym{*}  &     -0.0689         &     -0.0651         &      -0.173         &      -0.197         &       0.158\sym{**} &      -0.293\sym{*}  &      -0.314\sym{*}  &      -0.293\sym{*}  &      -0.331\sym{*}  &      -0.197         &      -0.309\sym{***}\\
            &     (0.114)         &    (0.0723)         &    (0.0751)         &     (0.105)         &     (0.148)         &    (0.0494)         &     (0.141)         &     (0.128)         &     (0.132)         &     (0.167)         &     (0.235)         &    (0.0746)         \\
\addlinespace
Overweight  &      0.0906         &     -0.0147         &     -0.0444         &      -0.134         &     -0.0702         &   -0.000595         &       0.228\sym{**} &       0.219\sym{*}  &       0.168         &       0.291\sym{*}  &       0.254         &      0.0382         \\
            &    (0.0828)         &    (0.0890)         &    (0.0943)         &     (0.123)         &     (0.138)         &    (0.0431)         &    (0.0838)         &     (0.103)         &     (0.113)         &     (0.144)         &     (0.136)         &    (0.0507)         \\
\addlinespace
Good Health &     0.00229         &     -0.0654         &     -0.0716         &      -0.264         &      0.0643         &       0.310\sym{***}&     -0.0825         &     -0.0471         &      -0.100         &      -0.422\sym{*}  &     0.00298         &       0.570         \\
            &     (0.154)         &     (0.161)         &     (0.148)         &     (0.193)         &     (0.257)         &    (0.0624)         &     (0.119)         &     (0.136)         &     (0.164)         &     (0.203)         &     (0.395)         &     (0.352)         \\
\addlinespace
No Problematic Health Condition&      -0.135         &     -0.0481         &     -0.0326         &    -0.00210         &      0.0587         &     -0.0615         &     -0.0298         &      -0.103         &     -0.0548         &      -0.239         &      -0.139         &       0.114         \\
            &     (0.114)         &     (0.117)         &     (0.120)         &     (0.153)         &     (0.185)         &    (0.0603)         &     (0.181)         &     (0.196)         &     (0.215)         &     (0.237)         &     (0.270)         &    (0.0721)         \\
\addlinespace
Num. of Days Sick Past Month&       0.248\sym{*}  &       0.306\sym{**} &       0.319\sym{**} &       0.299\sym{*}  &       0.265         &       0.224\sym{***}&       0.120\sym{**} &       0.137\sym{**} &       0.146\sym{*}  &       0.331         &      0.0948         &      0.0507         \\
            &     (0.102)         &     (0.108)         &     (0.120)         &     (0.118)         &     (0.162)         &    (0.0660)         &    (0.0375)         &    (0.0481)         &    (0.0683)         &     (0.172)         &     (0.144)         &    (0.0384)         \\
\addlinespace
Ever Suspended from School&     -0.0449         &     -0.0578         &     -0.0533         &     -0.0431         &      -0.157         &     -0.0137         &     -0.0199         &     -0.0206         &     -0.0414         &      -0.100         &     -0.0139         &      0.0235         \\
            &    (0.0628)         &    (0.0681)         &    (0.0767)         &    (0.0856)         &     (0.101)         &    (0.0227)         &    (0.0731)         &    (0.0778)         &    (0.0771)         &     (0.100)         &    (0.0685)         &    (0.0332)         \\
\addlinespace
Age At First Drink&       3.878         &       0.603         &       0.891         &       0.817         &       1.830         &      -1.674         &       3.892         &       4.340\sym{*}  &       2.720         &       2.958         &       6.816\sym{*}  &       3.204\sym{*}  \\
            &     (2.100)         &     (1.344)         &     (1.170)         &     (2.021)         &     (2.316)         &     (0.894)         &     (2.529)         &     (1.996)         &     (2.178)         &     (2.470)         &     (3.312)         &     (1.323)         \\
\bottomrule
\end{tabular}
}
}
\vspace{1ex} \\
\footnotesize\raggedright{Note: This table shows the estimates of the coefficient for attending Reggio Approach preschools from the OLS and difference-in-difference (DiD) models.Column title indicates the corresponding age group, control set, and model. ``None30'' refers to the OLS estimate with only the age-30 cohort and with no control variables. ``BIC30'' refers to the OLS estimate with only the age-30 cohort and with controls selected by Bayesian Information Criterion (BIC). ``Full30'' refers to the OLS estimate with only the age-30 cohort and with the full set of controls. ``DidPm30'' refers to the difference-in-difference estimate of (Reggio Muni - Parma Muni) - (Reggio None - Parma None) for the age-30 cohort. ``DidPv30'' refers to the difference-in-difference estimate of (Reggio Muni - Padova Muni) - (Reggio None - Padova None) for the age-30 cohort.  Analogous meanings are applied to the age-40 cohort in columns (7) through (12). Robust standard errors are reported in parentheses. Stars show statistical significance as follows: * $p < 0.05$, ** $p < 0.01$, *** $p < 0.001$.}
\end{table}

%Table \ref{ols-N-reg} shows no consistently significant trends between the age-30 cohort and age-40 cohort. For the age-30 cohort, there are significantly decreasing effects on the optimistic look in life. Moreover, there are significantly increasing effects on whether the respondent is likely to do the same to someone who puts him/her in a difficult situation and to someone who insults him/her.

%For the age-40 cohort, results show the significantly decreasing effects on depression for the age-30 cohort. There are also significantly positive effects on satisfaction with income and with work for the age-40 cohort. However, there are significantly positive effects on whether the respondent thinks work is source of stress. 

\begin{table}[H] \caption{OLS and Diff-in-Diff for Non-cognitive, Preschools, Adult Cohorts} \label{ols-N-reg}
\scalebox{0.80}{
{
\def\sym#1{\ifmmode^{#1}\else\(^{#1}\)\fi}
\begin{tabular}{l*{10}{c}}
\toprule
            &\multicolumn{1}{c}{(1)}&\multicolumn{1}{c}{(2)}&\multicolumn{1}{c}{(3)}&\multicolumn{1}{c}{(4)}&\multicolumn{1}{c}{(5)}&\multicolumn{1}{c}{(6)}&\multicolumn{1}{c}{(7)}&\multicolumn{1}{c}{(8)}&\multicolumn{1}{c}{(9)}&\multicolumn{1}{c}{(10)}\\
            &\multicolumn{1}{c}{None30}&\multicolumn{1}{c}{BIC30}&\multicolumn{1}{c}{Full30}&\multicolumn{1}{c}{DidPm30}&\multicolumn{1}{c}{DidPv30}&\multicolumn{1}{c}{None40}&\multicolumn{1}{c}{BIC40}&\multicolumn{1}{c}{Full40}&\multicolumn{1}{c}{DidPm40}&\multicolumn{1}{c}{DidPv40}\\
\midrule
Locus of Control - positive&       0.406\sym{**} &       0.184         &       0.180         &     -0.0689         &     -0.0671         &       0.489         &       0.742\sym{**} &       0.718\sym{*}  &       0.659         &       0.705         \\
            &     (0.151)         &     (0.138)         &     (0.148)         &     (0.271)         &     (0.285)         &     (0.285)         &     (0.250)         &     (0.280)         &     (0.484)         &     (0.466)         \\
\addlinespace
Depression Score - positive&       3.356\sym{*}  &       0.710         &       0.987         &       2.336         &      -1.683         &       1.824         &       2.747         &       2.400         &       3.795         &      -0.856         \\
            &     (1.610)         &     (0.953)         &     (1.004)         &     (1.723)         &     (2.360)         &     (2.111)         &     (2.058)         &     (2.293)         &     (3.836)         &     (2.875)         \\
\addlinespace
Stress      &       0.188         &      0.0116         &      0.0142         &      0.0398         &      -0.249         &      0.0143         &      0.0263         &     0.00453         &      -0.110         &       0.325         \\
            &     (0.196)         &     (0.139)         &     (0.143)         &     (0.218)         &     (0.303)         &     (0.174)         &     (0.177)         &     (0.189)         &     (0.384)         &     (0.341)         \\
\addlinespace
Work is Source of Stress&     -0.0602         &     -0.0536         &     -0.0318         &      0.0944         &       0.523         &       0.185         &      0.0276         &     -0.0234         &       0.295         &      0.0528         \\
            &     (0.128)         &     (0.132)         &     (0.144)         &     (0.215)         &     (0.282)         &     (0.179)         &     (0.188)         &     (0.207)         &     (0.300)         &     (0.355)         \\
\addlinespace
Satisfied with Income&       0.169         &       0.163         &       0.222         &       0.395         &     -0.0845         &      -0.344         &      -0.403         &      -0.429         &      -0.261         &      -0.237         \\
            &     (0.181)         &     (0.188)         &     (0.192)         &     (0.284)         &     (0.381)         &     (0.241)         &     (0.248)         &     (0.270)         &     (0.431)         &     (0.422)         \\
\addlinespace
Satisfied with Work&       0.651\sym{**} &       0.574\sym{**} &       0.542\sym{**} &       0.680\sym{*}  &       0.821\sym{*}  &      -0.110         &     -0.0941         &      -0.177         &      -0.211         &      -0.141         \\
            &     (0.207)         &     (0.204)         &     (0.210)         &     (0.309)         &     (0.330)         &     (0.186)         &     (0.208)         &     (0.243)         &     (0.358)         &     (0.366)         \\
\addlinespace
Satisfied with Health&      0.0611         &     -0.0932         &     -0.0250         &      -0.129         &      -0.350         &      -0.329\sym{*}  &      -0.279         &      -0.373\sym{*}  &      -0.983\sym{**} &      -0.758         \\
            &     (0.165)         &     (0.170)         &     (0.176)         &     (0.197)         &     (0.268)         &     (0.148)         &     (0.158)         &     (0.172)         &     (0.312)         &     (0.394)         \\
\addlinespace
Satisfied with Family&     -0.0425         &      -0.126         &      -0.162         &       0.256         &       0.409         &      0.0667         &     -0.0112         &      0.0606         &       0.253         &       0.210         \\
            &     (0.172)         &     (0.188)         &     (0.197)         &     (0.247)         &     (0.378)         &     (0.282)         &     (0.276)         &     (0.304)         &     (0.490)         &     (0.439)         \\
\addlinespace
Optimistic Look in Life&       0.272\sym{**} &       0.331\sym{**} &       0.355\sym{**} &      0.0795         &     -0.0507         &      -0.334\sym{**} &      -0.253\sym{*}  &      -0.321\sym{**} &      -0.363\sym{*}  &      -0.218         \\
            &     (0.101)         &     (0.110)         &     (0.112)         &     (0.156)         &     (0.179)         &     (0.109)         &     (0.109)         &     (0.110)         &     (0.178)         &     (0.223)         \\
\addlinespace
Positive Reciprocity&       0.158         &      0.0321         &      0.0674         &     -0.0810         &      -0.296         &       0.106         &       0.260         &       0.243         &     -0.0928         &       0.660         \\
            &     (0.145)         &     (0.167)         &     (0.176)         &     (0.210)         &     (0.303)         &     (0.157)         &     (0.182)         &     (0.191)         &     (0.266)         &     (0.423)         \\
\addlinespace
Negative Reciprocity&      -0.169         &     -0.0200         &     -0.0599         &       0.570         &       1.135\sym{**} &      -0.194         &      -0.310         &      -0.249         &      -0.183         &       0.311         \\
            &     (0.207)         &     (0.179)         &     (0.184)         &     (0.308)         &     (0.432)         &     (0.300)         &     (0.306)         &     (0.312)         &     (0.502)         &     (0.515)         \\
\bottomrule
\end{tabular}
}
}
\vspace{1ex} \\
\footnotesize\raggedright{Note: This table shows the estimates of the coefficient for attending Reggio Approach preschools from the OLS and difference-in-difference (DiD) models.Column title indicates the corresponding age group, control set, and model. ``None30'' refers to the OLS estimate with only the age-30 cohort and with no control variables. ``BIC30'' refers to the OLS estimate with only the age-30 cohort and with controls selected by Bayesian Information Criterion (BIC). ``Full30'' refers to the OLS estimate with only the age-30 cohort and with the full set of controls. ``DidPm30'' refers to the difference-in-difference estimate of (Reggio Muni - Parma Muni) - (Reggio None - Parma None) for the age-30 cohort. ``DidPv30'' refers to the difference-in-difference estimate of (Reggio Muni - Padova Muni) - (Reggio None - Padova None) for the age-30 cohort.  Analogous meanings are applied to the age-40 cohort in columns (7) through (12). Robust standard errors are reported in parentheses. Stars show statistical significance as follows: * $p < 0.05$, ** $p < 0.01$, *** $p < 0.001$.}
\end{table}

%Table \ref{ols-S-reg} shows consistently significant and negative effects of the Reggio Approach on whether the respondent does volunteer work for both the age-30 and age-40 cohorts. Moreover, for the age-40 cohort, there are significantly negative effects on respondents having migrant friends and significantly positive effects on whether respondents have ever voted for municipal and regional elections. For the age-30 cohort, there are significantly negative effects on whether respondents have ever voted for national elections.

\begin{table}[H] \caption{OLS and Diff-in-Diff for Social Behavior, Preschools, Adult Cohorts} \label{ols-S-reg}
\scalebox{0.80}{
{
\def\sym#1{\ifmmode^{#1}\else\(^{#1}\)\fi}
\begin{tabular}{l*{12}{c}}
\toprule
            &\multicolumn{1}{c}{(1)}&\multicolumn{1}{c}{(2)}&\multicolumn{1}{c}{(3)}&\multicolumn{1}{c}{(4)}&\multicolumn{1}{c}{(5)}&\multicolumn{1}{c}{(6)}&\multicolumn{1}{c}{(7)}&\multicolumn{1}{c}{(8)}&\multicolumn{1}{c}{(9)}&\multicolumn{1}{c}{(10)}&\multicolumn{1}{c}{(11)}&\multicolumn{1}{c}{(12)}\\
            &\multicolumn{1}{c}{None30}&\multicolumn{1}{c}{BIC30}&\multicolumn{1}{c}{Full30}&\multicolumn{1}{c}{DidPm30}&\multicolumn{1}{c}{DidPv30}&\multicolumn{1}{c}{PSM30}&\multicolumn{1}{c}{None40}&\multicolumn{1}{c}{BIC40}&\multicolumn{1}{c}{Full40}&\multicolumn{1}{c}{DidPm40}&\multicolumn{1}{c}{DidPv40}&\multicolumn{1}{c}{PSM40}\\
\midrule
Favorable to Migrants&      0.0847         &       0.104         &       0.105         &    -0.00766         &       0.561\sym{*}  &       0.192\sym{**} &      0.0906         &       0.189         &       0.163         &       0.592\sym{*}  &       0.393         &       0.125         \\
            &     (0.103)         &     (0.107)         &     (0.113)         &     (0.201)         &     (0.269)         &    (0.0697)         &     (0.160)         &     (0.161)         &     (0.174)         &     (0.287)         &     (0.373)         &    (0.0855)         \\
\addlinespace
Num. of Friends&       2.346\sym{*}  &       1.874         &       1.239         &       2.904         &       1.724         &      -1.294         &       2.004         &       2.315         &       2.003         &       3.225         &       5.375         &       0.755         \\
            &     (1.046)         &     (1.306)         &     (1.157)         &     (1.852)         &     (2.082)         &     (1.024)         &     (1.868)         &     (2.054)         &     (1.460)         &     (2.366)         &     (3.335)         &     (1.500)         \\
\addlinespace
Has Migrant Friends&      0.0503         &     0.00935         &     0.00514         &     0.00623         &       0.277         &      0.0169         &     -0.0339         &     -0.0114         &       0.102         &     -0.0982         &      0.0777         &       0.332\sym{***}\\
            &     (0.102)         &    (0.0956)         &    (0.0993)         &     (0.132)         &     (0.165)         &    (0.0540)         &     (0.131)         &     (0.144)         &     (0.136)         &     (0.197)         &     (0.217)         &    (0.0698)         \\
\addlinespace
Volunteers  &      0.0632\sym{*}  &      0.0466         &      0.0307         &      -0.163\sym{*}  &     -0.0543         &      -0.131\sym{**} &       0.103\sym{***}&      0.0718         &       0.112         &      0.0744         &     0.00705         &      0.0866         \\
            &    (0.0252)         &    (0.0240)         &    (0.0291)         &    (0.0819)         &     (0.111)         &    (0.0398)         &    (0.0312)         &    (0.0416)         &    (0.0604)         &     (0.110)         &    (0.0883)         &    (0.0659)         \\
\addlinespace
Child Eats Meal with Fam&      0.0714         &       1.000\sym{***}&       0.667         &      -0.230         &      -0.357         &      -0.116         &      -0.549         &      -0.567         &      -0.371         &      -0.754         &      -0.612         &      -0.135         \\
            &     (0.454)         &  (1.77e-15)         &         (.)         &     (0.623)         &     (0.272)         &     (0.276)         &     (0.489)         &     (0.402)         &     (0.369)         &     (0.574)         &     (0.479)         &     (0.135)         \\
\addlinespace
Ever Voted for Municipal&       0.161         &      0.0380         &      0.0885         &     -0.0787         &       0.441\sym{**} &      0.0396         &     -0.0110         &      0.0911         &     -0.0396         &      -0.116         &       0.266         &      0.0686         \\
            &     (0.116)         &    (0.0974)         &    (0.0985)         &     (0.122)         &     (0.162)         &    (0.0461)         &     (0.146)         &     (0.154)         &     (0.162)         &     (0.205)         &     (0.252)         &    (0.0640)         \\
\addlinespace
Ever Voted for Regional&      0.0740         &     -0.0483         &      0.0134         &      -0.106         &       0.369\sym{*}  &      0.0488         &     -0.0549         &      0.0348         &     -0.0537         &      -0.123         &     -0.0563         &       0.164\sym{*}  \\
            &     (0.117)         &     (0.106)         &     (0.108)         &     (0.130)         &     (0.173)         &    (0.0482)         &     (0.139)         &     (0.161)         &     (0.167)         &     (0.202)         &     (0.284)         &    (0.0661)         \\
\addlinespace
Ever Voted for National&       0.146         &       0.189         &       0.216         &       0.336\sym{**} &       0.507\sym{**} &     -0.0627         &       0.110         &      0.0719         &      0.0753         &      0.0657         &      0.0544         &      0.0244         \\
            &     (0.109)         &     (0.108)         &     (0.115)         &     (0.116)         &     (0.160)         &    (0.0372)         &     (0.123)         &     (0.121)         &     (0.138)         &     (0.144)         &     (0.209)         &    (0.0463)         \\
\bottomrule
\end{tabular}
}
}
\vspace{1ex} \\
\footnotesize\raggedright{Note: This table shows the estimates of the coefficient for attending Reggio Approach preschools from the OLS and difference-in-difference (DiD) models.Column title indicates the corresponding age group, control set, and model. ``None30'' refers to the OLS estimate with only the age-30 cohort and with no control variables. ``BIC30'' refers to the OLS estimate with only the age-30 cohort and with controls selected by Bayesian Information Criterion (BIC). ``Full30'' refers to the OLS estimate with only the age-30 cohort and with the full set of controls. ``DidPm30'' refers to the difference-in-difference estimate of (Reggio Muni - Parma Muni) - (Reggio None - Parma None) for the age-30 cohort. ``DidPv30'' refers to the difference-in-difference estimate of (Reggio Muni - Padova Muni) - (Reggio None - Padova None) for the age-30 cohort.  Analogous meanings are applied to the age-40 cohort in columns (7) through (12). Robust standard errors are reported in parentheses. Stars show statistical significance as follows: * $p < 0.05$, ** $p < 0.01$, *** $p < 0.001$.}
\end{table}









% ==============================================================================%
\subsection{Municipal vs. Religious}\label{appendix:religious}
% ==============================================================================% 
\subsubsection{Children}
\begin{table}[H] \caption{OLS and Diff-in-Diff for Cognitive and Noncognitive, Preschools, Children Cohort} \label{ols-E-reg}
\scalebox{0.80}{{
\def\sym#1{\ifmmode^{#1}\else\(^{#1}\)\fi}
\begin{tabular}{l*{10}{c}}
\toprule
            &\multicolumn{1}{c}{(1)}&\multicolumn{1}{c}{(2)}&\multicolumn{1}{c}{(3)}&\multicolumn{1}{c}{(4)}&\multicolumn{1}{c}{(5)}&\multicolumn{1}{c}{(6)}&\multicolumn{1}{c}{(7)}&\multicolumn{1}{c}{(8)}&\multicolumn{1}{c}{(9)}&\multicolumn{1}{c}{(10)}\\
            &\multicolumn{1}{c}{NoneIt}&\multicolumn{1}{c}{BICIt}&\multicolumn{1}{c}{FullIt}&\multicolumn{1}{c}{DidPmIt}&\multicolumn{1}{c}{DidPvIt}&\multicolumn{1}{c}{NoneMg}&\multicolumn{1}{c}{BICMg}&\multicolumn{1}{c}{FullMg}&\multicolumn{1}{c}{DidPmMg}&\multicolumn{1}{c}{DidPvMg}\\
\midrule
IQ Factor   &      -0.316\sym{**} &      -0.279\sym{**} &      -0.289\sym{**} &      -0.234         &      0.0123         &      -0.339         &      -0.301         &      -0.536         &       0.202         &       0.330         \\
            &     (0.105)         &     (0.104)         &     (0.108)         &     (0.159)         &     (0.177)         &     (0.288)         &     (0.334)         &     (0.329)         &     (0.407)         &     (0.392)         \\
\addlinespace
IQ Score    &     -0.0662\sym{*}  &     -0.0603\sym{*}  &     -0.0659\sym{*}  &     -0.0425         &     0.00206         &      -0.115         &     -0.0894         &      -0.165\sym{*}  &      0.0422         &      0.0312         \\
            &    (0.0258)         &    (0.0253)         &    (0.0261)         &    (0.0375)         &    (0.0407)         &    (0.0735)         &    (0.0829)         &    (0.0815)         &     (0.104)         &    (0.0963)         \\
\addlinespace
SDQ Composite - Child&       0.934         &       1.368\sym{*}  &       1.671\sym{**} &       0.319         &       2.109\sym{*}  &      -1.522         &      -0.527         &      -0.856         &       0.241         &       0.276         \\
            &     (0.587)         &     (0.577)         &     (0.578)         &     (0.865)         &     (0.892)         &     (1.143)         &     (1.062)         &     (1.028)         &     (1.540)         &     (1.575)         \\
\addlinespace
SDQ Pro-social - Child&       0.253         &       0.175         &      0.0705         &     -0.0827         &       0.305         &       1.015         &       0.992         &       1.876\sym{**} &       3.160\sym{***}&       1.098         \\
            &     (0.231)         &     (0.234)         &     (0.265)         &     (0.323)         &     (0.326)         &     (0.581)         &     (0.641)         &     (0.571)         &     (0.914)         &     (0.750)         \\
\addlinespace
SDQ Peer problems - Child&      -0.220         &      -0.144         &     -0.0697         &      -0.329         &       0.217         &       0.616         &       0.625         &       0.532         &       1.545\sym{*}  &       0.863         \\
            &     (0.154)         &     (0.157)         &     (0.163)         &     (0.278)         &     (0.274)         &     (0.394)         &     (0.435)         &     (0.475)         &     (0.740)         &     (0.550)         \\
\addlinespace
SDQ Hyper - Child&       0.164         &       0.328         &       0.499         &    -0.00684         &       0.127         &      -1.304\sym{*}  &      -0.685         &      -0.975         &      -0.911         &      -1.408         \\
            &     (0.299)         &     (0.304)         &     (0.319)         &     (0.417)         &     (0.417)         &     (0.622)         &     (0.573)         &     (0.561)         &     (0.897)         &     (0.815)         \\
\addlinespace
SDQ Emotional - Child&       0.564\sym{*}  &       0.643\sym{**} &       0.658\sym{**} &       0.556         &       1.152\sym{***}&      -0.469         &      -0.263         &     -0.0323         &      -0.502         &       0.477         \\
            &     (0.234)         &     (0.224)         &     (0.241)         &     (0.325)         &     (0.326)         &     (0.404)         &     (0.406)         &     (0.621)         &     (0.702)         &     (0.622)         \\
\addlinespace
SDQ Conduct - Child&       0.425\sym{*}  &       0.540\sym{**} &       0.584\sym{**} &      0.0982         &       0.614\sym{*}  &      -0.366         &      -0.205         &      -0.380         &       0.109         &       0.344         \\
            &     (0.188)         &     (0.187)         &     (0.199)         &     (0.277)         &     (0.281)         &     (0.373)         &     (0.349)         &     (0.391)         &     (0.792)         &     (0.496)         \\
\bottomrule
\end{tabular}
}
}
\vspace{1ex} \\
\footnotesize\raggedright{Note: This table shows the estimates of the coefficient for attending Reggio Approach preschools from the OLS and difference-in-difference (DiD) models.Column title indicates the corresponding age group, control set, and model. ``NoneIt'' refers to the OLS estimate with only the Italian children cohort and with no control variables. ``BICIt'' refers to the OLS estimate with only the Italian children cohort and with controls selected by Bayesian Information Criterion (BIC) and additional controls for caregiver's religion. ``FullIt'' refers to the OLS estimate with only the Italian children cohort and with the full set of controls. ``DidPmIt'' refers to the difference-in-difference estimate of (Reggio Muni - Parma Muni) - (Reggio None - Parma None) for the Italian children cohort. ``DidPvIt'' refers to the difference-in-difference estimate of (Reggio Muni - Padova Muni) - (Reggio None - Padova None) for the Italian children cohort.  Analogous meanings are applied to the migrant children cohort in columns (7) through (12). Robust standard errors are reported in parentheses. Stars show statistical significance as follows: * $p < 0.05$, ** $p < 0.01$, *** $p < 0.001$.}
\end{table}

%Table \ref{ols-W-reg} shows consistently significant positive effects of the Reggio Approach on hours worked per week. Moreover, there are significant effects on the indicator for the lowest household income category for the age-40 cohort. Note that the measures for subject labor income, including ``Monthly Wage'' presented in the table, have very few observations. Household income suffers less from the missing value problem.

\begin{table}[H] \caption{OLS and Diff-in-Diff for Health and Risk, Preschools, Children Cohort} \label{ols-W-reg}
\scalebox{0.76}{
{
\def\sym#1{\ifmmode^{#1}\else\(^{#1}\)\fi}
\begin{tabular}{l*{10}{c}}
\toprule
            &\multicolumn{1}{c}{(1)}&\multicolumn{1}{c}{(2)}&\multicolumn{1}{c}{(3)}&\multicolumn{1}{c}{(4)}&\multicolumn{1}{c}{(5)}&\multicolumn{1}{c}{(6)}&\multicolumn{1}{c}{(7)}&\multicolumn{1}{c}{(8)}&\multicolumn{1}{c}{(9)}&\multicolumn{1}{c}{(10)}\\
            &\multicolumn{1}{c}{NoneIt}&\multicolumn{1}{c}{BICIt}&\multicolumn{1}{c}{FullIt}&\multicolumn{1}{c}{DidPmIt}&\multicolumn{1}{c}{DidPvIt}&\multicolumn{1}{c}{NoneMg}&\multicolumn{1}{c}{BICMg}&\multicolumn{1}{c}{FullMg}&\multicolumn{1}{c}{DidPmMg}&\multicolumn{1}{c}{DidPvMg}\\
\midrule
Obese       &       0.114\sym{*}  &       0.107         &       0.164\sym{**} &      0.0997         &     -0.0160         &      -0.146         &      -0.157         &      -0.171         &      0.0298         &      -0.143         \\
            &    (0.0571)         &    (0.0587)         &    (0.0631)         &    (0.0975)         &    (0.0906)         &     (0.153)         &     (0.180)         &     (0.186)         &     (0.367)         &     (0.222)         \\
\addlinespace
Overweight  &      0.0288         &      0.0401         &      0.0473         &     -0.0328         &      0.0788         &       0.116         &       0.166         &       0.180         &      -0.185         &      0.0470         \\
            &    (0.0425)         &    (0.0429)         &    (0.0516)         &    (0.0904)         &    (0.0600)         &    (0.0904)         &    (0.0903)         &     (0.132)         &     (0.193)         &     (0.139)         \\
\addlinespace
Health is Good&    -0.00978         &     -0.0167         &      0.0161         &       0.132         &     -0.0558         &     -0.0982         &     -0.0784         &      0.0679         &      -0.139         &     -0.0253         \\
            &    (0.0612)         &    (0.0624)         &    (0.0676)         &     (0.115)         &    (0.0905)         &     (0.131)         &     (0.135)         &     (0.132)         &     (0.376)         &     (0.160)         \\
\addlinespace
Number of Sick Days&      -0.135         &      -0.154         &      -0.194         &     -0.0751         &     -0.0562         &      0.0280         &      -0.101         &     -0.0364         &       0.981         &      -0.211         \\
            &     (0.109)         &     (0.111)         &     (0.113)         &     (0.165)         &     (0.161)         &     (0.229)         &     (0.239)         &     (0.237)         &     (0.775)         &     (0.310)         \\
\bottomrule
\end{tabular}
}
}
\vspace{1ex} \\
\footnotesize\raggedright{Note: This table shows the estimates of the coefficient for attending Reggio Approach preschools from the OLS and difference-in-difference (DiD) models.Column title indicates the corresponding age group, control set, and model. ``NoneIt'' refers to the OLS estimate with only the Italian children cohort and with no control variables. ``BICIt'' refers to the OLS estimate with only the Italian children cohort and with controls selected by Bayesian Information Criterion (BIC) and additional controls for caregiver's religion. ``FullIt'' refers to the OLS estimate with only the Italian children cohort and with the full set of controls. ``DidPmIt'' refers to the difference-in-difference estimate of (Reggio Muni - Parma Muni) - (Reggio None - Parma None) for the Italian children cohort. ``DidPvIt'' refers to the difference-in-difference estimate of (Reggio Muni - Padova Muni) - (Reggio None - Padova None) for the Italian children cohort.  Analogous meanings are applied to the migrant children cohort in columns (7) through (12). Robust standard errors are reported in parentheses. Stars show statistical significance as follows: * $p < 0.05$, ** $p < 0.01$, *** $p < 0.001$.}
\end{table}

%Table \ref{ols-L-reg} shows significant effects of the Reggio Approach on the status of being divorced for the age-30 cohort. Other results do not show any significant trend. 

\begin{table}[H] \caption{OLS and Diff-in-Diff for Behaviors, Preschools, Children Cohort} \label{ols-L-reg}
\scalebox{0.80}{
{
\def\sym#1{\ifmmode^{#1}\else\(^{#1}\)\fi}
\begin{tabular}{l*{12}{c}}
\toprule
            &\multicolumn{1}{c}{(1)}&\multicolumn{1}{c}{(2)}&\multicolumn{1}{c}{(3)}&\multicolumn{1}{c}{(4)}&\multicolumn{1}{c}{(5)}&\multicolumn{1}{c}{(6)}&\multicolumn{1}{c}{(7)}&\multicolumn{1}{c}{(8)}&\multicolumn{1}{c}{(9)}&\multicolumn{1}{c}{(10)}&\multicolumn{1}{c}{(11)}&\multicolumn{1}{c}{(12)}\\
            &\multicolumn{1}{c}{NoneIt}&\multicolumn{1}{c}{BICIt}&\multicolumn{1}{c}{FullIt}&\multicolumn{1}{c}{DidPmIt}&\multicolumn{1}{c}{DidPvIt}&\multicolumn{1}{c}{IPWIt}&\multicolumn{1}{c}{NoneMg}&\multicolumn{1}{c}{BICMg}&\multicolumn{1}{c}{FullMg}&\multicolumn{1}{c}{DidPmMg}&\multicolumn{1}{c}{DidPvMg}&\multicolumn{1}{c}{IPWMg}\\
\midrule
Not Excited to Learn&     0.00236         &     0.00810         &     0.00371         &      0.0365         &     -0.0386         &     -0.0277         &      0.0980\sym{*}  &      0.0756         &      0.0928         &       0.559\sym{*}  &      0.0197         &     -0.0400         \\
            &    (0.0194)         &    (0.0193)         &    (0.0222)         &    (0.0367)         &    (0.0398)         &    (0.0169)         &    (0.0423)         &    (0.0426)         &    (0.0799)         &     (0.276)         &    (0.0626)         &    (0.0495)         \\
\addlinespace
Problems Sitting Still&     -0.0414         &     -0.0326         &     -0.0978\sym{*}  &     -0.0282         &      -0.155\sym{*}  &      0.0205         &       0.137\sym{**} &       0.181\sym{*}  &       0.203\sym{*}  &      0.0563         &       0.144         &      0.0748         \\
            &    (0.0477)         &    (0.0487)         &    (0.0492)         &    (0.0703)         &    (0.0688)         &    (0.0306)         &    (0.0489)         &    (0.0760)         &    (0.0809)         &     (0.132)         &     (0.122)         &    (0.0706)         \\
\addlinespace
How Much Child Likes School&       0.119         &       0.127         &       0.160         &       0.203         &       0.364\sym{**} &       0.145\sym{**} &      -0.399\sym{***}&      -0.299\sym{*}  &      -0.421\sym{*}  &      -0.101         &      -0.514\sym{**} &      -0.247\sym{*}  \\
            &    (0.0795)         &    (0.0793)         &    (0.0867)         &     (0.105)         &     (0.123)         &    (0.0554)         &     (0.118)         &     (0.123)         &     (0.171)         &     (0.167)         &     (0.192)         &     (0.106)         \\
\addlinespace
Happy in General&      0.0152         &      0.0179         &    -0.00711         &    -0.00953         &       0.290         &       0.187         &      -0.230         &     -0.0143         &      -0.555         &      -0.913         &       0.545         &      -0.314         \\
            &     (0.206)         &     (0.221)         &     (0.239)         &     (0.303)         &     (0.298)         &     (0.162)         &     (0.382)         &     (0.406)         &     (0.617)         &     (1.041)         &     (0.553)         &     (0.279)         \\
\bottomrule
\end{tabular}
}
}
\vspace{1ex} \\
\footnotesize\raggedright{Note: This table shows the estimates of the coefficient for attending Reggio Approach preschools from the OLS and difference-in-difference (DiD) models.Column title indicates the corresponding age group, control set, and model. ``NoneIt'' refers to the OLS estimate with only the Italian children cohort and with no control variables. ``BICIt'' refers to the OLS estimate with only the Italian children cohort and with controls selected by Bayesian Information Criterion (BIC) and additional controls for caregiver's religion. ``FullIt'' refers to the OLS estimate with only the Italian children cohort and with the full set of controls. ``DidPmIt'' refers to the difference-in-difference estimate of (Reggio Muni - Parma Muni) - (Reggio None - Parma None) for the Italian children cohort. ``DidPvIt'' refers to the difference-in-difference estimate of (Reggio Muni - Padova Muni) - (Reggio None - Padova None) for the Italian children cohort.  Analogous meanings are applied to the migrant children cohort in columns (7) through (12). Robust standard errors are reported in parentheses. Stars show statistical significance as follows: * $p < 0.05$, ** $p < 0.01$, *** $p < 0.001$.}
\end{table}

%Table \ref{ols-H-reg} show that the significantly increasing effects of the Reggio Approach on numbers of days sick for the past month at the time of the interview for the age-30 cohort. Moreover, the results show the significant decreasing effects on whether the respondent has ever been suspended from school. 

\begin{table}[H] \caption{OLS and Diff-in-Diff for Social Behaviors, Preschools, Children} \label{ols-H-reg}
\scalebox{0.77}{
{
\def\sym#1{\ifmmode^{#1}\else\(^{#1}\)\fi}
\begin{tabular}{l*{10}{c}}
\toprule
            &\multicolumn{1}{c}{(1)}&\multicolumn{1}{c}{(2)}&\multicolumn{1}{c}{(3)}&\multicolumn{1}{c}{(4)}&\multicolumn{1}{c}{(5)}&\multicolumn{1}{c}{(6)}&\multicolumn{1}{c}{(7)}&\multicolumn{1}{c}{(8)}&\multicolumn{1}{c}{(9)}&\multicolumn{1}{c}{(10)}\\
            &\multicolumn{1}{c}{NoneIt}&\multicolumn{1}{c}{BICIt}&\multicolumn{1}{c}{FullIt}&\multicolumn{1}{c}{DidPmIt}&\multicolumn{1}{c}{DidPvIt}&\multicolumn{1}{c}{NoneMg}&\multicolumn{1}{c}{BICMg}&\multicolumn{1}{c}{FullMg}&\multicolumn{1}{c}{DidPmMg}&\multicolumn{1}{c}{DidPvMg}\\
\midrule
Num. of Friends&      -0.240         &     -0.0240         &      0.0360         &       0.950         &      -0.196         &       0.148         &       0.628         &      0.0427         &      -1.590         &       1.696         \\
            &     (0.284)         &     (0.279)         &     (0.318)         &     (0.799)         &     (0.777)         &     (0.641)         &     (0.800)         &     (0.803)         &     (1.836)         &     (1.272)         \\
\addlinespace
Musical Instrument at Home&     -0.0410         &    -0.00662         &     0.00808         &      0.0242         &      -0.109         &      -0.181         &      -0.195         &      -0.264         &      -0.545\sym{**} &       0.112         \\
            &    (0.0652)         &    (0.0669)         &    (0.0696)         &    (0.0961)         &    (0.0945)         &     (0.141)         &     (0.140)         &     (0.148)         &     (0.170)         &     (0.196)         \\
\addlinespace
Tell Worry at Home&     -0.0354         &     0.00129         &      0.0159         &    -0.00309         &       0.106         &      0.0490         &       0.107         &      0.0800         &      -0.268         &       0.131         \\
            &    (0.0617)         &    (0.0631)         &    (0.0659)         &    (0.0886)         &    (0.0892)         &     (0.153)         &     (0.174)         &     (0.138)         &     (0.376)         &     (0.214)         \\
\addlinespace
Tell Worry to Teacher&      0.0439         &      0.0664         &      0.0692         &       0.128         &      0.0771         &      0.0532         &     0.00866         &      0.0826         &      -0.244         &      -0.273         \\
            &    (0.0602)         &    (0.0613)         &    (0.0623)         &    (0.0896)         &    (0.0837)         &     (0.111)         &     (0.114)         &     (0.145)         &     (0.321)         &     (0.163)         \\
\addlinespace
Tell Worry to Friends&      0.0466         &      0.0299         &      0.0513         &      0.0988         &    -0.00196         &      0.0140         &      0.0387         &      0.0359         &       0.557\sym{*}  &      -0.149         \\
            &    (0.0488)         &    (0.0502)         &    (0.0558)         &    (0.0750)         &    (0.0763)         &     (0.108)         &     (0.124)         &    (0.0758)         &     (0.270)         &     (0.147)         \\
\addlinespace
Keep Worry to Myself&     -0.0365         &     -0.0680         &     -0.0865         &     -0.0863         &     -0.0834         &      -0.122         &      -0.109         &      -0.175         &      -0.112         &      -0.138         \\
            &    (0.0465)         &    (0.0469)         &    (0.0523)         &    (0.0641)         &    (0.0634)         &     (0.143)         &     (0.158)         &     (0.139)         &     (0.169)         &     (0.188)         \\
\bottomrule
\end{tabular}
}
}
\vspace{1ex} \\
\footnotesize\raggedright{Note: This table shows the estimates of the coefficient for attending Reggio Approach preschools from the OLS and difference-in-difference (DiD) models.Column title indicates the corresponding age group, control set, and model. ``NoneIt'' refers to the OLS estimate with only the Italian children cohort and with no control variables. ``BICIt'' refers to the OLS estimate with only the Italian children cohort and with controls selected by Bayesian Information Criterion (BIC) and additional controls for caregiver's religion. ``FullIt'' refers to the OLS estimate with only the Italian children cohort and with the full set of controls. ``DidPmIt'' refers to the difference-in-difference estimate of (Reggio Muni - Parma Muni) - (Reggio None - Parma None) for the Italian children cohort. ``DidPvIt'' refers to the difference-in-difference estimate of (Reggio Muni - Padova Muni) - (Reggio None - Padova None) for the Italian children cohort.  Analogous meanings are applied to the migrant children cohort in columns (7) through (12). Robust standard errors are reported in parentheses. Stars show statistical significance as follows: * $p < 0.05$, ** $p < 0.01$, *** $p < 0.001$.}
\end{table}
\end{landscape}


\subsubsection{Adolescents}

\begin{table}[H] \caption{OLS and Diff-in-Diff for Cognitive and Noncognitive, Preschools, Adolescent Cohort} \label{ols-E-reg}
\scalebox{0.72}{{
\def\sym#1{\ifmmode^{#1}\else\(^{#1}\)\fi}
\begin{tabular}{l*{5}{c}}
\toprule
            &\multicolumn{1}{c}{(1)}&\multicolumn{1}{c}{(2)}&\multicolumn{1}{c}{(3)}&\multicolumn{1}{c}{(4)}&\multicolumn{1}{c}{(5)}\\
            &\multicolumn{1}{c}{None}&\multicolumn{1}{c}{BIC}&\multicolumn{1}{c}{Full}&\multicolumn{1}{c}{DidPm}&\multicolumn{1}{c}{DidPv}\\
\midrule
IQ Factor   &      -0.111         &      -0.109         &     -0.0519         &     -0.0696         &     -0.0239         \\
            &     (0.100)         &     (0.105)         &     (0.103)         &     (0.141)         &     (0.174)         \\
\addlinespace
IQ Score    &     -0.0279         &     -0.0233         &    -0.00383         &     -0.0171         &    -0.00504         \\
            &    (0.0307)         &    (0.0317)         &    (0.0311)         &    (0.0439)         &    (0.0508)         \\
\addlinespace
SDQ Composite - Child&       0.686         &       0.823         &       1.259         &     -0.0770         &       0.341         \\
            &     (0.691)         &     (0.709)         &     (0.746)         &     (1.033)         &     (0.896)         \\
\addlinespace
SDQ Pro-social - Child&      0.0938         &     -0.0283         &      -0.307         &       0.222         &      -0.122         \\
            &     (0.245)         &     (0.255)         &     (0.263)         &     (0.371)         &     (0.339)         \\
\addlinespace
SDQ Peer problems - Child&      0.0707         &      0.0884         &       0.210         &      -0.490         &      -0.125         \\
            &     (0.215)         &     (0.221)         &     (0.236)         &     (0.311)         &     (0.284)         \\
\addlinespace
SDQ Hyper - Child&       0.258         &       0.316         &       0.416         &       0.126         &       0.191         \\
            &     (0.250)         &     (0.245)         &     (0.257)         &     (0.397)         &     (0.349)         \\
\addlinespace
SDQ Emotional - Child&       0.243         &       0.206         &       0.272         &      0.0841         &       0.129         \\
            &     (0.276)         &     (0.281)         &     (0.304)         &     (0.421)         &     (0.366)         \\
\addlinespace
SDQ Conduct - Child&       0.114         &       0.212         &       0.360         &       0.203         &       0.136         \\
            &     (0.194)         &     (0.201)         &     (0.213)         &     (0.303)         &     (0.266)         \\
\addlinespace
SDQ Composite&       1.286         &       1.166         &       1.295         &       0.890         &       0.377         \\
            &     (0.685)         &     (0.731)         &     (0.780)         &     (1.017)         &     (1.019)         \\
\addlinespace
SDQ Pro-social&      0.0749         &     0.00454         &      -0.110         &     -0.0496         &      -0.450         \\
            &     (0.231)         &     (0.234)         &     (0.244)         &     (0.348)         &     (0.340)         \\
\addlinespace
SDQ Peer problems&     -0.0110         &    0.000395         &     -0.0138         &      -0.145         &    -0.00505         \\
            &     (0.196)         &     (0.204)         &     (0.209)         &     (0.275)         &     (0.301)         \\
\addlinespace
SDQ Hyper   &       0.471         &       0.475         &       0.499         &       0.284         &       0.245         \\
            &     (0.270)         &     (0.283)         &     (0.306)         &     (0.432)         &     (0.394)         \\
\addlinespace
SDQ Emotional&       0.319         &       0.192         &       0.201         &       0.161         &      -0.352         \\
            &     (0.293)         &     (0.304)         &     (0.326)         &     (0.437)         &     (0.419)         \\
\addlinespace
SDQ Conduct &       0.507\sym{*}  &       0.498\sym{*}  &       0.608\sym{**} &       0.590         &       0.489         \\
            &     (0.210)         &     (0.221)         &     (0.236)         &     (0.324)         &     (0.303)         \\
\addlinespace
Depression Score - positive&       1.628         &       1.525         &       1.870         &       0.679         &       1.170         \\
            &     (0.864)         &     (0.940)         &     (0.978)         &     (1.178)         &     (1.239)         \\
\bottomrule
\end{tabular}
}
}
\vspace{1ex} \\
\footnotesize\raggedright{Note: This table shows the estimates of the coefficient for attending Reggio Approach preschools from the OLS and difference-in-difference (DiD) models.Column title indicates the corresponding age group, control set, and model. ``None'' refers to the OLS estimate with only the Italian children cohort and with no control variables. ``BIC'' refers to the OLS estimate with only the Italian children cohort and with controls selected by Bayesian Information Criterion (BIC) and additional controls for caregiver's religion. ``Full'' refers to the OLS estimate with only the Italian children cohort and with the full set of controls. ``DidPm'' refers to the difference-in-difference estimate of (Reggio Muni - Parma Muni) - (Reggio None - Parma None) for the Italian children cohort. ``DidPv'' refers to the difference-in-difference estimate of (Reggio Muni - Padova Muni) - (Reggio None - Padova None) for the Italian children cohort.  Robust standard errors are reported in parentheses. Stars show statistical significance as follows: * $p < 0.05$, ** $p < 0.01$, *** $p < 0.001$.}
\end{table}

%Table \ref{ols-W-reg} shows consistently significant positive effects of the Reggio Approach on hours worked per week. Moreover, there are significant effects on the indicator for the lowest household income category for the age-40 cohort. Note that the measures for subject labor income, including ``Monthly Wage'' presented in the table, have very few observations. Household income suffers less from the missing value problem.


\begin{table}[H] \caption{OLS and Diff-in-Diff for Health and Risk, Preschools, Adolescent Cohort} \label{ols-W-reg}
\scalebox{0.76}{
{
\def\sym#1{\ifmmode^{#1}\else\(^{#1}\)\fi}
\begin{tabular}{l*{6}{c}}
\toprule
            &\multicolumn{1}{c}{(1)}&\multicolumn{1}{c}{(2)}&\multicolumn{1}{c}{(3)}&\multicolumn{1}{c}{(4)}&\multicolumn{1}{c}{(5)}&\multicolumn{1}{c}{(6)}\\
            &\multicolumn{1}{c}{None}&\multicolumn{1}{c}{BIC}&\multicolumn{1}{c}{Full}&\multicolumn{1}{c}{DidPm}&\multicolumn{1}{c}{DidPv}&\multicolumn{1}{c}{IPW}\\
\midrule
Obese       &       0.103\sym{*}  &       0.104\sym{*}  &      0.0882\sym{*}  &   -0.000598         &      0.0504         &     -0.0587         \\
            &    (0.0402)         &    (0.0425)         &    (0.0443)         &    (0.0614)         &    (0.0718)         &    (0.0396)         \\
\addlinespace
Overweight  &     0.00501         &      0.0131         &      0.0228         &     -0.0642         &      0.0288         &     0.00341         \\
            &    (0.0233)         &    (0.0250)         &    (0.0260)         &    (0.0385)         &    (0.0305)         &    (0.0222)         \\
\addlinespace
Health is Good&      0.0355         &      0.0626         &      0.0662         &       0.174         &      0.0717         &       0.115\sym{*}  \\
            &    (0.0609)         &    (0.0630)         &    (0.0648)         &    (0.0955)         &    (0.0902)         &    (0.0482)         \\
\addlinespace
Number of Sick Days&     -0.0202         &     -0.0470         &     -0.0414         &      -0.136         &       0.109         &     -0.0197         \\
            &     (0.110)         &     (0.108)         &     (0.118)         &     (0.147)         &     (0.147)         &    (0.0759)         \\
\addlinespace
Ever Suspended from School&      0.0205         &      0.0120         &      0.0176         &    -0.00596         &     -0.0261         &      0.0554\sym{*}  \\
            &    (0.0308)         &    (0.0351)         &    (0.0373)         &    (0.0433)         &    (0.0381)         &    (0.0249)         \\
\addlinespace
Num. of Cigarettes Per Day&       1.878         &       1.809         &       3.363         &       1.856         &      -1.715         &       2.319\sym{*}  \\
            &     (1.413)         &     (1.534)         &     (2.230)         &     (1.690)         &     (2.588)         &     (0.948)         \\
\bottomrule
\end{tabular}
}
}
\vspace{1ex} \\
\footnotesize\raggedright{Note: This table shows the estimates of the coefficient for attending Reggio Approach preschools from the OLS and difference-in-difference (DiD) models.Column title indicates the corresponding age group, control set, and model. ``None'' refers to the OLS estimate with only the Italian children cohort and with no control variables. ``BIC'' refers to the OLS estimate with only the Italian children cohort and with controls selected by Bayesian Information Criterion (BIC) and additional controls for caregiver's religion. ``Full'' refers to the OLS estimate with only the Italian children cohort and with the full set of controls. ``DidPm'' refers to the difference-in-difference estimate of (Reggio Muni - Parma Muni) - (Reggio None - Parma None) for the Italian children cohort. ``DidPv'' refers to the difference-in-difference estimate of (Reggio Muni - Padova Muni) - (Reggio None - Padova None) for the Italian children cohort.  Robust standard errors are reported in parentheses. Stars show statistical significance as follows: * $p < 0.05$, ** $p < 0.01$, *** $p < 0.001$.}
\end{table}

%Table \ref{ols-L-reg} shows significant effects of the Reggio Approach on the status of being divorced for the age-30 cohort. Other results do not show any significant trend. 

\begin{table}[H] \caption{OLS and Diff-in-Diff for Behaviors, Preschools, Adolescent Cohort} \label{ols-L-reg}
\scalebox{0.80}{
{
\def\sym#1{\ifmmode^{#1}\else\(^{#1}\)\fi}
\begin{tabular}{l*{5}{c}}
\toprule
            &\multicolumn{1}{c}{(1)}&\multicolumn{1}{c}{(2)}&\multicolumn{1}{c}{(3)}&\multicolumn{1}{c}{(4)}&\multicolumn{1}{c}{(5)}\\
            &\multicolumn{1}{c}{None}&\multicolumn{1}{c}{BIC}&\multicolumn{1}{c}{Full}&\multicolumn{1}{c}{DidPm}&\multicolumn{1}{c}{DidPv}\\
\midrule
Not Excited to Learn&     -0.0156         &    -0.00738         &     -0.0189         &     -0.0579         &    -0.00524         \\
            &    (0.0245)         &    (0.0229)         &    (0.0246)         &    (0.0378)         &    (0.0325)         \\
\addlinespace
Problems Sitting Still&     0.00307         &      0.0147         &      0.0119         &     -0.0317         &     0.00294         \\
            &    (0.0323)         &    (0.0334)         &    (0.0335)         &    (0.0792)         &    (0.0495)         \\
\addlinespace
Go To School&      0.0261         &      0.0192         &      0.0197         &     0.00208         &      0.0460         \\
            &    (0.0265)         &    (0.0259)         &    (0.0255)         &    (0.0342)         &    (0.0309)         \\
\addlinespace
How Much Child Likes School&     -0.0536         &     -0.0973         &     -0.0879         &      0.0299         &     -0.0496         \\
            &     (0.119)         &     (0.130)         &     (0.136)         &     (0.210)         &     (0.173)         \\
\addlinespace
Bothered by Migrants&       0.293\sym{**} &       0.229\sym{*}  &       0.251\sym{*}  &       0.278         &       0.151         \\
            &     (0.113)         &     (0.116)         &     (0.119)         &     (0.218)         &     (0.162)         \\
\addlinespace
Trust Score &     -0.0318         &      -0.120         &    -0.00929         &      0.0965         &      -0.102         \\
            &     (0.195)         &     (0.208)         &     (0.217)         &     (0.334)         &     (0.279)         \\
\addlinespace
Days of Sport (Weekly)&      -0.258         &      -0.125         &     -0.0526         &      -0.417         &      -0.265         \\
            &     (0.253)         &     (0.267)         &     (0.288)         &     (0.432)         &     (0.368)         \\
\bottomrule
\end{tabular}
}
}
\vspace{1ex} \\
\footnotesize\raggedright{Note: This table shows the estimates of the coefficient for attending Reggio Approach preschools from the OLS and difference-in-difference (DiD) models.Column title indicates the corresponding age group, control set, and model. ``None'' refers to the OLS estimate with only the Italian children cohort and with no control variables. ``BIC'' refers to the OLS estimate with only the Italian children cohort and with controls selected by Bayesian Information Criterion (BIC) and additional controls for caregiver's religion. ``Full'' refers to the OLS estimate with only the Italian children cohort and with the full set of controls. ``DidPm'' refers to the difference-in-difference estimate of (Reggio Muni - Parma Muni) - (Reggio None - Parma None) for the Italian children cohort. ``DidPv'' refers to the difference-in-difference estimate of (Reggio Muni - Padova Muni) - (Reggio None - Padova None) for the Italian children cohort.  Robust standard errors are reported in parentheses. Stars show statistical significance as follows: * $p < 0.05$, ** $p < 0.01$, *** $p < 0.001$.}
\end{table}

%Table \ref{ols-H-reg} show that the significantly increasing effects of the Reggio Approach on numbers of days sick for the past month at the time of the interview for the age-30 cohort. Moreover, the results show the significant decreasing effects on whether the respondent has ever been suspended from school. 

\begin{table}[H] \caption{OLS and Diff-in-Diff for Social Behaviors, Preschools, Adolescent Cohort} \label{ols-H-reg}
\scalebox{0.77}{
{
\def\sym#1{\ifmmode^{#1}\else\(^{#1}\)\fi}
\begin{tabular}{l*{6}{c}}
\toprule
            &\multicolumn{1}{c}{(1)}&\multicolumn{1}{c}{(2)}&\multicolumn{1}{c}{(3)}&\multicolumn{1}{c}{(4)}&\multicolumn{1}{c}{(5)}&\multicolumn{1}{c}{(6)}\\
            &\multicolumn{1}{c}{None}&\multicolumn{1}{c}{BIC}&\multicolumn{1}{c}{Full}&\multicolumn{1}{c}{DidPm}&\multicolumn{1}{c}{DidPv}&\multicolumn{1}{c}{IPW}\\
\midrule
Num. of Friends&       0.426         &       1.171         &       1.119         &       0.153         &       1.075         &      -0.784         \\
            &     (1.394)         &     (1.298)         &     (1.325)         &     (2.156)         &     (2.248)         &     (1.158)         \\
\addlinespace
Doesn't Talk About Activities&       0.128         &      0.0973         &      0.0555         &      0.0669         &      0.0769         &      -0.130         \\
            &    (0.0804)         &    (0.0843)         &    (0.0894)         &     (0.132)         &     (0.119)         &    (0.0692)         \\
\addlinespace
Doesn't Talk About School&     0.00575         &     -0.0243         &     -0.0344         &     -0.0912         &     0.00208         &     -0.0588         \\
            &    (0.0804)         &    (0.0813)         &    (0.0856)         &     (0.121)         &     (0.115)         &    (0.0585)         \\
\bottomrule
\end{tabular}
}
}
\vspace{1ex} \\
\footnotesize\raggedright{Note: This table shows the estimates of the coefficient for attending Reggio Approach preschools from the OLS and difference-in-difference (DiD) models.Column title indicates the corresponding age group, control set, and model. ``None'' refers to the OLS estimate with only the Italian children cohort and with no control variables. ``BIC'' refers to the OLS estimate with only the Italian children cohort and with controls selected by Bayesian Information Criterion (BIC) and additional controls for caregiver's religion. ``Full'' refers to the OLS estimate with only the Italian children cohort and with the full set of controls. ``DidPm'' refers to the difference-in-difference estimate of (Reggio Muni - Parma Muni) - (Reggio None - Parma None) for the Italian children cohort. ``DidPv'' refers to the difference-in-difference estimate of (Reggio Muni - Padova Muni) - (Reggio None - Padova None) for the Italian children cohort.  Robust standard errors are reported in parentheses. Stars show statistical significance as follows: * $p < 0.05$, ** $p < 0.01$, *** $p < 0.001$.}
\end{table}









\subsubsection{Adults}
\begin{landscape}

\begin{table}[H] \caption{OLS and Diff-in-Diff for Cognitive and Education, Preschools, Adult Cohorts} \label{ols-E-reg}
\scalebox{0.80}{{
\def\sym#1{\ifmmode^{#1}\else\(^{#1}\)\fi}
\begin{tabular}{l*{10}{c}}
\toprule
            &\multicolumn{1}{c}{(1)}&\multicolumn{1}{c}{(2)}&\multicolumn{1}{c}{(3)}&\multicolumn{1}{c}{(4)}&\multicolumn{1}{c}{(5)}&\multicolumn{1}{c}{(6)}&\multicolumn{1}{c}{(7)}&\multicolumn{1}{c}{(8)}&\multicolumn{1}{c}{(9)}&\multicolumn{1}{c}{(10)}\\
            &\multicolumn{1}{c}{None30}&\multicolumn{1}{c}{BIC30}&\multicolumn{1}{c}{Full30}&\multicolumn{1}{c}{DidPm30}&\multicolumn{1}{c}{DidPv30}&\multicolumn{1}{c}{None40}&\multicolumn{1}{c}{BIC40}&\multicolumn{1}{c}{Full40}&\multicolumn{1}{c}{DidPm40}&\multicolumn{1}{c}{DidPv40}\\
\midrule
IQ Factor   &      -0.568\sym{***}&      -0.582\sym{***}&      -0.578\sym{**} &      -0.878\sym{***}&      -0.459\sym{*}  &      -0.287\sym{*}  &      -0.255\sym{*}  &      -0.251\sym{*}  &      -0.294         &      0.0708         \\
            &     (0.155)         &     (0.172)         &     (0.194)         &     (0.190)         &     (0.210)         &     (0.122)         &     (0.122)         &     (0.126)         &     (0.168)         &     (0.210)         \\
\addlinespace
High School Grade&       4.264\sym{**} &       3.353\sym{*}  &       3.439         &       4.114         &       4.690         &       0.453         &       1.292         &       1.102         &       4.263         &       3.961         \\
            &     (1.611)         &     (1.643)         &     (1.862)         &     (3.145)         &     (3.552)         &     (1.720)         &     (1.792)         &     (1.879)         &     (3.459)         &     (3.168)         \\
\addlinespace
University Grade&       9.368\sym{***}&       8.591\sym{**} &       5.624         &       11.24\sym{**} &       10.05\sym{**} &      -5.082\sym{*}  &      -2.750         &      -1.847         &       1.199         &      -4.124         \\
            &     (2.121)         &     (2.917)         &     (4.510)         &     (3.790)         &     (3.162)         &     (2.104)         &     (1.930)         &     (2.750)         &     (2.793)         &     (3.535)         \\
\addlinespace
Graduate from High School&     -0.0226         &    -0.00941         &     -0.0242         &      0.0680         &     -0.0823         &      0.0486         &      0.0338         &      0.0519         &      0.0433         &     -0.0424         \\
            &    (0.0648)         &    (0.0641)         &    (0.0713)         &    (0.0831)         &    (0.0824)         &    (0.0762)         &    (0.0766)         &    (0.0782)         &    (0.0937)         &     (0.113)         \\
\addlinespace
Max Edu: University&       0.107         &      0.0963         &      0.0769         &       0.243\sym{*}  &       0.241         &      0.0745         &      0.0539         &      0.0339         &     -0.0773         &      0.0801         \\
            &    (0.0691)         &    (0.0705)         &    (0.0890)         &     (0.119)         &     (0.124)         &    (0.0617)         &    (0.0590)         &    (0.0544)         &     (0.115)         &     (0.120)         \\
\addlinespace
Max Edu: Graduate School&           0         &           0         &           0         &    -0.00509         &       0.141\sym{***}&           0         &           0         &           0         &     -0.0390         &     0.00514         \\
            &         (.)         &         (.)         &         (.)         &    (0.0322)         &    (0.0350)         &         (.)         &         (.)         &         (.)         &    (0.0558)         &    (0.0444)         \\
\bottomrule
\end{tabular}
}
}
\vspace{1ex} \\
\footnotesize\raggedright{Note: This table shows the estimates of the coefficient for attending Reggio Approach preschools from the OLS and difference-in-difference (DiD) models.Column title indicates the corresponding age group, control set, and model. ``None30'' refers to the OLS estimate with only the age-30 cohort and with no control variables. ``BIC30'' refers to the OLS estimate with only the age-30 cohort and with controls selected by Bayesian Information Criterion (BIC). ``Full30'' refers to the OLS estimate with only the age-30 cohort and with the full set of controls. ``DidPm30'' refers to the difference-in-difference estimate of (Reggio Muni - Parma Muni) - (Reggio None - Parma None) for the age-30 cohort. ``DidPv30'' refers to the difference-in-difference estimate of (Reggio Muni - Padova Muni) - (Reggio None - Padova None) for the age-30 cohort.  Analogous meanings are applied to the age-40 cohort in columns (7) through (12). Robust standard errors are reported in parentheses. Stars show statistical significance as follows: * $p < 0.05$, ** $p < 0.01$, *** $p < 0.001$.}
\end{table}

%Table \ref{ols-W-reg} shows consistently significant positive effects of the Reggio Approach on hours worked per week. Moreover, there are significant effects on the indicator for the lowest household income category for the age-40 cohort. Note that the measures for subject labor income, including ``Monthly Wage'' presented in the table, have very few observations. Household income suffers less from the missing value problem.

\begin{table}[H] \caption{OLS and Diff-in-Diff for Employment and Income, Preschools, Adult Cohorts} \label{ols-W-reg}
\scalebox{0.76}{
{
\def\sym#1{\ifmmode^{#1}\else\(^{#1}\)\fi}
\begin{tabular}{l*{12}{c}}
\toprule
            &\multicolumn{1}{c}{(1)}&\multicolumn{1}{c}{(2)}&\multicolumn{1}{c}{(3)}&\multicolumn{1}{c}{(4)}&\multicolumn{1}{c}{(5)}&\multicolumn{1}{c}{(6)}&\multicolumn{1}{c}{(7)}&\multicolumn{1}{c}{(8)}&\multicolumn{1}{c}{(9)}&\multicolumn{1}{c}{(10)}&\multicolumn{1}{c}{(11)}&\multicolumn{1}{c}{(12)}\\
            &\multicolumn{1}{c}{None30}&\multicolumn{1}{c}{BIC30}&\multicolumn{1}{c}{Full30}&\multicolumn{1}{c}{DidPm30}&\multicolumn{1}{c}{DidPv30}&\multicolumn{1}{c}{PSM30}&\multicolumn{1}{c}{None40}&\multicolumn{1}{c}{BIC40}&\multicolumn{1}{c}{Full40}&\multicolumn{1}{c}{DidPm40}&\multicolumn{1}{c}{DidPv40}&\multicolumn{1}{c}{PSM40}\\
\midrule
Employed    &     -0.0526\sym{*}  &     -0.0581\sym{*}  &     -0.0289         &      0.0493         &      -0.118         &      0.0292         &    -0.00871         &     0.00343         &    -0.00444         &     0.00424         &     -0.0615         &     -0.0119         \\
            &    (0.0231)         &    (0.0270)         &    (0.0224)         &    (0.0649)         &    (0.0736)         &    (0.0298)         &    (0.0283)         &    (0.0336)         &    (0.0328)         &    (0.0466)         &    (0.0781)         &    (0.0247)         \\
\addlinespace
Self-Employed&     -0.0450         &     -0.0657         &     -0.0545         &      -0.104         &     0.00641         &     0.00227         &     -0.0501         &     -0.0574         &     -0.0797         &     -0.0600         &      0.0670         &      0.0362         \\
            &    (0.0673)         &    (0.0659)         &    (0.0729)         &    (0.0788)         &    (0.0848)         &    (0.0298)         &    (0.0671)         &    (0.0658)         &    (0.0638)         &    (0.0988)         &    (0.0743)         &    (0.0307)         \\
\addlinespace
Hours Worked Per Week&       1.871         &       1.741         &       0.910         &       0.243         &       1.991         &       1.009         &      -1.118         &      -1.961         &      -2.109         &      -1.611         &       0.444         &       16.00\sym{***}\\
            &     (1.092)         &     (1.138)         &     (1.311)         &     (2.312)         &     (1.811)         &     (1.000)         &     (1.390)         &     (1.516)         &     (1.550)         &     (1.781)         &     (3.162)         &     (1.787)         \\
\addlinespace
Monthly Wage&       61.49         &      -49.38         &       72.18         &      -221.7         &       209.6         &      -238.4         &       203.6         &      -54.99         &      -83.75         &     -1345.5         &      1051.2         &       957.3         \\
            &     (127.7)         &     (99.05)         &     (136.1)         &     (379.3)         &     (164.7)         &     (135.1)         &     (565.7)         &     (434.5)         &     (475.6)         &     (971.1)         &     (840.7)         &     (518.5)         \\
\addlinespace
Income: 5,000 Euros of Less&     -0.0301         &     -0.0219         &     -0.0472         &     -0.0180         &     -0.0475         &      0.0516\sym{*}  &           0         &           0         &           0         &           0         &           0         &     0.00356         \\
            &    (0.0613)         &    (0.0562)         &    (0.0570)         &    (0.0567)         &    (0.0749)         &    (0.0250)         &         (.)         &         (.)         &         (.)         &         (.)         &         (.)         &   (0.00431)         \\
\addlinespace
Income: 5,001-10,000 Euros&      0.0105         &      0.0179         &      0.0119         &      0.0147         &      0.0336         &     0.00315         &           0         &           0         &           0         &           0         &     -0.0712         &     -0.0102         \\
            &    (0.0106)         &    (0.0179)         &    (0.0135)         &    (0.0149)         &    (0.0204)         &   (0.00929)         &         (.)         &         (.)         &         (.)         &         (.)         &    (0.0528)         &   (0.00627)         \\
\addlinespace
Income: 10,001-25,000 Euros&     -0.0406         &     -0.0352         &     -0.0285         &      -0.126         &     -0.0251         &      -0.193\sym{***}&     -0.0176         &     -0.0116         &    -0.00530         &      0.0158         &      -0.131         &      -0.146         \\
            &    (0.0915)         &    (0.0913)         &     (0.104)         &     (0.140)         &     (0.139)         &    (0.0545)         &    (0.0774)         &    (0.0783)         &    (0.0830)         &     (0.121)         &     (0.141)         &     (0.181)         \\
\addlinespace
Income: 25,001-50,000 Euros&    -0.00301         &     -0.0372         &     -0.0146         &      0.0881         &     -0.0936         &      0.0835         &       0.126         &       0.138         &       0.133         &       0.253         &       0.199         &      0.0414         \\
            &    (0.0986)         &    (0.0974)         &     (0.110)         &     (0.145)         &     (0.144)         &    (0.0595)         &    (0.0892)         &    (0.0931)         &    (0.0980)         &     (0.139)         &     (0.151)         &     (0.186)         \\
\addlinespace
Income: 50,001-100,000 Euros&      0.0632\sym{*}  &      0.0764\sym{*}  &      0.0784\sym{*}  &     0.00851         &       0.123\sym{*}  &      0.0600         &     -0.0731         &     -0.0915         &     -0.0891         &      -0.162\sym{*}  &      0.0111         &      0.0811         \\
            &    (0.0252)         &    (0.0311)         &    (0.0361)         &    (0.0549)         &    (0.0541)         &    (0.0389)         &    (0.0613)         &    (0.0608)         &    (0.0627)         &    (0.0757)         &    (0.0862)         &    (0.0634)         \\
\addlinespace
Income: 100,001-250,000 Euros&           0         &           0         &           0         &      0.0325         &     0.00916         &    -0.00540         &     -0.0357         &     -0.0353         &     -0.0383         &      -0.107         &    -0.00853         &      0.0306         \\
            &         (.)         &         (.)         &         (.)         &    (0.0320)         &   (0.00941)         &   (0.00388)         &    (0.0414)         &    (0.0429)         &    (0.0401)         &    (0.0583)         &    (0.0422)         &    (0.0270)         \\
\addlinespace
Income: More than 250,000 Euros&           0         &           0         &           0         &           0         &           0         &           0         &           0         &           0         &           0         &           0         &           0         &           0         \\
            &         (.)         &         (.)         &         (.)         &         (.)         &         (.)         &         (.)         &         (.)         &         (.)         &         (.)         &         (.)         &         (.)         &         (.)         \\
\bottomrule
\end{tabular}
}
}
\vspace{1ex} \\
\footnotesize\raggedright{Note: This table shows the estimates of the coefficient for attending Reggio Approach preschools from the OLS and difference-in-difference (DiD) models.Column title indicates the corresponding age group, control set, and model. ``None30'' refers to the OLS estimate with only the age-30 cohort and with no control variables. ``BIC30'' refers to the OLS estimate with only the age-30 cohort and with controls selected by Bayesian Information Criterion (BIC). ``Full30'' refers to the OLS estimate with only the age-30 cohort and with the full set of controls. ``DidPm30'' refers to the difference-in-difference estimate of (Reggio Muni - Parma Muni) - (Reggio None - Parma None) for the age-30 cohort. ``DidPv30'' refers to the difference-in-difference estimate of (Reggio Muni - Padova Muni) - (Reggio None - Padova None) for the age-30 cohort.  Analogous meanings are applied to the age-40 cohort in columns (7) through (12). Robust standard errors are reported in parentheses. Stars show statistical significance as follows: * $p < 0.05$, ** $p < 0.01$, *** $p < 0.001$.}
\end{table}

%Table \ref{ols-L-reg} shows significant effects of the Reggio Approach on the status of being divorced for the age-30 cohort. Other results do not show any significant trend. 

\begin{table}[H] \caption{OLS and Diff-in-Diff for Living Environment, Preschools, Adult Cohorts} \label{ols-L-reg}
\scalebox{0.80}{
{
\def\sym#1{\ifmmode^{#1}\else\(^{#1}\)\fi}
\begin{tabular}{l*{12}{c}}
\toprule
            &\multicolumn{1}{c}{(1)}&\multicolumn{1}{c}{(2)}&\multicolumn{1}{c}{(3)}&\multicolumn{1}{c}{(4)}&\multicolumn{1}{c}{(5)}&\multicolumn{1}{c}{(6)}&\multicolumn{1}{c}{(7)}&\multicolumn{1}{c}{(8)}&\multicolumn{1}{c}{(9)}&\multicolumn{1}{c}{(10)}&\multicolumn{1}{c}{(11)}&\multicolumn{1}{c}{(12)}\\
            &\multicolumn{1}{c}{None30}&\multicolumn{1}{c}{BIC30}&\multicolumn{1}{c}{Full30}&\multicolumn{1}{c}{DidPm30}&\multicolumn{1}{c}{DidPv30}&\multicolumn{1}{c}{PSM30}&\multicolumn{1}{c}{None40}&\multicolumn{1}{c}{BIC40}&\multicolumn{1}{c}{Full40}&\multicolumn{1}{c}{DidPm40}&\multicolumn{1}{c}{DidPv40}&\multicolumn{1}{c}{PSM40}\\
\midrule
Married or Cohabitating&     0.00451         &     0.00365         &      0.0351         &      0.0851         &       0.201         &      -0.128\sym{*}  &      0.0176         &      0.0180         &      0.0259         &     -0.0772         &       0.188         &       0.475\sym{***}\\
            &    (0.0947)         &     (0.101)         &     (0.112)         &     (0.139)         &     (0.138)         &    (0.0523)         &    (0.0774)         &    (0.0741)         &    (0.0786)         &     (0.127)         &     (0.138)         &    (0.0641)         \\
\addlinespace
Divorced    &     -0.0571         &     -0.0457         &     -0.0489         &     -0.0663         &     -0.0408         &     -0.0113         &     0.00550         &      0.0293         &      0.0287         &      -0.111         &      -0.101         &      0.0327         \\
            &    (0.0395)         &    (0.0313)         &    (0.0344)         &    (0.0382)         &    (0.0377)         &   (0.00591)         &    (0.0459)         &    (0.0493)         &    (0.0546)         &    (0.0834)         &    (0.0899)         &    (0.0357)         \\
\addlinespace
Num. of Children in House&     0.00902         &      0.0271         &    -0.00198         &      0.0185         &       0.312\sym{*}  &      -0.132\sym{**} &      -0.211         &      -0.196         &      -0.215         &      -0.311         &     -0.0789         &       0.235\sym{**} \\
            &    (0.0622)         &    (0.0623)         &    (0.0806)         &     (0.118)         &     (0.122)         &    (0.0466)         &     (0.124)         &     (0.115)         &     (0.112)         &     (0.222)         &     (0.246)         &    (0.0872)         \\
\addlinespace
Own House   &       0.143         &       0.141         &       0.120         &      0.0924         &       0.317\sym{*}  &      0.0448         &     -0.0204         &     -0.0655         &     -0.0637         &      -0.129         &      -0.108         &       0.316\sym{***}\\
            &    (0.0988)         &    (0.0978)         &     (0.114)         &     (0.143)         &     (0.144)         &    (0.0543)         &    (0.0829)         &    (0.0804)         &    (0.0819)         &     (0.117)         &     (0.115)         &    (0.0718)         \\
\addlinespace
Live With Parents&      -0.141         &      -0.145         &      -0.112         &      -0.331\sym{**} &      -0.323\sym{*}  &     -0.0231         &     -0.0238         &     -0.0256         &     -0.0371         &     0.00363         &    -0.00208         &      -0.545\sym{***}\\
            &    (0.0815)         &    (0.0760)         &    (0.0797)         &     (0.115)         &     (0.128)         &    (0.0479)         &    (0.0342)         &    (0.0337)         &    (0.0397)         &    (0.0654)         &    (0.0969)         &    (0.0439)         \\
\bottomrule
\end{tabular}
}
}
\vspace{1ex} \\
\footnotesize\raggedright{Note: This table shows the estimates of the coefficient for attending Reggio Approach preschools from the OLS and difference-in-difference (DiD) models.Column title indicates the corresponding age group, control set, and model. ``None30'' refers to the OLS estimate with only the age-30 cohort and with no control variables. ``BIC30'' refers to the OLS estimate with only the age-30 cohort and with controls selected by Bayesian Information Criterion (BIC). ``Full30'' refers to the OLS estimate with only the age-30 cohort and with the full set of controls. ``DidPm30'' refers to the difference-in-difference estimate of (Reggio Muni - Parma Muni) - (Reggio None - Parma None) for the age-30 cohort. ``DidPv30'' refers to the difference-in-difference estimate of (Reggio Muni - Padova Muni) - (Reggio None - Padova None) for the age-30 cohort.  Analogous meanings are applied to the age-40 cohort in columns (7) through (12). Robust standard errors are reported in parentheses. Stars show statistical significance as follows: * $p < 0.05$, ** $p < 0.01$, *** $p < 0.001$.}
\end{table}

%Table \ref{ols-H-reg} show that the significantly increasing effects of the Reggio Approach on numbers of days sick for the past month at the time of the interview for the age-30 cohort. Moreover, the results show the significant decreasing effects on whether the respondent has ever been suspended from school. 

\begin{table}[H] \caption{OLS and Diff-in-Diff for Health and Risk, Preschools, Adult Cohorts} \label{ols-H-reg}
\scalebox{0.77}{
{
\def\sym#1{\ifmmode^{#1}\else\(^{#1}\)\fi}
\begin{tabular}{l*{12}{c}}
\toprule
            &\multicolumn{1}{c}{(1)}&\multicolumn{1}{c}{(2)}&\multicolumn{1}{c}{(3)}&\multicolumn{1}{c}{(4)}&\multicolumn{1}{c}{(5)}&\multicolumn{1}{c}{(6)}&\multicolumn{1}{c}{(7)}&\multicolumn{1}{c}{(8)}&\multicolumn{1}{c}{(9)}&\multicolumn{1}{c}{(10)}&\multicolumn{1}{c}{(11)}&\multicolumn{1}{c}{(12)}\\
            &\multicolumn{1}{c}{None30}&\multicolumn{1}{c}{BIC30}&\multicolumn{1}{c}{Full30}&\multicolumn{1}{c}{DidPm30}&\multicolumn{1}{c}{DidPv30}&\multicolumn{1}{c}{IPW30}&\multicolumn{1}{c}{None40}&\multicolumn{1}{c}{BIC40}&\multicolumn{1}{c}{Full40}&\multicolumn{1}{c}{DidPm40}&\multicolumn{1}{c}{DidPv40}&\multicolumn{1}{c}{IPW40}\\
\midrule
Tried Marijuana&     -0.0782         &     -0.0781         &     -0.0729         &      0.0205         &      -0.102         &     -0.0501         &      0.0880         &      0.0780         &      0.0682         &      0.0310         &      0.0418         &      0.0852\sym{*}  \\
            &    (0.0843)         &    (0.0862)         &    (0.0783)         &     (0.114)         &     (0.119)         &    (0.0403)         &    (0.0526)         &    (0.0551)         &    (0.0540)         &    (0.0802)         &    (0.0921)         &    (0.0390)         \\
\addlinespace
Num. of Cigarettes Per Day&     -0.0551         &       0.987         &       2.104         &       0.362         &       3.168         &       4.952\sym{***}&       3.876\sym{*}  &       4.264\sym{*}  &       3.193         &       3.562         &       7.298\sym{**} &       4.541\sym{**} \\
            &     (1.341)         &     (1.409)         &     (1.591)         &     (3.079)         &     (3.867)         &     (1.133)         &     (1.705)         &     (1.977)         &     (1.852)         &     (3.085)         &     (2.533)         &     (1.602)         \\
\addlinespace
BMI         &     -0.0178         &      -0.111         &     -0.0647         &      -1.776\sym{**} &       0.390         &      0.0863         &       0.295         &       0.371         &       0.152         &       1.033         &      -0.386         &     -0.0117         \\
            &     (0.425)         &     (0.408)         &     (0.449)         &     (0.661)         &     (0.721)         &     (0.284)         &     (0.486)         &     (0.504)         &     (0.521)         &     (0.802)         &     (0.932)         &     (0.460)         \\
\addlinespace
Obese       &       0.110         &       0.136         &      0.0908         &       0.192         &       0.116         &       0.158\sym{**} &      0.0784         &      0.0737         &      0.0290         &      -0.115         &      -0.170         &      -0.309\sym{***}\\
            &    (0.0746)         &    (0.0744)         &    (0.0855)         &     (0.111)         &     (0.117)         &    (0.0494)         &    (0.0755)         &    (0.0717)         &    (0.0766)         &     (0.121)         &     (0.129)         &    (0.0746)         \\
\addlinespace
Overweight  &    -0.00752         &     -0.0172         &     0.00620         &      -0.174         &      0.0104         &   -0.000595         &     -0.0344         &     -0.0360         &     -0.0264         &       0.133         &      0.0508         &      0.0382         \\
            &    (0.0834)         &    (0.0771)         &    (0.0836)         &     (0.113)         &     (0.111)         &    (0.0431)         &    (0.0849)         &    (0.0813)         &    (0.0818)         &     (0.127)         &     (0.119)         &    (0.0507)         \\
\addlinespace
Good Health &      -0.108         &     -0.0808         &     -0.0838         &     -0.0860         &     -0.0681         &       0.310\sym{***}&      -0.171         &      -0.167         &      -0.168         &      -0.151         &      -0.499\sym{*}  &       0.570         \\
            &     (0.110)         &    (0.0986)         &     (0.118)         &     (0.158)         &     (0.174)         &    (0.0624)         &     (0.103)         &     (0.109)         &     (0.112)         &     (0.177)         &     (0.223)         &     (0.352)         \\
\addlinespace
No Problematic Health Condition&      0.0894         &      0.0899         &     -0.0113         &       0.129         &     -0.0234         &     -0.0615         &       0.118         &      0.0984         &      0.0871         &       0.231         &    -0.00182         &       0.114         \\
            &     (0.101)         &     (0.102)         &     (0.106)         &     (0.144)         &     (0.152)         &    (0.0603)         &    (0.0931)         &    (0.0971)         &    (0.0992)         &     (0.146)         &     (0.150)         &    (0.0721)         \\
\addlinespace
Num. of Days Sick Past Month&       0.127         &       0.111         &       0.251\sym{*}  &       0.219         &       0.186         &       0.224\sym{***}&      0.0498         &      0.0788         &      0.0805         &      0.0834         &      0.0336         &      0.0507         \\
            &     (0.102)         &    (0.0940)         &     (0.102)         &     (0.141)         &     (0.131)         &    (0.0660)         &    (0.0542)         &    (0.0557)         &    (0.0534)         &    (0.0801)         &     (0.114)         &    (0.0384)         \\
\addlinespace
Ever Suspended from School&     -0.0150         &     -0.0224         &     -0.0493         &      0.0106         &      -0.109         &     -0.0137         &     -0.0151         &     -0.0371         &     -0.0137         &      -0.137         &   -0.000534         &      0.0235         \\
            &    (0.0447)         &    (0.0476)         &    (0.0545)         &    (0.0686)         &    (0.0794)         &    (0.0227)         &    (0.0437)         &    (0.0413)         &    (0.0453)         &    (0.0744)         &    (0.0523)         &    (0.0332)         \\
\addlinespace
Age At First Drink&      -2.958\sym{*}  &      -3.309\sym{*}  &      -2.179         &      -5.500\sym{**} &      -5.351\sym{**} &      -1.674         &      -1.932         &      -2.191         &      -2.123         &      -2.679         &      -2.745         &       3.204\sym{*}  \\
            &     (1.382)         &     (1.410)         &     (1.704)         &     (1.967)         &     (2.050)         &     (0.894)         &     (1.414)         &     (1.387)         &     (1.423)         &     (1.819)         &     (2.166)         &     (1.323)         \\
\bottomrule
\end{tabular}
}
}
\vspace{1ex} \\
\footnotesize\raggedright{Note: This table shows the estimates of the coefficient for attending Reggio Approach preschools from the OLS and difference-in-difference (DiD) models.Column title indicates the corresponding age group, control set, and model. ``None30'' refers to the OLS estimate with only the age-30 cohort and with no control variables. ``BIC30'' refers to the OLS estimate with only the age-30 cohort and with controls selected by Bayesian Information Criterion (BIC). ``Full30'' refers to the OLS estimate with only the age-30 cohort and with the full set of controls. ``DidPm30'' refers to the difference-in-difference estimate of (Reggio Muni - Parma Muni) - (Reggio None - Parma None) for the age-30 cohort. ``DidPv30'' refers to the difference-in-difference estimate of (Reggio Muni - Padova Muni) - (Reggio None - Padova None) for the age-30 cohort.  Analogous meanings are applied to the age-40 cohort in columns (7) through (12). Robust standard errors are reported in parentheses. Stars show statistical significance as follows: * $p < 0.05$, ** $p < 0.01$, *** $p < 0.001$.}
\end{table}

%Table \ref{ols-N-reg} shows no consistently significant trends between the age-30 cohort and age-40 cohort. For the age-30 cohort, there are significantly decreasing effects on the optimistic look in life. Moreover, there are significantly increasing effects on whether the respondent is likely to do the same to someone who puts him/her in a difficult situation and to someone who insults him/her.

%For the age-40 cohort, results show the significantly decreasing effects on depression for the age-30 cohort. There are also significantly positive effects on satisfaction with income and with work for the age-40 cohort. However, there are significantly positive effects on whether the respondent thinks work is source of stress. 

\begin{table}[H] \caption{OLS and Diff-in-Diff for Non-cognitive, Preschools, Adult Cohorts} \label{ols-N-reg}
\scalebox{0.80}{
{
\def\sym#1{\ifmmode^{#1}\else\(^{#1}\)\fi}
\begin{tabular}{l*{12}{c}}
\toprule
            &\multicolumn{1}{c}{(1)}&\multicolumn{1}{c}{(2)}&\multicolumn{1}{c}{(3)}&\multicolumn{1}{c}{(4)}&\multicolumn{1}{c}{(5)}&\multicolumn{1}{c}{(6)}&\multicolumn{1}{c}{(7)}&\multicolumn{1}{c}{(8)}&\multicolumn{1}{c}{(9)}&\multicolumn{1}{c}{(10)}&\multicolumn{1}{c}{(11)}&\multicolumn{1}{c}{(12)}\\
            &\multicolumn{1}{c}{None30}&\multicolumn{1}{c}{BIC30}&\multicolumn{1}{c}{Full30}&\multicolumn{1}{c}{DidPm30}&\multicolumn{1}{c}{DidPv30}&\multicolumn{1}{c}{IPW30}&\multicolumn{1}{c}{None40}&\multicolumn{1}{c}{BIC40}&\multicolumn{1}{c}{Full40}&\multicolumn{1}{c}{DidPm40}&\multicolumn{1}{c}{DidPv40}&\multicolumn{1}{c}{IPW40}\\
\midrule
Locus of Control - positive&     -0.0995         &     -0.0817         &     -0.0996         &       0.226         &       0.284         &      -0.140         &     -0.0143         &     -0.0318         &     -0.0668         &      -0.128         &       0.452         &       0.640\sym{***}\\
            &     (0.152)         &     (0.141)         &     (0.166)         &     (0.230)         &     (0.231)         &    (0.0941)         &     (0.139)         &     (0.148)         &     (0.157)         &     (0.238)         &     (0.231)         &     (0.128)         \\
\addlinespace
Depression Score - positive&      -1.499         &      -1.536         &      -1.236         &      -1.224         &      -0.706         &      -2.452\sym{***}&       0.234         &       0.738         &       0.441         &       1.562         &       3.009         &      -1.528         \\
            &     (1.092)         &     (0.829)         &     (0.884)         &     (1.239)         &     (1.540)         &     (0.575)         &     (0.923)         &     (0.926)         &     (1.000)         &     (1.490)         &     (1.652)         &     (0.813)         \\
\addlinespace
Stress      &      -0.232         &      -0.227\sym{*}  &      -0.272\sym{*}  &      -0.507\sym{**} &      -0.594\sym{***}&      0.0375         &       0.132         &       0.153         &       0.133         &       0.360         &       0.378         &       0.256         \\
            &     (0.130)         &     (0.115)         &     (0.129)         &     (0.193)         &     (0.177)         &    (0.0796)         &     (0.122)         &     (0.130)         &     (0.134)         &     (0.198)         &     (0.199)         &     (0.221)         \\
\addlinespace
Work is Source of Stress&       0.180         &       0.261         &       0.262         &       0.498\sym{*}  &       0.250         &      0.0908         &       0.324\sym{**} &       0.268\sym{*}  &       0.239         &       0.292         &       0.306         &      0.0247         \\
            &     (0.165)         &     (0.183)         &     (0.206)         &     (0.209)         &     (0.243)         &    (0.0779)         &     (0.108)         &     (0.120)         &     (0.125)         &     (0.174)         &     (0.191)         &    (0.0812)         \\
\addlinespace
Satisfied with Income&       0.119         &       0.117         &      0.0561         &       0.572\sym{**} &       0.144         &       0.345\sym{***}&       0.229         &       0.203         &       0.214         &       0.352         &       0.369         &      0.0260         \\
            &     (0.155)         &     (0.145)         &     (0.167)         &     (0.219)         &     (0.249)         &    (0.0986)         &     (0.151)         &     (0.150)         &     (0.156)         &     (0.236)         &     (0.279)         &     (0.215)         \\
\addlinespace
Satisfied with Work&      0.0526         &      0.0307         &     -0.0191         &       0.324         &      0.1000         &       0.215         &       0.198         &       0.238         &       0.211         &       0.246         &     0.00972         &       0.136         \\
            &     (0.112)         &     (0.107)         &     (0.133)         &     (0.237)         &     (0.220)         &     (0.110)         &     (0.138)         &     (0.150)         &     (0.150)         &     (0.222)         &     (0.282)         &     (0.147)         \\
\addlinespace
Satisfied with Health&      -0.340\sym{**} &      -0.317\sym{**} &      -0.407\sym{**} &      -0.401\sym{*}  &      -0.482\sym{**} &     -0.0438         &     -0.0866         &     -0.0624         &     -0.0730         &     -0.0777         &      -0.128         &       0.103         \\
            &     (0.126)         &     (0.116)         &     (0.139)         &     (0.189)         &     (0.168)         &    (0.0825)         &    (0.0962)         &    (0.0958)         &    (0.0997)         &     (0.205)         &     (0.213)         &     (0.350)         \\
\addlinespace
Satisfied with Family&       0.228         &       0.218         &       0.228         &       0.659\sym{**} &       0.474         &      -0.171         &      0.0722         &       0.104         &       0.135         &       0.429         &       0.177         &       0.104         \\
            &     (0.194)         &     (0.193)         &     (0.209)         &     (0.244)         &     (0.316)         &     (0.103)         &     (0.147)         &     (0.145)         &     (0.148)         &     (0.221)         &     (0.263)         &    (0.0868)         \\
\addlinespace
Optimistic Look in Life&    -0.00538         &     -0.0249         &      -0.134         &      -0.151         &      -0.301\sym{*}  &    -0.00772         &      0.0396         &      0.0408         &      0.0270         &     0.00678         &      0.0368         &       0.350\sym{***}\\
            &     (0.106)         &    (0.0972)         &     (0.105)         &     (0.147)         &     (0.132)         &    (0.0609)         &    (0.0932)         &    (0.0922)         &    (0.0979)         &     (0.142)         &     (0.159)         &    (0.0747)         \\
\addlinespace
Positive Reciprocity&      -0.180         &      -0.149         &     -0.0742         &      -0.303         &      -0.116         &     -0.0831         &      -0.249\sym{*}  &      -0.203\sym{*}  &      -0.260\sym{*}  &      -0.325\sym{*}  &       0.239         &       0.264\sym{*}  \\
            &     (0.106)         &    (0.0908)         &    (0.0997)         &     (0.170)         &     (0.234)         &    (0.0765)         &     (0.103)         &    (0.0991)         &     (0.107)         &     (0.164)         &     (0.307)         &     (0.108)         \\
\addlinespace
Negative Reciprocity&       0.342\sym{*}  &       0.348\sym{*}  &       0.460\sym{*}  &       0.551\sym{*}  &       0.591         &       0.546\sym{***}&       0.156         &       0.122         &       0.178         &       0.156         &      -0.156         &      -0.174         \\
            &     (0.155)         &     (0.158)         &     (0.184)         &     (0.238)         &     (0.321)         &     (0.118)         &     (0.147)         &     (0.151)         &     (0.158)         &     (0.261)         &     (0.282)         &     (0.131)         \\
\bottomrule
\end{tabular}
}
}
\vspace{1ex} \\
\footnotesize\raggedright{Note: This table shows the estimates of the coefficient for attending Reggio Approach preschools from the OLS and difference-in-difference (DiD) models.Column title indicates the corresponding age group, control set, and model. ``None30'' refers to the OLS estimate with only the age-30 cohort and with no control variables. ``BIC30'' refers to the OLS estimate with only the age-30 cohort and with controls selected by Bayesian Information Criterion (BIC). ``Full30'' refers to the OLS estimate with only the age-30 cohort and with the full set of controls. ``DidPm30'' refers to the difference-in-difference estimate of (Reggio Muni - Parma Muni) - (Reggio None - Parma None) for the age-30 cohort. ``DidPv30'' refers to the difference-in-difference estimate of (Reggio Muni - Padova Muni) - (Reggio None - Padova None) for the age-30 cohort.  Analogous meanings are applied to the age-40 cohort in columns (7) through (12). Robust standard errors are reported in parentheses. Stars show statistical significance as follows: * $p < 0.05$, ** $p < 0.01$, *** $p < 0.001$.}
\end{table}

%Table \ref{ols-S-reg} shows consistently significant and negative effects of the Reggio Approach on whether the respondent does volunteer work for both the age-30 and age-40 cohorts. Moreover, for the age-40 cohort, there are significantly negative effects on respondents having migrant friends and significantly positive effects on whether respondents have ever voted for municipal and regional elections. For the age-30 cohort, there are significantly negative effects on whether respondents have ever voted for national elections.

\begin{table}[H] \caption{OLS and Diff-in-Diff for Social Behavior, Preschools, Adult Cohorts} \label{ols-S-reg}
\scalebox{0.80}{
{
\def\sym#1{\ifmmode^{#1}\else\(^{#1}\)\fi}
\begin{tabular}{l*{10}{c}}
\toprule
            &\multicolumn{1}{c}{(1)}&\multicolumn{1}{c}{(2)}&\multicolumn{1}{c}{(3)}&\multicolumn{1}{c}{(4)}&\multicolumn{1}{c}{(5)}&\multicolumn{1}{c}{(6)}&\multicolumn{1}{c}{(7)}&\multicolumn{1}{c}{(8)}&\multicolumn{1}{c}{(9)}&\multicolumn{1}{c}{(10)}\\
            &\multicolumn{1}{c}{None30}&\multicolumn{1}{c}{BIC30}&\multicolumn{1}{c}{Full30}&\multicolumn{1}{c}{DidPm30}&\multicolumn{1}{c}{DidPv30}&\multicolumn{1}{c}{None40}&\multicolumn{1}{c}{BIC40}&\multicolumn{1}{c}{Full40}&\multicolumn{1}{c}{DidPm40}&\multicolumn{1}{c}{DidPv40}\\
\midrule
Favorable to Migrants&       0.195         &       0.218\sym{*}  &       0.141         &       0.114         &       0.299         &      0.0354         &      0.0428         &      0.0385         &       0.206         &       0.342         \\
            &     (0.102)         &     (0.105)         &     (0.108)         &     (0.207)         &     (0.225)         &    (0.0964)         &     (0.102)         &     (0.105)         &     (0.200)         &     (0.235)         \\
\addlinespace
Num. of Friends&      -1.473         &      -1.144         &      -0.420         &       4.096         &      -1.006         &      0.0286         &      -0.646         &      -0.136         &       1.411         &       1.323         \\
            &     (1.198)         &     (1.611)         &     (2.475)         &     (2.200)         &     (1.993)         &     (0.941)         &     (0.820)         &     (0.964)         &     (1.352)         &     (1.445)         \\
\addlinespace
Has Migrant Friends&       0.218\sym{*}  &       0.239\sym{*}  &       0.205         &       0.318\sym{*}  &       0.336\sym{*}  &      0.0582         &     0.00748         &     -0.0173         &     -0.0481         &       0.133         \\
            &    (0.0943)         &    (0.0959)         &     (0.109)         &     (0.132)         &     (0.140)         &    (0.0870)         &    (0.0849)         &    (0.0874)         &     (0.134)         &     (0.139)         \\
\addlinespace
Volunteers  &      0.0632\sym{*}  &      0.0759\sym{*}  &       0.106\sym{**} &      0.0102         &     -0.0151         &    -0.00802         &     -0.0318         &     -0.0115         &      0.0170         &      0.0603         \\
            &    (0.0252)         &    (0.0316)         &    (0.0402)         &     (0.100)         &    (0.0957)         &    (0.0565)         &    (0.0508)         &    (0.0532)         &     (0.101)         &    (0.0891)         \\
\addlinespace
Child Eats Meal with Fam&       0.238         &       0.382         &       0.125         &     -0.0469         &       0.181         &       0.229         &       0.191         &       0.186         &      -0.197         &       0.269         \\
            &     (0.369)         &     (0.398)         &     (0.342)         &     (0.845)         &     (0.781)         &     (0.124)         &     (0.121)         &     (0.128)         &     (0.237)         &     (0.202)         \\
\addlinespace
Ever Voted for Municipal&     -0.0513         &    -0.00825         &    -0.00577         &    -0.00730         &       0.185         &      0.0362         &       0.119         &       0.112         &       0.204         &      0.0645         \\
            &    (0.0971)         &    (0.0709)         &    (0.0966)         &     (0.104)         &     (0.128)         &    (0.0913)         &    (0.0741)         &    (0.0774)         &     (0.119)         &     (0.140)         \\
\addlinespace
Ever Voted for Regional&     -0.0513         &    -0.00392         &     -0.0162         &     -0.0172         &       0.222         &      0.0692         &       0.153\sym{*}  &       0.136         &       0.274\sym{*}  &       0.204         \\
            &    (0.0971)         &    (0.0689)         &    (0.0922)         &    (0.0994)         &     (0.128)         &    (0.0908)         &    (0.0740)         &    (0.0787)         &     (0.118)         &     (0.154)         \\
\addlinespace
Ever Voted for National&      0.0626         &      0.0607         &      0.0823         &       0.114         &       0.337\sym{*}  &      0.0837         &      0.0683         &      0.0849         &     -0.0209         &       0.206         \\
            &    (0.0869)         &    (0.0945)         &     (0.103)         &     (0.118)         &     (0.133)         &    (0.0703)         &    (0.0772)         &    (0.0827)         &     (0.105)         &     (0.110)         \\
\bottomrule
\end{tabular}
}
}
\vspace{1ex} \\
\footnotesize\raggedright{Note: This table shows the estimates of the coefficient for attending Reggio Approach preschools from the OLS and difference-in-difference (DiD) models.Column title indicates the corresponding age group, control set, and model. ``None30'' refers to the OLS estimate with only the age-30 cohort and with no control variables. ``BIC30'' refers to the OLS estimate with only the age-30 cohort and with controls selected by Bayesian Information Criterion (BIC). ``Full30'' refers to the OLS estimate with only the age-30 cohort and with the full set of controls. ``DidPm30'' refers to the difference-in-difference estimate of (Reggio Muni - Parma Muni) - (Reggio None - Parma None) for the age-30 cohort. ``DidPv30'' refers to the difference-in-difference estimate of (Reggio Muni - Padova Muni) - (Reggio None - Padova None) for the age-30 cohort.  Analogous meanings are applied to the age-40 cohort in columns (7) through (12). Robust standard errors are reported in parentheses. Stars show statistical significance as follows: * $p < 0.05$, ** $p < 0.01$, *** $p < 0.001$.}
\end{table}

\end{landscape}

