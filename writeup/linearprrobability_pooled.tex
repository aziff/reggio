\documentclass[12pt]{article}
\usepackage[top=1in, bottom=1in, left=1in, right=1in]{geometry}
\parindent 22pt

\usepackage{adjustbox}
\usepackage{amsmath}
\usepackage{amssymb}
\usepackage{array}
\usepackage{booktabs}
\usepackage{fancyhdr}
\usepackage{float}
\usepackage{graphicx}
\usepackage[colorlinks=true,linkcolor=blue,urlcolor=blue,anchorcolor=blue,citecolor=blue]{hyperref}
\usepackage{lscape}
\usepackage{multirow}
\usepackage{natbib}
\usepackage{setspace}
\usepackage{tabularx}
\usepackage[colorinlistoftodos,linecolor=black]{todonotes}
\usepackage{appendix}
\usepackage{pgffor}
\usepackage{caption} 
\usepackage{threeparttable}
\captionsetup[table]{skip=3pt}



\newcolumntype{L}[1]{>{\raggedright\arraybackslash}p{#1}}
\newcolumntype{C}[1]{>{\centering\arraybackslash}p{#1}}
\newcolumntype{R}[1]{>{\raggedleft\arraybackslash}p{#1}}

\begin{document}

\title{Preliminary Reggio Analysis: Results for Linear Probability Model, Gender-Pooled Sample}
\author{Reggio Team}
\date{Original Version: June 10, 2016 \\ Current Version: \today}
\maketitle 

\section{Estimation Strategy}
\doublespacing
Before presenting an estimation framework, we first briefly discuss the data. Reggio data contains respondents from 3 cities (Reggio, Parma, Padova) and 6 cohorts (children, adolescents, adults in their 30's, adults in their 40's, adults in their 50's, and migrants who are children). Early childhood schools are broadly categorized into Asilo, which is for children aged 0-3, and Materna, which is for children aged 3-5. Asilo has 3 types: (i) Municipal, (ii) Religious, and (iii) Private. Materna has 4 types: (i) Municipal, (ii) Religious, (iii) Private, and (iv) State. For the purpose of estimating linear probability model, we denote ``not attending" any Asilo and Materna schools, respectively, also as a type of early childhood schools. Hence, for our purpose, we have 9 early childhood school types in total, which are 4 types of Asilo and 5 types of Materna. 

We now present a framework for the linear probability model that estimates the  marginal effects of covariates on the probability of attending a certain type of early childhood schools. Let $Y_{i,p}^{c,a}$ be an indicator of attending a preschool of type $p$ for individual $i$ in city $c$ and in cohort $a$. Let $\mathbf{X}_{i}^{c,a}$ be a vector of binary covariates for individual $i$ in city $c$ and in cohort $a$. Covariates include individual, parental, and household baseline characteristics. The estimation equation is:
\begin{equation} \label{eq:lpm}
Y_{i,p}^{c,a} = \mathbf{X}_{i}^{c,a}\beta_{i,p}^{c,a} + \varepsilon_{i,p}^{c,a} \text{, $\forall$ $i$ in city $c$ and in cohort $a$}
\end{equation}

Equation \ref{eq:lpm} is estimated for each group of people who are in same city \textit{and} in same cohort. We first present the estimation results for attending types of Asilo schools for each group, and then present results for Materna schools. For the tables of results for Asilo types, each column indicates the type of Asilo school that is used as our dependent variable in the regression. Hence, each column shows the marginal effects of covariates on the probability of attending a type of Asilo school indicated in the column title, for the group of people in a certain city \textit{and} in a certain cohort. The description for the Materna tables is analogous. 

\clearpage
\singlespacing
\listoftables

%\section{Asilo (Age 0-3)}
%\subsection{Children}
%\begin{table}[H]
\caption{LPM Estimation - Reggio - Children, Asilo}
\centering
\scalebox{0.7}{
\begin{tabular}{lcccc}
\toprule
 & \textbf{None} & \textbf{Municipal} & \textbf{Religious} & \textbf{Private} \\
\midrule
\textbf{Respondent's Baseline Info} \\
\quad Male & \textbf{    -0.10} & \textbf{     0.09} &      0.00 &      0.01 \\
\quad  & \textbf{(     0.05 )} & \textbf{(     0.06 )}  & (     0.03 )  & (     0.02 )  \\
\quad CAPI &     -0.02 &      0.03 &      0.01 &     -0.02 \\
\quad  & (     0.06 ) & (     0.06 )  & (     0.03 )  & (     0.02 )  \\
\quad Low Birthweight &      0.04 &      0.06 & \textbf{    -0.14} &      0.04 \\
\quad  & (     0.14 ) & (     0.15 )  & \textbf{(     0.09 )}  & (     0.04 )  \\
\quad Premature at Birth & \textbf{    -0.25} &      0.16 &      0.08 &      0.01 \\
\quad  & \textbf{(     0.13 )} & (     0.13 )  & (     0.08 )  & (     0.04 )  \\
\midrule
\textbf{Mother's Baseline Info} \\
\quad Max Education: Middle School &     -0.04 &      0.08 &     -0.04 &      0.00 \\
\quad  & (     0.12 ) & (     0.12 )  & (     0.07 )  & (     0.04 )  \\
\quad Max Education: High School & \textbf{    -0.18} & \textbf{     0.20} &     -0.03 &      0.02 \\
\quad  & \textbf{(     0.08 )} & \textbf{(     0.08 )}  & (     0.05 )  & (     0.02 )  \\
\quad Max Education: University & \textbf{    -0.28} & \textbf{     0.29} &     -0.04 &      0.03 \\
\quad  & \textbf{(     0.09 )} & \textbf{(     0.10 )}  & (     0.06 )  & (     0.03 )  \\
\quad Born in Province & \textbf{     0.11} &     -0.09 &     -0.03 &      0.00 \\
\quad  & \textbf{(     0.06 )} & (     0.06 )  & (     0.04 )  & (     0.02 )  \\
\midrule
\textbf{Father's Baseline Info} \\
\quad Max Education: Middle School &     -0.08 &      0.10 &     -0.01 &     -0.00 \\
\quad  & (     0.11 ) & (     0.12 )  & (     0.07 )  & (     0.03 )  \\
\quad Max Education: High School &     -0.06 &      0.05 &     -0.02 &      0.03 \\
\quad  & (     0.07 ) & (     0.08 )  & (     0.04 )  & (     0.02 )  \\
\quad Max Education: University &     -0.08 &     -0.05 & \textbf{     0.11} &      0.01 \\
\quad  & (     0.08 ) & (     0.09 )  & \textbf{(     0.05 )}  & (     0.03 )  \\
\quad Born in Province &      0.08 & \textbf{    -0.12} &      0.05 &     -0.00 \\
\quad  & (     0.06 ) & \textbf{(     0.06 )}  & (     0.04 )  & (     0.02 )  \\
\midrule
\textbf{Household Baseline Info} \\
\quad Caregiver Has Religion & \textbf{     0.14} &     -0.09 &     -0.02 &     -0.02 \\
\quad  & \textbf{(     0.08 )} & (     0.08 )  & (     0.05 )  & (     0.02 )  \\
\quad Owns House &      0.03 &     -0.07 &      0.05 &     -0.01 \\
\quad  & (     0.06 ) & (     0.06 )  & (     0.03 )  & (     0.02 )  \\
\quad Income 5K-10K Euro &     -0.10 &      0.16 &     -0.04 &     -0.02 \\
\quad  & (     0.24 ) & (     0.25 )  & (     0.15 )  & (     0.07 )  \\
\quad Income 10K-25K Euro & \textbf{    -0.19} & \textbf{     0.20} &      0.01 &     -0.01 \\
\quad  & \textbf{(     0.09 )} & \textbf{(     0.09 )}  & (     0.05 )  & (     0.03 )  \\
\quad Income 25K-50K Euro & \textbf{    -0.20} & \textbf{     0.21} &      0.04 & \textbf{    -0.05} \\
\quad  & \textbf{(     0.07 )} & \textbf{(     0.08 )}  & (     0.05 )  & \textbf{(     0.02 )}  \\
\quad Income 50K-100K Euro & \textbf{    -0.21} & \textbf{     0.19} &      0.03 &     -0.01 \\
\quad  & \textbf{(     0.08 )} & \textbf{(     0.09 )}  & (     0.05 )  & (     0.03 )  \\
\quad Income 100K-250K Euro &     -0.23 &      0.26 &      0.01 &     -0.04 \\
\quad  & (     0.18 ) & (     0.19 )  & (     0.11 )  & (     0.06 )  \\
\midrule
Observations & 310 & 310 & 310 & 310 \\
Fraction Attending Each Type &      0.38 &      0.52 &      0.09 &      0.02 \\
\midrule
$ R^2$ &      0.17 &      0.16 &      0.07 &      0.05 \\
\bottomrule
\end{tabular}}
\end{table}
\begin{scriptsize}
\noindent\underline{Note:} This table presents the linear probability model estimations for attending each type of Asilo schools, indicated by each column. The samples used in this estimation are those who were children at the time of the survey living in Reggio. All dependent variables are binary. Observation indicates the number of people included in this sample. Bold number indicates that the p-value is less than or equal to 0.1. Standard errors are reported in parentheses.
\end{scriptsize}

%\begin{table}[H]
\caption{LPM Estimation - Parma - Children, Asilo}
\centering
\scalebox{0.7}{
\begin{tabular}{lcccc}
\toprule
 & \textbf{None} & \textbf{Municipal} & \textbf{Religious} & \textbf{Private} \\
\midrule
\textbf{Respondent's Baseline Info} \\
\quad Male &     -0.03 &      0.06 &     -0.01 &     -0.01 \\
\quad  & (     0.06 ) & (     0.06 )  & (     0.02 )  & (     0.04 )  \\
\quad CAPI &      0.04 &      0.01 &     -0.01 &     -0.03 \\
\quad  & (     0.06 ) & (     0.06 )  & (     0.02 )  & (     0.04 )  \\
\quad Low Birthweight & \textbf{     0.30} &     -0.09 &     -0.01 & \textbf{    -0.20} \\
\quad  & \textbf{(     0.14 )} & (     0.15 )  & (     0.05 )  & \textbf{(     0.10 )}  \\
\quad Premature at Birth &     -0.09 &     -0.06 &     -0.01 & \textbf{     0.17} \\
\quad  & (     0.13 ) & (     0.14 )  & (     0.04 )  & \textbf{(     0.10 )}  \\
\midrule
\textbf{Mother's Baseline Info} \\
\quad Max Education: Middle School &     -0.22 &      0.24 &      0.04 &     -0.05 \\
\quad  & (     0.18 ) & (     0.19 )  & (     0.06 )  & (     0.13 )  \\
\quad Max Education: High School &     -0.17 &      0.10 &      0.06 &      0.02 \\
\quad  & (     0.12 ) & (     0.13 )  & (     0.04 )  & (     0.09 )  \\
\quad Max Education: University & \textbf{    -0.25} &      0.18 &      0.05 &      0.03 \\
\quad  & \textbf{(     0.12 )} & (     0.13 )  & (     0.04 )  & (     0.09 )  \\
\quad Teenager at Birth &      0.01 &      0.15 &     -0.01 &     -0.15 \\
\quad  & (     0.54 ) & (     0.58 )  & (     0.18 )  & (     0.39 )  \\
\quad Born in Province &     -0.00 &     -0.04 &     -0.03 & \textbf{     0.07} \\
\quad  & (     0.06 ) & (     0.07 )  & (     0.02 )  & \textbf{(     0.04 )}  \\
\midrule
\textbf{Father's Baseline Info} \\
\quad Max Education: Middle School & \textbf{     0.26} & \textbf{    -0.28} &     -0.04 &      0.06 \\
\quad  & \textbf{(     0.13 )} & \textbf{(     0.14 )}  & (     0.04 )  & (     0.09 )  \\
\quad Max Education: High School & \textbf{     0.16} &     -0.11 &     -0.04 &     -0.01 \\
\quad  & \textbf{(     0.09 )} & (     0.09 )  & (     0.03 )  & (     0.06 )  \\
\quad Max Education: University &      0.13 &     -0.10 &     -0.01 &     -0.01 \\
\quad  & (     0.09 ) & (     0.10 )  & (     0.03 )  & (     0.07 )  \\
\quad Teenager at Birth &      0.60 &     -0.41 &     -0.09 &     -0.10 \\
\quad  & (     0.72 ) & (     0.77 )  & (     0.24 )  & (     0.52 )  \\
\quad Born in Province &     -0.00 &     -0.04 &      0.01 &      0.03 \\
\quad  & (     0.06 ) & (     0.07 )  & (     0.02 )  & (     0.05 )  \\
\midrule
\textbf{Household Baseline Info} \\
\quad Caregiver Has Religion &      0.09 &     -0.07 & \textbf{    -0.05} &      0.03 \\
\quad  & (     0.08 ) & (     0.09 )  & \textbf{(     0.03 )}  & (     0.06 )  \\
\quad Owns House & \textbf{     0.13} &     -0.05 &     -0.03 &     -0.05 \\
\quad  & \textbf{(     0.06 )} & (     0.07 )  & (     0.02 )  & (     0.05 )  \\
\quad Income 5K-10K Euro &     -0.39 &      0.44 &     -0.07 &      0.03 \\
\quad  & (     0.25 ) & (     0.27 )  & (     0.08 )  & (     0.18 )  \\
\quad Income 10K-25K Euro &     -0.03 &      0.12 & \textbf{    -0.08} &     -0.01 \\
\quad  & (     0.10 ) & (     0.10 )  & \textbf{(     0.03 )}  & (     0.07 )  \\
\quad Income 25K-50K Euro &     -0.12 & \textbf{     0.23} & \textbf{    -0.05} &     -0.05 \\
\quad  & (     0.08 ) & \textbf{(     0.09 )}  & \textbf{(     0.03 )}  & (     0.06 )  \\
\quad Income 50K-100K Euro &     -0.12 & \textbf{     0.23} & \textbf{    -0.08} &     -0.03 \\
\quad  & (     0.09 ) & \textbf{(     0.10 )}  & \textbf{(     0.03 )}  & (     0.07 )  \\
\quad Income 100K-250K Euro &     -0.26 &      0.27 &     -0.07 &      0.06 \\
\quad  & (     0.21 ) & (     0.22 )  & (     0.07 )  & (     0.15 )  \\
\midrule
Observations & 290 & 290 & 290 & 290 \\
Fraction Attending Each Type &      0.34 &      0.51 &      0.02 &      0.12 \\
\midrule
$ R^2$ &      0.10 &      0.08 &      0.08 &      0.04 \\
\bottomrule
\end{tabular}}
\end{table}
\begin{scriptsize}
\noindent\underline{Note:} This table presents the linear probability model estimations for attending each type of Asilo schools, indicated by each column. The samples used in this estimation are those who were children at the time of the survey living in Parma. All dependent variables are binary. Observation indicates the number of people included in this sample. Bold number indicates that the p-value is less than or equal to 0.1. Standard errors are reported in parentheses.
\end{scriptsize}

%\begin{table}[H]
\caption{LPM Estimation - Padova - Children, Asilo}
\centering
\scalebox{0.7}{
\begin{tabular}{lcccc}
\toprule
 & \textbf{None} & \textbf{Municipal} & \textbf{Religious} & \textbf{Private} \\
\midrule
\textbf{Respondent's Baseline Info} \\
\quad Male & \textbf{    -0.12} &      0.02 & \textbf{     0.06} &      0.04 \\
\quad  & \textbf{(     0.05 )} & (     0.05 )  & \textbf{(     0.04 )}  & (     0.04 )  \\
\quad CAPI &      0.06 &      0.05 &     -0.04 &     -0.07 \\
\quad  & (     0.06 ) & (     0.05 )  & (     0.04 )  & (     0.05 )  \\
\quad Low Birthweight &     -0.08 &      0.04 &      0.00 &      0.03 \\
\quad  & (     0.15 ) & (     0.14 )  & (     0.10 )  & (     0.12 )  \\
\quad Premature at Birth &     -0.11 & \textbf{     0.28} &     -0.04 &     -0.13 \\
\quad  & (     0.12 ) & \textbf{(     0.12 )}  & (     0.08 )  & (     0.10 )  \\
\midrule
\textbf{Mother's Baseline Info} \\
\quad Max Education: Middle School &      0.04 &      0.02 & \textbf{    -0.16} &      0.11 \\
\quad  & (     0.13 ) & (     0.12 )  & \textbf{(     0.09 )}  & (     0.11 )  \\
\quad Max Education: High School &      0.06 &     -0.05 &     -0.10 &      0.08 \\
\quad  & (     0.10 ) & (     0.09 )  & (     0.07 )  & (     0.08 )  \\
\quad Max Education: University &     -0.16 &      0.09 &     -0.10 & \textbf{     0.18} \\
\quad  & (     0.10 ) & (     0.10 )  & (     0.07 )  & \textbf{(     0.09 )}  \\
\quad Teenager at Birth &      0.18 &     -0.16 &      0.03 &     -0.05 \\
\quad  & (     0.36 ) & (     0.34 )  & (     0.24 )  & (     0.29 )  \\
\quad Born in Province & \textbf{     0.13} &     -0.04 &     -0.03 &     -0.05 \\
\quad  & \textbf{(     0.06 )} & (     0.06 )  & (     0.04 )  & (     0.05 )  \\
\midrule
\textbf{Father's Baseline Info} \\
\quad Max Education: Middle School &      0.17 &     -0.18 &      0.01 &      0.00 \\
\quad  & (     0.12 ) & (     0.12 )  & (     0.08 )  & (     0.10 )  \\
\quad Max Education: High School & \textbf{     0.15} &     -0.10 &     -0.07 &      0.01 \\
\quad  & \textbf{(     0.09 )} & (     0.08 )  & (     0.06 )  & (     0.07 )  \\
\quad Max Education: University & \textbf{     0.23} &     -0.13 &     -0.03 &     -0.08 \\
\quad  & \textbf{(     0.09 )} & (     0.09 )  & (     0.06 )  & (     0.08 )  \\
\quad Teenager at Birth &      0.27 &     -0.10 &     -0.09 &     -0.08 \\
\quad  & (     0.35 ) & (     0.34 )  & (     0.24 )  & (     0.29 )  \\
\quad Born in Province & \textbf{     0.12} &     -0.08 &     -0.01 &     -0.03 \\
\quad  & \textbf{(     0.07 )} & (     0.06 )  & (     0.04 )  & (     0.05 )  \\
\midrule
\textbf{Household Baseline Info} \\
\quad Caregiver Has Religion &     -0.06 &     -0.09 & \textbf{     0.09} &      0.07 \\
\quad  & (     0.07 ) & (     0.07 )  & \textbf{(     0.05 )}  & (     0.06 )  \\
\quad Owns House &     -0.04 &      0.04 &      0.03 &     -0.03 \\
\quad  & (     0.06 ) & (     0.06 )  & (     0.04 )  & (     0.05 )  \\
\quad Income 5K-10K Euro &     -0.00 &     -0.09 &     -0.02 &      0.12 \\
\quad  & (     0.22 ) & (     0.21 )  & (     0.15 )  & (     0.18 )  \\
\quad Income 10K-25K Euro & \textbf{    -0.35} & \textbf{     0.19} &      0.08 &      0.08 \\
\quad  & \textbf{(     0.08 )} & \textbf{(     0.08 )}  & (     0.06 )  & (     0.07 )  \\
\quad Income 25K-50K Euro & \textbf{    -0.35} & \textbf{     0.22} &      0.06 &      0.08 \\
\quad  & \textbf{(     0.07 )} & \textbf{(     0.07 )}  & (     0.05 )  & (     0.06 )  \\
\quad Income 50K-100K Euro & \textbf{    -0.41} & \textbf{     0.18} &      0.10 & \textbf{     0.13} \\
\quad  & \textbf{(     0.09 )} & \textbf{(     0.09 )}  & (     0.06 )  & \textbf{(     0.08 )}  \\
\quad Income 100K-250K Euro &     -0.24 & \textbf{     0.27} &      0.08 &     -0.10 \\
\quad  & (     0.16 ) & \textbf{(     0.16 )}  & (     0.11 )  & (     0.13 )  \\
\midrule
Observations & 278 & 278 & 278 & 278 \\
Fraction Attending Each Type &      0.51 &      0.24 &      0.09 &      0.15 \\
\midrule
$ R^2$ &      0.32 &      0.17 &      0.08 &      0.11 \\
\bottomrule
\end{tabular}}
\end{table}
\begin{scriptsize}
\noindent\underline{Note:} This table presents the linear probability model estimations for attending each type of Asilo schools, indicated by each column. The samples used in this estimation are those who were children at the time of the survey living in Padova. All dependent variables are binary. Observation indicates the number of people included in this sample. Bold number indicates that the p-value is less than or equal to 0.1. Standard errors are reported in parentheses.
\end{scriptsize}


%\subsection{Adolescents}
%\begin{table}[H]
\caption{LPM Estimation - Reggio - Adolescents, Asilo}
\centering
\scalebox{0.7}{
\begin{tabular}{lcccc}
\toprule
 & \textbf{None} & \textbf{Municipal} & \textbf{Religious} & \textbf{Private} \\
\midrule
\textbf{Respondent's Baseline Info} \\
\quad Male &     -0.01 &     -0.01 &      0.01 &      0.01 \\
\quad  & (     0.06 ) & (     0.06 )  & (     0.02 )  & (     0.01 )  \\
\quad CAPI &     -0.07 &      0.05 &      0.02 &     -0.00 \\
\quad  & (     0.06 ) & (     0.06 )  & (     0.02 )  & (     0.01 )  \\
\quad Low Birthweight &     -0.05 &      0.02 &      0.04 &     -0.00 \\
\quad  & (     0.15 ) & (     0.15 )  & (     0.05 )  & (     0.03 )  \\
\quad Premature at Birth &      0.07 &     -0.06 &      0.01 &     -0.01 \\
\quad  & (     0.14 ) & (     0.14 )  & (     0.05 )  & (     0.03 )  \\
\midrule
\textbf{Mother's Baseline Info} \\
\quad Max Education: Middle School & \textbf{    -0.24} &      0.19 &      0.04 &      0.01 \\
\quad  & \textbf{(     0.13 )} & (     0.13 )  & (     0.05 )  & (     0.03 )  \\
\quad Max Education: High School & \textbf{    -0.16} &      0.12 &      0.02 &      0.02 \\
\quad  & \textbf{(     0.09 )} & (     0.09 )  & (     0.03 )  & (     0.02 )  \\
\quad Max Education: University & \textbf{    -0.17} &      0.12 &      0.04 &      0.01 \\
\quad  & \textbf{(     0.10 )} & (     0.10 )  & (     0.04 )  & (     0.02 )  \\
\quad Teenager at Birth &      0.34 &     -0.28 &     -0.04 &     -0.02 \\
\quad  & (     0.26 ) & (     0.26 )  & (     0.09 )  & (     0.05 )  \\
\quad Born in Province &     -0.07 &      0.11 &     -0.01 &     -0.02 \\
\quad  & (     0.07 ) & (     0.07 )  & (     0.02 )  & (     0.01 )  \\
\midrule
\textbf{Father's Baseline Info} \\
\quad Max Education: Middle School &      0.20 & \textbf{    -0.22} &      0.01 &      0.01 \\
\quad  & (     0.13 ) & \textbf{(     0.13 )}  & (     0.04 )  & (     0.03 )  \\
\quad Max Education: High School &     -0.01 &     -0.02 &      0.01 &      0.02 \\
\quad  & (     0.08 ) & (     0.08 )  & (     0.03 )  & (     0.02 )  \\
\quad Max Education: University &     -0.11 &      0.04 &      0.04 &      0.02 \\
\quad  & (     0.10 ) & (     0.10 )  & (     0.03 )  & (     0.02 )  \\
\quad Born in Province &     -0.01 &     -0.00 &      0.03 &     -0.01 \\
\quad  & (     0.07 ) & (     0.07 )  & (     0.02 )  & (     0.01 )  \\
\midrule
\textbf{Household Baseline Info} \\
\quad Caregiver Has Religion &      0.07 & \textbf{    -0.13} & \textbf{     0.04} &      0.01 \\
\quad  & (     0.07 ) & \textbf{(     0.07 )}  & \textbf{(     0.02 )}  & (     0.01 )  \\
\quad Owns House &     -0.13 &      0.09 &      0.03 &      0.02 \\
\quad  & (     0.09 ) & (     0.09 )  & (     0.03 )  & (     0.02 )  \\
\quad Income 5K-10K Euro &     -0.32 &      0.40 &     -0.06 &     -0.02 \\
\quad  & (     0.31 ) & (     0.31 )  & (     0.11 )  & (     0.06 )  \\
\quad Income 10K-25K Euro &     -0.10 &      0.08 &      0.02 &      0.00 \\
\quad  & (     0.10 ) & (     0.10 )  & (     0.03 )  & (     0.02 )  \\
\quad Income 25K-50K Euro &     -0.10 &      0.12 &     -0.01 &     -0.01 \\
\quad  & (     0.08 ) & (     0.08 )  & (     0.03 )  & (     0.02 )  \\
\quad Income 50K-100K Euro &     -0.12 & \textbf{     0.20} & \textbf{    -0.06} &     -0.02 \\
\quad  & (     0.09 ) & \textbf{(     0.09 )}  & \textbf{(     0.03 )}  & (     0.02 )  \\
\quad Income 100K-250K Euro &     -0.13 &      0.06 & \textbf{     0.09} &     -0.03 \\
\quad  & (     0.15 ) & (     0.16 )  & \textbf{(     0.05 )}  & (     0.03 )  \\
\quad Income More Than 250K Euro &     -0.38 &      0.51 &     -0.10 &     -0.03 \\
\quad  & (     0.50 ) & (     0.50 )  & (     0.18 )  & (     0.10 )  \\
\midrule
Observations & 295 & 295 & 295 & 295 \\
Fraction Attending Each Type &      0.44 &      0.52 &      0.03 &      0.01 \\
\midrule
$ R^2$ &      0.11 &      0.10 &      0.08 &      0.04 \\
\bottomrule
\end{tabular}}
\end{table}
\begin{scriptsize}
\noindent\underline{Note:} This table presents the linear probability model estimations for attending each type of Asilo schools, indicated by each column. The samples used in this estimation are those who were adolescents at the time of the survey living in Reggio. All dependent variables are binary. Observation indicates the number of people included in this sample. Bold number indicates that the p-value is less than or equal to 0.1. Standard errors are reported in parentheses.
\end{scriptsize}

%\begin{table}[H]
\caption{LPM Estimation - Parma - Adolescents, Asilo}
\centering
\scalebox{0.7}{
\begin{tabular}{lcccc}
\toprule
 & \textbf{None} & \textbf{Municipal} & \textbf{Religious} & \textbf{Private} \\
\midrule
\textbf{Respondent's Baseline Info} \\
\quad Male &     -0.07 &      0.06 &      0.01 &      0.00 \\
\quad  & (     0.07 ) & (     0.07 )  & (     0.03 )  & (     0.03 )  \\
\quad CAPI &     -0.01 &     -0.09 &      0.03 & \textbf{     0.06} \\
\quad  & (     0.07 ) & (     0.07 )  & (     0.03 )  & \textbf{(     0.03 )}  \\
\quad Low Birthweight &     -0.00 &      0.05 &     -0.04 &     -0.01 \\
\quad  & (     0.16 ) & (     0.16 )  & (     0.06 )  & (     0.07 )  \\
\quad Premature at Birth &     -0.03 &      0.06 &     -0.01 &     -0.02 \\
\quad  & (     0.14 ) & (     0.13 )  & (     0.05 )  & (     0.06 )  \\
\midrule
\textbf{Mother's Baseline Info} \\
\quad Max Education: Middle School &     -0.04 &      0.18 & \textbf{    -0.10} &     -0.03 \\
\quad  & (     0.15 ) & (     0.14 )  & \textbf{(     0.06 )}  & (     0.06 )  \\
\quad Max Education: High School &      0.08 &     -0.04 &     -0.03 &     -0.01 \\
\quad  & (     0.12 ) & (     0.12 )  & (     0.05 )  & (     0.05 )  \\
\quad Max Education: University &     -0.01 &      0.03 &     -0.07 &      0.05 \\
\quad  & (     0.13 ) & (     0.13 )  & (     0.05 )  & (     0.06 )  \\
\quad Teenager at Birth &     -0.31 &      0.36 &     -0.01 &     -0.03 \\
\quad  & (     0.27 ) & (     0.26 )  & (     0.10 )  & (     0.12 )  \\
\quad Born in Province &      0.05 &     -0.11 &     -0.00 & \textbf{     0.06} \\
\quad  & (     0.07 ) & (     0.07 )  & (     0.03 )  & \textbf{(     0.03 )}  \\
\midrule
\textbf{Father's Baseline Info} \\
\quad Max Education: Middle School &     -0.20 & \textbf{     0.33} &     -0.04 &     -0.09 \\
\quad  & (     0.15 ) & \textbf{(     0.14 )}  & (     0.06 )  & (     0.07 )  \\
\quad Max Education: High School &     -0.10 &      0.14 &      0.02 &     -0.06 \\
\quad  & (     0.10 ) & (     0.10 )  & (     0.04 )  & (     0.05 )  \\
\quad Max Education: University &     -0.10 &      0.10 &     -0.01 &      0.01 \\
\quad  & (     0.11 ) & (     0.11 )  & (     0.04 )  & (     0.05 )  \\
\quad Teenager at Birth &     -0.13 &      0.26 &     -0.08 &     -0.05 \\
\quad  & (     0.36 ) & (     0.35 )  & (     0.14 )  & (     0.16 )  \\
\quad Born in Province &      0.09 & \textbf{    -0.13} &      0.02 &      0.03 \\
\quad  & (     0.08 ) & \textbf{(     0.08 )}  & (     0.03 )  & (     0.04 )  \\
\midrule
\textbf{Household Baseline Info} \\
\quad Caregiver Has Religion &      0.12 &     -0.10 &     -0.02 &      0.00 \\
\quad  & (     0.10 ) & (     0.10 )  & (     0.04 )  & (     0.04 )  \\
\quad Owns House &      0.06 &     -0.09 &     -0.03 &      0.05 \\
\quad  & (     0.09 ) & (     0.09 )  & (     0.04 )  & (     0.04 )  \\
\quad Income 5K-10K Euro &      0.30 &     -0.25 &     -0.01 &     -0.04 \\
\quad  & (     0.37 ) & (     0.36 )  & (     0.15 )  & (     0.16 )  \\
\quad Income 10K-25K Euro &     -0.15 &      0.12 &     -0.02 &      0.05 \\
\quad  & (     0.11 ) & (     0.10 )  & (     0.04 )  & (     0.05 )  \\
\quad Income 25K-50K Euro & \textbf{    -0.23} &      0.12 & \textbf{     0.11} &      0.01 \\
\quad  & \textbf{(     0.09 )} & (     0.09 )  & \textbf{(     0.03 )}  & (     0.04 )  \\
\quad Income 50K-100K Euro & \textbf{    -0.19} &      0.12 &      0.06 &      0.02 \\
\quad  & \textbf{(     0.10 )} & (     0.09 )  & (     0.04 )  & (     0.04 )  \\
\quad Income 100K-250K Euro &      0.14 &     -0.11 &      0.02 &     -0.05 \\
\quad  & (     0.20 ) & (     0.19 )  & (     0.08 )  & (     0.09 )  \\
\midrule
Observations & 250 & 250 & 250 & 250 \\
Fraction Attending Each Type &      0.52 &      0.39 &      0.04 &      0.05 \\
\midrule
$ R^2$ &      0.09 &      0.11 &      0.10 &      0.10 \\
\bottomrule
\end{tabular}}
\end{table}
\begin{scriptsize}
\noindent\underline{Note:} This table presents the linear probability model estimations for attending each type of Asilo schools, indicated by each column. The samples used in this estimation are those who were adolescents at the time of the survey living in Parma. All dependent variables are binary. Observation indicates the number of people included in this sample. Bold number indicates that the p-value is less than or equal to 0.1. Standard errors are reported in parentheses.
\end{scriptsize}

%\begin{table}[H]
\caption{LPM Estimation - Padova - Adolescents, Asilo}
\centering
\scalebox{0.7}{
\begin{tabular}{lcccc}
\toprule
 & \textbf{None} & \textbf{Municipal} & \textbf{Religious} & \textbf{Private} \\
\midrule
\textbf{Respondent's Baseline Info} \\
\quad Male &     -0.00 &     -0.02 &      0.02 &      0.00 \\
\quad  & (     0.05 ) & (     0.05 )  & (     0.02 )  & (        . )  \\
\quad CAPI &      0.02 &     -0.01 &     -0.01 &      0.00 \\
\quad  & (     0.06 ) & (     0.05 )  & (     0.02 )  & (        . )  \\
\quad Low Birthweight &      0.00 &      0.06 &     -0.07 &      0.00 \\
\quad  & (     0.13 ) & (     0.13 )  & (     0.05 )  & (        . )  \\
\quad Premature at Birth &     -0.12 &      0.06 &      0.06 &      0.00 \\
\quad  & (     0.11 ) & (     0.11 )  & (     0.05 )  & (        . )  \\
\midrule
\textbf{Mother's Baseline Info} \\
\quad Max Education: Middle School &     -0.03 &      0.06 &     -0.03 &      0.00 \\
\quad  & (     0.11 ) & (     0.11 )  & (     0.04 )  & (        . )  \\
\quad Max Education: High School & \textbf{    -0.15} & \textbf{     0.14} &      0.00 &      0.00 \\
\quad  & \textbf{(     0.08 )} & \textbf{(     0.07 )}  & (     0.03 )  & (        . )  \\
\quad Max Education: University & \textbf{    -0.27} & \textbf{     0.30} &     -0.03 &      0.00 \\
\quad  & \textbf{(     0.09 )} & \textbf{(     0.08 )}  & (     0.04 )  & (        . )  \\
\quad Born in Province &      0.07 &     -0.07 &     -0.00 &      0.00 \\
\quad  & (     0.07 ) & (     0.06 )  & (     0.03 )  & (        . )  \\
\midrule
\textbf{Father's Baseline Info} \\
\quad Max Education: Middle School &     -0.11 &      0.13 &     -0.02 &      0.00 \\
\quad  & (     0.11 ) & (     0.11 )  & (     0.04 )  & (        . )  \\
\quad Max Education: High School &      0.02 &     -0.02 &     -0.01 &      0.00 \\
\quad  & (     0.08 ) & (     0.07 )  & (     0.03 )  & (        . )  \\
\quad Max Education: University &      0.09 &     -0.08 &     -0.01 &      0.00 \\
\quad  & (     0.09 ) & (     0.09 )  & (     0.04 )  & (        . )  \\
\quad Teenager at Birth &     -0.66 & \textbf{     0.77} &     -0.11 &      0.00 \\
\quad  & (     0.43 ) & \textbf{(     0.41 )}  & (     0.17 )  & (        . )  \\
\quad Born in Province & \textbf{     0.14} &     -0.10 & \textbf{    -0.04} &      0.00 \\
\quad  & \textbf{(     0.06 )} & (     0.06 )  & \textbf{(     0.03 )}  & (        . )  \\
\midrule
\textbf{Household Baseline Info} \\
\quad Caregiver Has Religion &      0.10 &     -0.04 & \textbf{    -0.06} &      0.00 \\
\quad  & (     0.06 ) & (     0.06 )  & \textbf{(     0.02 )}  & (        . )  \\
\quad Owns House &     -0.05 &      0.04 &      0.00 &      0.00 \\
\quad  & (     0.06 ) & (     0.06 )  & (     0.03 )  & (        . )  \\
\quad Income 5K-10K Euro &      0.16 &     -0.13 &     -0.02 &      0.00 \\
\quad  & (     0.31 ) & (     0.29 )  & (     0.12 )  & (        . )  \\
\quad Income 10K-25K Euro &     -0.04 &      0.09 &     -0.05 &      0.00 \\
\quad  & (     0.09 ) & (     0.09 )  & (     0.04 )  & (        . )  \\
\quad Income 25K-50K Euro &     -0.05 &      0.08 &     -0.02 &      0.00 \\
\quad  & (     0.07 ) & (     0.06 )  & (     0.03 )  & (        . )  \\
\quad Income 50K-100K Euro &     -0.04 &      0.07 &     -0.02 &      0.00 \\
\quad  & (     0.09 ) & (     0.09 )  & (     0.04 )  & (        . )  \\
\quad Income 100K-250K Euro & \textbf{     0.31} & \textbf{    -0.27} &     -0.04 &      0.00 \\
\quad  & \textbf{(     0.17 )} & \textbf{(     0.16 )}  & (     0.07 )  & (        . )  \\
\midrule
Observations & 279 & 279 & 279 & 279 \\
Fraction Attending Each Type &      0.75 &      0.22 &      0.03 &      0.00 \\
\midrule
$ R^2$ &      0.12 &      0.13 &      0.06 &         . \\
\bottomrule
\end{tabular}}
\end{table}
\begin{scriptsize}
\noindent\underline{Note:} This table presents the linear probability model estimations for attending each type of Asilo schools, indicated by each column. The samples used in this estimation are those who were adolescents at the time of the survey living in Padova. All dependent variables are binary. Observation indicates the number of people included in this sample. Bold number indicates that the p-value is less than or equal to 0.1. Standard errors are reported in parentheses.
\end{scriptsize}


%\subsection{Adults at Age 30}
%\begin{table}[H]
\caption{LPM Estimation - Reggio - Adults (Age 30), Asilo}
\centering
\scalebox{0.7}{
\begin{tabular}{lcccc}
\toprule
 & \textbf{None} & \textbf{Municipal} & \textbf{Religious} & \textbf{Private} \\
\midrule
\textbf{Respondent's Baseline Info} \\
\quad Male &     -0.01 &      0.01 &     -0.01 &      0.02 \\
\quad  & (     0.05 ) & (     0.05 )  & (     0.01 )  & (     0.01 )  \\
\quad CAPI &     -0.07 & \textbf{     0.10} &     -0.02 &     -0.01 \\
\quad  & (     0.06 ) & \textbf{(     0.05 )}  & (     0.01 )  & (     0.01 )  \\
\midrule
\textbf{Mother's Baseline Info} \\
\quad Max Education: Middle School &     -0.27 &      0.14 &     -0.00 &      0.13 \\
\quad  & (     0.47 ) & (     0.45 )  & (     0.11 )  & (     0.11 )  \\
\quad Max Education: High School &     -0.12 &      0.03 &      0.01 &      0.08 \\
\quad  & (     0.48 ) & (     0.46 )  & (     0.12 )  & (     0.12 )  \\
\quad Max Education: University &     -0.08 &      0.01 &     -0.02 &      0.09 \\
\quad  & (     0.48 ) & (     0.47 )  & (     0.12 )  & (     0.12 )  \\
\quad Born in Province & \textbf{     0.14} & \textbf{    -0.17} &      0.01 &      0.01 \\
\quad  & \textbf{(     0.08 )} & \textbf{(     0.07 )}  & (     0.02 )  & (     0.02 )  \\
\midrule
\textbf{Father's Baseline Info} \\
\quad Max Education: Middle School &      0.06 &     -0.16 &     -0.02 & \textbf{     0.11} \\
\quad  & (     0.21 ) & (     0.20 )  & (     0.05 )  & \textbf{(     0.05 )}  \\
\quad Max Education: University &     -0.09 &      0.07 &      0.01 &      0.01 \\
\quad  & (     0.07 ) & (     0.07 )  & (     0.02 )  & (     0.02 )  \\
\quad Born in Province &     -0.05 &      0.03 &      0.01 &      0.01 \\
\quad  & (     0.08 ) & (     0.08 )  & (     0.02 )  & (     0.02 )  \\
\midrule
\textbf{Household Baseline Info} \\
\quad Caregiver Has Religion & \textbf{     0.10} & \textbf{    -0.09} &     -0.00 &     -0.01 \\
\quad  & \textbf{(     0.05 )} & \textbf{(     0.05 )}  & (     0.01 )  & (     0.01 )  \\
\midrule
Observations & 277 & 277 & 277 & 277 \\
Fraction Attending Each Type &      0.76 &      0.22 &      0.01 &      0.01 \\
\midrule
$ R^2$ &      0.04 &      0.05 &      0.03 &      0.07 \\
\bottomrule
\end{tabular}}
\end{table}
\begin{scriptsize}
\noindent\underline{Note:} This table presents the linear probability model estimations for attending each type of Asilo schools, indicated by each column. The samples used in this estimation are those who were adults in their 30's at the time of the survey living in Reggio. All dependent variables are binary. Observation indicates the number of people included in this sample. Bold number indicates that the p-value is less than or equal to 0.1. Standard errors are reported in parentheses.
\end{scriptsize}

%\begin{table}[H]
\caption{LPM Estimation - Parma - Adults (Age 30), Asilo}
\centering
\scalebox{0.7}{
\begin{tabular}{lcccc}
\toprule
 & \textbf{None} & \textbf{Municipal} & \textbf{Religious} & \textbf{Private} \\
\midrule
\textbf{Respondent's Baseline Info} \\
\quad Male &      0.06 &     -0.05 &      0.02 &     -0.03 \\
\quad  & (     0.05 ) & (     0.05 )  & (     0.02 )  & (     0.02 )  \\
\quad CAPI &     -0.01 &     -0.02 & \textbf{     0.04} &     -0.02 \\
\quad  & (     0.06 ) & (     0.05 )  & \textbf{(     0.02 )}  & (     0.03 )  \\
\midrule
\textbf{Mother's Baseline Info} \\
\quad Max Education: High School &      0.17 & \textbf{    -0.22} &      0.02 &      0.03 \\
\quad  & (     0.15 ) & \textbf{(     0.13 )}  & (     0.06 )  & (     0.07 )  \\
\quad Max Education: University &      0.18 &     -0.18 &     -0.01 &      0.01 \\
\quad  & (     0.15 ) & (     0.13 )  & (     0.06 )  & (     0.07 )  \\
\quad Born in Province &      0.07 &     -0.04 &     -0.01 &     -0.01 \\
\quad  & (     0.06 ) & (     0.05 )  & (     0.03 )  & (     0.03 )  \\
\midrule
\textbf{Father's Baseline Info} \\
\quad Max Education: High School &     -0.06 &      0.06 &      0.00 &      0.01 \\
\quad  & (     0.14 ) & (     0.12 )  & (     0.06 )  & (     0.06 )  \\
\quad Max Education: University &     -0.09 &     -0.01 &      0.04 &      0.05 \\
\quad  & (     0.15 ) & (     0.12 )  & (     0.06 )  & (     0.07 )  \\
\quad Born in Province &      0.06 &      0.03 &     -0.03 & \textbf{    -0.06} \\
\quad  & (     0.07 ) & (     0.06 )  & (     0.03 )  & \textbf{(     0.03 )}  \\
\midrule
\textbf{Household Baseline Info} \\
\quad Caregiver Has Religion & \textbf{     0.12} & \textbf{    -0.18} &      0.01 & \textbf{     0.05} \\
\quad  & \textbf{(     0.06 )} & \textbf{(     0.05 )}  & (     0.03 )  & \textbf{(     0.03 )}  \\
\midrule
Observations & 250 & 250 & 250 & 250 \\
Fraction Attending Each Type &      0.77 &      0.16 &      0.03 &      0.04 \\
\midrule
$ R^2$ &      0.04 &      0.08 &      0.03 &      0.05 \\
\bottomrule
\end{tabular}}
\end{table}
\begin{scriptsize}
\noindent\underline{Note:} This table presents the linear probability model estimations for attending each type of Asilo schools, indicated by each column. The samples used in this estimation are those who were adults in their 30's at the time of the survey living in Parma. All dependent variables are binary. Observation indicates the number of people included in this sample. Bold number indicates that the p-value is less than or equal to 0.1. Standard errors are reported in parentheses.
\end{scriptsize}

%\begin{table}[H]
\caption{LPM Estimation - Padova - Adults (Age 30), Asilo}
\centering
\scalebox{0.7}{
\begin{tabular}{lcccc}
\toprule
 & \textbf{None} & \textbf{Municipal} & \textbf{Religious} & \textbf{Private} \\
\midrule
\textbf{Respondent's Baseline Info} \\
\quad Male &     -0.03 &      0.03 &      0.00 &     -0.01 \\
\quad  & (     0.04 ) & (     0.03 )  & (     0.02 )  & (     0.01 )  \\
\quad CAPI &      0.04 &     -0.05 &      0.03 &     -0.01 \\
\quad  & (     0.04 ) & (     0.04 )  & (     0.02 )  & (     0.01 )  \\
\midrule
\textbf{Mother's Baseline Info} \\
\quad Max Education: Middle School &      0.30 & \textbf{    -0.33} &      0.03 &      0.01 \\
\quad  & (     0.21 ) & \textbf{(     0.18 )}  & (     0.10 )  & (     0.07 )  \\
\quad Max Education: High School &      0.23 &     -0.25 &      0.00 &      0.01 \\
\quad  & (     0.20 ) & (     0.17 )  & (     0.09 )  & (     0.07 )  \\
\quad Max Education: University &      0.24 &     -0.26 &      0.03 &     -0.00 \\
\quad  & (     0.20 ) & (     0.17 )  & (     0.09 )  & (     0.07 )  \\
\quad Born in Province &      0.02 &     -0.04 &     -0.00 &      0.02 \\
\quad  & (     0.05 ) & (     0.04 )  & (     0.02 )  & (     0.02 )  \\
\midrule
\textbf{Father's Baseline Info} \\
\quad Max Education: Middle School &      0.07 &     -0.10 &      0.04 &     -0.01 \\
\quad  & (     0.19 ) & (     0.16 )  & (     0.08 )  & (     0.07 )  \\
\quad Max Education: High School &      0.13 &     -0.12 &     -0.00 &     -0.01 \\
\quad  & (     0.17 ) & (     0.14 )  & (     0.08 )  & (     0.06 )  \\
\quad Max Education: University &      0.07 &     -0.10 &      0.02 &      0.02 \\
\quad  & (     0.17 ) & (     0.14 )  & (     0.08 )  & (     0.06 )  \\
\quad Born in Province &      0.05 &     -0.03 &      0.00 &     -0.02 \\
\quad  & (     0.05 ) & (     0.04 )  & (     0.02 )  & (     0.02 )  \\
\midrule
\textbf{Household Baseline Info} \\
\quad Caregiver Has Religion &     -0.01 &      0.01 &      0.00 &     -0.00 \\
\quad  & (     0.05 ) & (     0.04 )  & (     0.02 )  & (     0.02 )  \\
\midrule
Observations & 249 & 249 & 249 & 249 \\
Fraction Attending Each Type &      0.89 &      0.08 &      0.02 &      0.01 \\
\midrule
$ R^2$ &      0.03 &      0.04 &      0.02 &      0.03 \\
\bottomrule
\end{tabular}}
\end{table}
\begin{scriptsize}
\noindent\underline{Note:} This table presents the linear probability model estimations for attending each type of Asilo schools, indicated by each column. The samples used in this estimation are those who were adults in their 30's at the time of the survey living in Padova. All dependent variables are binary. Observation indicates the number of people included in this sample. Bold number indicates that the p-value is less than or equal to 0.1. Standard errors are reported in parentheses.
\end{scriptsize}


%\subsection{Adults at Age 40}
%\begin{table}[H]
\caption{LPM Estimation - Reggio - Adults (Age 40), Asilo}
\centering
\scalebox{0.7}{
\begin{tabular}{lcccc}
\toprule
 & \textbf{None} & \textbf{Municipal} & \textbf{Religious} & \textbf{Private} \\
\midrule
\textbf{Respondent's Baseline Info} \\
\quad Male &     -0.06 &      0.06 &      0.00 &      0.00 \\
\quad  & (     0.04 ) & (     0.04 )  & (        . )  & (        . )  \\
\quad CAPI &     -0.03 &      0.03 &      0.00 &      0.00 \\
\quad  & (     0.04 ) & (     0.04 )  & (        . )  & (        . )  \\
\midrule
\textbf{Mother's Baseline Info} \\
\quad Max Education: Middle School & \textbf{    -0.34} & \textbf{     0.34} &      0.00 &      0.00 \\
\quad  & \textbf{(     0.19 )} & \textbf{(     0.19 )}  & (        . )  & (        . )  \\
\quad Max Education: High School & \textbf{    -0.32} & \textbf{     0.32} &      0.00 &      0.00 \\
\quad  & \textbf{(     0.19 )} & \textbf{(     0.19 )}  & (        . )  & (        . )  \\
\quad Max Education: University & \textbf{    -0.32} & \textbf{     0.32} &      0.00 &      0.00 \\
\quad  & \textbf{(     0.19 )} & \textbf{(     0.19 )}  & (        . )  & (        . )  \\
\quad Born in Province &     -0.03 &      0.03 &      0.00 &      0.00 \\
\quad  & (     0.05 ) & (     0.05 )  & (        . )  & (        . )  \\
\midrule
\textbf{Father's Baseline Info} \\
\quad Max Education: Middle School &     -0.03 &      0.03 &      0.00 &      0.00 \\
\quad  & (     0.17 ) & (     0.17 )  & (        . )  & (        . )  \\
\quad Max Education: High School &      0.21 &     -0.21 &      0.00 &      0.00 \\
\quad  & (     0.17 ) & (     0.17 )  & (        . )  & (        . )  \\
\quad Max Education: University &      0.22 &     -0.22 &      0.00 &      0.00 \\
\quad  & (     0.17 ) & (     0.17 )  & (        . )  & (        . )  \\
\quad Born in Province &     -0.07 &      0.07 &      0.00 &      0.00 \\
\quad  & (     0.05 ) & (     0.05 )  & (        . )  & (        . )  \\
\midrule
\textbf{Household Baseline Info} \\
\quad Caregiver Has Religion &      0.05 &     -0.05 &      0.00 &      0.00 \\
\quad  & (     0.04 ) & (     0.04 )  & (        . )  & (        . )  \\
\midrule
Observations & 280 & 280 & 280 & 280 \\
Fraction Attending Each Type &      0.87 &      0.13 &      0.00 &      0.00 \\
\midrule
$ R^2$ &      0.14 &      0.14 &         . &         . \\
\bottomrule
\end{tabular}}
\end{table}
\begin{scriptsize}
\noindent\underline{Note:} This table presents the linear probability model estimations for attending each type of Asilo schools, indicated by each column. The samples used in this estimation are those who were adults in their 40's at the time of the survey living in Reggio. All dependent variables are binary. Observation indicates the number of people included in this sample. Bold number indicates that the p-value is less than or equal to 0.1. Standard errors are reported in parentheses.
\end{scriptsize}

%\begin{table}[H]
\caption{LPM Estimation - Parma - Adults (Age 40), Asilo}
\centering
\scalebox{0.7}{
\begin{tabular}{lcccc}
\toprule
 & \textbf{None} & \textbf{Municipal} & \textbf{Religious} & \textbf{Private} \\
\midrule
\textbf{Respondent's Baseline Info} \\
\quad Male &      0.00 &     -0.00 &     -0.01 &      0.01 \\
\quad  & (     0.04 ) & (     0.04 )  & (     0.01 )  & (     0.01 )  \\
\quad CAPI &     -0.04 &      0.03 &     -0.00 &      0.01 \\
\quad  & (     0.04 ) & (     0.04 )  & (     0.02 )  & (     0.01 )  \\
\midrule
\textbf{Mother's Baseline Info} \\
\quad Max Education: Middle School &     -0.06 &      0.04 &      0.01 &      0.01 \\
\quad  & (     0.45 ) & (     0.42 )  & (     0.16 )  & (     0.09 )  \\
\quad Max Education: High School &     -0.01 &     -0.02 &      0.01 &      0.02 \\
\quad  & (     0.45 ) & (     0.42 )  & (     0.16 )  & (     0.09 )  \\
\quad Max Education: University &      0.05 &     -0.03 &     -0.01 &     -0.01 \\
\quad  & (     0.46 ) & (     0.43 )  & (     0.16 )  & (     0.09 )  \\
\quad Born in Province &     -0.07 & \textbf{     0.08} &      0.01 & \textbf{    -0.02} \\
\quad  & (     0.05 ) & \textbf{(     0.04 )}  & (     0.02 )  & \textbf{(     0.01 )}  \\
\midrule
\textbf{Father's Baseline Info} \\
\quad Max Education: Middle School &     -0.19 &      0.19 &     -0.00 &      0.00 \\
\quad  & (     0.32 ) & (     0.30 )  & (     0.11 )  & (     0.06 )  \\
\quad Max Education: High School &     -0.07 &      0.09 &     -0.01 &     -0.00 \\
\quad  & (     0.32 ) & (     0.30 )  & (     0.11 )  & (     0.07 )  \\
\quad Max Education: University &     -0.18 &      0.14 &      0.02 &      0.02 \\
\quad  & (     0.33 ) & (     0.30 )  & (     0.11 )  & (     0.07 )  \\
\quad Born in Province &      0.04 &     -0.06 &      0.01 &      0.01 \\
\quad  & (     0.06 ) & (     0.05 )  & (     0.02 )  & (     0.01 )  \\
\midrule
\textbf{Household Baseline Info} \\
\quad Caregiver Has Religion &      0.02 &      0.00 &     -0.03 &      0.01 \\
\quad  & (     0.05 ) & (     0.04 )  & (     0.02 )  & (     0.01 )  \\
\midrule
Observations & 254 & 254 & 254 & 254 \\
Fraction Attending Each Type &      0.89 &      0.09 &      0.01 &      0.00 \\
\midrule
$ R^2$ &      0.06 &      0.06 &      0.03 &      0.05 \\
\bottomrule
\end{tabular}}
\end{table}
\begin{scriptsize}
\noindent\underline{Note:} This table presents the linear probability model estimations for attending each type of Asilo schools, indicated by each column. The samples used in this estimation are those who were adults in their 40's at the time of the survey living in Parma. All dependent variables are binary. Observation indicates the number of people included in this sample. Bold number indicates that the p-value is less than or equal to 0.1. Standard errors are reported in parentheses.
\end{scriptsize}

%\begin{table}[H]
\caption{LPM Estimation - Padova - Adults (Age 40), Asilo}
\centering
\scalebox{0.7}{
\begin{tabular}{lcccc}
\toprule
 & \textbf{None} & \textbf{Municipal} & \textbf{Religious} & \textbf{Private} \\
\midrule
\textbf{Respondent's Baseline Info} \\
\quad Male &     -0.03 & \textbf{     0.05} &     -0.02 &      0.00 \\
\quad  & (     0.04 ) & \textbf{(     0.03 )}  & (     0.02 )  & (        . )  \\
\quad CAPI & \textbf{    -0.10} & \textbf{     0.12} &     -0.02 &      0.00 \\
\quad  & \textbf{(     0.04 )} & \textbf{(     0.03 )}  & (     0.02 )  & (        . )  \\
\midrule
\textbf{Mother's Baseline Info} \\
\quad Max Education: Middle School &     -0.03 &      0.01 &      0.01 &      0.00 \\
\quad  & (     0.16 ) & (     0.13 )  & (     0.09 )  & (        . )  \\
\quad Max Education: High School &     -0.10 &      0.09 &      0.00 &      0.00 \\
\quad  & (     0.15 ) & (     0.13 )  & (     0.09 )  & (        . )  \\
\quad Max Education: University &     -0.10 &      0.06 &      0.04 &      0.00 \\
\quad  & (     0.16 ) & (     0.13 )  & (     0.09 )  & (        . )  \\
\quad Born in Province &      0.04 &     -0.02 &     -0.02 &      0.00 \\
\quad  & (     0.04 ) & (     0.03 )  & (     0.02 )  & (        . )  \\
\midrule
\textbf{Father's Baseline Info} \\
\quad Max Education: Middle School & \textbf{     0.88} & \textbf{    -0.92} &      0.04 &      0.00 \\
\quad  & \textbf{(     0.20 )} & \textbf{(     0.17 )}  & (     0.12 )  & (        . )  \\
\quad Max Education: High School & \textbf{     0.96} & \textbf{    -0.96} &     -0.01 &      0.00 \\
\quad  & \textbf{(     0.20 )} & \textbf{(     0.16 )}  & (     0.11 )  & (        . )  \\
\quad Max Education: University & \textbf{     1.02} & \textbf{    -1.02} &     -0.00 &      0.00 \\
\quad  & \textbf{(     0.20 )} & \textbf{(     0.16 )}  & (     0.11 )  & (        . )  \\
\quad Born in Province &      0.06 & \textbf{    -0.06} &      0.01 &      0.00 \\
\quad  & (     0.04 ) & \textbf{(     0.03 )}  & (     0.02 )  & (        . )  \\
\midrule
\textbf{Household Baseline Info} \\
\quad Caregiver Has Religion &     -0.03 &      0.03 &      0.00 &      0.00 \\
\quad  & (     0.04 ) & (     0.04 )  & (     0.02 )  & (        . )  \\
\midrule
Observations & 252 & 252 & 252 & 252 \\
Fraction Attending Each Type &      0.90 &      0.08 &      0.03 &      0.00 \\
\midrule
$ R^2$ &      0.16 &      0.25 &      0.03 &         . \\
\bottomrule
\end{tabular}}
\end{table}
\begin{scriptsize}
\noindent\underline{Note:} This table presents the linear probability model estimations for attending each type of Asilo schools, indicated by each column. The samples used in this estimation are those who were adults in their 40's at the time of the survey living in Padova. All dependent variables are binary. Observation indicates the number of people included in this sample. Bold number indicates that the p-value is less than or equal to 0.1. Standard errors are reported in parentheses.
\end{scriptsize}


%\subsection{Adults at Age 50}
%\begin{table}[H]
\caption{LPM Estimation - Reggio - Adults (Age 50), Asilo}
\centering
\scalebox{0.7}{
\begin{tabular}{lcccc}
\toprule
 & \textbf{None} & \textbf{Municipal} & \textbf{Religious} & \textbf{Private} \\
\midrule
\textbf{Respondent's Baseline Info} \\
\quad Male &      0.01 &      0.00 &     -0.01 &      0.00 \\
\quad  & (     0.01 ) & (        . )  & (     0.01 )  & (        . )  \\
\quad CAPI &      0.01 &      0.00 &     -0.01 &      0.00 \\
\quad  & (     0.01 ) & (        . )  & (     0.01 )  & (        . )  \\
\midrule
\textbf{Mother's Baseline Info} \\
\quad Max Education: Middle School &      0.00 &      0.00 &     -0.00 &      0.00 \\
\quad  & (     0.06 ) & (        . )  & (     0.06 )  & (        . )  \\
\quad Max Education: High School &      0.00 &      0.00 &     -0.00 &      0.00 \\
\quad  & (     0.06 ) & (        . )  & (     0.06 )  & (        . )  \\
\quad Max Education: University &     -0.00 &      0.00 &      0.00 &      0.00 \\
\quad  & (     0.06 ) & (        . )  & (     0.06 )  & (        . )  \\
\quad Born in Province &     -0.00 &      0.00 &      0.00 &      0.00 \\
\quad  & (     0.01 ) & (        . )  & (     0.01 )  & (        . )  \\
\midrule
\textbf{Father's Baseline Info} \\
\quad Max Education: Middle School &     -0.02 &      0.00 &      0.02 &      0.00 \\
\quad  & (     0.06 ) & (        . )  & (     0.06 )  & (        . )  \\
\quad Max Education: High School &     -0.01 &      0.00 &      0.01 &      0.00 \\
\quad  & (     0.06 ) & (        . )  & (     0.06 )  & (        . )  \\
\quad Max Education: University &     -0.01 &      0.00 &      0.01 &      0.00 \\
\quad  & (     0.06 ) & (        . )  & (     0.06 )  & (        . )  \\
\quad Born in Province &     -0.00 &      0.00 &      0.00 &      0.00 \\
\quad  & (     0.01 ) & (        . )  & (     0.01 )  & (        . )  \\
\midrule
\textbf{Household Baseline Info} \\
\quad Caregiver Has Religion &      0.01 &      0.00 &     -0.01 &      0.00 \\
\quad  & (     0.01 ) & (        . )  & (     0.01 )  & (        . )  \\
\midrule
Observations & 199 & 199 & 199 & 199 \\
Fraction Attending Each Type &      0.99 &      0.00 &      0.01 &      0.00 \\
\midrule
$ R^2$ &      0.03 &         . &      0.03 &         . \\
\bottomrule
\end{tabular}}
\end{table}
\begin{scriptsize}
\noindent\underline{Note:} This table presents the linear probability model estimations for attending each type of Asilo schools, indicated by each column. The samples used in this estimation are those who were adults in their 50's at the time of the survey living in Reggio. All dependent variables are binary. Observation indicates the number of people included in this sample. Bold number indicates that the p-value is less than or equal to 0.1. Standard errors are reported in parentheses.
\end{scriptsize}

%\begin{table}[H]
\caption{LPM Estimation - Parma - Adults (Age 50), Asilo}
\centering
\scalebox{0.7}{
\begin{tabular}{lcccc}
\toprule
 & \textbf{None} & \textbf{Municipal} & \textbf{Religious} & \textbf{Private} \\
\midrule
\textbf{Respondent's Baseline Info} \\
\quad Male &      0.04 &     -0.04 &      0.00 &      0.00 \\
\quad  & (     0.07 ) & (     0.06 )  & (     0.04 )  & (        . )  \\
\quad CAPI & \textbf{     0.18} & \textbf{    -0.16} &     -0.02 &      0.00 \\
\quad  & \textbf{(     0.07 )} & \textbf{(     0.06 )}  & (     0.04 )  & (        . )  \\
\midrule
\textbf{Mother's Baseline Info} \\
\quad Max Education: Middle School &      0.17 &      0.03 &     -0.21 &      0.00 \\
\quad  & (     0.24 ) & (     0.22 )  & (     0.14 )  & (        . )  \\
\quad Max Education: High School &      0.35 &     -0.12 &     -0.23 &      0.00 \\
\quad  & (     0.25 ) & (     0.23 )  & (     0.15 )  & (        . )  \\
\quad Max Education: University &      0.43 &     -0.18 &     -0.25 &      0.00 \\
\quad  & (     0.28 ) & (     0.25 )  & (     0.16 )  & (        . )  \\
\quad Born in Province &     -0.09 &      0.07 &      0.03 &      0.00 \\
\quad  & (     0.09 ) & (     0.08 )  & (     0.05 )  & (        . )  \\
\midrule
\textbf{Father's Baseline Info} \\
\quad Max Education: Middle School &     -0.19 &      0.21 &     -0.02 &      0.00 \\
\quad  & (     0.21 ) & (     0.19 )  & (     0.13 )  & (        . )  \\
\quad Max Education: High School &     -0.00 &      0.07 &     -0.07 &      0.00 \\
\quad  & (     0.21 ) & (     0.19 )  & (     0.13 )  & (        . )  \\
\quad Max Education: University &     -0.13 &      0.16 &     -0.04 &      0.00 \\
\quad  & (     0.24 ) & (     0.22 )  & (     0.14 )  & (        . )  \\
\quad Born in Province &     -0.10 & \textbf{     0.11} &     -0.02 &      0.00 \\
\quad  & (     0.08 ) & \textbf{(     0.07 )}  & (     0.05 )  & (        . )  \\
\midrule
\textbf{Household Baseline Info} \\
\quad Caregiver Has Religion &     -0.10 &      0.08 &      0.02 &      0.00 \\
\quad  & (     0.08 ) & (     0.07 )  & (     0.05 )  & (        . )  \\
\midrule
Observations & 103 & 103 & 103 & 103 \\
Fraction Attending Each Type &      0.83 &      0.13 &      0.04 &      0.00 \\
\midrule
$ R^2$ &      0.30 &      0.28 &      0.09 &         . \\
\bottomrule
\end{tabular}}
\end{table}
\begin{scriptsize}
\noindent\underline{Note:} This table presents the linear probability model estimations for attending each type of Asilo schools, indicated by each column. The samples used in this estimation are those who were adults in their 50's at the time of the survey living in Parma. All dependent variables are binary. Observation indicates the number of people included in this sample. Bold number indicates that the p-value is less than or equal to 0.1. Standard errors are reported in parentheses.
\end{scriptsize}

%\begin{table}[H]
\caption{LPM Estimation - Padova - Adults (Age 50), Asilo}
\centering
\scalebox{0.7}{
\begin{tabular}{lcccc}
\toprule
 & \textbf{None} & \textbf{Municipal} & \textbf{Religious} & \textbf{Private} \\
\midrule
\textbf{Respondent's Baseline Info} \\
\quad Male &     -0.04 &      0.00 &      0.04 &      0.00 \\
\quad  & (     0.03 ) & (        . )  & (     0.03 )  & (        . )  \\
\quad CAPI &      0.02 &      0.00 &     -0.02 &      0.00 \\
\quad  & (     0.04 ) & (        . )  & (     0.04 )  & (        . )  \\
\midrule
\textbf{Mother's Baseline Info} \\
\quad Max Education: Middle School &      0.02 &      0.00 &     -0.02 &      0.00 \\
\quad  & (     0.14 ) & (        . )  & (     0.14 )  & (        . )  \\
\quad Max Education: High School &     -0.11 &      0.00 &      0.11 &      0.00 \\
\quad  & (     0.15 ) & (        . )  & (     0.15 )  & (        . )  \\
\quad Max Education: University &      0.04 &      0.00 &     -0.04 &      0.00 \\
\quad  & (     0.15 ) & (        . )  & (     0.15 )  & (        . )  \\
\quad Born in Province &      0.04 &      0.00 &     -0.04 &      0.00 \\
\quad  & (     0.04 ) & (        . )  & (     0.04 )  & (        . )  \\
\midrule
\textbf{Father's Baseline Info} \\
\quad Max Education: Middle School &     -0.04 &      0.00 &      0.04 &      0.00 \\
\quad  & (     0.12 ) & (        . )  & (     0.12 )  & (        . )  \\
\quad Max Education: High School &     -0.04 &      0.00 &      0.04 &      0.00 \\
\quad  & (     0.13 ) & (        . )  & (     0.13 )  & (        . )  \\
\quad Max Education: University &     -0.03 &      0.00 &      0.03 &      0.00 \\
\quad  & (     0.13 ) & (        . )  & (     0.13 )  & (        . )  \\
\quad Born in Province &     -0.01 &      0.00 &      0.01 &      0.00 \\
\quad  & (     0.05 ) & (        . )  & (     0.05 )  & (        . )  \\
\midrule
\textbf{Household Baseline Info} \\
\quad Caregiver Has Religion &     -0.04 &      0.00 &      0.04 &      0.00 \\
\quad  & (     0.04 ) & (        . )  & (     0.04 )  & (        . )  \\
\midrule
Observations & 144 & 144 & 144 & 144 \\
Fraction Attending Each Type &      0.96 &      0.00 &      0.04 &      0.00 \\
\midrule
$ R^2$ &      0.09 &         . &      0.09 &         . \\
\bottomrule
\end{tabular}}
\end{table}
\begin{scriptsize}
\noindent\underline{Note:} This table presents the linear probability model estimations for attending each type of Asilo schools, indicated by each column. The samples used in this estimation are those who were adults in their 50's at the time of the survey living in Padova. All dependent variables are binary. Observation indicates the number of people included in this sample. Bold number indicates that the p-value is less than or equal to 0.1. Standard errors are reported in parentheses.
\end{scriptsize}


%\subsection{Migrants}
%\begin{table}[H]
\caption{LPM Estimation - Reggio - Migrants, Asilo}
\centering
\scalebox{0.7}{
\begin{tabular}{lcccc}
\toprule
 & \textbf{None} & \textbf{Municipal} & \textbf{Religious} & \textbf{Private} \\
\midrule
\textbf{Respondent's Baseline Info} \\
\quad Male &      0.01 &     -0.01 &     -0.00 &      0.00 \\
\quad  & (     0.10 ) & (     0.10 )  & (     0.03 )  & (        . )  \\
\quad CAPI &      0.18 &     -0.06 & \textbf{    -0.12} &      0.00 \\
\quad  & (     0.12 ) & (     0.12 )  & \textbf{(     0.04 )}  & (        . )  \\
\quad Low Birthweight &     -0.08 &      0.07 &      0.00 &      0.00 \\
\quad  & (     0.19 ) & (     0.19 )  & (     0.07 )  & (        . )  \\
\quad Premature at Birth &     -0.32 &      0.32 &      0.00 &      0.00 \\
\quad  & (     0.21 ) & (     0.21 )  & (     0.07 )  & (        . )  \\
\midrule
\textbf{Mother's Baseline Info} \\
\quad Max Education: High School &      0.00 &      0.01 &     -0.01 &      0.00 \\
\quad  & (     0.11 ) & (     0.10 )  & (     0.04 )  & (        . )  \\
\quad Max Education: University &     -0.25 &      0.29 &     -0.04 &      0.00 \\
\quad  & (     0.20 ) & (     0.19 )  & (     0.07 )  & (        . )  \\
\quad Teenager at Birth &      0.04 &      0.02 &     -0.06 &      0.00 \\
\quad  & (     0.21 ) & (     0.21 )  & (     0.07 )  & (        . )  \\
\midrule
\textbf{Father's Baseline Info} \\
\quad Max Education: High School &      0.05 &     -0.04 &     -0.01 &      0.00 \\
\quad  & (     0.11 ) & (     0.11 )  & (     0.04 )  & (        . )  \\
\quad Max Education: University &      0.09 &     -0.10 &      0.01 &      0.00 \\
\quad  & (     0.25 ) & (     0.25 )  & (     0.09 )  & (        . )  \\
\midrule
\textbf{Household Baseline Info} \\
\quad Caregiver Has Religion &     -0.19 &      0.17 &      0.02 &      0.00 \\
\quad  & (     0.14 ) & (     0.14 )  & (     0.05 )  & (        . )  \\
\quad Owns House &     -0.15 &      0.16 &     -0.01 &      0.00 \\
\quad  & (     0.14 ) & (     0.14 )  & (     0.05 )  & (        . )  \\
\quad Income 5K-10K Euro &     -0.19 &      0.26 &     -0.07 &      0.00 \\
\quad  & (     0.26 ) & (     0.26 )  & (     0.09 )  & (        . )  \\
\quad Income 10K-25K Euro &     -0.14 &      0.16 &     -0.02 &      0.00 \\
\quad  & (     0.12 ) & (     0.12 )  & (     0.04 )  & (        . )  \\
\quad Income 25K-50K Euro &     -0.02 &      0.04 &     -0.03 &      0.00 \\
\quad  & (     0.16 ) & (     0.16 )  & (     0.05 )  & (        . )  \\
\quad Income 50K-100K Euro &      0.31 &     -0.27 &     -0.04 &      0.00 \\
\quad  & (     0.38 ) & (     0.38 )  & (     0.13 )  & (        . )  \\
\midrule
Observations & 109 & 109 & 109 & 109 \\
Fraction Attending Each Type &      0.57 &      0.40 &      0.03 &      0.00 \\
\midrule
$ R^2$ &      0.16 &      0.15 &      0.10 &         . \\
\bottomrule
\end{tabular}}
\end{table}
\begin{scriptsize}
\noindent\underline{Note:} This table presents the linear probability model estimations for attending each type of Asilo schools, indicated by each column. The samples used in this estimation are those who were migrants at the time of the survey living in Reggio. All dependent variables are binary. Observation indicates the number of people included in this sample. Bold number indicates that the p-value is less than or equal to 0.1. Standard errors are reported in parentheses.
\end{scriptsize}

%\begin{table}[H]
\caption{LPM Estimation - Parma - Migrants, Asilo}
\centering
\scalebox{0.7}{
\begin{tabular}{lcccc}
\toprule
 & \textbf{None} & \textbf{Municipal} & \textbf{Religious} & \textbf{Private} \\
\midrule
\textbf{Respondent's Baseline Info} \\
\quad Male &     -0.08 &      0.08 &      0.04 &     -0.03 \\
\quad  & (     0.15 ) & (     0.15 )  & (     0.03 )  & (     0.04 )  \\
\quad CAPI &     -0.14 &      0.14 &     -0.02 &      0.02 \\
\quad  & (     0.17 ) & (     0.18 )  & (     0.04 )  & (     0.05 )  \\
\quad Low Birthweight &      0.07 &     -0.13 &      0.01 &      0.04 \\
\quad  & (     0.34 ) & (     0.36 )  & (     0.08 )  & (     0.10 )  \\
\quad Premature at Birth &     -0.19 &      0.26 &      0.01 &     -0.08 \\
\quad  & (     0.33 ) & (     0.34 )  & (     0.08 )  & (     0.10 )  \\
\midrule
\textbf{Mother's Baseline Info} \\
\quad Max Education: High School &     -0.19 &      0.14 &      0.00 &      0.05 \\
\quad  & (     0.15 ) & (     0.16 )  & (     0.04 )  & (     0.04 )  \\
\quad Max Education: University &     -0.04 &     -0.23 & \textbf{     0.25} &      0.02 \\
\quad  & (     0.28 ) & (     0.29 )  & \textbf{(     0.07 )}  & (     0.08 )  \\
\quad Teenager at Birth &     -0.46 &      0.58 &     -0.05 &     -0.07 \\
\quad  & (     0.52 ) & (     0.54 )  & (     0.12 )  & (     0.15 )  \\
\midrule
\textbf{Father's Baseline Info} \\
\quad Max Education: High School &     -0.12 &      0.02 &      0.06 &      0.04 \\
\quad  & (     0.16 ) & (     0.17 )  & (     0.04 )  & (     0.05 )  \\
\quad Max Education: University &     -0.05 &      0.11 &     -0.05 &     -0.01 \\
\quad  & (     0.24 ) & (     0.25 )  & (     0.06 )  & (     0.07 )  \\
\quad Born in Province & \textbf{     1.17} &     -0.93 &     -0.17 &     -0.07 \\
\quad  & \textbf{(     0.55 )} & (     0.57 )  & (     0.13 )  & (     0.16 )  \\
\midrule
\textbf{Household Baseline Info} \\
\quad Caregiver Has Religion &      0.58 &     -0.58 &      0.00 &     -0.00 \\
\quad  & (     0.40 ) & (     0.41 )  & (     0.09 )  & (     0.12 )  \\
\quad Owns House &     -0.35 &      0.24 & \textbf{     0.14} &     -0.03 \\
\quad  & (     0.24 ) & (     0.25 )  & \textbf{(     0.06 )}  & (     0.07 )  \\
\quad Income 5K-10K Euro &      0.27 &     -0.14 &     -0.07 &     -0.07 \\
\quad  & (     0.22 ) & (     0.23 )  & (     0.05 )  & (     0.06 )  \\
\quad Income 10K-25K Euro &     -0.17 &      0.20 &      0.02 &     -0.05 \\
\quad  & (     0.16 ) & (     0.17 )  & (     0.04 )  & (     0.05 )  \\
\quad Income 25K-50K Euro &     -0.03 &      0.18 &     -0.09 &     -0.05 \\
\quad  & (     0.31 ) & (     0.32 )  & (     0.07 )  & (     0.09 )  \\
\midrule
Observations & 58 & 58 & 58 & 58 \\
Fraction Attending Each Type &      0.41 &      0.55 &      0.02 &      0.02 \\
\midrule
$ R^2$ &      0.30 &      0.26 &      0.44 &      0.14 \\
\bottomrule
\end{tabular}}
\end{table}
\begin{scriptsize}
\noindent\underline{Note:} This table presents the linear probability model estimations for attending each type of Asilo schools, indicated by each column. The samples used in this estimation are those who were migrants at the time of the survey living in Parma. All dependent variables are binary. Observation indicates the number of people included in this sample. Bold number indicates that the p-value is less than or equal to 0.1. Standard errors are reported in parentheses.
\end{scriptsize}

%\begin{table}[H]
\caption{LPM Estimation - Padova - Migrants, Asilo}
\centering
\scalebox{0.7}{
\begin{tabular}{lcccc}
\toprule
 & \textbf{None} & \textbf{Municipal} & \textbf{Religious} & \textbf{Private} \\
\midrule
\textbf{Respondent's Baseline Info} \\
\quad Male &      0.13 &     -0.15 &      0.01 &      0.00 \\
\quad  & (     0.10 ) & (     0.10 )  & (     0.03 )  & (     0.03 )  \\
\quad CAPI &     -0.08 &      0.05 &      0.05 &     -0.02 \\
\quad  & (     0.10 ) & (     0.10 )  & (     0.03 )  & (     0.03 )  \\
\quad Low Birthweight &     -0.17 &      0.19 &     -0.01 &     -0.01 \\
\quad  & (     0.20 ) & (     0.19 )  & (     0.07 )  & (     0.06 )  \\
\quad Premature at Birth &      0.32 &     -0.28 &     -0.03 &     -0.02 \\
\quad  & (     0.24 ) & (     0.24 )  & (     0.08 )  & (     0.08 )  \\
\midrule
\textbf{Mother's Baseline Info} \\
\quad Max Education: High School &      0.12 &     -0.12 &      0.04 &     -0.04 \\
\quad  & (     0.10 ) & (     0.10 )  & (     0.03 )  & (     0.03 )  \\
\quad Max Education: University &     -0.22 &      0.26 &      0.06 &     -0.09 \\
\quad  & (     0.19 ) & (     0.19 )  & (     0.06 )  & (     0.06 )  \\
\quad Teenager at Birth &      0.63 &     -0.58 &     -0.07 &      0.02 \\
\quad  & (     0.51 ) & (     0.51 )  & (     0.17 )  & (     0.16 )  \\
\midrule
\textbf{Father's Baseline Info} \\
\quad Max Education: High School &      0.05 &     -0.05 &     -0.03 &      0.04 \\
\quad  & (     0.10 ) & (     0.10 )  & (     0.03 )  & (     0.03 )  \\
\quad Max Education: University &     -0.12 &     -0.06 & \textbf{     0.16} &      0.01 \\
\quad  & (     0.24 ) & (     0.24 )  & \textbf{(     0.08 )}  & (     0.08 )  \\
\midrule
\textbf{Household Baseline Info} \\
\quad Caregiver Has Religion &      0.43 &     -0.50 &      0.08 &     -0.00 \\
\quad  & (     0.51 ) & (     0.50 )  & (     0.17 )  & (     0.16 )  \\
\quad Owns House &      0.05 &     -0.03 &     -0.03 &      0.01 \\
\quad  & (     0.11 ) & (     0.11 )  & (     0.04 )  & (     0.03 )  \\
\quad Income 5K-10K Euro &     -0.12 &      0.10 &      0.01 &      0.01 \\
\quad  & (     0.17 ) & (     0.17 )  & (     0.06 )  & (     0.05 )  \\
\quad Income 10K-25K Euro &      0.17 &     -0.16 &     -0.04 &      0.03 \\
\quad  & (     0.13 ) & (     0.13 )  & (     0.04 )  & (     0.04 )  \\
\quad Income 25K-50K Euro &     -0.51 &      0.04 &     -0.04 & \textbf{     0.51} \\
\quad  & (     0.39 ) & (     0.39 )  & (     0.13 )  & \textbf{(     0.13 )}  \\
\midrule
Observations & 111 & 111 & 111 & 111 \\
Fraction Attending Each Type &      0.56 &      0.39 &      0.03 &      0.03 \\
\midrule
$ R^2$ &      0.17 &      0.15 &      0.12 &      0.21 \\
\bottomrule
\end{tabular}}
\end{table}
\begin{scriptsize}
\noindent\underline{Note:} This table presents the linear probability model estimations for attending each type of Asilo schools, indicated by each column. The samples used in this estimation are those who were migrants at the time of the survey living in Padova. All dependent variables are binary. Observation indicates the number of people included in this sample. Bold number indicates that the p-value is less than or equal to 0.1. Standard errors are reported in parentheses.
\end{scriptsize}




\section{Materna (Age 3-5)}
\subsection{Children}
\subsubsection{Unconditional Mean}
\begin{table}[H]
\caption{Baseline, Reggio, Children}
\scalebox{0.85}{
\begin{tabular}{l c c c c c c }
\toprule
& \textbf{Municipal} & \textbf{State} & \textbf{Religious} & \textbf{Private} & \textbf{None} \\
\midrule
Male indicator &      0.55 &      0.53 &      0.53 &      0.40 &      0.50 \\
\midrule
Observations &       166 &        45 &        92 &         5 &         2
Age &      6.80 & \textbf{     6.92} &      6.74 &      6.69 &      6.93 \\
\midrule
Observations &       166 &        45 &        92 &         5 &         2
Low birthweight &      0.10 &      0.07 &      0.05 & \textbf{     0.00} & \textbf{     0.00} \\
\midrule
Observations &       166 &        45 &        92 &         5 &         2
Premature birth &      0.10 &      0.09 &      0.10 & \textbf{     0.00} & \textbf{     0.00} \\
\midrule
Observations &       166 &        45 &        92 &         5 &         2
CAPI &      0.60 & \textbf{     0.40} &      0.55 &      0.40 &      0.50 \\
\midrule
Observations &       166 &        45 &        92 &         5 &         2
Mother: age at birth &     33.13 &     32.97 &     32.47 &     37.28 &     34.65 \\
\midrule
Observations &       165 &        44 &        92 &         5 &         2
Mother: born in province &      0.51 &      0.38 &      0.59 &      0.40 & \textbf{     1.00} \\
\midrule
Observations &       166 &        45 &        92 &         5 &         2
Mother max. edu.: less than middle school &      0.15 & \textbf{     0.29} &      0.14 & \textbf{     0.00} & \textbf{     1.00} \\
\midrule
Observations &       166 &        45 &        92 &         5 &         2
Mother max. edu.: middle school &      0.08 & \textbf{     0.02} &      0.11 & \textbf{     0.00} & \textbf{     0.00} \\
\midrule
Observations &       166 &        45 &        92 &         5 &         2
Mother max. edu.: high school &      0.46 &      0.44 &      0.43 &      0.60 & \textbf{     0.00} \\
\midrule
Observations &       166 &        45 &        92 &         5 &         2
Mother max. edu.: university &      0.29 & \textbf{     0.18} &      0.32 &      0.40 & \textbf{     0.00} \\
\midrule
Observations &       166 &        45 &        92 &         5 &         2
Father: age at birth &     35.48 &     36.75 &     35.55 &     35.95 &     36.54 \\
\midrule
Observations &       148 &        41 &        89 &         4 &         2
Father: born in province &      0.51 &      0.44 &      0.57 &      0.40 & \textbf{     1.00} \\
\midrule
Observations &       166 &        45 &        92 &         5 &         2
Father max. edu.: less than middle school &      0.22 &      0.29 &      0.24 & \textbf{     0.00} & \textbf{     0.00} \\
\midrule
Observations &       166 &        45 &        92 &         5 &         2
Father max. edu.: middle school &      0.08 &      0.11 &      0.05 & \textbf{     0.00} &      0.50 \\
\midrule
Observations &       166 &        45 &        92 &         5 &         2
Father max. edu.: high school &      0.34 &      0.31 &      0.38 &      0.40 &      0.50 \\
\midrule
Observations &       166 &        45 &        92 &         5 &         2
Father max. edu.: university &      0.23 &      0.18 &      0.29 &      0.40 & \textbf{     0.00} \\
\midrule
Observations &       166 &        45 &        92 &         5 &         2
Number of siblings &      1.01 &      1.02 &      1.04 &      0.60 &      0.50 \\
\midrule
Observations &       166 &        45 &        92 &         5 &         2
Religious caregiver indicator &      0.81 &      0.89 & \textbf{     0.90} &      0.80 &      0.50 \\
\midrule
Observations &       166 &        45 &        92 &         5 &         2
Mother: born outside of Italy &      0.08 &      0.16 & \textbf{     0.02} & \textbf{     0.00} & \textbf{     0.00} \\
\midrule
Observations &       166 &        45 &        92 &         5 &         2
Income: 5,000 euros or less &      0.00 &      0.02 & \textbf{     0.03} &      0.00 &      0.00 \\
\midrule
Observations &       166 &        45 &        92 &         5 &         2
Income: 5,001-10,000 euros &      0.01 &      0.00 &      0.02 &      0.00 &      0.00 \\
\midrule
Observations &       166 &        45 &        92 &         5 &         2
Income: 10,001-25,000 euros &      0.17 &      0.22 &      0.16 & \textbf{     0.00} & \textbf{     0.00} \\
\midrule
Observations &       166 &        45 &        92 &         5 &         2
Income: 25,001-50,000 euros &      0.34 &      0.29 &      0.30 &      0.40 & \textbf{     0.00} \\
\midrule
Observations &       166 &        45 &        92 &         5 &         2
Income: 50,001-100,000 euros &      0.19 & \textbf{     0.07} &      0.27 &      0.20 & \textbf{     0.00} \\
\midrule
Observations &       166 &        45 &        92 &         5 &         2
Income: 100,001-250,000 euros &      0.04 & \textbf{     0.00} &      0.01 & \textbf{     0.00} & \textbf{     0.00} \\
\midrule
Observations &       166 &        45 &        92 &         5 &         2
Income: more than 250,000 euros &      0.00 &      0.00 &      0.00 &      0.00 &      0.00 \\
\midrule
Observations &       166 &        45 &        92 &         5 &         2
\bottomrule
\end{tabular}

}
\end{table}

\begin{table}[H]
\caption{Baseline, Parma, Children}
\scalebox{0.85}{
\begin{tabular}{l c c c c c c }
\toprule
& \textbf{Municipal} & \textbf{State} & \textbf{Religious} & \textbf{Private} & \textbf{None} \\
\midrule
Male indicator &      0.54 &      0.56 &      0.57 &      0.67 &      0.67 \\
\midrule
Observations &       154 &        43 &        77 &         9 &         6
Age & \textbf{     6.70} &      6.71 & \textbf{     6.70} &      6.83 &      6.80 \\
\midrule
Observations &       154 &        43 &        77 &         9 &         6
Low birthweight & \textbf{     0.04} &      0.14 &      0.08 &      0.11 & \textbf{     0.00} \\
\midrule
Observations &       154 &        43 &        77 &         9 &         6
Premature birth & \textbf{     0.03} &      0.14 &      0.12 &      0.22 & \textbf{     0.00} \\
\midrule
Observations &       154 &        43 &        77 &         9 &         6
CAPI & \textbf{     0.42} & \textbf{     0.37} & \textbf{     0.44} &      0.78 &      0.50 \\
\midrule
Observations &       154 &        43 &        77 &         9 &         6
Mother: age at birth & \textbf{    34.27} &     33.96 &     33.21 &     33.11 & \textbf{    35.30} \\
\midrule
Observations &       153 &        43 &        76 &         9 &         6
Mother: born in province & \textbf{     0.60} & \textbf{     0.72} &      0.56 &      0.67 &      0.50 \\
\midrule
Observations &       154 &        43 &        77 &         9 &         6
Mother max. edu.: less than middle school & \textbf{     0.08} & \textbf{     0.00} &      0.09 & \textbf{     0.00} & \textbf{     0.00} \\
\midrule
Observations &       154 &        43 &        77 &         9 &         6
Mother max. edu.: middle school &      0.06 & \textbf{     0.02} &      0.08 & \textbf{     0.00} & \textbf{     0.00} \\
\midrule
Observations &       154 &        43 &        77 &         9 &         6
Mother max. edu.: high school &      0.38 &      0.44 &      0.44 &      0.56 &      0.17 \\
\midrule
Observations &       154 &        43 &        77 &         9 &         6
Mother max. edu.: university & \textbf{     0.46} & \textbf{     0.53} &      0.39 &      0.44 & \textbf{     0.83} \\
\midrule
Observations &       154 &        43 &        77 &         9 &         6
Father: age at birth & \textbf{    36.63} &     35.30 &     35.61 &     35.40 &     38.01 \\
\midrule
Observations &       141 &        42 &        73 &         9 &         6
Father: born in province &      0.58 & \textbf{     0.70} & \textbf{     0.62} &      0.33 & \textbf{     0.17} \\
\midrule
Observations &       154 &        43 &        77 &         9 &         6
Father max. edu.: less than middle school &      0.16 & \textbf{     0.05} & \textbf{     0.08} &      0.11 &      0.33 \\
\midrule
Observations &       154 &        43 &        77 &         9 &         6
Father max. edu.: middle school &      0.08 &      0.12 &      0.13 &      0.11 & \textbf{     0.00} \\
\midrule
Observations &       154 &        43 &        77 &         9 &         6
Father max. edu.: high school &      0.34 &      0.37 &      0.39 &      0.44 &      0.33 \\
\midrule
Observations &       154 &        43 &        77 &         9 &         6
Father max. edu.: university & \textbf{     0.33} & \textbf{     0.44} & \textbf{     0.35} &      0.33 &      0.33 \\
\midrule
Observations &       154 &        43 &        77 &         9 &         6
Number of siblings &      1.11 &      1.02 & \textbf{     0.77} &      1.33 &      0.83 \\
\midrule
Observations &       154 &        43 &        77 &         9 &         6
Religious caregiver indicator &      0.86 & \textbf{     0.91} &      0.86 & \textbf{     1.00} &      0.83 \\
\midrule
Observations &       154 &        43 &        77 &         9 &         6
Mother: born outside of Italy & \textbf{     0.03} & \textbf{     0.02} & \textbf{     0.00} & \textbf{     0.00} & \textbf{     0.00} \\
\midrule
Observations &       154 &        43 &        77 &         9 &         6
Income: 5,000 euros or less & \textbf{     0.02} &      0.02 &      0.03 &      0.11 &      0.00 \\
\midrule
Observations &       154 &        43 &        77 &         9 &         6
Income: 5,001-10,000 euros &      0.01 &      0.00 &      0.04 &      0.00 &      0.00 \\
\midrule
Observations &       154 &        43 &        77 &         9 &         6
Income: 10,001-25,000 euros &      0.21 &      0.21 &      0.13 & \textbf{     0.00} &      0.33 \\
\midrule
Observations &       154 &        43 &        77 &         9 &         6
Income: 25,001-50,000 euros & \textbf{     0.45} &      0.37 &      0.31 &      0.44 &      0.67 \\
\midrule
Observations &       154 &        43 &        77 &         9 &         6
Income: 50,001-100,000 euros &      0.17 &      0.26 &      0.23 &      0.11 & \textbf{     0.00} \\
\midrule
Observations &       154 &        43 &        77 &         9 &         6
Income: 100,001-250,000 euros &      0.02 &      0.02 &      0.03 & \textbf{     0.00} & \textbf{     0.00} \\
\midrule
Observations &       154 &        43 &        77 &         9 &         6
Income: more than 250,000 euros &      0.00 &      0.00 &      0.00 &      0.00 &      0.00 \\
\midrule
Observations &       154 &        43 &        77 &         9 &         6
\bottomrule
\end{tabular}

}
\end{table}

\begin{table}[H]
\caption{Baseline, Padova, Children}
\scalebox{0.85}{
\begin{tabular}{l c c c c c c }
\toprule
& \textbf{Municipal} & \textbf{State} & \textbf{Religious} & \textbf{Private} & \textbf{None} \\
\midrule
Male indicator &      0.59 &      0.63 &      0.48 & \textbf{     0.25} &      0.50 \\
\midrule
Observations &        82 &        40 &       141 &        12 &         2
Age & \textbf{     6.66} & \textbf{     6.66} & \textbf{     6.67} &      6.74 & \textbf{     6.39} \\
\midrule
Observations &        82 &        40 &       141 &        12 &         2
Low birthweight &      0.07 &      0.05 & \textbf{     0.03} &      0.08 & \textbf{     0.00} \\
\midrule
Observations &        82 &        40 &       141 &        12 &         2
Premature birth &      0.06 &      0.07 &      0.08 & \textbf{     0.00} & \textbf{     0.00} \\
\midrule
Observations &        82 &        40 &       141 &        12 &         2
CAPI & \textbf{     0.45} &      0.55 & \textbf{     0.48} &      0.42 & \textbf{     0.00} \\
\midrule
Observations &        82 &        40 &       141 &        12 &         2
Mother: age at birth & \textbf{    34.60} & \textbf{    34.50} & \textbf{    34.05} & \textbf{    36.41} &     34.92 \\
\midrule
Observations &        82 &        39 &       141 &        12 &         2
Mother: born in province & \textbf{     0.62} & \textbf{     0.70} & \textbf{     0.74} &      0.58 & \textbf{     0.00} \\
\midrule
Observations &        82 &        40 &       141 &        12 &         2
Mother max. edu.: less than middle school & \textbf{     0.07} &      0.17 &      0.09 & \textbf{     0.00} &      0.50 \\
\midrule
Observations &        82 &        40 &       141 &        12 &         2
Mother max. edu.: middle school &      0.06 &      0.05 &      0.12 &      0.08 & \textbf{     0.00} \\
\midrule
Observations &        82 &        40 &       141 &        12 &         2
Mother max. edu.: high school &      0.43 &      0.40 &      0.47 &      0.58 & \textbf{     0.00} \\
\midrule
Observations &        82 &        40 &       141 &        12 &         2
Mother max. edu.: university & \textbf{     0.44} &      0.35 &      0.32 &      0.33 &      0.50 \\
\midrule
Observations &        82 &        40 &       141 &        12 &         2
Father: age at birth & \textbf{    38.08} & \textbf{    37.43} & \textbf{    37.20} & \textbf{    39.22} &     40.28 \\
\midrule
Observations &        73 &        34 &       130 &        11 &         2
Father: born in province &      0.61 &      0.50 & \textbf{     0.70} &      0.67 &      0.50 \\
\midrule
Observations &        82 &        40 &       141 &        12 &         2
Father max. edu.: less than middle school & \textbf{     0.11} & \textbf{     0.10} & \textbf{     0.09} & \textbf{     0.00} &      0.50 \\
\midrule
Observations &        82 &        40 &       141 &        12 &         2
Father max. edu.: middle school &      0.09 &      0.05 &      0.11 & \textbf{     0.00} & \textbf{     0.00} \\
\midrule
Observations &        82 &        40 &       141 &        12 &         2
Father max. edu.: high school &      0.40 &      0.40 &      0.43 &      0.50 & \textbf{     0.00} \\
\midrule
Observations &        82 &        40 &       141 &        12 &         2
Father max. edu.: university &      0.29 &      0.30 &      0.30 &      0.42 &      0.50 \\
\midrule
Observations &        82 &        40 &       141 &        12 &         2
Number of siblings & \textbf{     1.26} &      0.93 & \textbf{     0.82} &      0.75 & \textbf{     2.00} \\
\midrule
Observations &        82 &        40 &       141 &        12 &         2
Religious caregiver indicator &      0.76 &      0.75 &      0.83 &      0.92 & \textbf{     1.00} \\
\midrule
Observations &        82 &        40 &       141 &        12 &         2
Mother: born outside of Italy & \textbf{     0.02} & \textbf{     0.03} & \textbf{     0.01} &      0.08 & \textbf{     0.00} \\
\midrule
Observations &        82 &        40 &       141 &        12 &         2
Income: 5,000 euros or less &      0.01 &      0.03 & \textbf{     0.05} &      0.00 &      0.00 \\
\midrule
Observations &        82 &        40 &       141 &        12 &         2
Income: 5,001-10,000 euros &      0.05 &      0.00 &      0.00 &      0.00 &      0.00 \\
\midrule
Observations &        82 &        40 &       141 &        12 &         2
Income: 10,001-25,000 euros &      0.13 &      0.25 &      0.16 & \textbf{     0.00} & \textbf{     0.00} \\
\midrule
Observations &        82 &        40 &       141 &        12 &         2
Income: 25,001-50,000 euros &      0.34 &      0.25 &      0.33 &      0.17 & \textbf{     0.00} \\
\midrule
Observations &        82 &        40 &       141 &        12 &         2
Income: 50,001-100,000 euros &      0.17 &      0.10 &      0.13 & \textbf{     0.00} & \textbf{     0.00} \\
\midrule
Observations &        82 &        40 &       141 &        12 &         2
Income: 100,001-250,000 euros &      0.01 &      0.03 &      0.03 &      0.08 &      0.50 \\
\midrule
Observations &        82 &        40 &       141 &        12 &         2
Income: more than 250,000 euros &      0.00 &      0.00 &      0.00 &      0.00 &      0.00 \\
\midrule
Observations &        82 &        40 &       141 &        12 &         2
\bottomrule
\end{tabular}

}
\end{table}

\subsubsection{Linear Probability Model}
\begin{table}[H]
\caption{LPM Estimation Reggio - Children, Materna}
\centering
\scalebox{0.7}{
\begin{tabular}{lccccc}
\toprule
 & \textbf{None} & \textbf{Municipal} & \textbf{Religious} & \textbf{Private} & \textbf{State} \\
\midrule
\textbf{Respondent's Baseline Info} \\
\quad Male &     -0.01 &      0.02 &     -0.01 &     -0.01 &      0.00 \\
\quad  & (     0.01 ) & (     0.06 )  & (     0.05 )  & (     0.02 ) & (     0.04 ) \\
\quad CAPI &      0.00 & \textbf{     0.12} &     -0.01 &     -0.01 & \textbf{    -0.11} \\
\quad  & (     0.01 ) & \textbf{(     0.06 )}  & (     0.05 )  & (     0.02 ) & \textbf{(     0.04 )} \\
\quad Low Birthweight &      0.00 & \textbf{     0.28} &     -0.20 &     -0.01 &     -0.07 \\
\quad  & (     0.02 ) & \textbf{(     0.15 )}  & (     0.14 )  & (     0.04 ) & (     0.11 ) \\
\quad Premature at Birth &     -0.01 &     -0.14 &      0.14 &     -0.01 &      0.02 \\
\quad  & (     0.02 ) & (     0.14 )  & (     0.12 )  & (     0.04 ) & (     0.10 ) \\
\midrule
\textbf{Mother's Baseline Info} \\
\quad Max Education: Middle School & \textbf{    -0.05} &      0.14 &      0.14 &      0.01 & \textbf{    -0.24} \\
\quad  & \textbf{(     0.02 )} & (     0.13 )  & (     0.12 )  & (     0.03 ) & \textbf{(     0.09 )} \\
\quad Max Education: High School & \textbf{    -0.04} &      0.08 &      0.02 &      0.02 &     -0.09 \\
\quad  & \textbf{(     0.01 )} & (     0.09 )  & (     0.08 )  & (     0.02 ) & (     0.06 ) \\
\quad Max Education: University & \textbf{    -0.03} &      0.07 &      0.05 &      0.02 &     -0.12 \\
\quad  & \textbf{(     0.02 )} & (     0.10 )  & (     0.09 )  & (     0.03 ) & (     0.07 ) \\
\quad Born in Province &      0.01 &      0.00 &      0.05 &     -0.01 &     -0.06 \\
\quad  & (     0.01 ) & (     0.06 )  & (     0.06 )  & (     0.02 ) & (     0.04 ) \\
\midrule
\textbf{Father's Baseline Info} \\
\quad Max Education: Middle School & \textbf{     0.05} &     -0.05 &     -0.09 &     -0.01 &      0.10 \\
\quad  & \textbf{(     0.02 )} & (     0.12 )  & (     0.11 )  & (     0.03 ) & (     0.08 ) \\
\quad Max Education: High School &      0.02 &     -0.05 &      0.02 &      0.01 &      0.01 \\
\quad  & (     0.01 ) & (     0.08 )  & (     0.07 )  & (     0.02 ) & (     0.05 ) \\
\quad Max Education: University &      0.01 &     -0.06 &      0.03 &      0.01 &      0.01 \\
\quad  & (     0.01 ) & (     0.09 )  & (     0.08 )  & (     0.02 ) & (     0.06 ) \\
\quad Born in Province &      0.01 &      0.01 &      0.00 &     -0.01 &     -0.01 \\
\quad  & (     0.01 ) & (     0.07 )  & (     0.06 )  & (     0.02 ) & (     0.05 ) \\
\midrule
\textbf{Household Baseline Info} \\
\quad Caregiver Has Religion &     -0.02 & \textbf{    -0.19} & \textbf{     0.16} &     -0.01 &      0.06 \\
\quad  & (     0.01 ) & \textbf{(     0.08 )}  & \textbf{(     0.08 )}  & (     0.02 ) & (     0.06 ) \\
\quad Owns House &     -0.00 & \textbf{    -0.13} & \textbf{     0.18} &     -0.01 &     -0.04 \\
\quad  & (     0.01 ) & \textbf{(     0.06 )}  & \textbf{(     0.05 )}  & (     0.02 ) & (     0.04 ) \\
\quad Income 5K-10K Euro &     -0.01 &      0.01 &      0.18 &     -0.02 &     -0.16 \\
\quad  & (     0.04 ) & (     0.26 )  & (     0.23 )  & (     0.07 ) & (     0.18 ) \\
\quad Income 10K-25K Euro &     -0.01 &     -0.04 &      0.08 &     -0.03 &     -0.01 \\
\quad  & (     0.01 ) & (     0.09 )  & (     0.08 )  & (     0.02 ) & (     0.07 ) \\
\quad Income 25K-50K Euro &     -0.01 &      0.04 &      0.02 &     -0.01 &     -0.04 \\
\quad  & (     0.01 ) & (     0.08 )  & (     0.07 )  & (     0.02 ) & (     0.06 ) \\
\quad Income 50K-100K Euro &     -0.01 &      0.04 &      0.11 &     -0.02 & \textbf{    -0.12} \\
\quad  & (     0.01 ) & (     0.09 )  & (     0.08 )  & (     0.02 ) & \textbf{(     0.06 )} \\
\quad Income 100K-250K Euro &     -0.00 & \textbf{     0.46} &     -0.21 &     -0.04 &     -0.20 \\
\quad  & (     0.03 ) & \textbf{(     0.20 )}  & (     0.18 )  & (     0.05 ) & (     0.14 ) \\
\midrule
Observations & 310 & 310 & 310 & 310 & 310 \\
Fraction Attending Each Type &      0.01 &      0.54 &      0.30 &      0.02 &      0.15 \\
\midrule
$ R^2$ &      0.08 &      0.07 &      0.09 &      0.02 &      0.10 \\
\bottomrule
\end{tabular}}
\end{table}
\begin{scriptsize}
\noindent\underline{Note:} This table presents the linear probability model estimations for attending each type of Materna schools, indicated by each column. The samples used in this estimation are those who were children at the time of the survey living in Reggio. All dependent variables are binary. Observation indicates the number of people included in this sample. Bold number indicates that the p-value is less than or equal to 0.1. Standard errors are reported in parentheses.
\end{scriptsize}

\begin{table}[H]
\caption{LPM Estimation Parma - Children, Materna}
\centering
\scalebox{0.7}{
\begin{tabular}{lccccc}
\toprule
 & \textbf{None} & \textbf{Municipal} & \textbf{Religious} & \textbf{Private} & \textbf{State} \\
\midrule
\textbf{Respondent's Baseline Info} \\
\quad Male &      0.01 &      0.01 &     -0.01 &      0.01 &     -0.02 \\
\quad  & (     0.02 ) & (     0.06 )  & (     0.05 )  & (     0.02 ) & (     0.04 ) \\
\quad CAPI &      0.00 &     -0.01 &      0.01 & \textbf{     0.04} &     -0.04 \\
\quad  & (     0.02 ) & (     0.06 )  & (     0.05 )  & \textbf{(     0.02 )} & (     0.04 ) \\
\quad Low Birthweight &     -0.00 &     -0.01 &     -0.06 &     -0.04 &      0.11 \\
\quad  & (     0.04 ) & (     0.15 )  & (     0.13 )  & (     0.05 ) & (     0.11 ) \\
\quad Premature at Birth &     -0.02 & \textbf{    -0.30} &      0.18 &      0.08 &      0.06 \\
\quad  & (     0.04 ) & \textbf{(     0.14 )}  & (     0.13 )  & (     0.05 ) & (     0.10 ) \\
\midrule
\textbf{Mother's Baseline Info} \\
\quad Max Education: Middle School &      0.03 &      0.09 &     -0.03 &     -0.05 &     -0.04 \\
\quad  & (     0.06 ) & (     0.19 )  & (     0.17 )  & (     0.07 ) & (     0.14 ) \\
\quad Max Education: High School &      0.03 &     -0.11 &     -0.07 &      0.03 &      0.12 \\
\quad  & (     0.04 ) & (     0.13 )  & (     0.11 )  & (     0.04 ) & (     0.09 ) \\
\quad Max Education: University & \textbf{     0.06} &     -0.06 &     -0.15 &      0.02 &      0.12 \\
\quad  & \textbf{(     0.04 )} & (     0.13 )  & (     0.12 )  & (     0.05 ) & (     0.09 ) \\
\quad Teenager at Birth &      0.01 &     -0.83 &      0.55 &      0.05 &      0.22 \\
\quad  & (     0.17 ) & (     0.57 )  & (     0.51 )  & (     0.20 ) & (     0.41 ) \\
\quad Born in Province &     -0.00 &      0.00 &     -0.06 &      0.02 &      0.04 \\
\quad  & (     0.02 ) & (     0.07 )  & (     0.06 )  & (     0.02 ) & (     0.05 ) \\
\midrule
\textbf{Father's Baseline Info} \\
\quad Max Education: Middle School &     -0.02 & \textbf{    -0.35} &      0.16 &      0.06 &      0.16 \\
\quad  & (     0.04 ) & \textbf{(     0.14 )}  & (     0.12 )  & (     0.05 ) & (     0.10 ) \\
\quad Max Education: High School &     -0.01 & \textbf{    -0.18} &      0.10 &      0.02 &      0.08 \\
\quad  & (     0.03 ) & \textbf{(     0.09 )}  & (     0.08 )  & (     0.03 ) & (     0.07 ) \\
\quad Max Education: University &     -0.01 & \textbf{    -0.20} &      0.08 &      0.01 &      0.12 \\
\quad  & (     0.03 ) & \textbf{(     0.10 )}  & (     0.09 )  & (     0.03 ) & (     0.07 ) \\
\quad Teenager at Birth &      0.04 &      0.41 &      0.06 &     -0.10 &     -0.41 \\
\quad  & (     0.22 ) & (     0.77 )  & (     0.68 )  & (     0.27 ) & (     0.55 ) \\
\quad Born in Province & \textbf{    -0.04} &      0.03 &      0.04 & \textbf{    -0.05} &      0.02 \\
\quad  & \textbf{(     0.02 )} & (     0.07 )  & (     0.06 )  & \textbf{(     0.02 )} & (     0.05 ) \\
\midrule
\textbf{Household Baseline Info} \\
\quad Caregiver Has Religion &     -0.00 &     -0.05 &     -0.03 &      0.04 &      0.05 \\
\quad  & (     0.03 ) & (     0.09 )  & (     0.08 )  & (     0.03 ) & (     0.06 ) \\
\quad Owns House &      0.01 &     -0.02 &     -0.00 &      0.01 &     -0.00 \\
\quad  & (     0.02 ) & (     0.07 )  & (     0.06 )  & (     0.02 ) & (     0.05 ) \\
\quad Income 5K-10K Euro &      0.01 &      0.13 &      0.08 &     -0.09 &     -0.13 \\
\quad  & (     0.08 ) & (     0.27 )  & (     0.24 )  & (     0.09 ) & (     0.19 ) \\
\quad Income 10K-25K Euro &      0.04 &      0.13 & \textbf{    -0.18} & \textbf{    -0.08} &      0.08 \\
\quad  & (     0.03 ) & (     0.10 )  & \textbf{(     0.09 )}  & \textbf{(     0.04 )} & (     0.07 ) \\
\quad Income 25K-50K Euro &      0.04 & \textbf{     0.16} & \textbf{    -0.16} &     -0.04 &     -0.00 \\
\quad  & (     0.02 ) & \textbf{(     0.09 )}  & \textbf{(     0.08 )}  & (     0.03 ) & (     0.06 ) \\
\quad Income 50K-100K Euro &     -0.00 &      0.05 &     -0.03 &     -0.05 &      0.04 \\
\quad  & (     0.03 ) & (     0.10 )  & (     0.09 )  & (     0.03 ) & (     0.07 ) \\
\quad Income 100K-250K Euro &     -0.03 &      0.08 &      0.02 &     -0.10 &      0.03 \\
\quad  & (     0.06 ) & (     0.22 )  & (     0.19 )  & (     0.08 ) & (     0.16 ) \\
\midrule
Observations & 289 & 289 & 289 & 289 & 289 \\
Fraction Attending Each Type &      0.02 &      0.53 &      0.27 &      0.03 &      0.15 \\
\midrule
$ R^2$ &      0.06 &      0.09 &      0.09 &      0.08 &      0.07 \\
\bottomrule
\end{tabular}}
\end{table}
\begin{scriptsize}
\noindent\underline{Note:} This table presents the linear probability model estimations for attending each type of Materna schools, indicated by each column. The samples used in this estimation are those who were children at the time of the survey living in Parma. All dependent variables are binary. Observation indicates the number of people included in this sample. Bold number indicates that the p-value is less than or equal to 0.1. Standard errors are reported in parentheses.
\end{scriptsize}

\begin{table}[H]
\caption{LPM Estimation Padova - Children, Materna}
\centering
\scalebox{0.7}{
\begin{tabular}{lccccc}
\toprule
 & \textbf{None} & \textbf{Municipal} & \textbf{Religious} & \textbf{Private} & \textbf{State} \\
\midrule
\textbf{Respondent's Baseline Info} \\
\quad Male &      0.00 &      0.07 &     -0.10 & \textbf{    -0.05} & \textbf{     0.07} \\
\quad  & (     0.01 ) & (     0.06 )  & (     0.06 )  & \textbf{(     0.03 )} & \textbf{(     0.04 )} \\
\quad CAPI &     -0.01 &     -0.00 &      0.00 &     -0.03 &      0.04 \\
\quad  & (     0.01 ) & (     0.06 )  & (     0.06 )  & (     0.03 ) & (     0.05 ) \\
\quad Low Birthweight &     -0.01 &      0.24 & \textbf{    -0.32} &      0.05 &      0.03 \\
\quad  & (     0.03 ) & (     0.15 )  & \textbf{(     0.17 )}  & (     0.07 ) & (     0.12 ) \\
\quad Premature at Birth &      0.00 &     -0.12 &      0.18 &     -0.06 &     -0.01 \\
\quad  & (     0.02 ) & (     0.13 )  & (     0.14 )  & (     0.06 ) & (     0.10 ) \\
\midrule
\textbf{Mother's Baseline Info} \\
\quad Max Education: Middle School &     -0.03 &      0.00 &      0.14 &      0.09 & \textbf{    -0.21} \\
\quad  & (     0.03 ) & (     0.13 )  & (     0.15 )  & (     0.06 ) & \textbf{(     0.11 )} \\
\quad Max Education: High School &     -0.03 &      0.09 &      0.04 &      0.07 & \textbf{    -0.18} \\
\quad  & (     0.02 ) & (     0.10 )  & (     0.11 )  & (     0.05 ) & \textbf{(     0.08 )} \\
\quad Max Education: University &     -0.02 &      0.17 &     -0.03 &      0.05 & \textbf{    -0.17} \\
\quad  & (     0.02 ) & (     0.11 )  & (     0.12 )  & (     0.05 ) & \textbf{(     0.09 )} \\
\quad Teenager at Birth &      0.01 & \textbf{     0.82} &     -0.66 &     -0.00 &     -0.17 \\
\quad  & (     0.07 ) & \textbf{(     0.37 )}  & (     0.41 )  & (     0.17 ) & (     0.29 ) \\
\quad Born in Province & \textbf{    -0.03} &     -0.10 &      0.12 & \textbf{    -0.05} &      0.06 \\
\quad  & \textbf{(     0.01 )} & (     0.07 )  & (     0.07 )  & \textbf{(     0.03 )} & (     0.05 ) \\
\midrule
\textbf{Father's Baseline Info} \\
\quad Max Education: Middle School &     -0.03 &     -0.07 &      0.10 &     -0.05 &      0.05 \\
\quad  & (     0.02 ) & (     0.13 )  & (     0.14 )  & (     0.06 ) & (     0.10 ) \\
\quad Max Education: High School &     -0.02 &     -0.13 &      0.06 &      0.03 &      0.07 \\
\quad  & (     0.02 ) & (     0.09 )  & (     0.10 )  & (     0.04 ) & (     0.07 ) \\
\quad Max Education: University &     -0.01 & \textbf{    -0.17} &      0.07 &      0.04 &      0.08 \\
\quad  & (     0.02 ) & \textbf{(     0.10 )}  & (     0.11 )  & (     0.04 ) & (     0.08 ) \\
\quad Teenager at Birth &     -0.00 &     -0.12 &      0.30 &     -0.04 &     -0.13 \\
\quad  & (     0.07 ) & (     0.37 )  & (     0.41 )  & (     0.17 ) & (     0.29 ) \\
\quad Born in Province &      0.01 &      0.07 &      0.04 &     -0.01 & \textbf{    -0.10} \\
\quad  & (     0.01 ) & (     0.07 )  & (     0.08 )  & (     0.03 ) & \textbf{(     0.05 )} \\
\midrule
\textbf{Household Baseline Info} \\
\quad Caregiver Has Religion &      0.02 &     -0.06 &      0.02 &      0.03 &     -0.01 \\
\quad  & (     0.01 ) & (     0.07 )  & (     0.08 )  & (     0.03 ) & (     0.06 ) \\
\quad Owns House &     -0.01 &      0.01 &      0.02 &      0.01 &     -0.03 \\
\quad  & (     0.01 ) & (     0.06 )  & (     0.07 )  & (     0.03 ) & (     0.05 ) \\
\quad Income 5K-10K Euro &     -0.01 & \textbf{     0.76} & \textbf{    -0.53} &     -0.10 &     -0.13 \\
\quad  & (     0.04 ) & \textbf{(     0.23 )}  & \textbf{(     0.26 )}  & (     0.10 ) & (     0.18 ) \\
\quad Income 10K-25K Euro &     -0.02 &     -0.02 &      0.03 & \textbf{    -0.11} & \textbf{     0.12} \\
\quad  & (     0.02 ) & (     0.09 )  & (     0.10 )  & \textbf{(     0.04 )} & \textbf{(     0.07 )} \\
\quad Income 25K-50K Euro &     -0.01 &      0.07 &      0.06 & \textbf{    -0.09} &     -0.04 \\
\quad  & (     0.01 ) & (     0.07 )  & (     0.08 )  & \textbf{(     0.03 )} & (     0.06 ) \\
\quad Income 50K-100K Euro &     -0.02 &      0.13 &      0.06 & \textbf{    -0.14} &     -0.03 \\
\quad  & (     0.02 ) & (     0.10 )  & (     0.11 )  & \textbf{(     0.04 )} & (     0.08 ) \\
\quad Income 100K-250K Euro & \textbf{     0.11} &     -0.13 &      0.00 &      0.00 &      0.01 \\
\quad  & \textbf{(     0.03 )} & (     0.17 )  & (     0.19 )  & (     0.08 ) & (     0.14 ) \\
\midrule
Observations & 277 & 277 & 277 & 277 & 277 \\
Fraction Attending Each Type &      0.01 &      0.30 &      0.51 &      0.04 &      0.14 \\
\midrule
$ R^2$ &      0.11 &      0.12 &      0.09 &      0.10 &      0.07 \\
\bottomrule
\end{tabular}}
\end{table}
\begin{scriptsize}
\noindent\underline{Note:} This table presents the linear probability model estimations for attending each type of Materna schools, indicated by each column. The samples used in this estimation are those who were children at the time of the survey living in Padova. All dependent variables are binary. Observation indicates the number of people included in this sample. Bold number indicates that the p-value is less than or equal to 0.1. Standard errors are reported in parentheses.
\end{scriptsize}




\subsection{Adolescents}
\subsubsection{Unconditional Mean}
\begin{table}[H]
\caption{Baseline, Reggio, Adolescents}
\scalebox{0.85}{
\begin{tabular}{l c c c c c c }
\toprule
& \textbf{Municipal} & \textbf{State} & \textbf{Religious} & \textbf{Private} & \textbf{None} \\
\midrule
Male indicator &      0.42 &      0.55 &      0.40 &      0.50 &      0.57 \\
\midrule
Observations &       166 &        22 &        96 &         6 &         7
Age &     18.70 &     18.75 &     18.73 &     18.64 &     18.67 \\
\midrule
Observations &       166 &        22 &        96 &         6 &         7
Low birthweight &      0.05 & \textbf{     0.00} &      0.05 &      0.17 &      0.14 \\
\midrule
Observations &       166 &        22 &        96 &         6 &         7
Premature birth &      0.04 &      0.09 &      0.08 &      0.17 &      0.14 \\
\midrule
Observations &       166 &        22 &        96 &         6 &         7
CAPI &      0.47 &      0.41 &      0.38 &      0.33 &      0.43 \\
\midrule
Observations &       166 &        22 &        96 &         6 &         7
Mother: age at birth &     30.26 &     31.53 &     30.15 &     30.62 &     29.96 \\
\midrule
Observations &       161 &        22 &        94 &         6 &         7
Mother: born in province &      0.72 & \textbf{     0.50} &      0.64 &      0.83 &      0.57 \\
\midrule
Observations &       166 &        22 &        96 &         6 &         7
Mother max. edu.: less than middle school &      0.14 &      0.27 &      0.14 &      0.17 &      0.29 \\
\midrule
Observations &       166 &        22 &        96 &         6 &         7
Mother max. edu.: middle school &      0.09 &      0.09 &      0.11 & \textbf{     0.00} & \textbf{     0.00} \\
\midrule
Observations &       166 &        22 &        96 &         6 &         7
Mother max. edu.: high school &      0.51 &      0.36 &      0.42 &      0.67 &      0.71 \\
\midrule
Observations &       166 &        22 &        96 &         6 &         7
Mother max. edu.: university &      0.22 &      0.27 &      0.31 &      0.17 & \textbf{     0.00} \\
\midrule
Observations &       166 &        22 &        96 &         6 &         7
Father: age at birth &     32.83 &     32.92 &     33.50 &     35.73 &     33.70 \\
\midrule
Observations &       141 &        18 &        85 &         5 &         7
Father: born in province &      0.61 &      0.45 &      0.57 &      0.67 &      0.43 \\
\midrule
Observations &       166 &        22 &        96 &         6 &         7
Father max. edu.: less than middle school &      0.19 &      0.23 &      0.18 & \textbf{     0.00} &      0.43 \\
\midrule
Observations &       166 &        22 &        96 &         6 &         7
Father max. edu.: middle school &      0.07 &      0.09 &      0.10 &      0.33 & \textbf{     0.00} \\
\midrule
Observations &       166 &        22 &        96 &         6 &         7
Father max. edu.: high school &      0.40 &      0.41 &      0.40 &      0.50 &      0.14 \\
\midrule
Observations &       166 &        22 &        96 &         6 &         7
Father max. edu.: university &      0.19 &      0.09 &      0.21 & \textbf{     0.00} &      0.43 \\
\midrule
Observations &       166 &        22 &        96 &         6 &         7
Number of siblings &      1.14 &      1.27 &      1.31 &      0.83 &      1.71 \\
\midrule
Observations &       166 &        22 &        96 &         6 &         7
Religious caregiver indicator &      0.69 &      0.73 & \textbf{     0.91} & \textbf{     1.00} &      0.86 \\
\midrule
Observations &       166 &        22 &        96 &         6 &         7
Mother: born outside of Italy &      0.00 &      0.00 &      0.01 &      0.17 &      0.29 \\
\midrule
Observations &       166 &        22 &        96 &         6 &         7
Income: 5,000 euros or less &      0.00 &      0.00 &      0.01 &      0.00 &      0.00 \\
\midrule
Observations &       166 &        22 &        96 &         6 &         7
Income: 5,001-10,000 euros &      0.01 &      0.00 &      0.01 &      0.00 &      0.00 \\
\midrule
Observations &       166 &        22 &        96 &         6 &         7
Income: 10,001-25,000 euros &      0.17 & \textbf{     0.36} &      0.17 & \textbf{     0.00} &      0.29 \\
\midrule
Observations &       166 &        22 &        96 &         6 &         7
Income: 25,001-50,000 euros &      0.32 &      0.27 &      0.34 &      0.17 &      0.43 \\
\midrule
Observations &       166 &        22 &        96 &         6 &         7
Income: 50,001-100,000 euros &      0.25 & \textbf{     0.09} &      0.25 &      0.33 & \textbf{     0.00} \\
\midrule
Observations &       166 &        22 &        96 &         6 &         7
Income: 100,001-250,000 euros &      0.04 & \textbf{     0.00} &      0.06 & \textbf{     0.00} & \textbf{     0.00} \\
\midrule
Observations &       166 &        22 &        96 &         6 &         7
Income: more than 250,000 euros &      0.00 &      0.00 &      0.01 &      0.00 &      0.00 \\
\midrule
Observations &       166 &        22 &        96 &         6 &         7
\bottomrule
\end{tabular}

}
\end{table}

\begin{table}[H]
\caption{Baseline, Parma, Adolescents}
\scalebox{0.85}{
\begin{tabular}{l c c c c c c }
\toprule
& \textbf{Municipal} & \textbf{State} & \textbf{Religious} & \textbf{Private} & \textbf{None} \\
\midrule
Male indicator &      0.40 &      0.42 &      0.52 &      0.67 &      0.50 \\
\midrule
Observations &       116 &        43 &        82 &         6 &         4
Age &     18.77 &     18.75 & \textbf{    18.80} &     18.81 &     18.59 \\
\midrule
Observations &       116 &        43 &        82 &         6 &         4
Low birthweight &      0.06 &      0.05 &      0.07 & \textbf{     0.00} &      0.25 \\
\midrule
Observations &       116 &        43 &        82 &         6 &         4
Premature birth & \textbf{     0.09} &      0.07 & \textbf{     0.12} & \textbf{     0.00} &      0.25 \\
\midrule
Observations &       116 &        43 &        82 &         6 &         4
CAPI &      0.53 &      0.47 & \textbf{     0.59} & \textbf{     1.00} &      0.50 \\
\midrule
Observations &       116 &        43 &        82 &         6 &         4
Mother: age at birth &     30.84 &     30.05 &     30.95 & \textbf{    27.15} &     26.80 \\
\midrule
Observations &       115 &        41 &        81 &         6 &         4
Mother: born in province &      0.68 &      0.74 &      0.66 &      0.67 &      0.25 \\
\midrule
Observations &       116 &        43 &        82 &         6 &         4
Mother max. edu.: less than middle school &      0.09 &      0.16 &      0.12 & \textbf{     0.00} & \textbf{     0.00} \\
\midrule
Observations &       116 &        43 &        82 &         6 &         4
Mother max. edu.: middle school &      0.09 &      0.16 &      0.10 & \textbf{     0.00} & \textbf{     0.00} \\
\midrule
Observations &       116 &        43 &        82 &         6 &         4
Mother max. edu.: high school &      0.44 &      0.47 &      0.45 & \textbf{     0.00} &      0.75 \\
\midrule
Observations &       116 &        43 &        82 &         6 &         4
Mother max. edu.: university & \textbf{     0.37} &      0.16 & \textbf{     0.33} & \textbf{     1.00} &      0.25 \\
\midrule
Observations &       116 &        43 &        82 &         6 &         4
Father: age at birth &     33.77 &     32.01 & \textbf{    34.34} &     31.43 &     32.42 \\
\midrule
Observations &        98 &        38 &        70 &         4 &         2
Father: born in province &      0.58 &      0.70 &      0.63 &      0.50 & \textbf{     0.00} \\
\midrule
Observations &       116 &        43 &        82 &         6 &         4
Father max. edu.: less than middle school &      0.16 &      0.21 &      0.15 & \textbf{     0.00} & \textbf{     0.00} \\
\midrule
Observations &       116 &        43 &        82 &         6 &         4
Father max. edu.: middle school &      0.09 &      0.12 & \textbf{     0.02} & \textbf{     0.00} & \textbf{     0.00} \\
\midrule
Observations &       116 &        43 &        82 &         6 &         4
Father max. edu.: high school &      0.35 &      0.40 &      0.38 &      0.17 &      0.25 \\
\midrule
Observations &       116 &        43 &        82 &         6 &         4
Father max. edu.: university &      0.24 &      0.16 & \textbf{     0.32} &      0.50 &      0.25 \\
\midrule
Observations &       116 &        43 &        82 &         6 &         4
Number of siblings &      1.07 &      1.21 &      1.01 &      1.17 &      1.00 \\
\midrule
Observations &       116 &        43 &        82 &         6 &         4
Religious caregiver indicator & \textbf{     0.88} & \textbf{     0.86} & \textbf{     0.87} &      0.83 &      0.75 \\
\midrule
Observations &       116 &        43 &        82 &         6 &         4
Mother: born outside of Italy &      0.00 &      0.02 & \textbf{     0.04} &      0.00 &      0.00 \\
\midrule
Observations &       116 &        43 &        82 &         6 &         4
Income: 5,000 euros or less & \textbf{     0.03} &      0.02 &      0.02 &      0.00 &      0.00 \\
\midrule
Observations &       116 &        43 &        82 &         6 &         4
Income: 5,001-10,000 euros &      0.01 &      0.00 &      0.00 &      0.00 &      0.00 \\
\midrule
Observations &       116 &        43 &        82 &         6 &         4
Income: 10,001-25,000 euros &      0.16 &      0.28 &      0.12 &      0.50 &      0.25 \\
\midrule
Observations &       116 &        43 &        82 &         6 &         4
Income: 25,001-50,000 euros &      0.25 &      0.37 &      0.30 &      0.50 &      0.25 \\
\midrule
Observations &       116 &        43 &        82 &         6 &         4
Income: 50,001-100,000 euros &      0.26 &      0.23 &      0.26 & \textbf{     0.00} & \textbf{     0.00} \\
\midrule
Observations &       116 &        43 &        82 &         6 &         4
Income: 100,001-250,000 euros &      0.03 & \textbf{     0.00} &      0.05 & \textbf{     0.00} & \textbf{     0.00} \\
\midrule
Observations &       116 &        43 &        82 &         6 &         4
Income: more than 250,000 euros &      0.00 &      0.00 &      0.00 &      0.00 &      0.00 \\
\midrule
Observations &       116 &        43 &        82 &         6 &         4
\bottomrule
\end{tabular}

}
\end{table}

\begin{table}[H]
\caption{Baseline, Padova, Adolescents}
\scalebox{0.85}{
\begin{tabular}{l c c c c c c }
\toprule
& \textbf{Municipal} & \textbf{State} & \textbf{Religious} & \textbf{Private} & \textbf{None} \\
\midrule
Male indicator &      0.44 &      0.45 &      0.50 &      0.50 &      1.00 \\
\midrule
Observations &        93 &        47 &       131 &         6 &         1
Age &     18.75 & \textbf{    18.82} &     18.64 &     18.74 &     18.56 \\
\midrule
Observations &        93 &        47 &       131 &         6 &         1
Low birthweight &      0.06 &      0.06 &      0.02 & \textbf{     0.00} &      1.00 \\
\midrule
Observations &        93 &        47 &       131 &         6 &         1
Premature birth &      0.10 &      0.09 &      0.04 & \textbf{     0.00} &      1.00 \\
\midrule
Observations &        93 &        47 &       131 &         6 &         1
CAPI &      0.43 &      0.55 &      0.53 &      0.33 &      0.00 \\
\midrule
Observations &        93 &        47 &       131 &         6 &         1
Mother: age at birth & \textbf{    32.02} & \textbf{    31.89} & \textbf{    32.06} &     32.37 &     33.03 \\
\midrule
Observations &        89 &        46 &       128 &         5 &         1
Mother: born in province &      0.71 &      0.77 & \textbf{     0.82} & \textbf{     1.00} &      1.00 \\
\midrule
Observations &        93 &        47 &       131 &         6 &         1
Mother max. edu.: less than middle school &      0.11 &      0.11 &      0.18 & \textbf{     0.00} &      0.00 \\
\midrule
Observations &        93 &        47 &       131 &         6 &         1
Mother max. edu.: middle school &      0.15 &      0.11 &      0.08 &      0.17 &      0.00 \\
\midrule
Observations &        93 &        47 &       131 &         6 &         1
Mother max. edu.: high school & \textbf{     0.39} &      0.43 &      0.45 &      0.50 &      1.00 \\
\midrule
Observations &        93 &        47 &       131 &         6 &         1
Mother max. edu.: university &      0.31 &      0.34 &      0.27 &      0.33 &      0.00 \\
\midrule
Observations &        93 &        47 &       131 &         6 &         1
Father: age at birth & \textbf{    34.98} &     34.05 & \textbf{    34.90} &     33.77 &     35.41 \\
\midrule
Observations &        81 &        43 &       121 &         4 &         1
Father: born in province &      0.65 &      0.72 & \textbf{     0.79} &      0.67 &      1.00 \\
\midrule
Observations &        93 &        47 &       131 &         6 &         1
Father max. edu.: less than middle school &      0.13 & \textbf{     0.09} &      0.17 & \textbf{     0.00} &      0.00 \\
\midrule
Observations &        93 &        47 &       131 &         6 &         1
Father max. edu.: middle school &      0.11 &      0.13 &      0.08 &      0.17 &      0.00 \\
\midrule
Observations &        93 &        47 &       131 &         6 &         1
Father max. edu.: high school &      0.32 &      0.47 &      0.41 &      0.17 &      1.00 \\
\midrule
Observations &        93 &        47 &       131 &         6 &         1
Father max. edu.: university & \textbf{     0.31} &      0.23 & \textbf{     0.27} &      0.50 &      0.00 \\
\midrule
Observations &        93 &        47 &       131 &         6 &         1
Number of siblings &      1.01 & \textbf{     0.72} &      1.05 &      0.83 &      1.00 \\
\midrule
Observations &        93 &        47 &       131 &         6 &         1
Religious caregiver indicator &      0.77 &      0.62 & \textbf{     0.78} &      0.50 &      1.00 \\
\midrule
Observations &        93 &        47 &       131 &         6 &         1
Mother: born outside of Italy &      0.00 &      0.00 &      0.00 &      0.00 &      0.00 \\
\midrule
Observations &        93 &        47 &       131 &         6 &         1
Income: 5,000 euros or less & \textbf{     0.03} &      0.02 & \textbf{     0.05} &      0.00 &      0.00 \\
\midrule
Observations &        93 &        47 &       131 &         6 &         1
Income: 5,001-10,000 euros &      0.00 &      0.02 &      0.01 &      0.00 &      0.00 \\
\midrule
Observations &        93 &        47 &       131 &         6 &         1
Income: 10,001-25,000 euros &      0.12 & \textbf{     0.09} & \textbf{     0.11} & \textbf{     0.00} &      0.00 \\
\midrule
Observations &        93 &        47 &       131 &         6 &         1
Income: 25,001-50,000 euros &      0.30 & \textbf{     0.09} &      0.27 &      0.17 &      0.00 \\
\midrule
Observations &        93 &        47 &       131 &         6 &         1
Income: 50,001-100,000 euros & \textbf{     0.10} & \textbf{     0.13} & \textbf{     0.11} &      0.17 &      0.00 \\
\midrule
Observations &        93 &        47 &       131 &         6 &         1
Income: 100,001-250,000 euros & \textbf{     0.01} & \textbf{     0.00} &      0.05 & \textbf{     0.00} &      0.00 \\
\midrule
Observations &        93 &        47 &       131 &         6 &         1
Income: more than 250,000 euros &      0.00 &      0.00 &      0.00 &      0.00 &      0.00 \\
\midrule
Observations &        93 &        47 &       131 &         6 &         1
\bottomrule
\end{tabular}

}
\end{table}

\subsubsection{Linear Probability Model}
\begin{table}[H]
\caption{LPM Estimation Reggio - Adolescents, Materna}
\centering
\scalebox{0.7}{
\begin{tabular}{lccccc}
\toprule
 & \textbf{None} & \textbf{Municipal} & \textbf{Religious} & \textbf{Private} & \textbf{State} \\
\midrule
\textbf{Respondent's Baseline Info} \\
\quad Male &      0.02 &     -0.03 &     -0.04 &      0.01 &      0.05 \\
\quad  & (     0.02 ) & (     0.06 )  & (     0.06 )  & (     0.02 ) & (     0.03 ) \\
\quad CAPI &      0.01 &      0.09 &     -0.08 &     -0.01 &     -0.01 \\
\quad  & (     0.02 ) & (     0.06 )  & (     0.06 )  & (     0.02 ) & (     0.03 ) \\
\quad Low Birthweight &      0.04 &      0.09 &     -0.07 &      0.03 &     -0.09 \\
\quad  & (     0.05 ) & (     0.15 )  & (     0.14 )  & (     0.04 ) & (     0.08 ) \\
\quad Premature at Birth &      0.02 & \textbf{    -0.26} &      0.13 &      0.02 &      0.09 \\
\quad  & (     0.04 ) & \textbf{(     0.14 )}  & (     0.13 )  & (     0.04 ) & (     0.08 ) \\
\midrule
\textbf{Mother's Baseline Info} \\
\quad Max Education: Middle School &     -0.03 &      0.10 &      0.02 & \textbf{    -0.07} &     -0.03 \\
\quad  & (     0.04 ) & (     0.13 )  & (     0.12 )  & \textbf{(     0.04 )} & (     0.07 ) \\
\quad Max Education: High School &      0.01 &      0.08 &     -0.07 &     -0.00 &     -0.03 \\
\quad  & (     0.03 ) & (     0.09 )  & (     0.08 )  & (     0.03 ) & (     0.05 ) \\
\quad Max Education: University &     -0.05 &     -0.01 &      0.05 &     -0.00 &      0.01 \\
\quad  & (     0.03 ) & (     0.10 )  & (     0.10 )  & (     0.03 ) & (     0.05 ) \\
\quad Teenager at Birth &     -0.05 &      0.35 &     -0.13 &     -0.03 &     -0.14 \\
\quad  & (     0.08 ) & (     0.26 )  & (     0.25 )  & (     0.08 ) & (     0.14 ) \\
\quad Born in Province &      0.00 &      0.11 &     -0.06 &      0.01 &     -0.06 \\
\quad  & (     0.02 ) & (     0.07 )  & (     0.06 )  & (     0.02 ) & (     0.04 ) \\
\midrule
\textbf{Father's Baseline Info} \\
\quad Max Education: Middle School &     -0.01 &     -0.14 &      0.05 & \textbf{     0.09} &      0.02 \\
\quad  & (     0.04 ) & (     0.12 )  & (     0.12 )  & \textbf{(     0.04 )} & (     0.07 ) \\
\quad Max Education: High School &     -0.00 &     -0.02 &      0.02 &     -0.00 &      0.01 \\
\quad  & (     0.02 ) & (     0.08 )  & (     0.07 )  & (     0.02 ) & (     0.04 ) \\
\quad Max Education: University & \textbf{     0.08} &      0.02 &     -0.04 &     -0.03 &     -0.03 \\
\quad  & \textbf{(     0.03 )} & (     0.10 )  & (     0.09 )  & (     0.03 ) & (     0.05 ) \\
\quad Born in Province &     -0.01 &      0.02 &      0.01 &     -0.00 &     -0.01 \\
\quad  & (     0.02 ) & (     0.07 )  & (     0.06 )  & (     0.02 ) & (     0.04 ) \\
\midrule
\textbf{Household Baseline Info} \\
\quad Caregiver Has Religion &      0.01 & \textbf{    -0.29} & \textbf{     0.26} &      0.03 &     -0.01 \\
\quad  & (     0.02 ) & \textbf{(     0.07 )}  & \textbf{(     0.07 )}  & (     0.02 ) & (     0.04 ) \\
\quad Owns House & \textbf{    -0.08} &      0.02 &      0.06 &     -0.00 &      0.01 \\
\quad  & \textbf{(     0.03 )} & (     0.08 )  & (     0.08 )  & (     0.02 ) & (     0.05 ) \\
\quad Income 5K-10K Euro &     -0.09 &      0.36 &     -0.04 &     -0.08 &     -0.15 \\
\quad  & (     0.09 ) & (     0.31 )  & (     0.29 )  & (     0.09 ) & (     0.17 ) \\
\quad Income 10K-25K Euro &      0.01 &     -0.04 &      0.03 & \textbf{    -0.05} &      0.04 \\
\quad  & (     0.03 ) & (     0.10 )  & (     0.09 )  & \textbf{(     0.03 )} & (     0.05 ) \\
\quad Income 25K-50K Euro &      0.00 &      0.02 &      0.06 & \textbf{    -0.04} &     -0.04 \\
\quad  & (     0.03 ) & (     0.08 )  & (     0.08 )  & \textbf{(     0.02 )} & (     0.05 ) \\
\quad Income 50K-100K Euro &     -0.03 &      0.04 &      0.06 &     -0.02 &     -0.06 \\
\quad  & (     0.03 ) & (     0.09 )  & (     0.09 )  & (     0.03 ) & (     0.05 ) \\
\quad Income 100K-250K Euro &     -0.04 &     -0.06 &      0.21 &     -0.03 &     -0.08 \\
\quad  & (     0.05 ) & (     0.15 )  & (     0.15 )  & (     0.04 ) & (     0.08 ) \\
\quad Income More Than 250K Euro &     -0.05 &     -0.46 &      0.64 &     -0.04 &     -0.09 \\
\quad  & (     0.15 ) & (     0.50 )  & (     0.47 )  & (     0.14 ) & (     0.27 ) \\
\midrule
Observations & 297 & 297 & 297 & 297 & 297 \\
Fraction Attending Each Type &      0.02 &      0.56 &      0.32 &      0.02 &      0.07 \\
\midrule
$ R^2$ &      0.09 &      0.11 &      0.10 &      0.07 &      0.06 \\
\bottomrule
\end{tabular}}
\end{table}
\begin{scriptsize}
\noindent\underline{Note:} This table presents the linear probability model estimations for attending each type of Materna schools, indicated by each column. The samples used in this estimation are those who were adolescents at the time of the survey living in Reggio. All dependent variables are binary. Observation indicates the number of people included in this sample. Bold number indicates that the p-value is less than or equal to 0.1. Standard errors are reported in parentheses.
\end{scriptsize}

\begin{table}[H]
\caption{LPM Estimation Parma - Adolescents, Materna}
\centering
\scalebox{0.7}{
\begin{tabular}{lccccc}
\toprule
 & \textbf{None} & \textbf{Municipal} & \textbf{Religious} & \textbf{Private} & \textbf{State} \\
\midrule
\textbf{Respondent's Baseline Info} \\
\quad Male &      0.01 &     -0.08 &      0.06 &      0.01 &     -0.01 \\
\quad  & (     0.02 ) & (     0.07 )  & (     0.06 )  & (     0.02 ) & (     0.05 ) \\
\quad CAPI &     -0.00 &     -0.04 &      0.10 & \textbf{     0.03} & \textbf{    -0.09} \\
\quad  & (     0.02 ) & (     0.07 )  & (     0.06 )  & \textbf{(     0.02 )} & \textbf{(     0.05 )} \\
\quad Low Birthweight &      0.04 &      0.03 &     -0.01 &      0.02 &     -0.08 \\
\quad  & (     0.04 ) & (     0.16 )  & (     0.15 )  & (     0.05 ) & (     0.12 ) \\
\quad Premature at Birth &      0.00 &     -0.08 &      0.11 &     -0.03 &      0.00 \\
\quad  & (     0.03 ) & (     0.14 )  & (     0.13 )  & (     0.04 ) & (     0.10 ) \\
\midrule
\textbf{Mother's Baseline Info} \\
\quad Max Education: Middle School &      0.02 &      0.04 &     -0.07 &      0.02 &     -0.01 \\
\quad  & (     0.04 ) & (     0.15 )  & (     0.14 )  & (     0.04 ) & (     0.11 ) \\
\quad Max Education: High School &      0.03 &      0.16 &     -0.08 &      0.01 &     -0.13 \\
\quad  & (     0.03 ) & (     0.12 )  & (     0.11 )  & (     0.04 ) & (     0.09 ) \\
\quad Max Education: University &      0.02 & \textbf{     0.26} &     -0.15 & \textbf{     0.10} & \textbf{    -0.23} \\
\quad  & (     0.03 ) & \textbf{(     0.13 )}  & (     0.12 )  & \textbf{(     0.04 )} & \textbf{(     0.10 )} \\
\quad Teenager at Birth &     -0.02 &      0.31 &     -0.34 &     -0.01 &      0.05 \\
\quad  & (     0.07 ) & (     0.27 )  & (     0.25 )  & (     0.08 ) & (     0.20 ) \\
\quad Born in Province &     -0.03 &     -0.02 &     -0.03 &      0.01 &      0.07 \\
\quad  & (     0.02 ) & (     0.07 )  & (     0.07 )  & (     0.02 ) & (     0.05 ) \\
\midrule
\textbf{Father's Baseline Info} \\
\quad Max Education: Middle School &      0.00 &      0.12 & \textbf{    -0.26} &      0.01 &      0.14 \\
\quad  & (     0.04 ) & (     0.15 )  & \textbf{(     0.14 )}  & (     0.04 ) & (     0.11 ) \\
\quad Max Education: High School &     -0.00 &     -0.06 &      0.02 &      0.02 &      0.03 \\
\quad  & (     0.03 ) & (     0.10 )  & (     0.10 )  & (     0.03 ) & (     0.08 ) \\
\quad Max Education: University &      0.00 &     -0.16 &      0.07 &      0.03 &      0.06 \\
\quad  & (     0.03 ) & (     0.12 )  & (     0.11 )  & (     0.03 ) & (     0.09 ) \\
\quad Teenager at Birth &     -0.04 &      0.04 &     -0.43 &     -0.01 & \textbf{     0.44} \\
\quad  & (     0.09 ) & (     0.36 )  & (     0.34 )  & (     0.11 ) & \textbf{(     0.27 )} \\
\quad Born in Province & \textbf{    -0.04} &     -0.01 &      0.03 &     -0.01 &      0.02 \\
\quad  & \textbf{(     0.02 )} & (     0.08 )  & (     0.08 )  & (     0.02 ) & (     0.06 ) \\
\midrule
\textbf{Household Baseline Info} \\
\quad Caregiver Has Religion &     -0.02 &      0.04 &      0.03 &     -0.00 &     -0.04 \\
\quad  & (     0.03 ) & (     0.10 )  & (     0.09 )  & (     0.03 ) & (     0.07 ) \\
\quad Owns House & \textbf{     0.04} &     -0.10 &      0.03 &      0.00 &      0.03 \\
\quad  & \textbf{(     0.02 )} & (     0.09 )  & (     0.09 )  & (     0.03 ) & (     0.07 ) \\
\quad Income 5K-10K Euro &      0.00 &      0.43 &     -0.36 &      0.01 &     -0.09 \\
\quad  & (     0.13 ) & (     0.53 )  & (     0.49 )  & (     0.16 ) & (     0.39 ) \\
\quad Income 10K-25K Euro &     -0.01 &     -0.12 & \textbf{    -0.18} & \textbf{     0.08} & \textbf{     0.23} \\
\quad  & (     0.03 ) & (     0.11 )  & \textbf{(     0.10 )}  & \textbf{(     0.03 )} & \textbf{(     0.08 )} \\
\quad Income 25K-50K Euro &     -0.02 & \textbf{    -0.15} &     -0.03 &      0.04 & \textbf{     0.16} \\
\quad  & (     0.02 ) & \textbf{(     0.09 )}  & (     0.08 )  & (     0.03 ) & \textbf{(     0.07 )} \\
\quad Income 50K-100K Euro &     -0.03 &     -0.03 &     -0.02 &     -0.04 & \textbf{     0.12} \\
\quad  & (     0.02 ) & (     0.10 )  & (     0.09 )  & (     0.03 ) & \textbf{(     0.07 )} \\
\quad Income 100K-250K Euro &     -0.04 &     -0.04 &      0.16 &     -0.03 &     -0.04 \\
\quad  & (     0.05 ) & (     0.21 )  & (     0.20 )  & (     0.06 ) & (     0.16 ) \\
\midrule
Observations & 251 & 251 & 251 & 251 & 251 \\
Fraction Attending Each Type &      0.02 &      0.46 &      0.33 &      0.02 &      0.17 \\
\midrule
$ R^2$ &      0.07 &      0.07 &      0.08 &      0.13 &      0.12 \\
\bottomrule
\end{tabular}}
\end{table}
\begin{scriptsize}
\noindent\underline{Note:} This table presents the linear probability model estimations for attending each type of Materna schools, indicated by each column. The samples used in this estimation are those who were adolescents at the time of the survey living in Parma. All dependent variables are binary. Observation indicates the number of people included in this sample. Bold number indicates that the p-value is less than or equal to 0.1. Standard errors are reported in parentheses.
\end{scriptsize}

\begin{table}[H]
\caption{LPM Estimation Padova - Adolescents, Materna}
\centering
\scalebox{0.7}{
\begin{tabular}{lccccc}
\toprule
 & \textbf{None} & \textbf{Municipal} & \textbf{Religious} & \textbf{Private} & \textbf{State} \\
\midrule
\textbf{Respondent's Baseline Info} \\
\quad Male &      0.01 &     -0.06 &      0.06 &      0.00 &     -0.01 \\
\quad  & (     0.01 ) & (     0.06 )  & (     0.06 )  & (     0.02 ) & (     0.05 ) \\
\quad CAPI &     -0.01 &     -0.02 &      0.06 &     -0.02 &      0.00 \\
\quad  & (     0.01 ) & (     0.06 )  & (     0.06 )  & (     0.02 ) & (     0.05 ) \\
\quad Low Birthweight & \textbf{     0.06} &      0.09 &     -0.15 &     -0.02 &      0.02 \\
\quad  & \textbf{(     0.02 )} & (     0.15 )  & (     0.16 )  & (     0.05 ) & (     0.12 ) \\
\quad Premature at Birth & \textbf{     0.03} &      0.08 &     -0.16 &     -0.02 &      0.06 \\
\quad  & \textbf{(     0.02 )} & (     0.13 )  & (     0.13 )  & (     0.04 ) & (     0.10 ) \\
\midrule
\textbf{Mother's Baseline Info} \\
\quad Max Education: Middle School &     -0.00 & \textbf{     0.24} & \textbf{    -0.28} &      0.02 &      0.03 \\
\quad  & (     0.02 ) & \textbf{(     0.12 )}  & \textbf{(     0.13 )}  & (     0.04 ) & (     0.10 ) \\
\quad Max Education: High School &      0.00 &      0.07 &     -0.12 &      0.03 &      0.03 \\
\quad  & (     0.01 ) & (     0.09 )  & (     0.09 )  & (     0.03 ) & (     0.07 ) \\
\quad Max Education: University &     -0.00 &      0.05 &     -0.16 &      0.01 &      0.11 \\
\quad  & (     0.01 ) & (     0.10 )  & (     0.10 )  & (     0.03 ) & (     0.08 ) \\
\quad Born in Province &      0.00 & \textbf{    -0.14} & \textbf{     0.14} &      0.03 &     -0.03 \\
\quad  & (     0.01 ) & \textbf{(     0.07 )}  & \textbf{(     0.08 )}  & (     0.02 ) & (     0.06 ) \\
\midrule
\textbf{Father's Baseline Info} \\
\quad Max Education: Middle School &      0.00 &     -0.06 &     -0.03 &      0.02 &      0.07 \\
\quad  & (     0.02 ) & (     0.12 )  & (     0.13 )  & (     0.04 ) & (     0.10 ) \\
\quad Max Education: High School &      0.01 &     -0.07 &     -0.01 &     -0.01 &      0.07 \\
\quad  & (     0.01 ) & (     0.09 )  & (     0.09 )  & (     0.03 ) & (     0.07 ) \\
\quad Max Education: University &      0.01 &      0.04 &     -0.03 &      0.02 &     -0.04 \\
\quad  & (     0.01 ) & (     0.10 )  & (     0.10 )  & (     0.03 ) & (     0.08 ) \\
\quad Born in Province &      0.01 &     -0.08 & \textbf{     0.12} &     -0.01 &     -0.03 \\
\quad  & (     0.01 ) & (     0.07 )  & \textbf{(     0.07 )}  & (     0.02 ) & (     0.06 ) \\
\midrule
\textbf{Household Baseline Info} \\
\quad Caregiver Has Religion &      0.00 &      0.01 &      0.10 &     -0.03 & \textbf{    -0.09} \\
\quad  & (     0.01 ) & (     0.07 )  & (     0.07 )  & (     0.02 ) & \textbf{(     0.05 )} \\
\quad Owns House &     -0.01 &     -0.03 & \textbf{     0.13} &      0.01 &     -0.09 \\
\quad  & (     0.01 ) & (     0.07 )  & \textbf{(     0.07 )}  & (     0.02 ) & (     0.06 ) \\
\quad Income 5K-10K Euro &      0.00 &     -0.40 &      0.17 &     -0.06 &      0.28 \\
\quad  & (     0.04 ) & (     0.34 )  & (     0.35 )  & (     0.11 ) & (     0.27 ) \\
\quad Income 10K-25K Euro &     -0.00 &      0.04 &      0.02 &     -0.02 &     -0.03 \\
\quad  & (     0.01 ) & (     0.10 )  & (     0.11 )  & (     0.03 ) & (     0.08 ) \\
\quad Income 25K-50K Euro &     -0.01 &      0.06 &      0.12 &     -0.01 & \textbf{    -0.16} \\
\quad  & (     0.01 ) & (     0.08 )  & (     0.08 )  & (     0.02 ) & \textbf{(     0.06 )} \\
\quad Income 50K-100K Euro &     -0.01 &     -0.07 &      0.11 &     -0.01 &     -0.02 \\
\quad  & (     0.01 ) & (     0.10 )  & (     0.11 )  & (     0.03 ) & (     0.08 ) \\
\quad Income 100K-250K Euro &     -0.00 &     -0.23 & \textbf{     0.49} &     -0.05 &     -0.21 \\
\quad  & (     0.02 ) & (     0.19 )  & \textbf{(     0.19 )}  & (     0.06 ) & (     0.15 ) \\
\midrule
Observations & 278 & 278 & 278 & 278 & 278 \\
Fraction Attending Each Type &      0.00 &      0.33 &      0.47 &      0.02 &      0.17 \\
\midrule
$ R^2$ &      0.12 &      0.07 &      0.12 &      0.04 &      0.09 \\
\bottomrule
\end{tabular}}
\end{table}
\begin{scriptsize}
\noindent\underline{Note:} This table presents the linear probability model estimations for attending each type of Materna schools, indicated by each column. The samples used in this estimation are those who were adolescents at the time of the survey living in Padova. All dependent variables are binary. Observation indicates the number of people included in this sample. Bold number indicates that the p-value is less than or equal to 0.1. Standard errors are reported in parentheses.
\end{scriptsize}




\subsection{Adults at Age 30}
\subsubsection{Unconditional Mean}
\begin{table}[H]
\caption{Baseline, Reggio, Adult 30's}
\scalebox{0.8}{
\begin{tabular}{l c c c c c c c c c c c c c c c c c c}
\toprule
& \multicolumn{3}{c}{Municipal} & \multicolumn{3}{c}{State} & \multicolumn{3}{c}{Religious} & \multicolumn{3}{c}{Private} & \multicolumn{3}{c}{None} & Unconditional R^2 & Conditional R^2 & Conditional N\\
& \scriptsize Mean & \scriptsize C. Mean & \scriptsize N & \scriptsize Mean & \scriptsize C. Mean & \scriptsize N & \scriptsize Mean & \scriptsize C. Mean & \scriptsize C. Mean & \scriptsize C. Mean & \scriptsize N & \scriptsize Mean & \scriptsize C. Mean & \scriptsize N & \scriptsize Mean & & \scriptsize C. Mean \scriptsize N & & & \\
\midrule
Male indicator &         . &         . & & \textbf{        .} &         . & &         . &         . & &         . &         . & & \textbf{        .} &         . & &      0.00 &      0.00 &      3974 \\
Age &         . &         . & &         . &         . & & \textbf{        .} &         . & &         . &         . & &         . &         . & &      0.26 &      0.26 &      3974 \\
CAPI &         . &         . & & \textbf{        .} &         . & & \textbf{        .} &         . & &         . &         . & & \textbf{        .} &         . & &      0.01 &      0.01 &      3974 \\
Mother: born in province &         . &         . & &         . &         . & &         . &         . & &         . &         . & &         . &         . & &      0.01 &      0.01 &      3974 \\
Mother max. edu.: less than middle school &         . &         . & &         . &         . & &         . &         . & &         . &         . & &         . &         . & &      0.02 &      0.02 &      3974 \\
Mother max. edu.: middle school &         . &         . & & \textbf{        .} &         . & &         . &         . & &         . &         . & & \textbf{        .} &         . & &      0.01 &      0.01 &      3974 \\
Mother max. edu.: high school &         . &         . & &         . &         . & &         . &         . & &         . &         . & & \textbf{        .} &         . & &      0.00 &      0.00 &      3974 \\
Mother max. edu.: university &         . &         . & &         . &         . & & \textbf{        .} &         . & &         . &         . & & \textbf{        .} &         . & &      0.00 &      0.00 &      3974 \\
Father: born in province &         . &         . & &         . &         . & &         . &         . & &         . &         . & &         . &         . & &      0.02 &      0.02 &      3974 \\
Father max. edu.: less than middle school &         . &         . & &         . &         . & &         . &         . & &         . &         . & &         . &         . & &      0.01 &      0.01 &      3974 \\
Father max. edu.: middle school &         . &         . & &         . &         . & &         . &         . & &         . &         . & & \textbf{        .} &         . & &      0.01 &      0.01 &      3974 \\
Father max. edu.: high school &         . &         . & &         . &         . & &         . &         . & &         . &         . & & \textbf{        .} &         . & &      0.00 &      0.00 &      3974 \\
Father max. edu.: university &         . &         . & &         . &         . & &         . &         . & &         . &         . & & \textbf{        .} &         . & &      0.01 &      0.01 &      3974 \\
Religious caregiver indicator &         . &         . & &         . &         . & & \textbf{        .} &         . & &         . &         . & & \textbf{        .} &         . & &      0.01 &      0.01 &      3974 \\
\bottomrule
\end{tabular}

}
\end{table}

\begin{table}[H]
\caption{Baseline, Parma, Adult 30's}
\scalebox{0.8}{
\begin{tabular}{l c c c c c c }
\toprule
& \textbf{Municipal} & \textbf{State} & \textbf{Religious} & \textbf{Private} & \textbf{None} \\
\midrule
Male indicator & \textbf{     0.54} & \textbf{     0.51} & \textbf{     0.50} &      0.40 &      0.57 \\
\midrule
Observations &        98 &        51 &        50 &         5 &        44
Age &     32.69 &     32.69 &     32.78 & \textbf{    32.43} &     32.84 \\
\midrule
Observations &        98 &        51 &        50 &         5 &        44
CAPI & \textbf{     0.27} & \textbf{     0.41} &      0.64 &      0.40 & \textbf{     0.41} \\
\midrule
Observations &        98 &        51 &        50 &         5 &        44
Mother: born in province & \textbf{     0.62} &      0.75 &      0.76 &      0.60 &      0.77 \\
\midrule
Observations &        98 &        51 &        50 &         5 &        44
Mother max. edu.: less than middle school &      0.00 &      0.00 &      0.00 &      0.00 &      0.00 \\
\midrule
Observations &        98 &        51 &        50 &         5 &        44
Mother max. edu.: middle school &      0.09 &      0.08 &      0.02 & \textbf{     0.00} &      0.07 \\
\midrule
Observations &        98 &        51 &        50 &         5 &        44
Mother max. edu.: high school & \textbf{     0.24} & \textbf{     0.29} & \textbf{     0.24} &      0.60 &      0.45 \\
\midrule
Observations &        98 &        51 &        50 &         5 &        44
Mother max. edu.: university & \textbf{     0.66} &      0.63 & \textbf{     0.74} &      0.40 &      0.48 \\
\midrule
Observations &        98 &        51 &        50 &         5 &        44
Father: born in province & \textbf{     0.74} &      0.88 & \textbf{     0.76} &      0.60 &      0.82 \\
\midrule
Observations &        98 &        51 &        50 &         5 &        44
Father max. edu.: less than middle school &      0.00 &      0.00 &      0.00 &      0.00 &      0.00 \\
\midrule
Observations &        98 &        51 &        50 &         5 &        44
Father max. edu.: middle school & \textbf{     0.08} &      0.06 &      0.06 & \textbf{     0.00} & \textbf{     0.14} \\
\midrule
Observations &        98 &        51 &        50 &         5 &        44
Father max. edu.: high school &      0.34 &      0.33 & \textbf{     0.20} &      0.40 &      0.41 \\
\midrule
Observations &        98 &        51 &        50 &         5 &        44
Father max. edu.: university &      0.58 &      0.61 & \textbf{     0.74} &      0.60 &      0.45 \\
\midrule
Observations &        98 &        51 &        50 &         5 &        44
Religious caregiver indicator & \textbf{     0.74} &      0.51 & \textbf{     0.80} &      0.80 & \textbf{     0.84} \\
\midrule
Observations &        98 &        51 &        50 &         5 &        44
Caregiver is Catholic &      0.00 &      0.00 &      0.00 &      0.00 &      0.00 \\
\midrule
Observations &        98 &        51 &        50 &         5 &        44
Caregiver is Catholic AND more faithful than the average. &      0.00 &      0.00 &      0.00 &      0.00 &      0.00 \\
\midrule
Observations &        98 &        51 &        50 &         5 &        44
\bottomrule
\end{tabular}

}
\end{table}

\begin{table}[H]
\caption{Baseline, Padova, Adult 30's}
\scalebox{0.8}{
\begin{tabular}{l c c c c c c }
\toprule
& \textbf{Municipal} & \textbf{State} & \textbf{Religious} & \textbf{Private} & \textbf{None} \\
\midrule
Male indicator &      0.51 & \textbf{     0.38} & \textbf{     0.53} &      0.00 &      0.70 \\
\midrule
Observations &        35 &        26 &       140 &         1 &        47
Age &     32.96 &     32.89 & \textbf{    33.04} &     32.07 & \textbf{    33.12} \\
\midrule
Observations &        35 &        26 &       140 &         1 &        47
CAPI & \textbf{     0.20} & \textbf{     0.31} & \textbf{     0.36} &      0.00 & \textbf{     0.47} \\
\midrule
Observations &        35 &        26 &       140 &         1 &        47
Mother: born in province & \textbf{     0.69} &      0.73 & \textbf{     0.71} &      0.00 & \textbf{     0.70} \\
\midrule
Observations &        35 &        26 &       140 &         1 &        47
Mother max. edu.: less than middle school &      0.00 &      0.00 &      0.00 &      0.00 &      0.00 \\
\midrule
Observations &        35 &        26 &       140 &         1 &        47
Mother max. edu.: middle school &      0.11 &      0.12 & \textbf{     0.09} &      0.00 &      0.11 \\
\midrule
Observations &        35 &        26 &       140 &         1 &        47
Mother max. edu.: high school &      0.37 &      0.35 & \textbf{     0.33} &      0.00 &      0.40 \\
\midrule
Observations &        35 &        26 &       140 &         1 &        47
Mother max. edu.: university &      0.49 &      0.50 &      0.58 &      1.00 &      0.49 \\
\midrule
Observations &        35 &        26 &       140 &         1 &        47
Father: born in province & \textbf{     0.71} &      0.77 & \textbf{     0.74} &      0.00 &      0.85 \\
\midrule
Observations &        35 &        26 &       140 &         1 &        47
Father max. edu.: less than middle school &      0.00 &      0.00 &      0.00 &      0.00 &      0.00 \\
\midrule
Observations &        35 &        26 &       140 &         1 &        47
Father max. edu.: middle school &      0.09 & \textbf{     0.15} & \textbf{     0.11} &      0.00 &      0.04 \\
\midrule
Observations &        35 &        26 &       140 &         1 &        47
Father max. edu.: high school &      0.34 &      0.42 & \textbf{     0.29} &      0.00 &      0.34 \\
\midrule
Observations &        35 &        26 &       140 &         1 &        47
Father max. edu.: university &      0.57 &      0.42 &      0.58 &      1.00 &      0.60 \\
\midrule
Observations &        35 &        26 &       140 &         1 &        47
Religious caregiver indicator & \textbf{     0.57} &      0.54 & \textbf{     0.77} &      1.00 & \textbf{     0.79} \\
\midrule
Observations &        35 &        26 &       140 &         1 &        47
Caregiver is Catholic &      0.00 &      0.00 &      0.00 &      0.00 &      0.00 \\
\midrule
Observations &        35 &        26 &       140 &         1 &        47
Caregiver is Catholic AND more faithful than the average. &      0.00 &      0.00 &      0.00 &      0.00 &      0.00 \\
\midrule
Observations &        35 &        26 &       140 &         1 &        47
\bottomrule
\end{tabular}

}
\end{table}


\subsubsection{Linear Probability Model}
\begin{table}[H]
\caption{LPM Estimation Reggio - Adults (Age 30), Materna}
\centering
\scalebox{0.7}{
\begin{tabular}{lccccc}
\toprule
 & \textbf{None} & \textbf{Municipal} & \textbf{Religious} & \textbf{Private} & \textbf{State} \\
\midrule
\textbf{Respondent's Baseline Info} \\
\quad Male &     -0.08 & \textbf{     0.10} &      0.07 &      0.00 & \textbf{    -0.09} \\
\quad  & (     0.05 ) & \textbf{(     0.06 )}  & (     0.04 )  & (     0.01 ) & \textbf{(     0.04 )} \\
\quad CAPI &     -0.02 & \textbf{     0.14} &     -0.04 &      0.01 & \textbf{    -0.08} \\
\quad  & (     0.05 ) & \textbf{(     0.06 )}  & (     0.04 )  & (     0.01 ) & \textbf{(     0.04 )} \\
\midrule
\textbf{Mother's Baseline Info} \\
\quad Max Education: Middle School &      0.10 &      0.37 & \textbf{    -0.73} & \textbf{     0.15} &      0.12 \\
\quad  & (     0.43 ) & (     0.53 )  & \textbf{(     0.37 )}  & \textbf{(     0.06 )} & (     0.34 ) \\
\quad Max Education: High School &      0.19 &      0.19 & \textbf{    -0.78} &      0.09 &      0.30 \\
\quad  & (     0.45 ) & (     0.55 )  & \textbf{(     0.38 )}  & (     0.06 ) & (     0.35 ) \\
\quad Max Education: University &      0.35 &      0.28 & \textbf{    -0.95} &      0.10 &      0.21 \\
\quad  & (     0.45 ) & (     0.55 )  & \textbf{(     0.39 )}  & (     0.06 ) & (     0.35 ) \\
\quad Born in Province &      0.06 &     -0.07 &     -0.02 &      0.01 &      0.03 \\
\quad  & (     0.07 ) & (     0.09 )  & (     0.06 )  & (     0.01 ) & (     0.05 ) \\
\midrule
\textbf{Father's Baseline Info} \\
\quad Max Education: Middle School &     -0.03 &      0.61 &      0.10 & \textbf{     0.10} & \textbf{    -0.78} \\
\quad  & (     0.45 ) & (     0.55 )  & (     0.38 )  & \textbf{(     0.06 )} & \textbf{(     0.35 )} \\
\quad Max Education: High School &      0.02 &      0.75 &      0.17 &     -0.00 & \textbf{    -0.93} \\
\quad  & (     0.41 ) & (     0.50 )  & (     0.35 )  & (     0.06 ) & \textbf{(     0.31 )} \\
\quad Max Education: University &     -0.04 &      0.68 &      0.22 &     -0.00 & \textbf{    -0.86} \\
\quad  & (     0.41 ) & (     0.50 )  & (     0.35 )  & (     0.06 ) & \textbf{(     0.32 )} \\
\quad Born in Province &     -0.01 &      0.03 &     -0.02 &      0.00 &     -0.00 \\
\quad  & (     0.08 ) & (     0.09 )  & (     0.07 )  & (     0.01 ) & (     0.06 ) \\
\midrule
\textbf{Household Baseline Info} \\
\quad Caregiver Has Religion & \textbf{     0.12} & \textbf{    -0.17} & \textbf{     0.15} &     -0.00 & \textbf{    -0.09} \\
\quad  & \textbf{(     0.05 )} & \textbf{(     0.06 )}  & \textbf{(     0.04 )}  & (     0.01 ) & \textbf{(     0.04 )} \\
\midrule
Observations & 278 & 278 & 278 & 278 & 278 \\
Fraction Attending Each Type &      0.21 &      0.54 &      0.14 &      0.00 &      0.11 \\
\midrule
$ R^2$ &      0.09 &      0.09 &      0.10 &      0.16 &      0.10 \\
\bottomrule
\end{tabular}}
\end{table}
\begin{scriptsize}
\noindent\underline{Note:} This table presents the linear probability model estimations for attending each type of Materna schools, indicated by each column. The samples used in this estimation are those who were adults in their 30's at the time of the survey living in Reggio. All dependent variables are binary. Observation indicates the number of people included in this sample. Bold number indicates that the p-value is less than or equal to 0.1. Standard errors are reported in parentheses.
\end{scriptsize}

\begin{table}[H]
\caption{LPM Estimation Parma - Adults (Age 30), Materna}
\centering
\scalebox{0.7}{
\begin{tabular}{lccccc}
\toprule
 & \textbf{None} & \textbf{Municipal} & \textbf{Religious} & \textbf{Private} & \textbf{State} \\
\midrule
\textbf{Respondent's Baseline Info} \\
\quad Male &      0.03 &      0.01 &      0.00 &     -0.01 &     -0.03 \\
\quad  & (     0.05 ) & (     0.06 )  & (     0.05 )  & (     0.02 ) & (     0.05 ) \\
\quad CAPI &      0.04 & \textbf{    -0.22} & \textbf{     0.19} &      0.00 &     -0.00 \\
\quad  & (     0.05 ) & \textbf{(     0.06 )}  & \textbf{(     0.05 )}  & (     0.02 ) & (     0.05 ) \\
\midrule
\textbf{Mother's Baseline Info} \\
\quad Max Education: Middle School &     -0.10 &      0.10 &     -0.15 &      0.01 &      0.13 \\
\quad  & (     0.14 ) & (     0.17 )  & (     0.14 )  & (     0.05 ) & (     0.14 ) \\
\quad Max Education: High School &      0.10 & \textbf{    -0.15} &     -0.00 & \textbf{     0.04} &      0.02 \\
\quad  & (     0.07 ) & \textbf{(     0.09 )}  & (     0.07 )  & \textbf{(     0.03 )} & (     0.07 ) \\
\quad Born in Province &      0.03 &     -0.11 &      0.05 &     -0.01 &      0.05 \\
\quad  & (     0.05 ) & (     0.07 )  & (     0.06 )  & (     0.02 ) & (     0.06 ) \\
\midrule
\textbf{Father's Baseline Info} \\
\quad Max Education: Middle School &      0.17 &     -0.10 &      0.11 &     -0.02 &     -0.16 \\
\quad  & (     0.13 ) & (     0.16 )  & (     0.13 )  & (     0.05 ) & (     0.13 ) \\
\quad Max Education: University &     -0.04 &     -0.06 &      0.09 &      0.02 &     -0.01 \\
\quad  & (     0.07 ) & (     0.08 )  & (     0.07 )  & (     0.02 ) & (     0.07 ) \\
\quad Born in Province &      0.03 &     -0.07 &     -0.04 &     -0.02 &      0.10 \\
\quad  & (     0.06 ) & (     0.08 )  & (     0.06 )  & (     0.02 ) & (     0.06 ) \\
\midrule
\textbf{Household Baseline Info} \\
\quad Caregiver Has Religion & \textbf{     0.09} &      0.03 &      0.09 &      0.00 & \textbf{    -0.22} \\
\quad  & \textbf{(     0.05 )} & (     0.07 )  & (     0.06 )  & (     0.02 ) & \textbf{(     0.06 )} \\
\midrule
Observations & 248 & 248 & 248 & 248 & 248 \\
Fraction Attending Each Type &      0.18 &      0.40 &      0.20 &      0.02 &      0.21 \\
\midrule
$ R^2$ &      0.05 &      0.09 &      0.09 &      0.02 &      0.08 \\
\bottomrule
\end{tabular}}
\end{table}
\begin{scriptsize}
\noindent\underline{Note:} This table presents the linear probability model estimations for attending each type of Materna schools, indicated by each column. The samples used in this estimation are those who were adults in their 30's at the time of the survey living in Parma. All dependent variables are binary. Observation indicates the number of people included in this sample. Bold number indicates that the p-value is less than or equal to 0.1. Standard errors are reported in parentheses.
\end{scriptsize}

\begin{table}[H]
\caption{LPM Estimation Padova - Adults (Age 30), Materna}
\centering
\scalebox{0.7}{
\begin{tabular}{lccccc}
\toprule
 & \textbf{None} & \textbf{Municipal} & \textbf{Religious} & \textbf{Private} & \textbf{State} \\
\midrule
\textbf{Respondent's Baseline Info} \\
\quad Male & \textbf{     0.13} &      0.00 &     -0.05 &     -0.01 & \textbf{    -0.07} \\
\quad  & \textbf{(     0.05 )} & (     0.04 )  & (     0.06 )  & (     0.01 ) & \textbf{(     0.04 )} \\
\quad CAPI & \textbf{     0.09} & \textbf{    -0.11} &      0.05 &     -0.01 &     -0.02 \\
\quad  & \textbf{(     0.05 )} & \textbf{(     0.05 )}  & (     0.07 )  & (     0.01 ) & (     0.04 ) \\
\midrule
\textbf{Mother's Baseline Info} \\
\quad Max Education: Middle School &      0.41 &     -0.27 &      0.28 &     -0.01 & \textbf{    -0.40} \\
\quad  & (     0.30 ) & (     0.27 )  & (     0.38 )  & (     0.05 ) & \textbf{(     0.24 )} \\
\quad Max Education: High School &      0.31 &     -0.36 &      0.45 &     -0.02 & \textbf{    -0.38} \\
\quad  & (     0.28 ) & (     0.25 )  & (     0.36 )  & (     0.05 ) & \textbf{(     0.22 )} \\
\quad Max Education: University &      0.18 &     -0.40 &      0.55 &     -0.01 &     -0.31 \\
\quad  & (     0.28 ) & (     0.25 )  & (     0.36 )  & (     0.05 ) & (     0.22 ) \\
\quad Born in Province &      0.01 &     -0.04 &      0.05 &     -0.01 &     -0.00 \\
\quad  & (     0.06 ) & (     0.05 )  & (     0.07 )  & (     0.01 ) & (     0.04 ) \\
\midrule
\textbf{Father's Baseline Info} \\
\quad Max Education: Middle School &     -0.39 &      0.04 &      0.12 &     -0.00 &      0.22 \\
\quad  & (     0.25 ) & (     0.22 )  & (     0.32 )  & (     0.04 ) & (     0.20 ) \\
\quad Max Education: High School &     -0.17 &      0.13 &     -0.12 &     -0.00 &      0.16 \\
\quad  & (     0.23 ) & (     0.21 )  & (     0.29 )  & (     0.04 ) & (     0.18 ) \\
\quad Max Education: University &     -0.07 &      0.14 &     -0.13 &     -0.00 &      0.06 \\
\quad  & (     0.23 ) & (     0.20 )  & (     0.29 )  & (     0.04 ) & (     0.18 ) \\
\quad Born in Province &      0.08 &     -0.03 &     -0.05 & \textbf{    -0.02} &      0.01 \\
\quad  & (     0.06 ) & (     0.05 )  & (     0.08 )  & \textbf{(     0.01 )} & (     0.05 ) \\
\midrule
\textbf{Household Baseline Info} \\
\quad Caregiver Has Religion &      0.04 & \textbf{    -0.09} & \textbf{     0.13} &      0.01 & \textbf{    -0.08} \\
\quad  & (     0.06 ) & \textbf{(     0.05 )}  & \textbf{(     0.07 )}  & (     0.01 ) & \textbf{(     0.04 )} \\
\midrule
Observations & 249 & 249 & 249 & 249 & 249 \\
Fraction Attending Each Type &      0.19 &      0.14 &      0.56 &      0.00 &      0.10 \\
\midrule
$ R^2$ &      0.08 &      0.05 &      0.05 &      0.03 &      0.06 \\
\bottomrule
\end{tabular}}
\end{table}
\begin{scriptsize}
\noindent\underline{Note:} This table presents the linear probability model estimations for attending each type of Materna schools, indicated by each column. The samples used in this estimation are those who were adults in their 30's at the time of the survey living in Padova. All dependent variables are binary. Observation indicates the number of people included in this sample. Bold number indicates that the p-value is less than or equal to 0.1. Standard errors are reported in parentheses.
\end{scriptsize}




\subsection{Adults at Age 40}
\subsubsection{Unconditional Mean}
\begin{table}[H]
\caption{Baseline, Reggio, Adult 40's}
\scalebox{0.8}{
\begin{tabular}{l c c c c c c }
\toprule
& \textbf{Municipal} & \textbf{State} & \textbf{Religious} & \textbf{Private} & \textbf{None} \\
\midrule
Male indicator &      0.58 &      0.53 &      0.56 &      0.60 &      0.46 \\
\midrule
Observations &       128 &        17 &        52 &         5 &        80
Age &     43.66 &     43.73 &     43.56 &     43.52 &     43.75 \\
\midrule
Observations &       128 &        17 &        52 &         5 &        80
CAPI &      0.59 & \textbf{     0.88} &      0.69 &      0.60 &      0.61 \\
\midrule
Observations &       128 &        17 &        52 &         5 &        80
Mother: born in province &      0.92 &      0.82 &      0.87 &      0.60 & \textbf{     0.56} \\
\midrule
Observations &       128 &        17 &        52 &         5 &        80
Mother max. edu.: less than middle school &      0.02 &      0.00 &      0.00 &      0.00 &      0.04 \\
\midrule
Observations &       128 &        17 &        52 &         5 &        80
Mother max. edu.: middle school &      0.28 &      0.29 & \textbf{     0.13} &      0.40 & \textbf{     0.04} \\
\midrule
Observations &       128 &        17 &        52 &         5 &        80
Mother max. edu.: high school &      0.47 & \textbf{     0.24} & \textbf{     0.62} &      0.40 &      0.45 \\
\midrule
Observations &       128 &        17 &        52 &         5 &        80
Mother max. edu.: university &      0.23 & \textbf{     0.47} &      0.25 &      0.20 & \textbf{     0.45} \\
\midrule
Observations &       128 &        17 &        52 &         5 &        80
Father: born in province &      0.87 &      0.76 &      0.79 &      0.40 & \textbf{     0.65} \\
\midrule
Observations &       128 &        17 &        52 &         5 &        80
Father max. edu.: less than middle school &      0.02 & \textbf{     0.00} & \textbf{     0.00} & \textbf{     0.00} &      0.04 \\
\midrule
Observations &       128 &        17 &        52 &         5 &        80
Father max. edu.: middle school &      0.28 &      0.29 & \textbf{     0.13} &      0.20 & \textbf{     0.05} \\
\midrule
Observations &       128 &        17 &        52 &         5 &        80
Father max. edu.: high school &      0.41 &      0.41 & \textbf{     0.63} &      0.60 &      0.40 \\
\midrule
Observations &       128 &        17 &        52 &         5 &        80
Father max. edu.: university &      0.28 &      0.29 &      0.23 &      0.20 & \textbf{     0.47} \\
\midrule
Observations &       128 &        17 &        52 &         5 &        80
Religious caregiver indicator &      0.39 &      0.47 & \textbf{     0.54} &      0.60 & \textbf{     0.65} \\
\midrule
Observations &       128 &        17 &        52 &         5 &        80
Caregiver is Catholic &      0.00 &      0.00 &      0.00 &      0.00 &      0.00 \\
\midrule
Observations &       128 &        17 &        52 &         5 &        80
Caregiver is Catholic AND more faithful than the average. &      0.00 &      0.00 &      0.00 &      0.00 &      0.00 \\
\midrule
Observations &       128 &        17 &        52 &         5 &        80
\bottomrule
\end{tabular}

}
\end{table}

\begin{table}[H]
\caption{Baseline, Parma, Adult 40's}
\scalebox{0.8}{
\begin{tabular}{l c c c c c c c c c c c c c c c c c c}
\toprule
& \multicolumn{3}{c}{Municipal} & \multicolumn{3}{c}{State} & \multicolumn{3}{c}{Religious} & \multicolumn{3}{c}{Private} & \multicolumn{3}{c}{None} & Unconditional R^2 & Conditional R^2 & Conditional N\\
& \scriptsize Mean & \scriptsize C. Mean & \scriptsize N & \scriptsize Mean & \scriptsize C. Mean & \scriptsize N & \scriptsize Mean & \scriptsize C. Mean & \scriptsize C. Mean & \scriptsize C. Mean & \scriptsize N & \scriptsize Mean & \scriptsize C. Mean & \scriptsize N & \scriptsize Mean & & \scriptsize C. Mean \scriptsize N & & & \\
\midrule
Male indicator &         . &         . & &         . &         . & &         . &         . & &         . &         . & &         . &         . & &      0.00 &      0.00 &      3974 \\
Age &         . &         . & & \textbf{        .} &         . & &         . &         . & &         . &         . & &         . &         . & &      0.26 &      0.26 &      3974 \\
CAPI & \textbf{        .} &         . & & \textbf{        .} &         . & &         . &         . & &         . &         . & & \textbf{        .} &         . & &      0.01 &      0.01 &      3974 \\
Mother: born in province & \textbf{        .} &         . & &         . &         . & & \textbf{        .} &         . & &         . &         . & & \textbf{        .} &         . & &      0.01 &      0.01 &      3974 \\
Mother max. edu.: less than middle school &         . &         . & &         . &         . & &         . &         . & &         . &         . & &         . &         . & &      0.02 &      0.02 &      3974 \\
Mother max. edu.: middle school &         . &         . & & \textbf{        .} &         . & &         . &         . & &         . &         . & &         . &         . & &      0.01 &      0.01 &      3974 \\
Mother max. edu.: high school & \textbf{        .} &         . & &         . &         . & &         . &         . & &         . &         . & & \textbf{        .} &         . & &      0.00 &      0.00 &      3974 \\
Mother max. edu.: university & \textbf{        .} &         . & &         . &         . & & \textbf{        .} &         . & &         . &         . & & \textbf{        .} &         . & &      0.00 &      0.00 &      3974 \\
Father: born in province &         . &         . & &         . &         . & &         . &         . & &         . &         . & &         . &         . & &      0.02 &      0.02 &      3974 \\
Father max. edu.: less than middle school & \textbf{        .} &         . & & \textbf{        .} &         . & & \textbf{        .} &         . & &         . &         . & &         . &         . & &      0.01 &      0.01 &      3974 \\
Father max. edu.: middle school &         . &         . & & \textbf{        .} &         . & & \textbf{        .} &         . & &         . &         . & &         . &         . & &      0.01 &      0.01 &      3974 \\
Father max. edu.: high school & \textbf{        .} &         . & &         . &         . & &         . &         . & &         . &         . & &         . &         . & &      0.00 &      0.00 &      3974 \\
Father max. edu.: university & \textbf{        .} &         . & &         . &         . & & \textbf{        .} &         . & &         . &         . & &         . &         . & &      0.01 &      0.01 &      3974 \\
Religious caregiver indicator & \textbf{        .} &         . & & \textbf{        .} &         . & & \textbf{        .} &         . & &         . &         . & & \textbf{        .} &         . & &      0.01 &      0.01 &      3974 \\
\bottomrule
\end{tabular}

}
\end{table}

\begin{table}[H]
\caption{Baseline, Padova, Adult 40's}
\scalebox{0.8}{
\begin{tabular}{l c c c c c c c c c c c c c c c c c c}
\toprule
& \multicolumn{3}{c}{Municipal} & \multicolumn{3}{c}{State} & \multicolumn{3}{c}{Religious} & \multicolumn{3}{c}{Private} & \multicolumn{3}{c}{None} & Unconditional R^2 & Conditional R^2 & Conditional N\\
& \scriptsize Mean & \scriptsize C. Mean & \scriptsize N & \scriptsize Mean & \scriptsize C. Mean & \scriptsize N & \scriptsize Mean & \scriptsize C. Mean & \scriptsize C. Mean & \scriptsize C. Mean & \scriptsize N & \scriptsize Mean & \scriptsize C. Mean & \scriptsize N & \scriptsize Mean & & \scriptsize C. Mean \scriptsize N & & & \\
\midrule
Male indicator & \textbf{        .} &         . & &         . &         . & & \textbf{        .} &         . & &         . &         . & &         . &         . & &      0.00 &      0.00 &      3974 \\
Age & \textbf{        .} &         . & & \textbf{        .} &         . & & \textbf{        .} &         . & &         . &         . & & \textbf{        .} &         . & &      0.26 &      0.26 &      3974 \\
CAPI &         . &         . & & \textbf{        .} &         . & & \textbf{        .} &         . & &         . &         . & & \textbf{        .} &         . & &      0.01 &      0.01 &      3974 \\
Mother: born in province & \textbf{        .} &         . & & \textbf{        .} &         . & & \textbf{        .} &         . & &         . &         . & & \textbf{        .} &         . & &      0.01 &      0.01 &      3974 \\
Mother max. edu.: less than middle school &         . &         . & &         . &         . & &         . &         . & &         . &         . & &         . &         . & &      0.02 &      0.02 &      3974 \\
Mother max. edu.: middle school & \textbf{        .} &         . & & \textbf{        .} &         . & &         . &         . & &         . &         . & & \textbf{        .} &         . & &      0.01 &      0.01 &      3974 \\
Mother max. edu.: high school & \textbf{        .} &         . & &         . &         . & & \textbf{        .} &         . & &         . &         . & & \textbf{        .} &         . & &      0.00 &      0.00 &      3974 \\
Mother max. edu.: university &         . &         . & &         . &         . & & \textbf{        .} &         . & &         . &         . & & \textbf{        .} &         . & &      0.00 &      0.00 &      3974 \\
Father: born in province &         . &         . & &         . &         . & & \textbf{        .} &         . & &         . &         . & & \textbf{        .} &         . & &      0.02 &      0.02 &      3974 \\
Father max. edu.: less than middle school & \textbf{        .} &         . & &         . &         . & & \textbf{        .} &         . & &         . &         . & & \textbf{        .} &         . & &      0.01 &      0.01 &      3974 \\
Father max. edu.: middle school &         . &         . & & \textbf{        .} &         . & & \textbf{        .} &         . & &         . &         . & & \textbf{        .} &         . & &      0.01 &      0.01 &      3974 \\
Father max. edu.: high school &         . &         . & & \textbf{        .} &         . & & \textbf{        .} &         . & &         . &         . & & \textbf{        .} &         . & &      0.00 &      0.00 &      3974 \\
Father max. edu.: university &         . &         . & &         . &         . & & \textbf{        .} &         . & &         . &         . & & \textbf{        .} &         . & &      0.01 &      0.01 &      3974 \\
Religious caregiver indicator & \textbf{        .} &         . & & \textbf{        .} &         . & & \textbf{        .} &         . & &         . &         . & & \textbf{        .} &         . & &      0.01 &      0.01 &      3974 \\
\bottomrule
\end{tabular}

}
\end{table}

\subsubsection{Linear Probability Model}
\begin{table}[H]
\caption{LPM Estimation Reggio - Adults (Age 40), Materna}
\centering
\scalebox{0.7}{
\begin{tabular}{lccccc}
\toprule
 & \textbf{None} & \textbf{Municipal} & \textbf{Religious} & \textbf{Private} & \textbf{State} \\
\midrule
\textbf{Respondent's Baseline Info} \\
\quad Male &     -0.05 &      0.04 &     -0.00 &      0.00 &      0.01 \\
\quad  & (     0.05 ) & (     0.06 )  & (     0.05 )  & (     0.02 ) & (     0.03 ) \\
\quad CAPI &     -0.06 &     -0.08 &      0.07 &     -0.01 & \textbf{     0.08} \\
\quad  & (     0.05 ) & (     0.06 )  & (     0.05 )  & (     0.02 ) & \textbf{(     0.03 )} \\
\midrule
\textbf{Mother's Baseline Info} \\
\quad Max Education: Middle School &     -0.25 &      0.20 &     -0.01 &      0.06 &     -0.00 \\
\quad  & (     0.25 ) & (     0.28 )  & (     0.23 )  & (     0.08 ) & (     0.14 ) \\
\quad Max Education: High School &     -0.13 &      0.11 &      0.05 &      0.02 &     -0.04 \\
\quad  & (     0.24 ) & (     0.27 )  & (     0.23 )  & (     0.08 ) & (     0.14 ) \\
\quad Max Education: University &     -0.11 &     -0.05 &      0.06 &      0.01 &      0.09 \\
\quad  & (     0.24 ) & (     0.27 )  & (     0.23 )  & (     0.08 ) & (     0.14 ) \\
\quad Born in Province & \textbf{    -0.34} & \textbf{     0.24} &      0.09 &     -0.01 &      0.02 \\
\quad  & \textbf{(     0.07 )} & \textbf{(     0.08 )}  & (     0.06 )  & (     0.02 ) & (     0.04 ) \\
\midrule
\textbf{Father's Baseline Info} \\
\quad Max Education: Middle School &     -0.32 &      0.07 &      0.15 &     -0.03 &      0.13 \\
\quad  & (     0.22 ) & (     0.25 )  & (     0.21 )  & (     0.07 ) & (     0.13 ) \\
\quad Max Education: High School &     -0.26 &     -0.08 &      0.24 &      0.00 &      0.11 \\
\quad  & (     0.22 ) & (     0.25 )  & (     0.20 )  & (     0.07 ) & (     0.13 ) \\
\quad Max Education: University &     -0.20 &      0.08 &      0.11 &     -0.01 &      0.03 \\
\quad  & (     0.22 ) & (     0.25 )  & (     0.21 )  & (     0.07 ) & (     0.13 ) \\
\quad Born in Province &     -0.08 & \textbf{     0.12} &     -0.01 & \textbf{    -0.04} &      0.00 \\
\quad  & (     0.06 ) & \textbf{(     0.07 )}  & (     0.06 )  & \textbf{(     0.02 )} & (     0.04 ) \\
\midrule
\textbf{Household Baseline Info} \\
\quad Caregiver Has Religion &      0.08 & \textbf{    -0.14} &      0.04 &      0.01 &      0.01 \\
\quad  & (     0.05 ) & \textbf{(     0.06 )}  & (     0.05 )  & (     0.02 ) & (     0.03 ) \\
\midrule
Observations & 282 & 282 & 282 & 282 & 282 \\
Fraction Attending Each Type &      0.28 &      0.45 &      0.18 &      0.02 &      0.06 \\
\midrule
$ R^2$ &      0.22 &      0.16 &      0.05 &      0.03 &      0.06 \\
\bottomrule
\end{tabular}}
\end{table}
\begin{scriptsize}
\noindent\underline{Note:} This table presents the linear probability model estimations for attending each type of Materna schools, indicated by each column. The samples used in this estimation are those who were adults in their 40's at the time of the survey living in Reggio. All dependent variables are binary. Observation indicates the number of people included in this sample. Bold number indicates that the p-value is less than or equal to 0.1. Standard errors are reported in parentheses.
\end{scriptsize}

\begin{table}[H]
\caption{LPM Estimation Parma - Adults (Age 40), Materna}
\centering
\scalebox{0.7}{
\begin{tabular}{lccccc}
\toprule
 & \textbf{None} & \textbf{Municipal} & \textbf{Religious} & \textbf{Private} & \textbf{State} \\
\midrule
\textbf{Respondent's Baseline Info} \\
\quad Male &      0.01 &      0.06 &     -0.05 &      0.01 &     -0.03 \\
\quad  & (     0.06 ) & (     0.05 )  & (     0.05 )  & (     0.01 ) & (     0.04 ) \\
\quad CAPI & \textbf{    -0.18} &     -0.05 & \textbf{     0.16} &      0.01 &      0.06 \\
\quad  & \textbf{(     0.07 )} & (     0.06 )  & \textbf{(     0.06 )}  & (     0.01 ) & (     0.04 ) \\
\midrule
\textbf{Mother's Baseline Info} \\
\quad Max Education: Middle School &     -0.18 &      0.02 &      0.08 &      0.03 &      0.06 \\
\quad  & (     0.70 ) & (     0.59 )  & (     0.59 )  & (     0.09 ) & (     0.44 ) \\
\quad Max Education: High School &      0.05 &     -0.05 &     -0.04 &      0.02 &      0.03 \\
\quad  & (     0.70 ) & (     0.59 )  & (     0.59 )  & (     0.09 ) & (     0.44 ) \\
\quad Max Education: University &      0.18 &     -0.12 &     -0.13 &      0.03 &      0.04 \\
\quad  & (     0.71 ) & (     0.60 )  & (     0.60 )  & (     0.09 ) & (     0.44 ) \\
\quad Born in Province & \textbf{    -0.14} &      0.05 &      0.04 &      0.01 &      0.05 \\
\quad  & \textbf{(     0.07 )} & (     0.06 )  & (     0.06 )  & (     0.01 ) & (     0.05 ) \\
\midrule
\textbf{Father's Baseline Info} \\
\quad Max Education: Middle School &     -0.52 &      0.18 &      0.15 &     -0.02 &      0.20 \\
\quad  & (     0.50 ) & (     0.42 )  & (     0.42 )  & (     0.07 ) & (     0.31 ) \\
\quad Max Education: High School &     -0.59 &      0.18 &      0.34 &     -0.02 &      0.09 \\
\quad  & (     0.50 ) & (     0.42 )  & (     0.43 )  & (     0.07 ) & (     0.32 ) \\
\quad Max Education: University & \textbf{    -0.88} &      0.39 &      0.45 &     -0.01 &      0.05 \\
\quad  & \textbf{(     0.51 )} & (     0.43 )  & (     0.43 )  & (     0.07 ) & (     0.32 ) \\
\quad Born in Province &      0.13 &     -0.04 &     -0.07 &      0.00 &     -0.02 \\
\quad  & (     0.09 ) & (     0.07 )  & (     0.07 )  & (     0.01 ) & (     0.05 ) \\
\midrule
\textbf{Household Baseline Info} \\
\quad Caregiver Has Religion &      0.07 &     -0.00 &     -0.03 &     -0.01 &     -0.03 \\
\quad  & (     0.07 ) & (     0.06 )  & (     0.06 )  & (     0.01 ) & (     0.05 ) \\
\midrule
Observations & 250 & 250 & 250 & 250 & 250 \\
Fraction Attending Each Type &      0.46 &      0.21 &      0.22 &      0.00 &      0.10 \\
\midrule
$ R^2$ &      0.11 &      0.05 &      0.08 &      0.03 &      0.06 \\
\bottomrule
\end{tabular}}
\end{table}
\begin{scriptsize}
\noindent\underline{Note:} This table presents the linear probability model estimations for attending each type of Materna schools, indicated by each column. The samples used in this estimation are those who were adults in their 40's at the time of the survey living in Parma. All dependent variables are binary. Observation indicates the number of people included in this sample. Bold number indicates that the p-value is less than or equal to 0.1. Standard errors are reported in parentheses.
\end{scriptsize}

\begin{table}[H]
\caption{LPM Estimation Padova - Adults (Age 40), Materna}
\centering
\scalebox{0.7}{
\begin{tabular}{lccccc}
\toprule
 & \textbf{None} & \textbf{Municipal} & \textbf{Religious} & \textbf{Private} & \textbf{State} \\
\midrule
\textbf{Respondent's Baseline Info} \\
\quad Male &      0.07 & \textbf{    -0.07} &     -0.02 &      0.00 &      0.01 \\
\quad  & (     0.06 ) & \textbf{(     0.04 )}  & (     0.06 )  & (        . ) & (     0.04 ) \\
\quad CAPI &     -0.04 &      0.07 & \textbf{    -0.20} &      0.00 & \textbf{     0.16} \\
\quad  & (     0.06 ) & (     0.04 )  & \textbf{(     0.07 )}  & (        . ) & \textbf{(     0.04 )} \\
\midrule
\textbf{Mother's Baseline Info} \\
\quad Max Education: Middle School &      0.16 &     -0.03 &     -0.07 &      0.00 &     -0.06 \\
\quad  & (     0.26 ) & (     0.17 )  & (     0.28 )  & (        . ) & (     0.15 ) \\
\quad Max Education: High School &      0.25 &     -0.16 &     -0.15 &      0.00 &      0.07 \\
\quad  & (     0.25 ) & (     0.17 )  & (     0.27 )  & (        . ) & (     0.15 ) \\
\quad Max Education: University &      0.34 &     -0.18 &     -0.20 &      0.00 &      0.04 \\
\quad  & (     0.25 ) & (     0.17 )  & (     0.28 )  & (        . ) & (     0.15 ) \\
\quad Born in Province &      0.05 &     -0.04 &     -0.01 &      0.00 &      0.00 \\
\quad  & (     0.06 ) & (     0.04 )  & (     0.07 )  & (        . ) & (     0.04 ) \\
\midrule
\textbf{Father's Baseline Info} \\
\quad Max Education: Middle School &     -0.00 &      0.33 &     -0.00 &      0.00 &     -0.33 \\
\quad  & (     0.36 ) & (     0.24 )  & (     0.39 )  & (        . ) & (     0.22 ) \\
\quad Max Education: High School &      0.02 &      0.32 &     -0.07 &      0.00 &     -0.26 \\
\quad  & (     0.35 ) & (     0.23 )  & (     0.38 )  & (        . ) & (     0.21 ) \\
\quad Max Education: University &      0.10 &      0.27 &      0.05 &      0.00 & \textbf{    -0.41} \\
\quad  & (     0.35 ) & (     0.24 )  & (     0.39 )  & (        . ) & \textbf{(     0.21 )} \\
\quad Born in Province &      0.02 &      0.04 &     -0.05 &      0.00 &     -0.01 \\
\quad  & (     0.07 ) & (     0.04 )  & (     0.07 )  & (        . ) & (     0.04 ) \\
\midrule
\textbf{Household Baseline Info} \\
\quad Caregiver Has Religion &     -0.11 &     -0.06 & \textbf{     0.15} &      0.00 &      0.01 \\
\quad  & (     0.07 ) & (     0.05 )  & \textbf{(     0.07 )}  & (        . ) & (     0.04 ) \\
\midrule
Observations & 249 & 249 & 249 & 249 & 249 \\
Fraction Attending Each Type &      0.30 &      0.11 &      0.49 &      0.00 &      0.10 \\
\midrule
$ R^2$ &      0.08 &      0.09 &      0.06 &         . &      0.18 \\
\bottomrule
\end{tabular}}
\end{table}
\begin{scriptsize}
\noindent\underline{Note:} This table presents the linear probability model estimations for attending each type of Materna schools, indicated by each column. The samples used in this estimation are those who were adults in their 40's at the time of the survey living in Padova. All dependent variables are binary. Observation indicates the number of people included in this sample. Bold number indicates that the p-value is less than or equal to 0.1. Standard errors are reported in parentheses.
\end{scriptsize}




\subsection{Adults at Age 50}
\subsubsection{Unconditional Mean}
\begin{table}[H]
\caption{Baseline, Reggio, Adult 50's}
\scalebox{0.8}{
\begin{tabular}{l c c c c c c }
\toprule
& \textbf{Municipal} & \textbf{State} & \textbf{Religious} & \textbf{Private} & \textbf{None} \\
\midrule
Male indicator &      0.67 &      0.60 &      0.36 &      0.50 &      0.47 \\
\midrule
Observations &         9 &        10 &        28 &         2 &       147
Age &     56.81 &     56.44 &     56.12 & \textbf{    55.67} &     56.53 \\
\midrule
Observations &         9 &        10 &        28 &         2 &       147
CAPI &      0.78 &      0.50 &      0.54 &      0.50 & \textbf{     0.39} \\
\midrule
Observations &         9 &        10 &        28 &         2 &       147
Mother: born in province &      1.00 &      1.00 & \textbf{     0.86} &      1.00 & \textbf{     0.73} \\
\midrule
Observations &         9 &        10 &        28 &         2 &       147
Mother max. edu.: less than middle school &      0.00 &      0.00 &      0.00 &      0.00 &      0.01 \\
\midrule
Observations &         9 &        10 &        28 &         2 &       147
Mother max. edu.: middle school &      0.78 &      1.00 &      0.68 & \textbf{     0.00} & \textbf{     0.31} \\
\midrule
Observations &         9 &        10 &        28 &         2 &       147
Mother max. edu.: high school &      0.22 &      0.00 &      0.25 & \textbf{     1.00} &      0.39 \\
\midrule
Observations &         9 &        10 &        28 &         2 &       147
Mother max. edu.: university &      0.00 &      0.00 &      0.07 &      0.00 & \textbf{     0.29} \\
\midrule
Observations &         9 &        10 &        28 &         2 &       147
Father: born in province &      0.89 &      0.90 &      0.86 &      1.00 &      0.83 \\
\midrule
Observations &         9 &        10 &        28 &         2 &       147
Father max. edu.: less than middle school &      0.00 &      0.00 &      0.00 &      0.00 &      0.01 \\
\midrule
Observations &         9 &        10 &        28 &         2 &       147
Father max. edu.: middle school &      0.89 &      0.80 & \textbf{     0.61} & \textbf{     0.00} & \textbf{     0.24} \\
\midrule
Observations &         9 &        10 &        28 &         2 &       147
Father max. edu.: high school &      0.11 &      0.10 &      0.25 & \textbf{     1.00} & \textbf{     0.41} \\
\midrule
Observations &         9 &        10 &        28 &         2 &       147
Father max. edu.: university &      0.00 &      0.00 & \textbf{     0.14} &      0.00 & \textbf{     0.33} \\
\midrule
Observations &         9 &        10 &        28 &         2 &       147
Religious caregiver indicator &      0.11 &      0.10 & \textbf{     0.61} &      0.00 & \textbf{     0.71} \\
\midrule
Observations &         9 &        10 &        28 &         2 &       147
Caregiver is Catholic &      0.00 &      0.00 &      0.00 &      0.00 &      0.00 \\
\midrule
Observations &         9 &        10 &        28 &         2 &       147
Caregiver is Catholic AND more faithful than the average. &      0.00 &      0.00 &      0.00 &      0.00 &      0.00 \\
\midrule
Observations &         9 &        10 &        28 &         2 &       147
\bottomrule
\end{tabular}

}
\end{table}

\begin{table}[H]
\caption{Baseline, Parma, Adult 50's}
\scalebox{0.8}{
\begin{tabular}{l c c c c c c }
\toprule
& \textbf{Municipal} & \textbf{State} & \textbf{Religious} & \textbf{Private} & \textbf{None} \\
\midrule
Male indicator & \textbf{     0.17} &      0.29 &      0.55 &         . &      0.39 \\
\midrule
Observations &        12 &         7 &        11 & . &        72
Age &     56.08 &     56.67 &     56.49 &         . &     56.49 \\
\midrule
Observations &        12 &         7 &        11 & . &        72
CAPI & \textbf{     0.00} & \textbf{     0.14} & \textbf{     0.27} &         . & \textbf{     0.43} \\
\midrule
Observations &        12 &         7 &        11 & . &        72
Mother: born in province &      0.92 &      1.00 &      0.91 &         . & \textbf{     0.74} \\
\midrule
Observations &        12 &         7 &        11 & . &        72
Mother max. edu.: less than middle school &      0.08 &      0.00 &      0.00 &         . & \textbf{     0.04} \\
\midrule
Observations &        12 &         7 &        11 & . &        72
Mother max. edu.: middle school &      0.83 &      0.86 &      0.55 &         . & \textbf{     0.49} \\
\midrule
Observations &        12 &         7 &        11 & . &        72
Mother max. edu.: high school &      0.00 &      0.14 &      0.27 &         . &      0.31 \\
\midrule
Observations &        12 &         7 &        11 & . &        72
Mother max. edu.: university &      0.08 &      0.00 &      0.18 &         . & \textbf{     0.17} \\
\midrule
Observations &        12 &         7 &        11 & . &        72
Father: born in province &      1.00 &      1.00 &      0.82 &         . & \textbf{     0.53} \\
\midrule
Observations &        12 &         7 &        11 & . &        72
Father max. edu.: less than middle school &      0.00 &      0.00 &      0.00 &         . & \textbf{     0.06} \\
\midrule
Observations &        12 &         7 &        11 & . &        72
Father max. edu.: middle school &      0.83 &      0.86 & \textbf{     0.45} &         . & \textbf{     0.50} \\
\midrule
Observations &        12 &         7 &        11 & . &        72
Father max. edu.: high school &      0.08 &      0.14 &      0.18 &         . &      0.22 \\
\midrule
Observations &        12 &         7 &        11 & . &        72
Father max. edu.: university &      0.08 &      0.00 & \textbf{     0.36} &         . & \textbf{     0.21} \\
\midrule
Observations &        12 &         7 &        11 & . &        72
Religious caregiver indicator & \textbf{     0.67} & \textbf{     0.71} & \textbf{     0.73} &         . & \textbf{     0.71} \\
\midrule
Observations &        12 &         7 &        11 & . &        72
Caregiver is Catholic &      0.00 &      0.00 &      0.00 &         . &      0.00 \\
\midrule
Observations &        12 &         7 &        11 & . &        72
Caregiver is Catholic AND more faithful than the average. &      0.00 &      0.00 &      0.00 &         . &      0.00 \\
\midrule
Observations &        12 &         7 &        11 & . &        72
\bottomrule
\end{tabular}

}
\end{table}

\begin{table}[H]
\caption{Baseline, Padova, Adult 50's}
\scalebox{0.8}{
\begin{tabular}{l c c c c c c c c c c c c c c c c c c}
\toprule
& \multicolumn{3}{c}{Municipal} & \multicolumn{3}{c}{State} & \multicolumn{3}{c}{Religious} & \multicolumn{3}{c}{Private} & \multicolumn{3}{c}{None} & Unconditional R^2 & Conditional R^2 & Conditional N\\
& \scriptsize Mean & \scriptsize C. Mean & \scriptsize N & \scriptsize Mean & \scriptsize C. Mean & \scriptsize N & \scriptsize Mean & \scriptsize C. Mean & \scriptsize C. Mean & \scriptsize C. Mean & \scriptsize N & \scriptsize Mean & \scriptsize C. Mean & \scriptsize N & \scriptsize Mean & & \scriptsize C. Mean \scriptsize N & & & \\
\midrule
Male indicator &         . &         . & & \textbf{        .} &         . & &         . &         . & & \textbf{        .} &         . & &         . &         . & &      0.00 &      0.00 &      3974 \\
Age &         . &         . & & \textbf{        .} &         . & &         . &         . & &         . &         . & &         . &         . & &      0.26 &      0.26 &      3974 \\
CAPI & \textbf{        .} &         . & &         . &         . & & \textbf{        .} &         . & & \textbf{        .} &         . & & \textbf{        .} &         . & &      0.01 &      0.01 &      3974 \\
Mother: born in province & \textbf{        .} &         . & &         . &         . & & \textbf{        .} &         . & &         . &         . & & \textbf{        .} &         . & &      0.01 &      0.01 &      3974 \\
Mother max. edu.: less than middle school &         . &         . & &         . &         . & &         . &         . & &         . &         . & &         . &         . & &      0.02 &      0.02 &      3974 \\
Mother max. edu.: middle school &         . &         . & &         . &         . & &         . &         . & &         . &         . & &         . &         . & &      0.01 &      0.01 &      3974 \\
Mother max. edu.: high school &         . &         . & &         . &         . & &         . &         . & &         . &         . & &         . &         . & &      0.00 &      0.00 &      3974 \\
Mother max. edu.: university &         . &         . & &         . &         . & & \textbf{        .} &         . & &         . &         . & & \textbf{        .} &         . & &      0.00 &      0.00 &      3974 \\
Father: born in province &         . &         . & &         . &         . & &         . &         . & &         . &         . & &         . &         . & &      0.02 &      0.02 &      3974 \\
Father max. edu.: less than middle school &         . &         . & &         . &         . & &         . &         . & &         . &         . & &         . &         . & &      0.01 &      0.01 &      3974 \\
Father max. edu.: middle school &         . &         . & &         . &         . & & \textbf{        .} &         . & & \textbf{        .} &         . & & \textbf{        .} &         . & &      0.01 &      0.01 &      3974 \\
Father max. edu.: high school &         . &         . & &         . &         . & &         . &         . & &         . &         . & &         . &         . & &      0.00 &      0.00 &      3974 \\
Father max. edu.: university &         . &         . & &         . &         . & & \textbf{        .} &         . & &         . &         . & & \textbf{        .} &         . & &      0.01 &      0.01 &      3974 \\
Religious caregiver indicator & \textbf{        .} &         . & &         . &         . & & \textbf{        .} &         . & & \textbf{        .} &         . & & \textbf{        .} &         . & &      0.01 &      0.01 &      3974 \\
\bottomrule
\end{tabular}

}
\end{table}

\subsubsection{Linear Probability Model}
\begin{table}[H]
\caption{LPM Estimation Reggio - Adults (Age 50), Materna}
\centering
\scalebox{0.7}{
\begin{tabular}{lccccc}
\toprule
 & \textbf{None} & \textbf{Municipal} & \textbf{Religious} & \textbf{Private} & \textbf{State} \\
\midrule
\textbf{Respondent's Baseline Info} \\
\quad Male &      0.06 &      0.02 & \textbf{    -0.09} &     -0.00 &      0.01 \\
\quad  & (     0.06 ) & (     0.03 )  & \textbf{(     0.05 )}  & (     0.01 ) & (     0.03 ) \\
\quad CAPI &     -0.08 &      0.04 &      0.02 &      0.01 &     -0.00 \\
\quad  & (     0.06 ) & (     0.03 )  & (     0.05 )  & (     0.02 ) & (     0.03 ) \\
\midrule
\textbf{Mother's Baseline Info} \\
\quad Max Education: Middle School & \textbf{    -0.65} &      0.04 &      0.28 &     -0.03 & \textbf{     0.37} \\
\quad  & \textbf{(     0.33 )} & (     0.17 )  & (     0.29 )  & (     0.08 ) & \textbf{(     0.17 )} \\
\quad Max Education: High School & \textbf{    -0.60} &      0.11 &      0.18 &      0.00 & \textbf{     0.31} \\
\quad  & \textbf{(     0.34 )} & (     0.18 )  & (     0.30 )  & (     0.09 ) & \textbf{(     0.18 )} \\
\quad Max Education: University &     -0.53 &      0.11 &      0.09 &     -0.00 & \textbf{     0.33} \\
\quad  & (     0.35 ) & (     0.18 )  & (     0.30 )  & (     0.09 ) & \textbf{(     0.18 )} \\
\quad Born in Province & \textbf{    -0.14} &      0.03 &      0.07 &      0.02 &      0.02 \\
\quad  & \textbf{(     0.08 )} & (     0.04 )  & (     0.07 )  & (     0.02 ) & (     0.04 ) \\
\midrule
\textbf{Father's Baseline Info} \\
\quad Max Education: Middle School &      0.37 &      0.06 &      0.13 &      0.00 & \textbf{    -0.57} \\
\quad  & (     0.33 ) & (     0.17 )  & (     0.29 )  & (     0.08 ) & \textbf{(     0.17 )} \\
\quad Max Education: High School & \textbf{     0.57} &     -0.05 &      0.03 &      0.03 & \textbf{    -0.60} \\
\quad  & \textbf{(     0.34 )} & (     0.18 )  & (     0.30 )  & (     0.09 ) & \textbf{(     0.18 )} \\
\quad Max Education: University & \textbf{     0.63} &     -0.07 &      0.04 &      0.01 & \textbf{    -0.61} \\
\quad  & \textbf{(     0.35 )} & (     0.18 )  & (     0.30 )  & (     0.09 ) & \textbf{(     0.18 )} \\
\quad Born in Province &     -0.06 &      0.03 &     -0.00 &      0.02 &      0.02 \\
\quad  & (     0.08 ) & (     0.04 )  & (     0.07 )  & (     0.02 ) & (     0.04 ) \\
\midrule
\textbf{Household Baseline Info} \\
\quad Caregiver Has Religion &      0.05 & \textbf{    -0.07} & \textbf{     0.12} & \textbf{    -0.05} &     -0.06 \\
\quad  & (     0.07 ) & \textbf{(     0.04 )}  & \textbf{(     0.06 )}  & \textbf{(     0.02 )} & (     0.04 ) \\
\midrule
Observations & 196 & 196 & 196 & 196 & 196 \\
Fraction Attending Each Type &      0.75 &      0.05 &      0.14 &      0.01 &      0.05 \\
\midrule
$ R^2$ &      0.21 &      0.10 &      0.09 &      0.08 &      0.15 \\
\bottomrule
\end{tabular}}
\end{table}
\begin{scriptsize}
\noindent\underline{Note:} This table presents the linear probability model estimations for attending each type of Materna schools, indicated by each column. The samples used in this estimation are those who were adults in their 50's at the time of the survey living in Reggio. All dependent variables are binary. Observation indicates the number of people included in this sample. Bold number indicates that the p-value is less than or equal to 0.1. Standard errors are reported in parentheses.
\end{scriptsize}

\begin{table}[H]
\caption{LPM Estimation Parma - Adults (Age 50), Materna}
\centering
\scalebox{0.7}{
\begin{tabular}{lccccc}
\toprule
 & \textbf{None} & \textbf{Municipal} & \textbf{Religious} & \textbf{Private} & \textbf{State} \\
\midrule
\textbf{Respondent's Baseline Info} \\
\quad Male &      0.05 & \textbf{    -0.12} &      0.10 &      0.00 &     -0.03 \\
\quad  & (     0.09 ) & \textbf{(     0.06 )}  & (     0.07 )  & (        . ) & (     0.05 ) \\
\quad CAPI & \textbf{     0.22} & \textbf{    -0.15} &     -0.03 &      0.00 &     -0.04 \\
\quad  & \textbf{(     0.09 )} & \textbf{(     0.07 )}  & (     0.07 )  & (        . ) & (     0.06 ) \\
\midrule
\textbf{Mother's Baseline Info} \\
\quad Max Education: Middle School &      0.19 & \textbf{    -0.41} &      0.14 &      0.00 &      0.08 \\
\quad  & (     0.30 ) & \textbf{(     0.22 )}  & (     0.23 )  & (        . ) & (     0.19 ) \\
\quad Max Education: High School &      0.38 & \textbf{    -0.57} &      0.14 &      0.00 &      0.04 \\
\quad  & (     0.31 ) & \textbf{(     0.23 )}  & (     0.24 )  & (        . ) & (     0.20 ) \\
\quad Max Education: University &      0.54 & \textbf{    -0.54} &      0.00 &      0.00 &      0.01 \\
\quad  & (     0.34 ) & \textbf{(     0.25 )}  & (     0.26 )  & (        . ) & (     0.22 ) \\
\quad Born in Province &     -0.13 &      0.00 &      0.09 &      0.00 &      0.04 \\
\quad  & (     0.11 ) & (     0.08 )  & (     0.09 )  & (        . ) & (     0.07 ) \\
\midrule
\textbf{Father's Baseline Info} \\
\quad Max Education: Middle School & \textbf{    -0.49} & \textbf{     0.41} &      0.03 &      0.00 &      0.06 \\
\quad  & \textbf{(     0.26 )} & \textbf{(     0.19 )}  & (     0.20 )  & (        . ) & (     0.16 ) \\
\quad Max Education: High School &     -0.42 & \textbf{     0.32} &      0.07 &      0.00 &      0.03 \\
\quad  & (     0.27 ) & \textbf{(     0.19 )}  & (     0.20 )  & (        . ) & (     0.17 ) \\
\quad Max Education: University & \textbf{    -0.69} & \textbf{     0.40} &      0.27 &      0.00 &      0.02 \\
\quad  & \textbf{(     0.30 )} & \textbf{(     0.22 )}  & (     0.23 )  & (        . ) & (     0.19 ) \\
\quad Born in Province & \textbf{    -0.25} &      0.09 &      0.08 &      0.00 &      0.08 \\
\quad  & \textbf{(     0.09 )} & (     0.07 )  & (     0.07 )  & (        . ) & (     0.06 ) \\
\midrule
\textbf{Household Baseline Info} \\
\quad Caregiver Has Religion &     -0.01 &     -0.00 &      0.01 &      0.00 &      0.01 \\
\quad  & (     0.09 ) & (     0.07 )  & (     0.07 )  & (        . ) & (     0.06 ) \\
\midrule
Observations & 102 & 102 & 102 & 102 & 102 \\
Fraction Attending Each Type &      0.71 &      0.12 &      0.11 &      0.00 &      0.07 \\
\midrule
$ R^2$ &      0.28 &      0.24 &      0.10 &         . &      0.08 \\
\bottomrule
\end{tabular}}
\end{table}
\begin{scriptsize}
\noindent\underline{Note:} This table presents the linear probability model estimations for attending each type of Materna schools, indicated by each column. The samples used in this estimation are those who were adults in their 50's at the time of the survey living in Parma. All dependent variables are binary. Observation indicates the number of people included in this sample. Bold number indicates that the p-value is less than or equal to 0.1. Standard errors are reported in parentheses.
\end{scriptsize}

\begin{table}[H]
\caption{LPM Estimation Padova - Adults (Age 50), Materna}
\centering
\scalebox{0.7}{
\begin{tabular}{lccccc}
\toprule
 & \textbf{None} & \textbf{Municipal} & \textbf{Religious} & \textbf{Private} & \textbf{State} \\
\midrule
\textbf{Respondent's Baseline Info} \\
\quad Male &      0.06 &      0.03 &     -0.09 & \textbf{    -0.04} &      0.03 \\
\quad  & (     0.08 ) & (     0.05 )  & (     0.09 )  & \textbf{(     0.02 )} & (     0.02 ) \\
\quad CAPI &      0.10 &      0.03 &     -0.12 &     -0.01 &      0.01 \\
\quad  & (     0.10 ) & (     0.06 )  & (     0.10 )  & (     0.02 ) & (     0.02 ) \\
\midrule
\textbf{Mother's Baseline Info} \\
\quad Max Education: Middle School &     -0.32 &      0.02 &     -0.01 & \textbf{     0.28} &      0.02 \\
\quad  & (     0.33 ) & (     0.19 )  & (     0.34 )  & \textbf{(     0.07 )} & (     0.08 ) \\
\quad Max Education: High School &     -0.56 &      0.08 &      0.14 & \textbf{     0.32} &      0.02 \\
\quad  & (     0.36 ) & (     0.21 )  & (     0.37 )  & \textbf{(     0.08 )} & (     0.09 ) \\
\quad Max Education: University &     -0.25 &      0.11 &     -0.18 & \textbf{     0.29} &      0.03 \\
\quad  & (     0.37 ) & (     0.21 )  & (     0.38 )  & \textbf{(     0.08 )} & (     0.09 ) \\
\quad Born in Province & \textbf{     0.23} &     -0.06 & \textbf{    -0.20} &      0.01 &      0.02 \\
\quad  & \textbf{(     0.10 )} & (     0.06 )  & \textbf{(     0.10 )}  & (     0.02 ) & (     0.03 ) \\
\midrule
\textbf{Father's Baseline Info} \\
\quad Max Education: Middle School &      0.02 &      0.06 &      0.26 & \textbf{    -0.36} &      0.02 \\
\quad  & (     0.30 ) & (     0.17 )  & (     0.31 )  & \textbf{(     0.07 )} & (     0.07 ) \\
\quad Max Education: High School &      0.13 &      0.02 &      0.19 & \textbf{    -0.34} &     -0.00 \\
\quad  & (     0.32 ) & (     0.18 )  & (     0.33 )  & \textbf{(     0.07 )} & (     0.08 ) \\
\quad Max Education: University &      0.21 &     -0.04 &      0.22 & \textbf{    -0.39} &      0.00 \\
\quad  & (     0.32 ) & (     0.18 )  & (     0.33 )  & \textbf{(     0.07 )} & (     0.08 ) \\
\quad Born in Province &      0.07 &     -0.04 &     -0.05 &      0.02 &      0.01 \\
\quad  & (     0.11 ) & (     0.06 )  & (     0.12 )  & (     0.03 ) & (     0.03 ) \\
\midrule
\textbf{Household Baseline Info} \\
\quad Caregiver Has Religion &     -0.06 &      0.02 &      0.07 &     -0.00 &     -0.03 \\
\quad  & (     0.10 ) & (     0.06 )  & (     0.11 )  & (     0.02 ) & (     0.03 ) \\
\midrule
Observations & 140 & 140 & 140 & 140 & 140 \\
Fraction Attending Each Type &      0.41 &      0.08 &      0.49 &      0.01 &      0.01 \\
\midrule
$ R^2$ &      0.11 &      0.04 &      0.09 &      0.23 &      0.04 \\
\bottomrule
\end{tabular}}
\end{table}
\begin{scriptsize}
\noindent\underline{Note:} This table presents the linear probability model estimations for attending each type of Materna schools, indicated by each column. The samples used in this estimation are those who were adults in their 50's at the time of the survey living in Padova. All dependent variables are binary. Observation indicates the number of people included in this sample. Bold number indicates that the p-value is less than or equal to 0.1. Standard errors are reported in parentheses.
\end{scriptsize}




%\subsection{Migrants}
%\subsubsection{Unconditional Mean}

%\subsubsection{Linear Probability Model}
%\input{../Output/LPM/lpm_ReggioMigrants_m.tex}
%\input{../Output/LPM/lpm_ParmaMigrants_m.tex}
%\input{../Output/LPM/lpm_PadovaMigrants_m.tex}


\end{document}