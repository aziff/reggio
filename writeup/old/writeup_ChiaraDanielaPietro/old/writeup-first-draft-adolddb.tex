%2multibyte Version: 5.50.0.2960 CodePage: 1252

\documentclass[12pt]{article}
%%%%%%%%%%%%%%%%%%%%%%%%%%%%%%%%%%%%%%%%%%%%%%%%%%%%%%%%%%%%%%%%%%%%%%%%%%%%%%%%%%%%%%%%%%%%%%%%%%%%%%%%%%%%%%%%%%%%%%%%%%%%%%%%%%%%%%%%%%%%%%%%%%%%%%%%%%%%%%%%%%%%%%%%%%%%%%%%%%%%%%%%%%%%%%%%%%%%%%%%%%%%%%%%%%%%%%%%%%%%%%%%%%%%%%%%%%%%%%%%%%%%%%%%%%%%
\usepackage{eurosym}
\usepackage[top=1in, bottom=1in, left=1in, right=1in]{geometry}
\usepackage{adjustbox}
\usepackage{amsmath}
\usepackage{amssymb}
\usepackage{array}
\usepackage{booktabs}
\usepackage{fancyhdr}
\usepackage{float}
\usepackage{graphicx}
\usepackage[colorlinks=true,linkcolor=blue,urlcolor=blue,anchorcolor=blue,citecolor=blue]{hyperref}
\usepackage{lscape}
\usepackage{multirow}
\usepackage{natbib}
\usepackage{setspace}
\usepackage{tabularx}
\usepackage[colorinlistoftodos,linecolor=black]{todonotes}
\usepackage{appendix}
\usepackage{pgffor}

\setcounter{MaxMatrixCols}{10}
%TCIDATA{OutputFilter=LATEX.DLL}
%TCIDATA{Version=5.50.0.2960}
%TCIDATA{Codepage=1252}
%TCIDATA{<META NAME="SaveForMode" CONTENT="1">}
%TCIDATA{BibliographyScheme=BibTeX}
%TCIDATA{LastRevised=Sunday, July 31, 2016 12:41:19}
%TCIDATA{<META NAME="GraphicsSave" CONTENT="32">}

\parindent 22pt
\newcommand{\indep}{\rotatebox[origin=c]{90}{$\models$}}
\newcolumntype{L}[1]{>{\raggedright\arraybackslash}p{#1}}
\newcolumntype{C}[1]{>{\centering\arraybackslash}p{#1}}
\newcolumntype{R}[1]{>{\raggedleft\arraybackslash}p{#1}}
\input{tcilatex}
\begin{document}

\title{Analysis of the Reggio Approach}
\author{Pietro Biroli, Daniela Del Boca, and Chiara Pronzato }
\date[VERY PRELIMINARY DRAFT]{ \\
Original version: January 22, 2016 \\
Current version: \today }
\maketitle

\bigskip

%\listoftodos
\doublespacing

%\section{Summary}
This document describes the evaluation of the Reggio Children Approach to Early Childhood Education (for convenience, referred to as RA). This is a unique natural experiment which has been in place for fifty years in the city of
Reggio Emilia, Italy. Here a universal high-quality early child care system has developed a different
vision of the child – as an individual with rights and potential. The Reggio Children Approach has received
world-wide recognition and has been emulated in different countries and in a variety of settings,\footnote{The official \href{http://www.reggiochildren.it/network/?lang=en}{Reggio Children International Network} is present in 33 countries worldwide. Many other preschools around the world are ``inspired'' by the Reggio Children Approach but they are not officially part of these network.} but it has never been evaluated. 

The RA preschool system was introduced and grew non-randomly over the course of several decades. Therefore, the evaluation strategy has to deal with the lack of a well-defined control group, variation of treatment over time, and potential spillover effects. 

This round of analysis evaluates RA tackling these challenges. We consider several groupings of controls to account for differences by region, data collection, socioeconomic factors, and most importantly type of preschool chosen by the caregiver. %Observable parental characteristics, such as family religiosity and proximity to grandparents, are used to understand and control for selection into different types of early education. 
Different model specification allow for comparison with various control groups, allowing for a more nuanced understanding of the effects of RA, including the effects both within and between regions and in relation to other types of early education.

In general, we find \ldots \todo[backgroundcolor=orange!30,size=\tiny]{Summary of results}

Section \ref{sec:background} gives a description of RA and the cities involved with the data collection. More detail about this data collection, including sample structure and survey design, is given in Section \ref{sec:data}. The analysis is discussed in Section \ref{sec:analysis} with results presented in Section \ref{sec:OLS}.

\section{Summary (Introduction)}

This paper aims to evaluate the Reggio Approach preschool system. The Reggio
Approach includes infant-toddler centers (ages 0-2) and preschools (ages
3-5), both of which are for children before they enter primary school at age
6. The first Reggio infant-toddler center was established in 1971, while the
first Reggio preschool was established even earlier, in 1963.

The Reggio Approach (RA)\ is a unique natural experiment which has been in
place for fifty years in the city of Reggio Emilia, Italy. Here a universal
high-quality early child care system has developed a more progressive vision
of the child \ as an individual with rights and potential (Malaguzzi \
1993). The Reggio Approach has received world-wide recognition and was
defined as an exemplary model of early childhood education by Newsweek in
1971. It has received over the years several prizes and has been emulated in
different countries,\footnote{%
The official \href{http://www.reggiochildren.it/network/?lang=en}{Reggio
Children International Network} is present in 33 countries worldwide. Many
other preschools around the world are \textquotedblleft
inspired\textquotedblright\ by the Reggio Approach but they are not
officially part of these network.} but it has never been evaluated.

In order to evaluate the Reggio Approach, we have collected data on five age
cohorts, three cohorts of adults, one cohort of adolescents, and one cohort
of children in their first year of elementary school. The data were
collected in Reggio Emilia, and Padua and Parma. Parma and Padua are similar
to Reggio Emilia in several dimensions (size, geographic, demographic and
socio-economic structure and fertility dynamics), but do not provide the
same type of early childhood education.The schools that follow the Reggio
Approach are the municipal infant-toddler centers and preschools in Reggio
Emilia. All of the children who attended a municipal infant-toddler center
or preschool in Reggio Emilia are considered part of the treated group,
since they received the RA intervention. The other types of schools in
Reggio Emilia, and all the schools in Parma and Padova (including the
municipal schools) did not receive the RA intervention.

The evaluation of the Reggio Children Approach presents several challenges,
given the non-experimental nature of the program. The Reggio Approach\
preschool system was introduced and grew non randomly over the course of
several decades (QUESTO\ L HO\ \ LASCIATO\ MA\ NON\ HO\ CAPITO). Therefore
the evaluation strategy has to deal with lack of well defined control group,
variation of treatment overtime (?) and potential spillover effects. This
analysis evaluates RA aiming to adressing these issues.We consider several
groupings of controls to account for differences by region, data collection,
socioeconomic factors, and most importantly type of preschool chosen by the
families. Different model specifications allow for comparison with various
control groups, allowing for a more nuanced understanding of the effects of
RA, including the effects both within and between regions and in relation to
other types of early education.

In the next section, we review some of the previous work on the impact of
child care quality and curriculum on child outcomes. A brief description of
the RA is provided in Section 3 while a comparison between Reggio Emilia and
the other two cities involved with the data collection is provided in
Section \ref{sec:background}. More detail about this data collection,
including sample structure and survey design, is given in Section \ref%
{sec:data}. A simple model of school selection is sketched in Section \ref%
{sec:model}. The identification strategy and the empirical analysis are
discussed in Section \ref{sec:analysis}.

%\section{Background}
\label{sec:background}

% Quick description of Reggio Children Approach history
This section briefly describes the program and the cities in which the data were collected. For a more complete discussion of the Italian early childhood landscape, the educational philosophy and the research design, see \citet{biroli2015evaluating}. The Reggio Children Approach to early childhood education is a community effort of public investment in high-quality early childhood education spearheaded and led by the pedagogist Loris Malaguzzi (1920-1994), whose foresight has been the main influence on the approach. The system evolved slowly over the years from a parent cooperative in the outskirts of Reggio Emilia into a structured municipal system of 26 infant-toddler centers and 30 preschools. Building on existing pedagogical models, this educational project is centered around the interaction between children, families, teachers, and the community (\cite{Malaguzzi1993}).

The Reggio Children Approach includes infant-toddler centers (ages 0-2) and preschools (ages 3-5), both of which are for children before they enter primary school at age 6. The first Reggio infant-toddler center was established in 1971, while the first Reggio preschool was established even earlier, in 1963. 

% Reggio Emilia vs. Parma vs. Padova
In addition to collecting data from individuals in Reggio Emilia, data were collected from individuals in Parma and Padova, two cities that share several features with Reggio Emilia. Table (\ref{tab:comparison}) lists some characteristics of the three cities. More information on the data collection is given in Section (\ref{sec:data}).

\begin{table}[htbp]
\begin{center}
\caption{Demographic Comparison of Reggio Emilia, Parma, and Padova}
\label{tab:comparison}
\begin{tabular}{lccc}
\toprule
& \multicolumn{3}{c}{City} \\
\cmidrule{2-4}
& Reggio Emilia & Parma & Padova \\
\midrule
Population (2013)* & 172,525 &  187,938 & 209,678 \\
Average per-capita income (2011 euros)**  & 25,226 & 28,437 & 29,915 \\
\bottomrule
\end{tabular}
\end{center}
\footnotesize Notes: *ISTAT, \url{http://www.demo.istat.it/}; **Finance Minister, taxable income for 2011.
\end{table}
\todo[backgroundcolor=orange!30,size=\tiny]{Add information about (updated) italian and immigrant births to table \ref{tab:comparison}. Look for values in 1960 (start of the program, baseline)}

% Preschool availability and take-up

% Different types of preschool
As in every Italian city, early educational experiences are divided into two main age categories. The first, infant-toddler centers (\textit{asilo}), is available for children aged 0 through 3 while the second, preschool (\textit{materna}), is for children aged 3 through 6. 

For each of these age groups, there are schools established and managed by different entities: municipal, state, private, and religious. Table \ref{tab:types} summarizes the types of schools available for each age group, highlighting that there are no state-run infant-toddler centers.

\begin{table}[htbp]
\begin{center}
\caption{Types of Schools}\label{tab:types}
\begin{tabular}{ccc}
\toprule
& Infant-Toddler & Preschool \\
\midrule
Municipal & \checkmark & \checkmark \\
State & & \checkmark \\
Private & \checkmark & \checkmark \\
Religious & \checkmark & \checkmark \\
\bottomrule
\end{tabular}
\end{center}
\end{table}

The schools that follow the Reggio Children Approach are the \textbf{municipal} infant-toddler centers and preschools in Reggio Emilia. All of the children who attended a municipal infant-toddler center or preschool in Reggio Emilia are considered part of the treated group, since they received the RA intervention. The other types of schools in Reggio Emilia, and all the schools in Parma and Padova (including the municipal schools) did not receive the RA intervention. However, we cannot rule out that also these other school types borrowed aspects from the RA philosophy of early childhood education, especially in the city of Reggio Emilia. A detailed interview with the managers of the religious preschools in Reggio Emilia showed that [...] \todo[backgroundcolor=orange!30,size=\tiny]{Together with Claudia and Moira, who run the RA schools, we are collecting more specific information available about the religious schools in Reggio. No information is available for state and private}
In Padova, influences from Guerra Frabboni's philosophy prevail \citep{Frabboni1999}, while in Parma, there is a mixture of influences.


\section{\protect\bigskip Literature}

In the last few years economic \textbf{r}esearch has started analyzing the
impact of child care on child outcomes focusing not only on its availability
but also its quality and pedagogical philosophies. Felfe and Lalive (2014)
analyzed German data, using information on the quality of available
childcare for children under the age of three. They show that quality is a
very important determinant of several child outcomes at the age of 6,
particularly of socio-emotional maturity and school readiness. Specifically,
teachers' age and education, working hours and group size have a positive
impact on child outcomes. The effects are stronger for the children of less
educated mothers.

Love et al. (2003) using data from three studies in the different contexts
of the U.S., Australia and Israel (Early Head Start; the Sydney Family
Development Project; Haifa-NICHD \ merged data) compare a variety of
childcare centers differing in levels of regulation and of staff quality,
and provide consistent evidence that the quality of child care has an
important impact on child development. In particular, children from
low-income families experience a stronger positive impact of childcare
quality on a both cognitive and non-cognitive outcomes.

Duncan et al. (2012) explore the effects on later outcomes of high-versus
low-quality childcare both during infant--toddlerhood and during the
preschool years, using data from the National Institute of Child Health and
Human Development Study of Early Child Care. They show that quality matters
as well as the timing of attendance high quality education. Cognitive,
language and other skills were highest in children who experienced
high-quality care in both the infant--toddler and preschool periods,
somewhat lower in children who experienced high-quality child care during
only one of these periods, and lowest among children who experienced
low-quality care during both periods

Other studies have linked quality with the influence of certain program
curricula and pedagogical philosophies on the teaching strategies employed
in classrooms, roughly divided into the two main categories of
\textquotedblleft child centered\textquotedblright\ and \textquotedblleft
academic\textquotedblright\ approaches. The Reggio approach is a clear
example of a child centered approach, where the teacher is encouraged to
become the child's co-learner and collaborator. According to this approach,
teachers do not impose a specific curriculum, but facilitate the child's
learning by planning activities based on the child's interests, and engaging
in the activities alongside the child (Malaguzzi, 1993, Hewett, 2001). In an
academic approach, instead, the focus is on acquiring notions related to
different subject areas.

Using data from the Pre-K Survey of Beliefs and Practices, Marcon (1999)
identifies three different preschool models operating in an urban school
district: child centered, academic driven and a combination of the two. His
empirical evidence shows that by the end of preschool, children from
child-centered programs have acquired greater competence in social, basic
math and basic verbal skills than their peers in academically-driven
preschool environments.

Other studies have compared the effectiveness of several pedagogical models.
Miller and Bizzell, (1984) analyzes the achievement test and IQ data on
low-income black youths who had participated in traditional and more
progressive programs in pre and prekindergarten through ninth and tenth
grades. They find long-term positive effects of child-centered learning on
school achievement. Schweinhart et al., (1993) have evaluated long term
effects (through age 23) of the High/Scope program vs more traditional
preschool curriculum models. Their findings show that using a curriculum
model based on child-initiated learning activities improves social behavior
at a later age. Specifically they found at age 23, compared to the other
curriculum groups, the group who has experienced a more traditional
curriculum had three times as many felony arrests per person.

The traditional sequential and subject specific approaches are less
effective in promoting children's learning in the early years whereas a
holistic approach that sustains children's overall development across
several domains is more effective as it is supportive of children's learning
strategies (Bennett, Gordon, and Edelmann, 2013). As stated by Bennett
(2013): `The cognitive development of young children does not match a
traditional subject approach. Rather, it is focused on meaning-making --
his/her place in the family; the roles and work of significant adults;
forging a personal identity; how to communicate needs and desires; how to
interact successfully and make friends; how things work; the change of the
season and other remarkable events in the child's environment.

While RA\ has certainly a leadership in child-centered approach and a long
lasting pedagogical experience, these principles are increasingly becoming
part of \ large number of child care experiences ind different contexts in
Italy as well as in other countries (Lazzari et al 2012). 

%\section{Background}
\label{sec:background}

% Quick description of Reggio Children Approach history
This section briefly describes the program and the cities in which the data were collected. For a more complete discussion of the Italian early childhood landscape, the educational philosophy and the research design, see \citet{biroli2015evaluating}. The Reggio Children Approach to early childhood education is a community effort of public investment in high-quality early childhood education spearheaded and led by the pedagogist Loris Malaguzzi (1920-1994), whose foresight has been the main influence on the approach. The system evolved slowly over the years from a parent cooperative in the outskirts of Reggio Emilia into a structured municipal system of 26 infant-toddler centers and 30 preschools. Building on existing pedagogical models, this educational project is centered around the interaction between children, families, teachers, and the community (\cite{Malaguzzi1993}).

The Reggio Children Approach includes infant-toddler centers (ages 0-2) and preschools (ages 3-5), both of which are for children before they enter primary school at age 6. The first Reggio infant-toddler center was established in 1971, while the first Reggio preschool was established even earlier, in 1963. 

% Reggio Emilia vs. Parma vs. Padova
In addition to collecting data from individuals in Reggio Emilia, data were collected from individuals in Parma and Padova, two cities that share several features with Reggio Emilia. Table (\ref{tab:comparison}) lists some characteristics of the three cities. More information on the data collection is given in Section (\ref{sec:data}).

\begin{table}[htbp]
\begin{center}
\caption{Demographic Comparison of Reggio Emilia, Parma, and Padova}
\label{tab:comparison}
\begin{tabular}{lccc}
\toprule
& \multicolumn{3}{c}{City} \\
\cmidrule{2-4}
& Reggio Emilia & Parma & Padova \\
\midrule
Population (2013)* & 172,525 &  187,938 & 209,678 \\
Average per-capita income (2011 euros)**  & 25,226 & 28,437 & 29,915 \\
\bottomrule
\end{tabular}
\end{center}
\footnotesize Notes: *ISTAT, \url{http://www.demo.istat.it/}; **Finance Minister, taxable income for 2011.
\end{table}
\todo[backgroundcolor=orange!30,size=\tiny]{Add information about (updated) italian and immigrant births to table \ref{tab:comparison}. Look for values in 1960 (start of the program, baseline)}

% Preschool availability and take-up

% Different types of preschool
As in every Italian city, early educational experiences are divided into two main age categories. The first, infant-toddler centers (\textit{asilo}), is available for children aged 0 through 3 while the second, preschool (\textit{materna}), is for children aged 3 through 6. 

For each of these age groups, there are schools established and managed by different entities: municipal, state, private, and religious. Table \ref{tab:types} summarizes the types of schools available for each age group, highlighting that there are no state-run infant-toddler centers.

\begin{table}[htbp]
\begin{center}
\caption{Types of Schools}\label{tab:types}
\begin{tabular}{ccc}
\toprule
& Infant-Toddler & Preschool \\
\midrule
Municipal & \checkmark & \checkmark \\
State & & \checkmark \\
Private & \checkmark & \checkmark \\
Religious & \checkmark & \checkmark \\
\bottomrule
\end{tabular}
\end{center}
\end{table}

The schools that follow the Reggio Children Approach are the \textbf{municipal} infant-toddler centers and preschools in Reggio Emilia. All of the children who attended a municipal infant-toddler center or preschool in Reggio Emilia are considered part of the treated group, since they received the RA intervention. The other types of schools in Reggio Emilia, and all the schools in Parma and Padova (including the municipal schools) did not receive the RA intervention. However, we cannot rule out that also these other school types borrowed aspects from the RA philosophy of early childhood education, especially in the city of Reggio Emilia. A detailed interview with the managers of the religious preschools in Reggio Emilia showed that [...] \todo[backgroundcolor=orange!30,size=\tiny]{Together with Claudia and Moira, who run the RA schools, we are collecting more specific information available about the religious schools in Reggio. No information is available for state and private}
In Padova, influences from Guerra Frabboni's philosophy prevail \citep{Frabboni1999}, while in Parma, there is a mixture of influences.


\section{The Reggio Approach}

\label{sec:background}

% Quick description of Reggio Approach history
This section briefly describes the Reggio Approach (for a more complete
discussion of the Italian early childhood landscape, the educational
philosophy \citet{biroli2015evaluating})The Reggio Approach to early
childhood education is a community effort of public investment in
high-quality early childhood education spearheaded and led by the pedagogist
Loris Malaguzzi (1920-1994), whose foresight has been the main influence on
the approach. Since Malaguzzi became Pedagogical Consultant to Reggio Emilia
municipal scool in 1963, more and more progressive models of early childhood
education characterized the municipal schools in Reggio Emilia previously
dominated by Catholic traditional schools. The system evolved slowly over
the years from a parent cooperative in the outskirts of Reggio Emilia into a
structured municipal system of 26 infant-toddler centers and 30 preschools,
in alternative to the prevailing Catholic schools which have declined
rapidly after that.

Its education project built on existing pedagogical models combined with
many innovative features. As stated by Edwards, Gandini and Forman (1993)
this curriculum facilitates children'sconstruction their own powers of
thinking through all the expressive, communicative and cognitive languages.

The Reggio Children Approach views early childhood education as based on
relationships, and focuses on each child in relation with other children,
family, teachers, society and the environment. Its philosophy is based on
several principles (Rinaldi. 2005, Gandini, 1993)

First of all, children are seen as researchers: they develop theories and
adapt them, individually and with others, while they interpret and
reconstruct the world and their surroundings. These learning conditions are
favored by the creation of small working groups of children and adults
researching together .Collaborative group work is considered valuable and
necessary to advance cognitive development. Children are encouraged to
dialogue,, compare, negotiate, , and problem solve through group work.

Second, teachers are encouraged to facilitate the child's learning by
planning activities and lessons based on the child's interests, asking
questions to further understanding, and actively engaging in the activities
alongside the child, as partners to the child more than only an instructor
(Hewett, 2001). According to the principles of RA, each class is organized
with two teachers (often male and female), who are supported by a new
professional figure, the "atelierista", a teacher trained to foster the
children's expressive and creative languages.

Third, a continous documentation of children's work in progress is
constructed as an important tool for the learning process for children,
teachers, and parents. Pictures of children engaged in experiences, their
words as they discuss what they are doing, feeling and thinking, and the
children's interpretation of experience through the visual media are
displayed daily and used as assessment and memories. Other benefits of
documentation include: making visible children's learning process, sharing
their experiences with the parents and allowing teachers to record and
understand children learning dynamics.

A fourth important element concerns the teachers training. An ongoing and
continous job- training and collegial workplace for all educational figures
is established. \ and is an integral part of the educational experience.
training is achieved through internal weekly updates, classroom teaching and
meetings that become a tool for dialogue, discussion and growth among the
various professional figures at the school.

Another important aspect \ concerns the space of the educational experience.
Since the beginning, the aesthetic dimension was considered an integral part
of learning, from the architecture, furniture, to the materials used in the
school. The early municipal schools, opened in 1963 and 1964, were already
equipped with an internal kitchen, and its personnel was actively involved
in the educational project, an important and innovative choice for a public
service.The preschools are generally filled with indoor plants and vines. In
each room  the attention of both children and adults is captured through the
use of mirrors (on the walls, floors, and ceilings), photographs, and
children's work accompanied by transcriptions of their discussions. The
environment has been defined by Malaguzzi  the "third teacher" besides
children and teachers.

Finally, parents are a vital component to the Reggio philosophy. They are
viewed as partners, collaborators of their children's learning as they are
involved in every aspect of the curriculum. This philosophy does not end
when the child leaves the classroom. Teachers conduct frequent meetings with
parents to help educate them about children's social, emotional, creative
and academic development.The educational project of early childhood is
outcome of of the joint work of teachers, pedagogists and parents in
Consigli Infanzia Citt\`{a} which are elected every three years.

In the next sections we will try to analyze differences and similarities in
Reggio, Parma and Padova in order to understand and interpret better our
empirical results.

\section{Reggio, Parma and Padova}

% Reggio Emilia vs. Parma vs. Padova
In addition to collecting data from individuals in Reggio Emilia, data were
collected from individuals in Parma and Padova, two cities that share
several features with Reggio Emilia. More information on the data collection
is given in Section (\ref{sec:data}).

Table (\ref{tab:comparison}) lists some characteristics of the three cities.
While in the rest of the paper we will show several similarities between
Reggio Emilia, Padova and Parma concerning size, geographic, demographic and
socio-economic structure, in this part we discuss some of the differences
concerning the pedagogical approach and infant-toddler centres' and
preschools' organization (educators roles, type of training, parental
involvement etc) in the three educational systems.

First of all, the Reggio Emilia municipal infant-toddler centers and
preschools are certainly the oldest childcare system in Italy: it started in
1963 with the opening of the first municipal preschool and in 1971 with the
opening of the first municipal infant-toddler center, earlier than the 1971
National Law. Padova (1976-77) and Parma (1975) instead started after the
law, in the mid-late seventies..

More information on the data collection is given in Section (\ref{sec:data}).

\begin{table}[tbph]
\caption{Demographic Comparison of Reggio Emilia, Parma, and Padova}
\label{tab:comparison}
\begin{center}
\begin{tabular}{lccc}
\hline\hline
& \multicolumn{3}{c}{City} \\ 
\cmidrule{2-4} & Reggio Emilia & Parma & Padova \\ \hline
Population (2013)* & 172,525 & 187,938 & 209,678 \\ 
Average per-capita income (2011 euros)** & 25,226 & 28,437 & 29,915 \\ \hline
\end{tabular}%
\end{center}
\par
{\footnotesize Notes: *ISTAT, \url{http://www.demo.istat.it/}; **Finance
Minister, taxable income for 2011. }
\end{table}
%\todo[backgroundcolor=orange!30,size=\tiny]{Add information about (updated) italian and immigrant births to table \ref{tab:comparison}. Look for values in 1960 (start of the program, baseline)}

% Preschool availability and take-up

% Different types of preschool
As in every Italian city, early educational experiences are divided into two
main age categories. The first, infant-toddler centers (asilo nido), is
available for children aged 0 through 3 while the second, preschool (scuola
materna/dell'infanzia), is for children aged 3 through 6.

For each of these age groups, there are schools established and managed by
different entities: municipal, state, private, and religious. Table \ref%
{tab:types} summarizes the types of schools available for each age group,
highlighting that there are no state-run infant-toddler centers.

\begin{table}[tbph]
\caption{Types of Schools}
\label{tab:types}
\begin{center}
\begin{tabular}{ccc}
\hline\hline
& Infant-Toddler & Preschool \\ \hline
Municipal & \checkmark & \checkmark \\ 
State &  & \checkmark \\ 
Private & \checkmark & \checkmark \\ 
Religious & \checkmark & \checkmark \\ \hline
\end{tabular}%
\end{center}
\end{table}

The schools that follow the Reggio Approach are the municipal infant-toddler
centers and preschools in Reggio Emilia. All of the children who attended a
municipal infant-toddler center or preschool in Reggio Emilia are considered
part of the treated group, since they received the RA intervention. The
other types of schools in Reggio Emilia, and all the schools in Parma and
Padova (including the municipal schools) did not receive the RA
intervention. 
%A detailed interview with the managers of the religious preschools in Reggio Emilia showed that [...] \todo[backgroundcolor=orange!30,size=\tiny]{Together with Claudia and Moira, who run the RA schools, we are collecting more specific information available about the religious schools in Reggio. No information is available for state and private} 

\bigskip

In Italy, early childhood education is provided at the public level (State
and Municipal) and by private organizations (religious and secular). The
system is decentralized: the municipality is the main decision-maker, while
the regions define general management criteria;%
%TCIMACRO{%
%\TeXButton{Footnote}{\footnote{To date, in Italy there are 8,092 municipalities in 101 provinces and 20 regions.}} }%
%BeginExpansion
\footnote{To date, in Italy there are 8,092 municipalities in 101 provinces and 20 regions.}
%EndExpansion
The central government is only responsible for defining common objective
standards and resources allocation among regions. (Brilli et al 2016).

Child care is offered by the State, municipalities, religious institutions
and private companies Enacted in 1968, Law 444 provided for state-run early
childcare education and enabled municipalities to create their own
autonomous early childhood programming. In 1968, Law 444 assigned the costs
of building, equipment and playing materials to the State; municipalities
were to fund building maintenance, heating and operating costs including
salaries for an all-female staff of teachers under 35 years of age with a
high school diploma and In 1971, Italy enacted Law 1044 mandating regional
governments to construct and operate enough asili nido to meet the local
demand for childcare for children aged 3 to 36 months.( Brilli et al 2016)

Municipalities are \ also enabled to set eligibility criteria for whom
access to public childcare appears to be most valuable. Selection criteria
appear to be similar across munic- ipalities, however, the weighting of
distinct family characteristics varies. The Catholic Church offers the
majority of private religious early childhood programs. In the last years,
childcare supply from private providers has increased and developed
differently across Italian regions (Istituto Degli Innocenti, 2002 and
2009). Public childcare differs from private childcare in several ways. For
instance, public services are more strictly regulated both in terms of
service standards and in terms of management and personnel requirements
(Istituto Degli Innocenti, 2002). As recently stated in Budget Law 2002,%
%TCIMACRO{%
%\TeXButton{footnote}{\footnote{Law 448/2001 (Budget Law 2002) defined formal childcare as "structures aimed at granting the development and socialization of girls and boys aged between 3 months and 3 years and to support families and parents with young children".}} }%
%BeginExpansion
\footnote{Law 448/2001 (Budget Law 2002) defined formal childcare as "structures aimed at granting the development and socialization of girls and boys aged between 3 months and 3 years and to support families and parents with young children".}
%EndExpansion
one of the most important aim of public childcare is educational. This goal
has been implemented through the introduction of quality standards,
especially in regions with greater experience in childcare provision (such
as Emilia Romagna and Tuscany). Public childcare is also less expensive than
the private one, since it is highly subsidized for lower income families.

The Italian child care quality is relatively high vs other countries.
According to the European ranking italian public child care it is fourth
after Denmark, Finland and France (De Haneau et al 2008).

While high quality standards are common among public schools there is still
variation in the curriculum across municipalities and regions.

In this section we aim to compare Reggio Emilia and Parma and Padova. While
the first two are both provinces of Emilia Romagna, Padova is a province of
Veneto.In Padova, the pedagogical approach is influenced by Frabboni
(Frabboni and Guerra,1999) and is based on a more traditional cognitive
child approach, that is not child-centered as RA. Differently from RA,
teachers have a more traditional role as instructors who follow a sequential
and academic program. There is no teacher with specific art and creative
skills to stimulate and coordinate children's activities. Parents are
informed about the children activities and progress in specific meetings.

Teachers undergo frequent training but a specific times and in provided in
classroom with traditional lecture format. \ Also in Padova's muncipal
schools, teachers document teh activities and collect materials produced by
teachers and children. However this is done ex-post and used as a archive
for valuation of children's work and communication with the parents (Piano
dell Offerta Formativa of preschools and nidi in Padova 2016). Finally most
schools has their own kitchen but food are provided by an external company.
The presence of a majority of private/catholic schools have still strong
spillover on the philosophy of the muncipal and state child care. Padova is
one of the largest cities of Veneto, traditionally a very Catholic region,
which was for a long time the heartland of Christian Democracy.

Parma is instead the third largest city of Emilia Romagna, a region which
has been historically a stronghold of the Italian Communist Party, forming
the famous Italian "Red Quadrilateral" with Tuscany, Umbria and Marche. This
is due to the strength of the anti-fascist resistance around the time of
World War IIas well as a strong tradition of anti-clericalism dating from
the 19th century, when part of the region belonged to the Papal States.
However the political history of Parma has been different from Reggio Emilia
During the period 1950-1990 a moderate socialist party has been predominant
and  afterwards the center-right has controlled the city administration. In
Parma, the pedagogical approach is not based on a unique pedagogical
approach, but is influenced by a combination of different ones. Among others
Winnicott and Bion who have stressed the strenght of the link mother child
and imposed a very long and articulated and long adjustment period for the
child to be separated from his/her family in order \ to attend child care

According to the Piano dell' Offerta Formativa of preschools and nidi
(2016), children should have a central role in the learning process which
allows to achieving a dimension of knowlege not as a perfomance but as
possibility of supporting independent and original building of his/her
personality. In spite of some similarities in the role of children and
teachers in Parma there is no role for a teacher/coordinator with a creative
and artistic background (aterlierista).   The role of parents is strong and
integrated in the schools. Parents can intervene in the schools with ideas
proposals and experiences. 

Training are given for 45-50 hours a year at specific times. The training
consists in meetings with the pedagogical coordinator while formal training
is provided in classroom with traditional lecture format by university
professors .

As in Padova there is no art trained teacher entitled to stimulate and
encourage children's creativity and teachers'training is offered periodical
in a traditional setting.

Finally each school has their own kitchen but food are provided by an
external company.

Important differences characterize also the organization of child care (del
Boca et al 2016). The eligibility criteria in Padova is mainly geographical,
while in Parma priority is given to families in difficulty (health issues
and single parent families) as well as parental working hours, similarly to
Reggio \citep{Frabboni1999}

From these documents and informal interviews with coordinators of servizi
educativi of Parma and Padova and Reggio Emilia, it looks the curriculum in
Parma is a combination of Padova and Reggio Emilia.

% ------------------------------ Section DATA ------------------------------------------%
%\section{Data}
\label{sec:data}

This section discusses the survey data which has been used for the analysis in this report.\footnote{Also administrative data from the RA preschool system was collected. Its description and more details about the survey data are contained in \citet{biroli2015evaluating}} To evaluate the impact of Reggio, individuals living in Reggio Emilia, Parma, and Padova since their first year of life were interviewed. Data were collected on five age cohorts, including three cohorts of adults, one cohort of adolescents, and one cohort of children in their first year of elementary school. 

The structure of the cohorts is described in Table \ref{tab:cohorts}.

\begin{table}[htbp]
\begin{center}
\caption{Cohort Structure}\label{tab:cohorts}
\begin{tabular}{ccc}
\toprule
Cohort & Years of Birth & Age at Interview \\
\midrule
I & 1954--1959 & 54--60 \\
II & 1969--1970 & 43 \\
III & 1980--1981 & 32 \\
IV & 1994 & 18 \\
V & 2006 & 6 \\
\bottomrule  
\end{tabular}
\end{center}
\end{table}

\subsection{Outcomes considered}\label{sec:outcomes}
A rich set of outcomes were collected during the survey. Individuals at different stages of their lifecycle were asked about family composition, fertility, labor force participation, income, schooling, cognitive ability, social and emotional skills, health and healthy habits, social capital, interpersonal ties, as well as attitudes on immigration and integration. 

We focus our attention on four different outcomes, in four different domains: the mother reports to the Strength and Difficulties Questionnaire (SDQ), a widely used and validated scale from 0 to 40 measuring behavioral problems for children and adolescents; the percentage of respondents reporting to have good or excellent health;\footnote{For children and adolescent, the respondent is again the mother} the raw score of the Center for Epidemiological Studies Depression Scale (CESD), a scale ranging from 10 to 50 and measuring self-reported depression for adolescents and adults; and finally the share of respondents reporting to be `very' or  `quite' bothered by the immigration into the city. Tables (\ref{tab:outcomes-child-asilo}) to (\ref{tab:outcomes-adult-asilo}) report the summary statistics of these outcome variables, disaggregated by city and type of \textit{infant-toddler center} attended. Similarly, Tables (\ref{tab:outcomes-child-materna}) to (\ref{tab:outcomes-adult-materna}) report the average of these same variables, this time focusing on type of \textit{preschool} attended. % We see that


%% children, infant-toddler
\begin{table}[H] \label{tab:outcomes-child-asilo}
\caption{Outcomes of interest by infant-toddler-center type, children (age 6)}
% this is the top part of the tables that display the summary of the baseline characteristics, by city and child-care type
\centering
\begin{adjustbox}{width=1.2\textwidth,center=\textwidth}
\small
\begin{tabular}{m{4.0cm} cccccccccccc}
\hline \hline 
 & Reggio & Reggio & Reggio & Reggio & Parma & Parma & Parma & Parma & Padova & Padova & Padova & Padova \\
 & Municipal & Religious & Private & Not Attended & Municipal & Religious & Private & Not Attended & Municipal & Religious & Private & Not Attended \\

\hline 

SDQ score (mom rep.)   &   8.36   &   8.19   &   7.60   &   9.53**   &   7.93   &   8.29   &   7.11*   &   8.30   &   9.07   &   9.08   &   8.56   &   8.49 \\
   &   [0.32]   &   [0.79]   &   [1.78]   &   [0.43]   &   [0.33]   &   [1.44]   &   [0.52]   &   [0.43]   &   [0.54]   &   [0.98]   &   [0.75]   &   [0.33] \\
Child health is good (\%) - mom report   &   0.65   &   0.74   &   0.80   &   0.71   &   0.60   &   0.86   &   0.67   &   0.64   &   0.79**   &   0.77   &   0.71   &   0.73 \\
   &   [0.04]   &   [0.09]   &   [0.20]   &   [0.04]   &   [0.04]   &   [0.14]   &   [0.08]   &   [0.05]   &   [0.05]   &   [0.08]   &   [0.07]   &   [0.04] \\
%Child likes school (\%)   &   0.71   &   0.70   &   0.80   &   0.59**   &   0.71   &   0.71   &   0.78   &   0.74   &   0.46***   &   0.46**   &   0.59   &   0.54*** \\
%   &   [0.04]   &   [0.09]   &   [0.20]   &   [0.05]   &   [0.04]   &   [0.18]   &   [0.07]   &   [0.04]   &   [0.06]   &   [0.10]   &   [0.08]   &   [0.04] \\
%Child likes math (\%)   &   0.61   &   0.81*   &   1.00   &   0.62   &   0.68   &   0.86   &   0.69   &   0.58   &   0.68   &   0.62   &   0.65   &   0.62 \\
%   &   [0.04]   &   [0.08]   &   [0.00]   &   [0.05]   &   [0.04]   &   [0.14]   &   [0.08]   &   [0.05]   &   [0.06]   &   [0.10]   &   [0.08]   &   [0.04] \\
%Child likes reading/italian (\%)   &   0.59   &   0.59   &   0.80   &   0.49   &   0.67   &   0.86   &   0.78*   &   0.62   &   0.60   &   0.50   &   0.57   &   0.44*** \\
%   &   [0.04]   &   [0.10]   &   [0.20]   &   [0.05]   &   [0.04]   &   [0.14]   &   [0.07]   &   [0.05]   &   [0.06]   &   [0.10]   &   [0.08]   &   [0.04] \\
%Ability to sit still in a group when asked (difficulties in primary school)   &   0.12   &   0.07   &   0.00   &   0.19   &   0.15   &   0.14   &   0.06   &   0.14   &   0.19   &   0.15   &   0.10   &   0.06* \\
%   &   [0.03]   &   [0.05]   &   [0.00]   &   [0.04]   &   [0.03]   &   [0.14]   &   [0.04]   &   [0.04]   &   [0.05]   &   [0.07]   &   [0.05]   &   [0.02] \\
%Lack of excitement to learn (difficulties in primary school)   &   0.02   &   0.04   &   0.00   &   0.05   &   0.05   &   0.00   &   0.00   &   0.06*   &   0.07*   &   0.04   &   0.02   &   0.06 \\
%   &   [0.01]   &   [0.04]   &   [0.00]   &   [0.02]   &   [0.02]   &   [0.00]   &   [0.00]   &   [0.02]   &   [0.03]   &   [0.04]   &   [0.02]   &   [0.02] \\
%Ability to obey rules and directions (difficulties in primary school)   &   0.10   &   0.04   &   0.00   &   0.11   &   0.15   &   0.00   &   0.06   &   0.09   &   0.13   &   0.08   &   0.05   &   0.05 \\
%   &   [0.02]   &   [0.04]   &   [0.00]   &   [0.03]   &   [0.03]   &   [0.00]   &   [0.04]   &   [0.03]   &   [0.04]   &   [0.05]   &   [0.03]   &   [0.02] \\
%Fussy eater (difficulties in primary school)   &   0.11   &   0.19   &   0.17   &   0.03**   &   0.05*   &   0.14   &   0.03   &   0.18*   &   0.09   &   0.12   &   0.02   &   0.08 \\
%   &   [0.02]   &   [0.08]   &   [0.17]   &   [0.02]   &   [0.02]   &   [0.14]   &   [0.03]   &   [0.04]   &   [0.03]   &   [0.06]   &   [0.02]   &   [0.02] \\
% Difficulties encountered when starting primary school   &   4.29   &   4.44   &   4.83   &   3.92   &   4.17   &   4.43   &   4.64   &   3.98***   &   3.93   &   4.23   &   4.44   &   4.49* \\
%   &   [0.11]   &   [0.22]   &   [0.17]   &   [0.15]   &   [0.12]   &   [0.57]   &   [0.17]   &   [0.15]   &   [0.20]   &   [0.30]   &   [0.20]   &   [0.10] \\
\hline

% it contains the notes, assuming they are the same for all the tables.
\end{tabular}

\end{adjustbox}
\raggedright{
\footnotesize{Average of baseline characteristcs, by city and type of child-care attended. Standard errors of means in brackets. Test for difference in means between each column and the first column (Reggio Municipal, the treatment group) was performed; *** significant difference at 1\%, ** significant difference at 5\%, * significant difference at 10\%. Source: authors calculation using survey data.}
}
\end{table}
  
%\end{table}

%% adolescents, infant-toddler
\begin{table}[H] \label{tab:outcomes-adol-asilo}
\caption{Outcomes of interest by infant-toddler-center type, adolescents (age 18)}
% this is the top part of the tables that display the summary of the baseline characteristics, by city and child-care type
\centering
\begin{adjustbox}{width=1.2\textwidth,center=\textwidth}
\small
\begin{tabular}{m{4.0cm} cccccccccccc}
\hline \hline 
 & Reggio & Reggio & Reggio & Reggio & Parma & Parma & Parma & Parma & Padova & Padova & Padova & Padova \\
 & Municipal & Religious & Private & Not Attended & Municipal & Religious & Private & Not Attended & Municipal & Religious & Private & Not Attended \\

\hline 
  
%SDQ score (mom rep.)  &  7.95  &  8.89  &  9.67  &  8.47  &  8.02  &  9.70  &  9.38  &  8.01  &  8.15  &  6.38  & . &  8.32 \\
%  &  [0.36]  &  [2.64]  &  [2.33]  &  [0.37]  &  [0.42]  &  [2.39]  &  [1.45]  &  [0.38]  &  [0.57]  &  [0.91]  & .&  [0.26] \\
SDQ score &  9.96  &  11.89  &  7.67  &  10.04  &  9.11  &  9.10  &  9.46  &  8.95  &  10.52  &  7.75 & . &  9.54 \\
(self rep.)  &  [0.40]  &  [2.19]  &  [0.88]  &  [0.39]  &  [0.44]  &  [1.89]  &  [1.07]  &  [0.37]  &  [0.67]  &  [1.46]  & .&  [0.30] \\
Depression score &  23.08  &  23.78  &  20.00  &  22.75  &  22.08  &  21.20  &  21.15  &  22.35  &  22.87  &  20.88  & . &  20.90*** \\
(CESD)    &  [0.51]  &  [2.96]  &  [4.58]  &  [0.58]  &  [0.54]  &  [1.31]  &  [1.35]  &  [0.45]  &  [0.85]  &  [2.06]   & . &  [0.40] \\
Respondent health &  0.76  &  0.67  &  1.00  &  0.66*  &  0.54***  &  0.50  &  0.85  &  0.60***  &  0.70  &  0.57   & . &  0.76 \\
is good (\%)    &  [0.03]  &  [0.17]  &  [0.00]  &  [0.04]  &  [0.05]  &  [0.17]  &  [0.10]  &  [0.04]  &  [0.06]  &  [0.20]   & . &  [0.03] \\
%Child health is good (\%) - mom report  &  0.70  &  0.78  &  1.00  &  0.62  &  0.45***  &  0.50  &  0.77  &  0.57**  &  0.62  &  0.75   & . &  0.72 \\
%  &  [0.04]  &  [0.15]  &  [0.00]  &  [0.04]  &  [0.05]  &  [0.17]  &  [0.12]  &  [0.04]  &  [0.06]  &  [0.16]   & . &  [0.03] \\
%Bothered by migrants (\%)  &  0.32  &  0.11  &  0.00  &  0.29  &  0.22  &  0.00  &  0.31  &  0.24  &  0.28  &  0.25   & . &  0.26 \\
%  &  [0.04]  &  [0.11]  &  [0.00]  &  [0.04]  &  [0.04]  &  [0.00]  &  [0.13]  &  [0.04]  &  [0.06]  &  [0.16]  & .  &  [0.03] \\
%Child likes school (\%)  &  0.74  &  0.56  &  0.67  &  0.72  &  0.69  &  0.44  &  0.62  &  0.70  &  0.70  &  0.62   & . &  0.70 \\
%  &  [0.04]  &  [0.18]  &  [0.33]  &  [0.04]  &  [0.05]  &  [0.18]  &  [0.14]  &  [0.04]  &  [0.06]  &  [0.18]   & . &  [0.03] \\
%Child likes math (\%)  &  0.51  &  0.44  &  0.67  &  0.53  &  0.69***  &  0.56  &  0.54  &  0.57  &  0.52  &  0.62   & . &  0.58 \\
%  &  [0.04]  &  [0.18]  &  [0.33]  &  [0.05]  &  [0.05]  &  [0.18]  &  [0.14]  &  [0.04]  &  [0.07]  &  [0.18]   & . &  [0.03] \\
%Child likes reading/italian (\%)  &  0.67  &  0.56  &  0.33  &  0.72  &  0.78  &  0.44  &  0.62  &  0.81**  &  0.60  &  0.25**   & . &  0.65 \\
%  &  [0.04]  &  [0.18]  &  [0.33]  &  [0.04]  &  [0.04]  &  [0.18]  &  [0.14]  &  [0.03]  &  [0.06]  &  [0.16]   & . &  [0.03] \\
%Ability to sit still in a group when asked (difficulties in primary school)  &  0.06  &  0.00  &  0.00  &  0.08  &  0.10  &  0.10  &  0.15  &  0.11  &  0.10  &  0.12   & . &  0.09 \\
%  &  [0.02]  &  [0.00]  &  [0.00]  &  [0.02]  &  [0.03]  &  [0.10]  &  [0.10]  &  [0.03]  &  [0.04]  &  [0.12]   & . &  [0.02] \\
%Lack of excitement to learn (difficulties in primary school)  &  0.03  &  0.00  &  0.00  &  0.05  &  0.04  &  0.00  &  0.00  &  0.02  &  0.07  &  0.00   & . &  0.04 \\
%  &  [0.01]  &  [0.00]  &  [0.00]  &  [0.02]  &  [0.02]  &  [0.00]  &  [0.00]  &  [0.01]  &  [0.03]  &  [0.00]   & . &  [0.01] \\
%Ability to obey rules and directions (difficulties in primary school)  &  0.04  &  0.00  &  0.00  &  0.06  &  0.07  &  0.00  &  0.15  &  0.07  &  0.08  &  0.00   & . &  0.05 \\
%  &  [0.02]  &  [0.00]  &  [0.00]  &  [0.02]  &  [0.03]  &  [0.00]  &  [0.10]  &  [0.02]  &  [0.04]  &  [0.00]   & . &  [0.02] \\
%Fussy eater (difficulties in primary school)  &  0.09  &  0.00  &  0.00  &  0.06  &  0.09  &  0.20  &  0.00  &  0.13  &  0.05  &  0.00   & . &  0.10 \\
%  &  [0.02]  &  [0.00]  &  [0.00]  &  [0.02]  &  [0.03]  &  [0.13]  &  [0.00]  &  [0.03]  &  [0.03]  &  [0.00]   & . &  [0.02] \\
% Difficulties encountered when starting primary school  &  4.55  &  5.00  &  5.00  &  4.46  &  4.35  &  4.40  &  4.38  &  4.34  &  4.39  &  4.50   & . &  4.42 \\
%  &  [0.09]  &  [0.00]  &  [0.00]  &  [0.11]  &  [0.13]  &  [0.40]  &  [0.42]  &  [0.11]  &  [0.17]  &  [0.50]   & . &  [0.09] \\
\hline

% it contains the notes, assuming they are the same for all the tables.
\end{tabular}

\end{adjustbox}
\raggedright{
\footnotesize{Average of baseline characteristcs, by city and type of child-care attended. Standard errors of means in brackets. Test for difference in means between each column and the first column (Reggio Municipal, the treatment group) was performed; *** significant difference at 1\%, ** significant difference at 5\%, * significant difference at 10\%. Source: authors calculation using survey data.}
}
\end{table}
  
%\end{table}

%% adults, infant-toddler
\begin{table}[H] \label{tab:outcomes-adult-asilo}
\caption{Outcomes of interest by infant-toddler-center type, adults (age 30-50)}
% this is the top part of the tables that display the summary of the baseline characteristics, by city and child-care type
\centering
\begin{adjustbox}{width=1.2\textwidth,center=\textwidth}
\small
\begin{tabular}{m{4.0cm} cccccccccccc}
\hline \hline 
 & Reggio & Reggio & Reggio & Reggio & Parma & Parma & Parma & Parma & Padova & Padova & Padova & Padova \\
 & Municipal & Religious & Private & Not Attended & Municipal & Religious & Private & Not Attended & Municipal & Religious & Private & Not Attended \\

\hline 
  
Depression score (CESD)  &  21.28  &  17.75  &  22.33  &  21.99  &  17.91***  &  19.73  &  19.82  &  21.35  &  26.57***  &  22.71  &  20.33  &  21.32 \\
  &  [0.55]  &  [0.75]  &  [4.33]  &  [0.23]  &  [0.55]  &  [1.59]  &  [1.59]  &  [0.24]  &  [0.88]  &  [1.44]  &  [2.33]  &  [0.24] \\
Respondent health is good (\%)  &  0.93  &  1.00  &  1.00  &  0.71***  &  0.29***  &  0.53***  &  0.27***  &  0.60***  &  0.63***  &  0.24***  &  1.00  &  0.54*** \\
  &  [0.03]  &  [0.00]  &  [0.00]  &  [0.02]  &  [0.05]  &  [0.13]  &  [0.14]  &  [0.02]  &  [0.08]  &  [0.11]  &  [0.00]  &  [0.02] \\
Bothered by migrants (\%)  &  0.07  &  0.25  &  0.00  &  0.21***  &  0.21**  &  0.33**  &  0.36**  &  0.23***  &  0.47***  &  0.61***  &  0.67**  &  0.35*** \\
  &  [0.03]  &  [0.25]  &  [0.00]  &  [0.02]  &  [0.05]  &  [0.13]  &  [0.15]  &  [0.02]  &  [0.08]  &  [0.12]  &  [0.33]  &  [0.02] \\
\hline

% it contains the notes, assuming they are the same for all the tables.
\end{tabular}

\end{adjustbox}
\raggedright{
\footnotesize{Average of baseline characteristcs, by city and type of child-care attended. Standard errors of means in brackets. Test for difference in means between each column and the first column (Reggio Municipal, the treatment group) was performed; *** significant difference at 1\%, ** significant difference at 5\%, * significant difference at 10\%. Source: authors calculation using survey data.}
}
\end{table}
  
%\end{table}

%%% children, preschool
\begin{table}[H] \label{tab:outcomes-child-materna}
\caption{Outcomes of interest by preschool type, children (age 6)}
% this is the top part of the tables that display the summary of the baseline characteristics, by city and child-care type
\centering
\begin{adjustbox}{width=1.2\textwidth,center=\textwidth}
\small
\begin{tabular}{m{4.0cm} ccccccccccccccc}
\hline \hline 
 & Reggio & Reggio & Reggio & Reggio & Reggio & Parma & Parma & Parma & Parma & Parma & Padova & Padova & Padova & Padova & Padova \\
 & Municipal & State & Religious & Private & Not Attended & Municipal & State & Religious & Private & Not Attended & Municipal & State & Religious & Private & Not Attended \\

\hline 
  
SDQ score (mom rep.) & 8.33 & 10.02* & 9.00 & 9.00 & 10.00 & 7.80 & 7.19* & 8.66 & 8.56 & 8.67 & 9.07 & 8.85 & 8.23 & 10.00 & 12.50* \\
 & [0.32] & [0.82] & [0.42] & [2.71] & [4.00] & [0.32] & [0.58] & [0.48] & [0.96] & [1.50] & [0.50] & [0.62] & [0.34] & [1.69] & [1.50] \\
Child health is good (\%) - mom report & 0.67 & 0.71 & 0.68 & 0.75 & 0.50 & 0.60 & 0.63 & 0.68 & 0.78 & 0.50 & 0.78 & 0.80 & 0.71 & 0.75 & 1.00 \\
 & [0.04] & [0.07] & [0.05] & [0.25] & [0.50] & [0.04] & [0.07] & [0.05] & [0.15] & [0.22] & [0.05] & [0.06] & [0.04] & [0.13] & [0.00] \\
%Child likes school (\%) & 0.69 & 0.60 & 0.64 & 0.75 & 0.50 & 0.72 & 0.77 & 0.77 & 0.67 & 0.50 & 0.39*** & 0.50** & 0.58* & 0.58 & 1.00 \\
% & [0.04] & [0.07] & [0.05] & [0.25] & [0.50] & [0.04] & [0.07] & [0.05] & [0.17] & [0.22] & [0.05] & [0.08] & [0.04] & [0.15] & [0.00] \\
%Child likes math (\%) & 0.61 & 0.71 & 0.67 & 0.75 & 0.00 & 0.68 & 0.65 & 0.68 & 0.56 & 0.00 & 0.69 & 0.60 & 0.62 & 0.55 & 1.00 \\
% & [0.04] & [0.07] & [0.05] & [0.25] & [0.00] & [0.04] & [0.07] & [0.05] & [0.18] & [0.00] & [0.05] & [0.08] & [0.04] & [0.16] & [0.00] \\
%Child likes reading/italian (\%) & 0.57 & 0.56 & 0.53 & 1.00 & 0.00 & 0.67* & 0.72* & 0.64 & 0.67 & 0.80 & 0.57 & 0.47 & 0.50 & 0.27* & 0.50 \\
% & [0.04] & [0.07] & [0.05] & [0.00] & [0.00] & [0.04] & [0.07] & [0.06] & [0.17] & [0.20] & [0.06] & [0.08] & [0.04] & [0.14] & [0.50] \\
%Ability to sit still in a group when asked (difficulties in primary school) & 0.13 & 0.09 & 0.17 & 0.00 & 0.00 & 0.14 & 0.09 & 0.16 & 0.00 & 0.17 & 0.20 & 0.03* & 0.08 & 0.08 & 0.00 \\
% & [0.03] & [0.04] & [0.04] & [0.00] & [0.00] & [0.03] & [0.04] & [0.04] & [0.00] & [0.17] & [0.04] & [0.03] & [0.02] & [0.08] & [0.00] \\
%Lack of excitement to learn (difficulties in primary school) & 0.02 & 0.07 & 0.02 & 0.00 & 0.00 & 0.03 & 0.05 & 0.06 & 0.00 & 0.33** & 0.09** & 0.05 & 0.04 & 0.08 & 0.00 \\
% & [0.01] & [0.04] & [0.02] & [0.00] & [0.00] & [0.01] & [0.03] & [0.03] & [0.00] & [0.21] & [0.03] & [0.03] & [0.02] & [0.08] & [0.00] \\
%Ability to obey rules and directions (difficulties in primary school) & 0.10 & 0.11 & 0.08 & 0.00 & 0.50 & 0.10 & 0.14 & 0.16 & 0.11 & 0.00 & 0.10 & 0.12 & 0.05* & 0.00 & 0.00 \\
% & [0.02] & [0.05] & [0.03] & [0.00] & [0.50] & [0.02] & [0.05] & [0.04] & [0.11] & [0.00] & [0.03] & [0.05] & [0.02] & [0.00] & [0.00] \\
%Fussy eater (difficulties in primary school) & 0.07 & 0.09 & 0.11 & 0.40** & 0.00 & 0.06 & 0.07 & 0.16** & 0.11 & 0.33* & 0.09 & 0.07 & 0.09 & 0.00 & 0.00 \\
% & [0.02] & [0.04] & [0.03] & [0.24] & [0.00] & [0.02] & [0.04] & [0.04] & [0.11] & [0.21] & [0.03] & [0.04] & [0.02] & [0.00] & [0.00] \\
%Difficulties encountered when starting primary school & 4.25 & 4.22 & 4.04* & 4.60** & 4.00 & 4.25 & 4.30 & 3.95** & 4.67 & 3.00*** & 3.89** & 4.53 & 4.49* & 4.42 & 5.00 \\
% & [0.11] & [0.20] & [0.16] & [0.24] & [1.00] & [0.12] & [0.20] & [0.18] & [0.24] & [0.63] & [0.18] & [0.16] & [0.10] & [0.40] & [0.00] \\
\hline

% it contains the notes, assuming they are the same for all the tables.
\end{tabular}

\end{adjustbox}
\raggedright{
\footnotesize{Average of baseline characteristcs, by city and type of child-care attended. Standard errors of means in brackets. Test for difference in means between each column and the first column (Reggio Municipal, the treatment group) was performed; *** significant difference at 1\%, ** significant difference at 5\%, * significant difference at 10\%. Source: authors calculation using survey data.}
}
\end{table}
  
%\end{table}
%
%%% adolescents, preschool
\begin{table}[H] \label{tab:outcomes-adol-materna}
\caption{Outcomes of interest by preschool type, adolescents (age 18)}
% this is the top part of the tables that display the summary of the baseline characteristics, by city and child-care type
\centering
\begin{adjustbox}{width=1.2\textwidth,center=\textwidth}
\small
\begin{tabular}{m{4.0cm} ccccccccccccccc}
\hline \hline 
 & Reggio & Reggio & Reggio & Reggio & Reggio & Parma & Parma & Parma & Parma & Parma & Padova & Padova & Padova & Padova & Padova \\
 & Municipal & State & Religious & Private & Not Attended & Municipal & State & Religious & Private & Not Attended & Municipal & State & Religious & Private & Not Attended \\

\hline 
  
%SDQ score (mom rep.) & 8.16 & 6.64* & 8.70 & 9.33 & 8.86 & 7.86 & 7.95 & 8.73 & 8.17 & 9.75 & 7.68 & 9.02 & 8.32 & 8.17 & . \\
% & [0.33] & [0.58] & [0.53] & [1.43] & [1.40] & [0.39] & [0.60] & [0.54] & [2.27] & [2.32] & [0.34] & [0.60] & [0.37] & [1.45] & . \\
SDQ score & 9.82 & 7.73** & 10.77* & 11.00 & 7.43 & 9.11 & 8.00** & 9.53 & 8.67 & 8.75 & 9.56 & 9.00 & 10.18 & 7.50 & . \\
(self rep.)  & [0.37] & [0.79] & [0.47] & [1.32] & [1.63] & [0.39] & [0.68] & [0.47] & [1.50] & [0.85] & [0.47] & [0.55] & [0.43] & [1.26] & . \\
Depression score & 22.46 & 20.71 & 24.15** & 23.33 & 19.14 & 22.29 & 20.60 & 22.81 & 20.00 & 20.75 & 21.75 & 19.32*** & 21.77 & 19.67 & . \\
(CESD)  & [0.51] & [1.23] & [0.69] & [1.58] & [1.50] & [0.48] & [0.74] & [0.58] & [1.55] & [2.66] & [0.61] & [0.78] & [0.55] & [1.98] & . \\
Respondent health & 0.74 & 0.82 & 0.66 & 0.83 & 0.71 & 0.62** & 0.37*** & 0.63 & 1.00 & 0.75 & 0.68 & 0.85 & 0.75 & 0.67 & . \\
is good (\%)  & [0.03] & [0.08] & [0.05] & [0.17] & [0.18] & [0.05] & [0.07] & [0.05] & [0.00] & [0.25] & [0.05] & [0.05] & [0.04] & [0.21] & . \\
%Child health is good (\%) - mom report & 0.69 & 0.64 & 0.66 & 0.33* & 0.71 & 0.49*** & 0.40*** & 0.63 & 0.83 & 1.00 & 0.65 & 0.83* & 0.69 & 0.67 & . \\
% & [0.04] & [0.10] & [0.05] & [0.21] & [0.18] & [0.05] & [0.08] & [0.05] & [0.17] & [0.00] & [0.05] & [0.06] & [0.04] & [0.21] & . \\
%Bothered by migrants (\%) & 0.26 & 0.23 & 0.37* & 0.67** & 0.14 & 0.23 & 0.14 & 0.28 & 0.17 & 0.75* & 0.24 & 0.18 & 0.32 & 0.33 & 0.00 \\
% & [0.03] & [0.09] & [0.05] & [0.21] & [0.14] & [0.04] & [0.05] & [0.05] & [0.17] & [0.25] & [0.05] & [0.06] & [0.04] & [0.21] & . \\
%Child likes school (\%) & 0.73 & 0.85 & 0.72 & 0.33* & 0.60 & 0.68 & 0.79 & 0.62 & 0.50 & 1.00 & 0.69 & 0.68 & 0.71 & 0.83 & 0.00 \\
% & [0.03] & [0.08] & [0.05] & [0.21] & [0.24] & [0.04] & [0.06] & [0.05] & [0.22] & [0.00] & [0.05] & [0.07] & [0.04] & [0.17] & . \\
%Child likes math (\%) & 0.51 & 0.80** & 0.46 & 0.67 & 0.40 & 0.62* & 0.67* & 0.59 & 0.33 & 0.50 & 0.57 & 0.42 & 0.58 & 1.00 & 0.00 \\
% & [0.04] & [0.09] & [0.05] & [0.21] & [0.24] & [0.05] & [0.07] & [0.06] & [0.21] & [0.29] & [0.05] & [0.08] & [0.04] & [0.00] & . \\
%Child likes reading/italian (\%) & 0.71 & 0.65 & 0.66 & 0.50 & 0.80 & 0.80* & 0.81 & 0.73 & 0.67 & 0.50 & 0.63 & 0.67 & 0.62 & 0.50 & 1.00 \\
% & [0.04] & [0.11] & [0.05] & [0.22] & [0.20] & [0.04] & [0.06] & [0.05] & [0.21] & [0.29] & [0.05] & [0.07] & [0.04] & [0.22] & . \\
%Ability to sit still in a group when asked (difficulties in primary school) & 0.07 & 0.00 & 0.06 & 0.17 & 0.14 & 0.11 & 0.02 & 0.13* & 0.17 & 0.25 & 0.08 & 0.17** & 0.08 & 0.00 & 0.00 \\
% & [0.02] & [0.00] & [0.02] & [0.17] & [0.14] & [0.03] & [0.02] & [0.04] & [0.17] & [0.25] & [0.03] & [0.06] & [0.02] & [0.00] & . \\
%Lack of excitement to learn (difficulties in primary school) & 0.03 & 0.05 & 0.04 & 0.00 & 0.29** & 0.05 & 0.00 & 0.00 & 0.00 & 0.25 & 0.03 & 0.09 & 0.04 & 0.17 & 0.00 \\
% & [0.01] & [0.05] & [0.02] & [0.00] & [0.18] & [0.02] & [0.00] & [0.00] & [0.00] & [0.25] & [0.02] & [0.04] & [0.02] & [0.17] & . \\
%Ability to obey rules and directions (difficulties in primary school) & 0.04 & 0.05 & 0.04 & 0.00 & 0.14 & 0.08 & 0.09 & 0.09 & 0.00 & 0.00 & 0.06 & 0.11 & 0.05 & 0.00 & 0.00 \\
% & [0.02] & [0.05] & [0.02] & [0.00] & [0.14] & [0.02] & [0.04] & [0.03] & [0.00] & [0.00] & [0.03] & [0.05] & [0.02] & [0.00] & . \\
%Fussy eater (difficulties in primary school) & 0.07 & 0.05 & 0.08 & 0.33* & 0.00 & 0.15** & 0.05 & 0.11 & 0.00 & 0.00 & 0.05 & 0.17** & 0.07 & 0.17 & 0.00 \\
% & [0.02] & [0.05] & [0.03] & [0.21] & [0.00] & [0.03] & [0.03] & [0.03] & [0.00] & [0.00] & [0.02] & [0.06] & [0.02] & [0.17] & . \\
%Difficulties encountered when starting primary school & 4.57 & 4.82 & 4.52 & 4.00* & 3.57** & 4.24* & 4.67 & 4.34 & 4.33 & 3.25* & 4.57 & 3.98** & 4.46 & 4.33 & 5.00 \\
% & [0.09] & [0.14] & [0.11] & [0.63] & [0.69] & [0.13] & [0.13] & [0.15] & [0.67] & [1.03] & [0.12] & [0.23] & [0.11] & [0.49] & . \\
\hline

% it contains the notes, assuming they are the same for all the tables.
\end{tabular}

\end{adjustbox}
\raggedright{
\footnotesize{Average of baseline characteristcs, by city and type of child-care attended. Standard errors of means in brackets. Test for difference in means between each column and the first column (Reggio Municipal, the treatment group) was performed; *** significant difference at 1\%, ** significant difference at 5\%, * significant difference at 10\%. Source: authors calculation using survey data.}
}
\end{table}
  
%\end{table}
%
%%% adults, preschool
\begin{table}[H] \label{tab:outcomes-adult-materna}
\caption{Outcomes of interest by preschool type, adults (age 30-50)}
% this is the top part of the tables that display the summary of the baseline characteristics, by city and child-care type
\centering
\begin{adjustbox}{width=1.2\textwidth,center=\textwidth}
\small
\begin{tabular}{m{4.0cm} ccccccccccccccc}
\hline \hline 
 & Reggio & Reggio & Reggio & Reggio & Reggio & Parma & Parma & Parma & Parma & Parma & Padova & Padova & Padova & Padova & Padova \\
 & Municipal & State & Religious & Private & Not Attended & Municipal & State & Religious & Private & Not Attended & Municipal & State & Religious & Private & Not Attended \\

\hline 
  
Depression score (CESD) & 21.25 & 22.50 & 20.77 & 19.29 & 22.96*** & 20.50 & 18.96*** & 20.04 & 20.83 & 22.21*** & 22.11 & 25.20*** & 20.79 & 20.67 & 22.07* \\
 & [0.34] & [0.88] & [0.50] & [1.49] & [0.33] & [0.47] & [0.66] & [0.48] & [2.99] & [0.31] & [0.70] & [0.81] & [0.31] & [2.96] & [0.46] \\
Respondent health is good (\%) & 0.88 & 0.81 & 0.75*** & 0.75 & 0.57*** & 0.54*** & 0.48*** & 0.68*** & 0.33*** & 0.50*** & 0.59*** & 0.80 & 0.53*** & 0.50 & 0.47*** \\
 & [0.02] & [0.05] & [0.04] & [0.16] & [0.03] & [0.04] & [0.05] & [0.04] & [0.21] & [0.03] & [0.06] & [0.06] & [0.03] & [0.50] & [0.04] \\
Bothered by migrants (\%) & 0.12 & 0.16 & 0.25*** & 0.14 & 0.23*** & 0.30*** & 0.24*** & 0.30*** & 0.17 & 0.16 & 0.50*** & 0.32*** & 0.37*** & 0.33 & 0.34*** \\
 & [0.02] & [0.05] & [0.04] & [0.14] & [0.03] & [0.04] & [0.05] & [0.04] & [0.17] & [0.02] & [0.06] & [0.07] & [0.03] & [0.33] & [0.04] \\
\hline

% it contains the notes, assuming they are the same for all the tables.
\end{tabular}

\end{adjustbox}
\raggedright{
\footnotesize{Average of baseline characteristcs, by city and type of child-care attended. Standard errors of means in brackets. Test for difference in means between each column and the first column (Reggio Municipal, the treatment group) was performed; *** significant difference at 1\%, ** significant difference at 5\%, * significant difference at 10\%. Source: authors calculation using survey data.}
}
\end{table}
  
%\end{table}

%%%%%%%%%%%%
Finally, table (\ref{tab:outcomes-list}) compares the outcomes available in the our survey data with the ones available in ABC and Perry.
\begin{table}[htbp]\label{tab:outcomes-list}
\begin{center}

	\caption{Outcome Variables in Reggio and Other Early Childhood Studies}
\begin{tabular}{lllll}
\toprule
Outcome & Reggio & ABC & CARE & Perry \\
\midrule
\textbf{Cognitive} & & & & \\
\quad IQ & 6, 18, 32, 43, 54-60 & 6, 15 & 6 & 6, 14 \\
\textbf{Schooling} & & & & \\
\quad HS Graduation & 32, 43, 54-60 & 30 & 30 & 27, 40 \\
\quad Ever Suspended & 18 & & & 27 \\
\textbf{Socio-emotional} & & & & \\
\quad Strengths and Difficulties & 6, 18, 32, 43, 54-60 & & & \\
\quad Depression &18, 32, 43, 54-60 & 15, 21, 34 & 21, 34 & \\
\textbf{Health} & & & & \\
\quad Number of Cigarettes (per day) & 18, 32, 43, 54-60 & 21 & 21 & 27 \\
\quad BMI & 6, 32, 43, 54-60 & 34 & 34 & 40 \\
\quad Health Problems & 32, 43, 54-60 & 34 & 34 & 27 \\
\quad Ever Tried Drugs & 32, 43, 54-60 & 12, 15, 21, 30 & 12, 15, 21, 30 & 27 \\
\bottomrule
\end{tabular}
%
%
\end{center}
\footnotesize 
Notes: The columns for each dataset specify the ages in years at which the variables are available. IQ is measured by Raven's Progressive Matrices for Reggio, the Wechsler Intelligence Scale for Children (WISC) for ABC and CARE, and WISC and Stanford Binet for Perry. The ``Health Problems" item in Reggio, ABC, and CARE corresponds to the number of days in the past 30 days the subject was sick. In Perry, this variable is the number of days sick in the past year. ``Strengths and Difficulties" is the Strengths and Difficulties Questionnaire administered to adolescent and adult subjects in Reggio. ``Depression" is the Depression Scale in Reggio. In ABC, depression is measured from the Achenbach Youth Report at age 15. In ABC and CARE, depression is measured using the Brief Symptom Inventory at age 21 and is self-reported at age 34.
\end{table}

%\todo[backgroundcolor=orange!30,size=\tiny]{Tables with raw differences across cities and preschool types}


\section{Data}

\label{sec:data}

This section discusses the survey data which has been used for the analysis
in this report.\footnote{%
Also administrative data from the RA preschool system was collected. Its
description and more details about the survey data are contained in %
\citet{biroli2015evaluating}} To evaluate the impact of Reggio, individuals
living in Reggio Emilia, Parma, and Padova since their first year of life
were interviewed. Data were collected on five age cohorts, including three
cohorts of adults, one cohort of adolescents, and one cohort of children in
their first year of elementary school.

The structure of the cohorts is described in Table \ref{tab:cohorts}.

\begin{table}[tbph]
\caption{Cohort Structure}
\label{tab:cohorts}
\begin{center}
\begin{tabular}{ccc}
\hline\hline
Cohort & Years of Birth & Age at Interview \\ \hline
I & 1954--1959 & 54--60 \\ 
II & 1969--1970 & 43 \\ 
III & 1980--1981 & 32 \\ 
IV & 1994 & 18 \\ 
V & 2006 & 6 \\ \hline
\end{tabular}%
\end{center}
\end{table}

In order to fix ideas on the identification strategy, we currently focus our
attention only on the adolescents, the cohort for which we have the most
wide array of information on both the caregiver and the respondent of
interest.\textbf{I would add all cohorts in this table to show that
different cohorts have differnet options of school types}

Table (\ref{tab:sample-adol}) shows the number of adolescents interviewed in
our sample who attended the different types of child-care in Reggio Emilia,
Parma, and Padova.

\begin{table}[tbph]
\caption{Sample Size by City and School Type; Adolescent }
\label{tab:sample-adol}
\begin{center}
\begin{tabular}{l|ccc|ccc}
\hline\hline
\multicolumn{1}{r|}{City:} & Reggio & Parma & Padova & Reggio & Parma & 
Padova \\ 
\textit{School} & \multicolumn{3}{c|}{\textit{Infant-Toddler Center}} & 
\multicolumn{3}{c}{\textit{Preschool}} \\ \hline
Not Attended & 130 & 130 & 210 & 7 & 4 & 1 \\ 
Municipal & 153 & 97 & 61 & 166 & 116 & 93 \\ 
State & . & . & . & 22 & 43 & 47 \\ 
Religious & 9 & 10 & 8 & 96 & 82 & 131 \\ 
Private & 3 & 13 & 0 & 6 & 6 & 6 \\ \hline
\end{tabular}%
\end{center}
\end{table}

\bigskip 

% ------------------------------ Section DATA ------------------------------------------%
%\section{Data}
\label{sec:data}

This section discusses the survey data which has been used for the analysis in this report.\footnote{Also administrative data from the RA preschool system was collected. Its description and more details about the survey data are contained in \citet{biroli2015evaluating}} To evaluate the impact of Reggio, individuals living in Reggio Emilia, Parma, and Padova since their first year of life were interviewed. Data were collected on five age cohorts, including three cohorts of adults, one cohort of adolescents, and one cohort of children in their first year of elementary school. 

The structure of the cohorts is described in Table \ref{tab:cohorts}.

\begin{table}[htbp]
\begin{center}
\caption{Cohort Structure}\label{tab:cohorts}
\begin{tabular}{ccc}
\toprule
Cohort & Years of Birth & Age at Interview \\
\midrule
I & 1954--1959 & 54--60 \\
II & 1969--1970 & 43 \\
III & 1980--1981 & 32 \\
IV & 1994 & 18 \\
V & 2006 & 6 \\
\bottomrule  
\end{tabular}
\end{center}
\end{table}

\subsection{Outcomes considered}\label{sec:outcomes}
A rich set of outcomes were collected during the survey. Individuals at different stages of their lifecycle were asked about family composition, fertility, labor force participation, income, schooling, cognitive ability, social and emotional skills, health and healthy habits, social capital, interpersonal ties, as well as attitudes on immigration and integration. 

We focus our attention on four different outcomes, in four different domains: the mother reports to the Strength and Difficulties Questionnaire (SDQ), a widely used and validated scale from 0 to 40 measuring behavioral problems for children and adolescents; the percentage of respondents reporting to have good or excellent health;\footnote{For children and adolescent, the respondent is again the mother} the raw score of the Center for Epidemiological Studies Depression Scale (CESD), a scale ranging from 10 to 50 and measuring self-reported depression for adolescents and adults; and finally the share of respondents reporting to be `very' or  `quite' bothered by the immigration into the city. Tables (\ref{tab:outcomes-child-asilo}) to (\ref{tab:outcomes-adult-asilo}) report the summary statistics of these outcome variables, disaggregated by city and type of \textit{infant-toddler center} attended. Similarly, Tables (\ref{tab:outcomes-child-materna}) to (\ref{tab:outcomes-adult-materna}) report the average of these same variables, this time focusing on type of \textit{preschool} attended. % We see that


%% children, infant-toddler
\begin{table}[H] \label{tab:outcomes-child-asilo}
\caption{Outcomes of interest by infant-toddler-center type, children (age 6)}
% this is the top part of the tables that display the summary of the baseline characteristics, by city and child-care type
\centering
\begin{adjustbox}{width=1.2\textwidth,center=\textwidth}
\small
\begin{tabular}{m{4.0cm} cccccccccccc}
\hline \hline 
 & Reggio & Reggio & Reggio & Reggio & Parma & Parma & Parma & Parma & Padova & Padova & Padova & Padova \\
 & Municipal & Religious & Private & Not Attended & Municipal & Religious & Private & Not Attended & Municipal & Religious & Private & Not Attended \\

\hline 

SDQ score (mom rep.)   &   8.36   &   8.19   &   7.60   &   9.53**   &   7.93   &   8.29   &   7.11*   &   8.30   &   9.07   &   9.08   &   8.56   &   8.49 \\
   &   [0.32]   &   [0.79]   &   [1.78]   &   [0.43]   &   [0.33]   &   [1.44]   &   [0.52]   &   [0.43]   &   [0.54]   &   [0.98]   &   [0.75]   &   [0.33] \\
Child health is good (\%) - mom report   &   0.65   &   0.74   &   0.80   &   0.71   &   0.60   &   0.86   &   0.67   &   0.64   &   0.79**   &   0.77   &   0.71   &   0.73 \\
   &   [0.04]   &   [0.09]   &   [0.20]   &   [0.04]   &   [0.04]   &   [0.14]   &   [0.08]   &   [0.05]   &   [0.05]   &   [0.08]   &   [0.07]   &   [0.04] \\
%Child likes school (\%)   &   0.71   &   0.70   &   0.80   &   0.59**   &   0.71   &   0.71   &   0.78   &   0.74   &   0.46***   &   0.46**   &   0.59   &   0.54*** \\
%   &   [0.04]   &   [0.09]   &   [0.20]   &   [0.05]   &   [0.04]   &   [0.18]   &   [0.07]   &   [0.04]   &   [0.06]   &   [0.10]   &   [0.08]   &   [0.04] \\
%Child likes math (\%)   &   0.61   &   0.81*   &   1.00   &   0.62   &   0.68   &   0.86   &   0.69   &   0.58   &   0.68   &   0.62   &   0.65   &   0.62 \\
%   &   [0.04]   &   [0.08]   &   [0.00]   &   [0.05]   &   [0.04]   &   [0.14]   &   [0.08]   &   [0.05]   &   [0.06]   &   [0.10]   &   [0.08]   &   [0.04] \\
%Child likes reading/italian (\%)   &   0.59   &   0.59   &   0.80   &   0.49   &   0.67   &   0.86   &   0.78*   &   0.62   &   0.60   &   0.50   &   0.57   &   0.44*** \\
%   &   [0.04]   &   [0.10]   &   [0.20]   &   [0.05]   &   [0.04]   &   [0.14]   &   [0.07]   &   [0.05]   &   [0.06]   &   [0.10]   &   [0.08]   &   [0.04] \\
%Ability to sit still in a group when asked (difficulties in primary school)   &   0.12   &   0.07   &   0.00   &   0.19   &   0.15   &   0.14   &   0.06   &   0.14   &   0.19   &   0.15   &   0.10   &   0.06* \\
%   &   [0.03]   &   [0.05]   &   [0.00]   &   [0.04]   &   [0.03]   &   [0.14]   &   [0.04]   &   [0.04]   &   [0.05]   &   [0.07]   &   [0.05]   &   [0.02] \\
%Lack of excitement to learn (difficulties in primary school)   &   0.02   &   0.04   &   0.00   &   0.05   &   0.05   &   0.00   &   0.00   &   0.06*   &   0.07*   &   0.04   &   0.02   &   0.06 \\
%   &   [0.01]   &   [0.04]   &   [0.00]   &   [0.02]   &   [0.02]   &   [0.00]   &   [0.00]   &   [0.02]   &   [0.03]   &   [0.04]   &   [0.02]   &   [0.02] \\
%Ability to obey rules and directions (difficulties in primary school)   &   0.10   &   0.04   &   0.00   &   0.11   &   0.15   &   0.00   &   0.06   &   0.09   &   0.13   &   0.08   &   0.05   &   0.05 \\
%   &   [0.02]   &   [0.04]   &   [0.00]   &   [0.03]   &   [0.03]   &   [0.00]   &   [0.04]   &   [0.03]   &   [0.04]   &   [0.05]   &   [0.03]   &   [0.02] \\
%Fussy eater (difficulties in primary school)   &   0.11   &   0.19   &   0.17   &   0.03**   &   0.05*   &   0.14   &   0.03   &   0.18*   &   0.09   &   0.12   &   0.02   &   0.08 \\
%   &   [0.02]   &   [0.08]   &   [0.17]   &   [0.02]   &   [0.02]   &   [0.14]   &   [0.03]   &   [0.04]   &   [0.03]   &   [0.06]   &   [0.02]   &   [0.02] \\
% Difficulties encountered when starting primary school   &   4.29   &   4.44   &   4.83   &   3.92   &   4.17   &   4.43   &   4.64   &   3.98***   &   3.93   &   4.23   &   4.44   &   4.49* \\
%   &   [0.11]   &   [0.22]   &   [0.17]   &   [0.15]   &   [0.12]   &   [0.57]   &   [0.17]   &   [0.15]   &   [0.20]   &   [0.30]   &   [0.20]   &   [0.10] \\
\hline

% it contains the notes, assuming they are the same for all the tables.
\end{tabular}

\end{adjustbox}
\raggedright{
\footnotesize{Average of baseline characteristcs, by city and type of child-care attended. Standard errors of means in brackets. Test for difference in means between each column and the first column (Reggio Municipal, the treatment group) was performed; *** significant difference at 1\%, ** significant difference at 5\%, * significant difference at 10\%. Source: authors calculation using survey data.}
}
\end{table}
  
%\end{table}

%% adolescents, infant-toddler
\begin{table}[H] \label{tab:outcomes-adol-asilo}
\caption{Outcomes of interest by infant-toddler-center type, adolescents (age 18)}
% this is the top part of the tables that display the summary of the baseline characteristics, by city and child-care type
\centering
\begin{adjustbox}{width=1.2\textwidth,center=\textwidth}
\small
\begin{tabular}{m{4.0cm} cccccccccccc}
\hline \hline 
 & Reggio & Reggio & Reggio & Reggio & Parma & Parma & Parma & Parma & Padova & Padova & Padova & Padova \\
 & Municipal & Religious & Private & Not Attended & Municipal & Religious & Private & Not Attended & Municipal & Religious & Private & Not Attended \\

\hline 
  
%SDQ score (mom rep.)  &  7.95  &  8.89  &  9.67  &  8.47  &  8.02  &  9.70  &  9.38  &  8.01  &  8.15  &  6.38  & . &  8.32 \\
%  &  [0.36]  &  [2.64]  &  [2.33]  &  [0.37]  &  [0.42]  &  [2.39]  &  [1.45]  &  [0.38]  &  [0.57]  &  [0.91]  & .&  [0.26] \\
SDQ score &  9.96  &  11.89  &  7.67  &  10.04  &  9.11  &  9.10  &  9.46  &  8.95  &  10.52  &  7.75 & . &  9.54 \\
(self rep.)  &  [0.40]  &  [2.19]  &  [0.88]  &  [0.39]  &  [0.44]  &  [1.89]  &  [1.07]  &  [0.37]  &  [0.67]  &  [1.46]  & .&  [0.30] \\
Depression score &  23.08  &  23.78  &  20.00  &  22.75  &  22.08  &  21.20  &  21.15  &  22.35  &  22.87  &  20.88  & . &  20.90*** \\
(CESD)    &  [0.51]  &  [2.96]  &  [4.58]  &  [0.58]  &  [0.54]  &  [1.31]  &  [1.35]  &  [0.45]  &  [0.85]  &  [2.06]   & . &  [0.40] \\
Respondent health &  0.76  &  0.67  &  1.00  &  0.66*  &  0.54***  &  0.50  &  0.85  &  0.60***  &  0.70  &  0.57   & . &  0.76 \\
is good (\%)    &  [0.03]  &  [0.17]  &  [0.00]  &  [0.04]  &  [0.05]  &  [0.17]  &  [0.10]  &  [0.04]  &  [0.06]  &  [0.20]   & . &  [0.03] \\
%Child health is good (\%) - mom report  &  0.70  &  0.78  &  1.00  &  0.62  &  0.45***  &  0.50  &  0.77  &  0.57**  &  0.62  &  0.75   & . &  0.72 \\
%  &  [0.04]  &  [0.15]  &  [0.00]  &  [0.04]  &  [0.05]  &  [0.17]  &  [0.12]  &  [0.04]  &  [0.06]  &  [0.16]   & . &  [0.03] \\
%Bothered by migrants (\%)  &  0.32  &  0.11  &  0.00  &  0.29  &  0.22  &  0.00  &  0.31  &  0.24  &  0.28  &  0.25   & . &  0.26 \\
%  &  [0.04]  &  [0.11]  &  [0.00]  &  [0.04]  &  [0.04]  &  [0.00]  &  [0.13]  &  [0.04]  &  [0.06]  &  [0.16]  & .  &  [0.03] \\
%Child likes school (\%)  &  0.74  &  0.56  &  0.67  &  0.72  &  0.69  &  0.44  &  0.62  &  0.70  &  0.70  &  0.62   & . &  0.70 \\
%  &  [0.04]  &  [0.18]  &  [0.33]  &  [0.04]  &  [0.05]  &  [0.18]  &  [0.14]  &  [0.04]  &  [0.06]  &  [0.18]   & . &  [0.03] \\
%Child likes math (\%)  &  0.51  &  0.44  &  0.67  &  0.53  &  0.69***  &  0.56  &  0.54  &  0.57  &  0.52  &  0.62   & . &  0.58 \\
%  &  [0.04]  &  [0.18]  &  [0.33]  &  [0.05]  &  [0.05]  &  [0.18]  &  [0.14]  &  [0.04]  &  [0.07]  &  [0.18]   & . &  [0.03] \\
%Child likes reading/italian (\%)  &  0.67  &  0.56  &  0.33  &  0.72  &  0.78  &  0.44  &  0.62  &  0.81**  &  0.60  &  0.25**   & . &  0.65 \\
%  &  [0.04]  &  [0.18]  &  [0.33]  &  [0.04]  &  [0.04]  &  [0.18]  &  [0.14]  &  [0.03]  &  [0.06]  &  [0.16]   & . &  [0.03] \\
%Ability to sit still in a group when asked (difficulties in primary school)  &  0.06  &  0.00  &  0.00  &  0.08  &  0.10  &  0.10  &  0.15  &  0.11  &  0.10  &  0.12   & . &  0.09 \\
%  &  [0.02]  &  [0.00]  &  [0.00]  &  [0.02]  &  [0.03]  &  [0.10]  &  [0.10]  &  [0.03]  &  [0.04]  &  [0.12]   & . &  [0.02] \\
%Lack of excitement to learn (difficulties in primary school)  &  0.03  &  0.00  &  0.00  &  0.05  &  0.04  &  0.00  &  0.00  &  0.02  &  0.07  &  0.00   & . &  0.04 \\
%  &  [0.01]  &  [0.00]  &  [0.00]  &  [0.02]  &  [0.02]  &  [0.00]  &  [0.00]  &  [0.01]  &  [0.03]  &  [0.00]   & . &  [0.01] \\
%Ability to obey rules and directions (difficulties in primary school)  &  0.04  &  0.00  &  0.00  &  0.06  &  0.07  &  0.00  &  0.15  &  0.07  &  0.08  &  0.00   & . &  0.05 \\
%  &  [0.02]  &  [0.00]  &  [0.00]  &  [0.02]  &  [0.03]  &  [0.00]  &  [0.10]  &  [0.02]  &  [0.04]  &  [0.00]   & . &  [0.02] \\
%Fussy eater (difficulties in primary school)  &  0.09  &  0.00  &  0.00  &  0.06  &  0.09  &  0.20  &  0.00  &  0.13  &  0.05  &  0.00   & . &  0.10 \\
%  &  [0.02]  &  [0.00]  &  [0.00]  &  [0.02]  &  [0.03]  &  [0.13]  &  [0.00]  &  [0.03]  &  [0.03]  &  [0.00]   & . &  [0.02] \\
% Difficulties encountered when starting primary school  &  4.55  &  5.00  &  5.00  &  4.46  &  4.35  &  4.40  &  4.38  &  4.34  &  4.39  &  4.50   & . &  4.42 \\
%  &  [0.09]  &  [0.00]  &  [0.00]  &  [0.11]  &  [0.13]  &  [0.40]  &  [0.42]  &  [0.11]  &  [0.17]  &  [0.50]   & . &  [0.09] \\
\hline

% it contains the notes, assuming they are the same for all the tables.
\end{tabular}

\end{adjustbox}
\raggedright{
\footnotesize{Average of baseline characteristcs, by city and type of child-care attended. Standard errors of means in brackets. Test for difference in means between each column and the first column (Reggio Municipal, the treatment group) was performed; *** significant difference at 1\%, ** significant difference at 5\%, * significant difference at 10\%. Source: authors calculation using survey data.}
}
\end{table}
  
%\end{table}

%% adults, infant-toddler
\begin{table}[H] \label{tab:outcomes-adult-asilo}
\caption{Outcomes of interest by infant-toddler-center type, adults (age 30-50)}
% this is the top part of the tables that display the summary of the baseline characteristics, by city and child-care type
\centering
\begin{adjustbox}{width=1.2\textwidth,center=\textwidth}
\small
\begin{tabular}{m{4.0cm} cccccccccccc}
\hline \hline 
 & Reggio & Reggio & Reggio & Reggio & Parma & Parma & Parma & Parma & Padova & Padova & Padova & Padova \\
 & Municipal & Religious & Private & Not Attended & Municipal & Religious & Private & Not Attended & Municipal & Religious & Private & Not Attended \\

\hline 
  
Depression score (CESD)  &  21.28  &  17.75  &  22.33  &  21.99  &  17.91***  &  19.73  &  19.82  &  21.35  &  26.57***  &  22.71  &  20.33  &  21.32 \\
  &  [0.55]  &  [0.75]  &  [4.33]  &  [0.23]  &  [0.55]  &  [1.59]  &  [1.59]  &  [0.24]  &  [0.88]  &  [1.44]  &  [2.33]  &  [0.24] \\
Respondent health is good (\%)  &  0.93  &  1.00  &  1.00  &  0.71***  &  0.29***  &  0.53***  &  0.27***  &  0.60***  &  0.63***  &  0.24***  &  1.00  &  0.54*** \\
  &  [0.03]  &  [0.00]  &  [0.00]  &  [0.02]  &  [0.05]  &  [0.13]  &  [0.14]  &  [0.02]  &  [0.08]  &  [0.11]  &  [0.00]  &  [0.02] \\
Bothered by migrants (\%)  &  0.07  &  0.25  &  0.00  &  0.21***  &  0.21**  &  0.33**  &  0.36**  &  0.23***  &  0.47***  &  0.61***  &  0.67**  &  0.35*** \\
  &  [0.03]  &  [0.25]  &  [0.00]  &  [0.02]  &  [0.05]  &  [0.13]  &  [0.15]  &  [0.02]  &  [0.08]  &  [0.12]  &  [0.33]  &  [0.02] \\
\hline

% it contains the notes, assuming they are the same for all the tables.
\end{tabular}

\end{adjustbox}
\raggedright{
\footnotesize{Average of baseline characteristcs, by city and type of child-care attended. Standard errors of means in brackets. Test for difference in means between each column and the first column (Reggio Municipal, the treatment group) was performed; *** significant difference at 1\%, ** significant difference at 5\%, * significant difference at 10\%. Source: authors calculation using survey data.}
}
\end{table}
  
%\end{table}

%%% children, preschool
\begin{table}[H] \label{tab:outcomes-child-materna}
\caption{Outcomes of interest by preschool type, children (age 6)}
% this is the top part of the tables that display the summary of the baseline characteristics, by city and child-care type
\centering
\begin{adjustbox}{width=1.2\textwidth,center=\textwidth}
\small
\begin{tabular}{m{4.0cm} ccccccccccccccc}
\hline \hline 
 & Reggio & Reggio & Reggio & Reggio & Reggio & Parma & Parma & Parma & Parma & Parma & Padova & Padova & Padova & Padova & Padova \\
 & Municipal & State & Religious & Private & Not Attended & Municipal & State & Religious & Private & Not Attended & Municipal & State & Religious & Private & Not Attended \\

\hline 
  
SDQ score (mom rep.) & 8.33 & 10.02* & 9.00 & 9.00 & 10.00 & 7.80 & 7.19* & 8.66 & 8.56 & 8.67 & 9.07 & 8.85 & 8.23 & 10.00 & 12.50* \\
 & [0.32] & [0.82] & [0.42] & [2.71] & [4.00] & [0.32] & [0.58] & [0.48] & [0.96] & [1.50] & [0.50] & [0.62] & [0.34] & [1.69] & [1.50] \\
Child health is good (\%) - mom report & 0.67 & 0.71 & 0.68 & 0.75 & 0.50 & 0.60 & 0.63 & 0.68 & 0.78 & 0.50 & 0.78 & 0.80 & 0.71 & 0.75 & 1.00 \\
 & [0.04] & [0.07] & [0.05] & [0.25] & [0.50] & [0.04] & [0.07] & [0.05] & [0.15] & [0.22] & [0.05] & [0.06] & [0.04] & [0.13] & [0.00] \\
%Child likes school (\%) & 0.69 & 0.60 & 0.64 & 0.75 & 0.50 & 0.72 & 0.77 & 0.77 & 0.67 & 0.50 & 0.39*** & 0.50** & 0.58* & 0.58 & 1.00 \\
% & [0.04] & [0.07] & [0.05] & [0.25] & [0.50] & [0.04] & [0.07] & [0.05] & [0.17] & [0.22] & [0.05] & [0.08] & [0.04] & [0.15] & [0.00] \\
%Child likes math (\%) & 0.61 & 0.71 & 0.67 & 0.75 & 0.00 & 0.68 & 0.65 & 0.68 & 0.56 & 0.00 & 0.69 & 0.60 & 0.62 & 0.55 & 1.00 \\
% & [0.04] & [0.07] & [0.05] & [0.25] & [0.00] & [0.04] & [0.07] & [0.05] & [0.18] & [0.00] & [0.05] & [0.08] & [0.04] & [0.16] & [0.00] \\
%Child likes reading/italian (\%) & 0.57 & 0.56 & 0.53 & 1.00 & 0.00 & 0.67* & 0.72* & 0.64 & 0.67 & 0.80 & 0.57 & 0.47 & 0.50 & 0.27* & 0.50 \\
% & [0.04] & [0.07] & [0.05] & [0.00] & [0.00] & [0.04] & [0.07] & [0.06] & [0.17] & [0.20] & [0.06] & [0.08] & [0.04] & [0.14] & [0.50] \\
%Ability to sit still in a group when asked (difficulties in primary school) & 0.13 & 0.09 & 0.17 & 0.00 & 0.00 & 0.14 & 0.09 & 0.16 & 0.00 & 0.17 & 0.20 & 0.03* & 0.08 & 0.08 & 0.00 \\
% & [0.03] & [0.04] & [0.04] & [0.00] & [0.00] & [0.03] & [0.04] & [0.04] & [0.00] & [0.17] & [0.04] & [0.03] & [0.02] & [0.08] & [0.00] \\
%Lack of excitement to learn (difficulties in primary school) & 0.02 & 0.07 & 0.02 & 0.00 & 0.00 & 0.03 & 0.05 & 0.06 & 0.00 & 0.33** & 0.09** & 0.05 & 0.04 & 0.08 & 0.00 \\
% & [0.01] & [0.04] & [0.02] & [0.00] & [0.00] & [0.01] & [0.03] & [0.03] & [0.00] & [0.21] & [0.03] & [0.03] & [0.02] & [0.08] & [0.00] \\
%Ability to obey rules and directions (difficulties in primary school) & 0.10 & 0.11 & 0.08 & 0.00 & 0.50 & 0.10 & 0.14 & 0.16 & 0.11 & 0.00 & 0.10 & 0.12 & 0.05* & 0.00 & 0.00 \\
% & [0.02] & [0.05] & [0.03] & [0.00] & [0.50] & [0.02] & [0.05] & [0.04] & [0.11] & [0.00] & [0.03] & [0.05] & [0.02] & [0.00] & [0.00] \\
%Fussy eater (difficulties in primary school) & 0.07 & 0.09 & 0.11 & 0.40** & 0.00 & 0.06 & 0.07 & 0.16** & 0.11 & 0.33* & 0.09 & 0.07 & 0.09 & 0.00 & 0.00 \\
% & [0.02] & [0.04] & [0.03] & [0.24] & [0.00] & [0.02] & [0.04] & [0.04] & [0.11] & [0.21] & [0.03] & [0.04] & [0.02] & [0.00] & [0.00] \\
%Difficulties encountered when starting primary school & 4.25 & 4.22 & 4.04* & 4.60** & 4.00 & 4.25 & 4.30 & 3.95** & 4.67 & 3.00*** & 3.89** & 4.53 & 4.49* & 4.42 & 5.00 \\
% & [0.11] & [0.20] & [0.16] & [0.24] & [1.00] & [0.12] & [0.20] & [0.18] & [0.24] & [0.63] & [0.18] & [0.16] & [0.10] & [0.40] & [0.00] \\
\hline

% it contains the notes, assuming they are the same for all the tables.
\end{tabular}

\end{adjustbox}
\raggedright{
\footnotesize{Average of baseline characteristcs, by city and type of child-care attended. Standard errors of means in brackets. Test for difference in means between each column and the first column (Reggio Municipal, the treatment group) was performed; *** significant difference at 1\%, ** significant difference at 5\%, * significant difference at 10\%. Source: authors calculation using survey data.}
}
\end{table}
  
%\end{table}
%
%%% adolescents, preschool
\begin{table}[H] \label{tab:outcomes-adol-materna}
\caption{Outcomes of interest by preschool type, adolescents (age 18)}
% this is the top part of the tables that display the summary of the baseline characteristics, by city and child-care type
\centering
\begin{adjustbox}{width=1.2\textwidth,center=\textwidth}
\small
\begin{tabular}{m{4.0cm} ccccccccccccccc}
\hline \hline 
 & Reggio & Reggio & Reggio & Reggio & Reggio & Parma & Parma & Parma & Parma & Parma & Padova & Padova & Padova & Padova & Padova \\
 & Municipal & State & Religious & Private & Not Attended & Municipal & State & Religious & Private & Not Attended & Municipal & State & Religious & Private & Not Attended \\

\hline 
  
%SDQ score (mom rep.) & 8.16 & 6.64* & 8.70 & 9.33 & 8.86 & 7.86 & 7.95 & 8.73 & 8.17 & 9.75 & 7.68 & 9.02 & 8.32 & 8.17 & . \\
% & [0.33] & [0.58] & [0.53] & [1.43] & [1.40] & [0.39] & [0.60] & [0.54] & [2.27] & [2.32] & [0.34] & [0.60] & [0.37] & [1.45] & . \\
SDQ score & 9.82 & 7.73** & 10.77* & 11.00 & 7.43 & 9.11 & 8.00** & 9.53 & 8.67 & 8.75 & 9.56 & 9.00 & 10.18 & 7.50 & . \\
(self rep.)  & [0.37] & [0.79] & [0.47] & [1.32] & [1.63] & [0.39] & [0.68] & [0.47] & [1.50] & [0.85] & [0.47] & [0.55] & [0.43] & [1.26] & . \\
Depression score & 22.46 & 20.71 & 24.15** & 23.33 & 19.14 & 22.29 & 20.60 & 22.81 & 20.00 & 20.75 & 21.75 & 19.32*** & 21.77 & 19.67 & . \\
(CESD)  & [0.51] & [1.23] & [0.69] & [1.58] & [1.50] & [0.48] & [0.74] & [0.58] & [1.55] & [2.66] & [0.61] & [0.78] & [0.55] & [1.98] & . \\
Respondent health & 0.74 & 0.82 & 0.66 & 0.83 & 0.71 & 0.62** & 0.37*** & 0.63 & 1.00 & 0.75 & 0.68 & 0.85 & 0.75 & 0.67 & . \\
is good (\%)  & [0.03] & [0.08] & [0.05] & [0.17] & [0.18] & [0.05] & [0.07] & [0.05] & [0.00] & [0.25] & [0.05] & [0.05] & [0.04] & [0.21] & . \\
%Child health is good (\%) - mom report & 0.69 & 0.64 & 0.66 & 0.33* & 0.71 & 0.49*** & 0.40*** & 0.63 & 0.83 & 1.00 & 0.65 & 0.83* & 0.69 & 0.67 & . \\
% & [0.04] & [0.10] & [0.05] & [0.21] & [0.18] & [0.05] & [0.08] & [0.05] & [0.17] & [0.00] & [0.05] & [0.06] & [0.04] & [0.21] & . \\
%Bothered by migrants (\%) & 0.26 & 0.23 & 0.37* & 0.67** & 0.14 & 0.23 & 0.14 & 0.28 & 0.17 & 0.75* & 0.24 & 0.18 & 0.32 & 0.33 & 0.00 \\
% & [0.03] & [0.09] & [0.05] & [0.21] & [0.14] & [0.04] & [0.05] & [0.05] & [0.17] & [0.25] & [0.05] & [0.06] & [0.04] & [0.21] & . \\
%Child likes school (\%) & 0.73 & 0.85 & 0.72 & 0.33* & 0.60 & 0.68 & 0.79 & 0.62 & 0.50 & 1.00 & 0.69 & 0.68 & 0.71 & 0.83 & 0.00 \\
% & [0.03] & [0.08] & [0.05] & [0.21] & [0.24] & [0.04] & [0.06] & [0.05] & [0.22] & [0.00] & [0.05] & [0.07] & [0.04] & [0.17] & . \\
%Child likes math (\%) & 0.51 & 0.80** & 0.46 & 0.67 & 0.40 & 0.62* & 0.67* & 0.59 & 0.33 & 0.50 & 0.57 & 0.42 & 0.58 & 1.00 & 0.00 \\
% & [0.04] & [0.09] & [0.05] & [0.21] & [0.24] & [0.05] & [0.07] & [0.06] & [0.21] & [0.29] & [0.05] & [0.08] & [0.04] & [0.00] & . \\
%Child likes reading/italian (\%) & 0.71 & 0.65 & 0.66 & 0.50 & 0.80 & 0.80* & 0.81 & 0.73 & 0.67 & 0.50 & 0.63 & 0.67 & 0.62 & 0.50 & 1.00 \\
% & [0.04] & [0.11] & [0.05] & [0.22] & [0.20] & [0.04] & [0.06] & [0.05] & [0.21] & [0.29] & [0.05] & [0.07] & [0.04] & [0.22] & . \\
%Ability to sit still in a group when asked (difficulties in primary school) & 0.07 & 0.00 & 0.06 & 0.17 & 0.14 & 0.11 & 0.02 & 0.13* & 0.17 & 0.25 & 0.08 & 0.17** & 0.08 & 0.00 & 0.00 \\
% & [0.02] & [0.00] & [0.02] & [0.17] & [0.14] & [0.03] & [0.02] & [0.04] & [0.17] & [0.25] & [0.03] & [0.06] & [0.02] & [0.00] & . \\
%Lack of excitement to learn (difficulties in primary school) & 0.03 & 0.05 & 0.04 & 0.00 & 0.29** & 0.05 & 0.00 & 0.00 & 0.00 & 0.25 & 0.03 & 0.09 & 0.04 & 0.17 & 0.00 \\
% & [0.01] & [0.05] & [0.02] & [0.00] & [0.18] & [0.02] & [0.00] & [0.00] & [0.00] & [0.25] & [0.02] & [0.04] & [0.02] & [0.17] & . \\
%Ability to obey rules and directions (difficulties in primary school) & 0.04 & 0.05 & 0.04 & 0.00 & 0.14 & 0.08 & 0.09 & 0.09 & 0.00 & 0.00 & 0.06 & 0.11 & 0.05 & 0.00 & 0.00 \\
% & [0.02] & [0.05] & [0.02] & [0.00] & [0.14] & [0.02] & [0.04] & [0.03] & [0.00] & [0.00] & [0.03] & [0.05] & [0.02] & [0.00] & . \\
%Fussy eater (difficulties in primary school) & 0.07 & 0.05 & 0.08 & 0.33* & 0.00 & 0.15** & 0.05 & 0.11 & 0.00 & 0.00 & 0.05 & 0.17** & 0.07 & 0.17 & 0.00 \\
% & [0.02] & [0.05] & [0.03] & [0.21] & [0.00] & [0.03] & [0.03] & [0.03] & [0.00] & [0.00] & [0.02] & [0.06] & [0.02] & [0.17] & . \\
%Difficulties encountered when starting primary school & 4.57 & 4.82 & 4.52 & 4.00* & 3.57** & 4.24* & 4.67 & 4.34 & 4.33 & 3.25* & 4.57 & 3.98** & 4.46 & 4.33 & 5.00 \\
% & [0.09] & [0.14] & [0.11] & [0.63] & [0.69] & [0.13] & [0.13] & [0.15] & [0.67] & [1.03] & [0.12] & [0.23] & [0.11] & [0.49] & . \\
\hline

% it contains the notes, assuming they are the same for all the tables.
\end{tabular}

\end{adjustbox}
\raggedright{
\footnotesize{Average of baseline characteristcs, by city and type of child-care attended. Standard errors of means in brackets. Test for difference in means between each column and the first column (Reggio Municipal, the treatment group) was performed; *** significant difference at 1\%, ** significant difference at 5\%, * significant difference at 10\%. Source: authors calculation using survey data.}
}
\end{table}
  
%\end{table}
%
%%% adults, preschool
\begin{table}[H] \label{tab:outcomes-adult-materna}
\caption{Outcomes of interest by preschool type, adults (age 30-50)}
% this is the top part of the tables that display the summary of the baseline characteristics, by city and child-care type
\centering
\begin{adjustbox}{width=1.2\textwidth,center=\textwidth}
\small
\begin{tabular}{m{4.0cm} ccccccccccccccc}
\hline \hline 
 & Reggio & Reggio & Reggio & Reggio & Reggio & Parma & Parma & Parma & Parma & Parma & Padova & Padova & Padova & Padova & Padova \\
 & Municipal & State & Religious & Private & Not Attended & Municipal & State & Religious & Private & Not Attended & Municipal & State & Religious & Private & Not Attended \\

\hline 
  
Depression score (CESD) & 21.25 & 22.50 & 20.77 & 19.29 & 22.96*** & 20.50 & 18.96*** & 20.04 & 20.83 & 22.21*** & 22.11 & 25.20*** & 20.79 & 20.67 & 22.07* \\
 & [0.34] & [0.88] & [0.50] & [1.49] & [0.33] & [0.47] & [0.66] & [0.48] & [2.99] & [0.31] & [0.70] & [0.81] & [0.31] & [2.96] & [0.46] \\
Respondent health is good (\%) & 0.88 & 0.81 & 0.75*** & 0.75 & 0.57*** & 0.54*** & 0.48*** & 0.68*** & 0.33*** & 0.50*** & 0.59*** & 0.80 & 0.53*** & 0.50 & 0.47*** \\
 & [0.02] & [0.05] & [0.04] & [0.16] & [0.03] & [0.04] & [0.05] & [0.04] & [0.21] & [0.03] & [0.06] & [0.06] & [0.03] & [0.50] & [0.04] \\
Bothered by migrants (\%) & 0.12 & 0.16 & 0.25*** & 0.14 & 0.23*** & 0.30*** & 0.24*** & 0.30*** & 0.17 & 0.16 & 0.50*** & 0.32*** & 0.37*** & 0.33 & 0.34*** \\
 & [0.02] & [0.05] & [0.04] & [0.14] & [0.03] & [0.04] & [0.05] & [0.04] & [0.17] & [0.02] & [0.06] & [0.07] & [0.03] & [0.33] & [0.04] \\
\hline

% it contains the notes, assuming they are the same for all the tables.
\end{tabular}

\end{adjustbox}
\raggedright{
\footnotesize{Average of baseline characteristcs, by city and type of child-care attended. Standard errors of means in brackets. Test for difference in means between each column and the first column (Reggio Municipal, the treatment group) was performed; *** significant difference at 1\%, ** significant difference at 5\%, * significant difference at 10\%. Source: authors calculation using survey data.}
}
\end{table}
  
%\end{table}

%%%%%%%%%%%%
Finally, table (\ref{tab:outcomes-list}) compares the outcomes available in the our survey data with the ones available in ABC and Perry.
\begin{table}[htbp]\label{tab:outcomes-list}
\begin{center}

	\caption{Outcome Variables in Reggio and Other Early Childhood Studies}
\begin{tabular}{lllll}
\toprule
Outcome & Reggio & ABC & CARE & Perry \\
\midrule
\textbf{Cognitive} & & & & \\
\quad IQ & 6, 18, 32, 43, 54-60 & 6, 15 & 6 & 6, 14 \\
\textbf{Schooling} & & & & \\
\quad HS Graduation & 32, 43, 54-60 & 30 & 30 & 27, 40 \\
\quad Ever Suspended & 18 & & & 27 \\
\textbf{Socio-emotional} & & & & \\
\quad Strengths and Difficulties & 6, 18, 32, 43, 54-60 & & & \\
\quad Depression &18, 32, 43, 54-60 & 15, 21, 34 & 21, 34 & \\
\textbf{Health} & & & & \\
\quad Number of Cigarettes (per day) & 18, 32, 43, 54-60 & 21 & 21 & 27 \\
\quad BMI & 6, 32, 43, 54-60 & 34 & 34 & 40 \\
\quad Health Problems & 32, 43, 54-60 & 34 & 34 & 27 \\
\quad Ever Tried Drugs & 32, 43, 54-60 & 12, 15, 21, 30 & 12, 15, 21, 30 & 27 \\
\bottomrule
\end{tabular}
%
%
\end{center}
\footnotesize 
Notes: The columns for each dataset specify the ages in years at which the variables are available. IQ is measured by Raven's Progressive Matrices for Reggio, the Wechsler Intelligence Scale for Children (WISC) for ABC and CARE, and WISC and Stanford Binet for Perry. The ``Health Problems" item in Reggio, ABC, and CARE corresponds to the number of days in the past 30 days the subject was sick. In Perry, this variable is the number of days sick in the past year. ``Strengths and Difficulties" is the Strengths and Difficulties Questionnaire administered to adolescent and adult subjects in Reggio. ``Depression" is the Depression Scale in Reggio. In ABC, depression is measured from the Achenbach Youth Report at age 15. In ABC and CARE, depression is measured using the Brief Symptom Inventory at age 21 and is self-reported at age 34.
\end{table}

%\todo[backgroundcolor=orange!30,size=\tiny]{Tables with raw differences across cities and preschool types}


\section{References}

Bennett, J. (2013) Early childhood curriculum for children from low-income
and immigrant backgrounds. Paper presented at the second meeting of the
Transatlantic Forum on Inclusive Early Years

held in New York, 10-12 July 2013.

Bennett, J., Gordon, J. and Edelmann, 2013. Early childhood education and
care (ECEC) for children from disadvantaged backgrounds: findings from a
European literature review and two case

studies. Studycommissioned by the Directorate General for Education and
Culture. Retrieved from:
http://ec.europa.eu/education/more-information/doc/ecec/report\_en.pdf

Bion, W. R. (1962). Learning from Experience, London: William Heinemann.
[Reprinted London: Karnac Books

Brilli Y, Del Boca D. and Chiara Pronzato"Does the availability of child
care impact child outcomes?" Review of the Economics of the Household March
, 2016

Del Boca D. Pronzato C. and Sorrenti G. "When rationing plays a role" 
CESIFO Economic Studies Summer 2016

De Henau, J., D. Meulders, and S. O'Dorchai (2008). Making time for working
parents: comparing public childcare provision. In D. del Boca and C. Wetzels
(Eds.), Social Policies, Labour Markets

and Motherhood, Chapter 2, pp. 28--51. Cambridge: Cambridge University Press

Duncan G., Li, W., Farkas, G. J., Burchinal, M. R., \& Vandell, D. L. (2012)
Timing of High-Quality Child Care and Cognitive, Language, and Preacademic
Development. Developmental Psychology

Edwards, C.; Gandini, L.; Forman, G., eds. "The Hundred Languages of
Children: The Reggio Emilia Approach to Early Childhood Education". Norwood,
NJ: Ablex Publishing Corporation. pp.

19--37.1993

Felfe, C., and R. Lalive. "Does Early Child Care Help or Hurt Children's
Development"? IZA Discussion Paper N. 8484, (2014). Institute for the Study
of Labor.

Gandini, L. (1993). "Fundamentals of the Reggio Emilia Approach to Early
Childhood Education". Young Children 49 (1): 4--8.

Hewett, Valarie (2001)."Examining the Reggio Emilia approach to early
childhood education". Early Childhood Education Journal, 29, 95-100.

Lazzari, A., and Vandenbroeck, M. (2012). Literature Review of the
Participation of Disadvantaged Children and families in ECEC Services in
Europe. In J. Bennett (Ed.), Early childhood education 

and care (ECEC) for children from disadvantaged backgrounds Brussels

Love J., Harrison L, Sagi-Schwartz A, van IJzendoorn MH, Ross C, Ungerer JA,
Raikes H, Brady-Smith C, Boller K, Brooks-Gunn J, Constantine J, Kisker EE,
Paulsell D, Chazan-Cohen

"Child care quality matters: how conclusions may vary with context"Child
Dev. 2003 Jul-Aug;74(4):1021-33

Marcon R 1999 "Differential Impact of Preschool Models on Development and
Early Leaming of Inner-City Children: A Three-Cohort Study Developmental
Psychology"

1999, Vol. 35, No. 2, 358-375

Miller, L. B., and Bizzell, R. P. (1984) ."Long-term effects of four
preschool programs: Ninth-and-tenth-grader results". Child Development, 55,
1570-1587

Malaguzzi, L. (1993). History, ideas, and basic philosophy: An interview
with Lella Gandini. In Edwards C., Gandini, and G. Forman (Eds.), The
hundred languages of children: The Reggio Emilia

approach..Norwood, NJ: Ablex Publishing Corporation. pp. 19--37.

Rinaldi. C. (2005) In dialogue with Reggio Emilia - Listening, researching
and learning. Routledge 

Rankin, B. (2004, October). "The Importance of Intentional Socialization
Among Children in Small Groups: A Conversation with Loris Malaguzzi." Early
Childhood Education 32 (2), 81--85.

Schweinhart, L. J., Weikart, D. P., and Larner, M. B. (1986). Consequences
of three preschool curriculum models through age 15. Early Childhood
Research Quarterly, 1, 15-45.

Winnicott D. (1965) The Maturational Processes and the Facilitating
Environment: Studies in the Theory of Emotional Development" International
Universities Press, 1965

\end{document}
