\documentclass[12pt]{article}
\usepackage[top=1in, bottom=1in, left=1in, right=1in]{geometry}
\parindent 22pt

\usepackage{adjustbox}
\usepackage{amsmath}
\usepackage{amssymb}
\usepackage{array}
\usepackage{booktabs}
\usepackage{fancyhdr}
\usepackage{float}
\usepackage{graphicx}
\usepackage[colorlinks=true,linkcolor=blue,urlcolor=blue,anchorcolor=blue,citecolor=blue]{hyperref}
\usepackage{lscape}
\usepackage{multirow}
\usepackage{natbib}
\usepackage{setspace}
\usepackage{tabularx}
\usepackage[colorinlistoftodos,linecolor=black]{todonotes}
\usepackage{appendix}
\usepackage{pgffor}

\newcommand{\indep}{\rotatebox[origin=c]{90}{$\models$}}
\newcolumntype{L}[1]{>{\raggedright\arraybackslash}p{#1}}
\newcolumntype{C}[1]{>{\centering\arraybackslash}p{#1}}
\newcolumntype{R}[1]{>{\raggedleft\arraybackslash}p{#1}}

\begin{document}

\title{Analysis of the Reggio Children Approach}
\author{%Reggio Team
Pietro Biroli,
Daniela Del Boca,
and Chiara Pronzato %\thanks{%
%Pietro Biroli is Assistant Professor of Economics at the University of Zurich. 
%Daniela Del Boca is Professor of Economics at the University of Turin, Italy. 
%Chiara Pronzato is Assistant Professor of Economics at the University of Turin, Italy. 
%We thank Chiara Baldelli and Alessandro Mattioli for outstanding research assistance in the collection of the administrative data in the city of Reggio Emilia, and Karl Schulz for assistance in the analysis of the survey data.
%We acknowledge the generous support of the Jacobs Foundation. The views expressed in this paper are those of the authors and not necessarily those of the funder named here.
%We thank Anna Ziff and Jessica Koh for extremely skillfull assistance in the analysis of the survey data.
%}\\[2mm]
}
\date{[VERY PRELIMINARY DRAFT] \\ Original version: January 22, 2016 \\ Current version: \today}
\maketitle 

%\listoftodos
\doublespacing

%\section{Summary}
This document describes the evaluation of the Reggio Children Approach to Early Childhood Education (for convenience, referred to as RA). This is a unique natural experiment which has been in place for fifty years in the city of
Reggio Emilia, Italy. Here a universal high-quality early child care system has developed a different
vision of the child – as an individual with rights and potential. The Reggio Children Approach has received
world-wide recognition and has been emulated in different countries and in a variety of settings,\footnote{The official \href{http://www.reggiochildren.it/network/?lang=en}{Reggio Children International Network} is present in 33 countries worldwide. Many other preschools around the world are ``inspired'' by the Reggio Children Approach but they are not officially part of these network.} but it has never been evaluated. 

The RA preschool system was introduced and grew non-randomly over the course of several decades. Therefore, the evaluation strategy has to deal with the lack of a well-defined control group, variation of treatment over time, and potential spillover effects. 

This round of analysis evaluates RA tackling these challenges. We consider several groupings of controls to account for differences by region, data collection, socioeconomic factors, and most importantly type of preschool chosen by the caregiver. %Observable parental characteristics, such as family religiosity and proximity to grandparents, are used to understand and control for selection into different types of early education. 
Different model specification allow for comparison with various control groups, allowing for a more nuanced understanding of the effects of RA, including the effects both within and between regions and in relation to other types of early education.

In general, we find \ldots \todo[backgroundcolor=orange!30,size=\tiny]{Summary of results}

Section \ref{sec:background} gives a description of RA and the cities involved with the data collection. More detail about this data collection, including sample structure and survey design, is given in Section \ref{sec:data}. The analysis is discussed in Section \ref{sec:analysis} with results presented in Section \ref{sec:OLS}.
\section{Summary}
This document describes the evaluation of the Reggio Children Approach to Early Childhood Education (for convenience, referred to as RA). This is a unique natural experiment which has been in place for fifty years in the city of Reggio Emilia, Italy. Here a universal high-quality early child care system has developed a different vision of the child – as an individual with rights and potential. The Reggio Children Approach has received world-wide recognition and has been emulated in different countries and in a variety of settings,\footnote{The official \href{http://www.reggiochildren.it/network/?lang=en}{Reggio Children International Network} is present in 33 countries worldwide. Many other preschools around the world are ``inspired'' by the Reggio Children Approach but they are not officially part of these network.} but it has never been evaluated. 

The RA preschool system was introduced and grew non-randomly over the course of several decades. Therefore, the evaluation strategy has to deal with the lack of a well-defined control group, variation of treatment over time, and potential spillover effects. 

This round of analysis evaluates RA tackling these challenges. We consider several groupings of controls to account for differences by region, data collection, socioeconomic factors, and most importantly type of preschool chosen by the caregiver. %Observable parental characteristics, such as family religiosity and proximity to grandparents, are used to understand and control for selection into different types of early education. 
Different model specification allow for comparison with various control groups, allowing for a more nuanced understanding of the effects of RA, including the effects both within and between regions and in relation to other types of early education.

%In general, we find \ldots \todo[backgroundcolor=orange!30,size=\tiny]{Summary of results}

A brief description of the RA and the cities involved with the data collection is provided in Section \ref{sec:background}. More detail about this data collection, including sample structure and survey design, is given in Section \ref{sec:data}. A simple model of school selection is sketched in Section \ref{sec:model}. The identification stragy and the empirical analysis are discussed in Section \ref{sec:analysis}.

%\section{Background}
\label{sec:background}

% Quick description of Reggio Children Approach history
This section briefly describes the program and the cities in which the data were collected. For a more complete discussion of the Italian early childhood landscape, the educational philosophy and the research design, see \citet{biroli2015evaluating}. The Reggio Children Approach to early childhood education is a community effort of public investment in high-quality early childhood education spearheaded and led by the pedagogist Loris Malaguzzi (1920-1994), whose foresight has been the main influence on the approach. The system evolved slowly over the years from a parent cooperative in the outskirts of Reggio Emilia into a structured municipal system of 26 infant-toddler centers and 30 preschools. Building on existing pedagogical models, this educational project is centered around the interaction between children, families, teachers, and the community (\cite{Malaguzzi1993}).

The Reggio Children Approach includes infant-toddler centers (ages 0-2) and preschools (ages 3-5), both of which are for children before they enter primary school at age 6. The first Reggio infant-toddler center was established in 1971, while the first Reggio preschool was established even earlier, in 1963. 

% Reggio Emilia vs. Parma vs. Padova
In addition to collecting data from individuals in Reggio Emilia, data were collected from individuals in Parma and Padova, two cities that share several features with Reggio Emilia. Table (\ref{tab:comparison}) lists some characteristics of the three cities. More information on the data collection is given in Section (\ref{sec:data}).

\begin{table}[htbp]
\begin{center}
\caption{Demographic Comparison of Reggio Emilia, Parma, and Padova}
\label{tab:comparison}
\begin{tabular}{lccc}
\toprule
& \multicolumn{3}{c}{City} \\
\cmidrule{2-4}
& Reggio Emilia & Parma & Padova \\
\midrule
Population (2013)* & 172,525 &  187,938 & 209,678 \\
Average per-capita income (2011 euros)**  & 25,226 & 28,437 & 29,915 \\
\bottomrule
\end{tabular}
\end{center}
\footnotesize Notes: *ISTAT, \url{http://www.demo.istat.it/}; **Finance Minister, taxable income for 2011.
\end{table}
\todo[backgroundcolor=orange!30,size=\tiny]{Add information about (updated) italian and immigrant births to table \ref{tab:comparison}. Look for values in 1960 (start of the program, baseline)}

% Preschool availability and take-up

% Different types of preschool
As in every Italian city, early educational experiences are divided into two main age categories. The first, infant-toddler centers (\textit{asilo}), is available for children aged 0 through 3 while the second, preschool (\textit{materna}), is for children aged 3 through 6. 

For each of these age groups, there are schools established and managed by different entities: municipal, state, private, and religious. Table \ref{tab:types} summarizes the types of schools available for each age group, highlighting that there are no state-run infant-toddler centers.

\begin{table}[htbp]
\begin{center}
\caption{Types of Schools}\label{tab:types}
\begin{tabular}{ccc}
\toprule
& Infant-Toddler & Preschool \\
\midrule
Municipal & \checkmark & \checkmark \\
State & & \checkmark \\
Private & \checkmark & \checkmark \\
Religious & \checkmark & \checkmark \\
\bottomrule
\end{tabular}
\end{center}
\end{table}

The schools that follow the Reggio Children Approach are the \textbf{municipal} infant-toddler centers and preschools in Reggio Emilia. All of the children who attended a municipal infant-toddler center or preschool in Reggio Emilia are considered part of the treated group, since they received the RA intervention. The other types of schools in Reggio Emilia, and all the schools in Parma and Padova (including the municipal schools) did not receive the RA intervention. However, we cannot rule out that also these other school types borrowed aspects from the RA philosophy of early childhood education, especially in the city of Reggio Emilia. A detailed interview with the managers of the religious preschools in Reggio Emilia showed that [...] \todo[backgroundcolor=orange!30,size=\tiny]{Together with Claudia and Moira, who run the RA schools, we are collecting more specific information available about the religious schools in Reggio. No information is available for state and private}
In Padova, influences from Guerra Frabboni's philosophy prevail \citep{Frabboni1999}, while in Parma, there is a mixture of influences.

\section{Background}
\label{sec:background}

% Quick description of Reggio Children Approach history
This section briefly describes the program and the cities in which the data were collected. For a more complete discussion of the Italian early childhood landscape, the educational philosophy and the research design, see \citet{biroli2015evaluating}. The Reggio Children Approach to early childhood education is a community effort of public investment in high-quality early childhood education spearheaded and led by the pedagogist Loris Malaguzzi (1920-1994), whose foresight has been the main influence on the approach. The system evolved slowly over the years from a parent cooperative in the outskirts of Reggio Emilia into a structured municipal system of 26 infant-toddler centers and 30 preschools. Building on existing pedagogical models, this educational project is centered around the interaction between children, families, teachers, and the community (\cite{Malaguzzi1993}).

The Reggio Children Approach includes infant-toddler centers (ages 0-2) and preschools (ages 3-5), both of which are for children before they enter primary school at age 6. The first Reggio infant-toddler center was established in 1971, while the first Reggio preschool was established even earlier, in 1963. 

% Reggio Emilia vs. Parma vs. Padova
In addition to collecting data from individuals in Reggio Emilia, data were collected from individuals in Parma and Padova, two cities that share several features with Reggio Emilia. Table (\ref{tab:comparison}) lists some characteristics of the three cities. More information on the data collection is given in Section (\ref{sec:data}).

\begin{table}[htbp]
\begin{center}
\caption{Demographic Comparison of Reggio Emilia, Parma, and Padova}
\label{tab:comparison}
\begin{tabular}{lccc}
\hline \hline
& \multicolumn{3}{c}{City} \\
\cmidrule{2-4}
& Reggio Emilia & Parma & Padova \\
\hline
Population (2013)* & 172,525 &  187,938 & 209,678 \\
Average per-capita income (2011 euros)**  & 25,226 & 28,437 & 29,915 \\
\hline
\end{tabular}
\end{center}
\footnotesize Notes: *ISTAT, \url{http://www.demo.istat.it/}; **Finance Minister, taxable income for 2011.
\end{table}
%\todo[backgroundcolor=orange!30,size=\tiny]{Add information about (updated) italian and immigrant births to table \ref{tab:comparison}. Look for values in 1960 (start of the program, baseline)}

% Preschool availability and take-up

% Different types of preschool
As in every Italian city, early educational experiences are divided into two main age categories. The first, infant-toddler centers (\textit{asilo nido}), is available for children aged 0 through 3 while the second, preschool (\textit{scuola materna/dell'infanzia}), is for children aged 3 through 6. 

For each of these age groups, there are schools established and managed by different entities: municipal, state, private, and religious. Table \ref{tab:types} summarizes the types of schools available for each age group, highlighting that there are no state-run infant-toddler centers.

\begin{table}[htbp]
\begin{center}
\caption{Types of Schools}\label{tab:types}
\begin{tabular}{ccc}
\hline \hline
& Infant-Toddler & Preschool \\
\hline
Municipal & \checkmark & \checkmark \\
State & & \checkmark \\
Private & \checkmark & \checkmark \\
Religious & \checkmark & \checkmark \\
\hline
\end{tabular}
\end{center}
\end{table}

The schools that follow the Reggio Children Approach are the \textbf{municipal} infant-toddler centers and preschools in Reggio Emilia. All of the children who attended a municipal infant-toddler center or preschool in Reggio Emilia are considered part of the treated group, since they received the RA intervention. The other types of schools in Reggio Emilia, and all the schools in Parma and Padova (including the municipal schools) did not receive the RA intervention. However, we cannot rule out that also these other school types borrowed aspects from the RA philosophy of early childhood education, especially in the city of Reggio Emilia. 
%A detailed interview with the managers of the religious preschools in Reggio Emilia showed that [...] \todo[backgroundcolor=orange!30,size=\tiny]{Together with Claudia and Moira, who run the RA schools, we are collecting more specific information available about the religious schools in Reggio. No information is available for state and private} 
In Padova, influences from Guerra Frabboni's philosophy prevail \citep{Frabboni1999}, while in Parma, there is a mixture of influences.

% ------------------------------ Section DATA ------------------------------------------%
%\section{Data}
\label{sec:data}

This section discusses the survey data which has been used for the analysis in this report.\footnote{Also administrative data from the RA preschool system was collected. Its description and more details about the survey data are contained in \citet{biroli2015evaluating}} To evaluate the impact of Reggio, individuals living in Reggio Emilia, Parma, and Padova since their first year of life were interviewed. Data were collected on five age cohorts, including three cohorts of adults, one cohort of adolescents, and one cohort of children in their first year of elementary school. 

The structure of the cohorts is described in Table \ref{tab:cohorts}.

\begin{table}[htbp]
\begin{center}
\caption{Cohort Structure}\label{tab:cohorts}
\begin{tabular}{ccc}
\toprule
Cohort & Years of Birth & Age at Interview \\
\midrule
I & 1954--1959 & 54--60 \\
II & 1969--1970 & 43 \\
III & 1980--1981 & 32 \\
IV & 1994 & 18 \\
V & 2006 & 6 \\
\bottomrule  
\end{tabular}
\end{center}
\end{table}

\subsection{Outcomes considered}\label{sec:outcomes}
A rich set of outcomes were collected during the survey. Individuals at different stages of their lifecycle were asked about family composition, fertility, labor force participation, income, schooling, cognitive ability, social and emotional skills, health and healthy habits, social capital, interpersonal ties, as well as attitudes on immigration and integration. 

We focus our attention on four different outcomes, in four different domains: the mother reports to the Strength and Difficulties Questionnaire (SDQ), a widely used and validated scale from 0 to 40 measuring behavioral problems for children and adolescents; the percentage of respondents reporting to have good or excellent health;\footnote{For children and adolescent, the respondent is again the mother} the raw score of the Center for Epidemiological Studies Depression Scale (CESD), a scale ranging from 10 to 50 and measuring self-reported depression for adolescents and adults; and finally the share of respondents reporting to be `very' or  `quite' bothered by the immigration into the city. Tables (\ref{tab:outcomes-child-asilo}) to (\ref{tab:outcomes-adult-asilo}) report the summary statistics of these outcome variables, disaggregated by city and type of \textit{infant-toddler center} attended. Similarly, Tables (\ref{tab:outcomes-child-materna}) to (\ref{tab:outcomes-adult-materna}) report the average of these same variables, this time focusing on type of \textit{preschool} attended. % We see that


%% children, infant-toddler
\begin{table}[H] \label{tab:outcomes-child-asilo}
\caption{Outcomes of interest by infant-toddler-center type, children (age 6)}
% this is the top part of the tables that display the summary of the baseline characteristics, by city and child-care type
\centering
\begin{adjustbox}{width=1.2\textwidth,center=\textwidth}
\small
\begin{tabular}{m{4.0cm} cccccccccccc}
\hline \hline 
 & Reggio & Reggio & Reggio & Reggio & Parma & Parma & Parma & Parma & Padova & Padova & Padova & Padova \\
 & Municipal & Religious & Private & Not Attended & Municipal & Religious & Private & Not Attended & Municipal & Religious & Private & Not Attended \\

\hline 

SDQ score (mom rep.)   &   8.36   &   8.19   &   7.60   &   9.53**   &   7.93   &   8.29   &   7.11*   &   8.30   &   9.07   &   9.08   &   8.56   &   8.49 \\
   &   [0.32]   &   [0.79]   &   [1.78]   &   [0.43]   &   [0.33]   &   [1.44]   &   [0.52]   &   [0.43]   &   [0.54]   &   [0.98]   &   [0.75]   &   [0.33] \\
Child health is good (\%) - mom report   &   0.65   &   0.74   &   0.80   &   0.71   &   0.60   &   0.86   &   0.67   &   0.64   &   0.79**   &   0.77   &   0.71   &   0.73 \\
   &   [0.04]   &   [0.09]   &   [0.20]   &   [0.04]   &   [0.04]   &   [0.14]   &   [0.08]   &   [0.05]   &   [0.05]   &   [0.08]   &   [0.07]   &   [0.04] \\
%Child likes school (\%)   &   0.71   &   0.70   &   0.80   &   0.59**   &   0.71   &   0.71   &   0.78   &   0.74   &   0.46***   &   0.46**   &   0.59   &   0.54*** \\
%   &   [0.04]   &   [0.09]   &   [0.20]   &   [0.05]   &   [0.04]   &   [0.18]   &   [0.07]   &   [0.04]   &   [0.06]   &   [0.10]   &   [0.08]   &   [0.04] \\
%Child likes math (\%)   &   0.61   &   0.81*   &   1.00   &   0.62   &   0.68   &   0.86   &   0.69   &   0.58   &   0.68   &   0.62   &   0.65   &   0.62 \\
%   &   [0.04]   &   [0.08]   &   [0.00]   &   [0.05]   &   [0.04]   &   [0.14]   &   [0.08]   &   [0.05]   &   [0.06]   &   [0.10]   &   [0.08]   &   [0.04] \\
%Child likes reading/italian (\%)   &   0.59   &   0.59   &   0.80   &   0.49   &   0.67   &   0.86   &   0.78*   &   0.62   &   0.60   &   0.50   &   0.57   &   0.44*** \\
%   &   [0.04]   &   [0.10]   &   [0.20]   &   [0.05]   &   [0.04]   &   [0.14]   &   [0.07]   &   [0.05]   &   [0.06]   &   [0.10]   &   [0.08]   &   [0.04] \\
%Ability to sit still in a group when asked (difficulties in primary school)   &   0.12   &   0.07   &   0.00   &   0.19   &   0.15   &   0.14   &   0.06   &   0.14   &   0.19   &   0.15   &   0.10   &   0.06* \\
%   &   [0.03]   &   [0.05]   &   [0.00]   &   [0.04]   &   [0.03]   &   [0.14]   &   [0.04]   &   [0.04]   &   [0.05]   &   [0.07]   &   [0.05]   &   [0.02] \\
%Lack of excitement to learn (difficulties in primary school)   &   0.02   &   0.04   &   0.00   &   0.05   &   0.05   &   0.00   &   0.00   &   0.06*   &   0.07*   &   0.04   &   0.02   &   0.06 \\
%   &   [0.01]   &   [0.04]   &   [0.00]   &   [0.02]   &   [0.02]   &   [0.00]   &   [0.00]   &   [0.02]   &   [0.03]   &   [0.04]   &   [0.02]   &   [0.02] \\
%Ability to obey rules and directions (difficulties in primary school)   &   0.10   &   0.04   &   0.00   &   0.11   &   0.15   &   0.00   &   0.06   &   0.09   &   0.13   &   0.08   &   0.05   &   0.05 \\
%   &   [0.02]   &   [0.04]   &   [0.00]   &   [0.03]   &   [0.03]   &   [0.00]   &   [0.04]   &   [0.03]   &   [0.04]   &   [0.05]   &   [0.03]   &   [0.02] \\
%Fussy eater (difficulties in primary school)   &   0.11   &   0.19   &   0.17   &   0.03**   &   0.05*   &   0.14   &   0.03   &   0.18*   &   0.09   &   0.12   &   0.02   &   0.08 \\
%   &   [0.02]   &   [0.08]   &   [0.17]   &   [0.02]   &   [0.02]   &   [0.14]   &   [0.03]   &   [0.04]   &   [0.03]   &   [0.06]   &   [0.02]   &   [0.02] \\
% Difficulties encountered when starting primary school   &   4.29   &   4.44   &   4.83   &   3.92   &   4.17   &   4.43   &   4.64   &   3.98***   &   3.93   &   4.23   &   4.44   &   4.49* \\
%   &   [0.11]   &   [0.22]   &   [0.17]   &   [0.15]   &   [0.12]   &   [0.57]   &   [0.17]   &   [0.15]   &   [0.20]   &   [0.30]   &   [0.20]   &   [0.10] \\
\hline

% it contains the notes, assuming they are the same for all the tables.
\end{tabular}

\end{adjustbox}
\raggedright{
\footnotesize{Average of baseline characteristcs, by city and type of child-care attended. Standard errors of means in brackets. Test for difference in means between each column and the first column (Reggio Municipal, the treatment group) was performed; *** significant difference at 1\%, ** significant difference at 5\%, * significant difference at 10\%. Source: authors calculation using survey data.}
}
\end{table}
  
%\end{table}

%% adolescents, infant-toddler
\begin{table}[H] \label{tab:outcomes-adol-asilo}
\caption{Outcomes of interest by infant-toddler-center type, adolescents (age 18)}
% this is the top part of the tables that display the summary of the baseline characteristics, by city and child-care type
\centering
\begin{adjustbox}{width=1.2\textwidth,center=\textwidth}
\small
\begin{tabular}{m{4.0cm} cccccccccccc}
\hline \hline 
 & Reggio & Reggio & Reggio & Reggio & Parma & Parma & Parma & Parma & Padova & Padova & Padova & Padova \\
 & Municipal & Religious & Private & Not Attended & Municipal & Religious & Private & Not Attended & Municipal & Religious & Private & Not Attended \\

\hline 
  
%SDQ score (mom rep.)  &  7.95  &  8.89  &  9.67  &  8.47  &  8.02  &  9.70  &  9.38  &  8.01  &  8.15  &  6.38  & . &  8.32 \\
%  &  [0.36]  &  [2.64]  &  [2.33]  &  [0.37]  &  [0.42]  &  [2.39]  &  [1.45]  &  [0.38]  &  [0.57]  &  [0.91]  & .&  [0.26] \\
SDQ score &  9.96  &  11.89  &  7.67  &  10.04  &  9.11  &  9.10  &  9.46  &  8.95  &  10.52  &  7.75 & . &  9.54 \\
(self rep.)  &  [0.40]  &  [2.19]  &  [0.88]  &  [0.39]  &  [0.44]  &  [1.89]  &  [1.07]  &  [0.37]  &  [0.67]  &  [1.46]  & .&  [0.30] \\
Depression score &  23.08  &  23.78  &  20.00  &  22.75  &  22.08  &  21.20  &  21.15  &  22.35  &  22.87  &  20.88  & . &  20.90*** \\
(CESD)    &  [0.51]  &  [2.96]  &  [4.58]  &  [0.58]  &  [0.54]  &  [1.31]  &  [1.35]  &  [0.45]  &  [0.85]  &  [2.06]   & . &  [0.40] \\
Respondent health &  0.76  &  0.67  &  1.00  &  0.66*  &  0.54***  &  0.50  &  0.85  &  0.60***  &  0.70  &  0.57   & . &  0.76 \\
is good (\%)    &  [0.03]  &  [0.17]  &  [0.00]  &  [0.04]  &  [0.05]  &  [0.17]  &  [0.10]  &  [0.04]  &  [0.06]  &  [0.20]   & . &  [0.03] \\
%Child health is good (\%) - mom report  &  0.70  &  0.78  &  1.00  &  0.62  &  0.45***  &  0.50  &  0.77  &  0.57**  &  0.62  &  0.75   & . &  0.72 \\
%  &  [0.04]  &  [0.15]  &  [0.00]  &  [0.04]  &  [0.05]  &  [0.17]  &  [0.12]  &  [0.04]  &  [0.06]  &  [0.16]   & . &  [0.03] \\
%Bothered by migrants (\%)  &  0.32  &  0.11  &  0.00  &  0.29  &  0.22  &  0.00  &  0.31  &  0.24  &  0.28  &  0.25   & . &  0.26 \\
%  &  [0.04]  &  [0.11]  &  [0.00]  &  [0.04]  &  [0.04]  &  [0.00]  &  [0.13]  &  [0.04]  &  [0.06]  &  [0.16]  & .  &  [0.03] \\
%Child likes school (\%)  &  0.74  &  0.56  &  0.67  &  0.72  &  0.69  &  0.44  &  0.62  &  0.70  &  0.70  &  0.62   & . &  0.70 \\
%  &  [0.04]  &  [0.18]  &  [0.33]  &  [0.04]  &  [0.05]  &  [0.18]  &  [0.14]  &  [0.04]  &  [0.06]  &  [0.18]   & . &  [0.03] \\
%Child likes math (\%)  &  0.51  &  0.44  &  0.67  &  0.53  &  0.69***  &  0.56  &  0.54  &  0.57  &  0.52  &  0.62   & . &  0.58 \\
%  &  [0.04]  &  [0.18]  &  [0.33]  &  [0.05]  &  [0.05]  &  [0.18]  &  [0.14]  &  [0.04]  &  [0.07]  &  [0.18]   & . &  [0.03] \\
%Child likes reading/italian (\%)  &  0.67  &  0.56  &  0.33  &  0.72  &  0.78  &  0.44  &  0.62  &  0.81**  &  0.60  &  0.25**   & . &  0.65 \\
%  &  [0.04]  &  [0.18]  &  [0.33]  &  [0.04]  &  [0.04]  &  [0.18]  &  [0.14]  &  [0.03]  &  [0.06]  &  [0.16]   & . &  [0.03] \\
%Ability to sit still in a group when asked (difficulties in primary school)  &  0.06  &  0.00  &  0.00  &  0.08  &  0.10  &  0.10  &  0.15  &  0.11  &  0.10  &  0.12   & . &  0.09 \\
%  &  [0.02]  &  [0.00]  &  [0.00]  &  [0.02]  &  [0.03]  &  [0.10]  &  [0.10]  &  [0.03]  &  [0.04]  &  [0.12]   & . &  [0.02] \\
%Lack of excitement to learn (difficulties in primary school)  &  0.03  &  0.00  &  0.00  &  0.05  &  0.04  &  0.00  &  0.00  &  0.02  &  0.07  &  0.00   & . &  0.04 \\
%  &  [0.01]  &  [0.00]  &  [0.00]  &  [0.02]  &  [0.02]  &  [0.00]  &  [0.00]  &  [0.01]  &  [0.03]  &  [0.00]   & . &  [0.01] \\
%Ability to obey rules and directions (difficulties in primary school)  &  0.04  &  0.00  &  0.00  &  0.06  &  0.07  &  0.00  &  0.15  &  0.07  &  0.08  &  0.00   & . &  0.05 \\
%  &  [0.02]  &  [0.00]  &  [0.00]  &  [0.02]  &  [0.03]  &  [0.00]  &  [0.10]  &  [0.02]  &  [0.04]  &  [0.00]   & . &  [0.02] \\
%Fussy eater (difficulties in primary school)  &  0.09  &  0.00  &  0.00  &  0.06  &  0.09  &  0.20  &  0.00  &  0.13  &  0.05  &  0.00   & . &  0.10 \\
%  &  [0.02]  &  [0.00]  &  [0.00]  &  [0.02]  &  [0.03]  &  [0.13]  &  [0.00]  &  [0.03]  &  [0.03]  &  [0.00]   & . &  [0.02] \\
% Difficulties encountered when starting primary school  &  4.55  &  5.00  &  5.00  &  4.46  &  4.35  &  4.40  &  4.38  &  4.34  &  4.39  &  4.50   & . &  4.42 \\
%  &  [0.09]  &  [0.00]  &  [0.00]  &  [0.11]  &  [0.13]  &  [0.40]  &  [0.42]  &  [0.11]  &  [0.17]  &  [0.50]   & . &  [0.09] \\
\hline

% it contains the notes, assuming they are the same for all the tables.
\end{tabular}

\end{adjustbox}
\raggedright{
\footnotesize{Average of baseline characteristcs, by city and type of child-care attended. Standard errors of means in brackets. Test for difference in means between each column and the first column (Reggio Municipal, the treatment group) was performed; *** significant difference at 1\%, ** significant difference at 5\%, * significant difference at 10\%. Source: authors calculation using survey data.}
}
\end{table}
  
%\end{table}

%% adults, infant-toddler
\begin{table}[H] \label{tab:outcomes-adult-asilo}
\caption{Outcomes of interest by infant-toddler-center type, adults (age 30-50)}
% this is the top part of the tables that display the summary of the baseline characteristics, by city and child-care type
\centering
\begin{adjustbox}{width=1.2\textwidth,center=\textwidth}
\small
\begin{tabular}{m{4.0cm} cccccccccccc}
\hline \hline 
 & Reggio & Reggio & Reggio & Reggio & Parma & Parma & Parma & Parma & Padova & Padova & Padova & Padova \\
 & Municipal & Religious & Private & Not Attended & Municipal & Religious & Private & Not Attended & Municipal & Religious & Private & Not Attended \\

\hline 
  
Depression score (CESD)  &  21.28  &  17.75  &  22.33  &  21.99  &  17.91***  &  19.73  &  19.82  &  21.35  &  26.57***  &  22.71  &  20.33  &  21.32 \\
  &  [0.55]  &  [0.75]  &  [4.33]  &  [0.23]  &  [0.55]  &  [1.59]  &  [1.59]  &  [0.24]  &  [0.88]  &  [1.44]  &  [2.33]  &  [0.24] \\
Respondent health is good (\%)  &  0.93  &  1.00  &  1.00  &  0.71***  &  0.29***  &  0.53***  &  0.27***  &  0.60***  &  0.63***  &  0.24***  &  1.00  &  0.54*** \\
  &  [0.03]  &  [0.00]  &  [0.00]  &  [0.02]  &  [0.05]  &  [0.13]  &  [0.14]  &  [0.02]  &  [0.08]  &  [0.11]  &  [0.00]  &  [0.02] \\
Bothered by migrants (\%)  &  0.07  &  0.25  &  0.00  &  0.21***  &  0.21**  &  0.33**  &  0.36**  &  0.23***  &  0.47***  &  0.61***  &  0.67**  &  0.35*** \\
  &  [0.03]  &  [0.25]  &  [0.00]  &  [0.02]  &  [0.05]  &  [0.13]  &  [0.15]  &  [0.02]  &  [0.08]  &  [0.12]  &  [0.33]  &  [0.02] \\
\hline

% it contains the notes, assuming they are the same for all the tables.
\end{tabular}

\end{adjustbox}
\raggedright{
\footnotesize{Average of baseline characteristcs, by city and type of child-care attended. Standard errors of means in brackets. Test for difference in means between each column and the first column (Reggio Municipal, the treatment group) was performed; *** significant difference at 1\%, ** significant difference at 5\%, * significant difference at 10\%. Source: authors calculation using survey data.}
}
\end{table}
  
%\end{table}

%%% children, preschool
\begin{table}[H] \label{tab:outcomes-child-materna}
\caption{Outcomes of interest by preschool type, children (age 6)}
% this is the top part of the tables that display the summary of the baseline characteristics, by city and child-care type
\centering
\begin{adjustbox}{width=1.2\textwidth,center=\textwidth}
\small
\begin{tabular}{m{4.0cm} ccccccccccccccc}
\hline \hline 
 & Reggio & Reggio & Reggio & Reggio & Reggio & Parma & Parma & Parma & Parma & Parma & Padova & Padova & Padova & Padova & Padova \\
 & Municipal & State & Religious & Private & Not Attended & Municipal & State & Religious & Private & Not Attended & Municipal & State & Religious & Private & Not Attended \\

\hline 
  
SDQ score (mom rep.) & 8.33 & 10.02* & 9.00 & 9.00 & 10.00 & 7.80 & 7.19* & 8.66 & 8.56 & 8.67 & 9.07 & 8.85 & 8.23 & 10.00 & 12.50* \\
 & [0.32] & [0.82] & [0.42] & [2.71] & [4.00] & [0.32] & [0.58] & [0.48] & [0.96] & [1.50] & [0.50] & [0.62] & [0.34] & [1.69] & [1.50] \\
Child health is good (\%) - mom report & 0.67 & 0.71 & 0.68 & 0.75 & 0.50 & 0.60 & 0.63 & 0.68 & 0.78 & 0.50 & 0.78 & 0.80 & 0.71 & 0.75 & 1.00 \\
 & [0.04] & [0.07] & [0.05] & [0.25] & [0.50] & [0.04] & [0.07] & [0.05] & [0.15] & [0.22] & [0.05] & [0.06] & [0.04] & [0.13] & [0.00] \\
%Child likes school (\%) & 0.69 & 0.60 & 0.64 & 0.75 & 0.50 & 0.72 & 0.77 & 0.77 & 0.67 & 0.50 & 0.39*** & 0.50** & 0.58* & 0.58 & 1.00 \\
% & [0.04] & [0.07] & [0.05] & [0.25] & [0.50] & [0.04] & [0.07] & [0.05] & [0.17] & [0.22] & [0.05] & [0.08] & [0.04] & [0.15] & [0.00] \\
%Child likes math (\%) & 0.61 & 0.71 & 0.67 & 0.75 & 0.00 & 0.68 & 0.65 & 0.68 & 0.56 & 0.00 & 0.69 & 0.60 & 0.62 & 0.55 & 1.00 \\
% & [0.04] & [0.07] & [0.05] & [0.25] & [0.00] & [0.04] & [0.07] & [0.05] & [0.18] & [0.00] & [0.05] & [0.08] & [0.04] & [0.16] & [0.00] \\
%Child likes reading/italian (\%) & 0.57 & 0.56 & 0.53 & 1.00 & 0.00 & 0.67* & 0.72* & 0.64 & 0.67 & 0.80 & 0.57 & 0.47 & 0.50 & 0.27* & 0.50 \\
% & [0.04] & [0.07] & [0.05] & [0.00] & [0.00] & [0.04] & [0.07] & [0.06] & [0.17] & [0.20] & [0.06] & [0.08] & [0.04] & [0.14] & [0.50] \\
%Ability to sit still in a group when asked (difficulties in primary school) & 0.13 & 0.09 & 0.17 & 0.00 & 0.00 & 0.14 & 0.09 & 0.16 & 0.00 & 0.17 & 0.20 & 0.03* & 0.08 & 0.08 & 0.00 \\
% & [0.03] & [0.04] & [0.04] & [0.00] & [0.00] & [0.03] & [0.04] & [0.04] & [0.00] & [0.17] & [0.04] & [0.03] & [0.02] & [0.08] & [0.00] \\
%Lack of excitement to learn (difficulties in primary school) & 0.02 & 0.07 & 0.02 & 0.00 & 0.00 & 0.03 & 0.05 & 0.06 & 0.00 & 0.33** & 0.09** & 0.05 & 0.04 & 0.08 & 0.00 \\
% & [0.01] & [0.04] & [0.02] & [0.00] & [0.00] & [0.01] & [0.03] & [0.03] & [0.00] & [0.21] & [0.03] & [0.03] & [0.02] & [0.08] & [0.00] \\
%Ability to obey rules and directions (difficulties in primary school) & 0.10 & 0.11 & 0.08 & 0.00 & 0.50 & 0.10 & 0.14 & 0.16 & 0.11 & 0.00 & 0.10 & 0.12 & 0.05* & 0.00 & 0.00 \\
% & [0.02] & [0.05] & [0.03] & [0.00] & [0.50] & [0.02] & [0.05] & [0.04] & [0.11] & [0.00] & [0.03] & [0.05] & [0.02] & [0.00] & [0.00] \\
%Fussy eater (difficulties in primary school) & 0.07 & 0.09 & 0.11 & 0.40** & 0.00 & 0.06 & 0.07 & 0.16** & 0.11 & 0.33* & 0.09 & 0.07 & 0.09 & 0.00 & 0.00 \\
% & [0.02] & [0.04] & [0.03] & [0.24] & [0.00] & [0.02] & [0.04] & [0.04] & [0.11] & [0.21] & [0.03] & [0.04] & [0.02] & [0.00] & [0.00] \\
%Difficulties encountered when starting primary school & 4.25 & 4.22 & 4.04* & 4.60** & 4.00 & 4.25 & 4.30 & 3.95** & 4.67 & 3.00*** & 3.89** & 4.53 & 4.49* & 4.42 & 5.00 \\
% & [0.11] & [0.20] & [0.16] & [0.24] & [1.00] & [0.12] & [0.20] & [0.18] & [0.24] & [0.63] & [0.18] & [0.16] & [0.10] & [0.40] & [0.00] \\
\hline

% it contains the notes, assuming they are the same for all the tables.
\end{tabular}

\end{adjustbox}
\raggedright{
\footnotesize{Average of baseline characteristcs, by city and type of child-care attended. Standard errors of means in brackets. Test for difference in means between each column and the first column (Reggio Municipal, the treatment group) was performed; *** significant difference at 1\%, ** significant difference at 5\%, * significant difference at 10\%. Source: authors calculation using survey data.}
}
\end{table}
  
%\end{table}
%
%%% adolescents, preschool
\begin{table}[H] \label{tab:outcomes-adol-materna}
\caption{Outcomes of interest by preschool type, adolescents (age 18)}
% this is the top part of the tables that display the summary of the baseline characteristics, by city and child-care type
\centering
\begin{adjustbox}{width=1.2\textwidth,center=\textwidth}
\small
\begin{tabular}{m{4.0cm} ccccccccccccccc}
\hline \hline 
 & Reggio & Reggio & Reggio & Reggio & Reggio & Parma & Parma & Parma & Parma & Parma & Padova & Padova & Padova & Padova & Padova \\
 & Municipal & State & Religious & Private & Not Attended & Municipal & State & Religious & Private & Not Attended & Municipal & State & Religious & Private & Not Attended \\

\hline 
  
%SDQ score (mom rep.) & 8.16 & 6.64* & 8.70 & 9.33 & 8.86 & 7.86 & 7.95 & 8.73 & 8.17 & 9.75 & 7.68 & 9.02 & 8.32 & 8.17 & . \\
% & [0.33] & [0.58] & [0.53] & [1.43] & [1.40] & [0.39] & [0.60] & [0.54] & [2.27] & [2.32] & [0.34] & [0.60] & [0.37] & [1.45] & . \\
SDQ score & 9.82 & 7.73** & 10.77* & 11.00 & 7.43 & 9.11 & 8.00** & 9.53 & 8.67 & 8.75 & 9.56 & 9.00 & 10.18 & 7.50 & . \\
(self rep.)  & [0.37] & [0.79] & [0.47] & [1.32] & [1.63] & [0.39] & [0.68] & [0.47] & [1.50] & [0.85] & [0.47] & [0.55] & [0.43] & [1.26] & . \\
Depression score & 22.46 & 20.71 & 24.15** & 23.33 & 19.14 & 22.29 & 20.60 & 22.81 & 20.00 & 20.75 & 21.75 & 19.32*** & 21.77 & 19.67 & . \\
(CESD)  & [0.51] & [1.23] & [0.69] & [1.58] & [1.50] & [0.48] & [0.74] & [0.58] & [1.55] & [2.66] & [0.61] & [0.78] & [0.55] & [1.98] & . \\
Respondent health & 0.74 & 0.82 & 0.66 & 0.83 & 0.71 & 0.62** & 0.37*** & 0.63 & 1.00 & 0.75 & 0.68 & 0.85 & 0.75 & 0.67 & . \\
is good (\%)  & [0.03] & [0.08] & [0.05] & [0.17] & [0.18] & [0.05] & [0.07] & [0.05] & [0.00] & [0.25] & [0.05] & [0.05] & [0.04] & [0.21] & . \\
%Child health is good (\%) - mom report & 0.69 & 0.64 & 0.66 & 0.33* & 0.71 & 0.49*** & 0.40*** & 0.63 & 0.83 & 1.00 & 0.65 & 0.83* & 0.69 & 0.67 & . \\
% & [0.04] & [0.10] & [0.05] & [0.21] & [0.18] & [0.05] & [0.08] & [0.05] & [0.17] & [0.00] & [0.05] & [0.06] & [0.04] & [0.21] & . \\
%Bothered by migrants (\%) & 0.26 & 0.23 & 0.37* & 0.67** & 0.14 & 0.23 & 0.14 & 0.28 & 0.17 & 0.75* & 0.24 & 0.18 & 0.32 & 0.33 & 0.00 \\
% & [0.03] & [0.09] & [0.05] & [0.21] & [0.14] & [0.04] & [0.05] & [0.05] & [0.17] & [0.25] & [0.05] & [0.06] & [0.04] & [0.21] & . \\
%Child likes school (\%) & 0.73 & 0.85 & 0.72 & 0.33* & 0.60 & 0.68 & 0.79 & 0.62 & 0.50 & 1.00 & 0.69 & 0.68 & 0.71 & 0.83 & 0.00 \\
% & [0.03] & [0.08] & [0.05] & [0.21] & [0.24] & [0.04] & [0.06] & [0.05] & [0.22] & [0.00] & [0.05] & [0.07] & [0.04] & [0.17] & . \\
%Child likes math (\%) & 0.51 & 0.80** & 0.46 & 0.67 & 0.40 & 0.62* & 0.67* & 0.59 & 0.33 & 0.50 & 0.57 & 0.42 & 0.58 & 1.00 & 0.00 \\
% & [0.04] & [0.09] & [0.05] & [0.21] & [0.24] & [0.05] & [0.07] & [0.06] & [0.21] & [0.29] & [0.05] & [0.08] & [0.04] & [0.00] & . \\
%Child likes reading/italian (\%) & 0.71 & 0.65 & 0.66 & 0.50 & 0.80 & 0.80* & 0.81 & 0.73 & 0.67 & 0.50 & 0.63 & 0.67 & 0.62 & 0.50 & 1.00 \\
% & [0.04] & [0.11] & [0.05] & [0.22] & [0.20] & [0.04] & [0.06] & [0.05] & [0.21] & [0.29] & [0.05] & [0.07] & [0.04] & [0.22] & . \\
%Ability to sit still in a group when asked (difficulties in primary school) & 0.07 & 0.00 & 0.06 & 0.17 & 0.14 & 0.11 & 0.02 & 0.13* & 0.17 & 0.25 & 0.08 & 0.17** & 0.08 & 0.00 & 0.00 \\
% & [0.02] & [0.00] & [0.02] & [0.17] & [0.14] & [0.03] & [0.02] & [0.04] & [0.17] & [0.25] & [0.03] & [0.06] & [0.02] & [0.00] & . \\
%Lack of excitement to learn (difficulties in primary school) & 0.03 & 0.05 & 0.04 & 0.00 & 0.29** & 0.05 & 0.00 & 0.00 & 0.00 & 0.25 & 0.03 & 0.09 & 0.04 & 0.17 & 0.00 \\
% & [0.01] & [0.05] & [0.02] & [0.00] & [0.18] & [0.02] & [0.00] & [0.00] & [0.00] & [0.25] & [0.02] & [0.04] & [0.02] & [0.17] & . \\
%Ability to obey rules and directions (difficulties in primary school) & 0.04 & 0.05 & 0.04 & 0.00 & 0.14 & 0.08 & 0.09 & 0.09 & 0.00 & 0.00 & 0.06 & 0.11 & 0.05 & 0.00 & 0.00 \\
% & [0.02] & [0.05] & [0.02] & [0.00] & [0.14] & [0.02] & [0.04] & [0.03] & [0.00] & [0.00] & [0.03] & [0.05] & [0.02] & [0.00] & . \\
%Fussy eater (difficulties in primary school) & 0.07 & 0.05 & 0.08 & 0.33* & 0.00 & 0.15** & 0.05 & 0.11 & 0.00 & 0.00 & 0.05 & 0.17** & 0.07 & 0.17 & 0.00 \\
% & [0.02] & [0.05] & [0.03] & [0.21] & [0.00] & [0.03] & [0.03] & [0.03] & [0.00] & [0.00] & [0.02] & [0.06] & [0.02] & [0.17] & . \\
%Difficulties encountered when starting primary school & 4.57 & 4.82 & 4.52 & 4.00* & 3.57** & 4.24* & 4.67 & 4.34 & 4.33 & 3.25* & 4.57 & 3.98** & 4.46 & 4.33 & 5.00 \\
% & [0.09] & [0.14] & [0.11] & [0.63] & [0.69] & [0.13] & [0.13] & [0.15] & [0.67] & [1.03] & [0.12] & [0.23] & [0.11] & [0.49] & . \\
\hline

% it contains the notes, assuming they are the same for all the tables.
\end{tabular}

\end{adjustbox}
\raggedright{
\footnotesize{Average of baseline characteristcs, by city and type of child-care attended. Standard errors of means in brackets. Test for difference in means between each column and the first column (Reggio Municipal, the treatment group) was performed; *** significant difference at 1\%, ** significant difference at 5\%, * significant difference at 10\%. Source: authors calculation using survey data.}
}
\end{table}
  
%\end{table}
%
%%% adults, preschool
\begin{table}[H] \label{tab:outcomes-adult-materna}
\caption{Outcomes of interest by preschool type, adults (age 30-50)}
% this is the top part of the tables that display the summary of the baseline characteristics, by city and child-care type
\centering
\begin{adjustbox}{width=1.2\textwidth,center=\textwidth}
\small
\begin{tabular}{m{4.0cm} ccccccccccccccc}
\hline \hline 
 & Reggio & Reggio & Reggio & Reggio & Reggio & Parma & Parma & Parma & Parma & Parma & Padova & Padova & Padova & Padova & Padova \\
 & Municipal & State & Religious & Private & Not Attended & Municipal & State & Religious & Private & Not Attended & Municipal & State & Religious & Private & Not Attended \\

\hline 
  
Depression score (CESD) & 21.25 & 22.50 & 20.77 & 19.29 & 22.96*** & 20.50 & 18.96*** & 20.04 & 20.83 & 22.21*** & 22.11 & 25.20*** & 20.79 & 20.67 & 22.07* \\
 & [0.34] & [0.88] & [0.50] & [1.49] & [0.33] & [0.47] & [0.66] & [0.48] & [2.99] & [0.31] & [0.70] & [0.81] & [0.31] & [2.96] & [0.46] \\
Respondent health is good (\%) & 0.88 & 0.81 & 0.75*** & 0.75 & 0.57*** & 0.54*** & 0.48*** & 0.68*** & 0.33*** & 0.50*** & 0.59*** & 0.80 & 0.53*** & 0.50 & 0.47*** \\
 & [0.02] & [0.05] & [0.04] & [0.16] & [0.03] & [0.04] & [0.05] & [0.04] & [0.21] & [0.03] & [0.06] & [0.06] & [0.03] & [0.50] & [0.04] \\
Bothered by migrants (\%) & 0.12 & 0.16 & 0.25*** & 0.14 & 0.23*** & 0.30*** & 0.24*** & 0.30*** & 0.17 & 0.16 & 0.50*** & 0.32*** & 0.37*** & 0.33 & 0.34*** \\
 & [0.02] & [0.05] & [0.04] & [0.14] & [0.03] & [0.04] & [0.05] & [0.04] & [0.17] & [0.02] & [0.06] & [0.07] & [0.03] & [0.33] & [0.04] \\
\hline

% it contains the notes, assuming they are the same for all the tables.
\end{tabular}

\end{adjustbox}
\raggedright{
\footnotesize{Average of baseline characteristcs, by city and type of child-care attended. Standard errors of means in brackets. Test for difference in means between each column and the first column (Reggio Municipal, the treatment group) was performed; *** significant difference at 1\%, ** significant difference at 5\%, * significant difference at 10\%. Source: authors calculation using survey data.}
}
\end{table}
  
%\end{table}

%%%%%%%%%%%%
Finally, table (\ref{tab:outcomes-list}) compares the outcomes available in the our survey data with the ones available in ABC and Perry.
\begin{table}[htbp]\label{tab:outcomes-list}
\begin{center}

	\caption{Outcome Variables in Reggio and Other Early Childhood Studies}
\begin{tabular}{lllll}
\toprule
Outcome & Reggio & ABC & CARE & Perry \\
\midrule
\textbf{Cognitive} & & & & \\
\quad IQ & 6, 18, 32, 43, 54-60 & 6, 15 & 6 & 6, 14 \\
\textbf{Schooling} & & & & \\
\quad HS Graduation & 32, 43, 54-60 & 30 & 30 & 27, 40 \\
\quad Ever Suspended & 18 & & & 27 \\
\textbf{Socio-emotional} & & & & \\
\quad Strengths and Difficulties & 6, 18, 32, 43, 54-60 & & & \\
\quad Depression &18, 32, 43, 54-60 & 15, 21, 34 & 21, 34 & \\
\textbf{Health} & & & & \\
\quad Number of Cigarettes (per day) & 18, 32, 43, 54-60 & 21 & 21 & 27 \\
\quad BMI & 6, 32, 43, 54-60 & 34 & 34 & 40 \\
\quad Health Problems & 32, 43, 54-60 & 34 & 34 & 27 \\
\quad Ever Tried Drugs & 32, 43, 54-60 & 12, 15, 21, 30 & 12, 15, 21, 30 & 27 \\
\bottomrule
\end{tabular}
%
%
\end{center}
\footnotesize 
Notes: The columns for each dataset specify the ages in years at which the variables are available. IQ is measured by Raven's Progressive Matrices for Reggio, the Wechsler Intelligence Scale for Children (WISC) for ABC and CARE, and WISC and Stanford Binet for Perry. The ``Health Problems" item in Reggio, ABC, and CARE corresponds to the number of days in the past 30 days the subject was sick. In Perry, this variable is the number of days sick in the past year. ``Strengths and Difficulties" is the Strengths and Difficulties Questionnaire administered to adolescent and adult subjects in Reggio. ``Depression" is the Depression Scale in Reggio. In ABC, depression is measured from the Achenbach Youth Report at age 15. In ABC and CARE, depression is measured using the Brief Symptom Inventory at age 21 and is self-reported at age 34.
\end{table}

%\todo[backgroundcolor=orange!30,size=\tiny]{Tables with raw differences across cities and preschool types}

\section{Data}
\label{sec:data}

This section discusses the survey data which has been used for the analysis in this report.\footnote{Also administrative data from the RA preschool system was collected. Its description and more details about the survey data are contained in \citet{biroli2015evaluating}} To evaluate the impact of Reggio, individuals living in Reggio Emilia, Parma, and Padova since their first year of life were interviewed. Data were collected on five age cohorts, including three cohorts of adults, one cohort of adolescents, and one cohort of children in their first year of elementary school. 

The structure of the cohorts is described in Table \ref{tab:cohorts}.

\begin{table}[htbp]
\begin{center}
\caption{Cohort Structure}\label{tab:cohorts}
\begin{tabular}{ccc}
\hline \hline
Cohort & Years of Birth & Age at Interview \\
\hline
I & 1954--1959 & 54--60 \\
II & 1969--1970 & 43 \\
III & 1980--1981 & 32 \\
IV & 1994 & 18 \\
V & 2006 & 6 \\
\hline  
\end{tabular}
\end{center}
\end{table}

In order to fix ideas on the identification strategy, we currently focus our attention only on the adolescents, the cohort for which we have the most wide array of information on both the caregiver and the respondent of interest.

Table (\ref{tab:sample-adol}) shows the number of adolescents interviewed in our sample who attended the different types of child-care in Reggio Emilia, Parma, and Padova. 

\begin{table}[htbp]
\begin{center}
\caption{Sample Size by City and School Type; Adolescent }\label{tab:sample-adol}
\begin{tabular}{l| ccc | ccc}
\hline \hline
\multicolumn{1}{r|}{City:} & Reggio & Parma & Padova & Reggio & Parma & Padova \\
\textit{School} & \multicolumn{3}{c|}{\textit{Infant-Toddler Center}}  & \multicolumn{3}{c}{\textit{Preschool}} \\
\hline 
Not Attended & 130 & 130 & 210   &   7 &   4 &   1 \\
Municipal 	 & 153 &  97 &  61   & 166 & 116 &  93 \\
State 		 &  .  &  .  &  .    &  22 &  43 &  47 \\
Religious 	 &   9 &  10 &   8   &  96 &  82 & 131 \\
Private 		 &   3 &  13 &   0   &   6 &   6 &   6 \\
\hline
\end{tabular}
\end{center}
\end{table}


\subsection{Outcomes considered}\label{sec:outcomes}
A rich set of outcomes were collected during the survey. Individuals at different stages of their lifecycle were asked about family composition, fertility, labor force participation, income, schooling, cognitive ability, social and emotional skills, health and healthy habits, social capital, interpersonal ties, as well as attitudes on immigration and integration. 

We focus our attention on three different outcomes, in two different domains: the adolescent report to the Strength and Difficulties Questionnaire (SDQ), a widely used and validated scale from 0 to 40 measuring behavioral problems for children and adolescents; the percentage of respondents reporting to have good or excellent health; the raw score of the Center for Epidemiological Studies Depression Scale (CESD), a scale ranging from 10 to 50 and measuring self-reported depression for adolescents and adults.%; and finally the share of respondents reporting to be `very' or  `quite' bothered by the immigration into the city. 
Tables (\ref{tab:outcomes-adol-asilo-reggio}) and (\ref{tab:outcomes-adol-asilo-pp}) report the summary statistics of these outcome variables, disaggregated by type of \textit{infant-toddler center} attended. Similarly, Tables (\ref{tab:outcomes-adol-materna-reggio}) and (\ref{tab:outcomes-adol-materna-pp}) report the average of these same variables, this time focusing on type of \textit{preschool} attended. 
% We see that

%% adolescents, infant-toddler
\begin{table}[H] 
\caption{Outcomes of interest by infant-toddler-center type, adolescents (age 18), Reggio}
\label{tab:outcomes-adol-asilo-reggio}

\centering
\begin{adjustbox}{width=0.9\textwidth,center=\textwidth}
\small
\begin{tabular}{m{4.0cm} ccccc}
\hline \hline 
& Reggio & Reggio & Reggio & Reggio \\
 & Municipal & Religious & Private & Not Attended \\
 \hline
SDQ score &  9.96  &  11.89  &  7.67  &  10.04  \\ 
(self rep.)  &  [0.40]  &  [2.19]  &  [0.88]  &  [0.39]  \\
Depression score &  23.08  &  23.78  &  20.00  &  22.75  \\
(CESD)    &  [0.51]  &  [2.96]  &  [4.58]  &  [0.58]  \\
Respondent health &  0.76  &  0.67  &  1.00  &  0.66* \\  
is good (\%)    &  [0.03]  &  [0.17]  &  [0.00]  &  [0.04]  \\
\hline
\end{tabular}

\end{adjustbox}
\raggedright{
\footnotesize{Average of baseline characteristics in Reggio, by type of child-care attended. Standard errors of means in brackets. Test for difference in means between each column and the first column (Reggio Municipal, the treatment group) was performed; *** significant difference at 1\%, ** significant difference at 5\%, * significant difference at 10\%. Source: authors calculation using survey data. CESD: Center for Epidemiologic Studies Depression Scale.}
}

\end{table}

\begin{table}[H]
\caption{Outcomes of interest by infant-toddler-center type, adolescents (age 18), Parma and Padova}
\label{tab:outcomes-adol-asilo-pp}
\centering
\begin{adjustbox}{width=1\textwidth,center=\textwidth}
\small
\begin{tabular}{m{4.0cm} cccccccc}
\hline \hline 
& Parma & Parma & Parma & Parma & Padova & Padova & Padova & Padova \\
 & Municipal & Religious & Private & Not Attended & Municipal & Religious & Private & Not Attended \\
 \hline
SDQ score &  9.11  &  9.10  &  9.46  &  8.95  &  10.52  &  7.75 & . &  9.54 \\
(self rep.) &  [0.44]  &  [1.89]  &  [1.07]  &  [0.37]  &  [0.67]  &  [1.46]  & .&  [0.30] \\
Depression score &  22.08  &  21.20  &  21.15  &  22.35  &  22.87  &  20.88  & . &  20.90*** \\
(CESD) &  [0.54]  &  [1.31]  &  [1.35]  &  [0.45]  &  [0.85]  &  [2.06]   & . &  [0.40] \\
Respondent health &  0.54***  &  0.50  &  0.85  &  0.60***  &  0.70  &  0.57   & . &  0.76 \\
is good (\%) &  [0.05]  &  [0.17]  &  [0.10]  &  [0.04]  &  [0.06]  &  [0.20]   & . &  [0.03] \\
\hline
\end{tabular}

\end{adjustbox}
\raggedright{
\footnotesize{Average of baseline characteristics in Parma and Padova, by type of child-care attended. Standard errors of means in brackets. Test for difference in means between each column and the group that received the Reggio intervention (Reggio Municipal, the treatment group, whose information is seen in the first column of Table (\ref{tab:outcomes-adol-asilo-reggio})) was performed; *** significant difference at 1\%, ** significant difference at 5\%, * significant difference at 10\%. Source: authors calculation using survey data.}
}



\end{table}

%%% adolescents, preschool
\begin{table}[H] 
\caption{Outcomes of interest by preschool type, adolescents (age 18), Reggio}
\label{tab:outcomes-adol-materna-reggio}
\centering
\begin{adjustbox}{width=0.9\textwidth,center=\textwidth}
\small
\begin{tabular}{m{4.0cm} ccccc}
\hline \hline 
 & Reggio & Reggio & Reggio & Reggio & Reggio \\
 & Municipal & State & Religious & Private & Not Attended  \\

\hline 
SDQ score & 9.82 & 7.73** & 10.77* & 11.00 & 7.43 \\ 
(self rep.)  & [0.37] & [0.79] & [0.47] & [1.32] & [1.63] \\
Depression score & 22.46 & 20.71 & 24.15** & 23.33 & 19.14 \\ 
(CESD)  & [0.51] & [1.23] & [0.69] & [1.58] & [1.50] \\ 
Respondent health & 0.74 & 0.82 & 0.66 & 0.83 & 0.71 \\ 
is good (\%)  & [0.03] & [0.08] & [0.05] & [0.17] & [0.18] \\
\hline 
\end{tabular}

\end{adjustbox}
\raggedright{
\footnotesize{Average of baseline characteristics in Reggio, by type of child-care attended. Standard errors of means in brackets. Test for difference in means between each column and the first column (Reggio Municipal, the treatment group) was performed; *** significant difference at 1\%, ** significant difference at 5\%, * significant difference at 10\%. Source: authors calculation using survey data. CESD: Center for Epidemiologic Studies Depression Scale.}
}


  
\end{table}

\begin{table}[H] 
\caption{Outcomes of interest by preschool type, adolescents (age 18), Parma and Padova}
\label{tab:outcomes-adol-materna-pp}

\centering
\begin{adjustbox}{width=1\textwidth,center=\textwidth}
\small
\begin{tabular}{m{4.0cm} cccccccccc}
\hline \hline 
 & Parma & Parma & Parma & Parma & Parma & Padova & Padova & Padova & Padova & Padova \\
 & Municipal & State & Religious & Private & Not Attended & Municipal & State & Religious & Private & Not Attended \\
\hline 
SDQ score  & 9.11 & 8.00** & 9.53 & 8.67 & 8.75 & 9.56 & 9.00 & 10.18 & 7.50 & . \\
(self reported)  & [0.39] & [0.68] & [0.47] & [1.50] & [0.85] & [0.47] & [0.55] & [0.43] & [1.26] & . \\
Depression score  & 22.29 & 20.60 & 22.81 & 20.00 & 20.75 & 21.75 & 19.32*** & 21.77 & 19.67 & . \\
(CESD)  & [0.48] & [0.74] & [0.58] & [1.55] & [2.66] & [0.61] & [0.78] & [0.55] & [1.98] & . \\
Respondent health  & 0.62** & 0.37*** & 0.63 & 1.00 & 0.75 & 0.68 & 0.85 & 0.75 & 0.67 & . \\
is good  & [0.05] & [0.07] & [0.05] & [0.00] & [0.25] & [0.05] & [0.05] & [0.04] & [0.21] & . \\
\hline
\end{tabular}

\end{adjustbox}
\raggedright{
\footnotesize{Average of baseline characteristics in Parma and Padova, by type of child-care attended. Standard errors of means in brackets. Test for difference in means between each column and the group that received the Reggio intervention (Reggio Municipal, the treatment group, whose information is seen in the first column of Table (\ref{tab:outcomes-adol-asilo-reggio})) was performed; *** significant difference at 1\%, ** significant difference at 5\%, * significant difference at 10\%. Source: authors calculation using survey data. CESD: Center for Epidemiologic Studies Depression Scale.}
}
  
\end{table}
%
%%-------------------- Theoretical Approach ----------------------------------------------%
\section{Model of School Selection (Sketch)}
\label{sec:model}
%\section{Model of School Selection (Sketch)}
%\label{sec:model}

In this section we sketch a theoretical framework which helps motivate and interpret our empirical results. In order to better understand the family decision to send the child to a Reggio Children Approach (RA) child-care center, consider a generalized Roy model framework. $Y_{1,i}$ represents the potential outcome of the child $i$ when attending an RA center, and $Y_{0,i}$ the potential outcome of the child when not attending an RA center. Define potential outcomes as follows:

\begin{align*}
Y_{1,i}& =\mu_{1}(X_{i})+U_{1,i} \\
Y_{0,i}& =\mu_{0}(X_{i})+U_{0,i}
\end{align*}%
where $\mu_{d}(X_{i})=E(Y_{d,i}|X=x)$ for $d=0,1$. The individual return to the Reggio Children Approach can be defined as the difference between these two potential outcomes, $Y_{1,i}-Y_{0,i}=\mu_{1}(X_{i})-\mu_{0}(X_{i})+U_{1,i}-U_{0,i}$.

The total costs of attending the two types of schools will, in general, systematically differ as well. Let the costs associated with attending school type $d$ for family $i$ be given by
\begin{equation*}
V_{d}(Z_{i})=\delta_{d}(Z_{i})+\varepsilon_{d,i},\ d=0,1,
\end{equation*}%
where $Z_{i}$ are observable characteristics of household $i$ and $EV_{d}(Z_{i})=\delta_{d}(Z_{i}).$ A family will choose to send their child to a Reggio Children Approach center, $R_{i}=1$, depending on the perceived net benefit$,I_{R,i}$, which is given by
\begin{eqnarray*}
I_{R,i} &=&(Y_{1,i}-V_{1,i})-(Y_{0,i}-V_{0,i}) \\
&=&(Y_{1,i}-Y_{0,1})-(V_{1,i}-V_{0,i}) \\
&=&(\mu_{1}(X_{i})-\mu_{0}(X_{i})+U_{1,i}-U_{0,i})-(\delta_{1}(Z_{i})-\delta_{0}(Z_{i})+\varepsilon_{1,i}-\varepsilon_{0,i}) \\
&=&\mu_{1}(X_{i})-\mu_{0}(X_{i})-\delta_{1}(Z_{i})+\delta_{0}(Z_{i})-W_{i},
\end{eqnarray*}%
where $W_{i}\equiv U_{0,i}-U_{1,i}+\varepsilon_{1,i}-\varepsilon_{0,i}.$ Then define the net systematic return to choice $i$ as $R_{d}(X_{i},Z_{i})=\mu_{d}(X_{i})-\delta_{d}(Z_{i}).$ Then household $i$ chooses RA when $I_{R,i} \geq 0$, that is when \[R_{1}(X_{i},Z_{i}) - R_{0}(X_{i},Z_{i}) \geq W_{i}\]

We note that the observable variables $Z_{i}$ are considered to be characteristics of the household that shift the cost but do not affect the potential academic and non-academic benefits of attending the two types of school. That is, the difference in attendance likelihoods for household $i$ are a function of the $X_{i}$ and $Z_{i},$ but the difference in school benefits between an RA school and a non-RA school are functions only of the
observable characteristics $X_{i}.$

The variables observable to the econometrician are $R_{i},Y_{R_{i},i},X_{i},$ and $Z_{i}.$ Then we have%
\begin{equation}
EY_{R_{i},i}=\mu_{R_{i}}(X_{i})+E(U_{R_{i},i}|R_{i}).  \label{heckit}
\end{equation}%
Now assume for simplicity that all of the conditional expectation functions in the model are linear, so that%
\begin{equation*}
EY_{R_{i},i}=X_{i}\beta_{R_{i}},
\end{equation*}%
and%
\begin{equation*}
R_{1}(X_{i},Z_{i})-R_{0}(X_{i},Z_{i})=X_{i}(\beta_{1}-\beta_{0})-Z_{i}(\gamma_{1}-\gamma_{0}),
\end{equation*}%
and assume that all of the disturbance terms in the model are distributed as a multivariate normal. Then $W_{i}$ is normally distributed, and the conditional expectation $E(U_{R_{i},i}|R_{i})$ in (\ref{heckit}) can be consistently estimated (up to a scalar constant) from a first stage probit, as in Heckman (1976). We can then estimate the regression function with the estimated (up to scale) value $E(U_{R_{i},i}|R_{i}),$ and this enables identification of $\beta_{1},\beta_{0}$ and $(\gamma_{1}-\gamma_{0})$ and a small set of elements of the covariance matrix of $W_{i}.$ More efficient estimates of the identified parameters can be obtained through joint estimation of the $R_{i}$ and $Y_{R_{i},i}$ dependent variables.

%\end{document}

%
%%--------------------   Empirical Approach ----------------------------------------------%
\section{Empirical Analysis}
\label{sec:analysis}
This section reports the preliminary results on the multiple strategies used to identify the effect of participating in one of the schools following the Reggio Children Approach. The evaluation of the Reggio Children Approach presents several challenges, given the non-experimental nature of the intervention. Nonetheless, it also constitutes a unique opportunity: it would hardly be possible to carry out an evaluation of a long-lasting and evolving programme such as this using a simple randomized experiment. %Such a long-term assessment is particularly important for programmes which make use of local resources--which make them both more sustainable and inexpensive to implement on one side, but also more dependent on the motivation of the community members on the other.

\subsection{Identification Strategy}
\label{sec:identification}
%\subsection{Balance of observable characteristics}
%\label{sec:balance}
Central to our analysis is the identification of a credible counterfactual that would allow the estimation of participants' potential outcomes in the absence of the Reggio Children Approach. Our dataset leverages the particular institutional design of the Italian child care system in order to provide different comparison groups which we can use to estimate our counterfactuals. Specifically, we can use variation across child-care types (municipal, state, religious, private, or none) and across cities (Reggio Emilia, Parma, or Padova).\footnote{Another level of interesting variation is across cohorts, which were subject to different levels of treatment availability due to the historical roll-out of the RA. Currently we focus on the adolescent sample, and therefore we are not using this variation.}

Throughout this analysis we use the outcomes of individuals who attended the Reggio Children Approach child-care, and contrast them to the outcomes of three different comparison groups:
the \textit{first group} comprises all the respondents living in Reggio Emilia who did \textit{not} attend any Reggio Children Approach child-care, that is respondents who either have not attended any child-care, or have attended state, religious, or private child-care. This group is composed of families living in the same city as the treatment group, and are therefore similar since they face the same institutional and background characteristics.
The \textit{second group} is composed of all the respondents living in Parma or Padova, regardless of whether they attended child-care or which type of child-care. The families in this group did not have the chance of sending their children to RA even if they would have wanted to, since such child-care approach was not available in their community; for this reason they can represent an interesting comparison sample.
Finally the \textit{third group} combines the previous two, that is all the respondents in Parma and Padova, and all the respondents in Reggio Emilia who did \textit{not} attend a municipal child-care.

In order to evaluate the effect of RA, we use three different methods: ordinary least squares (OLS), instrumental variables (IV), and propensity score matching (PSM).

Consider the following specification:

\begin{align}
Y_{ic} & = \delta^{s} D^{s}_{ic} + \beta_{X}X_{ic} + \alpha_{c} +  u_{ic} \label{eq:secondstage} \\ 
D^{s}_{ic} & = \gamma_1 + \gamma_{X}X_{ic} + \gamma_{Z}Z_{ic} + \varepsilon_{ic} \label{eq:firststage}
\end{align}

where $Y_{ic}$ is the outcome of individual $i$ in city $c$ and $D^{s}_{ic}$ is a dummy for attendance of RA child-care of type $s \in \{ITC,PS\}$, where $s$ can be either infant-toddler centers (ITC, age 0-3) or preschool (PS, age 3-6); the coefficients of interest are $\delta^{ITC}$ and $\delta^{PS}$. 

$X_{ic}$ are baseline family characteristics of individual $i$, in city $c$;
the following controls considered in all specifications: adolescent age and gender, poor baseline health (either low-birth weight or premature birth), number of older siblings, mother with college education, father with college education, family with income higher than 25,000 euros, family owns home, distance from the center, religious caregiver, dummy for interview mode (computer vs paper). When analyzing the effect of infant-toddler center $D^{ITC}_{ic}$, we include also a dummy for participation in any tipe of ITC, so that the coefficient $\delta^{ITC}$ captures the difference between RA and other types of centers.
Finallly, $Z_{ic}$ are cost shifters influencing the probability of attending RA but not the outcomes of interest. As shifter for the participation in infant-toddler center we consider the following: caregiver attended ITC as a child, below median score given from the Reggio Children admission criteria in the application to RA, cubic polynomial of distance from closest municipal ITC, interaction between distance and mother born in the province, father born in the province; as shifter for the participation in preschool we consider the following: caregiver attended preschool as a child, below median score in application to RA, distance from closest municipal preschool, mother born in the province, father born in the province.

\medskip

In the first approach, OLS, we simply consider the potential for selection based on observables, and report the conditional differences in outcomes depending on whether individuals attended an RA child-care or not. The underlying assumption is the absence of selection on unobservable characteristics, so that $D \indep u|X$

In the second, IV, we model equation (\ref{eq:firststage}) as a simple linear relationship, and we use the cost shifters $Z$ to instrument for participation into RA child-care. The underlying assumption is all the cost shifters influence participation but not the outcomes, so that $D \indep u|X,Z$.

In the third and final approach, PSM, we consider two different specifications: one is a standard propensity score matching, the other models both demand and supply of child-care using a partial observability model. 

In the first approach (PSM 1 in the tables), based on equation (\ref{eq:firststage}) we match treated children in Reggio with untreated children with close probability of attending an RA infant-toddler center or preschool. We first estimate the propensity score matching (using only observations from Reggio Emilia), where beyond the usual control variables $X_{ic}$, we include the cost shifters $Z_{ic}$ %score given from the Reggio Children admission criteria, the distance to the closest municipal childcare center, the parents being born in the province they live nowadays, and the caregiver having attended an infant toddler center as a child. 
After estimation, we predict the probability of attending a RA school for each child in the sample. We match each treated child with the untreated child with the closest probability of attending a RA school (with replacement). The RA effect is given by the difference in the average outcome between the two groups, after controlling for city dummies and infant-toddler center participation where necessary. 

In the second approach (PSM 2), we match treated children in Reggio Emilia with untreated children with close probability of demanding the RA service and close probability of being offered the RA service in case they were applying to RA childcare. We first estimate a partial observability model (using only observations from Reggio Emilia), where beyond the usual control variables, we let the supply of RA service depend on a subset of $Z_{ic}$, notably the score given from the Reggio Children admission criteria, and the demand on the rest of $Z_{ic}$, notably distance to the closest municipal childcare center, parents being born in the province they live nowadays, the caregiver having attended an infant toddler center as a child. 
\begin{align*}
S_i & = A_i \theta + \omega_s \\
D_i & = Z_i \sigma + \omega_d \\
\end{align*}
where $\omega_s$ and $\omega_d$ are jointly normally distributed. After estimation, we predict the probability of demanding and being offered the service for each child in the sample. We match each treated child with the untreated child with the closest probability of demanding a RA slot and the closest probability of demanding childcare (with replacement). The RA effect is given by the difference in the average outcome between the two groups, after controlling for city dummies and infant-toddler center participation where necessary. 

\bigskip 

In the following sections we report the results of all of these approaches, after evaluating the extent of selection into RA based on observables characteristics. 

%% Selection on observables
\subsection{Selection on Observables}
\label{sec:selection}

In order to properly construct counterfactuals, we need to understand the extent of selection into RA across these different groups. First of all we consider the extent of selection by looking for systemic differences in observables characteristics at baseline, comparing the characteristics of the family where the respondent was born. Tables (\ref{tab:adol_CONTROLasilo}) reports the average value of these observable baseline family characteristics broken down by comparison group, and contrasts them to the average of those who attended a RA infant-toddler center. Table (\ref{tab:adol_CONTROLasilo}) carries out the same procedure but focusing on preschool attendance.\footnote{While the values for comparison group 2 (Parma and Padova) stay the same across the two tables, the other groups change because of differences in the participation in RA infant-toddler center and preschool. Tables (\ref{tab:adol_PREasilo}) and (\ref{tab:adol_PREmaterna}) in the appendix report the same averages, but disaggregated by city and child-care type.} 
Overall, we find that families sending their children to the RA infant-toddler centers are slightly more affluent (more likely to report annual family income above 50,000 euros and owning a home, as shown in Table (\ref{tab:adol_CONTROLasilo}); these differences disappear when comparing attendance to preschool within Reggio Emilia (as shown in the first two columns of table (\ref{tab:adol_CONTROLmaterna}). 

\singlespacing

%% adolescents, infant-toddler
\begin{table}[H]
\caption{Baseline characteristics by infant-toddler-center type, children (age 6)}
\label{tab:child_CONTROLasilo}
\centering
%\begin{adjustbox}{width=1.2\textwidth,center=\textwidth}
\small
%\begin{tabular}{m{2.0cm} cccc}
\begin{tabular}{l cccc}
\hline \hline 
 & Reggio & Reggio & Parma and & Any \\
 & Approach & Other & Padova & Other \\
 &   (1)    &  (2)  & (3)    &  (4) \\
\hline 

%SDQ score (mom rep.) & 8.36 & 9.22 & 8.33 & 8.51 \\
% & [0.32] & [0.37] & [0.17] & [0.16] \\
%Happy child & 9.86 & 10.07* & 9.68** & 9.76 \\
% & [0.12] & [0.13] & [0.06] & [0.06] \\
%Child health is good (\%) - mom report & 0.65 & 0.72 & 0.68 & 0.69 \\
% & [0.04] & [0.04] & [0.02] & [0.02] \\
Age & 6.77 & 6.83 & 6.69*** & 6.72* \\
 & [0.03] & [0.03] & [0.01] & [0.01] \\
Male dummy & 0.57 & 0.5 & 0.54 & 0.53 \\
 & [0.04] & [0.04] & [0.02] & [0.02] \\
Poor health at birth & 0.16 & 0.07** & 0.09** & 0.09** \\
 & [0.03] & [0.02] & [0.01] & [0.01] \\
Older siblings  & 0.04 & 0.07 & 0.04 & 0.05 \\
 & [0.02] & [0.03] & [0.01] & [0.01] \\
Mom university edu  & 0.33 & 0.22** & 0.41* & 0.37 \\
 & [0.04] & [0.03] & [0.02] & [0.02] \\
Dad university edu  & 0.23 & 0.27 & 0.33** & 0.31** \\
 & [0.03] & [0.04] & [0.02] & [0.02] \\
Fam income over 50k  & 0.23 & 0.19 & 0.17 & 0.18 \\
 & [0.03] & [0.03] & [0.02] & [0.01] \\
Fam income not reported  & 0.17 & 0.40*** & 0.27** & 0.30*** \\
 & [0.03] & [0.04] & [0.02] & [0.02] \\
Own home  & 0.55 & 0.61 & 0.68*** & 0.67*** \\
 & [0.04] & [0.04] & [0.02] & [0.02] \\
Distance to center  & 3.11 & 3.29 & 2.96 & 3.03 \\
 & [0.20] & [0.23] & [0.08] & [0.08] \\
Caregiver Religious & 0.82 & 0.87 & 0.83 & 0.84 \\
 & [0.03] & [0.03] & [0.02] & [0.01] \\
CAPI  & 0.6 & 0.51 & 0.45*** & 0.47*** \\
 & [0.04] & [0.04] & [0.02] & [0.02] \\
%cgAsilo_F & 0.24 & 0.09*** & 0.13*** & 0.12*** \\
% & [0.03] & [0.02] & [0.01] & [0.01] \\
%RA score second quartile & 0.18 & 0.24 & 0.23 & 0.24 \\
% & [0.03] & [0.03] & [0.02] & [0.02] \\
%Distance from closest municipal infant-toddler center & 1.15 & 1.35 & 0.79 & 0.91 \\
% & [0.10] & [0.13] & [0.03] & [0.04] \\
%distanza2 & 2.93 & 4.28 & 1.15 & 1.81 \\
% & [0.48] & [0.79] & [0.11] & [0.19] \\
%distanza3 & 10.6 & 20.09 & 2.61 & 6.27 \\
% & [2.49] & [5.13] & [0.41] & [1.15] \\
%Mom born in province & 0.44 & 0.59*** & 0.64*** & 0.63*** \\
% & [0.04] & [0.04] & [0.02] & [0.02] \\
%Dad born in province & 0.45 & 0.60*** & 0.62*** & 0.61*** \\
% & [0.04] & [0.04] & [0.02] & [0.02] \\
%distA_MomBorn & 0.49 & 0.89*** & 0.53*** & 0.61*** \\
% & [0.08] & [0.12] & [0.03] & [0.03] \\
\hline
\end{tabular}
%\end{adjustbox}
\vspace{1ex}

\raggedright{
\footnotesize{Average of baseline characteristics, by treatment (column 1) and comparison group.
The first comparison group (column 2) are individuals living in Reggio Emilia who did not attend the RA. The second comparison group (column 3) are individuals living in Parma or Padova. The final comparison group (column 4) are individuals living in Reggio Emilia, Parma, or Padova who did not attend RA.
Standard errors of means in brackets. Test for difference in means between each column and the first column (Reggio Municipal, the treatment group) was performed; *** significant difference at 1\%, ** significant difference at 5\%, * significant difference at 10\%. Source: authors calculation using survey data.}
}
\end{table}

%\end{table}

%% adolescents, infant-toddler
\begin{table}[H]
\caption{Baseline characteristics by Preschool type, children (age 6)}
\label{tab:child_CONTROLmaterna}
\centering
%\begin{adjustbox}{width=1.2\textwidth,center=\textwidth}
\begin{adjustbox}{max height=\dimexpr\textheight-5.5cm\relax,max width=\textwidth}
\small
%\begin{tabular}{m{2.0cm} cccc}
\begin{tabular}{l cccc}
\hline \hline 
 & Reggio & Reggio & Parma and & Any \\
 & Approach & Other & Padova & Other \\
 &   (1)    &  (2)  & (3)    &  (4) \\
\hline 

%SDQ score (mom rep.) & 8.33 & 9.34* & 8.33 & 8.52 \\
% & [0.32] & [0.38] & [0.17] & [0.16] \\
%Happy child & 9.94 & 9.98 & 9.68*** & 9.74** \\
% & [0.12] & [0.14] & [0.06] & [0.06] \\
%Child health is good (\%) - mom report & 0.67 & 0.69 & 0.68 & 0.69 \\
% & [0.04] & [0.04] & [0.02] & [0.02] \\
Age & 6.80 & 6.80 & 6.69*** & 6.71*** \\
 & [0.03] & [0.03] & [0.01] & [0.01] \\
Male dummy & 0.55 & 0.53 & 0.54 & 0.54 \\
 & [0.04] & [0.04] & [0.02] & [0.02] \\
Poor health at birth & 0.13 & 0.10 & 0.09 & 0.09 \\
 & [0.03] & [0.02] & [0.01] & [0.01] \\
Older siblings  & 0.03 & 0.09 & 0.04 & 0.05 \\
 & [0.02] & [0.03] & [0.01] & [0.01] \\
Mom university edu  & 0.29 & 0.27 & 0.41*** & 0.38** \\
 & [0.04] & [0.04] & [0.02] & [0.02] \\
Dad university edu  & 0.23 & 0.26 & 0.33** & 0.31** \\
 & [0.03] & [0.04] & [0.02] & [0.02] \\
Fam income over 50k  & 0.22 & 0.20 & 0.17 & 0.18 \\
 & [0.03] & [0.03] & [0.02] & [0.01] \\
%Fam income not reported  & 0.27 & 0.30 & 0.27 & 0.28 \\
% & [0.03] & [0.04] & [0.02] & [0.02] \\
Own home  & 0.54 & 0.62 & 0.68*** & 0.67*** \\
 & [0.04] & [0.04] & [0.02] & [0.02] \\
Distance to center  & 3.01 & 3.36 & 2.96 & 3.05 \\
 & [0.17] & [0.25] & [0.08] & [0.08] \\
Caregiver religious  & 0.81 & 0.89* & 0.83 & 0.85 \\
 & [0.03] & [0.03] & [0.02] & [0.01] \\
CAPI  & 0.60 & 0.50* & 0.45*** & 0.46*** \\
 & [0.04] & [0.04] & [0.02] & [0.02] \\
%cgMaterna_F & 0.80 & 0.78 & 0.84 & 0.83 \\
% & [0.03] & [0.03] & [0.02] & [0.01] \\
%RA score second quartile & 0.17 & 0.24 & 0.23 & 0.24* \\
% & [0.03] & [0.04] & [0.02] & [0.02] \\
%Distance from closest municipal preschool & 0.97 & 1.37* & 0.96 & 1.04 \\
% & [0.07] & [0.13] & [0.04] & [0.04] \\
%Mom born in province & 0.51 & 0.52 & 0.64*** & 0.62*** \\
% & [0.04] & [0.04] & [0.02] & [0.02] \\
%Dad born in province & 0.51 & 0.53 & 0.62** & 0.60* \\
% & [0.04] & [0.04] & [0.02] & [0.02] \\
\hline
\end{tabular}
\end{adjustbox}
% \vspace{1ex}

% \raggedright{
% \footnotesize{Average of baseline characteristics, by treatment (column 1) and comparison group.
% The first comparison group (column 2) are individuals living in Reggio Emilia who did not attend the RA. The second comparison group (column 3) are individuals living in Parma or Padova. The final comparison group (column 4) are individuals living in Reggio Emilia, Parma, or Padova who did not attend RCA.
% Standard errors of means in brackets. Test for difference in means between each column and the first column (Reggio Municipal, the treatment group) was performed; *** significant difference at 1\%, ** significant difference at 5\%, * significant difference at 10\%. Source: authors calculation using survey data.}
% }
\end{table}

%\end{table}

%% adolescents, infant-toddler
\begin{table}[H]
\caption{Baseline characteristics by infant-toddler-center type, adolescents (age 18)}
\label{tab:adol_CONTROLasilo}
\centering
%\begin{adjustbox}{width=1.2\textwidth,center=\textwidth}
\small
%\begin{tabular}{m{2.0cm} cccc}
\begin{tabular}{l cccc}
\hline \hline 
 & Reggio & Reggio & Parma and & Any \\
 & Approach & Other & Padova & Other \\
 &   (1)    &  (2)  & (3)    &  (4) \\
\hline 

%asiloG_1_2   &   0   &   1   &   2   &   3 \\ 
Age   &   18.71   &   18.71   &   18.74   &   18.73 \\ 
   &   [0.03]   &   [0.03]   &   [0.01]   &   [0.01] \\ 
Male dummy   &   0.42   &   0.42   &   0.46   &   0.45 \\ 
   &   [0.04]   &   [0.04]   &   [0.02]   &   [0.02] \\ 
Poor Health at birth   &   0.08   &   0.09   &   0.1   &   0.1 \\ 
   &   [0.02]   &   [0.02]   &   [0.01]   &   [0.01] \\ 
Older siblings   &   0.39   &   0.46   &   0.4   &   0.41 \\ 
   &   [0.04]   &   [0.07]   &   [0.03]   &   [0.03] \\ 
Mom university edu   &   0.29   &   0.22   &   0.31   &   0.29 \\ 
   &   [0.04]   &   [0.03]   &   [0.02]   &   [0.02] \\ 
Dad university edu   &   0.22   &   0.15   &   0.27   &   0.25 \\ 
   &   [0.03]   &   [0.03]   &   [0.02]   &   [0.02] \\ 
Fam income over 50k   &   0.33   &   0.20***   &   0.20***   &   0.20*** \\ 
   &   [0.04]   &   [0.03]   &   [0.02]   &   [0.02] \\ 
%Fam income not reported   &   0.16   &   0.27**   &   0.38***   &   0.36*** \\ 
%   &   [0.03]   &   [0.04]   &   [0.02]   &   [0.02] \\ 
Own home   &   0.89   &   0.80**   &   0.79***   &   0.79*** \\ 
   &   [0.03]   &   [0.03]   &   [0.02]   &   [0.02] \\ 
Distance to center   &   2.82   &   3.52*   &   2.84   &   2.98 \\ 
   &   [0.17]   &   [0.22]   &   [0.08]   &   [0.08] \\ 
Caregiver religious   &   0.74   &   0.81   &   0.80*   &   0.80* \\ 
   &   [0.04]   &   [0.03]   &   [0.02]   &   [0.02] \\ 
CAPI   &   0.44   &   0.4   &   0.52   &   0.49 \\ 
   &   [0.04]   &   [0.04]   &   [0.02]   &   [0.02] \\ 
%Caregiver to ITC   &   0.08   &   0.03*   &   0.05   &   0.05 \\ 
%   &   [0.02]   &   [0.01]   &   [0.01]   &   [0.01] \\ 
%RA score second quartile   &   0.3   &   0.20**   &   0.26   &   0.25 \\ 
%   &   [0.04]   &   [0.03]   &   [0.02]   &   [0.02] \\ 
%Distance from closest municipal preschool   &   1.03   &   1.43   &   0.79   &   0.92 \\ 
%   &   [0.09]   &   [0.13]   &   [0.03]   &   [0.04] \\ 
%distanza2   &   2.39   &   4.29   &   1.04   &   1.72 \\ 
%   &   [0.52]   &   [0.65]   &   [0.10]   &   [0.17] \\ 
%distanza3   &   9.23   &   17.21   &   2.1   &   5.28 \\ 
%   &   [3.20]   &   [3.48]   &   [0.40]   &   [0.84] \\ 
%Mom born in province   &   0.75   &   0.61***   &   0.73   &   0.71 \\ 
%   &   [0.04]   &   [0.04]   &   [0.02]   &   [0.02] \\ 
%Dad born in province   &   0.62   &   0.56   &   0.67   &   0.65 \\ 
%   &   [0.04]   &   [0.04]   &   [0.02]   &   [0.02] \\ 
%distA_MomBorn   &   0.75   &   0.97   &   0.6   &   0.68 \\ 
%   &   [0.09]   &   [0.12]   &   [0.03]   &   [0.03] \\ 
\hline 
\end{tabular}
%\end{adjustbox}
\vspace{1ex}

\raggedright{
\footnotesize{Average of baseline characteristics, by treatment (column 1) and comparison group.
The first comparison group (column 2) are individuals living in Reggio Emilia who did not attend the RA. The second comparison group (column 3) are individuals living in Parma or Padova. The final comparison group (column 4) are individuals living in Reggio Emilia, Parma, or Padova who did not attend RA.
Standard errors of means in brackets. Test for difference in means between each column and the first column (Reggio Municipal, the treatment group) was performed; *** significant difference at 1\%, ** significant difference at 5\%, * significant difference at 10\%. Source: authors calculation using survey data.}
}
\end{table}

%\end{table}

%% adolescents, infant-toddler
\begin{table}[H]
\caption{Baseline characteristics by Preschool type, adolescents (age 18)}
\label{tab:adol_CONTROLmaterna}
\centering
%\begin{adjustbox}{width=1.2\textwidth,center=\textwidth}
\begin{adjustbox}{max height=\dimexpr\textheight-5.5cm\relax,max width=\textwidth}
\small
%\begin{tabular}{m{2.0cm} cccc}
\begin{tabular}{l cccc}
\hline \hline 
 & Reggio & Reggio & Parma and & Any \\
 & Approach & Other & Padova & Other \\
 &   (1)    &  (2)  & (3)    &  (4) \\
\hline 

Age   &   18.70   &   18.73   &   18.74   &   18.74 \\ 
   &   [0.03]   &   [0.03]   &   [0.01]   &   [0.01] \\ 
Male dummy   &   0.42   &   0.44   &   0.46   &   0.46 \\ 
   &   [0.04]   &   [0.04]   &   [0.02]   &   [0.02] \\ 
Poor health at birth   &   0.07   &   0.11   &   0.10   &   0.10 \\ 
   &   [0.02]   &   [0.03]   &   [0.01]   &   [0.01] \\ 
Older siblings   &   0.40   &   0.44   &   0.40   &   0.40 \\ 
   &   [0.04]   &   [0.07]   &   [0.03]   &   [0.03] \\ 
Mom university edu   &   0.23   &   0.28   &   0.31**   &   0.31** \\ 
   &   [0.03]   &   [0.04]   &   [0.02]   &   [0.02] \\ 
Dad university edu   &   0.19   &   0.19   &   0.27**   &   0.25* \\ 
   &   [0.03]   &   [0.03]   &   [0.02]   &   [0.02] \\ 
Fam income over 50k   &   0.27   &   0.26   &   0.20**   &   0.21* \\ 
   &   [0.03]   &   [0.04]   &   [0.02]   &   [0.02] \\ 
%Fam income not reported   &   0.23   &   0.20   &   0.38***   &   0.34*** \\ 
%   &   [0.03]   &   [0.04]   &   [0.02]   &   [0.02] \\ 
Own home   &   0.85   &   0.82   &   0.79*   &   0.79* \\ 
   &   [0.03]   &   [0.03]   &   [0.02]   &   [0.02] \\ 
Distance to center   &   3.11   &   3.15   &   2.84   &   2.89 \\ 
   &   [0.17]   &   [0.23]   &   [0.08]   &   [0.08] \\ 
Caregiver religious   &   0.68   &   0.88***   &   0.80***   &   0.82*** \\ 
   &   [0.04]   &   [0.03]   &   [0.02]   &   [0.01] \\ 
CAPI   &   0.47   &   0.38   &   0.52   &   0.49 \\ 
   &   [0.04]   &   [0.04]   &   [0.02]   &   [0.02] \\ 
%Caregiver to ITC   &   0.07   &   0.04   &   0.05   &   0.05 \\ 
%   &   [0.02]   &   [0.02]   &   [0.01]   &   [0.01] \\ 
%RA score second quartile   &   0.29   &   0.20   &   0.26   &   0.25 \\ 
%   &   [0.04]   &   [0.04]   &   [0.02]   &   [0.02] \\ 
%Distance from closest municipal preschool   &   0.93   &   1.24*   &   0.85   &   0.93 \\ 
%   &   [0.07]   &   [0.12]   &   [0.03]   &   [0.03] \\ 
%Mom born in province   &   0.72   &   0.62*   &   0.73   &   0.71 \\ 
%   &   [0.04]   &   [0.04]   &   [0.02]   &   [0.02] \\ 
%Dad born in province   &   0.60   &   0.56   &   0.67   &   0.65 \\ 
%   &   [0.04]   &   [0.04]   &   [0.02]   &   [0.02] \\ 
\hline 
\end{tabular}
\end{adjustbox}
% \vspace{1ex}

% \raggedright{
% \footnotesize{Average of baseline characteristics, by treatment (column 1) and comparison group.
% The first comparison group (column 2) are individuals living in Reggio Emilia who did not attend the RA. The second comparison group (column 3) are individuals living in Parma or Padova. The final comparison group (column 4) are individuals living in Reggio Emilia, Parma, or Padova who did not attend RA.
% Standard errors of means in brackets. Test for difference in means between each column and the first column (Reggio Municipal, the treatment group) was performed; *** significant difference at 1\%, ** significant difference at 5\%, * significant difference at 10\%. Source: authors calculation using survey data.}
% }
\end{table}

%\end{table}

\doublespacing

\subsection{Empirical Results}
\label{sec:results}
The following tables show the empirical results of our four different identification approaches for the estimation of $\delta^{ITC}$ (top panel in all tables) and $\delta^{PS}$ (bottom panel in all tables), using different measures of outcomes: the Strength and Difficulties Questionnaire, a measure of self-reported depression, and a dummy for excellent or good self-reported health. 
Table (\ref{tab:resultsAdo-reggio}) uses the first comparison group, i.e. the analysis was run only within the city of Reggio Emilia, comparing those who attended or not RA. In this group, all of the individuals had the opportunity to go to RA, but some decided to either stay at home or attend state, religious, or private child-care. This is the only group where our instrument have relevant variation.
Table (\ref{tab:resultsAdo-parmapadova}) applies the same identification procedures, but with the second control group: the adolescents in Parma and Padova. These individuals did not have the chance of attending an RA child-care, even if their family would have wanted them to.
Finally, table (\ref{tab:resultsAdo-all}) reports the estimation coefficients when the all the comparison groups are combined together, i.e. compares the outcomes of adolescents who attended RA and the outcomes of all the adolescents in Parma, Padova and Reggio Emilia who did not attend RA.

\singlespacing
\setlength\tabcolsep{0.25em}
%\subsubsection{Adolescent Results: OLS, PSM, and IV}

\begin{table}[H]
\caption{Reggio Approach compared to other child-care in Reggio Emilia; Adolescents}
\label{tab:resultsAdo-reggio}
 \begin{centering} 
\vspace{1ex}
\begin{tabular}{ r c ccc} 
\hline \hline 
 & \textit{Outcome} & \textbf{SDQ}  & \textbf{Depression} & \textbf{Health}  \\ 
\textbf{Specification}  &  &  &  &  \\ 
\hline 
\multicolumn{5}{c}{\textit{Infant Toddler Center }} \\ 
\hline 

OLS	 & coeff.	 & -1.473	 & 0.011	 & -0.129	\\
	 & s.e.	 & [1.700]	 & [0.382]	 & [0.109]	\\
	 & \textit{obs.}	 & \textit{293}	 & \textit{292}	 & \textit{294}	\\
PSM 1	 & coeff.	 & -1.262	 & 0.426*	 & -0.190*	\\
	 & s.e.	 & [1.333]	 & [0.247]	 & [0.112]	\\
	 & \textit{obs.}	 & \textit{304}	 & \textit{302}	 & \textit{306}	\\
PSM 2	 & coeff.	 & -2.039	 & 0.051	 & -0.119	\\
	 & s.e.	 & [1.333]	 & [0.300]	 & [0.091]	\\
	 & \textit{obs.}	 & \textit{304}	 & \textit{303}	 & \textit{306}	\\
IV	 & coeff.	 & 3.202	 & 0.686	 & 0.853	\\
	 & s.e.	 & [7.300]	 & [1.559]	 & [0.844]	\\
	 & \textit{obs.}	 & \textit{293}	 & \textit{292}	 & \textit{294}	\\
\hline \multicolumn{5}{c}{\textit{Preschool }} \\ \hline
OLS	 & coeff.	 & -0.201	 & -0.116	 & 0.070	\\
	 & s.e.	 & [0.572]	 & [0.125]	 & [0.058]	\\
	 & \textit{obs.}	 & \textit{295}	 & \textit{294}	 & \textit{296}	\\
PSM 1	 & coeff.	 & -1.038*	 & -0.133	 & 0.113**	\\
	 & s.e.	 & [0.539]	 & [0.118]	 & [0.054]	\\
	 & \textit{obs.}	 & \textit{318}	 & \textit{316}	 & \textit{318}	\\
PSM 2	 & coeff.	 & -0.167	 & -0.005	 & 0.106**	\\
	 & s.e.	 & [0.487]	 & [0.106]	 & [0.053]	\\
	 & \textit{obs.}	 & \textit{326}	 & \textit{325}	 & \textit{327}	\\
IV	 & coeff.	 & 1.335	 & -0.170	 & 0.231	\\
	 & s.e.	 & [2.088]	 & [0.450]	 & [0.243]	\\
	 & \textit{obs.}	 & \textit{295}	 & \textit{294}	 & \textit{296}	\\
\hline 
\end{tabular} 
\par\end{centering} 
%\vspace{2ex}
%\begin{footnotesize}
%\textbf{Notes:} Each cell in the table reports the coefficient associated to a dummy for participation in the Reggio Approach infant-toddler or preschool, estimated according to the four different methods explained in section  \ref{sec:identification}. 
%The control group is represented by any other child-care option in Reggio Emilia. 
%Outcomes considered: Strength and Difficulties Questionnaire (SDQ), raw score; depression, raw score; dummy for excellent or good self-reported health. Controls considered: adolescent age and gender, poor baseline health (either low-birth weight or premature birth), number of older siblings, mother with college education, father with college education, family with income higher than 25,000 euros, family owns home, distance from the center, religious caregiver, dummy for interview mode (computer vs paper). 
%Instruments for infant-toddler center (ITC): caregiver attended ITC as a child, below median score in application to RA, cubic polynomial of distance from closest municipal ITC, distance $\times$ mother born in the province, father born in the province.
%Instruments for preschool (PS): caregiver attended PS as a child, below median score in application to RA, distance from closest municipal PS, mother born in the province, father born in the province.
%\end{footnotesize}
\end{table}

\begin{table}[H]		
\caption{Reggio Children Approach compared to Parma and Padova; Adolescents}					
\label{tab:resultsAdo-parmapadova}					
 \begin{centering}		
\vspace{1ex}					
\begin{tabular}{ r c ccc}		
\hline \hline		
 & \textbf{\textit{Outcome}} & \textbf{SDQ}  & \textbf{Depression} & \textbf{Health}  \\		
\textbf{Specification}  &  &  &  &  \\		
\hline		
\multicolumn{5}{c}{\textit{Infant Toddler Center }} \\		
\hline		
					
OLS	 & coeff.	 & 0.557	 & 0.072	 & 0.237***	\\
	 & s.e.	 & [0.558]	 & [0.106]	 & [0.056]	\\
	 & \textit{obs.}	 & \textit{677}	 & \textit{676}	 & \textit{682}	\\
PSM 1	 & coeff.	 & -0.425	 & -0.048	 & 0.156*	\\
	 & s.e.	 & [0.842]	 & [0.150]	 & [0.082]	\\
	 & \textit{obs.}	 & \textit{304}	 & \textit{302}	 & \textit{306}	\\
PSM 2	 & coeff.	 & 1.037	 & 0.226	 & 0.247***	\\
	 & s.e.	 & [0.877]	 & [0.165]	 & [0.091]	\\
	 & \textit{obs.}	 & \textit{301}	 & \textit{303}	 & \textit{306}	\\
IV	 & coeff.	 & 1.649	 & 0.052	 & 0.385	\\
	 & s.e.	 & [2.315]	 & [0.457]	 & [0.258]	\\
	 & \textit{obs.}	 & \textit{677}	 & \textit{676}	 & \textit{682}	\\
\hline \multicolumn{5}{c}{\textit{Preschool }} \\ \hline					
OLS	 & coeff.	 & 0.531	 & 0.026	 & 0.187***	\\
	 & s.e.	 & [0.489]	 & [0.097]	 & [0.050]	\\
	 & \textit{obs.}	 & \textit{686}	 & \textit{685}	 & \textit{693}	\\
PSM 1	 & coeff.	 & -0.045	 & -0.033	 & 0.168**	\\
	 & s.e.	 & [0.612]	 & [0.123]	 & [0.066]	\\
	 & \textit{obs.}	 & \textit{318}	 & \textit{316}	 & \textit{318}	\\
PSM 2	 & coeff.	 & 0.571	 & -0.013	 & 0.201***	\\
	 & s.e.	 & [0.626]	 & [0.127]	 & [0.066]	\\
	 & \textit{obs.}	 & \textit{323}	 & \textit{324}	 & \textit{328}	\\
%IV	 & coeff.	 & -0.839	 & 0.467	 & 0.464	\\
%	 & s.e.	 & [4.372]	 & [0.926]	 & [0.449]	\\
%	 & \textit{obs.}	 & \textit{686}	 & \textit{685}	 & \textit{693}	\\
\hline		
\end{tabular}		
\par\end{centering}		
\vspace{2ex}					
\begin{footnotesize}
\textbf{Notes:} Each cell in the table reports the coefficient associated to a dummy for participation in the Reggio Children Approach infant-toddler or preschool, estimated according to the four different methods explained in section  \ref{sec:identification}.
The control group is represented by any child-care option in Parma or Padova.
Outcomes considered: Strength and Difficulties Questionnaire (SDQ), raw score; depression, raw score; dummy for excellent or good self-reported health. Controls considered: adolescent age and gender, poor baseline health (either low-birth weight or premature birth), number of older siblings, mother with college education, father with college education, family with income higher than 25,000 euros, family owns home, distance from the center, religious caregiver, dummy for interview mode (computer vs paper). 
Instruments for infant-toddler center (ITC): caregiver attended ITC as a child, median score in application to RA, cubic polynomial of distance from closest municipal ITC, distance $\times$ mother born in the province, father born in the province.	Instruments for preschool (PS): caregiver attended PS as a child, below median score in application to RA, distance from closest municipal PS, mother born in the province, father born in the province.				
\end{footnotesize}					
\end{table}					

\begin{table}[H] 
\caption{Reggio Children Approach compared to other child-care in Reggio Emilia, Parma, or Padova; Adolescents}
\label{tab:resultsAdo-all}
 \begin{centering} 
\vspace{1ex}
\begin{tabular}{ r c ccc} 
\hline \hline 
 & \textbf{\textit{Outcome}} & \textbf{SDQ}  & \textbf{Depression} & \textbf{Health}  \\ 
\textbf{Specification}  &  &  &  &  \\ 
\hline 
\multicolumn{5}{c}{\textit{Infant Toddler Center }} \\ 
\hline 

OLS	 & coeff.	 & -0.625	 & -0.07	 & 0.11	\\
	 & s.e.	 & [0.694]	 & [0.142]	 & [0.069]	\\
	 & \textit{obs.}	 & \textit{818}	 & \textit{817}	 & \textit{823}	\\
PSM 1	 & coeff.	 & -0.815	 & -0.012	 & 0.116	\\
	 & s.e.	 & [0.809]	 & [0.156]	 & [0.081]	\\
	 & \textit{obs.}	 & \textit{304}	 & \textit{302}	 & \textit{306}	\\
PSM 2	 & coeff.	 & -0.571	 & 0.242	 & 0.139	\\
	 & s.e.	 & [1.078]	 & [0.236]	 & [0.122]	\\
	 & \textit{obs.}	 & \textit{303}	 & \textit{304}	 & \textit{306}	\\
IV	 & coeff.	 & 2.521	 & -0.205	 & 0.295	\\
	 & s.e.	 & [2.827]	 & [0.573]	 & [0.303]	\\
	 & \textit{obs.}	 & \textit{818}	 & \textit{817}	 & \textit{823}	\\
\hline \multicolumn{5}{c}{\textit{Preschool }} \\ \hline
OLS	 & coeff.	 & -0.377	 & -0.135	 & 0.049	\\
	 & s.e.	 & [0.556]	 & [0.119]	 & [0.057]	\\
	 & \textit{obs.}	 & \textit{819}	 & \textit{818}	 & \textit{825}	\\
PSM 1	 & coeff.	 & -0.994	 & 0.072	 & 0.192**	\\
	 & s.e.	 & [0.976]	 & [0.184]	 & [0.096]	\\
	 & \textit{obs.}	 & \textit{318}	 & \textit{316}	 & \textit{318}	\\
PSM 2	 & coeff.	 & -0.513	 & 0.072	 & 0.028	\\
	 & s.e.	 & [0.861]	 & [0.173]	 & [0.090]	\\
	 & \textit{obs.}	 & \textit{323}	 & \textit{324}	 & \textit{328}	\\
%IV	 & coeff.	 & 2.989	 & -0.803	 & 0.289	\\
%	 & s.e.	 & [3.161]	 & [0.614]	 & [0.352]	\\
%	 & \textit{obs.}	 & \textit{819}	 & \textit{818}	 & \textit{825}	\\
\hline 
\end{tabular} 
\par\end{centering} 
\vspace{2ex}
\begin{footnotesize} 
\textbf{Notes:} Each cell in the table reports the coefficient associated to a dummy for participation in the Reggio Children Approach infant-toddler or preschool, estimated according to the four different methods explained in section  \ref{sec:identification}.
The control group is represented by any other child-care option in Reggio Emilia, Parma, or Padova.
Outcomes considered: Strength and Difficulties Questionnaire (SDQ), raw score; depression, raw score; dummy for excellent or good self-reported health. Controls considered: adolescent age and gender, poor baseline health (either low-birth weight or premature birth), number of older siblings, mother with college education, father with college education, family with income higher than 25,000 euros, family owns home, distance from the center, religious caregiver, dummy for interview mode (computer vs paper). 
Instruments for infant-toddler center (ITC): caregiver attended ITC as a child, median score in application to RA, cubic polynomial of distance from closest municipal ITC, distance $\times$ mother born in the province, father born in the province.
Instruments for preschool (PS): caregiver attended PS as a child, below median score in application to RA, distance from closest municipal PS, mother born in the province, father born in the province.
\end{footnotesize}
\end{table}


\pagebreak
\bibliographystyle{chicago}
\bibliography{main-files/writeup-bib}

\appendix
\section{Empirical Analysis}
\subsection{Baseline Characteristics}
%% adolescents, infant-toddler
\begin{table}[H]
\caption{Baseline characteristics by infant-toddler-center type, adolescents (age 18)}
\label{tab:adol_PREasilo}
% this is the top part of the tables that display the summary of the baseline characteristics, by city and child-care type
\centering
\begin{adjustbox}{width=1.2\textwidth,center=\textwidth}
\small
\begin{tabular}{m{4.0cm} cccccccccccc}
\hline \hline 
 & Reggio & Reggio & Reggio & Reggio & Parma & Parma & Parma & Parma & Padova & Padova & Padova & Padova \\
 & Municipal & Religious & Private & Not Attended & Municipal & Religious & Private & Not Attended & Municipal & Religious & Private & Not Attended \\

\hline 
  
CAPI & 0.44 & 0.56 & 0.33 & 0.39 & 0.53 & 0.70 & 0.85*** & 0.52 & 0.43 & 0.50 & & 0.50\\
  &  [0.04]  &  [0.18]  &  [0.33]  &  [0.04]  &  [0.05]  &  [0.15]  &  [0.10]  &  [0.04]  &  [0.06]  &  [0.19]  & . & [0.03] \\
Male dummy  &  0.42  &  0.44  &  0.67  &  0.41  &  0.46  &  0.60  &  0.54  &  0.41  &  0.49  &  0.62  &  . & 0.47 \\
  &  [0.04]  &  [0.18]  &  [0.33]  &  [0.04]  &  [0.05]  &  [0.16]  &  [0.14]  &  [0.04]  &  [0.06]  &  [0.18]  & . & [0.03] \\
Age  &  18.71  &  18.83  &  18.96  &  18.69  &  18.84***  &  18.86  &  18.79  &  18.71  &  18.69  &  18.74  &  . & 18.72 \\
  &  [0.03]  &  [0.06]  &  [0.05]  &  [0.03]  &  [0.03]  &  [0.13]  &  [0.11]  &  [0.03]  &  [0.04]  &  [0.11]  & . & [0.02] \\
%Age sq.  &  350.15  &  354.75  &  359.37  &  349.53  &  355.15***  &  355.92  &  353.15  &  350.11  &  349.45  &  351.26  & . & 350.42 \\
%  &  [0.99]  &  [2.16]  &  [1.90]  &  [1.18]  &  [1.23]  &  [4.83]  &  [3.96]  &  [1.12]  &  [1.67]  &  [4.11]  & . & [0.90] \\
Middle school edu mom  &  0.08  &  0.11  &  0.00  &  0.10  &  0.15*  &  0.00  &  0.00  &  0.08  &  0.08  &  0.00  & . & 0.12 \\
  &  [0.02]  &  [0.11]  &  [0.00]  &  [0.03]  &  [0.04]  &  [0.00]  &  [0.00]  &  [0.02]  &  [0.04]  &  [0.00]  & . & [0.02] \\
High school edu mom  &  0.51  &  0.44  &  0.67  &  0.45  &  0.38*  &  0.60  &  0.23*  &  0.50  &  0.39  &  0.62  & . & 0.44 \\
  &  [0.04]  &  [0.18]  &  [0.33]  &  [0.04]  &  [0.05]  &  [0.16]  &  [0.12]  &  [0.04]  &  [0.06]  &  [0.18]  & . & [0.03] \\
University edu mom  &  0.29  &  0.44  &  0.33  &  0.20*  &  0.35  &  0.20  &  0.69***  &  0.30  &  0.46**  &  0.12  & . & 0.25 \\
  &  [0.04]  &  [0.18]  &  [0.33]  &  [0.04]  &  [0.05]  &  [0.13]  &  [0.13]  &  [0.04]  &  [0.06]  &  [0.12]  & . & [0.03] \\
Middle school edu dad  &  0.05  &  0.11  &  0.00  &  0.12*  &  0.10  &  0.00  &  0.00  &  0.06  &  0.13*  &  0.00  & . & 0.10 \\
  &  [0.02]  &  [0.11]  &  [0.00]  &  [0.03]  &  [0.03]  &  [0.00]  &  [0.00]  &  [0.02]  &  [0.04]  &  [0.00]  & . & [0.02] \\
High school edu dad  &  0.42  &  0.33  &  0.67  &  0.38  &  0.35  &  0.60  &  0.15*  &  0.38  &  0.36  &  0.38  & . & 0.40 \\
  &  [0.04]  &  [0.17]  &  [0.33]  &  [0.04]  &  [0.05]  &  [0.16]  &  [0.10]  &  [0.04]  &  [0.06]  &  [0.18]  & . & [0.03] \\
University edu dad  &  0.22  &  0.44  &  0.33  &  0.13*  &  0.25  &  0.10  &  0.54**  &  0.25  &  0.30  &  0.25  & . & 0.28 \\
  &  [0.03]  &  [0.18]  &  [0.33]  &  [0.03]  &  [0.04]  &  [0.10]  &  [0.14]  &  [0.04]  &  [0.06]  &  [0.16]  & . & [0.03] \\
Mom born in province  &  0.75  &  0.67  &  0.33  &  0.61**  &  0.61**  &  0.70  &  0.92  &  0.71  &  0.67  &  0.75  & . & 0.81 \\
  &  [0.04]  &  [0.17]  &  [0.33]  &  [0.04]  &  [0.05]  &  [0.15]  &  [0.08]  &  [0.04]  &  [0.06]  &  [0.16]  & . & [0.03] \\
Dad born in province  &  0.62  &  0.78  &  0.33  &  0.55  &  0.53  &  0.80  &  0.69  &  0.64  &  0.62  &  0.50  & . & 0.78*** \\
  &  [0.04]  &  [0.15]  &  [0.33]  &  [0.04]  &  [0.05]  &  [0.13]  &  [0.13]  &  [0.04]  &  [0.06]  &  [0.19]  & . & [0.03] \\
Caregiver is religious  &  0.74  &  1.00  &  1.00  &  0.79  &  0.86**  &  0.80  &  0.85  &  0.89***  &  0.72  &  0.38**  & . & 0.76 \\
  &  [0.04]  &  [0.00]  &  [0.00]  &  [0.04]  &  [0.04]  &  [0.13]  &  [0.10]  &  [0.03]  &  [0.06]  &  [0.18]  & . & [0.03] \\
Own Home  &  0.89  &  1.00  &  1.00  &  0.78**  &  0.77**  &  0.80  &  1.00  &  0.84  &  0.80  &  0.75  & . & 0.76*** \\
  &  [0.03]  &  [0.00]  &  [0.00]  &  [0.04]  &  [0.04]  &  [0.13]  &  [0.00]  &  [0.03]  &  [0.05]  &  [0.16]  & . & [0.03] \\
Income 5k-10k eur  &  0.01  &  0.00  &  0.00  &  0.01  &  0.00  &  0.00  &  0.00  &  0.02  &  0.00  &  0.00  & . & 0.01 \\
  &  [0.01]  &  [0.00]  &  [0.00]  &  [0.01]  &  [0.00]  &  [0.00]  &  [0.00]  &  [0.01]  &  [0.00]  &  [0.00]  & . & [0.01] \\
Income 10k-25k eur  &  0.16  &  0.22  &  0.33  &  0.20  &  0.19  &  0.00  &  0.23  &  0.16  &  0.11  &  0.00  &  . & 0.10* \\
  &  [0.03]  &  [0.15]  &  [0.33]  &  [0.04]  &  [0.04]  &  [0.00]  &  [0.12]  &  [0.03]  &  [0.04]  &  [0.00]  & . & [0.02] \\
Income 25k-50k eur  &  0.33  &  0.33  &  0.33  &  0.32  &  0.31  &  0.80***  &  0.23  &  0.25  &  0.33  &  0.12  & . & 0.22** \\
  &  [0.04]  &  [0.17]  &  [0.33]  &  [0.04]  &  [0.05]  &  [0.13]  &  [0.12]  &  [0.04]  &  [0.06]  &  [0.12]  & . & [0.03] \\
Income 50k-100k eur  &  0.29  &  0.00  &  0.00  &  0.18*  &  0.28  &  0.20  &  0.38  &  0.21  &  0.15**  &  0.12  & . & 0.10*** \\
  &  [0.04]  &  [0.00]  &  [0.00]  &  [0.03]  &  [0.05]  &  [0.13]  &  [0.14]  &  [0.04]  &  [0.05]  &  [0.12]  & . & [0.02] \\
Income 100k-250k eur  &  0.05  &  0.22*  &  0.00  &  0.03  &  0.02  &  0.00  &  0.00  &  0.05  &  0.00  &  0.00  & . & 0.03 \\
  &  [0.02]  &  [0.15]  &  [0.00]  &  [0.02]  &  [0.01]  &  [0.00]  &  [0.00]  &  [0.02]  &  [0.00]  &  [0.00]  & . & [0.01] \\
%Income more 250k eur  &  0.01  &  0.00  &  0.00  &  0.00  &  0.00  &  0.00  &  0.00  &  0.00  &  0.00  &  0.00  & . & 0.00 \\
%  &  [0.01]  &  [0.00]  &  [0.00]  &  [0.00]  &  [0.00]  &  [0.00]  &  [0.00]  &  [0.00]  &  [0.00]  &  [0.00]  & . & [0.00] \\
%Income below 5k eur  &  0.00  &  0.00  &  0.00  &  0.01  &  0.03*  &  0.00  &  0.00  &  0.02*  &  0.03*  &  0.00  & . & 0.04** \\
%  &  [0.00]  &  [0.00]  &  [0.00]  &  [0.01]  &  [0.02]  &  [0.00]  &  [0.00]  &  [0.01]  &  [0.02]  &  [0.00]  & . & [0.01] \\
Low birthweight  &  0.05  &  0.11  &  0.00  &  0.05  &  0.08  &  0.00  &  0.00  &  0.06  &  0.07  &  0.00  & . & 0.04 \\
  &  [0.02]  &  [0.11]  &  [0.00]  &  [0.02]  &  [0.03]  &  [0.00]  &  [0.00]  &  [0.02]  &  [0.03]  &  [0.00]  & . & [0.01] \\
Premature  &  0.06  &  0.11  &  0.00  &  0.07  &  0.13*  &  0.00  &  0.00  &  0.09  &  0.10  &  0.12  & . & 0.06 \\
 & [0.02] & [0.11] & [0.00] & [0.02] & [0.03] & [0.00] & [0.00] & [0.03] & [0.04] & [0.12] & & [0.02]\\ \hline


% it contains the notes, assuming they are the same for all the tables.
\end{tabular}

\end{adjustbox}
\raggedright{
\footnotesize{Average of baseline characteristcs, by city and type of child-care attended. Standard errors of means in brackets. Test for difference in means between each column and the first column (Reggio Municipal, the treatment group) was performed; *** significant difference at 1\%, ** significant difference at 5\%, * significant difference at 10\%. Source: authors calculation using survey data.}
}
\end{table}
  
%\end{table}

%%% adolescents, preschool
\begin{table}[H]
\caption{Baseline characteristics by preschool type, adolescents (age 18)}
\label{tab:adol_PREmaterna}
% this is the top part of the tables that display the summary of the baseline characteristics, by city and child-care type
\centering
\begin{adjustbox}{width=1.2\textwidth,center=\textwidth}
\small
\begin{tabular}{m{4.0cm} ccccccccccccccc}
\hline \hline 
 & Reggio & Reggio & Reggio & Reggio & Reggio & Parma & Parma & Parma & Parma & Parma & Padova & Padova & Padova & Padova & Padova \\
 & Municipal & State & Religious & Private & Not Attended & Municipal & State & Religious & Private & Not Attended & Municipal & State & Religious & Private & Not Attended \\

\hline 
  
CAPI  &  0.47  &  0.41  &  0.38  &  0.33  &  0.43  &  0.53  &  0.47  &  0.59  &  1.00  &  0.50  &  0.43  &  0.55  &  0.53  &  0.33  &  . \\
  &  [0.04]  &  [0.11]  &  [0.05]  &  [0.21]  &  [0.20]  &  [0.05]  &  [0.08]  &  [0.05]  &  [0.00]  &  [0.29]  &  [0.05]  &  [0.07]  &  [0.04]  &  [0.21]  &  . \\
Male dummy  &  0.42  &  0.55  &  0.40  &  0.50  &  0.57  &  0.40  &  0.42  &  0.52  &  0.67  &  0.50  &  0.44  &  0.45  &  0.50  &  0.50  &  . \\
  &  [0.04]  &  [0.11]  &  [0.05]  &  [0.22]  &  [0.20]  &  [0.05]  &  [0.08]  &  [0.06]  &  [0.21]  &  [0.29]  &  [0.05]  &  [0.07]  &  [0.04]  &  [0.22]  &  . \\
Age  &  18.70  &  18.75  &  18.73  &  18.64  &  18.67  &  18.77  &  18.75  &  18.80*  &  18.81  &  18.59  &  18.75  &  18.82**  &  18.64*  &  18.74  &  . \\
  &  [0.03]  &  [0.07]  &  [0.03]  &  [0.14]  &  [0.17]  &  [0.03]  &  [0.05]  &  [0.04]  &  [0.17]  &  [0.08]  &  [0.04]  &  [0.05]  &  [0.03]  &  [0.18]  &  . \\
%Age sq.  &  349.95  &  351.75  &  350.76  &  347.57  &  348.77  &  352.39  &  351.55  &  353.45*  &  354.07  &  345.74  &  351.60  &  354.38**  &  347.68*  &  351.29  &  . \\
%  &  [0.99]  &  [2.69]  &  [1.27]  &  [5.27]  &  [6.49]  &  [1.21]  &  [2.04]  &  [1.36]  &  [6.50]  &  [3.14]  &  [1.39]  &  [1.88]  &  [1.07]  &  [6.84]  &  . \\
Middle school edu mom  &  0.09  &  0.09  &  0.11  &  0.00  &  0.00  &  0.09  &  0.16  &  0.10  &  0.00  &  0.00  &  0.15  &  0.11  &  0.08  &  0.17  &  . \\
  &  [0.02]  &  [0.06]  &  [0.03]  &  [0.00]  &  [0.00]  &  [0.03]  &  [0.06]  &  [0.03]  &  [0.00]  &  [0.00]  &  [0.04]  &  [0.05]  &  [0.02]  &  [0.17]  &  . \\
High school edu mom  &  0.51  &  0.36  &  0.42  &  0.67  &  0.71  &  0.44  &  0.47  &  0.45  &  0.00  &  0.75  &  0.39*  &  0.43  &  0.45  &  0.50  &  . \\
  &  [0.04]  &  [0.10]  &  [0.05]  &  [0.21]  &  [0.18]  &  [0.05]  &  [0.08]  &  [0.06]  &  [0.00]  &  [0.25]  &  [0.05]  &  [0.07]  &  [0.04]  &  [0.22]  &  . \\
University edu mom  &  0.22  &  0.27  &  0.31  &  0.17  &  0.00  &  0.37***  &  0.16  &  0.33*  &  1.00  &  0.25  &  0.31  &  0.34  &  0.27  &  0.33  &  . \\
  &  [0.03]  &  [0.10]  &  [0.05]  &  [0.17]  &  [0.00]  &  [0.05]  &  [0.06]  &  [0.05]  &  [0.00]  &  [0.25]  &  [0.05]  &  [0.07]  &  [0.04]  &  [0.21]  &  . \\
Middle school edu dad  &  0.07  &  0.09  &  0.10  &  0.33*  &  0.00  &  0.09  &  0.12  &  0.02  &  0.00  &  0.00  &  0.11  &  0.13  &  0.08  &  0.17  &  . \\
  &  [0.02]  &  [0.06]  &  [0.03]  &  [0.21]  &  [0.00]  &  [0.03]  &  [0.05]  &  [0.02]  &  [0.00]  &  [0.00]  &  [0.03]  &  [0.05]  &  [0.02]  &  [0.17]  &  . \\
High school edu dad  &  0.40  &  0.41  &  0.40  &  0.50  &  0.14  &  0.35  &  0.40  &  0.38  &  0.17  &  0.25  &  0.32  &  0.47  &  0.41  &  0.17  &  . \\
  &  [0.04]  &  [0.11]  &  [0.05]  &  [0.22]  &  [0.14]  &  [0.04]  &  [0.08]  &  [0.05]  &  [0.17]  &  [0.25]  &  [0.05]  &  [0.07]  &  [0.04]  &  [0.17]  &  . \\
University edu dad  &  0.19  &  0.09  &  0.21  &  0.00  &  0.43  &  0.24  &  0.16  &  0.32**  &  0.50*  &  0.25  &  0.31**  &  0.23  &  0.27  &  0.50*  &  . \\
  &  [0.03]  &  [0.06]  &  [0.04]  &  [0.00]  &  [0.20]  &  [0.04]  &  [0.06]  &  [0.05]  &  [0.22]  &  [0.25]  &  [0.05]  &  [0.06]  &  [0.04]  &  [0.22]  &  . \\
Mom born in province&  0.72  &  0.50**  &  0.64  &  0.83  &  0.57  &  0.68  &  0.74  &  0.66  &  0.67  &  0.25*  &  0.71  &  0.77  &  0.82*  &  1.00  & . \\
  &  [0.03]  &  [0.11]  &  [0.05]  &  [0.17]  &  [0.20]  &  [0.04]  &  [0.07]  &  [0.05]  &  [0.21]  &  [0.25]  &  [0.05]  &  [0.06]  &  [0.03]  &  [0.00]  &  . \\
Dad born in province&  0.61  &  0.45  &  0.57  &  0.67  &  0.43  &  0.58  &  0.70  &  0.63  &  0.50  &  0.00  &  0.65  &  0.72  &  0.79***  &  0.67  &  . \\
  &  [0.04]  &  [0.11]  &  [0.05]  &  [0.21]  &  [0.20]  &  [0.05]  &  [0.07]  &  [0.05]  &  [0.22]  &  [0.00]  &  [0.05]  &  [0.07]  &  [0.04]  &  [0.21]  &  . \\
Caregiver is religious  &  0.69  &  0.73  &  0.91***  &  1.00  &  0.86  &  0.88***  &  0.86**  &  0.87***  &  0.83  &  0.75  &  0.77  &  0.62  &  0.78*  &  0.50  &  . \\
  &  [0.04]  &  [0.10]  &  [0.03]  &  [0.00]  &  [0.14]  &  [0.03]  &  [0.05]  &  [0.04]  &  [0.17]  &  [0.25]  &  [0.04]  &  [0.07]  &  [0.04]  &  [0.22]  &  . \\
Own Home  &  0.86  &  0.77  &  0.85  &  0.83  &  0.43**  &  0.79  &  0.81  &  0.83  &  0.83  &  1.00  &  0.75**  &  0.64***  &  0.82  &  0.83  &  . \\
  &  [0.03]  &  [0.09]  &  [0.04]  &  [0.17]  &  [0.20]  &  [0.04]  &  [0.06]  &  [0.04]  &  [0.17]  &  [0.00]  &  [0.04]  &  [0.07]  &  [0.03]  &  [0.17]  &  . \\
Income 5k-10k eur  &  0.01  &  0.00  &  0.01  &  0.00  &  0.00  &  0.01  &  0.00  &  0.00  &  0.00  &  0.00  &  0.00  &  0.02  &  0.01  &  0.00  &  . \\
  &  [0.01]  &  [0.00]  &  [0.01]  &  [0.00]  &  [0.00]  &  [0.01]  &  [0.00]  &  [0.00]  &  [0.00]  &  [0.00]  &  [0.00]  &  [0.02]  &  [0.01]  &  [0.00]  &  . \\
Income 10k-25k eur  &  0.17  &  0.36**  &  0.17  &  0.00  &  0.29  &  0.16  &  0.28  &  0.12  &  0.50*  &  0.25  &  0.12  &  0.09  &  0.11  &  0.00  &  . \\
  &  [0.03]  &  [0.10]  &  [0.04]  &  [0.00]  &  [0.18]  &  [0.03]  &  [0.07]  &  [0.04]  &  [0.22]  &  [0.25]  &  [0.03]  &  [0.04]  &  [0.03]  &  [0.00]  &  . \\
Income 25k-50k eur  &  0.32  &  0.27  &  0.34  &  0.17  &  0.43  &  0.25  &  0.37  &  0.30  &  0.50  &  0.25  &  0.30  &  0.09***  &  0.27  &  0.17  &  . \\
  &  [0.04]  &  [0.10]  &  [0.05]  &  [0.17]  &  [0.20]  &  [0.04]  &  [0.07]  &  [0.05]  &  [0.22]  &  [0.25]  &  [0.05]  &  [0.04]  &  [0.04]  &  [0.17]  &  . \\
Income 50k-100k eur  &  0.25  &  0.09  &  0.25  &  0.33  &  0.00  &  0.26  &  0.23  &  0.26  &  0.00  &  0.00  &  0.10***  &  0.13*  &  0.11***  &  0.17  &  . \\
  &  [0.03]  &  [0.06]  &  [0.04]  &  [0.21]  &  [0.00]  &  [0.04]  &  [0.07]  &  [0.05]  &  [0.00]  &  [0.00]  &  [0.03]  &  [0.05]  &  [0.03]  &  [0.17]  &  . \\
Income 100k-250k eur  &  0.04  &  0.00  &  0.06  &  0.00  &  0.00  &  0.03  &  0.00  &  0.05  &  0.00  &  0.00  &  0.01  &  0.00  &  0.05  &  0.00  &  . \\
  &  [0.02]  &  [0.00]  &  [0.02]  &  [0.00]  &  [0.00]  &  [0.01]  &  [0.00]  &  [0.02]  &  [0.00]  &  [0.00]  &  [0.01]  &  [0.00]  &  [0.02]  &  [0.00]  &  . \\
%Income more 250k eur  &  0.00  &  0.00  &  0.01  &  0.00  &  0.00  &  0.00  &  0.00  &  0.00  &  0.00  &  0.00  &  0.00  &  0.00  &  0.00  &  0.00  &  . \\
%  &  [0.00]  &  [0.00]  &  [0.01]  &  [0.00]  &  [0.00]  &  [0.00]  &  [0.00]  &  [0.00]  &  [0.00]  &  [0.00]  &  [0.00]  &  [0.00]  &  [0.00]  &  [0.00]  &  . \\
%Income below 5k eur  &  0.00  &  0.00  &  0.01  &  0.00  &  0.00  &  0.03*  &  0.02  &  0.02  &  0.00  &  0.00  &  0.03**  &  0.02  &  0.05***  &  0.00  &  . \\
%  &  [0.00]  &  [0.00]  &  [0.01]  &  [0.00]  &  [0.00]  &  [0.01]  &  [0.02]  &  [0.02]  &  [0.00]  &  [0.00]  &  [0.02]  &  [0.02]  &  [0.02]  &  [0.00]  &  . \\
Low birthweight  &  0.05  &  0.00  &  0.05  &  0.17  &  0.14  &  0.06  &  0.05  &  0.07  &  0.00  &  0.25  &  0.06  &  0.06  &  0.02  &  0.00  &  . \\
  &  [0.02]  &  [0.00]  &  [0.02]  &  [0.17]  &  [0.14]  &  [0.02]  &  [0.03]  &  [0.03]  &  [0.00]  &  [0.25]  &  [0.03]  &  [0.04]  &  [0.01]  &  [0.00]  &  . \\
Premature  &  0.04  &  0.09  &  0.08  &  0.17  &  0.14  &  0.09*  &  0.07  &  0.12**  &  0.00  &  0.25  &  0.10  &  0.09  &  0.04  &  0.00  &  . \\
  &  [0.02]  &  [0.06]  &  [0.03]  &  [0.17]  &  [0.14]  &  [0.03]  &  [0.04]  &  [0.04]  &  [0.00]  &  [0.25]  &  [0.03]  &  [0.04]  &  [0.02]  &  [0.00]  &  . \\

\hline
% it contains the notes, assuming they are the same for all the tables.
\end{tabular}

\end{adjustbox}
\raggedright{
\footnotesize{Average of baseline characteristcs, by city and type of child-care attended. Standard errors of means in brackets. Test for difference in means between each column and the first column (Reggio Municipal, the treatment group) was performed; *** significant difference at 1\%, ** significant difference at 5\%, * significant difference at 10\%. Source: authors calculation using survey data.}
}
\end{table}
  
%\end{table}

%\section{Appendix: Analysis Robustness Checks} 
\label{app:analysis}
This section reports the long-form tables of the preliminary results of the analysis. 

\subsection{Regression Results}
\label{app:OLS}

As in the main section of the paper, we ran the following regression for each city and age group:
\[ 
y = \alpha + \sum_{sc} \delta_{sc} D_{sc} + \beta_{X}X + \varepsilon
\]

in this case, $D_{sc}$ are dummies for city-specific school type including: municipal, religious, state, and private schools in Reggio, Parma, and Padova as well as dummies for not-attended any type of school.\footnote{Note that there are no state infant-toddler centers, and virtually every child and adolescent attended some form of preschool.} The omitted category of comparison for every city is municipal schools; since the treated group are the municipal schools in Reggio Emilia, the coefficients for the city of Reggio Emilia can be read as the difference between the treatment group and the specific type of school attended.

As mentioned before, the age groups considered are 0-3, regarding attendance to infant-toddler center, and age 3-6, for attendance to preschool. 

The following controls are considered incrementally through all the analysis: the first column of each table only controls for school types $D_{sc}$ and city dummies; the second column introduces interviewer fixed effects and interview mode (computer or paper); the third column introduces demographic controls such as gender, age at interview, age$^2$, dummies for parental education, dummies for parents born in the region (province), dummies for family owns home, family income brackets; the fourth column includes also dummies for low birthweight, and premature birth (when present)\footnote{Recall that adults did not report information on prematurity}; the fifth column interacts all of the controls with city-dummies, allowing the effect of controls to vary by city; the sixth column ran the regression only for the sample of respondents from Reggio Emilia; the final column does not control for interviewer fixed effects.


\singlespacing
\subsubsection{Children Results}
\begin{small}
\begin{table}[H]
\caption{Pool Regression: Child Health - Infant-toddler center, Child}
\begin{tabular}{lccccccc} \hline
 & (1) & (2) & (3) & (4) & (5) & (6) & (7) \\
VARIABLES &  &  &  &  &  &  &  \\ \hline
 &  &  &  &  &  &  &  \\
Reggio None ITC & 0.048 & 0.031 & 0.038 & 0.033 & 0.030 & 0.030 & 0.041 \\
 & [0.059] & [0.064] & [0.066] & [0.067] & [0.070] & [0.069] & [0.062] \\
Reggio Reli ITC & 0.072 & 0.075 & 0.085 & 0.080 & 0.066 & 0.066 & 0.080 \\
 & [0.094] & [0.098] & [0.095] & [0.095] & [0.098] & [0.097] & [0.091] \\
Reggio Priv ITC & 0.131 & 0.145 & 0.131 & 0.140 & 0.062 & 0.062 & 0.129 \\
 & [0.185] & [0.200] & [0.180] & [0.166] & [0.161] & [0.158] & [0.153] \\
Parma None ITC & -0.040 & 0.051 & -0.012 & -0.011 & 0.027 &  & -0.027 \\
 & [0.065] & [0.074] & [0.101] & [0.102] & [0.111] &  & [0.067] \\
Parma Muni ITC & -0.071 & 0.021 & -0.045 & -0.047 & -0.027 &  & -0.059 \\
 & [0.059] & [0.075] & [0.107] & [0.108] & [0.104] &  & [0.060] \\
Parma Reli ITC & 0.081 & 0.148 & 0.057 & 0.045 & -0.047 &  & 0.042 \\
 & [0.222] & [0.240] & [0.265] & [0.264] & [0.313] &  & [0.235] \\
Parma Priv ITC & -0.069 & 0.036 & -0.031 & -0.028 &  &  & -0.055 \\
 & [0.099] & [0.120] & [0.144] & [0.145] &  &  & [0.099] \\
Padova None ITC & 0.067 & -0.458 & -0.553 & -0.555 & 0.048 &  & 0.055 \\
 & [0.056] & [0.362] & [0.354] & [0.350] & [0.102] &  & [0.062] \\
Padova Muni ITC & 0.146** & -0.377 & -0.488 & -0.483 & 0.124 &  & 0.132* \\
 & [0.067] & [0.369] & [0.361] & [0.358] & [0.104] &  & [0.070] \\
Padova Reli ITC & 0.070 & -0.338 & -0.444 & -0.447 & 0.179 &  & 0.065 \\
 & [0.101] & [0.374] & [0.367] & [0.364] & [0.126] &  & [0.104] \\
Padova Priv ITC & 0.021 & -0.496 & -0.618* & -0.618* &  &  & -0.005 \\
 & [0.095] & [0.377] & [0.369] & [0.366] &  &  & [0.095] \\
Male dummy &  &  & -0.065* & -0.064* & -0.116** & -0.116** & -0.073** \\
 &  &  & [0.035] & [0.035] & [0.059] & [0.058] & [0.034] \\
Constant & 0.669*** & 0.948*** & 7.500 & 7.870 & 5.166 & 19.800** & 7.994 \\
 & [0.040] & [0.075] & [5.404] & [5.410] & [12.958] & [8.843] & [5.350] \\
 &  &  &  &  &  &  &  \\
Observations & 765 & 765 & 765 & 765 & 765 & 280 & 765 \\
R-squared & 0.019 & 0.061 & 0.086 & 0.087 & 0.182 & 0.142 & 0.049 \\
 Controls & None & F.E. & Family & All & Inter & Reggio & no FE \\ \hline
\multicolumn{8}{c}{ Robust standard errors in brackets} \\
\multicolumn{8}{c}{ *** p$<$0.01, ** p$<$0.05, * p$<$0.1} \\
\multicolumn{8}{c}{ Dependent variable: Child health is good (\%) - mom report.} \\
\end{tabular}
  
\end{table}
\begin{table}[H]
\caption{Pool Regression: Child Health - Preschool, Child}
\begin{tabular}{lccccccc} \hline
 & (1) & (2) & (3) & (4) & (5) & (6) & (7) \\
VARIABLES &  &  &  &  &  &  &  \\ \hline
 &  &  &  &  &  &  &  \\
Reggio None Preschool & -0.187 & -0.112 & -0.169 & -0.177 & -0.292 & -0.292 & -0.277 \\
 & [0.359] & [0.375] & [0.387] & [0.392] & [0.391] & [0.384] & [0.381] \\
Reggio Stat Preschool & 0.031 & 0.008 & 0.001 & -0.001 & -0.019 & -0.019 & 0.022 \\
 & [0.082] & [0.084] & [0.085] & [0.085] & [0.092] & [0.090] & [0.083] \\
Reggio Reli Preschool & 0.010 & 0.007 & 0.025 & 0.023 & 0.013 & 0.013 & 0.025 \\
 & [0.063] & [0.064] & [0.064] & [0.064] & [0.064] & [0.063] & [0.063] \\
Reggio Priv Preschool & 0.313*** & 0.275*** & 0.256*** & 0.252*** & 0.311*** & 0.311*** & 0.273*** \\
 & [0.039] & [0.055] & [0.074] & [0.074] & [0.114] & [0.112] & [0.060] \\
Parma None Preschool & -0.187 &  &  &  & -0.330 &  & -0.180 \\
 & [0.210] &  &  &  & [0.258] &  & [0.232] \\
Parma Muni Preschool & -0.101* & 0.096 & 0.101 & 0.105 & -0.203 &  & -0.089 \\
 & [0.058] & [0.228] & [0.247] & [0.247] & [0.153] &  & [0.059] \\
Parma Stat Preschool & -0.072 & 0.121 & 0.136 & 0.149 & -0.151 &  & -0.043 \\
 & [0.088] & [0.238] & [0.257] & [0.258] & [0.170] &  & [0.092] \\
Parma Reli Preschool & -0.036 & 0.127 & 0.146 & 0.156 & -0.164 &  & -0.017 \\
 & [0.072] & [0.232] & [0.252] & [0.253] & [0.159] &  & [0.072] \\
Parma Priv Preschool & 0.091 & 0.252 & 0.291 & 0.309 &  &  & 0.129 \\
 & [0.145] & [0.272] & [0.286] & [0.284] &  &  & [0.142] \\
Padova None Preschool & 0.313*** & -0.166 & -0.339 & -0.347 & -7.067 &  & 0.249*** \\
 & [0.039] & [0.383] & [0.390] & [0.387] & [15.605] &  & [0.085] \\
Padova Muni Preschool & 0.088 & -0.476 & -0.570 & -0.566 & -7.166 &  & 0.087 \\
 & [0.063] & [0.372] & [0.369] & [0.366] & [15.599] &  & [0.065] \\
Padova Stat Preschool & 0.131* & -0.443 & -0.567 & -0.562 & -7.220 &  & 0.098 \\
 & [0.078] & [0.377] & [0.373] & [0.370] & [15.590] &  & [0.080] \\
Padova Reli Preschool & 0.019 & -0.508 & -0.613* & -0.611* & -7.232 &  & 0.011 \\
 & [0.057] & [0.372] & [0.370] & [0.367] & [15.591] &  & [0.060] \\
Padova Priv Preschool & 0.113 & -0.447 & -0.562 & -0.559 & -7.134 &  & 0.086 \\
 & [0.133] & [0.396] & [0.394] & [0.392] & [15.601] &  & [0.129] \\
Male dummy &  &  & -0.065* & -0.064* & -0.117** & -0.117** & -0.076** \\
 &  &  & [0.035] & [0.035] & [0.058] & [0.057] & [0.035] \\
Constant & 0.687*** & 0.984*** & 7.972 & 8.370 & 6.704 & 19.082** & 8.197 \\
 & [0.039] & [0.036] & [5.477] & [5.480] & [12.932] & [8.801] & [5.414] \\
 &  &  &  &  &  &  &  \\
Observations & 765 & 765 & 765 & 765 & 765 & 280 & 765 \\
R-squared & 0.023 & 0.061 & 0.086 & 0.088 & 0.185 & 0.148 & 0.054 \\
 Controls & None & F.E. & Family & All & Inter & Reggio & no FE \\ \hline
\multicolumn{8}{c}{ Robust standard errors in brackets} \\
\multicolumn{8}{c}{ *** p$<$0.01, ** p$<$0.05, * p$<$0.1} \\
\multicolumn{8}{c}{ Dependent variable: Child health is good (\%) - mom report.} \\
\end{tabular}

\end{table}
\begin{table}[H]
\caption{Pool Regression: Child SDQ Score - Infant-toddler center, Child}
\begin{tabular}{lccccccc} \hline
 & (1) & (2) & (3) & (4) & (5) & (6) & (7) \\
VARIABLES &  &  &  &  &  &  &  \\ \hline
 &  &  &  &  &  &  &  \\
Reggio None ITC & 1.354** & 1.480*** & 1.098* & 1.119* & 1.168* & 1.168* & 1.118** \\
 & [0.559] & [0.573] & [0.583] & [0.585] & [0.639] & [0.628] & [0.568] \\
Reggio Reli ITC & 0.115 & 0.168 & 0.401 & 0.376 & 0.592 & 0.592 & 0.337 \\
 & [0.846] & [0.855] & [0.823] & [0.821] & [0.812] & [0.797] & [0.813] \\
Reggio Priv ITC & -0.470 & -0.249 & 0.560 & 0.570 & 1.473 & 1.473 & 0.543 \\
 & [1.633] & [1.666] & [1.493] & [1.499] & [1.412] & [1.386] & [1.517] \\
Parma None ITC & 0.312 & -1.268 & -0.908 & -0.854 & 0.954 &  & 0.848 \\
 & [0.538] & [1.611] & [1.658] & [1.679] & [0.656] &  & [0.524] \\
Parma Muni ITC & 0.111 & -1.377 & -1.002 & -0.977 & 0.901 &  & 0.713 \\
 & [0.482] & [1.584] & [1.617] & [1.637] & [0.585] &  & [0.463] \\
Parma Reli ITC & 2.180 &  &  &  & 2.018 &  & 2.254 \\
 & [1.678] &  &  &  & [2.020] &  & [1.766] \\
Parma Priv ITC & -0.937 & -2.366 & -1.832 & -1.868 &  &  & -0.194 \\
 & [0.630] & [1.650] & [1.679] & [1.697] &  &  & [0.607] \\
Padova None ITC & 0.183 & 6.416*** & 4.075*** & 4.105*** & -138.864 &  & 0.486 \\
 & [0.468] & [1.719] & [1.048] & [1.067] & [133.950] &  & [0.505] \\
Padova Muni ITC & 1.022 & 7.570*** & 5.254*** & 5.247*** & -137.353 &  & 1.559** \\
 & [0.703] & [1.809] & [1.252] & [1.271] & [133.921] &  & [0.729] \\
Padova Reli ITC & 1.364 & 7.945*** & 5.211*** & 5.271*** & -137.258 &  & 1.846 \\
 & [1.098] & [2.122] & [1.648] & [1.653] & [133.709] &  & [1.135] \\
Padova Priv ITC & 0.343 & 7.154*** & 5.085*** & 5.155*** & -137.459 &  & 1.170 \\
 & [0.962] & [1.705] & [1.481] & [1.498] & [134.051] &  & [0.922] \\
Male dummy &  &  & 0.588* & 0.564* & -0.059 & -0.059 & 0.540* \\
 &  &  & [0.313] & [0.313] & [0.523] & [0.514] & [0.309] \\
Constant & 8.070*** & 0.911 & 108.255** & 107.187** & 216.805** & 83.576 & 126.418** \\
 & [0.316] & [0.660] & [50.439] & [50.543] & [96.409] & [88.295] & [51.622] \\
 &  &  &  &  &  &  &  \\
Observations & 766 & 766 & 766 & 766 & 766 & 280 & 766 \\
R-squared & 0.019 & 0.088 & 0.161 & 0.162 & 0.227 & 0.219 & 0.100 \\
 Controls & None & F.E. & Family & All & Inter & Reggio & no FE \\ \hline
\multicolumn{8}{c}{ Robust standard errors in brackets} \\
\multicolumn{8}{c}{ *** p$<$0.01, ** p$<$0.05, * p$<$0.1} \\
\multicolumn{8}{c}{ Dependent variable: SDQ score (mom rep.).} \\
\end{tabular}
  
\end{table}
\begin{table}[H]
\caption{Pool Regression: Child SDQ Score - Preschool, Child}
\begin{tabular}{lccccccc} \hline
 & (1) & (2) & (3) & (4) & (5) & (6) & (7) \\
VARIABLES &  &  &  &  &  &  &  \\ \hline
 &  &  &  &  &  &  &  \\
Reggio None Preschool & 1.939 & 2.009 & 1.421 & 1.463 & 1.211 & 1.211 & 1.126 \\
 & [2.874] & [2.712] & [2.127] & [2.125] & [1.666] & [1.636] & [2.085] \\
Reggio Stat Preschool & 1.682* & 1.775* & 1.483* & 1.485* & 1.623* & 1.623* & 1.622* \\
 & [0.973] & [0.987] & [0.894] & [0.892] & [0.870] & [0.854] & [0.885] \\
Reggio Reli Preschool & 0.928* & 0.931* & 1.004* & 0.986* & 1.167** & 1.167** & 1.000* \\
 & [0.533] & [0.538] & [0.537] & [0.535] & [0.548] & [0.538] & [0.524] \\
Reggio Priv Preschool & -1.728*** & -1.991*** & -1.065* & -1.095* & -0.687 & -0.687 & -0.828 \\
 & [0.633] & [0.607] & [0.626] & [0.628] & [1.217] & [1.195] & [0.520] \\
Parma None Preschool & 0.605 & 6.842*** & 6.493*** & 6.498*** & 79.990 &  & 1.746 \\
 & [1.417] & [2.194] & [2.162] & [2.157] & [137.036] &  & [1.331] \\
Parma Muni Preschool & 0.006 & 6.622*** & 5.763*** & 5.781*** & 79.589 &  & 0.683 \\
 & [0.469] & [1.755] & [1.777] & [1.775] & [137.037] &  & [0.462] \\
Parma Stat Preschool & -1.061 & 5.715*** & 5.265*** & 5.277*** & 78.989 &  & -0.056 \\
 & [0.666] & [1.805] & [1.813] & [1.814] & [137.082] &  & [0.654] \\
Parma Reli Preschool & 0.907 & 7.186*** & 6.043*** & 6.035*** & 79.935 &  & 1.237** \\
 & [0.617] & [1.706] & [1.734] & [1.731] & [137.006] &  & [0.610] \\
Parma Priv Preschool & 0.494 & 6.645*** & 6.173*** & 6.096*** & 79.613 &  & 1.284 \\
 & [0.966] & [1.650] & [1.663] & [1.657] & [137.079] &  & [0.894] \\
Padova None Preschool & 4.439*** & 7.260*** & 6.577*** & 6.557*** & 2.299 &  & 4.390** \\
 & [1.116] & [1.813] & [2.197] & [2.223] & [2.274] &  & [2.025] \\
Padova Muni Preschool & 1.080* & 3.458** & 3.573*** & 3.605*** & -1.233 &  & 1.636** \\
 & [0.643] & [1.382] & [1.254] & [1.296] & [1.532] &  & [0.663] \\
Padova Stat Preschool & 0.636 & 2.417 & 2.289 & 2.288 & -2.561 &  & 0.872 \\
 & [0.710] & [1.534] & [1.440] & [1.476] & [1.607] &  & [0.724] \\
Padova Reli Preschool & 0.005 & 2.182 & 2.322* & 2.311* & -2.490* &  & 0.557 \\
 & [0.493] & [1.392] & [1.304] & [1.343] & [1.478] &  & [0.501] \\
Padova Priv Preschool & 1.539 & 3.771* & 4.775** & 4.825** &  &  & 2.885 \\
 & [1.761] & [2.033] & [2.035] & [2.060] &  &  & [1.893] \\
Male dummy &  &  & 0.608** & 0.586* & -0.136 & -0.136 & 0.564* \\
 &  &  & [0.306] & [0.307] & [0.518] & [0.508] & [0.301] \\
Constant & 8.061*** & 2.360*** & 122.007** & 121.560** & 136.878 & 85.444 & 135.106*** \\
 & [0.314] & [0.338] & [50.609] & [50.664] & [97.316] & [88.208] & [51.471] \\
 &  &  &  &  &  &  &  \\
Observations & 766 & 766 & 766 & 766 & 766 & 280 & 766 \\
R-squared & 0.026 & 0.092 & 0.167 & 0.169 & 0.233 & 0.228 & 0.108 \\
 Controls & None & F.E. & Family & All & Inter & Reggio & no FE \\ \hline
\multicolumn{8}{c}{ Robust standard errors in brackets} \\
\multicolumn{8}{c}{ *** p$<$0.01, ** p$<$0.05, * p$<$0.1} \\
\multicolumn{8}{c}{ Dependent variable: SDQ score (mom rep.).} \\
\end{tabular}

\end{table}
\end{small}

\pagebreak
\subsubsection{Adolescents Results}
\begin{table}[H]
\caption{Pool Regression: Child Health - Infant-toddler center, Adolescent}
\begin{tabular}{lccccccc} \hline
 & (1) & (2) & (3) & (4) & (5) & (6) & (7) \\
VARIABLES &  &  &  &  &  &  &  \\ \hline
 &  &  &  &  &  &  &  \\
Reggio None ITC & -0.059 & -0.078 & -0.069 & -0.069 & -0.046 & -0.046 & -0.049 \\
 & [0.063] & [0.066] & [0.068] & [0.068] & [0.073] & [0.071] & [0.066] \\
Reggio Reli ITC & 0.067 & 0.066 & 0.044 & 0.051 & 0.074 & 0.074 & 0.040 \\
 & [0.160] & [0.152] & [0.155] & [0.157] & [0.176] & [0.173] & [0.174] \\
Reggio Priv ITC & 0.317*** & 0.289*** & 0.264*** & 0.256*** & 0.257** & 0.257** & 0.289*** \\
 & [0.043] & [0.051] & [0.086] & [0.085] & [0.120] & [0.117] & [0.083] \\
Parma None ITC & -0.085 & -0.133 & -0.134 & -0.127 & 0.256* &  & -0.098 \\
 & [0.066] & [0.159] & [0.171] & [0.171] & [0.153] &  & [0.068] \\
Parma Muni ITC & -0.253*** & -0.304* & -0.324* & -0.314* & 0.036 &  & -0.268*** \\
 & [0.073] & [0.162] & [0.175] & [0.176] & [0.159] &  & [0.075] \\
Parma Reli ITC & -0.183 & -0.298 & -0.310 & -0.310 &  &  & -0.215 \\
 & [0.165] & [0.216] & [0.221] & [0.221] &  &  & [0.165] \\
Parma Priv ITC & 0.094 &  &  &  & 0.354* &  & 0.052 \\
 & [0.146] &  &  &  & [0.211] &  & [0.153] \\
Padova None ITC & 0.049 & -0.470 & -0.994*** & -0.385 & -120.971 &  & 0.004 \\
 & [0.055] & [0.432] & [0.261] & [0.470] & [128.813] &  & [0.060] \\
Padova Muni ITC & -0.088 & -0.563 & -1.105*** & -0.502 & -121.098 &  & -0.126 \\
 & [0.084] & [0.442] & [0.277] & [0.477] & [128.809] &  & [0.088] \\
Padova Reli ITC & 0.317*** & -0.249 & -0.843*** & -0.236 & -120.815 &  & 0.193** \\
 & [0.043] & [0.439] & [0.280] & [0.477] & [128.816] &  & [0.076] \\
Male dummy &  &  & 0.054 & 0.053 & 0.084 & 0.084 & 0.054 \\
 &  &  & [0.038] & [0.038] & [0.067] & [0.066] & [0.038] \\
Constant & 0.683*** & 0.970*** & -12.310 & -10.567 & 50.784 & -46.773 & -11.766 \\
 & [0.043] & [0.232] & [51.411] & [51.421] & [98.954] & [87.909] & [50.479] \\
 &  &  &  &  &  &  &  \\
Observations & 652 & 652 & 652 & 652 & 652 & 240 & 652 \\
R-squared & 0.044 & 0.114 & 0.139 & 0.142 & 0.211 & 0.111 & 0.081 \\
 Controls & None & F.E. & Family & All & Inter & Reggio & no FE \\ \hline
\multicolumn{8}{c}{ Robust standard errors in brackets} \\
\multicolumn{8}{c}{ *** p$<$0.01, ** p$<$0.05, * p$<$0.1} \\
\multicolumn{8}{c}{ Dependent variable: Child health is good (\%) - mom report.} \\
\end{tabular}
  
\end{table}
\begin{table}[H]
\caption{Pool Regression: Child Health - Preschool, Adolescent}
\begin{tabular}{lccccccc} \hline
 & (1) & (2) & (3) & (4) & (5) & (6) & (7) \\
VARIABLES &  &  &  &  &  &  &  \\ \hline
 &  &  &  &  &  &  &  \\
Reggio None Preschool & 0.025 & 0.041 & 0.036 & 0.052 & 0.101 & 0.101 & 0.037 \\
 & [0.177] & [0.189] & [0.193] & [0.185] & [0.215] & [0.210] & [0.165] \\
Reggio Stat Preschool & 0.061 & 0.064 & 0.086 & 0.101 & 0.095 & 0.095 & 0.099 \\
 & [0.117] & [0.124] & [0.127] & [0.126] & [0.133] & [0.130] & [0.124] \\
Reggio Reli Preschool & -0.064 & -0.047 & -0.044 & -0.036 & -0.048 & -0.048 & -0.032 \\
 & [0.068] & [0.070] & [0.072] & [0.072] & [0.080] & [0.078] & [0.071] \\
Reggio Priv Preschool & -0.489*** & -0.528*** & -0.549*** & -0.524*** & -0.534** & -0.534** & -0.503*** \\
 & [0.186] & [0.195] & [0.199] & [0.196] & [0.212] & [0.208] & [0.189] \\
Parma None Preschool & 0.311*** & 2.075*** & 2.132*** & 0.203 &  &  & 0.386*** \\
 & [0.041] & [0.209] & [0.241] & [0.203] &  &  & [0.093] \\
Parma Muni Preschool & -0.184*** & 1.534*** & 1.615*** & -0.367** & -0.740*** &  & -0.194*** \\
 & [0.068] & [0.196] & [0.225] & [0.155] & [0.178] &  & [0.071] \\
Parma Stat Preschool & -0.336*** & 1.413*** & 1.506*** & -0.486*** & -0.844*** &  & -0.340*** \\
 & [0.092] & [0.212] & [0.241] & [0.173] & [0.201] &  & [0.093] \\
Parma Reli Preschool & -0.039 & 1.662*** & 1.746*** & -0.230 & -0.601*** &  & -0.045 \\
 & [0.073] & [0.189] & [0.214] & [0.149] & [0.179] &  & [0.076] \\
Parma Priv Preschool & 0.311*** & 1.898*** & 1.986*** &  & -0.278 &  & 0.254** \\
 & [0.041] & [0.128] & [0.219] &  & [0.270] &  & [0.103] \\
Padova None Preschool & -0.689*** &  &  & -0.852*** &  &  & -0.696*** \\
 & [0.041] &  &  & [0.242] &  &  & [0.127] \\
Padova Muni Preschool & -0.042 & 0.853*** & 0.915*** & 0.062 & 0.892*** &  & -0.070 \\
 & [0.070] & [0.085] & [0.094] & [0.238] & [0.111] &  & [0.074] \\
Padova Stat Preschool & 0.172** & 1.000 & 1.047*** & 0.192 & 1.040*** &  & 0.108 \\
 & [0.071] & [.] & [0.039] & [0.236] & [0.062] &  & [0.074] \\
Padova Reli Preschool & 0.016 & 0.852*** & 0.924*** & 0.070 & 0.895*** &  & -0.019 \\
 & [0.060] & [0.075] & [0.083] & [0.236] & [0.102] &  & [0.062] \\
Padova Priv Preschool & 0.061 & 0.855*** & 0.864*** &  & 0.762*** &  & -0.054 \\
 & [0.223] & [0.217] & [0.244] &  & [0.266] &  & [0.232] \\
Male dummy &  &  & 0.047 & 0.045 & 0.068 & 0.068 & 0.048 \\
 &  &  & [0.039] & [0.039] & [0.069] & [0.067] & [0.038] \\
Constant & 0.689*** & -0.868*** & -21.141 & -18.282 & 94.588 & -78.501 & -17.041 \\
 & [0.041] & [0.079] & [51.692] & [51.555] & [99.799] & [88.216] & [50.533] \\
 &  &  &  &  &  &  &  \\
Observations & 652 & 652 & 652 & 652 & 652 & 240 & 652 \\
R-squared & 0.061 & 0.125 & 0.148 & 0.153 & 0.222 & 0.131 & 0.099 \\
 Controls & None & F.E. & Family & All & Inter & Reggio & no FE \\ \hline
\multicolumn{8}{c}{ Robust standard errors in brackets} \\
\multicolumn{8}{c}{ *** p$<$0.01, ** p$<$0.05, * p$<$0.1} \\
\multicolumn{8}{c}{ Dependent variable: Child health is good (\%) - mom report.} \\
\end{tabular}

\end{table}
\begin{table}[H]
\caption{Pool Regression: Health - Infant-toddler center, Adolescent}
\begin{tabular}{lccccccc} \hline
 & (1) & (2) & (3) & (4) & (5) & (6) & (7) \\
VARIABLES &  &  &  &  &  &  &  \\ \hline
 &  &  &  &  &  &  &  \\
Reggio None ITC & -0.135** & -0.133** & -0.131** & -0.131** & -0.113* & -0.113* & -0.127** \\
 & [0.059] & [0.061] & [0.063] & [0.063] & [0.067] & [0.066] & [0.062] \\
Reggio Reli ITC & -0.165 & -0.185 & -0.250 & -0.248 & -0.198 & -0.198 & -0.229 \\
 & [0.177] & [0.190] & [0.194] & [0.195] & [0.196] & [0.192] & [0.194] \\
Reggio Priv ITC & 0.210*** & 0.217*** & 0.149* & 0.146* & 0.210* & 0.210* & 0.149** \\
 & [0.038] & [0.054] & [0.081] & [0.083] & [0.112] & [0.110] & [0.068] \\
Parma None ITC & -0.144** & -0.064 & -0.077 & -0.076 & 0.212 &  & -0.175*** \\
 & [0.062] & [0.158] & [0.157] & [0.157] & [0.178] &  & [0.063] \\
Parma Muni ITC & -0.304*** & -0.217 & -0.212 & -0.210 & 0.064 &  & -0.316*** \\
 & [0.070] & [0.162] & [0.163] & [0.163] & [0.178] &  & [0.072] \\
Parma Reli ITC & -0.290* & -0.286 & -0.249 & -0.249 &  &  & -0.296* \\
 & [0.164] & [0.218] & [0.220] & [0.220] &  &  & [0.163] \\
Parma Priv ITC & -0.012 &  &  &  & 0.277 &  & -0.069 \\
 & [0.145] &  &  &  & [0.247] &  & [0.149] \\
Padova None ITC & -0.011 & -0.546 & -0.237 & -0.543 & 0.369* &  & -0.069 \\
 & [0.049] & [0.402] & [0.190] & [0.453] & [0.193] &  & [0.052] \\
Padova Muni ITC & -0.067 & -0.531 & -0.238 & -0.546 & 0.334 &  & -0.108 \\
 & [0.076] & [0.409] & [0.205] & [0.458] & [0.203] &  & [0.079] \\
Padova Reli ITC & -0.190 & -0.816* & -0.574** & -0.886* &  &  & -0.314* \\
 & [0.224] & [0.446] & [0.266] & [0.487] &  &  & [0.174] \\
Male dummy &  &  & 0.073** & 0.073** & 0.073 & 0.073 & 0.070** \\
 &  &  & [0.036] & [0.036] & [0.065] & [0.064] & [0.036] \\
Constant & 0.790*** & 1.046*** & -13.111 & -13.737 & 30.349 & 5.870 & 3.539 \\
 & [0.038] & [0.168] & [48.763] & [48.712] & [95.470] & [84.492] & [47.521] \\
 &  &  &  &  &  &  &  \\
Observations & 651 & 651 & 651 & 651 & 651 & 240 & 651 \\
R-squared & 0.049 & 0.120 & 0.148 & 0.148 & 0.216 & 0.105 & 0.103 \\
 Controls & None & F.E. & Family & All & Inter & Reggio & no FE \\ \hline
\multicolumn{8}{c}{ Robust standard errors in brackets} \\
\multicolumn{8}{c}{ *** p$<$0.01, ** p$<$0.05, * p$<$0.1} \\
\multicolumn{8}{c}{ Dependent variable: Respondent health is good (\%).} \\
\end{tabular}

\end{table}
\begin{table}[H]
\caption{Pool Regression: Health - Preschool, Adolescent}
\begin{tabular}{lccccccc} \hline
 & (1) & (2) & (3) & (4) & (5) & (6) & (7) \\
VARIABLES &  &  &  &  &  &  &  \\ \hline
 &  &  &  &  &  &  &  \\
Reggio None Preschool & -0.057 & -0.055 & -0.010 & -0.005 & 0.006 & 0.006 & -0.006 \\
 & [0.177] & [0.173] & [0.162] & [0.157] & [0.185] & [0.181] & [0.153] \\
Reggio Stat Preschool & 0.042 & 0.044 & 0.017 & 0.016 & 0.053 & 0.053 & 0.023 \\
 & [0.105] & [0.111] & [0.117] & [0.117] & [0.125] & [0.123] & [0.116] \\
Reggio Reli Preschool & -0.141** & -0.135** & -0.129* & -0.128* & -0.134* & -0.134* & -0.118* \\
 & [0.066] & [0.068] & [0.070] & [0.070] & [0.076] & [0.074] & [0.069] \\
Reggio Priv Preschool & 0.029 & 0.040 & 0.028 & 0.037 & 0.009 & 0.009 & 0.026 \\
 & [0.185] & [0.186] & [0.186] & [0.176] & [0.189] & [0.185] & [0.174] \\
Parma None Preschool & 0.229*** &  &  & 0.234 &  &  & 0.276*** \\
 & [0.037] &  &  & [0.206] &  &  & [0.104] \\
Parma Muni Preschool & -0.139** & -0.349*** & -0.326*** & -0.127 & -0.384** &  & -0.158** \\
 & [0.064] & [0.089] & [0.119] & [0.151] & [0.164] &  & [0.067] \\
Parma Stat Preschool & -0.447*** & -0.634*** & -0.588*** & -0.392** & -0.627*** &  & -0.431*** \\
 & [0.089] & [0.123] & [0.148] & [0.168] & [0.192] &  & [0.092] \\
Parma Reli Preschool & -0.142* & -0.372*** & -0.351*** & -0.150 & -0.402** &  & -0.171** \\
 & [0.072] & [0.103] & [0.127] & [0.146] & [0.164] &  & [0.075] \\
Parma Priv Preschool & 0.229*** & -0.203 & -0.198 &  & -0.160 &  & 0.126 \\
 & [0.037] & [0.158] & [0.190] &  & [0.257] &  & [0.078] \\
Padova None Preschool & -0.771*** &  &  & -0.969*** &  &  & -0.770*** \\
 & [0.037] &  &  & [0.207] &  &  & [0.126] \\
Padova Muni Preschool & -0.095 & 0.910*** & 0.979*** & 0.009 & 0.975*** &  & -0.120* \\
 & [0.067] & [0.084] & [0.091] & [0.202] & [0.106] &  & [0.070] \\
Padova Stat Preschool & 0.090 & 1.000 & 1.068*** & 0.098 & 1.064*** &  & 0.031 \\
 & [0.069] & [.] & [0.037] & [0.200] & [0.058] &  & [0.074] \\
Padova Reli Preschool & 0.024 & 0.928*** & 0.997*** & 0.027 & 1.017*** &  & -0.028 \\
 & [0.054] & [0.078] & [0.088] & [0.201] & [0.106] &  & [0.056] \\
Padova Priv Preschool & -0.021 & 0.921*** & 0.975*** &  & 0.962*** &  & -0.091 \\
 & [0.222] & [0.193] & [0.208] &  & [0.183] &  & [0.225] \\
Male dummy &  &  & 0.068* & 0.067* & 0.060 & 0.060 & 0.063* \\
 &  &  & [0.037] & [0.037] & [0.068] & [0.066] & [0.036] \\
Constant & 0.771*** & 1.289*** & -13.150 & -13.856 & 56.149 & -15.907 & 3.268 \\
 & [0.037] & [0.095] & [48.528] & [48.382] & [92.707] & [84.667] & [47.057] \\
 &  &  &  &  &  &  &  \\
Observations & 651 & 651 & 651 & 651 & 651 & 240 & 651 \\
R-squared & 0.070 & 0.127 & 0.153 & 0.154 & 0.221 & 0.104 & 0.117 \\
 Controls & None & F.E. & Family & All & Inter & Reggio & no FE \\ \hline
\multicolumn{8}{c}{ Robust standard errors in brackets} \\
\multicolumn{8}{c}{ *** p$<$0.01, ** p$<$0.05, * p$<$0.1} \\
\multicolumn{8}{c}{ Dependent variable: Respondent health is good (\%).} \\
\end{tabular}

\end{table}
\begin{table}[H]
\caption{Pool Regression: SDQ Score - Infant-toddler center, Adolescent}
\begin{tabular}{lccccccc} \hline
 & (1) & (2) & (3) & (4) & (5) & (6) & (7) \\
VARIABLES &  &  &  &  &  &  &  \\ \hline
 &  &  &  &  &  &  &  \\
Reggio None ITC & 0.675 & 0.547 & 0.564 & 0.553 & 0.760 & 0.760 & 0.626 \\
 & [0.521] & [0.556] & [0.576] & [0.578] & [0.615] & [0.601] & [0.555] \\
Reggio Reli ITC & 1.750 & 1.301 & 2.082 & 2.043 & 1.287 & 1.287 & 2.420 \\
 & [2.801] & [2.829] & [2.859] & [2.883] & [2.734] & [2.673] & [2.830] \\
Reggio Priv ITC & 2.042 & 1.812 & 2.644 & 2.692 & 1.622 & 1.622 & 3.015 \\
 & [1.953] & [2.214] & [2.643] & [2.632] & [3.219] & [3.148] & [2.308] \\
Parma None ITC & -0.130 & 0.139 & 1.655 & 2.371 & -1.731 &  & 0.080 \\
 & [0.566] & [4.542] & [3.981] & [4.420] & [2.204] &  & [0.579] \\
Parma Muni ITC & 0.111 & 0.174 & 1.829 & 2.520 & -1.454 &  & 0.416 \\
 & [0.610] & [4.575] & [4.019] & [4.428] & [2.113] &  & [0.595] \\
Parma Reli ITC & 2.075 & 1.349 & 2.762 & 3.498 &  &  & 2.334 \\
 & [2.314] & [4.807] & [4.259] & [4.714] &  &  & [2.257] \\
Parma Priv ITC & 2.597 & 2.826 & 4.046 & 4.771 & 1.041 &  & 2.605 \\
 & [1.784] & [4.952] & [4.414] & [4.815] & [2.755] &  & [1.760] \\
Padova None ITC & 0.564 & 0.915 & 1.399 & 1.299 & -467.122 &  & 0.410 \\
 & [0.454] & [1.709] & [1.475] & [1.469] & [1,096.046] &  & [0.486] \\
Padova Muni ITC & 0.597 & 0.916 & 1.655 & 1.570 & -466.520 &  & 0.804 \\
 & [0.756] & [1.857] & [1.643] & [1.645] & [1,095.941] &  & [0.750] \\
Padova Reli ITC & -0.225 &  &  &  & -468.638 &  & -0.703 \\
 & [1.145] &  &  &  & [1,096.082] &  & [1.033] \\
Male dummy &  &  & 0.508 & 0.516 & -0.322 & -0.322 & 0.456 \\
 &  &  & [0.339] & [0.338] & [0.629] & [0.615] & [0.342] \\
Constant & 7.625*** & 8.085** & -125.646 & -116.780 & 330.862 & -625.874 & -127.570 \\
 & [0.351] & [4.030] & [403.486] & [405.630] & [848.217] & [618.561] & [396.958] \\
 &  &  &  &  &  &  &  \\
Observations & 648 & 648 & 648 & 648 & 648 & 241 & 648 \\
R-squared & 0.014 & 0.088 & 0.135 & 0.137 & 0.208 & 0.117 & 0.072 \\
 Controls & None & F.E. & Family & All & Inter & Reggio & no FE \\ \hline
\multicolumn{8}{c}{ Robust standard errors in brackets} \\
\multicolumn{8}{c}{ *** p$<$0.01, ** p$<$0.05, * p$<$0.1} \\
\multicolumn{8}{c}{ Dependent variable: SDQ score (mom rep.).} \\
\end{tabular}

\end{table}
\begin{table}[H]
\caption{Pool Regression: SDQ Score - Preschool, Adolescent}
\begin{tabular}{lccccccc} \hline
 & (1) & (2) & (3) & (4) & (5) & (6) & (7) \\
VARIABLES &  &  &  &  &  &  &  \\ \hline
 &  &  &  &  &  &  &  \\
Reggio None Preschool & 1.137 & 1.114 & 0.987 & 0.972 & 1.462 & 1.462 & 1.182 \\
 & [1.352] & [1.378] & [1.151] & [1.137] & [1.482] & [1.448] & [1.112] \\
Reggio Stat Preschool & -1.095* & -1.053* & -1.234 & -1.180 & -1.396 & -1.396* & -1.248* \\
 & [0.606] & [0.638] & [0.784] & [0.784] & [0.849] & [0.829] & [0.737] \\
Reggio Reli Preschool & 0.947 & 0.923 & 1.288* & 1.299* & 1.321* & 1.321* & 1.314* \\
 & [0.657] & [0.671] & [0.698] & [0.700] & [0.781] & [0.763] & [0.686] \\
Reggio Priv Preschool & 0.880 & 1.093 & 1.418 & 1.394 & 1.500 & 1.500 & 1.086 \\
 & [1.395] & [1.446] & [1.528] & [1.526] & [1.574] & [1.537] & [1.443] \\
Parma None Preschool & 0.780 &  &  &  &  &  & 2.083 \\
 & [3.234] &  &  &  &  &  & [3.268] \\
Parma Muni Preschool & -0.248 & -2.070 & -2.851 & -2.651 & -2.091 &  & 0.215 \\
 & [0.570] & [3.351] & [3.335] & [3.228] & [2.830] &  & [0.580] \\
Parma Stat Preschool & -0.249 & -2.187 & -3.287 & -3.081 & -2.491 &  & -0.139 \\
 & [0.715] & [3.415] & [3.413] & [3.282] & [2.898] &  & [0.711] \\
Parma Reli Preschool & 0.645 & -1.161 & -2.146 & -1.951 & -1.509 &  & 0.967 \\
 & [0.712] & [3.400] & [3.364] & [3.239] & [2.781] &  & [0.737] \\
Parma Priv Preschool & 4.780 & 3.606 & 2.680 & 2.886 & 2.321 &  & 5.374 \\
 & [4.661] & [5.910] & [5.985] & [5.930] & [5.487] &  & [4.702] \\
Padova None Preschool & 6.280*** & 6.630 & 7.810 & 11.428*** &  &  & 5.829*** \\
 & [0.311] & [4.134] & [5.106] & [1.283] &  &  & [1.096] \\
Padova Muni Preschool & 0.077 & -4.609 & -2.730 & 0.889 & -9.637*** &  & 0.197 \\
 & [0.511] & [4.142] & [5.088] & [1.221] & [0.851] &  & [0.542] \\
Padova Stat Preschool & 0.795 & -3.370 & -1.687 & 1.936 & -8.489*** &  & 0.789 \\
 & [0.718] & [4.134] & [5.091] & [1.226] & [0.535] &  & [0.739] \\
Padova Reli Preschool & 0.598 & -3.368 & -1.732 & 1.891 & -8.835*** &  & 0.571 \\
 & [0.508] & [4.084] & [5.056] & [1.258] & [0.886] &  & [0.532] \\
Padova Priv Preschool & -1.220 & -5.334 & -3.646 &  & -10.220*** &  & -1.110 \\
 & [1.088] & [4.289] & [5.192] &  & [1.218] &  & [1.153] \\
Male dummy &  &  & 0.503 & 0.508 & -0.107 & -0.107 & 0.453 \\
 &  &  & [0.338] & [0.337] & [0.636] & [0.621] & [0.340] \\
Constant & 7.720*** & 8.917** & -37.346 & -36.699 & -6.173 & -475.616 & -76.957 \\
 & [0.311] & [4.059] & [398.121] & [399.866] & [672.319] & [634.360] & [391.569] \\
 &  &  &  &  &  &  &  \\
Observations & 648 & 648 & 648 & 648 & 648 & 241 & 648 \\
R-squared & 0.021 & 0.104 & 0.152 & 0.154 & 0.219 & 0.140 & 0.079 \\
 Controls & None & F.E. & Family & All & Inter & Reggio & no FE \\ \hline
\multicolumn{8}{c}{ Robust standard errors in brackets} \\
\multicolumn{8}{c}{ *** p$<$0.01, ** p$<$0.05, * p$<$0.1} \\
\multicolumn{8}{c}{ Dependent variable: SDQ score (mom rep.).} \\
\end{tabular}

\end{table}
\begin{table}[H]
\caption{Pool Regression: Depression - Infant-toddler center, Adolescent}
\begin{tabular}{lccccccc} \hline
 & (1) & (2) & (3) & (4) & (5) & (6) & (7) \\
VARIABLES &  &  &  &  &  &  &  \\ \hline
 &  &  &  &  &  &  &  \\
Reggio None ITC & 0.087 & 0.092 & 0.065 & 0.051 & 0.725 & 0.725 & 0.040 \\
 & [0.854] & [0.902] & [0.900] & [0.899] & [0.935] & [0.915] & [0.864] \\
Reggio Reli ITC & 1.686 & 2.268 & 2.767 & 2.664 & 2.773 & 2.773 & 1.911 \\
 & [3.182] & [3.291] & [3.305] & [3.354] & [3.239] & [3.167] & [3.295] \\
Reggio Priv ITC & -2.564 & -2.276 & -1.657 & -1.529 & -1.530 & -1.530 & -1.885 \\
 & [3.819] & [3.761] & [3.794] & [3.817] & [4.878] & [4.770] & [3.991] \\
Parma None ITC & -0.949 & 0.516 & 0.378 & 0.283 & -0.223 &  & -0.685 \\
 & [0.788] & [1.689] & [1.774] & [1.774] & [1.468] &  & [0.800] \\
Parma Muni ITC & -0.778 & 0.427 & -0.105 & -0.253 & -0.380 &  & -0.918 \\
 & [0.836] & [1.684] & [1.781] & [1.784] & [1.508] &  & [0.882] \\
Parma Reli ITC & -1.364 & -0.579 & -1.335 & -1.329 &  &  & -1.459 \\
 & [1.386] & [2.085] & [2.197] & [2.188] &  &  & [1.444] \\
Parma Priv ITC & -1.009 &  &  &  & -0.099 &  & -0.307 \\
 & [1.613] &  &  &  & [2.195] &  & [1.747] \\
Padova None ITC & -1.993*** & 1.929 & 3.481 & 0.624 & -1,073.031 &  & -1.146 \\
 & [0.719] & [5.943] & [4.098] & [6.653] & [1,452.872] &  & [0.746] \\
Padova Muni ITC & 0.202 & 3.311 & 4.991 & 2.221 & -1,071.700 &  & 0.905 \\
 & [1.177] & [6.035] & [4.238] & [6.712] & [1,452.766] &  & [1.169] \\
Padova Reli ITC & -1.164 & 3.181 & 5.463 & 2.730 & -1,071.774 &  & 0.911 \\
 & [2.878] & [6.782] & [5.432] & [7.541] & [1,452.767] &  & [3.120] \\
Male dummy &  &  & -1.330*** & -1.311*** & -1.521* & -1.521* & -1.356*** \\
 &  &  & [0.469] & [0.470] & [0.895] & [0.875] & [0.467] \\
Constant & 22.564*** & 20.571*** & 247.000 & 240.365 & 716.015 & 11.027 & 380.580 \\
 & [0.580] & [2.301] & [621.097] & [617.432] & [1,007.312] & [1,227.718] & [611.722] \\
 &  &  &  &  &  &  &  \\
Observations & 634 & 634 & 634 & 634 & 634 & 234 & 634 \\
R-squared & 0.023 & 0.099 & 0.137 & 0.141 & 0.239 & 0.142 & 0.080 \\
 Controls & None & F.E. & Family & All & Inter & Reggio & no FE \\ \hline
\multicolumn{8}{c}{ Robust standard errors in brackets} \\
\multicolumn{8}{c}{ *** p$<$0.01, ** p$<$0.05, * p$<$0.1} \\
\multicolumn{8}{c}{ Dependent variable: Depression score (CESD).} \\
\end{tabular}

\end{table}
\begin{table}[H]
\caption{Pool Regression: Depression - Preschool, Adolescent}
\begin{tabular}{lccccccc} \hline
 & (1) & (2) & (3) & (4) & (5) & (6) & (7) \\
VARIABLES &  &  &  &  &  &  &  \\ \hline
 &  &  &  &  &  &  &  \\
Reggio None Preschool & -2.967* & -2.821* & -2.575* & -2.729* & -1.375 & -1.375 & -2.782** \\
 & [1.515] & [1.671] & [1.492] & [1.479] & [1.661] & [1.622] & [1.345] \\
Reggio Stat Preschool & -1.709 & -2.017 & -1.666 & -1.767 & -1.230 & -1.230 & -1.710 \\
 & [1.469] & [1.591] & [1.682] & [1.685] & [1.754] & [1.713] & [1.591] \\
Reggio Reli Preschool & 2.030** & 2.031** & 1.912* & 1.845* & 2.018* & 2.018* & 1.685* \\
 & [0.958] & [0.985] & [1.027] & [1.030] & [1.122] & [1.096] & [1.008] \\
Reggio Priv Preschool & 1.491 & 1.653 & 2.049 & 1.808 & 1.592 & 1.592 & 1.753 \\
 & [1.820] & [2.029] & [1.878] & [1.778] & [1.925] & [1.880] & [1.610] \\
Parma None Preschool & -2.609 &  &  & -4.697 & 962.403 &  & -2.362 \\
 & [3.975] &  &  & [3.757] & [1,460.570] &  & [3.764] \\
Parma Muni Preschool & -0.550 & 2.111 & 1.682 & -2.371 & 964.371 &  & -0.413 \\
 & [0.767] & [4.115] & [3.753] & [1.808] & [1,460.050] &  & [0.842] \\
Parma Stat Preschool & -1.928* & 1.177 & 0.296 & -3.661* & 963.284 &  & -2.345** \\
 & [1.041] & [4.203] & [3.873] & [1.949] & [1,459.916] &  & [1.094] \\
Parma Reli Preschool & 0.569 & 3.305 & 2.833 & -1.276 & 965.713 &  & 0.710 \\
 & [0.860] & [4.174] & [3.802] & [1.787] & [1,460.018] &  & [0.862] \\
Parma Priv Preschool & -0.109 & 3.654 & 4.006 &  & 966.344 &  & 1.154 \\
 & [2.218] & [4.473] & [4.163] &  & [1,459.593] &  & [1.892] \\
Padova None Preschool & 7.891*** & 19.842*** & 20.393*** & 16.273*** &  &  & 9.300*** \\
 & [0.559] & [5.478] & [5.316] & [2.346] &  &  & [1.405] \\
Padova Muni Preschool & -0.809 & 7.255 & 6.312 & 2.204 & -13.347*** &  & -0.267 \\
 & [0.888] & [5.463] & [5.287] & [2.336] & [1.533] &  & [0.929] \\
Padova Stat Preschool & -3.168*** & 4.842 & 3.994 & -0.097 & -15.517*** &  & -2.422** \\
 & [0.999] & [5.478] & [5.305] & [2.307] & [0.819] &  & [1.089] \\
Padova Reli Preschool & -0.551 & 7.495 & 6.661 & 2.563 & -13.297*** &  & 0.243 \\
 & [0.815] & [5.381] & [5.219] & [2.278] & [1.377] &  & [0.816] \\
Padova Priv Preschool & -3.359 & 4.416 & 3.978 &  & -15.994*** &  & -1.706 \\
 & [2.625] & [5.829] & [5.667] &  & [2.419] &  & [2.665] \\
Male dummy &  &  & -1.399*** & -1.370*** & -1.257 & -1.257 & -1.440*** \\
 &  &  & [0.462] & [0.464] & [0.870] & [0.849] & [0.462] \\
Constant & 22.109*** & 17.708*** & 298.777 & 284.939 & -253.132 & 335.928 & 501.743 \\
 & [0.559] & [4.404] & [623.109] & [620.723] & [1,068.242] & [1,256.598] & [610.616] \\
 &  &  &  &  &  &  &  \\
Observations & 634 & 634 & 634 & 634 & 634 & 234 & 634 \\
R-squared & 0.050 & 0.129 & 0.163 & 0.166 & 0.260 & 0.158 & 0.105 \\
 Controls & None & F.E. & Family & All & Inter & Reggio & no FE \\ \hline
\multicolumn{8}{c}{ Robust standard errors in brackets} \\
\multicolumn{8}{c}{ *** p$<$0.01, ** p$<$0.05, * p$<$0.1} \\
\multicolumn{8}{c}{ Dependent variable: Depression score (CESD).} \\
\end{tabular}

\end{table}
\begin{table}[H]
\caption{Pool Regression: Migration Taste - Infant-toddler center, Adolescent}
\begin{tabular}{lccccccc} \hline
 & (1) & (2) & (3) & (4) & (5) & (6) & (7) \\
VARIABLES &  &  &  &  &  &  &  \\ \hline
 &  &  &  &  &  &  &  \\
Reggio None ITC & -0.032 & -0.030 & -0.035 & -0.037 & -0.024 & -0.024 & -0.046 \\
 & [0.062] & [0.064] & [0.067] & [0.066] & [0.071] & [0.069] & [0.064] \\
Reggio Reli ITC & -0.319*** & -0.336*** & -0.306*** & -0.319*** & -0.355*** & -0.355*** & -0.294*** \\
 & [0.043] & [0.060] & [0.071] & [0.073] & [0.097] & [0.095] & [0.063] \\
Reggio Priv ITC & -0.319*** & -0.293*** & -0.241*** & -0.226*** & -0.245 & -0.245* & -0.243*** \\
 & [0.043] & [0.053] & [0.084] & [0.084] & [0.149] & [0.145] & [0.077] \\
Parma None ITC & -0.137** & -0.165 & -0.198 & -0.207 & -204.576* &  & -0.133** \\
 & [0.059] & [0.171] & [0.161] & [0.161] & [112.339] &  & [0.058] \\
Parma Muni ITC & -0.094 & -0.132 & -0.160 & -0.176 & -204.584* &  & -0.100 \\
 & [0.066] & [0.173] & [0.164] & [0.164] & [112.340] &  & [0.067] \\
Parma Reli ITC & -0.319*** & -0.355** & -0.382** & -0.376** & -204.735* &  & -0.303*** \\
 & [0.043] & [0.170] & [0.165] & [0.166] & [112.342] &  & [0.058] \\
Parma Priv ITC & 0.014 &  &  &  & -204.317* &  & 0.076 \\
 & [0.164] &  &  &  & [112.312] &  & [0.155] \\
Padova None ITC & -0.069 & 0.120 & 0.062 & 0.047 & -0.078 &  & -0.058 \\
 & [0.055] & [0.149] & [0.159] & [0.160] & [0.163] &  & [0.057] \\
Padova Muni ITC & -0.058 & 0.093 & 0.079 & 0.076 & -0.021 &  & 0.003 \\
 & [0.078] & [0.161] & [0.169] & [0.170] & [0.174] &  & [0.080] \\
Padova Reli ITC & -0.119 &  &  &  &  &  & -0.048 \\
 & [0.186] &  &  &  &  &  & [0.196] \\
Male dummy &  &  & 0.017 & 0.020 & 0.018 & 0.018 & 0.004 \\
 &  &  & [0.035] & [0.035] & [0.066] & [0.064] & [0.035] \\
Constant & 0.319*** & -0.120 & -8.577 & -8.373 & 63.795 & -42.912 & 25.107 \\
 & [0.043] & [0.149] & [47.054] & [46.749] & [90.656] & [87.298] & [46.463] \\
 &  &  &  &  &  &  &  \\
Observations & 635 & 635 & 635 & 635 & 635 & 238 & 635 \\
R-squared & 0.021 & 0.109 & 0.156 & 0.165 & 0.243 & 0.143 & 0.093 \\
 Controls & None & F.E. & Family & All & Inter & Reggio & no FE \\ \hline
\multicolumn{8}{c}{ Robust standard errors in brackets} \\
\multicolumn{8}{c}{ *** p$<$0.01, ** p$<$0.05, * p$<$0.1} \\
\multicolumn{8}{c}{ Dependent variable: Bothered by migrants (\%).} \\
\end{tabular}

\end{table}
\begin{table}[H]
\caption{Pool Regression: Migration Taste - Preschool, Adolescent}
\begin{tabular}{lccccccc} \hline
 & (1) & (2) & (3) & (4) & (5) & (6) & (7) \\
VARIABLES &  &  &  &  &  &  &  \\ \hline
 &  &  &  &  &  &  &  \\
Reggio None Preschool & -0.088 & -0.093 & -0.112 & -0.129 & -0.107 & -0.107 & -0.121 \\
 & [0.139] & [0.127] & [0.135] & [0.138] & [0.148] & [0.144] & [0.157] \\
Reggio Stat Preschool & 0.019 & 0.003 & 0.035 & 0.031 & 0.002 & 0.002 & 0.051 \\
 & [0.116] & [0.120] & [0.119] & [0.118] & [0.121] & [0.118] & [0.114] \\
Reggio Reli Preschool & 0.144** & 0.129* & 0.107 & 0.101 & 0.100 & 0.100 & 0.105 \\
 & [0.066] & [0.067] & [0.069] & [0.069] & [0.074] & [0.072] & [0.069] \\
Reggio Priv Preschool & 0.569*** & 0.639*** & 0.583*** & 0.556*** & 0.460** & 0.460** & 0.492*** \\
 & [0.185] & [0.194] & [0.206] & [0.199] & [0.212] & [0.207] & [0.190] \\
Parma None Preschool & 0.269 & -0.048 & 0.493 &  & -207.358* &  & 0.278 \\
 & [0.360] & [0.536] & [0.375] &  & [110.596] &  & [0.298] \\
Parma Muni Preschool & -0.016 & -0.400 & 0.100 & -0.311 & -207.620* &  & -0.014 \\
 & [0.059] & [0.389] & [0.153] & [0.286] & [110.565] &  & [0.060] \\
Parma Stat Preschool & -0.113* & -0.547 & -0.050 & -0.450 & -207.729* &  & -0.131** \\
 & [0.067] & [0.391] & [0.158] & [0.291] & [110.565] &  & [0.065] \\
Parma Reli Preschool & -0.014 & -0.384 & 0.102 & -0.314 & -207.632* &  & -0.023 \\
 & [0.066] & [0.385] & [0.138] & [0.289] & [110.574] &  & [0.065] \\
Parma Priv Preschool & -0.231*** & -0.477 & 0.100 & -0.306 & -207.461* &  & -0.064 \\
 & [0.037] & [0.391] & [0.158] & [0.294] & [110.566] &  & [0.076] \\
Padova None Preschool & -0.231*** &  &  &  &  &  & -0.410*** \\
 & [0.037] &  &  &  &  &  & [0.124] \\
Padova Muni Preschool & -0.028 & -0.095 & -0.102 & -0.098 & -0.161 &  & -0.012 \\
 & [0.062] & [0.080] & [0.088] & [0.089] & [0.104] &  & [0.063] \\
Padova Stat Preschool & -0.025 & 0.000 & 0.017 & 0.020 & -0.033 &  & 0.008 \\
 & [0.080] & [.] & [0.034] & [0.034] & [0.060] &  & [0.087] \\
Padova Reli Preschool & 0.068 & 0.093 & 0.069 & 0.074 & -0.005 &  & 0.076 \\
 & [0.058] & [0.078] & [0.088] & [0.088] & [0.100] &  & [0.060] \\
Padova Priv Preschool & 0.019 & 0.111 & 0.180 & 0.196 & 0.101 &  & 0.147 \\
 & [0.222] & [0.276] & [0.260] & [0.264] & [0.265] &  & [0.189] \\
Male dummy &  &  & 0.017 & 0.020 & 0.032 & 0.032 & 0.005 \\
 &  &  & [0.034] & [0.034] & [0.066] & [0.064] & [0.035] \\
Constant & 0.231*** & 0.421 & 11.847 & 11.751 & 80.246 & -3.102 & 40.929 \\
 & [0.037] & [0.385] & [46.232] & [46.042] & [89.527] & [86.446] & [45.736] \\
 &  &  &  &  &  &  &  \\
Observations & 635 & 635 & 635 & 635 & 635 & 238 & 635 \\
R-squared & 0.037 & 0.135 & 0.176 & 0.182 & 0.249 & 0.149 & 0.103 \\
 Controls & None & F.E. & Family & All & Inter & Reggio & no FE \\ \hline
\multicolumn{8}{c}{ Robust standard errors in brackets} \\
\multicolumn{8}{c}{ *** p$<$0.01, ** p$<$0.05, * p$<$0.1} \\
\multicolumn{8}{c}{ Dependent variable: Bothered by migrants (\%).} \\
\end{tabular}

\end{table}
\begin{table}[H]
\caption{Pool Regression: SDQ Score - Infant-toddler center, Adolescent}
\begin{tabular}{lccccccc} \hline
 & (1) & (2) & (3) & (4) & (5) & (6) & (7) \\
VARIABLES &  &  &  &  &  &  &  \\ \hline
 &  &  &  &  &  &  &  \\
Reggio None ITC & 0.507 & 0.405 & 0.359 & 0.354 & 0.633 & 0.633 & 0.399 \\
 & [0.604] & [0.639] & [0.659] & [0.660] & [0.685] & [0.671] & [0.632] \\
Reggio Reli ITC & 2.596 & 2.850 & 3.428 & 3.382 & 3.811 & 3.811 & 3.093 \\
 & [2.375] & [2.497] & [2.504] & [2.536] & [2.709] & [2.651] & [2.425] \\
Reggio Priv ITC & -1.863** & -1.806* & -1.425 & -1.372 & -1.688 & -1.688 & -1.247 \\
 & [0.849] & [0.981] & [1.026] & [1.027] & [1.127] & [1.103] & [0.911] \\
Parma None ITC & -1.071* & -0.991 & 0.487 & 0.367 & -0.257 &  & -0.839 \\
 & [0.601] & [2.291] & [3.008] & [3.023] & [1.527] &  & [0.615] \\
Parma Muni ITC & -0.724 & -0.802 & 0.407 & 0.264 & -0.264 &  & -0.746 \\
 & [0.686] & [2.388] & [3.071] & [3.084] & [1.528] &  & [0.712] \\
Parma Reli ITC & -0.429 & -1.079 & 0.112 & 0.034 &  &  & -0.394 \\
 & [1.859] & [2.733] & [3.348] & [3.354] &  &  & [1.858] \\
Parma Priv ITC & 0.471 & 0.695 & 2.039 & 1.957 & 1.425 &  & 0.704 \\
 & [1.265] & [2.620] & [3.279] & [3.293] & [1.948] &  & [1.358] \\
Padova None ITC & -0.266 & 1.072 & 0.921 & 0.872 & 1.592 &  & -0.032 \\
 & [0.542] & [1.773] & [1.873] & [1.887] & [2.069] &  & [0.577] \\
Padova Muni ITC & 1.173 & 1.983 & 1.848 & 1.834 & 2.354 &  & 1.440* \\
 & [0.853] & [1.889] & [1.957] & [1.968] & [2.072] &  & [0.847] \\
Padova Reli ITC & -1.329 &  &  &  &  &  & -0.783 \\
 & [1.391] &  &  &  &  &  & [1.540] \\
Male dummy &  &  & -0.311 & -0.302 & -0.345 & -0.345 & -0.351 \\
 &  &  & [0.352] & [0.353] & [0.665] & [0.651] & [0.353] \\
Constant & 9.529*** & 10.428*** & 252.409 & 248.916 & -104.871 & 715.220 & 322.646 \\
 & [0.439] & [2.084] & [442.524] & [442.512] & [711.326] & [849.179] & [441.961] \\
 &  &  &  &  &  &  &  \\
Observations & 649 & 649 & 649 & 649 & 649 & 239 & 649 \\
R-squared & 0.025 & 0.093 & 0.117 & 0.118 & 0.234 & 0.141 & 0.057 \\
 Controls & None & F.E. & Family & All & Inter & Reggio & no FE \\ \hline
\multicolumn{8}{c}{ Robust standard errors in brackets} \\
\multicolumn{8}{c}{ *** p$<$0.01, ** p$<$0.05, * p$<$0.1} \\
\multicolumn{8}{c}{ Dependent variable: SDQ score (self rep.).} \\
\end{tabular}

\end{table}
\begin{table}[H]
\caption{Pool Regression: SDQ Score - Preschool, Adolescent}
\begin{tabular}{lccccccc} \hline
 & (1) & (2) & (3) & (4) & (5) & (6) & (7) \\
VARIABLES &  &  &  &  &  &  &  \\ \hline
 &  &  &  &  &  &  &  \\
Reggio None Preschool & -2.095 & -1.956 & -1.876 & -1.927 & -0.698 & -0.698 & -2.068 \\
 & [1.584] & [1.497] & [1.603] & [1.540] & [1.813] & [1.772] & [1.571] \\
Reggio Stat Preschool & -2.148** & -2.329** & -2.437** & -2.471** & -2.142** & -2.142** & -2.410** \\
 & [0.911] & [0.930] & [1.046] & [1.035] & [1.043] & [1.019] & [1.004] \\
Reggio Reli Preschool & 1.465** & 1.478** & 1.628** & 1.605** & 1.514** & 1.514** & 1.539** \\
 & [0.661] & [0.682] & [0.701] & [0.706] & [0.766] & [0.749] & [0.687] \\
Reggio Priv Preschool & 0.477 & 0.654 & 0.879 & 0.800 & -0.506 & -0.506 & 0.653 \\
 & [1.037] & [1.123] & [1.046] & [1.051] & [1.268] & [1.239] & [0.982] \\
Parma None Preschool & -0.023 &  &  &  &  &  & 0.969 \\
 & [1.152] &  &  &  &  &  & [1.270] \\
Parma Muni Preschool & -1.017* & -2.645** & -3.521*** & -3.320*** & -3.285*** &  & -0.847 \\
 & [0.604] & [1.268] & [1.308] & [1.232] & [1.219] &  & [0.641] \\
Parma Stat Preschool & -1.641* & -3.595** & -4.625*** & -4.394*** & -4.382*** &  & -1.709* \\
 & [0.923] & [1.524] & [1.589] & [1.529] & [1.513] &  & [0.985] \\
Parma Reli Preschool & -0.184 & -1.892 & -2.699** & -2.517** & -2.907** &  & 0.088 \\
 & [0.694] & [1.339] & [1.366] & [1.260] & [1.223] &  & [0.709] \\
Parma Priv Preschool & 0.977 & -0.417 & -0.759 & -0.547 & -2.221 &  & 1.718 \\
 & [1.837] & [2.282] & [2.498] & [2.477] & [2.612] &  & [1.987] \\
Padova None Preschool & 3.477*** & -1.305 & -0.334 & 4.111** & -264.303 &  & 3.176*** \\
 & [0.418] & [2.519] & [2.803] & [1.639] & [1,061.820] &  & [1.137] \\
Padova Muni Preschool & -0.309 & -4.026 & -3.096 & 1.352 & -266.571 &  & -0.103 \\
 & [0.658] & [2.472] & [2.744] & [1.490] & [1,061.939] &  & [0.674] \\
Padova Stat Preschool & -1.023 & -4.305* & -3.645 & 0.809 & -267.366 &  & -1.019 \\
 & [0.778] & [2.519] & [2.807] & [1.584] & [1,061.802] &  & [0.878] \\
Padova Reli Preschool & 0.611 & -2.483 & -1.582 & 2.870* & -265.312 &  & 0.945 \\
 & [0.599] & [2.368] & [2.666] & [1.475] & [1,061.958] &  & [0.632] \\
Padova Priv Preschool & -2.273 & -5.627** & -4.487 &  & -268.409 &  & -1.557 \\
 & [1.629] & [2.637] & [3.048] &  & [1,062.002] &  & [1.922] \\
Male dummy &  &  & -0.311 & -0.302 & -0.097 & -0.097 & -0.371 \\
 &  &  & [0.350] & [0.352] & [0.667] & [0.652] & [0.352] \\
Constant & 9.523*** & 11.117*** & 375.107 & 367.252 & 129.535 & 831.336 & 424.997 \\
 & [0.418] & [1.608] & [433.976] & [434.310] & [695.101] & [872.762] & [434.357] \\
 &  &  &  &  &  &  &  \\
Observations & 649 & 649 & 649 & 649 & 649 & 239 & 649 \\
R-squared & 0.045 & 0.121 & 0.148 & 0.148 & 0.250 & 0.156 & 0.082 \\
 Controls & None & F.E. & Family & All & Inter & Reggio & no FE \\ \hline
\multicolumn{8}{c}{ Robust standard errors in brackets} \\
\multicolumn{8}{c}{ *** p$<$0.01, ** p$<$0.05, * p$<$0.1} \\
\multicolumn{8}{c}{ Dependent variable: SDQ score (self rep.).} \\
\end{tabular}

\end{table}

\pagebreak
\subsubsection{Adults Results}
\begin{table}[H]
\caption{Pool Regression: Depression - Infant-toddler center, Adult}
\begin{tabular}{lccccccc} \hline
 & (1) & (2) & (3) & (4) & (5) & (6) & (7) \\
VARIABLES &  &  &  &  &  &  &  \\ \hline
 &  &  &  &  &  &  &  \\
Reggio None ITC & 0.808 & 0.358 & -0.279 & -0.279 & -0.484 & -0.484 & -0.161 \\
 & [0.598] & [0.608] & [0.705] & [0.705] & [0.698] & [0.692] & [0.727] \\
Reggio Reli ITC & -3.482*** & -4.251*** & -4.226*** & -4.226*** & -4.261** & -4.261** & -4.025*** \\
 & [0.854] & [1.388] & [1.586] & [1.586] & [1.937] & [1.920] & [1.254] \\
Reggio Priv ITC & 1.102 & 0.014 & -0.511 & -0.511 & -1.248 & -1.248 & 0.291 \\
 & [3.592] & [3.353] & [3.423] & [3.423] & [3.000] & [2.973] & [3.109] \\
Parma None ITC & 0.124 & 1.732 & 2.465 & 2.465 & -78.340* &  & -1.495** \\
 & [0.608] & [1.588] & [1.790] & [1.790] & [42.787] &  & [0.756] \\
Parma Muni ITC & -3.437*** & 0.339 & 0.024 & 0.024 & -80.995* &  & -5.111*** \\
 & [0.806] & [1.699] & [1.962] & [1.962] & [42.762] &  & [1.093] \\
Parma Reli ITC & -1.498 & 1.457 & 2.649 & 2.649 & -78.636* &  & -1.843 \\
 & [1.640] & [2.268] & [3.310] & [3.310] & [42.755] &  & [2.479] \\
Parma Priv ITC & -1.413 &  &  &  & -81.024* &  & -3.766** \\
 & [1.617] &  &  &  & [42.829] &  & [1.731] \\
Padova None ITC & -0.089 & -3.719 & -3.310 & -3.310 & -2.917 &  & -2.194*** \\
 & [0.604] & [2.286] & [2.696] & [2.696] & [3.791] &  & [0.738] \\
Padova Muni ITC & 5.344*** & 0.417 & 0.288 & 0.288 & 0.668 &  & 1.795 \\
 & [1.093] & [2.547] & [3.013] & [3.013] & [4.022] &  & [1.386] \\
Padova Reli ITC & 1.922 & -1.116 & -0.267 & -0.267 &  &  & -0.078 \\
 & [1.727] & [2.781] & [4.729] & [4.729] &  &  & [4.218] \\
Padova Priv ITC & -0.898 & -4.851 & -3.719 & -3.719 & -2.665 &  & -2.902 \\
 & [1.989] & [3.389] & [3.720] & [3.720] & [4.638] &  & [2.151] \\
Reggio None * Age 40-50 &  &  & 4.012** & 4.012** & 4.294** & 4.294** & 3.171** \\
 &  &  & [1.667] & [1.667] & [2.045] & [2.028] & [1.307] \\
Reggio Muni * Age 40-50 &  &  & 2.997 & 2.997 & 2.911 & 2.911 & 2.579 \\
 &  &  & [2.086] & [2.086] & [2.405] & [2.384] & [1.774] \\
Reggio None * Age 40-50 &  &  & -7.559*** & -7.559*** & -1.772 & 1.156 & -10.402*** \\
 &  &  & [2.718] & [2.718] & [9.871] & [1.288] & [2.403] \\
Reggio Muni * Age 40-50 &  &  & -8.143*** & -8.143*** & -3.324 & -0.396 & -11.103*** \\
 &  &  & [2.761] & [2.761] & [9.908] & [1.299] & [2.451] \\
Reggio Stat * Age 40-50 &  &  & -8.208*** & -8.208*** & -2.591 & 0.337 & -11.063*** \\
 &  &  & [3.052] & [3.052] & [9.878] & [1.716] & [2.771] \\
Reggio Reli * Age 40-50 &  &  & -8.165*** & -8.165*** & -2.688 & 0.240 & -11.411*** \\
 &  &  & [2.771] & [2.771] & [9.851] & [1.304] & [2.470] \\
Reggio Priv * Age 40-50 &  &  & -8.555*** & -8.555*** & -2.928 &  & -12.203*** \\
 &  &  & [3.034] & [3.034] & [9.909] &  & [2.823] \\
Male dummy &  &  & -0.709*** & -0.709*** & -0.700* & -0.700* & -0.786*** \\
 &  &  & [0.255] & [0.255] & [0.421] & [0.417] & [0.260] \\
Constant & 21.232*** & 25.642*** & 45.702*** & 45.702*** & 67.165** & 37.100 & 49.792*** \\
 & [0.552] & [0.608] & [16.981] & [16.981] & [30.392] & [27.879] & [17.237] \\
 &  &  &  &  &  &  &  \\
Observations & 1,863 & 1,863 & 1,863 & 1,863 & 1,863 & 735 & 1,863 \\
R-squared & 0.037 & 0.150 & 0.184 & 0.184 & 0.220 & 0.171 & 0.105 \\
 Controls & None & F.E. & Family & All & Inter & Reggio & no FE \\ \hline
\multicolumn{8}{c}{ Robust standard errors in brackets} \\
\multicolumn{8}{c}{ *** p$<$0.01, ** p$<$0.05, * p$<$0.1} \\
\multicolumn{8}{c}{ Dependent variable: Depression score (CESD).} \\
\end{tabular}

\end{table}
\begin{table}[H]
\caption{Pool Regression: Depression - Preschool, Adult}
\begin{tabular}{lccccccc} \hline
 & (1) & (2) & (3) & (4) & (5) & (6) & (7) \\
VARIABLES &  &  &  &  &  &  &  \\ \hline
 &  &  &  &  &  &  &  \\
Reggio None Preschool & 1.764*** & 1.051** & 0.651 & 0.651 & 0.273 & 0.273 & 0.854 \\
 & [0.475] & [0.480] & [0.862] & [0.862] & [0.863] & [0.855] & [0.887] \\
Reggio Stat Preschool & 1.244 & 1.110 & 1.461 & 1.461 & 1.147 & 1.147 & 1.548 \\
 & [0.983] & [0.932] & [1.170] & [1.170] & [1.122] & [1.112] & [1.167] \\
Reggio Reli Preschool & -0.462 & -0.340 & -1.654* & -1.654* & -1.888** & -1.888** & -1.799** \\
 & [0.610] & [0.593] & [0.869] & [0.869] & [0.806] & [0.799] & [0.881] \\
Reggio Priv Preschool & -1.961 & -1.304 & -6.731*** & -6.731*** & -6.247*** & -6.247*** & -4.942*** \\
 & [1.428] & [1.375] & [0.730] & [0.730] & [0.983] & [0.975] & [0.627] \\
Parma None Preschool & 0.793* & 0.115 & -0.310 & -0.310 & 1.888 &  & -0.963 \\
 & [0.475] & [3.356] & [3.960] & [3.960] & [1.150] &  & [0.989] \\
Parma Muni Preschool & -0.925 & -1.172 & -1.587 & -1.587 & 0.842 &  & -1.941** \\
 & [0.592] & [3.360] & [3.925] & [3.925] & [1.039] &  & [0.806] \\
Parma Stat Preschool & -2.080*** & -1.523 & -2.284 & -2.284 &  &  & -2.813*** \\
 & [0.775] & [3.405] & [3.985] & [3.985] &  &  & [0.981] \\
Parma Reli Preschool & -0.818 & -1.063 & -1.863 & -1.863 & 0.271 &  & -2.021*** \\
 & [0.605] & [3.380] & [3.953] & [3.953] & [1.073] &  & [0.780] \\
Parma Priv Preschool & -0.413 &  &  &  & 2.174 &  & -1.375 \\
 & [2.765] &  &  &  & [4.184] &  & [3.242] \\
Padova None Preschool & 0.536 & -2.576 & -1.195 & -1.195 & -0.188 &  & -0.736 \\
 & [0.568] & [2.719] & [3.102] & [3.102] & [1.363] &  & [0.980] \\
Padova Muni Preschool & 0.619 & -3.514 & -1.224 & -1.224 & 0.189 &  & -0.724 \\
 & [0.785] & [2.808] & [3.273] & [3.273] & [1.539] &  & [1.047] \\
Padova Stat Preschool & 3.941*** & 0.085 & 1.385 & 1.385 & 2.613 &  & 0.948 \\
 & [0.896] & [2.846] & [3.368] & [3.368] & [1.729] &  & [1.307] \\
Padova Reli Preschool & -0.576 & -3.428 & -1.891 & -1.891 & -0.561 &  & -2.688*** \\
 & [0.468] & [2.718] & [3.106] & [3.106] & [1.224] &  & [0.648] \\
Padova Priv Preschool & -2.746 & -4.761 & -2.120 & -2.120 &  &  & -2.564*** \\
 & [2.508] & [2.952] & [3.170] & [3.170] &  &  & [0.678] \\
Reggio None * Age 40-50 &  &  & 5.608 & 5.608 & 7.897 & 7.956*** & -1.678 \\
 &  &  & [4.536] & [4.536] & [10.527] & [0.918] & [3.838] \\
Reggio Muni * Age 40-50 &  &  & 4.866 & 4.866 & 7.000 & 7.059*** & -2.117 \\
 &  &  & [4.652] & [4.652] & [10.630] & [1.475] & [3.987] \\
Reggio Reli * Age 40-50 &  &  & -2.283 & -2.283 & -0.059 &  & -8.549** \\
 &  &  & [4.561] & [4.561] & [10.657] &  & [3.874] \\
Reggio None * Age 40-50 &  &  & -6.395*** & -6.395*** & -5.357*** & -5.357*** & -4.040** \\
 &  &  & [1.732] & [1.732] & [1.720] & [1.705] & [1.797] \\
Reggio Muni * Age 40-50 &  &  & -6.340*** & -6.340*** & -6.629*** & -6.629*** & -3.863** \\
 &  &  & [1.677] & [1.677] & [1.702] & [1.687] & [1.748] \\
Reggio Stat * Age 40-50 &  &  & -7.856*** & -7.856*** & -7.044*** & -7.044*** & -5.367** \\
 &  &  & [2.293] & [2.293] & [2.185] & [2.166] & [2.320] \\
Reggio Reli * Age 40-50 &  &  & -4.702*** & -4.702*** & -4.113** & -4.113** & -2.384 \\
 &  &  & [1.802] & [1.802] & [1.737] & [1.722] & [1.859] \\
Male dummy &  &  & -0.614** & -0.614** & -0.589 & -0.589 & -0.692*** \\
 &  &  & [0.255] & [0.255] & [0.419] & [0.415] & [0.260] \\
Constant & 21.246*** & 24.890*** & 47.870*** & 47.870*** & 76.442*** & 28.192 & 53.999*** \\
 & [0.341] & [0.932] & [17.075] & [17.075] & [29.348] & [28.091] & [17.276] \\
 &  &  &  &  &  &  &  \\
Observations & 1,863 & 1,863 & 1,863 & 1,863 & 1,863 & 735 & 1,863 \\
R-squared & 0.042 & 0.153 & 0.185 & 0.185 & 0.220 & 0.177 & 0.106 \\
 Controls & None & F.E. & Family & All & Inter & Reggio & no FE \\ \hline
\multicolumn{8}{c}{ Robust standard errors in brackets} \\
\multicolumn{8}{c}{ *** p$<$0.01, ** p$<$0.05, * p$<$0.1} \\
\multicolumn{8}{c}{ Dependent variable: Depression score (CESD).} \\
\end{tabular}

\end{table}
\begin{table}[H]
\caption{Pool Regression: Health - Infant-toddler center, Adult}
\begin{tabular}{lccccccc} \hline
 & (1) & (2) & (3) & (4) & (5) & (6) & (7) \\
VARIABLES &  &  &  &  &  &  &  \\ \hline
 &  &  &  &  &  &  &  \\
Reggio None ITC & -0.221*** & -0.190*** & -0.051** & -0.051** & -0.052** & -0.052** & -0.055*** \\
 & [0.032] & [0.033] & [0.020] & [0.020] & [0.021] & [0.021] & [0.020] \\
Reggio Reli ITC & 0.074*** & 0.110*** & 0.076* & 0.076* & 0.075 & 0.075 & 0.065 \\
 & [0.027] & [0.037] & [0.045] & [0.045] & [0.055] & [0.054] & [0.056] \\
Reggio Priv ITC & 0.074*** & 0.088*** & 0.009 & 0.009 & -0.009 & -0.009 & 0.047 \\
 & [0.027] & [0.030] & [0.041] & [0.041] & [0.028] & [0.028] & [0.037] \\
Parma None ITC & -0.346*** & -0.892*** & -0.366 & -0.366 & -0.609 &  & -0.171*** \\
 & [0.036] & [0.248] & [0.247] & [0.247] & [3.620] &  & [0.033] \\
Parma Muni ITC & -0.676*** & -1.057*** & -0.663** & -0.663** & -0.906 &  & -0.553*** \\
 & [0.059] & [0.255] & [0.260] & [0.260] & [3.626] &  & [0.078] \\
Parma Reli ITC & -0.393*** & -0.913*** & -0.441 & -0.441 & -0.706 &  & -0.279* \\
 & [0.132] & [0.273] & [0.295] & [0.295] & [3.625] &  & [0.150] \\
Parma Priv ITC & -0.654*** & -1.297*** & -0.963*** & -0.963*** & -1.227 &  & -0.758*** \\
 & [0.137] & [0.286] & [0.281] & [0.281] & [3.626] &  & [0.123] \\
Padova None ITC & -0.379*** & -0.571*** & -0.313*** & -0.313*** & 0.234 &  & -0.216*** \\
 & [0.034] & [0.112] & [0.073] & [0.073] & [0.222] &  & [0.035] \\
Padova Muni ITC & -0.229*** & -0.459*** & -0.517*** & -0.517*** & 0.027 &  & -0.500*** \\
 & [0.085] & [0.133] & [0.125] & [0.125] & [0.243] &  & [0.120] \\
Padova Reli ITC & -0.619*** & -0.832*** & -0.575*** & -0.575*** &  &  & -0.396 \\
 & [0.131] & [0.172] & [0.219] & [0.219] &  &  & [0.247] \\
Padova Priv ITC & 0.074*** &  &  &  & 0.552** &  & 0.053 \\
 & [0.027] &  &  &  & [0.232] &  & [0.041] \\
Reggio None * Age 40-50 &  &  & -0.569*** & -0.569*** & -0.623*** & -0.623*** & -0.689*** \\
 &  &  & [0.074] & [0.074] & [0.091] & [0.090] & [0.077] \\
Reggio Muni * Age 40-50 &  &  & -0.615*** & -0.615*** & -0.705*** & -0.705*** & -0.724*** \\
 &  &  & [0.105] & [0.105] & [0.116] & [0.115] & [0.106] \\
Reggio None * Age 40-50 &  &  & -0.146 & -0.146 & -0.317 & -0.069 & -0.160 \\
 &  &  & [0.183] & [0.183] & [0.798] & [0.197] & [0.178] \\
Reggio Muni * Age 40-50 &  &  & -0.015 & -0.015 & -0.293 & -0.044 & 0.005 \\
 &  &  & [0.182] & [0.182] & [0.802] & [0.194] & [0.179] \\
Reggio Stat * Age 40-50 &  &  & -0.004 & -0.004 & -0.256 & -0.007 & 0.015 \\
 &  &  & [0.195] & [0.195] & [0.804] & [0.206] & [0.192] \\
Reggio Reli * Age 40-50 &  &  & -0.085 & -0.085 & -0.327 & -0.078 & -0.084 \\
 &  &  & [0.183] & [0.183] & [0.797] & [0.195] & [0.180] \\
Reggio Priv * Age 40-50 &  &  &  &  & -0.249 &  &  \\
 &  &  &  &  & [0.819] &  &  \\
Male dummy &  &  & 0.036* & 0.036* & 0.030 & 0.030 & 0.031 \\
 &  &  & [0.019] & [0.019] & [0.028] & [0.027] & [0.020] \\
Constant & 0.926*** & 1.190*** & 3.450*** & 3.450*** & 4.081 & 0.868 & 2.333* \\
 & [0.027] & [0.033] & [1.278] & [1.278] & [2.682] & [1.514] & [1.279] \\
 &  &  &  &  &  &  &  \\
Observations & 1,874 & 1,874 & 1,874 & 1,874 & 1,874 & 744 & 1,874 \\
R-squared & 0.070 & 0.169 & 0.327 & 0.327 & 0.347 & 0.357 & 0.282 \\
 Controls & None & F.E. & Family & All & Inter & Reggio & no FE \\ \hline
\multicolumn{8}{c}{ Robust standard errors in brackets} \\
\multicolumn{8}{c}{ *** p$<$0.01, ** p$<$0.05, * p$<$0.1} \\
\multicolumn{8}{c}{ Dependent variable: Respondent health is good (\%).} \\
\end{tabular}

\end{table}
\begin{table}[H]
\caption{Pool Regression: Health - Preschool, Adult}
\begin{tabular}{lccccccc} \hline
 & (1) & (2) & (3) & (4) & (5) & (6) & (7) \\
VARIABLES &  &  &  &  &  &  &  \\ \hline
 &  &  &  &  &  &  &  \\
Reggio None Preschool & -0.324*** & -0.302*** & -0.043 & -0.043 & -0.035 & -0.035 & -0.057 \\
 & [0.035] & [0.036] & [0.044] & [0.044] & [0.044] & [0.043] & [0.042] \\
Reggio Stat Preschool & -0.090 & -0.086 & -0.050 & -0.050 & -0.036 & -0.036 & -0.047 \\
 & [0.057] & [0.056] & [0.061] & [0.061] & [0.063] & [0.062] & [0.060] \\
Reggio Reli Preschool & -0.142*** & -0.128*** & 0.026 & 0.026 & 0.026 & 0.026 & 0.027 \\
 & [0.044] & [0.044] & [0.029] & [0.029] & [0.031] & [0.031] & [0.030] \\
Reggio Priv Preschool & -0.140 & -0.119 & 0.126*** & 0.126*** & 0.037 & 0.037 & 0.159*** \\
 & [0.155] & [0.144] & [0.047] & [0.047] & [0.065] & [0.065] & [0.040] \\
Parma None Preschool & -0.399*** & 0.181 & 0.581*** & 0.581*** & 0.174** &  & -0.140** \\
 & [0.040] & [0.206] & [0.204] & [0.204] & [0.083] &  & [0.067] \\
Parma Muni Preschool & -0.359*** & 0.219 & 0.436** & 0.436** & 0.015 &  & -0.253*** \\
 & [0.045] & [0.207] & [0.201] & [0.201] & [0.077] &  & [0.054] \\
Parma Stat Preschool & -0.415*** & 0.213 & 0.414** & 0.414** &  &  & -0.285*** \\
 & [0.060] & [0.209] & [0.205] & [0.205] &  &  & [0.066] \\
Parma Reli Preschool & -0.223*** & 0.296 & 0.525** & 0.525** & 0.102 &  & -0.152** \\
 & [0.050] & [0.209] & [0.207] & [0.207] & [0.088] &  & [0.062] \\
Parma Priv Preschool & -0.556*** &  &  &  & -0.424** &  & -0.738*** \\
 & [0.194] &  &  &  & [0.198] &  & [0.157] \\
Padova None Preschool & -0.407*** & -0.307*** & 0.101 & 0.101 & -1.468 &  & -0.115* \\
 & [0.043] & [0.076] & [0.092] & [0.092] & [3.653] &  & [0.062] \\
Padova Muni Preschool & -0.299*** & -0.152* & -0.033 & -0.033 & -1.609 &  & -0.278*** \\
 & [0.064] & [0.091] & [0.124] & [0.124] & [3.646] &  & [0.090] \\
Padova Stat Preschool & -0.077 & 0.003 & 0.097 & 0.097 & -1.475 &  & -0.095 \\
 & [0.060] & [0.092] & [0.127] & [0.127] & [3.649] &  & [0.089] \\
Padova Reli Preschool & -0.348*** & -0.231*** & -0.054 & -0.054 & -1.610 &  & -0.237*** \\
 & [0.034] & [0.069] & [0.091] & [0.091] & [3.649] &  & [0.047] \\
Padova Priv Preschool & 0.110*** &  &  &  &  &  &  \\
 & [0.019] &  &  &  &  &  &  \\
Reggio None * Age 40-50 &  &  & -1.233*** & -1.233*** & -1.520* & -0.747*** & -1.519*** \\
 &  &  & [0.303] & [0.303] & [0.784] & [0.077] & [0.252] \\
Reggio Muni * Age 40-50 &  &  & -1.228*** & -1.228*** & -1.550** & -0.777*** & -1.497*** \\
 &  &  & [0.311] & [0.311] & [0.787] & [0.103] & [0.262] \\
Reggio Reli * Age 40-50 &  &  & -0.537* & -0.537* & -0.773 &  & -0.710*** \\
 &  &  & [0.306] & [0.306] & [0.796] &  & [0.257] \\
Reggio None * Age 40-50 &  &  & 0.025 & 0.025 & 0.003 & 0.003 & 0.055 \\
 &  &  & [0.193] & [0.193] & [0.208] & [0.206] & [0.183] \\
Reggio Muni * Age 40-50 &  &  & 0.110 & 0.110 & -0.007 & -0.007 & 0.163 \\
 &  &  & [0.190] & [0.190] & [0.206] & [0.204] & [0.183] \\
Reggio Stat * Age 40-50 &  &  & 0.174 & 0.174 & 0.066 & 0.066 & 0.221 \\
 &  &  & [0.213] & [0.213] & [0.230] & [0.228] & [0.204] \\
Reggio Reli * Age 40-50 &  &  & 0.015 & 0.015 & -0.066 & -0.066 & 0.048 \\
 &  &  & [0.190] & [0.190] & [0.204] & [0.202] & [0.184] \\
Male dummy &  &  & 0.035* & 0.035* & 0.027 & 0.027 & 0.031 \\
 &  &  & [0.020] & [0.020] & [0.028] & [0.028] & [0.020] \\
Constant & 0.890*** & 1.086*** & 3.673*** & 3.673*** & 4.963* & 0.968 & 2.504* \\
 & [0.019] & [0.056] & [1.287] & [1.287] & [2.560] & [1.532] & [1.307] \\
 &  &  &  &  &  &  &  \\
Observations & 1,874 & 1,874 & 1,874 & 1,874 & 1,874 & 744 & 1,874 \\
R-squared & 0.089 & 0.188 & 0.319 & 0.319 & 0.339 & 0.356 & 0.270 \\
 Controls & None & F.E. & Family & All & Inter & Reggio & no FE \\ \hline
\multicolumn{8}{c}{ Robust standard errors in brackets} \\
\multicolumn{8}{c}{ *** p$<$0.01, ** p$<$0.05, * p$<$0.1} \\
\multicolumn{8}{c}{ Dependent variable: Respondent health is good (\%).} \\
\end{tabular}

\end{table}
\begin{table}[H]
\caption{Pool Regression: Migration Taste - Infant-toddler center, Adult}
\begin{tabular}{lccccccc} \hline
 & (1) & (2) & (3) & (4) & (5) & (6) & (7) \\
VARIABLES &  &  &  &  &  &  &  \\ \hline
 &  &  &  &  &  &  &  \\
Reggio None ITC & 0.139*** & 0.134*** & 0.062 & 0.062 & 0.066 & 0.066 & 0.064 \\
 & [0.030] & [0.030] & [0.042] & [0.042] & [0.044] & [0.044] & [0.041] \\
Reggio Reli ITC & 0.187 & 0.170 & -0.095** & -0.095** & -0.099* & -0.099* & -0.084** \\
 & [0.219] & [0.220] & [0.040] & [0.040] & [0.054] & [0.054] & [0.042] \\
Reggio Priv ITC & -0.063** & -0.067** & -0.090** & -0.090** & -0.082 & -0.082 & -0.098*** \\
 & [0.025] & [0.029] & [0.039] & [0.039] & [0.050] & [0.050] & [0.038] \\
Parma None ITC & 0.162*** & 0.480* & 0.361 & 0.361 & -7.037* &  & 0.145*** \\
 & [0.032] & [0.272] & [0.304] & [0.304] & [3.735] &  & [0.047] \\
Parma Muni ITC & 0.128** & 0.480* & 0.432 & 0.432 & -6.980* &  & 0.185** \\
 & [0.054] & [0.278] & [0.312] & [0.312] & [3.742] &  & [0.084] \\
Parma Reli ITC & 0.270** & 0.576* & 0.747** & 0.747** & -6.683* &  & 0.572*** \\
 & [0.125] & [0.295] & [0.349] & [0.349] & [3.735] &  & [0.184] \\
Parma Priv ITC & 0.300** & 0.616** & 0.469 & 0.469 & -6.967* &  & 0.248* \\
 & [0.148] & [0.308] & [0.336] & [0.336] & [3.744] &  & [0.149] \\
Padova None ITC & 0.286*** & -0.294 & -0.319 & -0.319 & 0.080 &  & 0.250*** \\
 & [0.032] & [0.241] & [0.244] & [0.244] & [0.195] &  & [0.048] \\
Padova Muni ITC & 0.406*** & -0.252 & -0.358 & -0.358 & 0.013 &  & 0.265** \\
 & [0.092] & [0.258] & [0.273] & [0.273] & [0.227] &  & [0.128] \\
Padova Reli ITC & 0.508*** & -0.044 & -0.353 & -0.353 &  &  & 0.143 \\
 & [0.135] & [0.273] & [0.305] & [0.305] &  &  & [0.177] \\
Padova Priv ITC & 0.604** &  &  &  & 0.366 &  & 0.594** \\
 & [0.274] &  &  &  & [0.310] &  & [0.273] \\
Reggio None * Age 40-50 &  &  & -0.880*** & -0.880*** & -0.914 & -0.918*** & -0.869*** \\
 &  &  & [0.065] & [0.065] & [0.830] & [0.085] & [0.063] \\
Reggio Muni * Age 40-50 &  &  & -0.933*** & -0.933*** & -0.959 & -0.963*** & -0.925*** \\
 &  &  & [0.087] & [0.087] & [0.829] & [0.102] & [0.082] \\
Reggio Reli * Age 40-50 &  &  &  &  & 0.004 &  &  \\
 &  &  &  &  & [0.836] &  &  \\
Reggio None * Age 40-50 &  &  & 0.101 & 0.101 & 0.093 & 0.093 & 0.089 \\
 &  &  & [0.155] & [0.155] & [0.150] & [0.148] & [0.146] \\
Reggio Muni * Age 40-50 &  &  & 0.041 & 0.041 & 0.061 & 0.061 & 0.035 \\
 &  &  & [0.154] & [0.154] & [0.146] & [0.145] & [0.146] \\
Reggio Stat * Age 40-50 &  &  & 0.096 & 0.096 & 0.112 & 0.112 & 0.085 \\
 &  &  & [0.170] & [0.170] & [0.163] & [0.162] & [0.163] \\
Reggio Reli * Age 40-50 &  &  & 0.122 & 0.122 & 0.131 & 0.131 & 0.120 \\
 &  &  & [0.157] & [0.157] & [0.149] & [0.148] & [0.150] \\
Male dummy &  &  & 0.031 & 0.031 & 0.048 & 0.048 & 0.037* \\
 &  &  & [0.020] & [0.020] & [0.030] & [0.029] & [0.020] \\
Constant & 0.063** & -0.134*** & -0.443 & -0.443 & 4.425* & -2.121 & 0.042 \\
 & [0.025] & [0.030] & [1.376] & [1.376] & [2.667] & [1.935] & [1.359] \\
 &  &  &  &  &  &  &  \\
Observations & 1,866 & 1,866 & 1,866 & 1,866 & 1,866 & 739 & 1,866 \\
R-squared & 0.041 & 0.076 & 0.112 & 0.112 & 0.134 & 0.066 & 0.086 \\
 Controls & None & F.E. & Family & All & Inter & Reggio & no FE \\ \hline
\multicolumn{8}{c}{ Robust standard errors in brackets} \\
\multicolumn{8}{c}{ *** p$<$0.01, ** p$<$0.05, * p$<$0.1} \\
\multicolumn{8}{c}{ Dependent variable: Bothered by migrants (\%).} \\
\end{tabular}

\end{table}
\begin{table}[H]
\caption{Pool Regression: Migration Taste - Preschool, Adult}
\begin{tabular}{lccccccc} \hline
 & (1) & (2) & (3) & (4) & (5) & (6) & (7) \\
VARIABLES &  &  &  &  &  &  &  \\ \hline
 &  &  &  &  &  &  &  \\
Reggio None Preschool & 0.117*** & 0.120*** & 0.044 & 0.044 & 0.049 & 0.049 & 0.049 \\
 & [0.032] & [0.033] & [0.055] & [0.055] & [0.056] & [0.055] & [0.053] \\
Reggio Stat Preschool & 0.056 & 0.057 & 0.024 & 0.024 & 0.025 & 0.025 & 0.025 \\
 & [0.055] & [0.055] & [0.068] & [0.068] & [0.068] & [0.067] & [0.065] \\
Reggio Reli Preschool & 0.134*** & 0.128*** & 0.134* & 0.134* & 0.136* & 0.136* & 0.142* \\
 & [0.044] & [0.045] & [0.074] & [0.074] & [0.076] & [0.076] & [0.073] \\
Reggio Priv Preschool & 0.029 & 0.019 & -0.088* & -0.088* & -0.053 & -0.053 & -0.111*** \\
 & [0.134] & [0.137] & [0.050] & [0.050] & [0.068] & [0.067] & [0.043] \\
Parma None Preschool & 0.044 & 0.024 & 0.006 & 0.006 & -0.052 &  & 0.079 \\
 & [0.032] & [0.143] & [0.186] & [0.186] & [0.089] &  & [0.066] \\
Parma Muni Preschool & 0.172*** & 0.157 & 0.062 & 0.062 & -0.008 &  & 0.144*** \\
 & [0.042] & [0.144] & [0.180] & [0.180] & [0.081] &  & [0.054] \\
Parma Stat Preschool & 0.104** & 0.080 & 0.063 & 0.063 &  &  & 0.158** \\
 & [0.051] & [0.148] & [0.186] & [0.186] &  &  & [0.067] \\
Parma Reli Preschool & 0.172*** & 0.126 & 0.143 & 0.143 & 0.057 &  & 0.250*** \\
 & [0.048] & [0.147] & [0.189] & [0.189] & [0.093] &  & [0.075] \\
Parma Priv Preschool & 0.053 &  &  &  & -0.084 &  & 0.123 \\
 & [0.154] &  &  &  & [0.174] &  & [0.192] \\
Padova None Preschool & 0.219*** & 0.263*** & 0.347*** & 0.347*** & 0.325** &  & 0.295*** \\
 & [0.041] & [0.100] & [0.116] & [0.116] & [0.129] &  & [0.078] \\
Padova Muni Preschool & 0.386*** & 0.366*** & 0.461*** & 0.461*** & 0.429*** &  & 0.439*** \\
 & [0.065] & [0.114] & [0.136] & [0.136] & [0.144] &  & [0.093] \\
Padova Stat Preschool & 0.199*** & 0.201* & 0.112 & 0.112 & 0.063 &  & 0.041 \\
 & [0.070] & [0.119] & [0.120] & [0.120] & [0.131] &  & [0.072] \\
Padova Reli Preschool & 0.245*** & 0.285*** & 0.288*** & 0.288*** & 0.249** &  & 0.212*** \\
 & [0.034] & [0.099] & [0.107] & [0.107] & [0.118] &  & [0.049] \\
Padova Priv Preschool & -0.114*** &  &  &  &  &  & -0.047 \\
 & [0.019] &  &  &  &  &  & [0.046] \\
Reggio None * Age 40-50 &  &  & -0.555** & -0.555** & -0.539 & -0.105 & -0.530** \\
 &  &  & [0.270] & [0.270] & [0.798] & [0.461] & [0.259] \\
Reggio Muni * Age 40-50 &  &  & -0.671** & -0.671** & -0.652 & -0.218 & -0.651** \\
 &  &  & [0.272] & [0.272] & [0.796] & [0.458] & [0.263] \\
Reggio Reli * Age 40-50 &  &  & 0.169 & 0.169 & 0.210 & 0.645 & 0.188 \\
 &  &  & [0.272] & [0.272] & [0.810] & [0.481] & [0.265] \\
Reggio None * Age 40-50 &  &  & -0.030 & -0.030 & -0.008 & -0.008 & -0.070 \\
 &  &  & [0.169] & [0.169] & [0.167] & [0.166] & [0.156] \\
Reggio Muni * Age 40-50 &  &  & -0.046 & -0.046 & 0.008 & 0.008 & -0.076 \\
 &  &  & [0.162] & [0.162] & [0.162] & [0.161] & [0.152] \\
Reggio Stat * Age 40-50 &  &  & -0.017 & -0.017 & 0.032 & 0.032 & -0.052 \\
 &  &  & [0.191] & [0.191] & [0.192] & [0.190] & [0.179] \\
Reggio Reli * Age 40-50 &  &  & -0.100 & -0.100 & -0.057 & -0.057 & -0.134 \\
 &  &  & [0.176] & [0.176] & [0.174] & [0.172] & [0.168] \\
Male dummy &  &  & 0.028 & 0.028 & 0.048 & 0.048 & 0.035* \\
 &  &  & [0.020] & [0.020] & [0.030] & [0.030] & [0.020] \\
Constant & 0.114*** & -0.057 & -0.140 & -0.140 & 4.179* & -1.604 & 0.357 \\
 & [0.019] & [0.055] & [1.372] & [1.372] & [2.528] & [2.012] & [1.346] \\
 &  &  &  &  &  &  &  \\
Observations & 1,866 & 1,866 & 1,866 & 1,866 & 1,866 & 739 & 1,866 \\
R-squared & 0.048 & 0.081 & 0.115 & 0.115 & 0.137 & 0.069 & 0.090 \\
 Controls & None & F.E. & Family & All & Inter & Reggio & no FE \\ \hline
\multicolumn{8}{c}{ Robust standard errors in brackets} \\
\multicolumn{8}{c}{ *** p$<$0.01, ** p$<$0.05, * p$<$0.1} \\
\multicolumn{8}{c}{ Dependent variable: Bothered by migrants (\%).} \\
\end{tabular}

\end{table}
%\section{Appendix: Analysis Robustness Checks} 
\label{app:analysis}
This section reports the long-form tables of the preliminary results of the analysis. 

\subsection{Regression Results}
\label{app:OLS}

As in the main section of the paper, we ran the following regression for each city and age group:
\[ 
y = \alpha + \sum_{sc} \delta_{sc} D_{sc} + \beta_{X}X + \varepsilon
\]

in this case, $D_{sc}$ are dummies for city-specific school type including: municipal, religious, state, and private schools in Reggio, Parma, and Padova as well as dummies for not-attended any type of school.\footnote{Note that there are no state infant-toddler centers, and virtually every child and adolescent attended some form of preschool.} The omitted category of comparison for every city is municipal schools; since the treated group are the municipal schools in Reggio Emilia, the coefficients for the city of Reggio Emilia can be read as the difference between the treatment group and the specific type of school attended.

As mentioned before, the age groups considered are 0-3, regarding attendance to infant-toddler center, and age 3-6, for attendance to preschool. 

The following controls are considered incrementally through all the analysis: the first column of each table only controls for school types $D_{sc}$ and city dummies; the second column introduces interviewer fixed effects and interview mode (computer or paper); the third column introduces demographic controls such as gender, age at interview, age$^2$, dummies for parental education, dummies for parents born in the region (province), dummies for family owns home, family income brackets; the fourth column includes also dummies for low birthweight, and premature birth (when present)\footnote{Recall that adults did not report information on prematurity}; the fifth column interacts all of the controls with city-dummies, allowing the effect of controls to vary by city; the sixth column ran the regression only for the sample of respondents from Reggio Emilia; the final column does not control for interviewer fixed effects.


\singlespacing
\subsubsection{Children Results}
\begin{small}
\begin{table}[H]
\caption{Pool Regression: Child Health - Infant-toddler center, Child}
\begin{tabular}{lcccccc} \hline
 & (1) & (2) & (3) & (4) & (5) & (6) \\
VARIABLES &  &  &  &  &  &  \\ \hline
 &  &  &  &  &  &  \\
RCH infant-toddler &  &  &  & 0.332 & 0.328 & 0.637 \\
 &  &  &  & [0.819] & [0.548] & [0.878] \\
Reggio None ITC & 0.051 & 0.037 & 0.032 & 0.294 & 0.291 & 0.546 \\
 & [0.059] & [0.066] & [0.070] & [0.685] & [0.461] & [0.735] \\
Reggio Reli ITC & 0.072 & 0.078 & 0.064 &  &  &  \\
 & [0.094] & [0.095] & [0.098] &  &  &  \\
Parma Muni ITC & -0.067 & -0.127 & -13.600 & -14.907 & -14.894 & -15.861 \\
 & [0.059] & [0.087] & [15.523] & [14.831] & [14.646] & [15.327] \\
Padova Muni ITC & 0.146** & 0.182 & -22.781* & -24.088** & -24.076** & -25.044** \\
 & [0.067] & [0.125] & [12.122] & [11.802] & [11.561] & [12.397] \\
Male dummy &  & -0.062* & -0.113* & -0.118** & -0.118** & -0.123** \\
 &  & [0.035] & [0.058] & [0.056] & [0.056] & [0.060] \\
Parma Reli ITC & 0.081 & -0.048 & -13.630 & -14.937 & -14.925 & -15.892 \\
 & [0.222] & [0.257] & [15.533] & [14.840] & [14.656] & [15.336] \\
Padova Reli ITC & 0.070 & 0.217* & -22.726* & -24.033** & -24.021** & -24.989** \\
 & [0.101] & [0.131] & [12.130] & [11.809] & [11.568] & [12.403] \\
Parma Priv ITC & -0.069 & -0.111 & -13.573 & -14.880 & -14.867 & -15.834 \\
 & [0.099] & [0.116] & [15.529] & [14.837] & [14.652] & [15.332] \\
Padova Priv ITC & 0.031 & 0.051 & -22.896* & -24.203** & -24.190** & -25.158** \\
 & [0.093] & [0.136] & [12.129] & [11.808] & [11.567] & [12.403] \\
Reggio Priv ITC & 0.131 & 0.141 & 0.060 & 0.339 & 0.335 & 0.604 \\
 & [0.185] & [0.166] & [0.162] & [0.731] & [0.500] & [0.781] \\
Parma None ITC & -0.040 & -0.092 & -13.545 & -14.852 & -14.839 & -15.806 \\
 & [0.065] & [0.093] & [15.531] & [14.839] & [14.654] & [15.334] \\
Padova None ITC & 0.067 & 0.109 & -22.858* & -24.165** & -24.152** & -25.120** \\
 & [0.056] & [0.109] & [12.122] & [11.802] & [11.561] & [12.397] \\
Constant & 0.669*** & 7.666 & 19.197** & 20.504** & 20.491** & 21.460** \\
 & [0.040] & [5.387] & [8.969] & [9.067] & [8.755] & [9.834] \\
 &  &  &  &  &  &  \\
Observations & 768 & 768 & 768 & 768 & 768 & 768 \\
R-squared & 0.018 & 0.087 & 0.182 & 0.163 & 0.163 & 0.121 \\
Controls & None & All & Inter &  &  &  \\
 IV &  &  &  & distance & distXsib & dist score \\ \hline
\multicolumn{7}{c}{ Robust standard errors in brackets} \\
\multicolumn{7}{c}{ *** p$<$0.01, ** p$<$0.05, * p$<$0.1} \\
\multicolumn{7}{c}{ Dependent variable: Child health is good (\%) - mom report.} \\
\end{tabular}
  
\end{table}
\begin{table}[H]
\caption{Pool Regression: Child Health - Preschool, Child}
\begin{tabular}{lcccccc} \hline
 & (1) & (2) & (3) & (4) & (5) & (6) \\
VARIABLES &  &  &  &  &  &  \\ \hline
 &  &  &  &  &  &  \\
RCH preschool &  &  &  & 0.456 & 0.054 & 0.320 \\
 &  &  &  & [0.328] & [0.238] & [0.305] \\
Reggio None Preschool & -0.187 & -0.177 & -0.291 & 0.048 & -0.254 & -0.054 \\
 & [0.359] & [0.392] & [0.390] & [0.430] & [0.397] & [0.420] \\
Reggio Reli Preschool & 0.010 & 0.024 & 0.013 &  &  &  \\
 & [0.063] & [0.064] & [0.064] &  &  &  \\
Parma Muni Preschool & -0.101* & -0.150* & -15.669 & -21.558 & -16.520 & -19.859 \\
 & [0.058] & [0.081] & [15.842] & [15.489] & [14.708] & [15.145] \\
Padova Muni Preschool & 0.088 & 0.124 & -19.055 & -24.943** & -19.905* & -23.245* \\
 & [0.063] & [0.107] & [12.483] & [12.596] & [11.620] & [12.196] \\
Male dummy &  & -0.061* & -0.117** & -0.125** & -0.118** & -0.123** \\
 &  & [0.035] & [0.058] & [0.060] & [0.054] & [0.057] \\
Parma Stat Preschool & -0.087 & -0.127 & -15.644 & -21.533 & -16.494 & -19.833 \\
 & [0.087] & [0.112] & [15.847] & [15.493] & [14.712] & [15.149] \\
Padova Stat Preschool & 0.131* & 0.128 & -19.108 & -24.997** & -19.959* & -23.299* \\
 & [0.078] & [0.116] & [12.472] & [12.586] & [11.610] & [12.184] \\
Reggio Stat Preschool & 0.038 & 0.007 & -0.018 & 0.274 & 0.013 & 0.186 \\
 & [0.081] & [0.084] & [0.090] & [0.231] & [0.176] & [0.217] \\
Parma Reli Preschool & -0.036 & -0.098 & -15.632 & -21.521 & -16.482 & -19.821 \\
 & [0.072] & [0.096] & [15.846] & [15.493] & [14.711] & [15.149] \\
Padova Reli Preschool & 0.019 & 0.080 & -19.121 & -25.009** & -19.971* & -23.311* \\
 & [0.057] & [0.111] & [12.474] & [12.588] & [11.612] & [12.188] \\
Parma Priv Preschool & 0.091 & 0.051 & -15.479 & -21.368 & -16.330 & -19.669 \\
 & [0.145] & [0.157] & [15.860] & [15.506] & [14.725] & [15.161] \\
Padova Priv Preschool & 0.113 & 0.132 & -19.022 & -24.911** & -19.873* & -23.213* \\
 & [0.133] & [0.177] & [12.485] & [12.597] & [11.622] & [12.198] \\
Reggio Priv Preschool & 0.313*** & 0.254*** & 0.311*** & 0.568*** & 0.336** & 0.490** \\
 & [0.039] & [0.075] & [0.114] & [0.213] & [0.164] & [0.201] \\
Parma None Preschool & -0.187 & -0.257 & -15.793 & -21.683 & -16.644 & -19.983 \\
 & [0.210] & [0.252] & [15.844] & [15.491] & [14.710] & [15.147] \\
Padova None Preschool & 0.313*** & 0.348** & -18.956 & -24.844** & -19.806* & -23.146* \\
 & [0.039] & [0.164] & [12.491] & [12.603] & [11.628] & [12.204] \\
Constant & 0.687*** & 7.427 & 18.933** & 24.821** & 19.783** & 23.122** \\
 & [0.039] & [5.499] & [8.947] & [9.692] & [8.389] & [9.122] \\
 &  &  &  &  &  &  \\
Observations & 767 & 767 & 767 & 767 & 767 & 767 \\
R-squared & 0.024 & 0.087 & 0.184 & 0.118 & 0.182 & 0.151 \\
Controls & None & All & Inter &  &  &  \\
 IV &  &  &  & distance & distXsib & dist score \\ \hline
\multicolumn{7}{c}{ Robust standard errors in brackets} \\
\multicolumn{7}{c}{ *** p$<$0.01, ** p$<$0.05, * p$<$0.1} \\
\multicolumn{7}{c}{ Dependent variable: Child health is good (\%) - mom report.} \\
\end{tabular}

\end{table}
\begin{table}[H]
\caption{Pool Regression: Child SDQ Score - Infant-toddler center, Child}
\begin{tabular}{lcccccc} \hline
 & (1) & (2) & (3) & (4) & (5) & (6) \\
VARIABLES &  &  &  &  &  &  \\ \hline
 &  &  &  &  &  &  \\
RCH infant-toddler &  &  &  & -4.859 & -1.016 & -9.390 \\
 &  &  &  & [6.777] & [3.982] & [7.788] \\
Reggio None ITC & 1.350** & 1.121* & 1.173* & -2.935 & 0.232 & -6.671 \\
 & [0.555] & [0.582] & [0.638] & [5.609] & [3.307] & [6.477] \\
Reggio Reli ITC & 0.115 & 0.380 & 0.588 &  &  &  \\
 & [0.846] & [0.819] & [0.811] &  &  &  \\
Parma Muni ITC & 0.101 & 0.346 & 120.831 & 133.654 & 121.587 & 147.880 \\
 & [0.480] & [0.684] & [130.679] & [126.567] & [121.915] & [137.825] \\
Padova Muni ITC & 1.022 & 3.567*** & -6.643 & 6.180 & -5.890 & 20.399 \\
 & [0.703] & [1.310] & [129.093] & [125.158] & [120.390] & [136.693] \\
Male dummy &  & 0.572* & -0.053 & 0.005 & -0.047 & 0.067 \\
 &  & [0.312] & [0.522] & [0.503] & [0.485] & [0.566] \\
Parma Reli ITC & 2.180 & 1.496 & 122.063 & 134.886 & 122.818 & 149.111 \\
 & [1.678] & [1.659] & [130.705] & [126.590] & [121.940] & [137.843] \\
Padova Reli ITC & 1.364 & 3.577** & -6.560 & 6.263 & -5.806 & 20.482 \\
 & [1.098] & [1.730] & [128.875] & [124.965] & [120.190] & [136.514] \\
Parma Priv ITC & -0.937 & -0.506 & 119.932 & 132.755 & 120.687 & 146.980 \\
 & [0.630] & [0.736] & [130.688] & [126.576] & [121.924] & [137.833] \\
Padova Priv ITC & 0.563 & 3.674** & -6.622 & 6.201 & -5.868 & 20.420 \\
 & [0.958] & [1.441] & [129.219] & [125.270] & [120.504] & [136.798] \\
Reggio Priv ITC & -0.470 & 0.563 & 1.468 & -2.821 & 0.509 & -6.750 \\
 & [1.633] & [1.493] & [1.412] & [6.027] & [3.729] & [6.938] \\
Parma None ITC & 0.312 & 0.480 & 120.876 & 133.700 & 121.632 & 147.925 \\
 & [0.538] & [0.788] & [130.701] & [126.587] & [121.936] & [137.844] \\
Padova None ITC & 0.183 & 2.383** & -8.165 & 4.658 & -7.411 & 18.878 \\
 & [0.468] & [1.178] & [129.124] & [125.185] & [120.418] & [136.716] \\
Constant & 8.070*** & 109.946** & 89.010 & 76.190 & 88.255 & 61.968 \\
 & [0.316] & [50.353] & [89.836] & [91.037] & [84.483] & [106.102] \\
 &  &  &  &  &  &  \\
Observations & 769 & 769 & 769 & 769 & 769 & 769 \\
R-squared & 0.019 & 0.161 & 0.229 & 0.201 & 0.228 & 0.110 \\
Controls & None & All & Inter &  &  &  \\
 IV &  &  &  & distance & distXsib & dist score \\ \hline
\multicolumn{7}{c}{ Robust standard errors in brackets} \\
\multicolumn{7}{c}{ *** p$<$0.01, ** p$<$0.05, * p$<$0.1} \\
\multicolumn{7}{c}{ Dependent variable: SDQ score (mom rep.).} \\
\end{tabular}
  
\end{table}
\begin{table}[H]
\caption{Pool Regression: Child SDQ Score - Preschool, Child}
\begin{tabular}{lcccccc} \hline
 & (1) & (2) & (3) & (4) & (5) & (6) \\
VARIABLES &  &  &  &  &  &  \\ \hline
 &  &  &  &  &  &  \\
RCH preschool &  &  &  & -4.779* & -3.524** & -4.313* \\
 &  &  &  & [2.494] & [1.734] & [2.353] \\
Reggio None Preschool & 1.939 & 1.528 & 1.317 & -2.566 & -1.622 & -2.216 \\
 & [2.874] & [2.137] & [1.697] & [2.537] & [2.109] & [2.446] \\
Reggio Reli Preschool & 0.928* & 0.979* & 1.160** &  &  &  \\
 & [0.533] & [0.535] & [0.549] &  &  &  \\
Parma Muni Preschool & 0.006 & 0.070 & 140.901 & 185.108 & 169.382 & 179.292 \\
 & [0.469] & [0.677] & [131.518] & [126.483] & [121.663] & [125.098] \\
Padova Muni Preschool & 1.080* & 3.223*** & 45.401 & 89.618 & 73.879 & 83.789 \\
 & [0.643] & [1.127] & [132.147] & [127.051] & [122.220] & [125.567] \\
Male dummy &  & 0.578* & -0.106 & -0.039 & -0.063 & -0.048 \\
 &  & [0.307] & [0.518] & [0.513] & [0.495] & [0.505] \\
Parma Stat Preschool & -0.911 & -0.300 & 140.472 & 184.679 & 168.952 & 178.863 \\
 & [0.670] & [0.837] & [131.558] & [126.519] & [121.701] & [125.136] \\
Padova Stat Preschool & 0.636 & 1.906 & 44.074 & 88.290 & 72.551 & 82.462 \\
 & [0.710] & [1.240] & [131.951] & [126.877] & [122.039] & [125.396] \\
Reggio Stat Preschool & 1.789* & 1.596* & 1.756** & -1.763 & -0.945 & -1.460 \\
 & [0.957] & [0.880] & [0.862] & [1.801] & [1.399] & [1.706] \\
Parma Reli Preschool & 0.907 & 0.331 & 141.262 & 185.469 & 169.742 & 179.653 \\
 & [0.617] & [0.832] & [131.485] & [126.453] & [121.632] & [125.069] \\
Padova Reli Preschool & 0.005 & 1.926* & 44.144 & 88.361 & 72.622 & 82.533 \\
 & [0.493] & [1.131] & [132.042] & [126.958] & [122.123] & [125.476] \\
Parma Priv Preschool & 0.494 & 0.405 & 141.010 & 185.217 & 169.491 & 179.401 \\
 & [0.966] & [0.982] & [131.558] & [126.519] & [121.700] & [125.138] \\
Padova Priv Preschool & 1.539 & 4.431** & 46.634 & 90.851 & 75.112 & 85.023 \\
 & [1.761] & [1.962] & [132.224] & [127.120] & [122.291] & [125.639] \\
Reggio Priv Preschool & -1.728*** & -1.159* & -0.772 & -4.021** & -3.297** & -3.752** \\
 & [0.633] & [0.625] & [1.204] & [1.855] & [1.503] & [1.788] \\
Parma None Preschool & 0.605 & 0.756 & 141.289 & 185.495 & 169.769 & 179.679 \\
 & [1.417] & [1.438] & [131.519] & [126.484] & [121.664] & [125.105] \\
Padova None Preschool & 4.439*** & 6.174*** & 48.933 & 93.150 & 77.411 & 87.322 \\
 & [1.116] & [2.130] & [132.054] & [126.969] & [122.135] & [125.484] \\
Constant & 8.061*** & 133.167*** & 94.143 & 49.926 & 65.663 & 55.756 \\
 & [0.314] & [50.618] & [89.787] & [90.065] & [83.178] & [88.275] \\
 &  &  &  &  &  &  \\
Observations & 768 & 768 & 768 & 768 & 768 & 768 \\
R-squared & 0.026 & 0.168 & 0.231 & 0.183 & 0.211 & 0.195 \\
Controls & None & All & Inter &  &  &  \\
 IV &  &  &  & distance & distXsib & dist score \\ \hline
\multicolumn{7}{c}{ Robust standard errors in brackets} \\
\multicolumn{7}{c}{ *** p$<$0.01, ** p$<$0.05, * p$<$0.1} \\
\multicolumn{7}{c}{ Dependent variable: SDQ score (mom rep.).} \\
\end{tabular}

\end{table}
\end{small}

\pagebreak
\subsubsection{Adolescents Results}
\begin{table}[H]
\caption{Pool Regression: Child Health - Infant-toddler center, Adolescent}
\begin{tabular}{lcccccc} \hline
 & (1) & (2) & (3) & (4) & (5) & (6) \\
VARIABLES &  &  &  &  &  &  \\ \hline
 &  &  &  &  &  &  \\
RCH infant-toddler &  &  &  & -0.993 & -1.243 & -1.042 \\
 &  &  &  & [1.208] & [0.834] & [0.963] \\
Reggio None ITC & -0.068 & -0.078 & -0.053 & -0.989 & -1.225 & -1.036 \\
 & [0.063] & [0.068] & [0.072] & [1.141] & [0.794] & [0.911] \\
Reggio Reli ITC & 0.064 & 0.048 & 0.076 &  &  &  \\
 & [0.160] & [0.156] & [0.176] &  &  &  \\
Parma Muni ITC & -0.255*** & -0.245** & 93.460 & 48.820 & 42.633 & 57.473 \\
 & [0.073] & [0.100] & [133.128] & [187.932] & [105.399] & [121.222] \\
Padova Muni ITC & -0.074 & -0.074 & -27.863 & -63.666 & -98.715 & -69.976 \\
 & [0.082] & [0.125] & [123.039] & [185.849] & [132.118] & [115.162] \\
Male dummy &  & 0.058 & 0.088 & 0.097 & 0.099 & 0.097 \\
 &  & [0.038] & [0.066] & [0.065] & [0.067] & [0.065] \\
Parma Reli ITC & -0.186 & -0.246 & 93.422 & 48.784 & 42.594 & 57.434 \\
 & [0.165] & [0.175] & [133.110] & [187.916] & [105.385] & [121.206] \\
Padova Reli ITC & 0.314*** & 0.167 & -27.587 & -63.389 & -98.441 & -69.700 \\
 & [0.043] & [0.113] & [123.043] & [185.855] & [132.123] & [115.166] \\
Parma Priv ITC & 0.092 & 0.073 & 93.780 & 49.142 & 42.953 & 57.793 \\
 & [0.146] & [0.187] & [133.114] & [187.919] & [105.388] & [121.209] \\
Reggio Priv ITC & 0.314*** & 0.248*** & 0.255** & -0.616 & -0.832 & -0.659 \\
 & [0.043] & [0.086] & [0.122] & [1.061] & [0.753] & [0.849] \\
Parma None ITC & -0.084 & -0.053 & 93.681 & 49.042 & 42.854 & 57.694 \\
 & [0.066] & [0.093] & [133.124] & [187.929] & [105.396] & [121.218] \\
Padova None ITC & 0.042 & 0.009 & -27.772 & -63.575 & -98.624 & -69.885 \\
 & [0.055] & [0.109] & [123.042] & [185.851] & [132.120] & [115.165] \\
Constant & 0.686*** & -12.647 & -44.629 & -5.944 & 10.469 & -5.804 \\
 & [0.043] & [51.480] & [89.546] & [133.622] & [86.283] & [87.741] \\
 &  &  &  &  &  &  \\
Observations & 658 & 658 & 658 & 658 & 658 & 658 \\
R-squared & 0.043 & 0.136 & 0.205 & 0.168 & 0.146 & 0.164 \\
Controls & None & All & Inter &  &  &  \\
 IV &  &  &  & distance & distXsib & dist score \\ \hline
\multicolumn{7}{c}{ Robust standard errors in brackets} \\
\multicolumn{7}{c}{ *** p$<$0.01, ** p$<$0.05, * p$<$0.1} \\
\multicolumn{7}{c}{ Dependent variable: Child health is good (\%) - mom report.} \\
\end{tabular}
  
\end{table}
\begin{table}[H]
\caption{Pool Regression: Child Health - Preschool, Adolescent}
\begin{tabular}{lcccccc} \hline
 & (1) & (2) & (3) & (4) & (5) & (6) \\
VARIABLES &  &  &  &  &  &  \\ \hline
 &  &  &  &  &  &  \\
RCH preschool &  &  &  & 0.459 & 0.323 & 0.369 \\
 &  &  &  & [0.342] & [0.315] & [0.303] \\
Reggio None Preschool & 0.028 & 0.046 & 0.083 & 0.387 & 0.299 & 0.323 \\
 & [0.177] & [0.184] & [0.214] & [0.303] & [0.282] & [0.276] \\
Reggio Reli Preschool & -0.057 & -0.024 & -0.025 &  &  &  \\
 & [0.068] & [0.072] & [0.079] &  &  &  \\
Parma Muni Preschool & -0.175*** & -0.154 & 148.434 & 189.250 & 192.661 & 158.732 \\
 & [0.067] & [0.096] & [135.095] & [150.228] & [138.270] & [104.950] \\
Padova Muni Preschool & -0.039 & -0.027 & -15.011 & 55.235 & 8.824 & -16.252 \\
 & [0.070] & [0.110] & [125.950] & [167.994] & [112.852] & [87.199] \\
Male dummy &  & 0.045 & 0.066 & 0.045 & 0.052 & 0.050 \\
 &  & [0.038] & [0.067] & [0.069] & [0.066] & [0.066] \\
Parma Stat Preschool & -0.315*** & -0.267** & 148.333 & 189.149 & 192.560 & 158.631 \\
 & [0.092] & [0.119] & [135.096] & [150.229] & [138.271] & [104.951] \\
Padova Stat Preschool & 0.129* & 0.054 & -14.910 & 55.336 & 8.925 & -16.151 \\
 & [0.075] & [0.115] & [125.955] & [167.998] & [112.856] & [87.204] \\
Reggio Stat Preschool & 0.078 & 0.114 & 0.124 & 0.421* & 0.335 & 0.364* \\
 & [0.112] & [0.119] & [0.129] & [0.228] & [0.204] & [0.208] \\
Parma Reli Preschool & -0.036 & -0.027 & 148.562 & 189.377 & 192.789 & 158.860 \\
 & [0.073] & [0.093] & [135.098] & [150.231] & [138.274] & [104.953] \\
Padova Reli Preschool & 0.019 & -0.029 & -15.022 & 55.225 & 8.812 & -16.265 \\
 & [0.060] & [0.109] & [125.953] & [167.998] & [112.854] & [87.201] \\
Parma Priv Preschool & 0.313*** & 0.196 & 148.848 & 189.667 & 193.071 & 159.148 \\
 & [0.041] & [0.157] & [135.044] & [150.190] & [138.221] & [104.902] \\
Padova Priv Preschool & 0.063 & -0.088 & -15.121 & 55.124 & 8.714 & -16.362 \\
 & [0.223] & [0.254] & [125.949] & [167.992] & [112.851] & [87.198] \\
Reggio Priv Preschool & -0.487*** & -0.504** & -0.505** & -0.273 & -0.338 & -0.313 \\
 & [0.185] & [0.197] & [0.214] & [0.250] & [0.240] & [0.239] \\
Parma None Preschool & 0.313*** & 0.423*** & 149.180 & 189.991 & 193.415 & 159.475 \\
 & [0.041] & [0.161] & [135.132] & [150.258] & [138.308] & [104.984] \\
Padova None Preschool & -0.687*** & -0.991*** & -15.955 & 54.292 & 7.880 & -17.197 \\
 & [0.041] & [0.118] & [125.957] & [168.000] & [112.857] & [87.205] \\
Constant & 0.687*** & -10.167 & -56.525 & -104.909 & -87.144 & -71.071 \\
 & [0.041] & [51.504] & [90.368] & [113.335] & [89.825] & [72.457] \\
 &  &  &  &  &  &  \\
Observations & 660 & 660 & 660 & 660 & 660 & 660 \\
R-squared & 0.054 & 0.145 & 0.203 & 0.153 & 0.180 & 0.172 \\
Controls & None & All & Inter &  &  &  \\
 IV &  &  &  & distance & distXsib & dist score \\ \hline
\multicolumn{7}{c}{ Robust standard errors in brackets} \\
\multicolumn{7}{c}{ *** p$<$0.01, ** p$<$0.05, * p$<$0.1} \\
\multicolumn{7}{c}{ Dependent variable: Child health is good (\%) - mom report.} \\
\end{tabular}

\end{table}
\begin{table}[H]
\caption{Pool Regression: Health - Infant-toddler center, Adolescent}
\begin{tabular}{lcccccc} \hline
 & (1) & (2) & (3) & (4) & (5) & (6) \\
VARIABLES &  &  &  &  &  &  \\ \hline
 &  &  &  &  &  &  \\
RCH infant-toddler &  &  &  & 1.185 & 0.729 & 0.945 \\
 &  &  &  & [1.316] & [1.011] & [1.181] \\
Reggio None ITC & -0.143** & -0.138** & -0.119* & 1.004 & 0.577 & 0.779 \\
 & [0.059] & [0.063] & [0.067] & [1.245] & [0.951] & [1.117] \\
Reggio Reli ITC & -0.167 & -0.241 & -0.190 &  &  &  \\
 & [0.177] & [0.195] & [0.196] &  &  &  \\
Parma Muni ITC & -0.306*** & -0.270*** & 16.522 & 63.768 & 51.023 & 33.335 \\
 & [0.070] & [0.096] & [128.802] & [163.608] & [163.884] & [148.523] \\
Padova Muni ITC & -0.077 & -0.006 & -101.300 & -58.373 & -63.417 & -69.970 \\
 & [0.075] & [0.108] & [120.919] & [139.144] & [114.664] & [148.415] \\
Male dummy &  & 0.071* & 0.081 & 0.072 & 0.075 & 0.074 \\
 &  & [0.036] & [0.065] & [0.066] & [0.062] & [0.064] \\
Parma Reli ITC & -0.292* & -0.309* & 16.456 & 63.701 & 50.956 & 33.270 \\
 & [0.164] & [0.178] & [128.796] & [163.599] & [163.874] & [148.517] \\
Padova Reli ITC & -0.192 & -0.328 & -101.614 & -58.687 & -63.730 & -70.284 \\
 & [0.224] & [0.199] & [120.926] & [139.150] & [114.669] & [148.422] \\
Parma Priv ITC & -0.014 & -0.054 & 16.739 & 63.984 & 51.240 & 33.554 \\
 & [0.145] & [0.164] & [128.776] & [163.583] & [163.857] & [148.502] \\
Reggio Priv ITC & 0.208*** & 0.148* & 0.211* & 1.263 & 0.868 & 1.056 \\
 & [0.037] & [0.080] & [0.115] & [1.168] & [0.887] & [1.047] \\
Parma None ITC & -0.142** & -0.130 & 16.673 & 63.918 & 51.174 & 33.486 \\
 & [0.062] & [0.088] & [128.800] & [163.605] & [163.881] & [148.521] \\
Padova None ITC & -0.011 & 0.008 & -101.262 & -58.336 & -63.379 & -69.933 \\
 & [0.049] & [0.101] & [120.922] & [139.146] & [114.667] & [148.418] \\
Constant & 0.792*** & -13.380 & 10.464 & -32.091 & -19.704 & -15.788 \\
 & [0.037] & [48.633] & [86.193] & [108.910] & [98.786] & [108.477] \\
 &  &  &  &  &  &  \\
Observations & 657 & 657 & 657 & 657 & 657 & 657 \\
R-squared & 0.049 & 0.142 & 0.210 & 0.163 & 0.196 & 0.183 \\
Controls & None & All & Inter &  &  &  \\
 IV &  &  &  & distance & distXsib & dist score \\ \hline
\multicolumn{7}{c}{ Robust standard errors in brackets} \\
\multicolumn{7}{c}{ *** p$<$0.01, ** p$<$0.05, * p$<$0.1} \\
\multicolumn{7}{c}{ Dependent variable: Respondent health is good (\%).} \\
\end{tabular}

\end{table}
\begin{table}[H]
\caption{Pool Regression: Health - Preschool, Adolescent}
\begin{tabular}{lcccccc} \hline
 & (1) & (2) & (3) & (4) & (5) & (6) \\
VARIABLES &  &  &  &  &  &  \\ \hline
 &  &  &  &  &  &  \\
RCH preschool &  &  &  & 0.188 & 0.317 & 0.144 \\
 &  &  &  & [0.275] & [0.291] & [0.278] \\
Reggio None Preschool & -0.060 & -0.014 & -0.006 & 0.158 & 0.241 & 0.130 \\
 & [0.177] & [0.154] & [0.182] & [0.228] & [0.243] & [0.236] \\
Reggio Reli Preschool & -0.140** & -0.126* & -0.130* &  &  &  \\
 & [0.065] & [0.069] & [0.075] &  &  &  \\
Parma Muni Preschool & -0.149** & -0.134 & 53.784 & 49.644 & 82.201 & 47.092 \\
 & [0.064] & [0.091] & [127.740] & [96.593] & [138.747] & [173.692] \\
Padova Muni Preschool & -0.098 & 0.020 & -49.784 & -55.152 & -35.646 & -63.565 \\
 & [0.067] & [0.101] & [122.644] & [114.228] & [120.567] & [201.924] \\
Male dummy &  & 0.078** & 0.069 & 0.067 & 0.061 & 0.070 \\
 &  & [0.037] & [0.065] & [0.061] & [0.062] & [0.061] \\
Parma Stat Preschool & -0.432*** & -0.377*** & 53.561 & 49.421 & 81.978 & 46.868 \\
 & [0.089] & [0.117] & [127.739] & [96.593] & [138.746] & [173.691] \\
Padova Stat Preschool & 0.068 & 0.109 & -49.687 & -55.056 & -35.550 & -63.468 \\
 & [0.070] & [0.107] & [122.644] & [114.228] & [120.567] & [201.924] \\
Reggio Stat Preschool & 0.049 & 0.032 & 0.075 & 0.240 & 0.321* & 0.213 \\
 & [0.100] & [0.112] & [0.121] & [0.187] & [0.193] & [0.190] \\
Parma Reli Preschool & -0.145** & -0.153 & 53.765 & 49.624 & 82.182 & 47.072 \\
 & [0.072] & [0.094] & [127.741] & [96.594] & [138.748] & [173.693] \\
Padova Reli Preschool & 0.020 & 0.035 & -49.744 & -55.113 & -35.607 & -63.526 \\
 & [0.053] & [0.101] & [122.652] & [114.236] & [120.574] & [201.934] \\
Parma Priv Preschool & 0.226*** & 0.009 & 54.005 & 49.866 & 82.419 & 47.313 \\
 & [0.037] & [0.156] & [127.716] & [96.575] & [138.722] & [173.669] \\
Padova Priv Preschool & -0.024 & 0.023 & -49.758 & -55.127 & -35.621 & -63.539 \\
 & [0.222] & [0.217] & [122.640] & [114.225] & [120.564] & [201.920] \\
Reggio Priv Preschool & 0.026 & 0.029 & 0.002 & 0.160 & 0.219 & 0.139 \\
 & [0.185] & [0.174] & [0.186] & [0.201] & [0.203] & [0.199] \\
Parma None Preschool & 0.226*** & 0.240 & 54.164 & 50.022 & 82.587 & 47.471 \\
 & [0.037] & [0.168] & [127.757] & [96.601] & [138.769] & [173.714] \\
Padova None Preschool & -0.774*** & -0.969*** & -50.770 & -56.139 & -36.633 & -64.551 \\
 & [0.037] & [0.113] & [122.638] & [114.223] & [120.562] & [201.921] \\
Constant & 0.774*** & -6.432 & -9.041 & -7.995 & -27.750 & -3.231 \\
 & [0.037] & [48.381] & [86.739] & [76.632] & [89.359] & [125.086] \\
 &  &  &  &  &  &  \\
Observations & 659 & 659 & 659 & 659 & 659 & 659 \\
R-squared & 0.067 & 0.149 & 0.209 & 0.207 & 0.196 & 0.208 \\
Controls & None & All & Inter &  &  &  \\
 IV &  &  &  & distance & distXsib & dist score \\ \hline
\multicolumn{7}{c}{ Robust standard errors in brackets} \\
\multicolumn{7}{c}{ *** p$<$0.01, ** p$<$0.05, * p$<$0.1} \\
\multicolumn{7}{c}{ Dependent variable: Respondent health is good (\%).} \\
\end{tabular}

\end{table}
\begin{table}[H]
\caption{Pool Regression: SDQ Score - Infant-toddler center, Adolescent}
\begin{tabular}{lcccccc} \hline
 & (1) & (2) & (3) & (4) & (5) & (6) \\
VARIABLES &  &  &  &  &  &  \\ \hline
 &  &  &  &  &  &  \\
RCH infant-toddler &  &  &  & -9.777 & -8.030 & 1.110 \\
 &  &  &  & [10.866] & [8.976] & [9.115] \\
Reggio None ITC & 0.709 & 0.574 & 0.792 & -8.444 & -6.806 & 1.775 \\
 & [0.524] & [0.578] & [0.614] & [10.259] & [8.470] & [8.615] \\
Reggio Reli ITC & 1.706 & 1.885 & 1.103 &  &  &  \\
 & [2.801] & [2.890] & [2.724] &  &  &  \\
Parma Muni ITC & 0.067 & 0.213 & 1,063.479 & 871.207 & 944.179 & 1,231.790 \\
 & [0.610] & [0.739] & [1,057.425] & [1,245.749] & [1,195.692] & [1,058.251] \\
Padova Muni ITC & 0.628 & 1.952 & 564.902 & 319.788 & 329.512 & 926.970 \\
 & [0.735] & [1.214] & [936.956] & [868.099] & [870.198] & [1,360.411] \\
Male dummy &  & 0.484 & -0.444 & -0.371 & -0.386 & -0.468 \\
 &  & [0.337] & [0.629] & [0.651] & [0.633] & [0.595] \\
Parma Reli ITC & 2.031 & 1.241 & 1,064.948 & 872.664 & 945.633 & 1,233.263 \\
 & [2.313] & [2.086] & [1,057.115] & [1,245.454] & [1,195.387] & [1,058.025] \\
Padova Reli ITC & -0.269 & 0.346 & 562.708 & 317.597 & 327.316 & 924.796 \\
 & [1.145] & [1.487] & [937.064] & [868.175] & [870.287] & [1,360.540] \\
Parma Priv ITC & 2.553 & 2.410 & 1,065.950 & 873.665 & 946.633 & 1,234.265 \\
 & [1.784] & [2.061] & [1,057.692] & [1,245.962] & [1,195.912] & [1,058.471] \\
Reggio Priv ITC & 1.997 & 2.621 & 1.565 & -7.066 & -5.549 & 2.382 \\
 & [1.953] & [2.618] & [3.202] & [10.209] & [8.477] & [8.818] \\
Parma None ITC & -0.210 & -0.012 & 1,063.189 & 870.913 & 943.885 & 1,231.500 \\
 & [0.564] & [0.686] & [1,057.410] & [1,245.731] & [1,195.672] & [1,058.235] \\
Padova None ITC & 0.536 & 1.648* & 564.305 & 319.191 & 328.915 & 926.372 \\
 & [0.453] & [0.987] & [937.050] & [868.171] & [870.278] & [1,360.496] \\
Constant & 7.669*** & -143.581 & -716.666 & -449.868 & -501.805 & -910.139 \\
 & [0.351] & [405.361] & [633.960] & [708.415] & [660.832] & [772.144] \\
 &  &  &  &  &  &  \\
Observations & 654 & 654 & 654 & 654 & 654 & 654 \\
R-squared & 0.015 & 0.138 & 0.212 & 0.168 & 0.184 & 0.209 \\
Controls & None & All & Inter &  &  &  \\
 IV &  &  &  & distance & distXsib & dist score \\ \hline
\multicolumn{7}{c}{ Robust standard errors in brackets} \\
\multicolumn{7}{c}{ *** p$<$0.01, ** p$<$0.05, * p$<$0.1} \\
\multicolumn{7}{c}{ Dependent variable: SDQ score (mom rep.).} \\
\end{tabular}

\end{table}
\begin{table}[H]
\caption{Pool Regression: SDQ Score - Preschool, Adolescent}
\begin{tabular}{lcccccc} \hline
 & (1) & (2) & (3) & (4) & (5) & (6) \\
VARIABLES &  &  &  &  &  &  \\ \hline
 &  &  &  &  &  &  \\
RCH preschool &  &  &  & 0.090 & 0.059 & 0.241 \\
 &  &  &  & [2.353] & [2.194] & [2.276] \\
Reggio None Preschool & 0.977 & 0.827 & 1.484 & 1.175 & 1.095 & 1.198 \\
 & [1.356] & [1.141] & [1.487] & [2.028] & [1.960] & [2.009] \\
Reggio Reli Preschool & 0.754 & 1.118 & 1.111 &  &  &  \\
 & [0.660] & [0.700] & [0.789] &  &  &  \\
Parma Muni Preschool & -0.358 & -0.026 & 1,196.914 & 1,625.024 & 1,147.054 & 1,161.008 \\
 & [0.577] & [0.742] & [1,046.426] & [1,187.411] & [829.584] & [847.483] \\
Padova Muni Preschool & -0.083 & 0.801 & 653.899 & 1,138.680 & 786.317 & 668.882 \\
 & [0.522] & [1.036] & [931.706] & [1,087.827] & [930.675] & [818.482] \\
Male dummy &  & 0.522 & -0.060 & -0.127 & -0.113 & -0.119 \\
 &  & [0.338] & [0.630] & [0.593] & [0.593] & [0.599] \\
Parma Stat Preschool & -0.252 & -0.508 & 1,196.446 & 1,624.553 & 1,146.589 & 1,160.543 \\
 & [0.724] & [0.951] & [1,046.450] & [1,187.430] & [829.606] & [847.506] \\
Padova Stat Preschool & 0.903 & 2.182** & 655.397 & 1,140.179 & 787.815 & 670.379 \\
 & [0.725] & [1.064] & [931.739] & [1,087.856] & [930.705] & [818.512] \\
Reggio Stat Preschool & -0.822 & -0.832 & -1.010 & -1.364 & -1.387 & -1.274 \\
 & [0.727] & [0.913] & [0.960] & [1.761] & [1.691] & [1.738] \\
Parma Reli Preschool & 0.484 & 0.636 & 1,197.436 & 1,625.549 & 1,147.573 & 1,161.528 \\
 & [0.719] & [0.784] & [1,046.505] & [1,187.479] & [829.650] & [847.555] \\
Padova Reli Preschool & 0.438 & 1.870 & 654.808 & 1,139.604 & 787.232 & 669.788 \\
 & [0.518] & [1.144] & [931.721] & [1,087.848] & [930.694] & [818.494] \\
Parma Priv Preschool & 4.619 & 5.491 & 1,201.064 & 1,629.136 & 1,151.243 & 1,165.184 \\
 & [4.661] & [4.950] & [1,046.316] & [1,187.293] & [829.520] & [847.398] \\
Padova Priv Preschool & -1.381 & -0.026 & 653.254 & 1,138.033 & 785.671 & 668.237 \\
 & [1.093] & [1.544] & [931.786] & [1,087.894] & [930.750] & [818.556] \\
Reggio Priv Preschool & 0.719 & 1.148 & 1.113 & 0.532 & 0.553 & 0.644 \\
 & [1.399] & [1.533] & [1.605] & [1.798] & [1.756] & [1.784] \\
Parma None Preschool & 0.619 & 2.620 & 1,199.053 & 1,627.233 & 1,149.123 & 1,163.101 \\
 & [3.236] & [3.266] & [1,046.841] & [1,187.804] & [829.891] & [847.838] \\
Padova None Preschool & 6.119*** & 11.659*** & 663.899 & 1,148.688 & 796.320 & 678.880 \\
 & [0.328] & [1.058] & [931.722] & [1,087.847] & [930.693] & [818.497] \\
Constant & 7.881*** & -66.482 & -655.111 & -968.258 & -719.733 & -694.300 \\
 & [0.328] & [398.587] & [645.947] & [738.196] & [592.861] & [553.835] \\
 &  &  &  &  &  &  \\
Observations & 656 & 656 & 656 & 656 & 656 & 656 \\
R-squared & 0.018 & 0.151 & 0.215 & 0.210 & 0.210 & 0.209 \\
Controls & None & All & Inter &  &  &  \\
 IV &  &  &  & distance & distXsib & dist score \\ \hline
\multicolumn{7}{c}{ Robust standard errors in brackets} \\
\multicolumn{7}{c}{ *** p$<$0.01, ** p$<$0.05, * p$<$0.1} \\
\multicolumn{7}{c}{ Dependent variable: SDQ score (mom rep.).} \\
\end{tabular}

\end{table}
\begin{table}[H]
\caption{Pool Regression: Depression - Infant-toddler center, Adolescent}
\begin{tabular}{lcccccc} \hline
 & (1) & (2) & (3) & (4) & (5) & (6) \\
VARIABLES &  &  &  &  &  &  \\ \hline
 &  &  &  &  &  &  \\
RCH infant-toddler &  &  &  & 6.597 & 3.000 & 7.377 \\
 &  &  &  & [14.626] & [13.601] & [14.501] \\
Reggio None ITC & 0.042 & 0.014 & 0.720 & 6.697 & 3.351 & 7.448 \\
 & [0.846] & [0.892] & [0.930] & [13.634] & [12.613] & [13.548] \\
Reggio Reli ITC & 1.657 & 2.605 & 2.719 &  &  &  \\
 & [3.181] & [3.351] & [3.237] &  &  &  \\
Parma Muni ITC & -0.808 & -0.690 & 739.388 & 916.038 & 1,074.987 & 1,366.649 \\
 & [0.833] & [1.198] & [1,607.270] & [1,172.845] & [1,789.776] & [2,363.521] \\
Padova Muni ITC & 0.141 & 0.614 & -341.762 & -149.959 & -2.874 & 331.855 \\
 & [1.138] & [1.734] & [1,625.822] & [1,394.413] & [1,593.882] & [2,450.801] \\
Male dummy &  & -1.322*** & -1.552* & -1.633** & -1.608** & -1.653* \\
 &  & [0.464] & [0.879] & [0.828] & [0.813] & [0.846] \\
Parma Reli ITC & -1.393 & -1.745 & 739.777 & 916.441 & 1,075.369 & 1,367.026 \\
 & [1.384] & [1.751] & [1,607.059] & [1,172.686] & [1,789.578] & [2,363.336] \\
Padova Reli ITC & -1.193 & 1.127 & -341.841 & -150.047 & -2.948 & 331.791 \\
 & [2.877] & [3.736] & [1,625.787] & [1,394.377] & [1,593.850] & [2,450.814] \\
Parma Priv ITC & -1.038 & -0.454 & 739.651 & 916.315 & 1,075.243 & 1,366.900 \\
 & [1.611] & [1.859] & [1,607.043] & [1,172.668] & [1,789.566] & [2,363.324] \\
Reggio Priv ITC & -2.593 & -1.567 & -1.551 & 3.783 & 0.674 & 4.457 \\
 & [3.818] & [3.803] & [4.866] & [13.897] & [12.650] & [13.727] \\
Parma None ITC & -0.963 & -0.171 & 739.536 & 916.190 & 1,075.134 & 1,366.794 \\
 & [0.781] & [1.149] & [1,607.236] & [1,172.823] & [1,789.742] & [2,363.486] \\
Padova None ITC & -1.996*** & -0.954 & -343.069 & -151.267 & -4.182 & 330.547 \\
 & [0.715] & [1.516] & [1,625.902] & [1,394.488] & [1,593.946] & [2,450.860] \\
Constant & 22.593*** & 233.382 & -11.729 & -271.832 & -305.235 & -563.379 \\
 & [0.576] & [616.795] & [1,252.379] & [1,108.556] & [1,350.607] & [1,820.327] \\
 &  &  &  &  &  &  \\
Observations & 640 & 640 & 640 & 640 & 640 & 640 \\
R-squared & 0.023 & 0.141 & 0.239 & 0.213 & 0.230 & 0.209 \\
Controls & None & All & Inter &  &  &  \\
 IV &  &  &  & distance & distXsib & dist score \\ \hline
\multicolumn{7}{c}{ Robust standard errors in brackets} \\
\multicolumn{7}{c}{ *** p$<$0.01, ** p$<$0.05, * p$<$0.1} \\
\multicolumn{7}{c}{ Dependent variable: Depression score (CESD).} \\
\end{tabular}

\end{table}
\begin{table}[H]
\caption{Pool Regression: Depression - Preschool, Adolescent}
\begin{tabular}{lcccccc} \hline
 & (1) & (2) & (3) & (4) & (5) & (6) \\
VARIABLES &  &  &  &  &  &  \\ \hline
 &  &  &  &  &  &  \\
RCH preschool &  &  &  & 3.134 & 3.068 & 2.321 \\
 &  &  &  & [4.143] & [3.885] & [3.793] \\
Reggio None Preschool & -2.811* & -2.561* & -1.165 & 0.084 & -0.025 & -0.488 \\
 & [1.516] & [1.474] & [1.669] & [2.979] & [2.787] & [2.695] \\
Reggio Reli Preschool & 2.109** & 1.902* & 2.022* &  &  &  \\
 & [0.954] & [1.025] & [1.116] &  &  &  \\
Parma Muni Preschool & -0.342 & 0.009 & 542.840 & 1,538.000 & 977.718 & 934.417 \\
 & [0.767] & [1.217] & [1,596.190] & [2,610.306] & [1,483.923] & [1,519.604] \\
Padova Muni Preschool & -0.654 & 0.164 & -368.563 & 533.518 & 157.172 & 63.197 \\
 & [0.889] & [1.633] & [1,649.931] & [2,339.134] & [1,571.363] & [1,664.644] \\
Male dummy &  & -1.464*** & -1.579* & -1.846** & -1.828** & -1.793** \\
 &  & [0.462] & [0.866] & [0.882] & [0.878] & [0.856] \\
Parma Stat Preschool & -1.895* & -1.445 & 541.656 & 1,536.801 & 976.536 & 933.234 \\
 & [1.027] & [1.303] & [1,596.080] & [2,610.193] & [1,483.829] & [1,519.507] \\
Padova Stat Preschool & -2.697*** & -1.801 & -370.449 & 531.634 & 155.287 & 61.312 \\
 & [1.030] & [1.755] & [1,649.944] & [2,339.143] & [1,571.372] & [1,664.658] \\
Reggio Stat Preschool & -1.579 & -1.715 & -1.184 & -0.097 & -0.146 & -0.594 \\
 & [1.392] & [1.606] & [1.709] & [3.095] & [2.946] & [2.902] \\
Parma Reli Preschool & 0.724 & 1.086 & 544.224 & 1,539.385 & 979.101 & 935.801 \\
 & [0.861] & [1.137] & [1,596.167] & [2,610.288] & [1,483.903] & [1,519.582] \\
Padova Reli Preschool & -0.395 & 0.602 & -368.449 & 533.647 & 157.291 & 63.313 \\
 & [0.816] & [1.570] & [1,649.985] & [2,339.199] & [1,571.412] & [1,664.701] \\
Parma Priv Preschool & 0.046 & 2.257 & 544.907 & 1,539.977 & 979.802 & 936.492 \\
 & [2.219] & [1.871] & [1,595.791] & [2,609.808] & [1,483.589] & [1,519.263] \\
Padova Priv Preschool & -3.204 & -1.956 & -371.114 & 530.969 & 154.622 & 60.646 \\
 & [2.625] & [2.711] & [1,649.956] & [2,339.152] & [1,571.383] & [1,664.673] \\
Reggio Priv Preschool & 1.646 & 2.029 & 1.841 & 2.238 & 2.244 & 1.888 \\
 & [1.820] & [1.833] & [1.970] & [2.758] & [2.697] & [2.657] \\
Parma None Preschool & -2.454 & -2.387 & 540.896 & 1,536.204 & 975.747 & 932.463 \\
 & [3.975] & [3.428] & [1,596.669] & [2,610.996] & [1,484.322] & [1,520.013] \\
Padova None Preschool & 8.046*** & 14.663*** & -354.997 & 547.089 & 170.740 & 76.764 \\
 & [0.561] & [1.781] & [1,650.005] & [2,339.192] & [1,571.425] & [1,664.714] \\
Constant & 21.954*** & 347.927 & 225.747 & -530.942 & -252.465 & -183.278 \\
 & [0.561] & [618.666] & [1,256.603] & [1,761.126] & [1,200.417] & [1,225.891] \\
 &  &  &  &  &  &  \\
Observations & 641 & 641 & 641 & 641 & 641 & 641 \\
R-squared & 0.046 & 0.164 & 0.258 & 0.208 & 0.209 & 0.222 \\
Controls & None & All & Inter &  &  &  \\
 IV &  &  &  & distance & distXsib & dist score \\ \hline
\multicolumn{7}{c}{ Robust standard errors in brackets} \\
\multicolumn{7}{c}{ *** p$<$0.01, ** p$<$0.05, * p$<$0.1} \\
\multicolumn{7}{c}{ Dependent variable: Depression score (CESD).} \\
\end{tabular}

\end{table}
\begin{table}[H]
\caption{Pool Regression: Migration Taste - Infant-toddler center, Adolescent}
\begin{tabular}{lcccccc} \hline
 & (1) & (2) & (3) & (4) & (5) & (6) \\
VARIABLES &  &  &  &  &  &  \\ \hline
 &  &  &  &  &  &  \\
RCH infant-toddler &  &  &  & 0.476 & 1.118 & 0.141 \\
 &  &  &  & [1.096] & [0.822] & [0.880] \\
Reggio None ITC & -0.023 & -0.030 & -0.016 & 0.454 & 1.053 & 0.139 \\
 & [0.061] & [0.066] & [0.071] & [1.029] & [0.777] & [0.823] \\
Reggio Reli ITC & -0.317*** & -0.317*** & -0.355*** &  &  &  \\
 & [0.043] & [0.073] & [0.096] &  &  &  \\
Parma Muni ITC & -0.091 & -0.072 & -93.525 & -64.157 & -72.206 & -110.400 \\
 & [0.066] & [0.079] & [111.205] & [146.849] & [105.844] & [93.799] \\
Padova Muni ITC & -0.046 & 0.171 & 118.456 & 165.065 & 130.373 & 101.195 \\
 & [0.078] & [0.115] & [126.634] & [192.007] & [105.926] & [98.578] \\
Male dummy &  & 0.021 & 0.015 & 0.012 & 0.007 & 0.017 \\
 &  & [0.034] & [0.065] & [0.061] & [0.062] & [0.060] \\
Parma Reli ITC & -0.317*** & -0.272*** & -93.674 & -64.307 & -72.355 & -110.549 \\
 & [0.043] & [0.086] & [111.207] & [146.849] & [105.846] & [93.803] \\
Padova Reli ITC & -0.117 & 0.104 & 118.506 & 165.118 & 130.422 & 101.245 \\
 & [0.185] & [0.166] & [126.640] & [192.016] & [105.929] & [98.581] \\
Parma Priv ITC & 0.017 & 0.103 & -93.262 & -63.895 & -71.943 & -110.137 \\
 & [0.164] & [0.163] & [111.177] & [146.829] & [105.819] & [93.773] \\
Reggio Priv ITC & -0.317*** & -0.223*** & -0.242 & 0.217 & 0.774 & -0.072 \\
 & [0.043] & [0.083] & [0.147] & [0.949] & [0.731] & [0.763] \\
Parma None ITC & -0.136** & -0.109 & -93.519 & -64.151 & -72.200 & -110.394 \\
 & [0.059] & [0.077] & [111.205] & [146.849] & [105.844] & [93.800] \\
Padova None ITC & -0.068 & 0.132 & 118.393 & 165.003 & 130.310 & 101.133 \\
 & [0.055] & [0.106] & [126.640] & [192.012] & [105.931] & [98.583] \\
Constant & 0.317*** & -6.689 & -43.965 & -68.556 & -66.306 & -30.552 \\
 & [0.043] & [46.909] & [88.901] & [122.230] & [82.180] & [77.303] \\
 &  &  &  &  &  &  \\
Observations & 641 & 641 & 641 & 641 & 641 & 641 \\
R-squared & 0.021 & 0.162 & 0.240 & 0.239 & 0.208 & 0.237 \\
Controls & None & All & Inter &  &  &  \\
 IV &  &  &  & distance & distXsib & dist score \\ \hline
\multicolumn{7}{c}{ Robust standard errors in brackets} \\
\multicolumn{7}{c}{ *** p$<$0.01, ** p$<$0.05, * p$<$0.1} \\
\multicolumn{7}{c}{ Dependent variable: Bothered by migrants (\%).} \\
\end{tabular}

\end{table}
\begin{table}[H]
\caption{Pool Regression: Migration Taste - Preschool, Adolescent}
\begin{tabular}{lcccccc} \hline
 & (1) & (2) & (3) & (4) & (5) & (6) \\
VARIABLES &  &  &  &  &  &  \\ \hline
 &  &  &  &  &  &  \\
RCH preschool &  &  &  & -0.032 & 0.189 & -0.102 \\
 &  &  &  & [0.291] & [0.270] & [0.291] \\
Reggio None Preschool & -0.092 & -0.130 & -0.097 & -0.148 & -0.011 & -0.198 \\
 & [0.139] & [0.137] & [0.145] & [0.232] & [0.222] & [0.234] \\
Reggio Reli Preschool & 0.136** & 0.090 & 0.089 &  &  &  \\
 & [0.066] & [0.069] & [0.073] &  &  &  \\
Parma Muni Preschool & -0.011 & 0.034 & -100.510 & -95.114 & -77.221 & -120.811 \\
 & [0.059] & [0.079] & [108.735] & [101.542] & [113.582] & [130.179] \\
Padova Muni Preschool & -0.032 & 0.152 & 110.207 & 141.293 & 109.346 & 77.093 \\
 & [0.062] & [0.103] & [125.346] & [132.387] & [110.570] & [114.265] \\
Male dummy &  & 0.022 & 0.033 & 0.030 & 0.020 & 0.034 \\
 &  & [0.034] & [0.064] & [0.059] & [0.061] & [0.059] \\
Parma Stat Preschool & -0.121* & -0.122 & -100.639 & -95.243 & -77.350 & -120.940 \\
 & [0.066] & [0.079] & [108.734] & [101.542] & [113.581] & [130.178] \\
Padova Stat Preschool & -0.013 & 0.275** & 110.337 & 141.423 & 109.476 & 77.223 \\
 & [0.079] & [0.121] & [125.354] & [132.394] & [110.577] & [114.272] \\
Reggio Stat Preschool & 0.000 & 0.012 & -0.011 & -0.064 & 0.074 & -0.108 \\
 & [0.111] & [0.112] & [0.113] & [0.219] & [0.210] & [0.220] \\
Parma Reli Preschool & -0.018 & 0.018 & -100.531 & -95.134 & -77.241 & -120.831 \\
 & [0.065] & [0.082] & [108.743] & [101.550] & [113.589] & [130.186] \\
Padova Reli Preschool & 0.064 & 0.327*** & 110.362 & 141.450 & 109.499 & 77.246 \\
 & [0.058] & [0.107] & [125.352] & [132.394] & [110.574] & [114.270] \\
Parma Priv Preschool & -0.235*** & 0.016 & -100.384 & -94.985 & -77.098 & -120.683 \\
 & [0.037] & [0.085] & [108.732] & [101.540] & [113.577] & [130.170] \\
Padova Priv Preschool & 0.015 & 0.454* & 110.489 & 141.574 & 109.628 & 77.375 \\
 & [0.222] & [0.270] & [125.318] & [132.360] & [110.547] & [114.243] \\
Reggio Priv Preschool & 0.565*** & 0.549*** & 0.442** & 0.379 & 0.488** & 0.349 \\
 & [0.185] & [0.200] & [0.211] & [0.244] & [0.241] & [0.242] \\
Parma None Preschool & 0.265 & 0.337 & -100.241 & -94.849 & -76.948 & -120.544 \\
 & [0.360] & [0.295] & [108.767] & [101.571] & [113.614] & [130.218] \\
Padova None Preschool & -0.235*** & 0.253** & 110.362 & 141.448 & 109.500 & 77.247 \\
 & [0.037] & [0.125] & [125.359] & [132.399] & [110.580] & [114.275] \\
Constant & 0.235*** & 18.948 & -15.178 & -28.240 & -31.441 & 0.958 \\
 & [0.037] & [45.891] & [87.480] & [84.131] & [86.327] & [90.251] \\
 &  &  &  &  &  &  \\
Observations & 643 & 643 & 643 & 643 & 643 & 643 \\
R-squared & 0.036 & 0.179 & 0.249 & 0.248 & 0.222 & 0.249 \\
Controls & None & All & Inter &  &  &  \\
 IV &  &  &  & distance & distXsib & dist score \\ \hline
\multicolumn{7}{c}{ Robust standard errors in brackets} \\
\multicolumn{7}{c}{ *** p$<$0.01, ** p$<$0.05, * p$<$0.1} \\
\multicolumn{7}{c}{ Dependent variable: Bothered by migrants (\%).} \\
\end{tabular}

\end{table}
\begin{table}[H]
\caption{Pool Regression: SDQ Score - Infant-toddler center, Adolescent}
\begin{tabular}{lcccccc} \hline
 & (1) & (2) & (3) & (4) & (5) & (6) \\
VARIABLES &  &  &  &  &  &  \\ \hline
 &  &  &  &  &  &  \\
RCH infant-toddler &  &  &  & -6.665 & -4.903 & -1.868 \\
 &  &  &  & [10.735] & [6.970] & [7.754] \\
Reggio None ITC & 0.484 & 0.323 & 0.629 & -5.862 & -4.204 & -1.333 \\
 & [0.611] & [0.664] & [0.686] & [10.158] & [6.644] & [7.303] \\
Reggio Reli ITC & 2.500 & 3.109 & 3.485 &  &  &  \\
 & [2.377] & [2.550] & [2.701] &  &  &  \\
Parma Muni ITC & -0.819 & -0.687 & -653.026 & -1,140.036 & -1,143.250 & -429.159 \\
 & [0.690] & [0.916] & [1,129.264] & [1,550.525] & [1,647.722] & [856.313] \\
Padova Muni ITC & 0.926 & 2.394* & -869.728 & -1,626.786 & -1,210.755 & -681.254 \\
 & [0.838] & [1.225] & [1,206.440] & [2,084.046] & [1,498.464] & [988.961] \\
Male dummy &  & -0.379 & -0.534 & -0.489 & -0.510 & -0.555 \\
 &  & [0.353] & [0.668] & [0.641] & [0.628] & [0.614] \\
Parma Reli ITC & -0.525 & -0.857 & -652.746 & -1,139.746 & -1,142.935 & -428.892 \\
 & [1.861] & [1.797] & [1,129.105] & [1,550.416] & [1,647.561] & [856.172] \\
Padova Reli ITC & -1.425 & 0.806 & -871.838 & -1,628.937 & -1,212.873 & -683.360 \\
 & [1.393] & [1.987] & [1,206.462] & [2,084.115] & [1,498.490] & [988.979] \\
Parma Priv ITC & 0.375 & 0.941 & -651.362 & -1,138.363 & -1,141.550 & -427.509 \\
 & [1.267] & [1.436] & [1,129.258] & [1,550.513] & [1,647.678] & [856.317] \\
Reggio Priv ITC & -1.958** & -1.466 & -1.810 & -8.053 & -6.526 & -3.892 \\
 & [0.852] & [1.010] & [1.164] & [9.482] & [6.224] & [6.791] \\
Parma None ITC & -1.192** & -0.647 & -653.033 & -1,140.041 & -1,143.248 & -429.170 \\
 & [0.604] & [0.852] & [1,129.237] & [1,550.507] & [1,647.693] & [856.287] \\
Padova None ITC & -0.358 & 1.600 & -870.300 & -1,627.357 & -1,211.326 & -681.826 \\
 & [0.547] & [1.100] & [1,206.432] & [2,084.037] & [1,498.458] & [988.954] \\
Constant & 9.625*** & 225.975 & 567.315 & 993.113 & 840.542 & 419.465 \\
 & [0.446] & [444.473] & [876.274] & [1,329.241] & [1,155.681] & [728.987] \\
 &  &  &  &  &  &  \\
Observations & 655 & 655 & 655 & 655 & 655 & 655 \\
R-squared & 0.025 & 0.117 & 0.233 & 0.228 & 0.232 & 0.232 \\
Controls & None & All & Inter &  &  &  \\
 IV &  &  &  & distance & distXsib & dist score \\ \hline
\multicolumn{7}{c}{ Robust standard errors in brackets} \\
\multicolumn{7}{c}{ *** p$<$0.01, ** p$<$0.05, * p$<$0.1} \\
\multicolumn{7}{c}{ Dependent variable: SDQ score (self rep.).} \\
\end{tabular}

\end{table}
\begin{table}[H]
\caption{Pool Regression: SDQ Score - Preschool, Adolescent}
\begin{tabular}{lcccccc} \hline
 & (1) & (2) & (3) & (4) & (5) & (6) \\
VARIABLES &  &  &  &  &  &  \\ \hline
 &  &  &  &  &  &  \\
RCH preschool &  &  &  & -0.997 & -0.764 & -1.397 \\
 &  &  &  & [2.686] & [2.542] & [2.601] \\
Reggio None Preschool & -2.087 & -1.806 & -0.568 & -1.712 & -1.612 & -1.952 \\
 & [1.582] & [1.540] & [1.807] & [2.268] & [2.264] & [2.221] \\
Reggio Reli Preschool & 1.399** & 1.525** & 1.439* &  &  &  \\
 & [0.657] & [0.701] & [0.759] &  &  &  \\
Parma Muni Preschool & -0.981 & -0.540 & -459.705 & -275.550 & -553.074 & -261.819 \\
 & [0.596] & [0.870] & [1,112.923] & [878.099] & [1,171.471] & [723.338] \\
Padova Muni Preschool & -0.301 & 1.395 & -824.900 & -713.446 & -986.122 & -669.921 \\
 & [0.654] & [1.161] & [1,186.044] & [962.520] & [1,369.402] & [871.601] \\
Male dummy &  & -0.394 & -0.303 & -0.331 & -0.332 & -0.315 \\
 &  & [0.350] & [0.652] & [0.601] & [0.607] & [0.602] \\
Parma Stat Preschool & -1.744* & -1.801 & -460.976 & -276.823 & -554.344 & -263.092 \\
 & [0.906] & [1.211] & [1,112.900] & [878.077] & [1,171.450] & [723.315] \\
Padova Stat Preschool & -0.948 & 0.884 & -825.718 & -714.264 & -986.941 & -670.738 \\
 & [0.762] & [1.314] & [1,185.917] & [962.389] & [1,369.289] & [871.469] \\
Reggio Stat Preschool & -1.986** & -2.289** & -1.921* & -3.077 & -2.935 & -3.325* \\
 & [0.879] & [0.978] & [0.999] & [1.928] & [1.857] & [1.889] \\
Parma Reli Preschool & -0.176 & 0.280 & -459.279 & -275.123 & -552.648 & -261.391 \\
 & [0.690] & [0.858] & [1,112.921] & [878.096] & [1,171.471] & [723.333] \\
Padova Reli Preschool & 0.619 & 2.945** & -823.646 & -712.191 & -984.878 & -668.663 \\
 & [0.595] & [1.148] & [1,186.061] & [962.537] & [1,369.432] & [871.612] \\
Parma Priv Preschool & 0.985 & 2.221 & -458.445 & -274.318 & -551.800 & -260.589 \\
 & [1.835] & [2.134] & [1,112.754] & [877.924] & [1,171.309] & [723.200] \\
Padova Priv Preschool & -2.265 & -0.020 & -826.745 & -715.291 & -987.967 & -671.765 \\
 & [1.627] & [1.802] & [1,186.099] & [962.575] & [1,369.451] & [871.657] \\
Reggio Priv Preschool & 0.485 & 0.774 & -0.494 & -1.736 & -1.602 & -1.929 \\
 & [1.034] & [1.063] & [1.284] & [1.799] & [1.753] & [1.769] \\
Parma None Preschool & -0.015 & 2.699* & -456.407 & -272.204 & -549.800 & -258.466 \\
 & [1.149] & [1.464] & [1,113.113] & [878.295] & [1,171.667] & [723.469] \\
Padova None Preschool & 3.485*** & 4.278*** & -822.649 & -711.195 & -983.876 & -667.668 \\
 & [0.411] & [1.334] & [1,185.933] & [962.406] & [1,369.309] & [871.483] \\
Constant & 9.515*** & 346.532 & 680.087 & 576.774 & 733.085 & 572.782 \\
 & [0.411] & [432.335] & [871.226] & [666.706] & [898.740] & [623.103] \\
 &  &  &  &  &  &  \\
Observations & 656 & 656 & 656 & 656 & 656 & 656 \\
R-squared & 0.043 & 0.143 & 0.242 & 0.243 & 0.242 & 0.243 \\
Controls & None & All & Inter &  &  &  \\
 IV &  &  &  & distance & distXsib & dist score \\ \hline
\multicolumn{7}{c}{ Robust standard errors in brackets} \\
\multicolumn{7}{c}{ *** p$<$0.01, ** p$<$0.05, * p$<$0.1} \\
\multicolumn{7}{c}{ Dependent variable: SDQ score (self rep.).} \\
\end{tabular}

\end{table}

\pagebreak
\subsubsection{Adults Results}
\begin{table}[H]
\caption{Pool Regression: Depression - Infant-toddler center, Adult}
\begin{tabular}{lcccccc} \hline
 & (1) & (2) & (3) & (4) & (5) & (6) \\
VARIABLES &  &  &  &  &  &  \\ \hline
 &  &  &  &  &  &  \\
RCH infant-toddler &  &  &  & -71.092 & 22.119 & -46.612 \\
 &  &  &  & [70.064] & [23.457] & [50.137] \\
Reggio None ITC & 0.725 & -0.076 & 0.195 & -67.771 & 21.486 & -44.301 \\
 & [0.594] & [0.614] & [0.634] & [67.818] & [22.594] & [48.446] \\
Reggio Reli ITC & -3.531*** & -4.927*** & -4.705*** &  &  &  \\
 & [0.852] & [1.640] & [1.712] &  &  &  \\
Parma Muni ITC & -3.487*** & -0.946 & 0.871 & -57.073 & 22.469 & -36.840 \\
 & [0.803] & [1.121] & [8.284] & [60.743] & [20.869] & [43.481] \\
Padova Muni ITC & 5.307*** & 3.741** & 14.207 & -43.737 & 35.730* & -23.504 \\
 & [1.067] & [1.593] & [8.844] & [60.819] & [21.176] & [43.683] \\
Male dummy &  & -0.700*** & -0.698* & 0.011 & -0.858* & -0.218 \\
 &  & [0.252] & [0.413] & [0.774] & [0.466] & [0.632] \\
Parma Reli ITC & -1.548 & 0.257 & 1.987 & -55.957 & 23.585 & -35.724 \\
 & [1.638] & [1.859] & [8.413] & [60.760] & [20.917] & [43.514] \\
Padova Reli ITC & 1.873 & 1.777 & 12.317 & -45.627 & 33.818 & -25.394 \\
 & [1.726] & [2.039] & [9.096] & [60.854] & [21.275] & [43.726] \\
Parma Priv ITC & -1.463 & -1.346 & 0.292 & -57.652 & 21.890 & -37.419 \\
 & [1.616] & [1.828] & [8.376] & [60.755] & [20.907] & [43.489] \\
Padova Priv ITC & -0.948 & -1.554 & 10.181 & -47.763 & 31.733 & -27.530 \\
 & [1.988] & [2.867] & [8.978] & [60.837] & [21.216] & [43.706] \\
Reggio Priv ITC & 1.052 & 0.078 & -0.574 & -68.644 & 20.875 & -45.141 \\
 & [3.592] & [3.402] & [2.948] & [67.778] & [22.805] & [48.491] \\
Parma None ITC & 0.089 & 0.632 & 2.475 & -55.469 & 24.073 & -35.235 \\
 & [0.604] & [0.870] & [8.328] & [60.749] & [20.885] & [43.495] \\
Padova None ITC & -0.108 & -0.719 & 9.744 & -48.199 & 31.228 & -27.966 \\
 & [0.600] & [1.302] & [8.776] & [60.809] & [21.143] & [43.669] \\
Constant & 21.281*** & 34.089*** & 28.480*** & 86.424 & 6.882 & 66.191 \\
 & [0.549] & [3.555] & [6.068] & [60.492] & [20.124] & [43.135] \\
 &  &  &  &  &  &  \\
Observations & 1,888 & 1,888 & 1,888 & 1,888 & 1,855 & 1,888 \\
R-squared & 0.037 & 0.170 & 0.206 &  & 0.186 & 0.043 \\
Controls & None & All & Inter &  &  &  \\
 IV &  &  &  & distance & distXsib & dist score \\ \hline
\multicolumn{7}{c}{ Robust standard errors in brackets} \\
\multicolumn{7}{c}{ *** p$<$0.01, ** p$<$0.05, * p$<$0.1} \\
\multicolumn{7}{c}{ Dependent variable: Depression score (CESD).} \\
\end{tabular}

\end{table}
\begin{table}[H]
\caption{Pool Regression: Depression - Preschool, Adult}
\begin{tabular}{lcccccc} \hline
 & (1) & (2) & (3) & (4) & (5) & (6) \\
VARIABLES &  &  &  &  &  &  \\ \hline
 &  &  &  &  &  &  \\
RCH preschool &  &  &  & 10.144 & 2.791 & -5.089 \\
 &  &  &  & [10.614] & [6.194] & [6.657] \\
Reggio None Preschool & 1.765*** & 0.386 & 0.944 & 7.372 & 2.653 & -2.079 \\
 & [0.472] & [0.528] & [0.594] & [6.635] & [3.799] & [4.198] \\
Reggio Reli Preschool & -0.476 & -0.672 & -0.355 &  &  &  \\
 & [0.605] & [0.591] & [0.596] &  &  &  \\
Parma Muni Preschool & -0.878 & 0.225 & 3.713 & 6.536 & 6.867 & 2.401 \\
 & [0.589] & [0.837] & [8.405] & [9.873] & [8.744] & [8.833] \\
Padova Muni Preschool & 0.620 & -0.620 & 12.293 & 20.515 & 18.054* & 8.273 \\
 & [0.784] & [1.303] & [9.017] & [13.418] & [10.892] & [10.695] \\
Male dummy &  & -0.627** & -0.658 & -0.882 & -0.686 & -0.533 \\
 &  & [0.254] & [0.415] & [0.544] & [0.432] & [0.450] \\
Parma Stat Preschool & -2.079*** & -0.187 & 3.224 & 7.140 & 6.679 & 1.365 \\
 & [0.773] & [0.990] & [8.441] & [10.362] & [8.999] & [9.101] \\
Padova Stat Preschool & 3.942*** & 2.899** & 15.410* & 23.631* & 21.103* & 11.390 \\
 & [0.895] & [1.472] & [9.080] & [13.458] & [10.943] & [10.726] \\
Reggio Stat Preschool & 1.254 & 0.834 & 1.019 & 8.489 & 3.236 & -2.582 \\
 & [0.937] & [0.888] & [0.855] & [7.829] & [4.622] & [5.025] \\
Parma Reli Preschool & -0.817 & 0.320 & 3.535 & 7.269 & 6.940 & 1.767 \\
 & [0.603] & [0.863] & [8.446] & [10.281] & [8.964] & [9.078] \\
Padova Reli Preschool & -0.534 & -0.622 & 12.045 & 20.267 & 17.829* & 8.025 \\
 & [0.466] & [1.252] & [8.941] & [13.369] & [10.826] & [10.637] \\
Parma Priv Preschool & -0.412 & 1.362 & 4.402 & 8.358 & 7.868 & 2.523 \\
 & [2.764] & [3.331] & [9.074] & [10.849] & [9.583] & [9.698] \\
Padova Priv Preschool & -2.746 & -3.260 & 8.433 & 16.655 & 14.183 & 4.413 \\
 & [2.508] & [2.101] & [9.288] & [13.593] & [11.118] & [10.960] \\
Reggio Priv Preschool & -1.960 & -1.432 & -0.761 & 5.997 & 1.156 & -3.967 \\
 & [1.427] & [1.418] & [1.328] & [7.018] & [4.184] & [4.604] \\
Parma None Preschool & 0.794* & 1.499** & 4.985 & 8.637 & 8.367 & 3.257 \\
 & [0.472] & [0.744] & [8.469] & [10.262] & [8.967] & [9.073] \\
Padova None Preschool & 0.537 & 0.189 & 12.323 & 20.545 & 18.110* & 8.303 \\
 & [0.565] & [1.289] & [9.014] & [13.416] & [10.889] & [10.688] \\
Constant & 21.246*** & 33.176*** & 26.358*** & 18.136 & 21.718** & 30.378*** \\
 & [0.337] & [3.677] & [6.188] & [11.789] & [8.614] & [8.842] \\
 &  &  &  &  &  &  \\
Observations & 1,875 & 1,875 & 1,875 & 1,875 & 1,847 & 1,875 \\
R-squared & 0.041 & 0.171 & 0.206 & 0.075 & 0.198 & 0.158 \\
Controls & None & All & Inter &  &  &  \\
 IV &  &  &  & distance & distXsib & dist score \\ \hline
\multicolumn{7}{c}{ Robust standard errors in brackets} \\
\multicolumn{7}{c}{ *** p$<$0.01, ** p$<$0.05, * p$<$0.1} \\
\multicolumn{7}{c}{ Dependent variable: Depression score (CESD).} \\
\end{tabular}

\end{table}
\begin{table}[H]
\caption{Pool Regression: Health - Infant-toddler center, Adult}
\begin{tabular}{lcccccc} \hline
 & (1) & (2) & (3) & (4) & (5) & (6) \\
VARIABLES &  &  &  &  &  &  \\ \hline
 &  &  &  &  &  &  \\
RCH infant-toddler &  &  &  & -0.722 & -1.382 & -3.326 \\
 &  &  &  & [2.252] & [1.538] & [3.471] \\
Reggio None ITC & -0.222*** & -0.053* & -0.025 & -0.728 & -1.355 & -3.225 \\
 & [0.032] & [0.031] & [0.032] & [2.163] & [1.484] & [3.357] \\
Reggio Reli ITC & 0.073*** & 0.199 & 0.258 &  &  &  \\
 & [0.027] & [0.137] & [0.158] &  &  &  \\
Parma Muni ITC & -0.677*** & -0.299*** & 1.869*** & 1.227 & 0.718 & -0.925 \\
 & [0.059] & [0.075] & [0.630] & [1.943] & [1.418] & [3.022] \\
Padova Muni ITC & -0.251*** & -0.162 & 2.097*** & 1.455 & 1.080 & -0.697 \\
 & [0.085] & [0.110] & [0.702] & [1.966] & [1.459] & [3.041] \\
Male dummy &  & 0.039** & 0.030 & 0.034 & 0.044 & 0.059 \\
 &  & [0.019] & [0.027] & [0.035] & [0.031] & [0.041] \\
Parma Reli ITC & -0.394*** & -0.130 & 1.997*** & 1.356 & 0.846 & -0.796 \\
 & [0.132] & [0.121] & [0.629] & [1.943] & [1.418] & [3.022] \\
Padova Reli ITC & -0.619*** & -0.467*** & 1.753** & 1.111 & 0.723 & -1.041 \\
 & [0.131] & [0.159] & [0.714] & [1.971] & [1.462] & [3.045] \\
Parma Priv ITC & -0.654*** & -0.612*** & 1.527** & 0.886 & 0.376 & -1.266 \\
 & [0.137] & [0.150] & [0.635] & [1.945] & [1.420] & [3.024] \\
Padova Priv ITC & 0.073*** & 0.112 & 2.356*** & 1.715 & 1.329 & -0.437 \\
 & [0.027] & [0.101] & [0.688] & [1.962] & [1.452] & [3.038] \\
Reggio Priv ITC & 0.073*** & -0.005 & 0.008 & -0.695 & -1.331 & -3.196 \\
 & [0.027] & [0.040] & [0.034] & [2.168] & [1.486] & [3.355] \\
Parma None ITC & -0.342*** & -0.129** & 2.043*** & 1.402 & 0.892 & -0.750 \\
 & [0.035] & [0.051] & [0.629] & [1.943] & [1.417] & [3.021] \\
Padova None ITC & -0.380*** & -0.222*** & 1.993*** & 1.352 & 0.964 & -0.800 \\
 & [0.034] & [0.080] & [0.697] & [1.965] & [1.456] & [3.040] \\
Constant & 0.927*** & 0.756*** & -0.498 & 0.143 & 0.653 & 2.295 \\
 & [0.027] & [0.284] & [0.396] & [1.883] & [1.335] & [2.984] \\
 &  &  &  &  &  &  \\
Observations & 1,899 & 1,899 & 1,899 & 1,899 & 1,865 & 1,899 \\
R-squared & 0.069 & 0.307 & 0.329 & 0.327 & 0.323 & 0.249 \\
Controls & None & All & Inter &  &  &  \\
 IV &  &  &  & distance & distXsib & dist score \\ \hline
\multicolumn{7}{c}{ Robust standard errors in brackets} \\
\multicolumn{7}{c}{ *** p$<$0.01, ** p$<$0.05, * p$<$0.1} \\
\multicolumn{7}{c}{ Dependent variable: Respondent health is good (\%).} \\
\end{tabular}

\end{table}
\begin{table}[H]
\caption{Pool Regression: Health - Preschool, Adult}
\begin{tabular}{lcccccc} \hline
 & (1) & (2) & (3) & (4) & (5) & (6) \\
VARIABLES &  &  &  &  &  &  \\ \hline
 &  &  &  &  &  &  \\
RCH preschool &  &  &  & -0.300 & 0.310 & 0.994* \\
 &  &  &  & [0.528] & [0.423] & [0.508] \\
Reggio None Preschool & -0.319*** & -0.089** & -0.026 & -0.206 & 0.175 & 0.595* \\
 & [0.035] & [0.036] & [0.039] & [0.331] & [0.257] & [0.319] \\
Reggio Reli Preschool & -0.135*** & -0.026 & -0.014 &  &  &  \\
 & [0.044] & [0.036] & [0.035] &  &  &  \\
Parma Muni Preschool & -0.358*** & -0.180*** & 1.795*** & 1.716*** & 1.929*** & 2.069*** \\
 & [0.045] & [0.053] & [0.641] & [0.662] & [0.660] & [0.748] \\
Padova Muni Preschool & -0.294*** & -0.169* & 2.036*** & 1.795** & 2.346*** & 2.844*** \\
 & [0.064] & [0.092] & [0.711] & [0.839] & [0.828] & [0.891] \\
Male dummy &  & 0.039** & 0.026 & 0.032 & 0.024 & 0.007 \\
 &  & [0.020] & [0.027] & [0.030] & [0.028] & [0.037] \\
Parma Stat Preschool & -0.410*** & -0.207*** & 1.774*** & 1.663** & 1.941*** & 2.155*** \\
 & [0.060] & [0.064] & [0.643] & [0.680] & [0.675] & [0.761] \\
Padova Stat Preschool & -0.072 & -0.014 & 2.202*** & 1.962** & 2.505*** & 3.010*** \\
 & [0.060] & [0.098] & [0.715] & [0.842] & [0.831] & [0.894] \\
Reggio Stat Preschool & -0.074 & -0.012 & 0.003 & -0.209 & 0.222 & 0.723* \\
 & [0.055] & [0.051] & [0.051] & [0.382] & [0.303] & [0.369] \\
Parma Reli Preschool & -0.218*** & -0.073 & 1.909*** & 1.803*** & 2.071*** & 2.272*** \\
 & [0.050] & [0.069] & [0.648] & [0.682] & [0.677] & [0.762] \\
Padova Reli Preschool & -0.347*** & -0.213*** & 1.981*** & 1.740** & 2.285*** & 2.789*** \\
 & [0.034] & [0.082] & [0.707] & [0.835] & [0.824] & [0.887] \\
Parma Priv Preschool & -0.551*** & -0.467** & 1.501** & 1.389** & 1.670** & 1.886** \\
 & [0.194] & [0.207] & [0.659] & [0.697] & [0.690] & [0.771] \\
Padova Priv Preschool & 0.115*** & 0.485*** & 2.460*** & 2.220*** & 2.756*** & 3.269*** \\
 & [0.019] & [0.116] & [0.687] & [0.819] & [0.806] & [0.873] \\
Reggio Priv Preschool & -0.135 & 0.046 & 0.041 & -0.144 & 0.246 & 0.678* \\
 & [0.155] & [0.154] & [0.164] & [0.363] & [0.304] & [0.369] \\
Parma None Preschool & -0.394*** & -0.105* & 1.885*** & 1.782*** & 2.044*** & 2.239*** \\
 & [0.040] & [0.054] & [0.646] & [0.679] & [0.675] & [0.760] \\
Padova None Preschool & -0.402*** & -0.215** & 1.968*** & 1.728** & 2.271*** & 2.777*** \\
 & [0.043] & [0.086] & [0.707] & [0.835] & [0.825] & [0.887] \\
Constant & 0.885*** & 0.889*** & -0.345 & -0.105 & -0.646 & -1.154* \\
 & [0.019] & [0.287] & [0.400] & [0.611] & [0.575] & [0.682] \\
 &  &  &  &  &  &  \\
Observations & 1,886 & 1,886 & 1,886 & 1,886 & 1,857 & 1,886 \\
R-squared & 0.089 & 0.308 & 0.327 & 0.302 & 0.320 & 0.148 \\
Controls & None & All & Inter &  &  &  \\
 IV &  &  &  & distance & distXsib & dist score \\ \hline
\multicolumn{7}{c}{ Robust standard errors in brackets} \\
\multicolumn{7}{c}{ *** p$<$0.01, ** p$<$0.05, * p$<$0.1} \\
\multicolumn{7}{c}{ Dependent variable: Respondent health is good (\%).} \\
\end{tabular}

\end{table}
\begin{table}[H]
\caption{Pool Regression: Migration Taste - Infant-toddler center, Adult}
\begin{tabular}{lcccccc} \hline
 & (1) & (2) & (3) & (4) & (5) & (6) \\
VARIABLES &  &  &  &  &  &  \\ \hline
 &  &  &  &  &  &  \\
RCH infant-toddler &  &  &  & 0.606 & 0.553 & 2.978 \\
 &  &  &  & [2.419] & [1.292] & [3.341] \\
Reggio None ITC & 0.135*** & 0.099*** & 0.083** & 0.658 & 0.605 & 2.932 \\
 & [0.031] & [0.033] & [0.035] & [2.315] & [1.238] & [3.225] \\
Reggio Reli ITC & 0.177 & 0.146 & 0.139 &  &  &  \\
 & [0.219] & [0.205] & [0.196] &  &  &  \\
Parma Muni ITC & 0.118** & 0.152** & 0.732 & 1.206 & 1.114 & 3.161 \\
 & [0.055] & [0.075] & [0.674] & [2.143] & [1.290] & [2.952] \\
Padova Muni ITC & 0.412*** & 0.548*** & 0.486 & 0.961 & 0.847 & 2.915 \\
 & [0.091] & [0.118] & [0.710] & [2.155] & [1.319] & [2.956] \\
Male dummy &  & 0.025 & 0.036 & 0.029 & 0.033 & 0.006 \\
 &  & [0.020] & [0.030] & [0.036] & [0.031] & [0.042] \\
Parma Reli ITC & 0.260** & 0.226* & 0.774 & 1.249 & 1.156 & 3.203 \\
 & [0.125] & [0.131] & [0.683] & [2.146] & [1.295] & [2.954] \\
Padova Reli ITC & 0.499*** & 0.759*** & 0.701 & 1.176 & 1.047 & 3.130 \\
 & [0.135] & [0.153] & [0.721] & [2.158] & [1.325] & [2.959] \\
Parma Priv ITC & 0.291** & 0.309** & 0.846 & 1.321 & 1.228 & 3.275 \\
 & [0.148] & [0.149] & [0.682] & [2.146] & [1.294] & [2.954] \\
Padova Priv ITC & 0.594** & 0.836*** & 0.777 & 1.252 & 1.126 & 3.206 \\
 & [0.274] & [0.249] & [0.739] & [2.164] & [1.334] & [2.964] \\
Reggio Priv ITC & -0.073*** & -0.074** & -0.060 & 0.516 & 0.464 & 2.793 \\
 & [0.027] & [0.033] & [0.042] & [2.321] & [1.238] & [3.225] \\
Parma None ITC & 0.151*** & 0.145*** & 0.728 & 1.203 & 1.111 & 3.157 \\
 & [0.033] & [0.054] & [0.674] & [2.144] & [1.290] & [2.953] \\
Padova None ITC & 0.276*** & 0.496*** & 0.463 & 0.937 & 0.810 & 2.891 \\
 & [0.034] & [0.082] & [0.706] & [2.154] & [1.318] & [2.956] \\
Constant & 0.073*** & 0.191 & -0.048 & -0.523 & -0.430 & -2.477 \\
 & [0.027] & [0.288] & [0.440] & [2.085] & [1.189] & [2.907] \\
 &  &  &  &  &  &  \\
Observations & 1,891 & 1,891 & 1,891 & 1,891 & 1,859 & 1,891 \\
R-squared & 0.040 & 0.091 & 0.113 & 0.107 & 0.106 & 0.011 \\
Controls & None & All & Inter &  &  &  \\
 IV &  &  &  & distance & distXsib & dist score \\ \hline
\multicolumn{7}{c}{ Robust standard errors in brackets} \\
\multicolumn{7}{c}{ *** p$<$0.01, ** p$<$0.05, * p$<$0.1} \\
\multicolumn{7}{c}{ Dependent variable: Bothered by migrants (\%).} \\
\end{tabular}

\end{table}
\begin{table}[H]
\caption{Pool Regression: Migration Taste - Preschool, Adult}
\begin{tabular}{lcccccc} \hline
 & (1) & (2) & (3) & (4) & (5) & (6) \\
VARIABLES &  &  &  &  &  &  \\ \hline
 &  &  &  &  &  &  \\
RCH preschool &  &  &  & 0.541 & 0.329 & -0.602 \\
 &  &  &  & [0.630] & [0.458] & [0.460] \\
Reggio None Preschool & 0.116*** & 0.083** & 0.056 & 0.353 & 0.210 & -0.355 \\
 & [0.032] & [0.036] & [0.040] & [0.393] & [0.281] & [0.289] \\
Reggio Reli Preschool & 0.130*** & 0.108** & 0.102** &  &  &  \\
 & [0.044] & [0.045] & [0.046] &  &  &  \\
Parma Muni Preschool & 0.175*** & 0.193*** & 0.689 & 0.820 & 0.745 & 0.503 \\
 & [0.042] & [0.057] & [0.678] & [0.730] & [0.699] & [0.707] \\
Padova Muni Preschool & 0.384*** & 0.530*** & 0.487 & 0.907 & 0.689 & -0.021 \\
 & [0.065] & [0.088] & [0.714] & [0.899] & [0.825] & [0.823] \\
Male dummy &  & 0.032 & 0.047 & 0.032 & 0.040 & 0.058* \\
 &  & [0.020] & [0.029] & [0.035] & [0.032] & [0.033] \\
Parma Stat Preschool & 0.102** & 0.114* & 0.618 & 0.806 & 0.709 & 0.367 \\
 & [0.051] & [0.069] & [0.683] & [0.753] & [0.715] & [0.721] \\
Padova Stat Preschool & 0.197*** & 0.373*** & 0.307 & 0.727 & 0.498 & -0.201 \\
 & [0.070] & [0.105] & [0.710] & [0.896] & [0.822] & [0.821] \\
Reggio Stat Preschool & 0.045 & 0.042 & 0.036 & 0.398 & 0.245 & -0.427 \\
 & [0.053] & [0.052] & [0.052] & [0.461] & [0.334] & [0.338] \\
Parma Reli Preschool & 0.170*** & 0.148** & 0.633 & 0.811 & 0.718 & 0.393 \\
 & [0.048] & [0.064] & [0.686] & [0.752] & [0.715] & [0.722] \\
Padova Reli Preschool & 0.247*** & 0.446*** & 0.391 & 0.811 & 0.595 & -0.117 \\
 & [0.034] & [0.085] & [0.711] & [0.897] & [0.822] & [0.820] \\
Parma Priv Preschool & 0.051 & 0.050 & 0.518 & 0.709 & 0.611 & 0.265 \\
 & [0.154] & [0.149] & [0.686] & [0.756] & [0.718] & [0.725] \\
Padova Priv Preschool & -0.116*** & 0.144 & 0.177 & 0.596 & 0.384 & -0.332 \\
 & [0.019] & [0.108] & [0.694] & [0.884] & [0.807] & [0.806] \\
Reggio Priv Preschool & 0.027 & -0.023 & -0.036 & 0.279 & 0.134 & -0.462 \\
 & [0.134] & [0.131] & [0.123] & [0.424] & [0.316] & [0.322] \\
Parma None Preschool & 0.042 & 0.033 & 0.555 & 0.729 & 0.638 & 0.320 \\
 & [0.032] & [0.057] & [0.689] & [0.754] & [0.718] & [0.724] \\
Padova None Preschool & 0.218*** & 0.391*** & 0.346 & 0.765 & 0.551 & -0.162 \\
 & [0.041] & [0.088] & [0.713] & [0.898] & [0.824] & [0.822] \\
Constant & 0.116*** & 0.149 & -0.018 & -0.438 & -0.251 & 0.490 \\
 & [0.019] & [0.293] & [0.432] & [0.707] & [0.597] & [0.606] \\
 &  &  &  &  &  &  \\
Observations & 1,877 & 1,877 & 1,877 & 1,877 & 1,850 & 1,877 \\
R-squared & 0.049 & 0.097 & 0.118 & 0.010 & 0.069 & 0.059 \\
Controls & None & All & Inter &  &  &  \\
 IV &  &  &  & distance & distXsib & dist score \\ \hline
\multicolumn{7}{c}{ Robust standard errors in brackets} \\
\multicolumn{7}{c}{ *** p$<$0.01, ** p$<$0.05, * p$<$0.1} \\
\multicolumn{7}{c}{ Dependent variable: Bothered by migrants (\%).} \\
\end{tabular}

\end{table}

\end{document}
